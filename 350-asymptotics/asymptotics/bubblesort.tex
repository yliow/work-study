%-*-latex-*-
\sectionthree{Bubblesort: double for-loops}
\begin{python0}
from solutions import *; clear()
\end{python0}


Here's bubblesort. Suppose $\texttt{a[i]}$ $(i=0,
$\ldots$, n-1)$ is an array of numbers (integers or floats or
doubles, we don't care). The following sorts $\texttt{a[i]}$
$(i=0,\ldots,n-1)$ in ascending order:
\begin{Verbatim}[frame=single, fontsize=\footnotesize]
for i = n - 2, n - 3, ...,  0:
    for j = 0, 1, 2, ..., i:
        if a[j] > a[j + 1]:
            t = a[j]
            a[j] = a[j + 1]
            a[j + 1] = t
\end{Verbatim}


Let's analyze the time complexity of the above algorithm.

\begin{Verbatim}[frame=single, fontsize=\footnotesize]
          i = n - 2
LOOP1:    if i == -1:
              goto ENDLOOP1

          j = 0
LOOP2:    if j > i:
              goto ENDLOOP2
          if a[j] > a[j + 1]
              goto ENDIF    
          t = a[j]
          a[j] = a[j + 1]
          a[j + 1] = t
ENDIF     j = j + 1
          goto LOOP2

ENDLOOP2: i = i - 1
          goto LOOP2

ENDLOOP1:
\end{Verbatim}

With time for each statement:
\begin{Verbatim}[frame=single, fontsize=\footnotesize]
          i = n - 2               t1
LOOP1:    if i == -1:             t2
              goto ENDLOOP1       t3

          j = 0                   t4
LOOP2:    if j > i:               t5
              goto ENDLOOP2       t6
          if a[i] > a[j + 1]:     t7
              goto ENDIF          t8   
          t = a[i]                t9
          a[i] = a[j]             t10
          a[j] = t                t11
ENDIF:    j = j + 1               t12
          goto LOOP2              t13

ENDLOOP2: i = i - 1               t14
          goto LOOP1              t15

ENDLOOP1:
\end{Verbatim}

Including the number of times each statement is executed,
the worst case runtime is:
\begin{Verbatim}[frame=single, fontsize=\footnotesize]
          i = n - 2             t1  1
LOOP1:    if i == -1:           t2  n
              goto ENDLOOP1     t3  1
                                
          j = 0                 t4  1+...+1     = n-1
LOOP2:    if j > i:             t5  n+...+2     = (n-1)(n+2)/2
              goto ENDLOOP2     t6  1+...+1     = n-1
          if a[i] > a[j + 1]:   t7  (n-1)+...+1 = (n-1)n/2
              goto ENDIF        t8  0 
          t = a[i]              t9  (n-1)+...+1 = (n-1)n/2
          a[i] = a[j]           t10 (n-1)+...+1 = (n-1)n/2
          a[j] = t              t11 (n-1)+...+1 = (n-1)n/2
          j = j + 1             t12 (n-1)+...+1 = (n-1)n/2
          goto LOOP2            t13 (n-1)+...+1 = (n-1)n/2

ENDLOOP2: i = i - 1             t14 n-1
          goto LOOP1            t15 n-1

ENDLOOP1:
\end{Verbatim}

For the case when $i = n - 2$, the inner loop 
will have $j$ running from $0$ to $i + 1 = n - 1$,
with the body running for $j = 0, ..., i = n - 2$ ($n - 1$ times) and 
exiting when $j = i + 1 = n - 1$ (once):
\begin{Verbatim}[frame=single, fontsize=\footnotesize]
              j = 0               t4   1
LOOP2:        if j > i:           t5   n
                  goto ENDLOOP2   t6   1
              if a[i] > a[j + 1]: t7   (n-1)
                  goto ENDIF      t8   0
              t = a[i]            t9   (n-1)
              a[i] = a[j]         t10  (n-1)
              a[j] = t            t11  (n-1)
              j = j + 1           t12  (n-1)
              goto LOOP2          t13  (n-1)

ENDLOOPS:     ...
\end{Verbatim}
All other cases of $i$ values are similar.

So the worst case runtime is
\begin{align*}
f(n) = \
& (t_1 + t_3) \\
&+ n \cdot t_2 \\
&+ (n-1) \cdot (t_4 + t_6 + t_{14} + t_{15}) \\
&+ \frac{(n-1)n}{2} \cdot (t_{7} + t_{9} + t_{10} + t_{11} + t_{12} + t_{13}) \\
&+ \frac{(n-1)(n+2)}{2} \cdot t_5
\end{align*}
Rewriting this as a polynomial in $n$, we get
\begin{align*}
T(n) = \
&\frac{t_5 + t_7 + t_9 + t_{10} + t_{11} + t_{12} + t_{13}}{2} \cdot  n^2 \\
&+ \left(
   t_2 + t_4 + t_6 + t_{14} + t_{15}
   - \frac{t_7 + t_9 + t_{10} + t_{11} + t_{12} + t_{13} + t_5}{2}
   \right) \cdot n  \\
&+ \left(
   t_1 + t_3 - t_4 - t_6 - t_{14} - t_{15}- t_5 
   \right)
\end{align*}
In other words the worst case runtime is of the form
\[
f(n) = An^2 + Bn + C
\]
where $A,B,C$ are constants.
And of course in this case
\[
f(n) = O(n^2)
\]

For the best case:
{\footnotesize
\begin{Verbatim}[frame=single, fontsize=\footnotesize]
          i = n - 2               t1   1
LOOP1:    if i == -1:             t2   n
              goto ENDLOOP1       t3   1

          j = 0                   t4   1+...+1     = n-1
LOOP2:    if j > i:               t5   n+...+2     = (n-1)(n+2)/2
              goto ENDLOOP2       t6   1+...+1     = n-1
          if a[i] < a[j + 1]:     t7   (n-1)+...+1 = (n-1)n/2
              goto ENDIF          t8   (n-1)+...+1 = (n-1)n/2
          t = a[i]                t9   0
          a[i] = a[j]             t10  0
          a[j] = t                t11  0
          j = j + 1               t12  (n-1)+...+1 = (n-1)n/2
          goto LOOP2              t13  (n-1)+...+1 = (n-1)n/2
                                  
ENDLOOP2: i = i - 1               t14  n-1
          goto LOOP1              t15  n-1

ENDLOOP1:
\end{Verbatim}
}
I'll leave it to you to check that the best case runtime is also 
$O(n^2)$.


\newpage%-*-latex-*-

\begin{ex} 
  \label{ex:prob-00}
  \tinysidebar{\debug{exercises/{disc-prob-28/question.tex}}}

  \solutionlink{sol:prob-00}
  \qed
\end{ex} 
\begin{python0}
from solutions import *
add(label="ex:prob-00",
    srcfilename='exercises/discrete-probability/prob-00/answer.tex') 
\end{python0}

\newpage%-*-latex-*-

\begin{ex} 
  \label{ex:prob-00}
  \tinysidebar{\debug{exercises/{disc-prob-28/question.tex}}}

  \solutionlink{sol:prob-00}
  \qed
\end{ex} 
\begin{python0}
from solutions import *
add(label="ex:prob-00",
    srcfilename='exercises/discrete-probability/prob-00/answer.tex') 
\end{python0}

\newpage%-*-latex-*-

\begin{ex} 
  \label{ex:prob-00}
  \tinysidebar{\debug{exercises/{disc-prob-28/question.tex}}}

  \solutionlink{sol:prob-00}
  \qed
\end{ex} 
\begin{python0}
from solutions import *
add(label="ex:prob-00",
    srcfilename='exercises/discrete-probability/prob-00/answer.tex') 
\end{python0}

\newpage%-*-latex-*-

\begin{ex} 
  \label{ex:prob-00}
  \tinysidebar{\debug{exercises/{disc-prob-28/question.tex}}}

  \solutionlink{sol:prob-00}
  \qed
\end{ex} 
\begin{python0}
from solutions import *
add(label="ex:prob-00",
    srcfilename='exercises/discrete-probability/prob-00/answer.tex') 
\end{python0}

\newpage%-*-latex-*-

\begin{ex} 
  \label{ex:prob-00}
  \tinysidebar{\debug{exercises/{disc-prob-28/question.tex}}}

  \solutionlink{sol:prob-00}
  \qed
\end{ex} 
\begin{python0}
from solutions import *
add(label="ex:prob-00",
    srcfilename='exercises/discrete-probability/prob-00/answer.tex') 
\end{python0}

\newpage%-*-latex-*-

\begin{ex} 
  \label{ex:prob-00}
  \tinysidebar{\debug{exercises/{disc-prob-28/question.tex}}}

  \solutionlink{sol:prob-00}
  \qed
\end{ex} 
\begin{python0}
from solutions import *
add(label="ex:prob-00",
    srcfilename='exercises/discrete-probability/prob-00/answer.tex') 
\end{python0}

\newpage%-*-latex-*-

\begin{ex} 
  \label{ex:prob-00}
  \tinysidebar{\debug{exercises/{disc-prob-28/question.tex}}}

  \solutionlink{sol:prob-00}
  \qed
\end{ex} 
\begin{python0}
from solutions import *
add(label="ex:prob-00",
    srcfilename='exercises/discrete-probability/prob-00/answer.tex') 
\end{python0}

\newpage%-*-latex-*-

\begin{ex} 
  \label{ex:prob-00}
  \tinysidebar{\debug{exercises/{disc-prob-28/question.tex}}}

  \solutionlink{sol:prob-00}
  \qed
\end{ex} 
\begin{python0}
from solutions import *
add(label="ex:prob-00",
    srcfilename='exercises/discrete-probability/prob-00/answer.tex') 
\end{python0}

\newpage%-*-latex-*-

\begin{ex} 
  \label{ex:prob-00}
  \tinysidebar{\debug{exercises/{disc-prob-28/question.tex}}}

  \solutionlink{sol:prob-00}
  \qed
\end{ex} 
\begin{python0}
from solutions import *
add(label="ex:prob-00",
    srcfilename='exercises/discrete-probability/prob-00/answer.tex') 
\end{python0}

\newpage%-*-latex-*-

\begin{ex} 
  \label{ex:prob-00}
  \tinysidebar{\debug{exercises/{disc-prob-28/question.tex}}}

  \solutionlink{sol:prob-00}
  \qed
\end{ex} 
\begin{python0}
from solutions import *
add(label="ex:prob-00",
    srcfilename='exercises/discrete-probability/prob-00/answer.tex') 
\end{python0}

\newpage%-*-latex-*-

\begin{ex} 
  \label{ex:prob-00}
  \tinysidebar{\debug{exercises/{disc-prob-28/question.tex}}}

  \solutionlink{sol:prob-00}
  \qed
\end{ex} 
\begin{python0}
from solutions import *
add(label="ex:prob-00",
    srcfilename='exercises/discrete-probability/prob-00/answer.tex') 
\end{python0}

\newpage%-*-latex-*-

\begin{ex} 
  \label{ex:prob-00}
  \tinysidebar{\debug{exercises/{disc-prob-28/question.tex}}}

  \solutionlink{sol:prob-00}
  \qed
\end{ex} 
\begin{python0}
from solutions import *
add(label="ex:prob-00",
    srcfilename='exercises/discrete-probability/prob-00/answer.tex') 
\end{python0}


