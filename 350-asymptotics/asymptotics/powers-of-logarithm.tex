%-*-latex-*-
\section{The big-O Classes: Logarithms and Their Powers}

What about powers of $\lg n$ and also composition of the log function
such as
\[
(\lg n)^2
\]
and
\[
\lg (\lg n)
\]

We define the following
\begin{itemize}

\li The $k$--th power of $\ln$ is
\[
\lg^k n = (\lg n)^k
\]
i.e., $\lg$ function to the power of $k$.
For instance
\[
\lg^3 n = (\lg n)^3
\]
Note that the first power of $\lg$ is just $\lg$:
\[
\lg^1 n = \lg n
\]

\li The $k$--th composition of $\lg$ is
\[
\lg^{(k)} n = \lg(...(\lg k)...)
\]
i.e., composition of $\lg$ function $k$ times.
For instance 
\[
\lg^{(3)} n = \lg(\lg(\lg n))
\]

\end{itemize}

How does $\lg n$ (or $\lg^k n$) compare with powers of $n^d$?

Here's $\lg n$ and $n$:

\begin{python}
from latextool_basic import *
p = FunctionPlot(domain='0:3', vars=['n'])
p.add('n', color='red', python=1, pin='left')
p.add('log(n,2)', color='blue', python=1, pin='below', pin_message='\lg n')
print(p)
\end{python}

The function $\lg n$ seems to bend downward.
It actually grows unboundedly,
i.e., 
\[
\lim_{n \rightarrow \infty} \lg n = \infty
\]
although it does grow rather slowly when compared to $n$.

Here's the picture when for $1 \leq x \leq 50$:
\begin{python}
from latextool_basic import *
p = FunctionPlot(domain='1:100', vars=['n'])

p.add('n', color='red', python=1, pin='above')
p.add('log(n,2)', color='blue', python=1, pin='above', pin_message='\lg n')

print(p)
\end{python}

From the graph you would guess
\[
\lg n = O(n)
\]
and that is in fact true: you can prove the above fact using l'H\^opital's 
rule:
\[
\lim_{n \rightarrow \infty} \frac{\ln n}{n}
= \lim_{n \rightarrow \infty} \frac{1/n}{1}
= \lim_{n \rightarrow \infty} \frac{1}{n}
= 0
\]
The above is for comparing $\ln n$ against $n$ (recall: $\ln n = \log_e n$).
For $\lg_2 n$, we just use this:
\[
\lg_2 n = \frac{\log_e n}{\log_e 2}
\]
which implies that $\log_e n = (\log_2 n)(\log_e 2)$.
So the above computation for $\log_2 n$ becomes
\[
\lim_{n \rightarrow \infty} \frac{\log_2 n}{n}
= \lim_{n \rightarrow \infty} \frac{1}{\log_e 2}\frac{\log_e n}{n}
\]
and we would still get $0$.

What if I beef up $\lg n$?
Let's look at powers of $\lg^k n$ for $k = 1, 2, 3$.

\begin{python}
from latextool_basic import *
p = FunctionPlot(domain='0:10', vars=['n'])
p.add('n', color='red', python=1, pin='above', pin_message='', num_points=2)
p.add('log(n,2)', color='blue', python=1, pin='below', pin_message='\lg n', num_points=300)
p.add('log(n,2)**2', color='blue', python=1, pin='above', pin_message='')
p.add('log(n,2)**3', color='blue', python=1, pin='above left', pin_message='\lg^3 n')
print(p)
\end{python}

Looks like $\ln^3 n$ beats $n$.
But you know that graphs cannot be trusted (I already told that, right?).
So we check the graphs for $n$ up to 20:
\begin{python}
from latextool_basic import *
p = FunctionPlot(domain='0:20', vars=['n'])
p.add('n', color='red', python=1, pin='above', pin_message='', num_points=2)
p.add('log(n,2)', color='blue', python=1, pin='below', pin_message='\lg n')
p.add('log(n,2)**2', color='blue', python=1, pin='above', pin_message='')
p.add('log(n,2)**3', color='blue', python=1, pin='above', pin_message='\lg^3 n')
print(p)
\end{python}

Phew! ... seems to be ok ... wait let's try up to $n = 30$:
\begin{python}
from latextool_basic import *
p = FunctionPlot(domain='0:30', vars=['n'])
p.add('n', color='red', python=1, pin='above', pin_message='', num_points=2)
p.add('log(n,2)', color='blue', python=1, pin='below', pin_message='')
p.add('log(n,2)**2', color='blue', python=1, pin='above', pin_message='')
p.add('log(n,2)**3', color='blue', python=1, pin='above', pin_message='\lg^3 n')
print(p)
\end{python}
But now you begin to see something disturbing ... do you notice that
the $\ln^3 n$ is bending ever so slightly download?
Here's the graph up to $n = 50$:
\begin{python}
from latextool_basic import *
p = FunctionPlot(domain='0:50', vars=['n'])
p.add('n', color='red', python=1, pin='above', pin_message='', num_points=2)
p.add('log(n,2)', color='blue', python=1, pin='below', pin_message='')
p.add('log(n,2)**2', color='blue', python=1, pin='above', pin_message='')
p.add('log(n,2)**3', color='blue', python=1, pin='left', pin_message='\lg^3 n')
print(p)
\end{python}

Now the concavity of $\lg^3 n$ becomes more pronounced!
Oh no!
Let's try up to $n = 200$:
\begin{python}
from latextool_basic import *
p = FunctionPlot(domain='0:200', vars=['n'])
p.add('n', color='red', python=1, pin='above', pin_message='', num_points=2)
p.add('log(n,2)', color='blue', python=1, pin='below', pin_message='')
p.add('log(n,2)**2', color='blue', python=1, pin='above', pin_message='')
p.add('log(n,2)**3', color='blue', python=1, pin='above', pin_message='\lg^3 n')
print(p)
\end{python}
and up to $n = 500$:
\begin{python}
from latextool_basic import *
p = FunctionPlot(domain='0:1000', vars=['n'])
p.add('n', color='red', python=1, pin='above', pin_message='', num_points=2)
p.add('log(n,2)', color='blue', python=1, pin='below', pin_message='')
p.add('log(n,2)**2', color='blue', python=1, pin='above', pin_message='')
p.add('log(n,2)**3', color='blue', python=1, pin='above', pin_message='\lg^3 n')
print(p)
\end{python}
Aarrghhhh! Even $\lg^3 n$ can't beat $n$

When does $n$ beat $\lg^k n$?
Well
\begin{align*}
                 & n \geq \lg^k n \\
\iff \,\,\,\,\,\,& \lg n \geq \lg(\lg^k n) \\
\iff \,\,\,\,\,\,& \lg n \geq k \lg(\lg n) \\
\iff \,\,\,\,\,\,& \frac{\lg n}{\lg(\lg n)} \geq k \\
\end{align*}
Now why is the last fact a good fact (if it's true at all)?
Because the left-hand-side is independent of $k$, that's why.
It's a comparison of $\lg n$ and $\lg\lg n$.
If you look at the above graph, you'd see that it's likely that 
$\lg n/\lg\lg n$ grows unboundedly, i.e. we guess:
\[
\lim_{n \rightarrow \infty} \frac{\lg n}{\lg \lg n} = \infty
\]
This looks like a job for l'H\^opital's rule again.
\[
\lim_{n \rightarrow \infty} \frac{\lg n}{\lg \lg n}
= \lim_{n \rightarrow \infty} \frac{1/n}{1/(\lg n) \cdot (1/n)}
= \lim_{n \rightarrow \infty} \lg n
= \infty
\]

This means that 
\[
\lg^k n = O(n)
\]
for all $k \geq 0$
(for $k = 0$, $\lg^0 n = 1$ and for $k = 1$, $\lg^1 n = \lg n$.)
Of course
\[
\lg^k n = O(\lg^{k+1} n)
\]

So to our complexity classes
\[
1 = O(n), 
n = O(n^2) 
n^2 = O(n^3) \cdots
\]
we can now add :
\begin{align*}
1 &= O(\lg n), \lg 2 = \lg^2 n, \lg^2 n = O(n) \\
n &= O(n\lg n), n \lg n = O(n\ln^2 n),  n\ln^2 n = O(n^2) \\
n^2 &= O(n^2\lg n), n^2 \lg n = O(n\ln^2 n),  n\ln^2 n = O(n^3) \\
&\cdots \end{align*}
In other words in $n^r =  O(n^{r+1})$ we expand to get
\[
n^r
= O(n^r\lg n),
n^r\lg n = O(n^r\ln^2 n),  
n^r\ln^2 n = O(n^r\ln^3 n)
\cdots 
\]







\begin{thm} \mbox{}
 \begin{enumerate}
  \item[(a)] $\log_a n = O(n)$.
 \end{enumerate}
\end{thm}

\textit{Proof}


(d) This following from l'H\^opitals rule:
\[
\lim_{x \rightarrow \infty} \frac{f(x)}{g(x)}
= 
\lim_{x \rightarrow \infty} \frac{f'(x)}{g'(x)}
\]
if the right hand side makes sense.
Therefore
\[
\lim_{n \rightarrow \infty} \frac{\ln n}{n}
= 
\lim_{n \rightarrow \infty} \frac{1/n}{1}
= 
\lim_{n \rightarrow \infty} \frac{1}{n}
= 0
\]
This implies that for any $\ep > 0$, there is some $N$ such that
for $n \geq N$,
$|(\ln n)/n| \leq \ep$, i.e.,
\[
|\ln n| \leq \ep |n|
\]
In particular, for $C = 1$, there is some $N$ such that
for $n \geq N$
\[
|\ln n| \leq C|n|
\]
Hence $\ln n = O(n)$.
Since 
\[
\log_a n = \frac{\ln n}{\ln a}
\]
we also have
$\log_a n = O(n)$.
\qed

By the way, the proof technique using l'H\^opital's rule
proves more generally that if 
$f(n)$ and $g(n)$ are positively valued for all large $n$,
and such that
\[
\lim_{n\rightarrow \infty} \frac{f(n)}{g(n)} = C
\]
for a constant $C$, then $f = O(g)$.
Why?
Because if that's the case, 
then for every $\ep > 0$, 
there is some $N$
such that for $n \geq N$, we have
\[
\left|
\frac{f(n)}{g(n)} - C
\right| < \ep
\]
Therefore if we choose $\ep = 1$, then
there is some $N$ such that
for $n \geq N$,
\[
\left|
\frac{f(n)}{g(n)} - C
\right| < 1
\]
i.e.,
\[
-1 <
\frac{f(n)}{g(n)} - C
 < 1
\]
i.e.,
\[
\frac{f(n)}{g(n)} < C + 1
\]
and therefore
\[
f(n) < (C + 1) g(n)
\]


\begin{ex}
Prove that $\ln (n^3) = O(\ln n)$. 
\qed
\end{ex}

\begin{ex}
Prove that $\ln (1000 + n^3) = O(\ln n)$. 
\qed
\end{ex}

\begin{ex}
Prove that $\ln (1000n + n^{1000}) = O(\ln n)$. 
\qed
\end{ex}

\begin{ex}
Prove that $\ln (n + 10^{10}) = O(\ln n)$. 
\qed
\end{ex}

\begin{ex} 
Find the best big-O in the form of $O(n^a\ln^b n)$ for the following function:
$\ln (10^{42} n^{12})$
\qed
\end{ex}

\begin{ex} 
Find the best big-O in the form of $O(n^a\ln^b n)$ for the following function:
$(n + 10)\ln (20 n^{12})$
\qed
\end{ex}

\begin{ex}
Find the best big-O in the form of $O(n^a\ln^b n)$ for the following function:
$\ln (10^{42} n^{12} + n^5)$. 
\qed
\end{ex}

\begin{ex}
Find the best big-O in the form of $O(n^a\ln^b n)$ for the following function:
$\ln (10^n + 4)$.
\qed
\end{ex}

\begin{ex}
Find the best big-O in the form of $O(n^a\ln^b n)$ for the following function:
$\ln (n^{n + \ln n} + \ln n) + n^2$. 
\qed
\end{ex}

\begin{ex}
Find the best big-O in the form of $O(n^a\ln^b n)$ for the following function:
$n^2 \ln (n^{n + 5} + 4)$. 
\qed
\end{ex}

\begin{ex}
Find the best big-O in the form of $O(n^a\ln^b n)$ for the following function:
$n^2 \ln (n^{n + 5} + n^6) + n^3$. 
\qed
\end{ex}

\begin{ex}
The following are plots of $y = n^{0.1}$ and $y = \ln n$
for $0 \leq n \leq 1000$:
\begin{python}
from latextool_basic import *
p = FunctionPlot(domain='0:1000', vars=['n'])
p.add('log(n,e)', color='blue', python=1, pin='below', pin_message='\ln n')
p.add('n**0.1', color='red', python=1, pin='above', pin_message='n^{0.1}')
print(p)
\end{python}
Prove or disprove $n^{0.1} = O(\ln n)$?
\qed
\end{ex}

