%-*-latex-*-
\tinysidebar{\debug{exercises/{bubblesort-3/question.tex}}}
The following is a variant of the bubblesort:
\begin{Verbatim}[frame=single, fontsize=\small]
for i = n - 2, ..., 0:
    swap = FALSE
    for j = 0 to i
        if a[i] < a[i + 1]:
            swap = true
            t = a[i]
            a[i] = a[j]
            a[j] = t
    if !swap:
        break
\end{Verbatim}
The point is that if there are no swaps in a pass, then
the array is already sorted.
The boolean variable \verb!swap! is used to remember if swaps
occurred during a pass.
Therefore if \verb!swap! is FALSE after a pass, the algorithm stops.
\begin{itemize}
\item[(a)] Compute the big-O of the best runtime of the above algorithm.
[Obviously the best case occurs when \verb!swap! is FALSE at the end of
the first pass.]
\item[(b)] Compute the big-O of the worst runtime of the above algorithm.
\end{itemize}
