%-*-latex-*-
\section{$n + c = O(n)$, $c \geq 0$ (continued)} 

... now let's complete this problem:

\begin{eg}
Let $f(n) = n$ and $g(n) = n + 1$.
Prove that $f(n) = O(g(n))$.
\end{eg}

\textit{Solution}.
We have the following fact:
\[
n \leq n + 1 \tag{1}
\]
Let $C = 1$ and $N = 0$.
Let $n \geq N$.
\begin{align*}
n &\geq N = 0 \\
\THEREFORE |n| &= n \tag{2}
\end{align*}
We also have
\begin{align*}
n &\geq N = 0 \\
\THEREFORE n + 1 &\geq n \geq 0 & & \text{by (1)} \\
\THEREFORE |n + 1| &= n + 1 \tag{3}
\end{align*}
We have:
\begin{align*}
n &\leq n + 1 & & \text{by (1)}\\
\THEREFORE |n| &\leq |n + 1| & & \text{by (2) and (3)}\\
\THEREFORE |n| &\leq 1 \cdot |n + 1| \\
\THEREFORE |n| &\leq C \cdot |n + 1| \\
\THEREFORE |f(n)| &\leq C \cdot |g(n)| \\
\end{align*}
Hence $f(n) = O(g(n))$. 
\qed

\begin{ex}
Let $f(n) = n$ and $g(n) = n + 10$.
Prove that $f(n) = O(g(n))$.
\end{ex}

\begin{ex}
Let $f(n) = n + 2$ and $g(n) = n + 8$.
Prove that $f(n) = O(g(n))$.
\end{ex}


Now let's see if we can prove $f(n) = O(g(n))$ if
$f(n) = n + 8$ and $g(n) = n + 2$.
Again we want to find $C$ and $N$ such that for 
$n \geq N$ such that 
\[
|f(n)| \leq C|g(n)|
\]
This means that I want
\[
|n + 8| \leq C|n + 2|
\]
Again to simplify the inequality, we can once again force $n + 8$
and $n + 2$ to be positive if $N = 0$.
Since in the case when $n \geq N$, both $n + 8$ and $n + 2$ are positive.
With $N = 0$, the above inequality becomes
\[
n + 8 \leq C(n + 2)
\]
Of course choosing $C = 1$ won't work since
\[
n + 8 \leq 1(n + 2) = n + 2
\]
is obviously not true for \textit{any} value of $n$.
We all know that $n + 8$ is definitely greater than $n + 2$!!!

\begin{python}
from latextool_basic import *
p = FunctionPlot(line_width=1, width='5in', height='3in', domain='0:10')
p.add('x + 2', color='red', python=1, pin='below')
p.add('x + 8', color='green', python=1, pin='below')
print(p)
\end{python}

Arrgh!!! ...

... But don't forget that we can choose $C$.
If we make $C$ large enough the above inequality \textit{can} be achieved.
Let's try $C = 2$.
Then 
\[
n + 8 \leq C(n + 2)
\]
becomes
\[
n + 8 \leq 2(n + 2) = 2n + 4
\]
which is the same as
\[
4 \leq n
\]
So if we insist that $n \geq 4$, we must have
\[
n + 8 \leq C(n + 2)
\]
Can we assume that $n \geq 4$?
Yes we can!
Why?
Because if we choose $N = 4$ instead of $N = 0$,
then when $n \geq N = 4$,
we would have what we want.

Here's a plot of $y = 2(n+2)$ matching $y = n + 8$
at $n = 4$ and beating it for $n >4$:
\begin{python}
from latextool_basic import *
p = FunctionPlot(line_width=1, width='5in', height='3in', domain='0:10')
p.add('x + 2', color='red', python=1, pin='below')
p.add('2*(x + 2)', color='blue', python=1, pin='left')
p.add('x + 8', color='green', python=1, pin='below')
print(p)
\end{python}


Now we're ready to finish this problem:

\begin{eg}
Let $f(n) = n + 8$ and $g(n) = n + 2$.
Then $f(n) = O(g(n))$.
\end{eg}

\textit{Solution.}
Let $N = 4$ and $C = 2$.
Let $n \geq N$.
Then
\begin{align*}
           n &\geq N = 4 \\
\THEREFORE 4 &\leq n \\
\THEREFORE 8 &\leq n + 4 \\
\THEREFORE n + 8 &\leq n + n + 4 \\
\THEREFORE n + 8 &\leq 2n + 4 \\
\THEREFORE n + 8 &\leq 2(n + 2) \\
\THEREFORE |n + 8| &\leq 2 \cdot |n + 2| \\
\THEREFORE |f(n)| &\leq C \cdot |g(n)|
\end{align*}
Hence $f(n) = O(g(n))$.
\qed

Note that we choose $C = 2$ to beef up $n + 2$ so that it can be greater
than $n + 8$.
Now I wonder if $C = 3$ works?
Let's try it.

Recall that I want
\[
n + 8 \leq C(n + 2)
\]
for $n \geq N$.
For $C = 3$, the above inequality becomes:
\[
n + 8 \leq 3(n + 2) = 3n + 6
\]
which is the same as
\[
2 \leq 2n
\]
which is the same as 
\[
1 \leq n
\]
So in this case (i.e. if we choose $C = 3$), we can achieve
the inequality
\[
n + 8 \leq C(n + 2)
\]
if $n \geq 1$, which \textit{is} the case if $N = 1$. 

\begin{python}
from latextool_basic import *
p = FunctionPlot(line_width=1, width='5in', height='4in', domain='0:10')
p.add('x + 2', color='red', python=1, pin='below')
p.add('2*(x + 2)', color='blue', python=1, pin='left', pin_x=9)
p.add('x + 8', color='green', python=1, pin='below', pin_x=9)
p.add('3*(x + 2)', color='black', python=1, pin='below', pin_x=9)
print(p)
\end{python}

So here's another solution to the above exercise:

\textit{Alternate Solution.}
Let $N = 1$ and $C = 3$. Then
for $n \geq N = 1$
\begin{align*}
1 &\leq n \\
\THEREFORE 2 &\leq 2n \\
\THEREFORE n + 2 &\leq n + 2n \\
\THEREFORE n + 2 &\leq 3n \\
\THEREFORE n + 8 &\leq 3n + 6\\
\THEREFORE n + 8 &\leq 3(n + 2)\\
\THEREFORE |n + 8| &\leq 3 \cdot |n + 2|\\
\THEREFORE |f(n)| &\leq C \cdot |g(n)|
\end{align*}
Therefore $f(n) = O(g(n))$.
\qed

\begin{ex}
For the above same problem, what is a good choice of $N$ 
if we choose $C = 4$?
\qed
\end{ex}

\begin{ex}
For the above same problem, what is a good choice of $N$ 
if we choose $C = 11/2$?
\qed
\end{ex}

Do you see that as long as $C > 1$, for large values of $n$,
$C(n + 2)$ must \lq\lq overcome'' $n + 8$, i.e.,
\[
n + 8 \leq C \cdot (n + 2)
\]
for positive $n$.

\begin{ex}
For the above same problem, what is a good choice of $N$ 
if we choose $C = 1.1$?
\qed
\end{ex}

\begin{ex}
For the above same problem, what is a good choice of $N$ 
if we choose $C = 1.01$?
\qed
\end{ex}

All the above indicates that as long as $C > 1$, there \textit{must} be 
an $N$ such that for $n \geq N$, we must have
\[
n + 8 \leq C \cdot (n + 2)
\]
Let's see if that's really the case.

Suppose $C > 1$, then
\[
n + 8 \leq C \cdot (n + 2)
\]
is the same as
\[
n + 8 \leq Cn + 2C
\]
which is the same as
\[
8 - 2C \leq Cn - n
\]
which is the same as
\[
2(4 - C) \leq (C - 1)n
\]
which is the same as
\[
\frac{2(4 - C)}{C - 1} \leq n
\]
(Note that since $C > 1$, $C - 1$ is greater than $0$.
Therefore the above inequality is correct.)
In summary, 
\[
n + 8 \leq C \cdot (n + 2)
\]
is true if
\[
\frac{2(4 - C)}{C - 1} \leq n
\]
Therefore if we set 
\[
N = \frac{2(4 - C)}{(C - 1)}
\]
and let $n \geq N$, then
\[
n + 8 \leq C \cdot (n + 2)
\]
must hold.

