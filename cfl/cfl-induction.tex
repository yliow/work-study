\sectionthree{Proving the Language is Generated by a CFG}
\begin{python0}
from solutions import *; clear()
\end{python0}

Usually once you've written down the CFG $G$, if the grammar is simple enough,
i.e., there aren't too many production rules and the right-hand side of the 
production rules are not too complicated, the
language accepted by $G$, i.e., $L(G)$ is usually pretty easy to write down.

However when the CFG is complicated, then you will need to prove the
the language generated by $G$ is what it is.

This is usually done using ... mathematical induction!!! 
(Were you expecting something else???)
I'll do a simple example so that you can follow the technique.

Let's consider this grammar:
\[
G : S \rightarrow aSb \mid \epsilon
\]
It should be clear by now that 
\[
L(G) = \{ a^nb^n \mid n \geq 0 \}
\]
But let's prove it using induction anyway.

First let me label the rules so that they are easier to reference:
$G$ is the CFG with the following production rules:
\begin{align*}
S &\rightarrow aSb \tag{1} \\
S &\rightarrow \ep \tag{2} \\
\end{align*}
We want to prove
\[
L(G) = \{ a^nb^n \mid n \geq 0\}
\]
Let's give the right-hand side a name (so that it's easier to write):
Let
\[
L = \{ a^nb^n \mid n \geq 0\}
\]
I want to prove 
\[
L(G) = L
\]
This means proving 
\[
L(G) \subseteq L
\,\,\, \text{ and } \,\,\,
L(G) \supseteq L
\]

\textsc{Part 1.}
First let me prove
\[
L(G) \supseteq L
\]
Let $x \in L$.
Then 
\[
x = a^n b^n
\]
for some $n \geq 0$ by definition of $L$.
I want to prove that 
\[
a^n b^n \in L(G)
\]
To prove this by induction, I define $P(n)$ to be the statement
\[
P(n): \,\,\, a^n b^n \in L(G)
\]
Now I will prove $P(n)$ is true for all $n \geq 0$ by induction.
First I show that $P(0)$ is true. This means I want to show
\[
a^0 b^0 \in L(G)
\]
Since $a^0 b^0 = \ep$, this means I need to prove
\[
\ep \in L(G)
\]
This means I need to find a derivation (using $G$) for $\ep$.
That's easy:
\[
S \implies \ep
\]
by production rule (2).
Therefore $P(0)$ is true.

Now I assume $P(n)$ is true; the goal is to prove $P(n+1)$.
Since $P(n)$ is true, we have
\[
a^n b^n \in L(G)
\]
This means
\[
S \implies^{\hspace{-0.2cm}*}\ a^nb^n \tag{3}
\]
Using production rule (1) and the above (3), we have
\begin{align*}
S 
&\implies aSb                            & & \text{by production rule (1)} \\
&\implies^{\hspace{-0.2cm}*}\ a(a^nb^n)b & & \text{by (3)}
\end{align*}
Therefore
\[
S \implies^{\hspace{-0.2cm}*}\ a(a^nb^n)b = a^{n+1} b^{n+1}  
\]
This implies that $a^{n+1}b^{n+1} \in L(G)$.
Therefore $P(n+1)$ is true.

By the principle of Mathematical induction, $P(n)$ is true for all 
$n \geq 0$.
We have proven that $L \subseteq L(G)$.


\textsc{Part 2.}
Now I want to prove 
\[
L(G) \subseteq L
\]
Let $x \in L(G)$.
I need to show that $x \in L$.
Since $x \in L(G)$, we have
\[
S \implies^{\hspace{-0.2cm}*}\ x \in L(G)
\]
I need to show that $x \in L$.
To show $x \in L$ is the same as showing that $x = a^n b^n$
for some $n$.
All in all, I need to show:
\[
\text{If } 
S \implies^{\hspace{-0.2cm}*} \ x,
\text{ then }
x = a^n b^n \text{ for some } n \geq 0
\]
The induction in this case is usually on the length of the
derivation.
In others the induction hypothesis is:
\[
P(k): 
\text{If } 
S \implies^{\hspace{-0.2cm} k}\ x \in L(G),
\text{ then }
x = a^n b^n
\]
for $k \geq 1$.
(Of course this $P$ has nothing to do with the previous
induction hypothesis $P$; to disambiguate, you can use something like $P'$
if you like.)
Make sure you see that the induction is on $k$ and not $n$!!!
The two might be the same or totally different for different problem.
One is the length of the derivation and another is a parameter describing
the form of the word in our language $L$.

However instead of proving 
\[
P(k): 
\text{If } 
S \implies^{\hspace{-0.2cm} k}\ x \in L(G),
\text{ then }
x = a^n b^n
\]
by induction, 
I will prove this instead:
if 
\[
S \implies^{\hspace{-0.2cm} k}\ x_1 S x_2 
\]
then $x_1 = a^n$ and $x_2 = b^n$ for some $n \geq 0$ by 
induction on $k$.
(I let you think about why I prefer this.)
So let's go ahead:

Let $P(k)$ be the following statements:
\[
P(k):
\text{If } 
S \implies^{\hspace{-0.2cm} k}\ x_1 S x_2 
\text{ then } x_1 = a^n, x_2 = b^n 
\text{ for some } n \geq 0 
\]

Note that if $k = 0$, then $x_1 = \ep = x_2$ and we're done.
Suppose $k > 0$.
If $k = 1$, then the first production rule was use and 
\[
S \implies aSb = x_1 S x_2
\]
This implies that $x_1 = a$ and $x_2 = b$.
Let $k > 1$.
Assume that $P(0), ..., P(k)$ is true.
Let
\[
S \implies^{\hspace{-0.2cm} k+1}\ x_1 S x_2 
\]
Since only the first production rule was used, we have
\[
S \implies^{\hspace{-0.2cm} k}\ y_1 S y_2 \implies y_1a S by_2 = x_1 S x_2 
\]
Now since 
\[
S \implies^{\hspace{-0.2cm} k}\ y_1 S y_2 
\]
is a derivation of length $k$, we have $y_1 = a^n$ and $y_2 = b^n$
for some $n \geq 0$.
Therefore $x_1 = y_1a = a^{n+1}$ and $x_2 = by_2 = bb^n = b^{n+1}$.
 
Therefore by the principle of mathematical induction, 
if 
\[
S \implies^{\hspace{-0.2cm} k}\  x_1 S x_2
\]
then $x_1 = a^n$ and $x_2 = b^n$.

Now I'll prove $L(G) \subseteq L$.
Let $x \in L(G)$.
This means that 
\[
S \implies^{\hspace{-0.2cm} k}\ x
\]
for some $k$.
Note that the above sequence of derivations is broken up as follows:
\[
S \implies^{\hspace{-0.2cm} k - 1}\ x_1 S x_2 \implies x
\]
(The last production rule used in the above is the second production rule.
To be absolutely formal, you can produce by induction
that is a sequence of derivations does not use the second rule,
then the resulting sentential form must contain the variable $S$.
This implies that if you reach a string of terminals, the second rule
must be used. I'll leave that fun stuff to you.)
Previous, I have shown that $x_1 = a^n$ and $x_2 = b^n$.
The above becomes
\[
S \implies^{\hspace{-0.2cm} k - 1} \ a^n S b^n \implies x
\]
Since the last production rule is $S \rightarrow \ep$, the above is also
\[
S \implies^{\hspace{-0.2cm} k - 1} \ a^n S b^n \implies a^n \ep b^n
\]
which means that 
\[
x = a^n b^n
\]

\newpage
\begin{ex}
Why did I prove 
\[
\text{
\lq\lq If $S \implies^{\hspace{-0,2cm} *}\ x_1S x_2$ then $x_1 = a^n$ and $x_2 = b^n$'' 
}
\]
instead of 
\[
\text{
\lq\lq If $S \implies^{\hspace{-0.3cm} *}\ x$ then $x = a^nb^n$''
}
\] 
directly? 
(Hint: Try to prove the second version by induction yourself
and see what happens.)
\qed
\end{ex}
