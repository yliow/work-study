\section{Regular $\subseteq$ CFL}

In this section, we will look at the big picture:
is there a relationship between regular languages and CFLs?

Well ... if you've been reading the notes carefully you should
know this by now ...

\begin{thm}
Every regular language is context free.
\end{thm}

Let $L$ be a regular language.
There is therefore a DFA, say $M$, accepting $L$.
A DFA is simply a PDA that doesn't use its stack!!!
In other words you take the state diagram of $M$, and relabel each transition
\[
p \xrightarrow{a} q
\]
by
\[
p \xrightarrow{a, \ep \rightarrow \ep} q
\]
Sweet isn't it? 
(Of course the same construction works for an NFA too.)
