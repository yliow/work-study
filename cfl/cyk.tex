\section{Cocke--Younger--Kasami Algorithm}

The
\defterm{Cocke--Younger--Kasami algorithm} (\defterm{CYK})
\tinysidebar{Cocke--Younger--Kasami algorithm\\CYK}
is an algorithm that determine if a string is generated by a CFG.
And if a string is generated by the CFG,
the algorithm can also give all the possible derivations.
In other words, it is a string parsing algorithm.
The time complexity is $O(n^3)$ where $n$ is the length of the string
(details below).
Later, faster algorithms were discovered.
For instance there is one with time complexity $O(n^{2.8})$.
The algorithm uses a very important technique in the design of algorithms
-- dynamic programming technique -- it builds up a solution by combining
solutions of subproblems of the given problem. 
Dynamic programming is an extremely important algorithm design technique
an is frequently used even in research.
(So it's not just a neat trick used in basic undergrad courses.)
One CYK dynamic programming example is definitely not enough.
See CISS358 for more details on dynamic programming.

Of course you can parse a string $w$ without CYK:
If $w$ is $\ep$, just check if $S$ is a nullable symbol.
Assuming the CFG is in Chomsky normal form, then every
production will increase the length of the sentential form
or replace a variable with a terminal.
So you need $|w|$ productions to introduce $|w|$ variables into the
derivation process and you need $|w|$ productions to replace the variables
with terminals. So there are at most $2|w|$ derivation steps, which is
of course finite.
So you just check every possible sequence of $2|w|$ productions starting
with $S$ and see if you can arrive at $w$.
Is this a good algorithm?
(Note sarcasm ...)

The main idea of CYK is very simple.
First we assume the grammar $G$ is in Chomsky normal form.
\sidebar{{\scriptsize Chomsky, Chomsky, Chomsky!}}
Now suppose our string $w$ has length $n$.
Let $w_{i,j}$ denote the substring of $w$ starting at position $i$ with 
length $j$.
The first position is $1$ and the last position is $n$.
The string $w$ is just
\[
w = w_{1,1} w_{2,1} w_{3,1} w_{4,1} \cdots w_{n,1}
\]

\begin{eg} 
Let $w = baabaa$. Then $w_{2,3} = aab$.
\end{eg}


\newpage
\begin{ex}
  \begin{tightlist}
    \item
      Let $w = baabaa$. What is $w_{2,3}$? What about $w_{5,2}$?
      What about $w_{2,5}$?
      What is the relationship between the three?
    \item
      If $w_{i,j}$ is the concatenation of $w_{p,q}$ and $w_{r,s}$,
      can you tell me something about $p,q,r,s$ in terms of $i,j$?
    \end{tightlist}
\end{ex}




\newpage
First of all we can always determine if each $w_{i,1}$ can be
derived:
$A$ derives $w_{i,1}$ if and only if $A \rightarrow w_{i,1}$ is a production
(don't forget that we're assuming the grammar is in Chomsky Normal Form).
\sidebar{{\scriptsize Chomsky, Chomsky, Chomsky!}}


What about the general case?
How do you generate $w_{i,j}$?
This is a string starting at position $i$ and of length $j$.
What happens when you cut up the string?
Suppose you break it up into two strings $w_{i,j} = xy$.
Well, $x$ starts at the same index position in $w$ as $w_{i,j}$ right?
So it must be of the form
\[
x = w_{i,k}
\]
Since $x$ is a substring of $w_{i,j}$, it's length must be at most $j$.
Therefore $k \leq j$.
But let's assume that we really did cut up the string so that $y$ is not the 
empty string.
So $k < j$.
OK.
So we have
\[
  w_{i,j} = w_{i,k} y \text{ where } k < j
\]
Now let's think about $y$.

$y$ starts at position $i + k$, right?
And what is the length of $y$?
It's $|w| - |x|$ which is $j - k$.
So
\[
  y = w_{i + k, j - k}
\]
Viola.

Altogether, 
$w_{i,j}$ is the concatenation of $w_{i,k}$ and $w_{i+k, j-k}$
where $k < j$ (i.e., $k = 1, 2, \ldots, j - 1$):
\[
w_{i,j} = w_{i,k} w_{i+k, j-k}, \hskip 1cm (k = 1, 2, \ldots, j - 1)
\]
So what we'll do is
to find a variable, say $X$, deriving $w_{i,k}$:
\[
  X \istar w_{i,k}
\]
and 
another, say $Y$, deriving $w_{i+k, j-k}$
\[
  X \istar w_{i,k}
\]
for all the given $k$'s.
That means that $XY$ derives $w_{i,j}$
\[
  XY \istar w_{i,j}
\]
If the grammar is in Chomsky Normal Form, we will then look at the 
production rules to find a rule of the form $Z \rightarrow XY$.
If this is found, then $Z$ must derive $XY$ which derives $w_{i,j}$:
\[
  Z \implies XY \istar w_{i,j}
\]
That's the main idea.

The important thing to understand is that we are breaking down
out $w$ into two substrings exactly because the grammar is in
Chomsky Normal Form.
\sidebar{{\scriptsize Chomsky, Chomsky, Chomsky!}}
If the grammar has a production rule of the form
$Z \rightarrow WXY$, then our string $w$
might be derived through this production rule and in that case
we would have to break up $w$ into \textit{three} substrings
where $W$ derives the first, $X$ derives the second, and $Y$
derives the third.
The more variations there are in the right hand side of the
production rules of the grammar, the more cases
we would need to analyze to cover
all possibilities in the derivation process.
Get it?

Now let's go through an example ...


\newpage
\begin{eg}
Consider the grammar:
\begin{align*}
S &\rightarrow AB \st AC \\
A &\rightarrow BC \st a \\
B &\rightarrow CB \st b \\
C &\rightarrow AA \st b
\end{align*}
Note that this is already in Chomsky Normal Form.
Is $baaab$ generated by the grammar?
\end{eg}

\SOLUTION
For this example $w = baaab$.
We will systematically tabulate the variables deriving $w_{i,j}$.
The variables deriving $w_{i,j}$ will be placed at row $j$ and column $i$:
\begin{python}
from latextool_basic import *

m = [['','','','','',''],
     ['','','','','',''],
     ['','','','','',''],
     ['','','','','',''],
     ['','','','','',''],
     ['','','','','',''],
     ]
p = Plot()
cyk(p, m, w='baaaab')
print(p)
\end{python}


\underline{First Row}:
We first work out $w_{i,1} = b$:
\begin{python}
from latextool_basic import *

m = [['?','','','','',''],
     ['','','','','',''],
     ['','','','','',''],
     ['','','','','',''],
     ['','','','','',''],
     ['','','','','',''],
     ]
p = Plot()
cyk(p, m, w='baaaab')
print(p)
\end{python}

$b$ is derived only by $B \rightarrow b$ and $C \rightarrow b$:
\begin{python}
from latextool_basic import *

m = [['$\{B,C\}$','','','','',''],
     ['','','','','',''],
     ['','','','','',''],
     ['','','','','',''],
     ['','','','','',''],
     ['','','','','',''],
     ]
p = Plot()
cyk(p, m, w='baaaab', fontsize='small')
print(p)
\end{python}
If you are interested in not just knowing if $baaaab$ is generated by the given grammar,
but you actually want to write down the derivation, it's helpful to include
some addition information for each variable in the cells.
I'll write these extra information in subscript.
I'll show you how these extra information will be used later.
\begin{python}
from latextool_basic import *

m = [['$\{B_b,C_b\}$','','','','',''],
     ['','','','','',''],
     ['','','','','',''],
     ['','','','','',''],
     ['','','','','',''],
     ['','','','','',''],
     ]
p = Plot()
cyk(p, m, w='baaaab', fontsize='small')
print(p)
\end{python}
In the above, the $B_b$ means \lq\lq $B$ will derive $b$''.

$a$ is derived by $A \rightarrow a$. This allows us to complete
the first row:
\begin{python}
from latextool_basic import *

m = [['$\{B_b,C_b\}$','$\{A_a\}$','$\{A_a\}$','$\{A_a\}$','$\{A_a\}$','$\{B_b,C_b\}$'],
     ['','','','','',''],
     ['','','','','',''],
     ['','','','','',''],
     ['','','','','',''],
     ['','','','','',''],
     ]
p = Plot()
cyk(p, m, w='baaaab',fontsize='small')
print(p)
\end{python}

In summary, to fill the first row, you need only to look at the
productions of the form $\langle variable \rangle \rightarrow \langle terminal
\rangle$.
(Don't memorize this. You should understand the goal and the math involved.)

\underline{Second Row}:
Now let's look at the $j=2, i=1$, i.e. $w_{1, 2}$:
\begin{python}
from latextool_basic import *

m = [['$\{B_b,C_b\}$','$\{A_a\}$','$\{A_a\}$','$\{A_a\}$','$\{A_a\}$','$\{B_b,C_b\}$'],
     ['?','','','','',''],
     ['','','','','',''],
     ['','','','','',''],
     ['','','','','',''],
     ['','','','','',''],
     ]
p = Plot()
cyk(p, m, w='baaaab',fontsize='small')
print(p)
\end{python}
Now
\[
w_{1,2} = w_{1,1} w_{2,1}
\]
(There's only one way to cut up $w_{1,2}$ into two proper substrings.)
But we already know that the variable deriving $w_{1,1}$ is at the $(1,1)$
entry in the table: it's either $B$ or $C$.
We also know that the variable deriving $w_{2,1}$ is at the $(1,2)$ entry;
this is $A$.
Therefore we want to find a variable $V$ such that 
\[
V \rightarrow BA
\]
or
\[
V \rightarrow CA
\]
is a production rule.
(Why do we consider only this scenario? Let's hear it ... 
\textit{ Chomsky, Chomsky, Chomsky, ...}
$BA$ and $CA$ does not appear on the right-hand side of the productions in 
the grammar.
READ THIS PARAGRAPH AGAIN!
So there are no such $V$s.
We will indicate this by putting $\emptyset$ at $(2, 1$).

The cells we looked at in order to fill the entry at $(2,1)$ are shaded.
\begin{python}
from latextool_basic import *

p = Plot()
m = [['$\{B_b,C_b\}$','$\{A_a\}$','$\{A_a\}$','$\{A_a\}$','$\{A_a\}$','$\{B_b,C_b\}$'],
     ['$\emptyset$','','','','',''],
     ['','','','','',''],
     ['','','','','',''],
     ['','','','','',''],
     ['','','','','',''],
     ]
cyk(p, m, w='baaaab', background={(0,0):'blue!10', (0,1):'blue!10'},fontsize='small')
print(p)
\end{python}

Using the same idea, we can fill in the entry at $(2,2)$:
\begin{python}
s = r'''
from latextool_basic import *

m = [['$\{B_b,C_b\}$','$\{A_a\}$','$\{A_a\}$','$\{A_a\}$','$\{A_a\}$','$\{B_b,C_b\}$'],
     ['$\emptyset$','$\{C_{(1,2),(1,3)}\}$','','','',''],
     ['','','','','',''],
     ['','','','','',''],
     ['','','','','',''],
     ['','','','','',''],
     ]
p = Plot()
cyk(p, m, w='baaaab', background={(0,1):'blue!10', (0,2):'blue!10'},fontsize='small')
print(p)
'''
from latextool_basic import *
execute(s)
\end{python}
Here, $C$ derives $AA$.
The subscript data of $C$, i.e. $(1,2)$ and $(1,3)$, tells us that $C$
derives $XY$ where variable $X$ is in cell $(1,2)$ and
$Y$ is in cell $(1, 3)$.

Here's the table with row 2 completed:
\begin{python}
from latextool_basic import *

p = Plot()
m = [['$\{B_b,C_b\}$','$\{A_a\}$','$\{A_a\}$','$\{A_a\}$','$\{A_a\}$','$\{B_b,C_b\}$'],
     ['$\emptyset$','$\{C_{(1,2),(1,3)}\}$','$\{C_{(1,3),(1,4)}\}$','$\{C_{(1,4),(1,5)}\}$','$\{S_{(1,5),(1,6)}\}$',''],
     ['','','','','',''],
     ['','','','','',''],
     ['','','','','',''],
     ['','','','','',''],
     ]
cyk(p, m, w='baaaab',fontsize='small')
print(p)
\end{python}

Again, recall the aim: We're trying to fill the $(j,i)$ 
entry of the table with a variable that can derive $w_{i,j}$.

\underline{Third Row}:
Now we look at $j = 3, i = 1$:
\begin{python}
from latextool_basic import *

m = [['','','','','',''],
     ['','','','','',''],
     ['','','','','',''],
     ['','','','','',''],
     ['','','','','',''],
     ['','','','','',''],
     ]
p = Plot()
m = [['$\{B_b,C_b\}$','$\{A_a\}$','$\{A_a\}$','$\{A_a\}$','$\{A_a\}$','$\{B_b,C_b\}$'],
     ['$\emptyset$','$\{C_{(1,2),(1,3)}\}$','$\{C_{(1,3),(1,4)}\}$','$\{C_{(1,4),(1,5)}\}$','$\{S_{(1,5),(1,6)}\}$',''],
     ['?','','','','',''],
     ['','','','','',''],
     ['','','','','',''],
     ['','','','','',''],
     ]
cyk(p, m, w='baaaab',fontsize='small')
print(p)
\end{python}
We want to to find a variable $V$ that derives $w_{1,3}$.
Recall from the above that we want to make use of previous results.
The following are the possible ways of breaking up $w_{1,3}$:
\[
w_{1,1}w_{2,2} \hskip 1cm \text{or} \hskip 1cm w_{1,2} w_{3,1}
\]
In other words if $w_{1, 3}$ is $xyz$ then the two ways to break up 
$w_{1,3}$ into two proper substrings are:
\[
x \cdot yz \hskip 1cm \text{or} \hskip 1cm  xy \cdot z
\]
We already know that the variable deriving $w_{1,1}$ is in the
entry at $(1,1)$ of the table: it's $B$ or $C$.
For $w_{2,2}$, just look at the entry at $(2,2)$; $C$ derives 
$w_{2,2}$.
Therefore $BC$ and $CC$ derives $w_{1,1}w_{2,2}$ which is just 
$w_{1, 3}$.
Now we want to find $V$ that derives $BC$ or $CC$:
\[
  V \implies XY \text{ where } X \in \{B, C\} \text{ and } Y \in \{C\} 
\]
and enter that
into the $(3,1)$ entry of the table.
Checking the grammar, we see that only $A$ can do this.
Therefore we have
\begin{python}
from latextool_basic import *

m = [['','','','','',''],
     ['','','','','',''],
     ['','','','','',''],
     ['','','','','',''],
     ['','','','','',''],
     ['','','','','',''],
     ]
p = Plot()
m = [['$\{B_b,C_b\}$','$\{A_a\}$','$\{A_a\}$','$\{A_a\}$','$\{A_a\}$','$\{B_b,C_b\}$'],
     ['$\emptyset$','$\{C_{(1,2),(1,3)}\}$','$\{C_{(1,3),(1,4)}\}$','$\{C_{(1,4),(1,5)}\}$','$\{S_{(1,5),(1,6)}\}$',''],
     ['$\{A_{(1,1),(2,2)}\}$','','','','',''],
     ['','','','','',''],
     ['','','','','',''],
     ['','','','','',''],
     ]
cyk(p, m, w='baaaab', background={(0,0):'blue!10', (1,1):'blue!10'},fontsize='small')
print(p)
\end{python}
We're not done yet because $w_{1,3}$ can also be $w_{1,2} w_{3,1}$.
\begin{python}
from latextool_basic import *

m = [['','','','','',''],
     ['','','','','',''],
     ['','','','','',''],
     ['','','','','',''],
     ['','','','','',''],
     ['','','','','',''],
     ]
p = Plot()
m = [['$\{B_b,C_b\}$','$\{A_a\}$','$\{A_a\}$','$\{A_a\}$','$\{A_a\}$','$\{B_b,C_b\}$'],
     ['$\emptyset$','$\{C_{(1,2),(1,3)}\}$','$\{C_{(1,3),(1,4)}\}$','$\{C_{(1,4),(1,5)}\}$','$\{S_{(1,5),(1,6)}\}$',''],
     ['$\{A_{(1,1),(2,2)}\}$','','','','',''],
     ['','','','','',''],
     ['','','','','',''],
     ['','','','','',''],
     ]
cyk(p, m, w='baaaab', background={(1,0):'red!10', (0,2):'red!10'},fontsize='small')
print(p)
\end{python}
When you check the entry at $(2,1)$, you see that no variable can derive $w_{1,2}$, i.e.,
there is no $V$ such that
\[
  V \implies XY \text{ where } X \in \emptyset, Y \in \{A\}
\]
Therefore we have nothing else to add to the entry at $(3,1)$.
Altogether we have:
\begin{python}
from latextool_basic import *

m = [['','','','','',''],
     ['','','','','',''],
     ['','','','','',''],
     ['','','','','',''],
     ['','','','','',''],
     ['','','','','',''],
     ]
p = Plot()
m = [['$\{B_b,C_b\}$','$\{A_a\}$','$\{A_a\}$','$\{A_a\}$','$\{A_a\}$','$\{B_b,C_b\}$'],
     ['$\emptyset$','$\{C_{(1,2),(1,3)}\}$','$\{C_{(1,3),(1,4)}\}$','$\{C_{(1,4),(1,5)}\}$','$\{S_{(1,5),(1,6)}\}$',''],
     ['$\{A_{(1,1),(2,2)}\}$','','','','',''],
     ['','','','','',''],
     ['','','','','',''],
     ['','','','','',''],
     ]
cyk(p, m, w='baaaab',fontsize='small')
print(p)
\end{python}

Notice that while working on the entry at $(3,1)$, we looked at 
$(1,1),(2,2)$ and also $(2,1),(1,3)$.
Do you notice that there's a pattern in the pairs of cells that
contribute to the two cases in filling the cell at $(3,1)$?
\begin{python}
from latextool_basic import *

p = Plot()
m = [['','','','','',''],
     ['','','','','',''],
     ['?','','','','',''],
     ['','','','','',''],
     ['','','','','',''],
     ['','','','','',''],
     ]
c = cyk(p, m, w='baaaab',
        background={(0,0):'blue!10', (1,1):'blue!10', (1,0):'red!10', (0,2):'red!10'},
        fontsize='small')
p += Line(points=[c[0][0].center(), c[1][0].center()], endstyle='>', linewidth=0.1)     
p += Line(points=[c[1][1].center(), c[0][2].center()], endstyle='>', linewidth=0.1)     
print(p)
\end{python}

Now we move on the the entry at $(3,2)$:
\begin{python}
from latextool_basic import *

m = [['','','','','',''],
     ['','','','','',''],
     ['','','','','',''],
     ['','','','','',''],
     ['','','','','',''],
     ['','','','','',''],
     ]
p = Plot()
m = [['$\{B_b,C_b\}$','$\{A_a\}$','$\{A_a\}$','$\{A_a\}$','$\{A_a\}$','$\{B_b,C_b\}$'],
     ['$\emptyset$','$\{C_{(1,2),(1,3)}\}$','$\{C_{(1,3),(1,4)}\}$','$\{C_{(1,4),(1,5)}\}$','$\{S_{(1,5),(1,6)}\}$',''],
     ['$\{A_{(1,1),(2,2)}\}$',r'?','','','',''],
     ['','','','','',''],
     ['','','','','',''],
     ['','','','','',''],
     ]
cyk(p, m, w='baaaab',fontsize='small')
print(p)
\end{python}

This corresponds to $w_{2,3}$ and
\[
w_{2,3} = w_{2,1}w_{3,2} \hskip 1cm \text{or} \hskip 1cm w_{2,2}w_{4,1}
\] 
First for $w_{2,3} = w_{2,1}w_{3,2}$, we want to find variables that can derive
$AC$; there's only one: $S$.
\begin{python}
from latextool_basic import *
p = Plot()
m = [['$\{B_b,C_b\}$','$\{A_a\}$','$\{A_a\}$','$\{A_a\}$','$\{A_a\}$','$\{B_b,C_b\}$'],
     ['$\emptyset$','$\{C_{(1,2),(1,3)}\}$','$\{C_{(1,3),(1,4)}\}$','$\{C_{(1,4),(1,5)}\}$','$\{S_{(1,5),(1,6)}\}$',''],
     ['$\{A_{(1,1),(2,2)}\}$',r'$\{S_{(1,2),(2,3)}\}$','','','',''],
     ['','','','','',''],
     ['','','','','',''],
     ['','','','','',''],
     ]
cyk(p, m, w='baaaab', background={(0,1):'blue!10', (1,2):'blue!10'},fontsize='small')
print(p)
\end{python}

For $w_{2,3} = w_{2,2}w_{4,1}$, we want to find variables 
that can derive $CA$: there isn't any.
So there's nothing to add:
\begin{python}
from latextool_basic import *

p = Plot()
m = [['$\{B_b,C_b\}$','$\{A_a\}$','$\{A_a\}$','$\{A_a\}$','$\{A_a\}$','$\{B_b,C_b\}$'],
     ['$\emptyset$','$\{C_{(1,2),(1,3)}\}$','$\{C_{(1,3),(1,4)}\}$','$\{C_{(1,4),(1,5)}\}$','$\{S_{(1,5),(1,6)}\}$',''],
     ['$\{A_{(1,1),(2,2)}\}$',r'$\{S_{(1,2),(2,3)}\}$','','','',''],
     ['','','','','',''],
     ['','','','','',''],
     ['','','','','',''],
     ]
cyk(p, m, w='baaaab', background={(1,1):'red!10', (0,3):'red!10'},fontsize='small')
print(p)
\end{python}

Notice the patterns of the pairs of cells that contribute to the 
computation at $(3,2)$:
\begin{python}
from latextool_basic import *

m = [['','','','','',''],
     ['','','','','',''],
     ['','?','','','',''],
     ['','','','','',''],
     ['','','','','',''],
     ['','','','','',''],
     ]
p = Plot()
c = cyk(p, m, w='baaaab', background={(0,1):'blue!10', (1,1):'red!10', (1,2):'blue!10', (0,3):'red!10'},fontsize='small')
p += Line(points=[c[0][1].center(), c[1][1].center()], endstyle='>', linewidth=0.1)
p += Line(points=[c[1][2].center(), c[0][3].center()], endstyle='>', linewidth=0.1)
print(p)
\end{python}

Now following the same method, complete the second row yourself.

\[
  \text{[PAUSE]}
\]

\newpage
The table should now look like this:
\begin{python}
from latextool_basic import *
p = Plot()
m = [['$\{B_b,C_b\}$','$\{A_a\}$','$\{A_a\}$','$\{A_a\}$','$\{A_a\}$','$\{B_b,C_b\}$'],
     ['$\emptyset$','$\{C_{(1,2),(1,3)}\}$','$\{C_{(1,3),(1,4)}\}$','$\{C_{(1,4),(1,5)}\}$','$\{S_{(1,5),(1,6)}\}$',''],
     ['$\{A_{(1,1),(2,2)}\}$',r'$\{S_{(1,2),(2,3)}\}$','$\{S_{(1,3),(2,4)}\}$','$\{B_{(2,4),(1,6)}\}$','',''],
     ['','','','','',''],
     ['','','','','',''],
     ['','','','','',''],
     ]
cyk(p, m, w='baaaab',fontsize='small')
print(p)
\end{python}

\underline{Fourth Row}:
Now we look at the entry for $(4,1)$.
I think you can see the pattern now.
Basically you need to look at the following cases:
\begin{python}
from latextool_basic import *
p = Plot()
m = [['$\{B_b,C_b\}$','$\{A_a\}$','$\{A_a\}$','$\{A_a\}$','$\{A_a\}$','$\{B_b,C_b\}$'],
     ['$\emptyset$','$\{C_{(1,2),(1,3)}\}$','$\{C_{(1,3),(1,4)}\}$','$\{C_{(1,4),(1,5)}\}$','$\{S_{(1,5),(1,6)}\}$',''],
     ['$\{A_{(1,1),(2,2)}\}$',r'$\{S_{(1,2),(2,3)}\}$','$\{S_{(1,3),(2,4)}\}$','$\{B_{(2,4),(1,6)}\}$','',''],
     ['?','','','','',''],
     ['','','','','',''],
     ['','','','','',''],
     ]
cyk(p, m, w='baaaab', fontsize='small',
    background={(0,0):'blue!10',  (2,1):'blue!10',
                (1,0):'red!10',   (1,2):'red!10',
                (2,0):'green!10', (0,3):'green!10',
    })
print(p)
\end{python}
Note the traversal pattern:
\begin{python}
from latextool_basic import *
p = Plot()
m = [['','','','','',''],
     ['','','','','',''],
     ['','','','','',''],
     ['?','','','','',''],
     ['','','','','',''],
     ['','','','','',''],
     ]
c = cyk(p, m, w='baaaab', fontsize='small',
    background={(0,0):'blue!10',  (2,1):'blue!10',
                (1,0):'red!10',   (1,2):'red!10',
                (2,0):'green!10', (0,3):'green!10',
    })
p += Line(points=[c[0][0].center(), c[2][0].center()], endstyle='>', linewidth=0.1)
p += Line(points=[c[2][1].center(), c[0][3].center()], endstyle='>', linewidth=0.1)
print(p)
\end{python}

For
\begin{python}
from latextool_basic import *
p = Plot()
m = [['$\{B_b,C_b\}$','$\{A_a\}$','$\{A_a\}$','$\{A_a\}$','$\{A_a\}$','$\{B_b,C_b\}$'],
     ['$\emptyset$','$\{C_{(1,2),(1,3)}\}$','$\{C_{(1,3),(1,4)}\}$','$\{C_{(1,4),(1,5)}\}$','$\{S_{(1,5),(1,6)}\}$',''],
     ['$\{A_{(1,1),(2,2)}\}$',r'$\{S_{(1,2),(2,3)}\}$','$\{S_{(1,3),(2,4)}\}$','$\{B_{(2,4),(1,6)}\}$','',''],
     ['?','','','','',''],
     ['','','','','',''],
     ['','','','','',''],
     ]
cyk(p, m, w='baaaab',fontsize='small', background={(0,0):'blue!10', (2,1):'blue!10'})
print(p)
\end{python}
we are looking for a variable $V$ such that
\[
  V \rightarrow XY \text{ where } X \in \{B,C\}, Y \in \{S\}
\]
There is no such $V$.
For
\begin{python}
from latextool_basic import *
p = Plot()
m = [['$\{B_b,C_b\}$','$\{A_a\}$','$\{A_a\}$','$\{A_a\}$','$\{A_a\}$','$\{B_b,C_b\}$'],
     ['$\emptyset$','$\{C_{(1,2),(1,3)}\}$','$\{C_{(1,3),(1,4)}\}$','$\{C_{(1,4),(1,5)}\}$','$\{S_{(1,5),(1,6)}\}$',''],
     ['$\{A_{(1,1),(2,2)}\}$',r'$\{S_{(1,2),(2,3)}\}$','$\{S_{(1,3),(2,4)}\}$','$\{B_{(2,4),(1,6)}\}$','',''],
     ['?','','','','',''],
     ['','','','','',''],
     ['','','','','',''],
     ]
cyk(p, m, w='baaaab',fontsize='small', background={(1,0):'red!10', (1,2):'red!10'})
print(p)
\end{python}
since there are no variables at $(2, 1)$, we don't have any $V$'s to add to $(4,1)$.
For
\begin{python}
from latextool_basic import *
p = Plot()
m = [['$\{B_b,C_b\}$','$\{A_a\}$','$\{A_a\}$','$\{A_a\}$','$\{A_a\}$','$\{B_b,C_b\}$'],
     ['$\emptyset$','$\{C_{(1,2),(1,3)}\}$','$\{C_{(1,3),(1,4)}\}$','$\{C_{(1,4),(1,5)}\}$','$\{S_{(1,5),(1,6)}\}$',''],
     ['$\{A_{(1,1),(2,2)}\}$',r'$\{S_{(1,2),(2,3)}\}$','$\{S_{(1,3),(2,4)}\}$','$\{B_{(2,4),(1,6)}\}$','',''],
     ['?','','','','',''],
     ['','','','','',''],
     ['','','','','',''],
     ]
cyk(p, m, w='baaaab',fontsize='small', background={(2,0):'green!10', (0,3):'green!10'})
print(p)
\end{python}
we want $V$ such that $V \rightarrow AA$. There's only one case: $V = C$.
Altogether we have
\begin{python}
from latextool_basic import *
p = Plot()
m = [['$\{B_b,C_b\}$','$\{A_a\}$','$\{A_a\}$','$\{A_a\}$','$\{A_a\}$','$\{B_b,C_b\}$'],
     ['$\emptyset$','$\{C_{(1,2),(1,3)}\}$','$\{C_{(1,3),(1,4)}\}$','$\{C_{(1,4),(1,5)}\}$','$\{S_{(1,5),(1,6)}\}$',''],
     ['$\{A_{(1,1),(2,2)}\}$',r'$\{S_{(1,2),(2,3)}\}$','$\{S_{(1,3),(2,4)}\}$','$\{B_{(2,4),(1,6)}\}$','',''],
     ['$\{C_{(3,1),(1,4)}\}$','','','','',''],
     ['','','','','',''],
     ['','','','','',''],
     ]
cyk(p, m, w='baaaab',fontsize='small')
print(p)
\end{python}


For entry (4,2) you should get:
\begin{python}
from latextool_basic import *
p = Plot()
m = [['$\{B_b,C_b\}$','$\{A_a\}$','$\{A_a\}$','$\{A_a\}$','$\{A_a\}$','$\{B_b,C_b\}$'],
     ['$\emptyset$','$\{C_{(1,2),(1,3)}\}$','$\{C_{(1,3),(1,4)}\}$','$\{C_{(1,4),(1,5)}\}$','$\{S_{(1,5),(1,6)}\}$',''],
     ['$\{A_{(1,1),(2,2)}\}$',r'$\{S_{(1,2),(2,3)}\}$','$\{S_{(1,3),(2,4)}\}$','$\{B_{(2,4),(1,6)}\}$','',''],
     ['$\{C_{(3,1),(1,4)}\}$','$\emptyset$','','','',''],
     ['','','','','',''],
     ['','','','','',''],
     ]
cyk(p, m, w='baaaab',fontsize='small',
    background={(0,1):'blue!10',  (2,2):'blue!10',
                (1,1):'red!10',   (1,3):'red!10',
                (2,1):'green!10', (0,4):'green!10',
    })
print(p)
\end{python}


For entry (4,3) you should get:
\begin{python}
from latextool_basic import *
p = Plot()
m = [['$\{B_b,C_b\}$','$\{A_a\}$','$\{A_a\}$','$\{A_a\}$','$\{A_a\}$','$\{B_b,C_b\}$'],
     ['$\emptyset$','$\{C_{(1,2),(1,3)}\}$','$\{C_{(1,3),(1,4)}\}$','$\{C_{(1,4),(1,5)}\}$','$\{S_{(1,5),(1,6)}\}$',''],
     ['$\{A_{(1,1),(2,2)}\}$',r'$\{S_{(1,2),(2,3)}\}$','$\{S_{(1,3),(2,4)}\}$','$\{B_{(2,4),(1,6)}\}$','',''],
     ['$\{C_{(3,1),(1,4)}\}$','$\emptyset$','$\{S_{(1,3),(3,4)}\}$','','',''],
     ['','','','','',''],
     ['','','','','',''],
     ]
cyk(p, m, w='baaaab',fontsize='small',
    background={(0,2):'blue!10',  (2,3):'blue!10',
                (1,2):'red!10',   (1,4):'red!10',
                (2,2):'green!10', (0,5):'green!10',
    })
print(p)
\end{python}


\underline{Fifth Row}:
\begin{python}
from latextool_basic import *
p = Plot()
m = [['$\{B_b,C_b\}$','$\{A_a\}$','$\{A_a\}$','$\{A_a\}$','$\{A_a\}$','$\{B_b,C_b\}$'],
     ['$\emptyset$','$\{C_{(1,2),(1,3)}\}$','$\{C_{(1,3),(1,4)}\}$','$\{C_{(1,4),(1,5)}\}$','$\{S_{(1,5),(1,6)}\}$',''],
     ['$\{A_{(1,1),(2,2)}\}$',r'$\{S_{(1,2),(2,3)}\}$','$\{S_{(1,3),(2,4)}\}$','$\{B_{(2,4),(1,6)}\}$','',''],
     ['$\{C_{(3,1),(1,4)}\}$','$\emptyset$','$\{S_{(1,3),(3,4)}\}$','','',''],
     ['$\{S_{(3,1),(2,4)}\}$','$\{B_{(2,2),(3,4)}\}$','','','',''],
     ['','','','','',''],
     ]
cyk(p, m, w='baaaab',fontsize='small',
    background={(0,2):'blue!10',  (2,3):'blue!10',
                (1,2):'red!10',   (1,4):'red!10',
                (2,2):'green!10', (0,5):'green!10',
    })
print(p)
\end{python}
At $(5, 1)$ we use $S \rightarrow AC$ and at $(5,2)$ we use $B \rightarrow CB$.


\underline{Sixth Row}.
Using the following traversal shown, you should get the following entry for (6,1):
... and we're done!!! Here's the final table:
\begin{python}
from latextool_basic import *
p = Plot()
m = [['$\{B_b,C_b\}$','$\{A_a\}$','$\{A_a\}$','$\{A_a\}$','$\{A_a\}$','$\{B_b,C_b\}$'],
     ['$\emptyset$','$\{C_{(1,2),(1,3)}\}$','$\{C_{(1,3),(1,4)}\}$','$\{C_{(1,4),(1,5)}\}$','$\{S_{(1,5),(1,6)}\}$',''],
     ['$\{A_{(1,1),(2,2)}\}$',r'$\{S_{(1,2),(2,3)}\}$','$\{S_{(1,3),(2,4)}\}$','$\{B_{(2,4),(1,6)}\}$','',''],
     ['$\{C_{(3,1),(1,4)}\}$','$\emptyset$','$\{S_{(1,3),(3,4)}\}$','','',''],
     ['$\{S_{(3,1),(2,4)}\}$','$\{B_{(2,2),(3,4)}\}$','','','',''],
     ['$\{B_{(1,1),(2,5)}$ $S_{(3,1),(3,4)}\}$','','','','',''],
     ]
def getrect():
    i = {0:0}
    width = 2.3
    height = 0.7
    size = len(m)
    def rect(x):
        row, col = i[0] / size, i[0] % size
        if i[0] >= 30:
            i[0] += 1
            return Rect(x0=0, y0=0, x1=width, y1=2 * height,
                            innersep=0.2,
                            s='%s' % x, align='t')
        else:
            i[0] += 1
            return Rect(x0=0, y0=0, x1=width, y1=height,
                            innersep=0.2,
                            s='%s' % x, align='t')
    return rect
    
cyk(p, m, w='baaaab',fontsize='small', rect=getrect())
print(p)
\end{python}


So what have we achieved? First of all $S$ is in the entry at $(6,1)$.
What does this mean? 
Recall again that this means $S$ can derive $w_{1,6}$, which is the
string we started with, i.e. $baaaab$.
Secondly, you can write down the productions that derive $baaaab$ from $S$.
Can you write them down using the table? [SPOILERS AHEAD ...]

Of course CYK gives you even more.
For instance we also know that $B$ is also in the entry at (6,1).
This means that $B$ can also derive $baaab$.
If you look at the entry at (2,5), you conclude that $B$ can derive $aaab$.
In other words besides addressing the question of whether $baaaab$ can be
generated by the grammar, we have also answered the question for all
substrings of $baaaab$.
Make sure you see this.

OK ... so you know that $S$ can derive $baaaab$ (i.e.
$baaaab \in L(G)$ for the above CFG $G$).
But what is the derivation?
If you've understood the mathematical reasoning so far,
you should have no problems writing down the derivation.
\newcommand\ddd{&\implies}
\begin{align*}
  S_{(3,1)(3,4)}
  &\implies A_{(1,1)(2,2)} \cdot B_{(2,4)(1,6)} \\
  &\implies B_b \cdot C_{(1,2)(1,3)} \cdot B_{(2,4)(1,6)} & & \text{(or $C_b$ instead of $B_b$)} \\
  &\implies b \cdot C_{(1,2)(1,3)} \cdot B_{(2,4)(1,6)} \\
  &\implies b \cdot A_a \cdot A_a \cdot B_{(2,4)(1,6)} \\
  &\implies b \cdot a \cdot A_a \cdot B_{(2,4)(1,6)} \\
  &\implies b \cdot a \cdot a \cdot B_{(2,4)(1,6)} \\
  &\implies b \cdot a \cdot a \cdot C_{(1,4)(1,5)} \cdot B_b & &  \text{(or $C_b$ instead of $B_b$)} \\
  &\implies b \cdot a \cdot a \cdot A_a \cdot A_a \cdot B_b \\
  &\implies b \cdot a \cdot a \cdot a \cdot A_a \cdot B_b \\
  &\implies b \cdot a \cdot a \cdot a \cdot a \cdot B_b \\
  &\implies b \cdot a \cdot a \cdot a \cdot a \cdot b
\end{align*}

Note that if you are only interested in whether $baaaab$ is generated by the grammar
(and not the actual derivation), then you only need to fill the CYK table in
the following way:
\begin{python}
s = r'''
from latextool_basic import *
p = Plot()
m = [['$\{B,C\}$','$\{A\}$','$\{A\}$','$\{A\}$','$\{A\}$','$\{B,C\}$'],
     ['$\emptyset$','$\{C\}$','$\{C\}$','$\{C\}$','$\{S\}$',''],
     ['$\{A\}$','$\{S\}$','$\{S\}$','$\{B\}$','',''],
     ['$\{C\}$','$\emptyset$','$\{S\}$','','',''],
     ['$\{S\}$','$\{B\}$','','','',''],
     ['$\{B,S\}$','','','','',''],
     ]
cyk(p, m, w='baaaab',fontsize='small')
print(p)
'''
from latextool_basic import *
execute(s, print_source=False)
\end{python}


The CYK algorithm requires us to fill data into half of an $n$--by--$n$
array -- there are altogether $(1/2)n^2$ cells to fill.
Each cell $X$ of the array requires us to go through at most $n$ pairs
of cells which are already fill and see if we can find a production rule
that can help in filling cell $X$.
(We can speed up the search for a suitable production rule
if we find a suitable organization for the collection
of production rules. Which one?)
Therefore the total runtime is $O(n^3)$.
The space complexity is of course $\Theta(n^2)$.
(If you want to take into account the grammar, then the runtime is
$O(n^3 |G|)$ where $|G|$ is the size of the grammar
which is the total count of all symbols that appear
in all the production rules.)

\newpage
\begin{ex}
The above table also tells you that $B$ can also derive $baaaab$.
Write down a dervation of $baaaab$ using variable $B$.
\end{ex}

\newpage
\begin{ex}
  Using the grammar $G$ above, check if the following words are generated
  by $G$ using the CKY algorithm. Provide the completed CYK tables.
  For words generated by $G$, write down all their
  possible derivations. 
  \begin{tightlist}
  \item $aaaa$
  \item $ababba$
  \item $bbbaaa$
  \item $abbaab$
  \end{tightlist}
\end{ex}

\newpage
\begin{ex}
  Consider the grammar $G$ given by
  \begin{align*}
    S &\rightarrow AB \mid BA \\
    A &\rightarrow AC \mid a \\
    B &\rightarrow SA \mid SB \mid b \\
    C &\rightarrow SS \mid BA \mid a \\
  \end{align*}
  Using the grammar $G$ above, check if the following words are generated
  by $G$ using the CKY algorithm. Provide the completed CYK tables.
  For words generated by $G$, write down all their
  possible derivations. 
  \begin{tightlist}
  \item $aa$
  \item $bab$
  \item $abab$
  \item $aabaab$
  \end{tightlist}
\end{ex}


\newpage
\begin{ex}
  Consider the grammar:
  \begin{align*}
    S &\rightarrow AB \mid BA \\
    A &\rightarrow AC \mid AD \mid a \\
    B &\rightarrow SA \mid SD \mid a \mid b \\
    C &\rightarrow DS \mid BA \mid b \\
    D &\rightarrow AS \mid BB \mid a \\
  \end{align*}
  Using the grammar $G$ above, check if the following words are generated
  by $G$ using the CKY algorithm. Provide the completed CYK tables.
  For words generated by $G$, write down all their
  possible derivations. 
  \begin{tightlist}
  \item $aa$
  \item $bab$
  \item $abab$
  \item $aabaab$
  \end{tightlist}
\end{ex}

1. 
\begin{python}
s = r'''
from latextool_basic import *
p = Plot()
m = [['$\{A_a,B_a,D_a\}$','$\{A_a,B_a,D_a\}$'],
     ['$\{S_{(1,1)(1,2)},$ $A_{(1,1)(1,2)}$, $C_{(1,1)(1,2)}$, $S_{(1,1)(1,2)}$, $D_{(1,1)(1,2)} \}$',''],
     ]
cyk(p, m, w='aa',width=6, height=1.4, fontsize='footnotesize')
print(p)
'''
from latextool_basic import *
execute(s, print_source=False)
\end{python}
The derivations are
\begin{align*}
  S \implies AA \implies aA \implies aa \\
  S \implies AB \implies aB \implies aa \\
  S \implies BA \implies aA \implies aa \\
\end{align*}

\newpage
\begin{ex}
Of course with the above information, you can now
write a simple program to check if a string is generated by a CFG (you
would need the grammar to be in Chomsky Normal Form).
By the way, you can also store the production rule in each cell if you
want your program to list the production rule for each derivation
step. Another thing is that you need to search the list of production 
rules to fill up the table.
You want to think about how you want to represent the list of
production rules and how they should be
organized in a container to allow fast searches relevant to
completing the CYK table.
Also, note that in our algorithm above, when we compute
the variables $V$ in a cell, we include the information
\[
  V_{(i,j)(k,l), ...}
\]
to indicate that $V$ can derive $XY$ where $X$ is in the
set of variables in cell $(i,j)$ and $Y$ is in the set of variables
in cell $(k,l)$.
You might want to include more information about $X$ and $Y$.
For instance you might want to
do
\[
  V_{(i,j,m)(k,l,n), ...}
\]
to include the fact that the $X$ is the $m$--th variable at cell $(i,j)$
(if you're using an array for each cell)
and $Y$ is the $n$--th variable at cell $(k,l)$.
\end{ex}

\newpage
\begin{ex}
What if you have a string that's not generated by the grammar?
What do you get when you perform the CYK algorithm on such a string?
\end{ex}

\newpage
\begin{ex}
Using the above table, are there strings of length 2 in $L(G)$?
What about 3? Or 4? Or 5?
Verify by writing down their derivations (if possible).
\end{ex}


