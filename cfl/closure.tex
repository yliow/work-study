\sectionthree{Closure Rules}
\begin{python0}
from solutions import *; clear()
\end{python0}

\begin{ex}
Suppose $G'$ and $G''$ are two CFGs.
How would you define a grammar $G$ such that $L(G) = L(G') \cup L(G'')$?
This is your closure law for CFL.
\end{ex}

\newpage
\begin{ex}
Suppose $G'$ and $G''$ are two CFGs.
How would you define a grammar $G$ such that $L(G) = L(G') L(G'')$?
\end{ex}

\newpage
\begin{ex}
Suppose $G$ is a CFG.
Construct a grammar $G'$ such that $L(G') = L(G)^*$. 
\end{ex}

\newpage
So now you know that union, contenation, and the Kleene star are closed 
operators of CFLs.

\newpage
\begin{ex}
Let $L$ be a CFL and $L'$ be regular, prove that $L \cap L'$ is a CFL.
\end{ex}

[HINT: We can't prove that a language is not context free yet.
However, I will tell you that 
\[
\{a^n b^n c^n \mid n \geq 0\}
\]
is not context free.]

\newpage
\begin{ex}
Prove that the statement \lq\lq $L$ is a CFL, $L'$ is a  CFL implies
$L \cap L'$ is a CFL'' is not true.
\qed
\end{ex}

\newpage
\begin{ex}
Prove that the statement \lq\lq $L$ is a CFL implies
$\overline{L}$ is a CFL'' is not true.
\qed
\end{ex}

\newpage
\begin{ex}
We have seen some examples where non-regular languages are actually accepted
by CFGs.
We have also seen that some regular languages are accepted by CFGs.
Let's try one regular languages accepted by a regex:
Find a CFG that generates the same language as the language accepted by 
the regular expression $101(01^* \cup 10^*)^+(\ep \cup 111)$.
\end{ex}

[ANOTHER VERSION]



\begin{prop} Here is a summary of the closure properties of CFLs.
 \begin{tightlist}
  \item[(a)] CFLs are closed under union, concatenation, Kleene star
  ${}^*$, intersection with regular languages
  \item[(b)] CFLs are \textit{ not} closed under intersection
  complement.
  However, CFLs are closed under \lq\lq intersection with
  regular languages.''
  In other words of $L$ is a CFL and $L'$ is regular,
  then $L \cap L'$ is also a CFL.
 \end{tightlist}
\end{prop}

\begin{proof}

(a): Let $L_i$ ($i=1,2$) be CFL generated by grammar with starting
symbol $S_i$ ($i=1,2$). Create a new grammar by taking the obvious
unions (for instance the variable set is the union of the variable
set of $G_1, G_2$, etc.)
\begin{mylist}
 \item What do you get when you add the production $S \rightarrow
 S_1 \,|\, S_2$?
 \item What do you get when you add the production $S \rightarrow
 S_1 S_2$?
 \item What do you get when you add the production $S \rightarrow
 S_1S \,|\, \ep$?
\end{mylist}
Of course we are assuming that the variable sets are disjoint and
the $S$ is an unused symbol.

The proof for intersection with regular languages is omitted.

(b): Here's why CFLs are not closed under intersection. Let $L_1 =
\{a^ib^ic^j \,|\, i,j \geq 0\}$ and $L_2 = \{a^ib^jc^j \,|\, i,j
\geq 0\}$. Then $L_1, L_2$ are CFLs (prove it!). The intersection
is $\{a^i b^i c^i \,|\, i \geq 0\}$ which is not a CFL.

The reason why CFLs are not closed under complement is easy.
Recall the fact we used in the section on closure properties for
regular languages: If $L_1, L_2$ are sets, then
\[
 \overline{ \overline{L_1} \cup \overline{L_2}} = L_1 \cap L_2
\]
Why should this help?
\end{proof}



\newpage
\begin{eg}
  Let $\Sigma = \{a,b\}$.
  Prove that the language $L = \{ww \,|\, w \in \Sigma \}$ is not a CFL.
\end{eg}

\textit{Proof}.
Suppose $L$ is a CFL. Since intersection with regular languages is a closed
operator for CFLs, when we intersect this with the
regular language $L(a^*b^*a^*b^*)$
we get the
language
\[
\{a^i b^j a^ib^j \,|\, i,j \geq 0 \}
\]
However this language is not context-free.
Hence $L$ is not a CFL.
\qed

\newpage
\begin{eg}
  Let $\Sigma = \{ a, b \}$.
  Prove that the language
  \[
  L = \{w \,|\, w \text{ has the same number of $a$'s and $b$'s} \}
  \]
  is not context-free.
\end{eg}
