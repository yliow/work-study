\sectionthree{Set cardinality, countability, and ordinals} \label{S:countability}
\begin{python0}
from solutions import *; clear()
\end{python0}

Throughout this section, $X$ is a set. Recall that I defined
$|X|$ for finite sets, i.e., $|X|$ is the number of elements in $X$.
$|X|$ is usually called the
\defone{size}
or the
\sidebarskip{12pt}\defone{cardinality}\sidebarskip{0pt}
of the set.
Now I want to talk about infinite sets.

Georg Cantor is usually considered the founder of modern
set theory.
He was the first to realized that although
the idea of a set is simple, in fact other than finite sets,
sets in general are very complex.
Many of the concepts in this chapter was first defined
by
Richard Dedekind.
But it was Cantor who realized their importance.
Cantor and Dedekind were contemporary and were close friends
and met frequently for discussion.
But Dedekind acted more like a sounding board to Cantor's ideas.
Cantor thoughts on set theory made him realized that
when looking at infinite sets, it is
still possible to differentiate between them.
In other words, there are many different types of infinities.
In a sense, this is somewhat similar to the study of functions:
$f(x) = x$ and $g(x) = \ln x$ both go to infinity as $x$ goes to
infinity. But their order of growth are different.
Cantor proved that there are not just different infinities,
but there are \textit{infinitely} many infinities!
Cantor's contribution to set theory has important
consequences to analysis, topology, logic, and
Turing machines in the theory of automata and
theoretical computational complexity.
As a consequence of Cantor's work, you can for instance show that
the set of C++ programs is countably infinite (I'll
define this precisely later), but the set of
boolean functions $\N \rightarrow \{0, 1\}$ is uncountably
infinite. Note that both set are infinite but in different ways:
there are more boolean functions $\N \rightarrow \{0, 1\}$
than there are C++ programs.
Therefore Cantor's set theory tells you right away that
there is a boolean function that cannot be computed by a C++
program.

Now ... on to infinite sets ...

Recall that a function $f: X \rightarrow Y$
is
\defone{one-to-one}
(I'll write 1--1) or
\sidebarskip{0pt}\defone{injective}\sidebarskip{0pt}
if
$f(x) = f(x')$ implies $x = x'$.
$f$ is
\sidebarskip{12pt}\defone{onto}\sidebarskip{0pt} or
\sidebarskip{24pt}\defone{surjective}\sidebarskip{0pt}
if for every $y \in Y$, there is some $x \in X$ such that
$f(x) = y$.
A
\sidebarskip{12pt}\defone{bijection} %\sidebarskip{0pt} 
is a function that is both 1--1 and onto.
If a function is a bijection I will say it is \lq\lq 1--1 and onto".
A bijection is also called a
\sidebarskip{1pt}\defone{one-to-one correspondence}.

If we are going to only talk about finite sets
then $|X| = |Y|$ means $X$ and $Y$ have the same number of
distinct elements,
$|X| < |Y|$ means that $Y$ has strictly more elements than
$X$.
Etc.
I'm going to generalize the above notions of
$|X| = |Y|$, $|X| < |Y|$, $|X| \leq |Y|$ to include cases when
$X$ and $Y$ are infinite.

\begin{defn}
  Let $X$ and $Y$ be sets.
  \begin{itemize}
  \item
    We will write $|X| \leq |Y|$ if there is a
    1--1 function $X \rightarrow Y$.
  \item
    If there is a bijection between
    $X$ and $Y$, we write $|X| = |Y|$.
    If $|X| = |Y|$, I will also say that $X$ and $Y$
    have the same
    \defone{cardinality}
    or that they are
    \sidebarskip{12pt}\defone{equinumerous}.
    I will write $|X| \neq |Y|$
    if $|X| = |Y|$ is not true, i.e., if there is no 1--1 onto function
    from $X$ to $Y$.
  \item
    I will write $|X| < |Y|$ if
    $|X| \leq |Y|$ but $|X| \neq |Y|$.
    In other words $|X| < |Y|$ means there is a 1--1 function
    from $X$ to $Y$ but there is no 1--1 and onto function
    from $X$ to $Y$.
  \end{itemize}
  Of course we define $|X| > |Y|$ if $|Y| < |X|$ and
  $|X| \geq |Y|$ if $|Y| \leq |X|$.
\end{defn}

Note that the above definition generalizes
the other simpler definition of $|X| \leq |Y|$ for the case
when $X$ and $Y$ are finite.
(Right?)

\begin{defn}
  $X$ is \defone{infinite} iff $|X| = |X - \{x\}|$ for any $x \in X$.
\end{defn}

%-*-latex-*-

\begin{ex} 
  \label{ex:prob-00}
  \tinysidebar{\debug{exercises/{disc-prob-28/question.tex}}}

  \solutionlink{sol:prob-00}
  \qed
\end{ex} 
\begin{python0}
from solutions import *
add(label="ex:prob-00",
    srcfilename='exercises/discrete-probability/prob-00/answer.tex') 
\end{python0}


\begin{defn}
  Let $X$ be a set.
  \begin{itemize}
  \item
    We will say that $X$ is \defone{countable} if
    either $X$ is finite or $|X| = |\N|$, i.e.,
    there is a bijection between $X$ and
    $\N$.
  \item
    $X$ is \defone{uncountable} if it is not countable.
  \end{itemize}
\end{defn}

%-*-latex-*-

\begin{ex} 
  \label{ex:prob-00}
  \tinysidebar{\debug{exercises/{disc-prob-28/question.tex}}}

  \solutionlink{sol:prob-00}
  \qed
\end{ex} 
\begin{python0}
from solutions import *
add(label="ex:prob-00",
    srcfilename='exercises/discrete-probability/prob-00/answer.tex') 
\end{python0}


Don't be fooled: The statement $|X| \leq |Y|, |Y| \leq |X| \implies
|X| = |Y|$ is true, but the proof is not immediate.
See next section on the Berstein--Schr\"oder theorm.

%-*-latex-*-

\begin{ex} 
  \label{ex:prob-00}
  \tinysidebar{\debug{exercises/{disc-prob-28/question.tex}}}

  \solutionlink{sol:prob-00}
  \qed
\end{ex} 
\begin{python0}
from solutions import *
add(label="ex:prob-00",
    srcfilename='exercises/discrete-probability/prob-00/answer.tex') 
\end{python0}



%-*-latex-*-

\begin{ex} 
  \label{ex:prob-00}
  \tinysidebar{\debug{exercises/{disc-prob-28/question.tex}}}

  \solutionlink{sol:prob-00}
  \qed
\end{ex} 
\begin{python0}
from solutions import *
add(label="ex:prob-00",
    srcfilename='exercises/discrete-probability/prob-00/answer.tex') 
\end{python0}


%-*-latex-*-

\begin{ex} 
  \label{ex:prob-00}
  \tinysidebar{\debug{exercises/{disc-prob-28/question.tex}}}

  \solutionlink{sol:prob-00}
  \qed
\end{ex} 
\begin{python0}
from solutions import *
add(label="ex:prob-00",
    srcfilename='exercises/discrete-probability/prob-00/answer.tex') 
\end{python0}


%-*-latex-*-

\begin{ex} 
  \label{ex:prob-00}
  \tinysidebar{\debug{exercises/{disc-prob-28/question.tex}}}

  \solutionlink{sol:prob-00}
  \qed
\end{ex} 
\begin{python0}
from solutions import *
add(label="ex:prob-00",
    srcfilename='exercises/discrete-probability/prob-00/answer.tex') 
\end{python0}


%-*-latex-*-

\begin{ex} 
  \label{ex:prob-00}
  \tinysidebar{\debug{exercises/{disc-prob-28/question.tex}}}

  \solutionlink{sol:prob-00}
  \qed
\end{ex} 
\begin{python0}
from solutions import *
add(label="ex:prob-00",
    srcfilename='exercises/discrete-probability/prob-00/answer.tex') 
\end{python0}


%-*-latex-*-

\begin{ex} 
  \label{ex:prob-00}
  \tinysidebar{\debug{exercises/{disc-prob-28/question.tex}}}

  \solutionlink{sol:prob-00}
  \qed
\end{ex} 
\begin{python0}
from solutions import *
add(label="ex:prob-00",
    srcfilename='exercises/discrete-probability/prob-00/answer.tex') 
\end{python0}


In the above \lq\lq countable union" means the union of
countably many sets.
So this means that the sets of the union is finite, for instance
\[
\bigcup_{i=1}^5 X_i
\]
where $X_i = [i, i + 1)$ (intervals of $\R$), or
\[
\bigcup_{i=1}^\infty X_i
\]
In the second union, there are countable many $X_i$'s.

An example of an uncountable union would be
\[
\bigcup_{r \in [0, 100)} X_r
\]
where $X_r = [r, r + 1)$.
  In this case there are uncountably many $X_r$'s since the
  $r$ runs through $[0, 100)$ which is not countable.
    For instance this union involves $[1, 2)$,
      $[3.5, 4.5)$,
      $[\sqrt{2}, \sqrt{2} + 1)$,
      $[\pi, \pi + 1)$, etc.
Clearly there are uncountably many such $X_r$.
            
So what about $\R$? $\R$ is in fact uncountable. As a matter of
fact, the interval $[0,1)$ is uncountable.
This was first proved by Cantor.
The method of proof for
this theorem discovered by Cantor
is very important. Some people call this method the
\defone{diagonalization argument}.

\begin{thm} \textnormal{(Cantor)}
$[0,1)$ is uncountable.
\end{thm}


If you have taken discrete math, you should be able to give a proof
(or a semi-proof close to the correct one.)
You should attmept it.

SPOILER ALERT ... turn the page for the proof.


\textit{Proof.} Every real number in $[0,1)$ looks like a decimal with no
integer part. For instance $0.123 \in [0,1)$.

We prove by contradiction. Suppose $[0,1)$ is countable. So let's say
the complete list of real numbers in $[0,1)$ is given by the list
$x_1, x_2, x_3, \ldots$. Let's construct a real number $x \in
[0,1)$ which is \textbf{not} in the above list. This will give us a
contradiction. Right?

OK. So let's begin. I will construct $x$ by giving you the decimal
expansion. I will also\lq\lq avoid" the list $x_1, x_2, x_3\ldots$
so that $x$ is not any of them.

Let's begin by \lq\lq avoiding" $x_1$. Now $x_1$ is of the form
$0.a...$ where $a$ is a digit from 0 to 9. If $a=0$, I will say
that $x=0.1\ldots$; otherwise, I will say that $x=0.0\ldots$. So
obviously $x \neq x_1$.

I'll repeat this for $x_2$. Say $x_2 =
0.bc\ldots$, then the second decimal place of $x$ is $1$ if $c=0$;
otherwise the second decimal place of $x$ is $0$.
I have $x \neq x_2$.

I'll then make the 3rd decimal place of $x$ different from the
third decimal place of $x_3$.

Etc. Get it?

For instance suppose my list of $x_1, x_2, x_3, ...$ looks like this:
%\begin{figure}
\begin{longtable}{cccccccccc}
  $x_1$ & = & 0 & . & \textbox{1} & 3 & 0 & 2 & 9 & 0  \\
  $x_2$ & = & 0 & . & 0 & \textbox{0} & 8 & 3 & 0 & 7  \\
  $x_3$ & = & 0 & . & 3 & 1 & \textbox{3} & 1 & 3 & 1  \\
  $x_4$ & = & 0 & . & 0 & 0 & 8 & \textbox{7} & 2 & 7  \\
  $x_5$ & = & 0 & . & 1 & 2 & 8 & 0 & \textbox{2} & 7  \\
  $x_6$ & = & 0 & . & 0 & 6 & 1 & 9 & 2 & \textbox{0}  \\
\end{longtable}
%\end{figure}

So for the decimal places of $x$, I choose not 1, not 0, not 3, not 7,
not 2, not 0, etc.
For instance I can choose 0, 1, 0, 0, 0, 1, ...
Hence for this case my $x$ looks like
\[
x = 0.010001...
\]

So let's have a formal proof. Suppose $x_i$ is
\[ x_i = 0.x_{i,1}x_{i,2}x_{i,3} \ldots x_{i,i} \ldots \]
for $i=1,2,3,\ldots$. Then we let $x$ be the number
\[ x = 0.y_1 y_2 y_3 \ldots y_i \ldots \]
where
\[ y_i =
\begin{cases}
 0 \text{ if } x_{i,i} \neq 0 \cr
 1 \text{ if } x_{i,i} = 0 \cr
\end{cases}
\]
Note in particular that $y_i \neq x_{i,i}$ for all $i>0$. Then $x
\neq x_i$ for all $i$. Why? Otherwise say $x = x_i$ for some $i$.
But that implies that they have the same decimal expansion. In
other words
\[ y_j = x_{i,j} \]
for all $j$. But by construction,
\[ y_i \neq x_{i,i} \]
Contradiction!!!

But wait, there's a ... \textbf{hole} in the proof. We assumed that if
two real numbers are the same, they must then have the same decimal
expansion. Is that true?



%-*-latex-*-

\begin{ex} 
  \label{ex:prob-00}
  \tinysidebar{\debug{exercises/{disc-prob-28/question.tex}}}

  \solutionlink{sol:prob-00}
  \qed
\end{ex} 
\begin{python0}
from solutions import *
add(label="ex:prob-00",
    srcfilename='exercises/discrete-probability/prob-00/answer.tex') 
\end{python0}


%-*-latex-*-

\begin{ex} 
  \label{ex:prob-00}
  \tinysidebar{\debug{exercises/{disc-prob-28/question.tex}}}

  \solutionlink{sol:prob-00}
  \qed
\end{ex} 
\begin{python0}
from solutions import *
add(label="ex:prob-00",
    srcfilename='exercises/discrete-probability/prob-00/answer.tex') 
\end{python0}


%-*-latex-*-

\begin{ex} 
  \label{ex:prob-00}
  \tinysidebar{\debug{exercises/{disc-prob-28/question.tex}}}

  \solutionlink{sol:prob-00}
  \qed
\end{ex} 
\begin{python0}
from solutions import *
add(label="ex:prob-00",
    srcfilename='exercises/discrete-probability/prob-00/answer.tex') 
\end{python0}

  This is something that is hard to believe:
  Prove that
  \[
  \text{$[0,1]$ and $[0,1] \times [0,1]$ are equinumerous!!!}
  \]
  The interval $[0,1]$ is called the \defone{unit interval}
  and the product $[0,1] \times [0,1]$ is called the \defone{unit square}.
  So basically there are as many points on the unit interval
  as there are on the unit square.
  (Cantor couldn't believe it when he proved the above fact.
  In a letter to Dedekind, he wrote,
  \lq\lq I see it, but I don’t believe it!")

%-*-latex-*-

\begin{ex} 
  \label{ex:prob-00}
  \tinysidebar{\debug{exercises/{disc-prob-28/question.tex}}}

  \solutionlink{sol:prob-00}
  \qed
\end{ex} 
\begin{python0}
from solutions import *
add(label="ex:prob-00",
    srcfilename='exercises/discrete-probability/prob-00/answer.tex') 
\end{python0}



%-*-latex-*-

\begin{ex} 
  \label{ex:prob-00}
  \tinysidebar{\debug{exercises/{disc-prob-28/question.tex}}}

  \solutionlink{sol:prob-00}
  \qed
\end{ex} 
\begin{python0}
from solutions import *
add(label="ex:prob-00",
    srcfilename='exercises/discrete-probability/prob-00/answer.tex') 
\end{python0}



%-*-latex-*-

\begin{ex} 
  \label{ex:prob-00}
  \tinysidebar{\debug{exercises/{disc-prob-28/question.tex}}}

  \solutionlink{sol:prob-00}
  \qed
\end{ex} 
\begin{python0}
from solutions import *
add(label="ex:prob-00",
    srcfilename='exercises/discrete-probability/prob-00/answer.tex') 
\end{python0}


You now know that $\R$ is not countable and $\Q$ is countable.
So $\R$ is sort of \lq\lq huge".
In particular the set of $\R - \Q$ (irrational numbers) is \lq\lq huge".
You can actually
subdivide $\R$ further.
There is an important set of numbers, call the
set of algebraic numbers in $\R$.
I'll write $A$ for the set of algebraic numbers.
Some algebraic numbers are complex while some are real.
Here's the definition of an algebraic number.

A number $\alpha \in \C$ is an \defone{algebraic number}
if $\alpha$ is the root of a polynomial with coefficients in $\Q$.
Just for practice, show that $\Q \subset A$.
Note that $\sqrt{2}$ is not rational.
However $\sqrt{2}$ is algebraic.
In fact show that if $x > 0$ and $y > 0$ are rational, then
$x^y$ is algebraic.
Now show that $i$ is algebraic where $i = \sqrt{-1}$.

The relationship between $\Z, \Q, A, \R, \C$ is:
\[
\Z \subset \Q \subset A \cap \R \subset \R \subset \C
\]
and
\[
\Z \subset \Q \subset A \subset \C
\]
In other words the set of algebraic numbers $A$ is a subset of $\C$
and some algebraic numbers are real, but not all.
Enough practice ... now for the real exercise:

%-*-latex-*-

\begin{ex} 
  \label{ex:prob-00}
  \tinysidebar{\debug{exercises/{disc-prob-28/question.tex}}}

  \solutionlink{sol:prob-00}
  \qed
\end{ex} 
\begin{python0}
from solutions import *
add(label="ex:prob-00",
    srcfilename='exercises/discrete-probability/prob-00/answer.tex') 
\end{python0}


Note that the set of algebraic numbers in $\R$ is countable:
$A \cap \R$ is countable.
But we know that $\R$ is uncountable.
This means that there are non-algebraic numbers in $\R$.
In fact there are uncountably many non-algebraic numbers in $\R$.
These are called transendental numbers.
In other words $\alpha \in \C$ is \defone{transcendental}
if $\alpha$ is not the root of a polynomial with rational coefficients.
Here are two examples: $\pi$ and $e$ are transcendental.

I'm sure you have heard in some previous classes that $\pi$ and $e$
are irrational.
But $\pi$ and $e$ are more than irrational.
They are transcendental.

Proving a number is transcendental is not easy!
For instance to prove $\pi$ is transcendental, by definition,
you have to prove $\pi$ is not algebraic.
That means $\pi$ is \textit{not} the solution of \textit{any} polynomial
with $\Q$ coefficients.

While the ancient Greek philosophers discovered the existence of
irrational numbers (example $\sqrt{2}$) around 300BC,
the first transcendental number was only discovered around 1850:
\[
\sum_{n = 1}^\infty 10^{-n!}
\]
Try to write down it's decimal representation up to say 50 decimal placed.
This number was artificially created by
\href{https://en.wikipedia.org/wiki/Joseph_Liouville}{Liouville}
to \lq\lq avoid"
polynomials with $\Q$ coefficients.

On the other hand $\pi$ was proven to be transcendental in
1882 by
\href{https://en.wikipedia.org/wiki/Ferdinand_von_Lindemann}{von Lindemann}
and $e$ was proven to be transcendental a couple of years early by
\href{https://en.wikipedia.org/wiki/Charles_Hermite}{Hermite} in 1873.
The fact that there are uncountably many transcendental numbers
was proven by Cantor in 1874, one year after Hermite's result.

We actually know very little about transcendental numbers.
For instance, we do not know if the following constant,
the Euler–Mascheroni constant, is
transcendental or not:
\[
\gamma = \lim_{n \rightarrow \infty}
\left(
\sum_{k = 1}^n \frac{1}{k}
-
\ln n
\
\right)
= 0.5772...
\]
($\ln = \log_e$).
In fact, scratch that, we don't even know if $\gamma$ is rational or
irrational!!!
This constant is ubiquitous and appears in CS, math, physics, etc.


%-*-latex-*-

\begin{ex} 
  \label{ex:prob-00}
  \tinysidebar{\debug{exercises/{disc-prob-28/question.tex}}}

  \solutionlink{sol:prob-00}
  \qed
\end{ex} 
\begin{python0}
from solutions import *
add(label="ex:prob-00",
    srcfilename='exercises/discrete-probability/prob-00/answer.tex') 
\end{python0}



%-*-latex-*-

\begin{ex} 
  \label{ex:prob-00}
  \tinysidebar{\debug{exercises/{disc-prob-28/question.tex}}}

  \solutionlink{sol:prob-00}
  \qed
\end{ex} 
\begin{python0}
from solutions import *
add(label="ex:prob-00",
    srcfilename='exercises/discrete-probability/prob-00/answer.tex') 
\end{python0}



\begin{thm} \textnormal{(Cantor)}
For any set $X$, $|X| < |P(X)|$.
\end{thm}

The fact $|X| \leq |P(X)|$ is easy.

Now I'm going to prove that you cannot find an onto function
from $X$ to $P(X)$.
Suppose on the contrary that 
$f : X \rightarrow P(X)$ is an onto function.
I need to arrive at a contradiction.
How?
Since $f$ is onto,
for any subset $Y$ of $X$, there is some $x \in X$ such that
$f(x) = Y$.
I need to construct some $Y$ that will cause some problem (contradiction).

Now $Y$ would look like this:
$Y = \{x \in X \mid P(x)\}$ where $P(x)$ is some condition.
The condition $P(x)$ depends on whatever I have now.
For instance $P(x)$ might depends on $X$, $P(X)$, $f$.
So I want
to say there is some $x' \in X$ such that 
\[
f(x') = \{x \in X \mid P(x) \}
\]
will lead to a contradiction.
The question is what should $P(x)$ be?
What about this $x'$?
How can it be used to craft a contradiction?
Well $x'$ is an element of $X$
and $f(x')$ is a subset of $X$.
The relationship between $x'$ and $f(x')$ is
either $x' \in f(x')$ or $x' \not\in f(x')$.
Suppose $x' \in f(x')$, i.e.,
\[
x' \in f(x') = \{x \in X \mid P(x) \}
\]
Since $x' \in \{x \in X \mid P(x) \}$, then of course $x'$
satisfy $P(x')$.
In other words
\[
x' \in f(x') \implies P(x')
\]
See a contradiction? Suppose $P(x)$ is \lq\lq$x \not\in f(x)$".
Then
\[
x' \in f(x') = \{x \in X \mid x \not\in f(x) \}
\]
i.e.,
\[
x' \in f(x') \implies x' \not\in f(x')
\]
which is clearly a contradiction.
However this is when I assume $x' \in f(x')$.
What if $x' \not\in f(x')$?
Would the set 
$\{x \in X \mid x \not\in f(x) \}$ still give me a contradiction?
Why yes!
Because if $x' \not\in f(x')$, then $x'$ does not
satisfy $P(x)$, i.e., it is not true that
$x' \not\in f(x')$.
So now we're ready to write the proof.

\proof
First we prove that $|X| \leq |P(X)$.
Define the function $f: X \rightarrow P(X)$ to be
$f(x) = \{x\}$.
This function is 1--1:
If $x,x' \in X$ and $f(x) = f(x')$, then $\{x\} = \{x'\}$
and hence $x = x'$.
Therefore $|X| \leq |P(X)|$.

Now we will prove that $|X| \neq |P(X)|$.
In other words, we will prove that there's no onto function
from $X$ to $P(X)$.
Assume on the contrary that $f: X \rightarrow P(X)$ is an onto function.
Let
\[
Y = \{x \in X \mid x \not\in f(x) \}
\]
$Y$ is a subset of $X$.
Since $f$ is onto, there is some $x' \in X$ such that
\[
f(x') = Y = \{x \in X \mid x \not\in f(x) \}
\]
We will consider two cases: $x' \in f(x')$ and $x' \not\in f(x')$
and show that in each cases, we will arrive at a contradiction.

If $x' \in f(x')$, then
\begin{align*}
  x' &\in f(x') = Y = \{x \in X \mid x \not\in f(x) \} \\
  \THEREFORE x' &\text{ satisfies the condition } x' \not\in f(x') \\
  \THEREFORE x' &\not\in f(x') 
\end{align*}
which is a contradiction.

If $x' \not\in f(x')$, then
\begin{align*}
x' &\not\in f(x') = Y = \{x \in X \mid x \not\in f(x) \} \\
\THEREFORE x' &\not\in \{x \in X \mid x \not\in f(x) \} \\
\THEREFORE x' &\text{ does not satisfy the condition } x' \not\in f(x') \\
\THEREFORE x' &\in f(x')
\end{align*}
which is a contradiction.

In both cases, I arrive at contradictions.
Hence our assumption on the existence of an onto function
from $X$ to $P(X)$ does not hold.
Hence there is no onto function from $X$ to $P(X)$.


% this following exercise was originally commented out when I came across it
% but i still made it a solution in case in the future it's needed

%%-*-latex-*-

\begin{ex} 
  \label{ex:prob-00}
  \tinysidebar{\debug{exercises/{disc-prob-28/question.tex}}}

  \solutionlink{sol:prob-00}
  \qed
\end{ex} 
\begin{python0}
from solutions import *
add(label="ex:prob-00",
    srcfilename='exercises/discrete-probability/prob-00/answer.tex') 
\end{python0}


\textsc{On non-uniqueness of decimal representations}.

Here's a real number
\[
1433.235246457234346
\]
Of course a decimal expanion need no terminate.
For instance $\pi$ does not terminate and furthermore the pattern
of the decimal expansion does not repeat.
There are decimal expansion that repeats.
For instance
\[
0.9999999999\ldots
\]
It can be shown (probably in precalc) that
\[
0.9999999999\ldots = 1
\]

The question is this:

%-*-latex-*-

\begin{ex} 
  \label{ex:prob-00}
  \tinysidebar{\debug{exercises/{disc-prob-28/question.tex}}}

  \solutionlink{sol:prob-00}
  \qed
\end{ex} 
\begin{python0}
from solutions import *
add(label="ex:prob-00",
    srcfilename='exercises/discrete-probability/prob-00/answer.tex') 
\end{python0}

