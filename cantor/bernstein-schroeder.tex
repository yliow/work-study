\sectionthree{Berstein-Schr\"oder Theorem}
\begin{python0}
from solutions import *; clear()
\end{python0}

Suppose you have two finite sets $X$ and $Y$.
If $X$ and $Y$ has the same number of elements,
then of course I can find a 1--1 and onto from $X$ and $Y$.

Now suppose I have two finite sets $X$ and $Y$
but I only tell you there is a 1--1 function from $X$ to $Y$.
What can you tell me about the sizes $|X$ and $|Y|$?
Then it must be true that
\[
|X| \leq |Y|
\]
What if I tell you there's also a 1--1 function from $Y$ to $X$,
Then of course
\[
|Y| \leq |X|
\]
Since $|X|$ and $|Y|$ are finite, I get
\[
|X| = |Y|
\]
right away.
That's because for integers $a$ and $b$,
\[
a \leq b
\text{ and }
b \leq a
\]
implies
\[
a = b
\]
However if $X$ and $Y$ are sets in general (i.e., not neceesarily finite)
and I know that
\[
|X| \leq |Y|
\text{ and }
|Y| \leq |X|
\]
it seems to be true that
\[
|X| = |Y|
\]
but hang on ...
you should not think of numbers here.
When it comes to infinities, you always have to be careful.
What I'm saying above is this:
If
\[
|X| \leq |Y|
\text{ and }
|Y| \leq |X|
\]
i.e. 
\[
\text{there is a 1--1 function $X \rightarrow Y$}
\]
and
\[
\text{there is a 1--1 function $Y \rightarrow X$}
\]
then
\[
|X| = |Y|
\]
i.e.,
\[
\text{there is a 1--1 onto function $X \rightarrow Y$}
\]
So I'm making this statement:
If there are 1--1 functions
\[
X \rightarrow Y, \,\,\, Y \rightarrow X
\]
then there is a 1--1 and onto function
\[
X \rightarrow Y
\]
If you think the proof easy, go ahead and try it.
This is not a trick question.
The above statement is actually true. 
I'll state it as a theorem for reference, but without proof.
The theorem is usually called the
Bernstein--Schr\"oder theorem.
The statement was first stated by Cantor in 1887.
Bernstein and Schr\"oder provided proofs 1897.
It was later found (1902) that Schr\"oder's proof is incorrect.
Unknown to everyone, Dedekind already
had a proof in 1887.

\begin{thm}
  \textnormal{\textbf{Bernstein--Schr\"oder}}
  \sidebar{\textnormal{Bernstein--Schr\"oder}}
  Let $X$ and $Y$ be sets such that
  \[
  |X| \leq |Y| \text{ and }
  |Y| \leq |X|
  \]
  Then
  \[
  |Y| = |X|
  \]
\end{thm}
  

What does the Berstein--Schr\"ofer theorem gives you?
Well if you want to show $|X| = |Y|$, you can try to find
a 1--1 and onto function from $X$ to $Y$.
Or, by Bernstein--Schroeder, you can find a 1--1 function
from $X$ to $Y$ and another 1--1 function from $Y$ to $X$.
In some cases, finding a 1--1 and onto function might be harder.
The reason is because you need to find one function that satisfies
\textit{two} conditions.
It's true that using Bernstein--Schroeder requires you
to find \textit{two} functions.
However in many cases finding
two functions each satisfying \textit{one} condition (i.e., 1--1)
is actually easier.

\newpage
\begin{ex}
  Note that Berstein-Schroeder says
  \[
  |X| \leq |Y|, \,\,\, |Y| \leq |X| \implies
  |X| = |Y|
  \]
  Show that the converse is true, i.e., show that
  \[
  |X| = |Y|
  \implies
  |X| \leq |Y|, \,\,\, |Y| \leq |X| 
  \]
  \qed
\end{ex}

\newpage
\begin{ex}
  Let $a < b$ and $c < d$ where $a,b,c,d$ are real
  numbers.
  Prove that $|(a, b)| = |(c, d)|$.
  Here $(a, b)$ is the open interval from $a$ to $b$.
  How about $(a, b)$ and $(c, d]$?
    Do they have the same cardinality?
    What about $(a, b)$ and $[c, d]$?
    \qed
\end{ex}
\begin{comment}
  |(0, 1)| = |(0, x)| where x > 0: easy.
  |(0, 1)| = |(a, a+1)|: easy.

  |(0,1)| = |(0, 1]|:
    \leq is easy by inclusion.
    \geq use map x -> x/2.
    Therefore by bernstein schroeder ... done.
\end{comment}   

\newpage
\begin{ex}
  Is it true that $|(0, 1)| = |\R|$?
  What about $|[0, 1)| = |\R|$ 
  \qed
\end{ex}

\begin{ex}
  Is it true that $|\R| = |\R^2|$?
  What about $|[0,1]| = |[0,1]^2|$?
  \qed
\end{ex}
\begin{comment}
  |[0,1]| = |[0,1]^2|:
  \leq is easy
  \geq: let (x, y) \in [0,1]^2. interleave digits of x,y to get z in [0,1].
\end{comment}


\newpage
\begin{ex}
  Prove that $|P(\N)| = |\R|$.
\end{ex}

SPOILER ALERT ... turn the page for the solution.

\newpage
\proof
I will show $|P(\N)| = |[0, 1)|$.
A real number $x$ in $[0, 1)$ can be writing in binary representation
  \[
  x = 0.x_1 x_2 x_3 \ldots
  \]
  I will assume that $x$ does not have a string of $1$s as the tail end
  of the binary sequence, i.e., there is no $N$ such that
  $x_i = 1$ for all $i > N$.
  Define
  a set
  \[
  f(x) \subseteq \N
  \]
  where
  \[
  n \in f(x) \iff x_n > 0
  \]
  For instance if $x = 0.001101$, then
  \[
  f(x) = \{3,4,6\}
  \]
  and if $x = 0.0101010101010\ldots$, then
  \[
  f(x) = \{2, 4, 6, 8,10, 12, ...\} 
  \]
  i.e., it's the set of positive even integer.
  Now I'll show that $f$ is 1--1.
  If $f(x) = f(x')$, then the
  $x_i = 1$ iff $x'_i = 1$.
  Hence $x$ and $x'$ has the same binary representation.
  Therefore $x = x'$.
  Hence $f$ is 1--1.
  To show $f$ is onto,
  if $X$ is a subset of $\N$, from $X$ I construct $x$ as
  $x = 0.x_1 x_2 x_3 \ldots$ where $x_i = 1$ iff $i \in X$.
  Hence $|P(\N)| = |[0,1)|$.
  \qed

\newpage
Note that Cantor's theorem
\[
|X| < |P(X)|
\]
implies that there there is no \lq\lq
largest" set since
\[
|X| < |P(X)| < |P(P(X))| < |P(P(P(X)))| < \cdots
\]
In particular we have $|\N| < |P(\N)|$.
From the above exercise, $|P(\N)| = |\R|$.
Hence
\[
|\N| < |\R|
\]
An interesting question is
whether you can find a set $X$ such that
\[
|\N| < |X| < |\R|
\]
In other words, is there anything in between $|\N|$ and $|\R|$?
In some books you will find the symbol
\defone{$\aleph_0$} (pronounced \lq\lq $\aleph$-null", google for the
pronounciation of $\aleph$)
which stands for
$|\N|$; we also write
\sidebarskip{12pt}\defone{$2^{\aleph_0}$}\sidebarskip{0pt}
for $|P(\N)|$.
The orders of
infinite sets together with natural numbers are called
\defone{ordinals}.
You can think of ordinals as sizes of sets.
Here are some ordinals:
\[
0 < 1 < 2 < 3 < \ldots < |\N| < |P(\N)| = |\R| < |P(\R)| < |P(P(\R))|
< |P(P(P(\R)))|
< \ldots
\]
So the
above question can be rephrased as whether there is any ordinal $x$
such that $\aleph_0 < x < 2^{\aleph_0}$.
Cantor asked this question 1878 and believed that there is no such $x$.
This is called the
\defone{continuum hypothesis}
\sidebarskip{12pt}\defone{CH}\sidebarskip{0pt}:

\textsc{Continuum Hypothesis}.
There is no ordinal $x$ between $|\N|$ and $|\R|$, i.e.,
there is no ordinal $x$ such that
\[
\aleph_0 < x < 2^{\alpha_0}
\]

Cantor spent his whole life trying to prove the CH but was not able to.
The continuum hypothesis is so important that it was
one of the famous
\href{https://en.wikipedia.org/wiki/Hilbert%27s_problems}{23 problems of David Hilbert}
announced at the
1900 centennial meeting of the international congress of mathematicians.
Without going into details, it was later proved that
CH can neither be proved right nor proved wrong.
Specifically, in 1940 G\"odel proved that the statement
\lq\lq there is \textit{some} $x$ such that $|\N| < x < |\R|$"
cannot be proven
and then in 1963 Paul Cohen proved that the statement
\lq\lq there is \textit{no} $x$ such that $|\N| < x < |\R|$"
cannot be proven, both
proved under some \lq\lq general and reasonably assumptions
on what is meant by sets and logic".
It's important to understand that
Godel and Cohen did not say that CH is true.
And they did not say it's false.
They are saying that CH 
and the logical opposite of CH, i.e., $\lnot\operatorname{CH}$ cannot be proved.
Set theory and logic of this type is still under active research.

