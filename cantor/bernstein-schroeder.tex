\sectionthree{Berstein-Schr\"oder Theorem}
\begin{python0}
from solutions import *; clear()
\end{python0}

Suppose you have two finite sets $X$ and $Y$.
If $X$ and $Y$ has the same number of elements,
then of course I can find a 1--1 and onto from $X$ and $Y$.

Now suppose I have two finite sets $X$ and $Y$
but I only tell you there is a 1--1 function from $X$ to $Y$.
What can you tell me about the sizes $|X$ and $|Y|$?
Then it must be true that
\[
|X| \leq |Y|
\]
What if I tell you there's also a 1--1 function from $Y$ to $X$,
Then of course
\[
|Y| \leq |X|
\]
Since $|X|$ and $|Y|$ are finite, I get
\[
|X| = |Y|
\]
right away.
That's because for integers $a$ and $b$,
\[
a \leq b
\text{ and }
b \leq a
\]
implies
\[
a = b
\]
However if $X$ and $Y$ are sets in general (i.e., not neceesarily finite)
and I know that
\[
|X| \leq |Y|
\text{ and }
|Y| \leq |X|
\]
it seems to be true that
\[
|X| = |Y|
\]
but hang on ...
you should not think of numbers here.
When it comes to infinities, you always have to be careful.
What I'm saying above is this:
If
\[
|X| \leq |Y|
\text{ and }
|Y| \leq |X|
\]
i.e. 
\[
\text{there is a 1--1 function $X \rightarrow Y$}
\]
and
\[
\text{there is a 1--1 function $Y \rightarrow X$}
\]
then
\[
|X| = |Y|
\]
i.e.,
\[
\text{there is a 1--1 onto function $X \rightarrow Y$}
\]
So I'm making this statement:
If there are 1--1 functions
\[
X \rightarrow Y, \,\,\, Y \rightarrow X
\]
then there is a 1--1 and onto function
\[
X \rightarrow Y
\]
If you think the proof easy, go ahead and try it.
This is not a trick question.
The above statement is actually true. 
I'll state it as a theorem for reference, but without proof.
The theorem is usually called the
Bernstein--Schr\"oder theorem.
The statement was first stated by Cantor in 1887.
Bernstein and Schr\"oder provided proofs 1897.
It was later found (1902) that Schr\"oder's proof is incorrect.
Unknown to everyone, Dedekind already
had a proof in 1887.

\begin{thm}
  \textnormal{\textbf{Bernstein--Schr\"oder}}
  \sidebar{\textnormal{Bernstein--Schr\"oder}}
  Let $X$ and $Y$ be sets such that
  \[
  |X| \leq |Y| \text{ and }
  |Y| \leq |X|
  \]
  Then
  \[
  |Y| = |X|
  \]
\end{thm}
  

What does the Berstein--Schr\"ofer theorem gives you?
Well if you want to show $|X| = |Y|$, you can try to find
a 1--1 and onto function from $X$ to $Y$.
Or, by Bernstein--Schroeder, you can find a 1--1 function
from $X$ to $Y$ and another 1--1 function from $Y$ to $X$.
In some cases, finding a 1--1 and onto function might be harder.
The reason is because you need to find one function that satisfies
\textit{two} conditions.
It's true that using Bernstein--Schroeder requires you
to find \textit{two} functions.
However in many cases finding
two functions each satisfying \textit{one} condition (i.e., 1--1)
is actually easier.

%-*-latex-*-

\begin{ex} 
  \label{ex:prob-00}
  \tinysidebar{\debug{exercises/{disc-prob-28/question.tex}}}

  \solutionlink{sol:prob-00}
  \qed
\end{ex} 
\begin{python0}
from solutions import *
add(label="ex:prob-00",
    srcfilename='exercises/discrete-probability/prob-00/answer.tex') 
\end{python0}


%-*-latex-*-

\begin{ex} 
  \label{ex:prob-00}
  \tinysidebar{\debug{exercises/{disc-prob-28/question.tex}}}

  \solutionlink{sol:prob-00}
  \qed
\end{ex} 
\begin{python0}
from solutions import *
add(label="ex:prob-00",
    srcfilename='exercises/discrete-probability/prob-00/answer.tex') 
\end{python0}

\begin{comment}
  |(0, 1)| = |(0, x)| where x > 0: easy.
  |(0, 1)| = |(a, a+1)|: easy.

  |(0,1)| = |(0, 1]|:
    \leq is easy by inclusion.
    \geq use map x -> x/2.
    Therefore by bernstein schroeder ... done.
\end{comment}   

%-*-latex-*-

\begin{ex} 
  \label{ex:prob-00}
  \tinysidebar{\debug{exercises/{disc-prob-28/question.tex}}}

  \solutionlink{sol:prob-00}
  \qed
\end{ex} 
\begin{python0}
from solutions import *
add(label="ex:prob-00",
    srcfilename='exercises/discrete-probability/prob-00/answer.tex') 
\end{python0}


%-*-latex-*-

\begin{ex} 
  \label{ex:prob-00}
  \tinysidebar{\debug{exercises/{disc-prob-28/question.tex}}}

  \solutionlink{sol:prob-00}
  \qed
\end{ex} 
\begin{python0}
from solutions import *
add(label="ex:prob-00",
    srcfilename='exercises/discrete-probability/prob-00/answer.tex') 
\end{python0}

\begin{comment}
  |[0,1]| = |[0,1]^2|:
  \leq is easy
  \geq: let (x, y) \in [0,1]^2. interleave digits of x,y to get z in [0,1].
\end{comment}


%-*-latex-*-

\begin{ex} 
  \label{ex:prob-00}
  \tinysidebar{\debug{exercises/{disc-prob-28/question.tex}}}

  \solutionlink{sol:prob-00}
  \qed
\end{ex} 
\begin{python0}
from solutions import *
add(label="ex:prob-00",
    srcfilename='exercises/discrete-probability/prob-00/answer.tex') 
\end{python0}

