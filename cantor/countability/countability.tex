\section{Set cardinality, countability, and ordinals} \label{S:countability}

Throughout this section, $X$ is a set. Recall that I defined
$|X|$ for finite sets, i.e., $|X|$ is the number of elements in $X$.
$|X|$ is usually called the
\defone{size}
or the
\sidebarskip{12pt}\defone{cardinality}\sidebarskip{0pt}
of the set.
Now I want to talk about infinite sets.

Georg Cantor is usually considered the founder of modern
set theory.
He was the first to realized that although
the idea of a set is simple, in fact other than finite sets,
sets in general are very complex.
Many of the concepts in this chapter was first defined
by
Richard Dedekind.
But it was Cantor who realized their importance.
Cantor and Dedekind were contemporary and were close friends
and met frequently for discussion.
But Dedekind acted more like a sounding board to Cantor's ideas.
Cantor thoughts on set theory made him realized that
when looking at infinite sets, it is
still possible to differentiate between them.
In other words, there are many different types of infinities.
In a sense, this is somewhat similar to the study of functions:
$f(x) = x$ and $g(x) = \ln x$ both go to infinity as $x$ goes to
infinity. But their order of growth are different.
Cantor proved that there are not just different infinities,
but there are \textit{infinitely} many infinities!
Cantor's contribution to set theory has important
consequences to analysis, topology, logic, and
Turing machines in the theory of automata and
theoretical computational complexity.
As a consequence of Cantor's work, you can for instance show that
the set of C++ programs is countably infinite (I'll
define this precisely later), but the set of
boolean functions $\N \rightarrow \{0, 1\}$ is uncountably
infinite. Note that both set are infinite but in different ways:
there are more boolean functions $\N \rightarrow \{0, 1\}$
than there are C++ programs.
Therefore Cantor's set theory tells you right away that
there is a boolean function that cannot be computed by a C++
program.

Now ... on to infinite sets ...

Recall that a function $f: X \rightarrow Y$
is
\defone{one-to-one}
(I'll write 1--1) or
\sidebarskip{0pt}\defone{injective}\sidebarskip{0pt}
if
$f(x) = f(x')$ implies $x = x'$.
$f$ is
\sidebarskip{12pt}\defone{onto}\sidebarskip{0pt} or
\sidebarskip{24pt}\defone{surjective}\sidebarskip{0pt}
if for every $y \in Y$, there is some $x \in X$ such that
$f(x) = y$.
A
\sidebarskip{12pt}\defone{bijection} %\sidebarskip{0pt} 
is a function that is both 1--1 and onto.
If a function is a bijection I will say it is \lq\lq 1--1 and onto".
A bijection is also called a
\sidebarskip{1pt}\defone{one-to-one correspondence}.

If we are going to only talk about finite sets
then $|X| = |Y|$ means $X$ and $Y$ have the same number of
distinct elements,
$|X| < |Y|$ means that $Y$ has strictly more elements than
$X$.
Etc.
I'm going to generalize the above notions of
$|X| = |Y|$, $|X| < |Y|$, $|X| \leq |Y|$ to include cases when
$X$ and $Y$ are infinite.

\begin{defn}
  Let $X$ and $Y$ be sets.
  \begin{itemize}
  \item
    We will write $|X| \leq |Y|$ if there is a
    1--1 function $X \rightarrow Y$.
  \item
    If there is a bijection between
    $X$ and $Y$, we write $|X| = |Y|$.
    If $|X| = |Y|$, I will also say that $X$ and $Y$
    have the same
    \defone{cardinality}
    or that they are
    \sidebarskip{12pt}\defone{equinumerous}.
    I will write $|X| \neq |Y|$
    if $|X| = |Y|$ is not true, i.e., if there is no 1--1 onto function
    from $X$ to $Y$.
  \item
    I will write $|X| < |Y|$ if
    $|X| \leq |Y|$ but $|X| \neq |Y|$.
    In other words $|X| < |Y|$ means there is a 1--1 function
    from $X$ to $Y$ but there is no 1--1 and onto function
    from $X$ to $Y$.
  \end{itemize}
  Of course we define $|X| > |Y|$ if $|Y| < |X|$ and
  $|X| \geq |Y|$ if $|Y| \leq |X|$.
\end{defn}

Note that the above definition generalizes
the other simpler definition of $|X| \leq |Y|$ for the case
when $X$ and $Y$ are finite.
(Right?)

\begin{defn}
  $X$ is \defone{infinite} iff $|X| = |X - \{x\}|$ for any $x \in X$.
\end{defn}

\begin{ex}
  A set $X$ is \textit{finite} if there is some $n \in \N$ such that
  $|X| = |\{1, 2, 3, ..., n\}|$.
  But I can also define $X$ to be finite if $X$ is not
  infinite.
  Prove that the definitions are the same.
\end{ex}

\begin{defn}
  Let $X$ be a set.
  \begin{itemize}
  \item
    We will say that $X$ is \defone{countable} if
    either $X$ is finite or $|X| = |\N|$, i.e.,
    there is a bijection between $X$ and
    $\N$.
  \item
    $X$ is \defone{uncountable} if it is not countable.
  \end{itemize}
\end{defn}

\newpage
\begin{ex}
  Prove the following:
  \begin{itemize}
  \item $|X| = |X|$ (easy)
  \item
  If $|X| \leq |Y|$ and $|Y| \leq |Z|$, then $|X| \leq |Z|$.
  Prove the same statement when $\leq$ is replaced by $<$ and then by
  $=$.
  \qed
  \end{itemize}
\end{ex}

Don't be fooled: The statement $|X| \leq |Y|, |Y| \leq |X| \implies
|X| = |Y|$ is true, but the proof is not immediate.
See next section on the Berstein--Schr\"oder theorm.

\newpage
\begin{ex}
Prove that $\N$ is countably infinite.
\end{ex}


\newpage
\begin{ex}
Let $E$ be the set of even numbers. Since $E$ is just half of $\N$,
we must have $|E| < |\N|$, right? WRONG! Prove that $|E| = |\N|$.
\end{ex}

\newpage
\begin{ex}
OK, but surely, $|\Q| > |\N|$, right? WRONG! Prove that $|\Q| =
|\N|$.
\end{ex}

\newpage
\begin{ex}
Prove that $|\N \times \N| = |\N|$. In fact $|\N^k| = |\N|$ for any
integer $k>0$. More generally, suppose $X_1, \ldots, X_n$ are
countable sets, prove that $X_1 \times \cdots \times X_n$ is also
countable.
\end{ex}

\newpage
\begin{ex}
  Prove that the subset of a countable set is countable.
  Prove that the superset of an uncountable set is uncountable.
\end{ex}

\newpage
\begin{ex}
Let $X_1, X_2, \ldots, X_n$ be countable sets. Prove that
$\bigcup_{i=1}^n X_i$ and $\bigcap_{i=1}^n X_i$ are both
countable.
In fact if you have countably many countable sets,
$X_1, X_2, \ldots$, then
$\bigcup_{i=1}^\infty X_i$ is countable.
One would say
\lq\lq a countable union of countable sets is countable".
\end{ex}

In the above \lq\lq countable union" means the union of
countably many sets.
So this means that the sets of the union is finite, for instance
\[
\bigcup_{i=1}^5 X_i
\]
where $X_i = [i, i + 1)$ (intervals of $\R$), or
\[
\bigcup_{i=1}^\infty X_i
\]
In the second union, there are countable many $X_i$'s.

An example of an uncountable union would be
\[
\bigcup_{r \in [0, 100)} X_r
\]
where $X_r = [r, r + 1)$.
  In this case there are uncountably many $X_r$'s since the
  $r$ runs through $[0, 100)$ which is not countable.
    For instance this union involves $[1, 2)$,
      $[3.5, 4.5)$,
      $[\sqrt{2}, \sqrt{2} + 1)$,
      $[\pi, \pi + 1)$, etc.
Clearly there are uncountably many such $X_r$.
            
\newpage
So what about $\R$? $\R$ is in fact uncountable. As a matter of
fact, the interval $[0,1)$ is uncountable.
This was first proved by Cantor.
The method of proof for
this theorem discovered by Cantor
is very important. Some people call this method the
\defone{diagonalization argument}.

\begin{thm} \textnormal{(Cantor)}
$[0,1)$ is uncountable.
\end{thm}


If you have taken discrete math, you should be able to give a proof
(or a semi-proof close to the correct one.)
You should attmept it.

SPOILER ALERT ... turn the page for the proof.


\newpage
\textit{Proof.} Every real number in $[0,1)$ looks like a decimal with no
integer part. For instance $0.123 \in [0,1)$.

We prove by contradiction. Suppose $[0,1)$ is countable. So let's say
the complete list of real numbers in $[0,1)$ is given by the list
$x_1, x_2, x_3, \ldots$. Let's construct a real number $x \in
[0,1)$ which is \textbf{not} in the above list. This will give us a
contradiction. Right?

OK. So let's begin. I will construct $x$ by giving you the decimal
expansion. I will also\lq\lq avoid" the list $x_1, x_2, x_3\ldots$
so that $x$ is not any of them.

Let's begin by \lq\lq avoiding" $x_1$. Now $x_1$ is of the form
$0.a...$ where $a$ is a digit from 0 to 9. If $a=0$, I will say
that $x=0.1\ldots$; otherwise, I will say that $x=0.0\ldots$. So
obviously $x \neq x_1$.

I'll repeat this for $x_2$. Say $x_2 =
0.bc\ldots$, then the second decimal place of $x$ is $1$ if $c=0$;
otherwise the second decimal place of $x$ is $0$.
I have $x \neq x_2$.

I'll then make the 3rd decimal place of $x$ different from the
third decimal place of $x_3$.

Etc. Get it?

For instance suppose my list of $x_1, x_2, x_3, ...$ looks like this:
%\begin{figure}
\begin{longtable}{cccccccccc}
  $x_1$ & = & 0 & . & \textbox{1} & 3 & 0 & 2 & 9 & 0  \\
  $x_2$ & = & 0 & . & 0 & \textbox{0} & 8 & 3 & 0 & 7  \\
  $x_3$ & = & 0 & . & 3 & 1 & \textbox{3} & 1 & 3 & 1  \\
  $x_4$ & = & 0 & . & 0 & 0 & 8 & \textbox{7} & 2 & 7  \\
  $x_5$ & = & 0 & . & 1 & 2 & 8 & 0 & \textbox{2} & 7  \\
  $x_6$ & = & 0 & . & 0 & 6 & 1 & 9 & 2 & \textbox{0}  \\
\end{longtable}
%\end{figure}

So for the decimal places of $x$, I choose not 1, not 0, not 3, not 7,
not 2, not 0, etc.
For instance I can choose 0, 1, 0, 0, 0, 1, ...
Hence for this case my $x$ looks like
\[
x = 0.010001...
\]

So let's have a formal proof. Suppose $x_i$ is
\[ x_i = 0.x_{i,1}x_{i,2}x_{i,3} \ldots x_{i,i} \ldots \]
for $i=1,2,3,\ldots$. Then we let $x$ be the number
\[ x = 0.y_1 y_2 y_3 \ldots y_i \ldots \]
where
\[ y_i =
\begin{cases}
 0 \text{ if } x_{i,i} \neq 0 \cr
 1 \text{ if } x_{i,i} = 0 \cr
\end{cases}
\]
Note in particular that $y_i \neq x_{i,i}$ for all $i>0$. Then $x
\neq x_i$ for all $i$. Why? Otherwise say $x = x_i$ for some $i$.
But that implies that they have the same decimal expansion. In
other words
\[ y_j = x_{i,j} \]
for all $j$. But by construction,
\[ y_i \neq x_{i,i} \]
Contradiction!!!

But wait, there's a ... \textbf{hole} in the proof. We assumed that if
two real numbers are the same, they must then have the same decimal
expansion. Is that true?



\newpage
\begin{ex}
Show that $0.09999\ldots = 0.1$ numerically although they are
written differently. [Hint: Remember geometric series?
\[ a + ar + ar^2 + ar^3 + \cdots = \frac{a}{1-r} \]
Don't just stare at it. Try to use this formula.]
\end{ex}

\newpage
\begin{ex}
  Complete the proof of Cantor's Theorem with the exercise on the previous
  page.
\end{ex}

\newpage
\begin{ex}
  This is something that is hard to believe:
  Prove that
  \[
  \text{$[0,1]$ and $[0,1] \times [0,1]$ are equinumerous!!!}
  \]
  The interval $[0,1]$ is called the \defone{unit interval}
  and the product $[0,1] \times [0,1]$ is called the \defone{unit square}.
  So basically there are as many points on the unit interval
  as there are on the unit square.
  (Cantor couldn't believe it when he proved the above fact.
  In a letter to Dedekind, he wrote,
  \lq\lq I see it, but I don’t believe it!")
\end{ex}

\newpage
\begin{ex}
Prove that there are uncountably many functions $\N \rightarrow
\{0,1\}$. [Hint: Diagonalization.]
\end{ex}


\newpage
\begin{ex}
  Is the set of polynomials with integer coefficients countable?
  Two polynomials are considered the same if their coefficients are the same.
  What about the set of polynomials with rational coefficients?
\end{ex}


\newpage
\begin{ex}
  We already know from a previous exercise that a finite
 product of countable sets is countable. Suppose $X_1, X_2, \ldots$
 are countable. Is $X_1 \times X_2 \times X_2 \times \cdots$
 countable?
\end{ex}

\newpage
You now know that $\R$ is not countable and $\Q$ is countable.
So $\R$ is sort of \lq\lq huge".
In particular the set of $\R - \Q$ (irrational numbers) is \lq\lq huge".
You can actually
subdivide $\R$ further.
There is an important set of numbers, call the
set of algebraic numbers in $\R$.
I'll write $A$ for the set of algebraic numbers.
Some algebraic numbers are complex while some are real.
Here's the definition of an algebraic number.

A number $\alpha \in \C$ is an \defone{algebraic number}
if $\alpha$ is the root of a polynomial with coefficients in $\Q$.
Just for practice, show that $\Q \subset A$.
Note that $\sqrt{2}$ is not rational.
However $\sqrt{2}$ is algebraic.
In fact show that if $x > 0$ and $y > 0$ are rational, then
$x^y$ is algebraic.
Now show that $i$ is algebraic where $i = \sqrt{-1}$.

The relationship between $\Z, \Q, A, \R, \C$ is:
\[
\Z \subset \Q \subset A \cap \R \subset \R \subset \C
\]
and
\[
\Z \subset \Q \subset A \subset \C
\]
In other words the set of algebraic numbers $A$ is a subset of $\C$
and some algebraic numbers are real, but not all.
Enough practice ... now for the real exercise:

\begin{ex}
  Prove that $A$ is countable.
  \qed
\end{ex}

Note that the set of algebraic numbers in $\R$ is countable:
$A \cap \R$ is countable.
But we know that $\R$ is uncountable.
This means that there are non-algebraic numbers in $\R$.
In fact there are uncountably many non-algebraic numbers in $\R$.
These are called transendental numbers.
In other words $\alpha \in \C$ is \defone{transcendental}
if $\alpha$ is not the root of a polynomial with rational coefficients.
Here are two examples: $\pi$ and $e$ are transcendental.

I'm sure you have heard in some previous classes that $\pi$ and $e$
are irrational.
But $\pi$ and $e$ are more than irrational.
They are transcendental.

Proving a number is transcendental is not easy!
For instance to prove $\pi$ is transcendental, by definition,
you have to prove $\pi$ is not algebraic.
That means $\pi$ is \textit{not} the solution of \textit{any} polynomial
with $\Q$ coefficients.

While the ancient Greek philosophers discovered the existence of
irrational numbers (example $\sqrt{2}$) around 300BC,
the first transcendental number was only discovered around 1850:
\[
\sum_{n = 1}^\infty 10^{-n!}
\]
Try to write down it's decimal representation up to say 50 decimal placed.
This number was artificially created by
\href{https://en.wikipedia.org/wiki/Joseph_Liouville}{Liouville}
to \lq\lq avoid"
polynomials with $\Q$ coefficients.

On the other hand $\pi$ was proven to be transcendental in
1882 by
\href{https://en.wikipedia.org/wiki/Ferdinand_von_Lindemann}{von Lindemann}
and $e$ was proven to be transcendental a couple of years early by
\href{https://en.wikipedia.org/wiki/Charles_Hermite}{Hermite} in 1873.
The fact that there are uncountably many transcendental numbers
was proven by Cantor in 1874, one year after Hermite's result.

We actually know very little about transcendental numbers.
For instance, we do not know if the following constant,
the Euler–Mascheroni constant, is
transcendental or not:
\[
\gamma = \lim_{n \rightarrow \infty}
\left(
\sum_{k = 1}^n \frac{1}{k}
-
\ln n
\
\right)
= 0.5772...
\]
($\ln = \log_e$).
In fact, scratch that, we don't even know if $\gamma$ is rational or
irrational!!!
This constant is ubiquitous and appears in CS, math, physics, etc.


\newpage
\begin{ex}
  Prove that $P(\N)$ is uncountable.
  This means $|\N| < |P(\N)|$.
  In fact even more is true:
  if $X$ is a set, then $|X| < |P(X)|$.
  (This is called Cantor's theorem and is extremely important.
  See later.)
\end{ex}


\newpage
\begin{ex}
  \mbox{}
 \begin{itemize}
 \item Prove that there are countably many C++ programs.
 \item Since there number of functions from $\N$ to $\{0,1\}$ is
  uncountable, what does that tell you?
 \end{itemize}
\end{ex}


\newpage
\begin{thm} \textnormal{(Cantor)}
For any set $X$, $|X| < |P(X)|$.
\end{thm}

The fact $|X| \leq |P(X)|$ is easy.

Now I'm going to prove that you cannot find an onto function
from $X$ to $P(X)$.
Suppose on the contrary that 
$f : X \rightarrow P(X)$ is an onto function.
I need to arrive at a contradiction.
How?
Since $f$ is onto,
for any subset $Y$ of $X$, there is some $x \in X$ such that
$f(x) = Y$.
I need to construct some $Y$ that will cause some problem (contradiction).

Now $Y$ would look like this:
$Y = \{x \in X \mid P(x)\}$ where $P(x)$ is some condition.
The condition $P(x)$ depends on whatever I have now.
For instance $P(x)$ might depends on $X$, $P(X)$, $f$.
So I want
to say there is some $x' \in X$ such that 
\[
f(x') = \{x \in X \mid P(x) \}
\]
will lead to a contradiction.
The question is what should $P(x)$ be?
What about this $x'$?
How can it be used to craft a contradiction?
Well $x'$ is an element of $X$
and $f(x')$ is a subset of $X$.
The relationship between $x'$ and $f(x')$ is
either $x' \in f(x')$ or $x' \not\in f(x')$.
Suppose $x' \in f(x')$, i.e.,
\[
x' \in f(x') = \{x \in X \mid P(x) \}
\]
Since $x' \in \{x \in X \mid P(x) \}$, then of course $x'$
satisfy $P(x')$.
In other words
\[
x' \in f(x') \implies P(x')
\]
See a contradiction? Suppose $P(x)$ is \lq\lq$x \not\in f(x)$".
Then
\[
x' \in f(x') = \{x \in X \mid x \not\in f(x) \}
\]
i.e.,
\[
x' \in f(x') \implies x' \not\in f(x')
\]
which is clearly a contradiction.
However this is when I assume $x' \in f(x')$.
What if $x' \not\in f(x')$?
Would the set 
$\{x \in X \mid x \not\in f(x) \}$ still give me a contradiction?
Why yes!
Because if $x' \not\in f(x')$, then $x'$ does not
satisfy $P(x)$, i.e., it is not true that
$x' \not\in f(x')$.
So now we're ready to write the proof.

\proof
First we prove that $|X| \leq |P(X)$.
Define the function $f: X \rightarrow P(X)$ to be
$f(x) = \{x\}$.
This function is 1--1:
If $x,x' \in X$ and $f(x) = f(x')$, then $\{x\} = \{x'\}$
and hence $x = x'$.
Therefore $|X| \leq |P(X)|$.

Now we will prove that $|X| \neq |P(X)|$.
In other words, we will prove that there's no onto function
from $X$ to $P(X)$.
Assume on the contrary that $f: X \rightarrow P(X)$ is an onto function.
Let
\[
Y = \{x \in X \mid x \not\in f(x) \}
\]
$Y$ is a subset of $X$.
Since $f$ is onto, there is some $x' \in X$ such that
\[
f(x') = Y = \{x \in X \mid x \not\in f(x) \}
\]
We will consider two cases: $x' \in f(x')$ and $x' \not\in f(x')$
and show that in each cases, we will arrive at a contradiction.

If $x' \in f(x')$, then
\begin{align*}
  x' &\in f(x') = Y = \{x \in X \mid x \not\in f(x) \} \\
  \THEREFORE x' &\text{ satisfies the condition } x' \not\in f(x') \\
  \THEREFORE x' &\not\in f(x') 
\end{align*}
which is a contradiction.

If $x' \not\in f(x')$, then
\begin{align*}
x' &\not\in f(x') = Y = \{x \in X \mid x \not\in f(x) \} \\
\THEREFORE x' &\not\in \{x \in X \mid x \not\in f(x) \} \\
\THEREFORE x' &\text{ does not satisfy the condition } x' \not\in f(x') \\
\THEREFORE x' &\in f(x')
\end{align*}
which is a contradiction.

In both cases, I arrive at contradictions.
Hence our assumption on the existence of an onto function
from $X$ to $P(X)$ does not hold.
Hence there is no onto function from $X$ to $P(X)$.
\qed


%\begin{ex} Prove or disprove the following. Let $X,X', Y,Y'$ be
%sets. Let $\hom(X,Y)$ denote the set of functions from $X$ to $Y$.
% \begin{mylist}
%  \item $X \subseteq X' \implies |\hom(X,Y)| \leq |\hom(X',Y)|$
%  \item $X \subsetneq X' \implies |\hom(X,Y)| < |\hom(X',Y)|$
%  \item $Y \subseteq Y' \implies |\hom(X,Y)| \leq |\hom(X,Y')|$
%  \item $Y \subsetneq Y' \implies |\hom(X,Y)| < |\hom(X',Y)|$
% \end{mylist}
%\end{ex}

\newpage
\textsc{On non-uniqueness of decimal representations}.

Here's a real number
\[
1433.235246457234346
\]
Of course a decimal expanion need no terminate.
For instance $\pi$ does not terminate and furthermore the pattern
of the decimal expansion does not repeat.
There are decimal expansion that repeats.
For instance
\[
0.9999999999\ldots
\]
It can be shown (probably in precalc) that
\[
0.9999999999\ldots = 1
\]

The question is this:

\begin{ex}
  When does two decimal expansion represent the same real number?
\end{ex}

If you have taken Calc 2, you should have enough background to answer
this yourself.
So you can treat this as an exercise and write a short paper on it.
You should think about this question and try it out on your own.
Maybe you want to chat or work with another student.

SPOILER ALERT ... turn the page for the answer.


\newpage
Note that
\[
0.99999\ldots
\]
and
\[
1.00000\ldots
\]
represent the same real number.
Also,
\[
0.1234599999\ldots
\]
and
\[
0.1234600000\ldots
\]
represent the same real number.
In general when does distinct two decimal representations represent the
same real number?
First I'll focus on the representation problem in the interval $[0,1)$.

Let
$x = 0.x_1x_2x_3...$
and
$y = 0.y_1y_2y_3...$.
Assume the sequence of decimal places are different:
\[
(x_i)_{i=1}^\infty
\neq
(y_i)_{i=1}^\infty
\]
I claim that $x = y$ iff there is some $N$ such that
$x_i = y_i$ for $i < N$,
$x_N = y_N - 1$, and
$x_j = 9, y_j = 0$ for $j > N$, or the conditions are switched between
$x$ and $y$.

$\impliedby$: This is easy.

$\implies$:
If
\[
(x_i)_{i=1}^\infty
\neq
(y_i)_{i=1}^\infty
\]
then there must be a smallest $N \geq 1$ such that
\[
x_i = y_i
\]
for $i < N$
and $x_N \neq y_N$.
Of course either $x_N < y_N$ or $x_N > y_N$.
Without loss of generalize, assume $x_N < y_N$.
I want to prove that $x_N, y_N$ differ by $1$.
Let's see what happens.
Since $x = y$ and $x_i = y_i$ for $i < N$, I have
\[
y - x = (y_N - x_N)10^{-N} + \sum_{j=N+1}^\infty (y_j - x_j)10^{-j}
\]
I claim that in this case $y - x > 0$.
Since $y - x = 0$,
\[
0 = (y_N - x_N)10^{-N} + \sum_{j=N+1}^\infty (y_j - x_j)10^{-j}
\]
i.e.,
\[
(x_N - y_N)10^{-N} = \sum_{j=N+1}^\infty (y_j - x_j)10^{-j}
\]
i.e.,
\[
x_N - y_N = \sum_{j=1}^\infty (y_{N+j} - x_{N+j})10^{-j}
\]
Now note that since $x_j, y_j$ are integers in $[0, 9]$,
\[
-9 \leq y_j - x_j \leq 9
\]
Hence
\[
-9 \sum_{j=1}^\infty 10^{-j} \leq
\sum_{j=1}^\infty (y_{N+j} - x_{N+j})10^{-j} \leq
9 \sum_{j=1}^\infty 10^{-j}
\]
i.e.,
\[
-9 \cdot \frac{1}{10^{1}} \cdot \frac{1}{1 - 1/10}
\leq
\sum_{j=1}^\infty (y_{N+j} - x_{N+j})10^{-j}
\leq
9 \cdot \frac{1}{10^{1}} \cdot \frac{1}{1 - 1/10} 
\]
i.e.,
\[
-1 
\leq
\sum_{j=N+1}^\infty (y_{N+j} - x_{N+j})10^{-j}
\leq
1
\]
Together with the above equation
\[
x_N - y_N = \sum_{j=1}^\infty (y_{N+j} - x_{N+j})10^{-j}
\]
I get
\[
-1 
\leq
x_N - y_N
\leq
1
\]
This means that
$x_N - y_N$
is $-1, 0, 1$.
But hang on: remember that $x_N \neq y_N$.
So I now know that
$x_N,  y_N$ differ by $1$.
Since I'm assuming $x_N < y_n$, I get
\[
x_N = y_N - 1
\]

So now from
\[
x_N - y_N = \sum_{j=1}^\infty (y_{N + j} - x_{N+j})10^{-j}
\]
I have
\[
-1 = x_N - y_N = \sum_{j=1}^\infty (y_{N+j} - x_{N+j})10^{-j}
\]
I claim that $y_{N+1} = 0$ and $x_{N+i} = 9$.
From
\[
-1 = \sum_{j=1}^\infty (y_{N+j} - x_{N+j})10^{-j}
\]
I get
\[
-1 - (y_{N+1} - x_{N+1})10^{-1} = \sum_{j=2}^\infty (y_{N+j} - x_{N+j})10^{-j}
\]
and therefore
\[
-10 - (y_{N+1} - x_{N+1}) = \sum_{j=2}^\infty (y_{N+j} - x_{N+j})10^{-j}
\]
i.e.,
\[
-10 - (y_{N+1} - x_{N+1}) = \sum_{j=1}^\infty (y_{N+1+j} - x_{N+1+j})10^{-j}
\]
I claim that $y_{N+1} = 0$ and $x_{N+1} = 9$.
The sum above can be bounded:
\[
-1 = 
-9 \cdot \frac{1}{10} \cdot \frac{1}{1 - 1/10}
\leq
\sum_{j=1}^\infty (y_{N+1+j} - x_{N+1+j})10^{-j}
\leq
9 \cdot \frac{1}{10} \cdot \frac{1}{1 - 1/10}
= 1
\]
Hence
\[
-1 \leq -10 - (y_{N+1} - x_{N+1}) \leq 1 
\]
i.e. 
\[
9 \leq - (y_{N+1} - x_{N+1}) \leq 11 
\]
i.e.,
\[
-9 \geq y_{N+1} - x_{N+1} \geq -11 
\]
However the since $x_{j}, y_{j}$ are in $[0, 9]$,
\[
-9 \leq y_{N+1} - x_{N+1} \leq 9
\]
Hence
\[
y_{N+1} - x_{N+1} = -9
\]
which is achieved only when
\[
y_{N+1} = 0, \,\,\, x_{N+1} = 9
\]
which is what I claimed earlier.
With this information,
\[
-10 - (y_{N+1} - x_{N+1}) = \sum_{j=1}^\infty (y_{N+1+j} - x_{N+1+j})10^{-j}
\]
becomes
\[
-1 = \sum_{j=1}^\infty (y_{N+1+j} - x_{N+1+j})10^{-j}
\]
and the same argument would yield
\[
y_{N+2} = 0, x_{N+2} = 9
\]
Etc.
By induction, one can prove that
\[
y_{N+j} = 0, x_{N+j} = 9
\]
for $j = 1, 2, 3, ...$.

I have now shown that if
$0.x_1x_2x_3...$
and
$0.y_1y_2y_3...$
are two decimal representations, then
either $x_i = y_i$ for all $i$ or, if not,
then there is a smallest $N$ such that
$x_i = y_i$ for $i < N$,
and $x_N \neq y_N$.
Furthermore,
$x_N$ and $y_N$ differs by exactly $1$.
Without loss of generality, if $x_N = y_N - 1$, then
$x_j = 9$ and $y_j = 0$ for $j > N$.

What about decimal representations with nonzero integer part?
Easy!
Consider
\[
x = x_{-m}x_{-m+1}x_{-m+2}\cdots x_0 \ . \ x_1x_2x_3...
\]
and
\[
y = y_{-n}y_{-n+1}y_{-n+2}\cdots y_0 \ . \ y_1y_2y_3...
\]
All I need to do is to multiply these two numbers
by $10^{-k}$ for a sufficiently large $k$
so that $x\cdot 10^{-k}$ and $y \cdot 10^{-k}$ are
two numbers in $[0,1)$.
  (Basically I'm moving their decimal point to the left
  by the same number of steps.)
  Then use the above result to get the following:
  Either $x_i = y_i$ for all $i$ of
  there is some $N$ such that $x_i = y_i$ for $i < N$,
  $x_N \neq y_N$.
  Furthermore $x_N, y_N$ differs by $1$.
  Assuming $x_N = y_N - 1$, then
  $x_j = 9$ and $y_j = 0$ for $j > N$.

  Let me state this as a theorem.
  I'm going to use descending index values.
  
  \begin{thm}
    Let $x$ and $y$ be real numbers.
    Suppose
    \[
    x = x_m x_{m - 1} \ldots x_0 \cdot x_{-1}x_{-2} \ldots
    \]
    and
    \[
    y = y_m y_{m - 1} \ldots y_0 \cdot y_{-1}y_{-2} \ldots
    \]
    be decimal representations for $x$ and $y$.
    Then $x = y$ iff
    either $x_i = y_i$ for all $i \leq m$ or
    there is some $N$ such that
    \begin{itemize}
    \item[\textnormal{(a)}] $x_i = y_i$ for $m \leq i < N$
    \item[\textnormal{(b)}] $x_N = y_N - 1$ (or $y_N = x_N - 1$),
      and
    \item[\textnormal{(c)}]
      $x_j = 9$, $y_j = 0$ for $j < N$
      (or $x_j = 0$, $y_j = 9$ for $j < N$, respectively)
    \end{itemize}
  \end{thm}

  The above is also true (and the proof is similar), if the base
  of the representation is changed to any base $B > 1$.
  In that case,
  the \lq\lq 9" in the statement of the theorem has to be changed to
  $B - 1$.
  
\qed

