%-*-latex-*-
\sectionthree{Infinite sample space and geometric distribution}
\begin{python0}
from solutions import *; clear()
\end{python0}

In the section on Binomial distribution, I perform a random experiment that is a fixed
length sequence of a Bernoulli trial (a random experiment with two outcomes).
Now I'm going to consider a sequence of experiments that does not have a fixed length.

I'm going to slightly generalize my definitions of
pdfs, random variables, expectation, etc. to the case when the sample space
is infinite.
Most of the changes are easy.
A pdf is still
\[
p: S \rightarrow [0,1]
\]
except that now $S$ can be an infinite set -- I will restrict
$S$ to be countable.
(See notes on Cantor set theory.)
I will still have the definition of events $A$ being subsets of $S$
and
\[
p(A) = \sum_{x \in S} p(x)
\]
and the requirement
\[
p(S) = 1
\]
Note that there are some trickyness when it comes to infinite sets
since summations might now be a sum of infinite terms.
I will assume that all the obviously algebraic operations are valid.

Consider this problem:
How many die rolls do I need to get a $\SIX$?
Of course this means \lq\lq What is the average number of
die rolls to get a $\SIX$?".

Note that for this random experiment, $S$ is the set of all tosses of
finite length where $\SIX$ occurs only as the last toss:
\begin{align*}
  S &= \{(\SIX), \\
  &\hspace{0.8cm} (\ONE, \SIX), (\TWO, \SIX), ... (\FIVE, \SIX), \\
  &\hspace{0.8cm} (\ONE, \ONE, \SIX), (\ONE, \TWO, \SIX), ..., (\FIVE, \FIVE, \SIX), \\
  &\hspace{0.8cm} ...\}
\end{align*}
Here's the pdf:
\[
p((x_0, ..., x_{n - 1})) = p^n
\]
where $p = 1/6$ and $q = 1 - p = 5/6$.
I want to count the number of tosses to reach the first six.
Therefore I use the following random variable $X$:
\[
X((x_0, ..., x_{n - 1})) = n
\]
where $n \geq 1$.

Therefore
\[
\Pr[X = n] = q^{n-1}p
\]
Hence
\[
\E[X] = \sum_{n \geq 1} n \cdot \Pr[X = n] = \sum_{n \geq 1} n q^{n-1}p = p \sum_{n \geq 1} nq^{n - 1}
\]
Note that if $|x| < 1$,
\[
\sum_{n \geq 1} nx^{n - 1} = \frac{1}{(1 - x)^2}
\]
Hence
\[
\E[X] = p \sum_{n \geq 1} nq^{n - 1} = p \frac{1}{(1 - q)^2} = p \frac{1}{p^2} = 1/p
\]
Therefore the average number of tosses to get a $\SIX$ is
\[
1/p = 1/(1/6) = 6
\]

%-*-latex-*-

\begin{ex} 
  \label{ex:prob-00}
  \tinysidebar{\debug{exercises/{disc-prob-28/question.tex}}}

  \solutionlink{sol:prob-00}
  \qed
\end{ex} 
\begin{python0}
from solutions import *
add(label="ex:prob-00",
    srcfilename='exercises/discrete-probability/prob-00/answer.tex') 
\end{python0}


Note that I can simplify the above computation a little.
I'll group up $\ONE, \TWO, ..., \FIVE$ as $\FAILURE$ -- using a random variable of course.
So I define a random variable 
\[
Y: \{\ONE, \TWO, ..., \FIVE, \SIX\} \rightarrow \{\SUCCESS, \FAILURE\}
\]
where
\[
Y(x) =
\begin{cases}
  \SUCCESS & \text{if } x = \SIX \\
  \FAILURE & \text{otherwise}
\end{cases}
\]  
Most of the proof is the same.
The experiment of each toss is now a Bernoulli trails where
the pdf on
\[
\{\SUCCESS, \FAILURE\} \rightarrow [0,1]
\]
given by
\begin{align*}
  \SUCCESS &\mapsto p = 1/6 \\
  \FAILURE &\mapsto q = 5/6
\end{align*}
This pdf is just $\Pr[Y = \bullet]$.
Now my sequence of outcomes have terms coming from $\{\SUCCESS, \FAILURE\}$, i.e.,
the sample space is
\begin{align*}
  S' &= \{(\SUCCESS), \\
  &\hspace{0.8cm} (\FAILURE, \SUCCESS), \\
  &\hspace{0.8cm} (\FAILURE, \FAILURE, \SUCCESS), \\
  &\hspace{0.8cm} ...\}
\end{align*}
Define random variable $X'$ on this $S'$ as the length of the tuples.
Then again we have
\[
\Pr[X' = n] = q^{n-1}p
\]
and
\[
\E[X'] = \sum_{n \geq 1} n \cdot \Pr[X' = n] = 1/p
\]

The above proves the following theorem:

\begin{prop}
  Consider a Bernoulli trial where the probability of success is $p$, i.e.,
  the pdf is $\{\SUCCESS, \FAILURE\}$ where $\SUCCESS \mapsto p$.
  As a random experiment, suppose we perform a sequence of this Bernoulli trial until
  the first success is achieved.
  Let $X$ be the random variable for the length of this sequence.
  \begin{myenum}
  \item
    $\Pr[X = n] = q^{n - 1}p$, i.e., 
    the probability that $n$ trials are needed to get a first success is $q^{n-1}p$. 
  \item
    $\E[X] = 1/p$, i.e., 
    the average number of trials needed to get a first success is $1/p$.
  \end{myenum}
\end{prop}
  

Note that in the above, the function $\Pr[X = n]$ and $\Pr[X' = n]$ looks like this:
\[
p: \{1, 2, 3, ... \} \rightarrow [0, 1]
\]
where $p(n) = q^{n-1}p$ where $p + q = 1$.
Note that $p(\{1, 2, 3, ...\}) = 1$.
This is called a \defone{geometric distribution}.

%-*-latex-*-

\begin{ex} 
  \label{ex:prob-00}
  \tinysidebar{\debug{exercises/{disc-prob-28/question.tex}}}

  \solutionlink{sol:prob-00}
  \qed
\end{ex} 
\begin{python0}
from solutions import *
add(label="ex:prob-00",
    srcfilename='exercises/discrete-probability/prob-00/answer.tex') 
\end{python0}
 

\begin{eg}
  The number of coin tosses to get a head is $2$.
  \qed
\end{eg}

\begin{eg}
  The number of die rolls to get either $1$ or $6$ is $3$.
  \qed
\end{eg}

In the above, I've shown
\[
\E[X] = 1/p 
\]
where $X$ is the length for the first successful trial.
There's another way to compute $\E[X]$.
Note that 
$p(\FAILURE, \FAILURE, \FAILURE, \SUCCESS)
= p(\FAILURE) p(\FAILURE) p(\FAILURE) p(\SUCCESS) = qqqp$.
Before I do the actual new computation of $E[X']$, here's a very important observation.
\begin{align*}
  \E[X']
  &= \sum_{n \geq 1} n \cdot \Pr[X' = n] \\
  &= 1 \cdot \Pr[X' = 1] + \sum_{n \geq 2} n \cdot \Pr[X' = n] \\
  &= p + \left( 2 qp + 3 qqp + 4 qqqp + \cdots \right) \\
  &= p + q \bigl( 2 p + 3 qp + 4 qqp + \cdots \bigr) \\
  &= p + q\bigl( p + qp + qqp + \cdots \\
  &\hspace{0.8cm} + 1 p + 2 qp + 3 qqp \cdots \bigr) \\
  &= p + q\left( p(S') + \E[X'] \right) \\
  &= p + q\left( 1 + \E[X'] \right)
\end{align*}
i.e., you get an equation in $\E[X']$.
Hence
\[
(1 - q)\E[X'] = p + q = 1
\]
and therefore
\[
\E[X'] = \frac{1}{1 - q} = 1/p
\]

By the way, the above computation
should remind you of the standard computation in
working with generating functions on recurrence relations.
And you know from MATH325 that generating functions are very
useful for analyzing recurrences such as finding closed forms.
Furthermore recall that generating functions are very helpful
in managing complex interactions between combinatorial problems.
For instance I use generating functions to count number of
solutions of $a + b + c + d = 1000$ where each of the terms
$a,b,c,d$ have their own constraints.
For the case for analyzing a sequence of
combinatorial problems such as a sequence of coin tosses or
die rolls, probability generating function is a very powerful tool.
See later section on probability generating function.

%-*-latex-*-

\begin{ex} 
  \label{ex:prob-00}
  \tinysidebar{\debug{exercises/{disc-prob-28/question.tex}}}

  \solutionlink{sol:prob-00}
  \qed
\end{ex} 
\begin{python0}
from solutions import *
add(label="ex:prob-00",
    srcfilename='exercises/discrete-probability/prob-00/answer.tex') 
\end{python0}

%-*-latex-*-

\begin{ex} 
  \label{ex:prob-00}
  \tinysidebar{\debug{exercises/{disc-prob-28/question.tex}}}

  \solutionlink{sol:prob-00}
  \qed
\end{ex} 
\begin{python0}
from solutions import *
add(label="ex:prob-00",
    srcfilename='exercises/discrete-probability/prob-00/answer.tex') 
\end{python0}

%-*-latex-*-

\begin{ex} 
  \label{ex:prob-00}
  \tinysidebar{\debug{exercises/{disc-prob-28/question.tex}}}

  \solutionlink{sol:prob-00}
  \qed
\end{ex} 
\begin{python0}
from solutions import *
add(label="ex:prob-00",
    srcfilename='exercises/discrete-probability/prob-00/answer.tex') 
\end{python0}

%-*-latex-*-

\begin{ex} 
  \label{ex:prob-00}
  \tinysidebar{\debug{exercises/{disc-prob-28/question.tex}}}

  \solutionlink{sol:prob-00}
  \qed
\end{ex} 
\begin{python0}
from solutions import *
add(label="ex:prob-00",
    srcfilename='exercises/discrete-probability/prob-00/answer.tex') 
\end{python0}

%-*-latex-*-

\begin{ex} 
  \label{ex:prob-00}
  \tinysidebar{\debug{exercises/{disc-prob-28/question.tex}}}

  \solutionlink{sol:prob-00}
  \qed
\end{ex} 
\begin{python0}
from solutions import *
add(label="ex:prob-00",
    srcfilename='exercises/discrete-probability/prob-00/answer.tex') 
\end{python0}

%-*-latex-*-

\begin{ex} 
  \label{ex:prob-00}
  \tinysidebar{\debug{exercises/{disc-prob-28/question.tex}}}

  \solutionlink{sol:prob-00}
  \qed
\end{ex} 
\begin{python0}
from solutions import *
add(label="ex:prob-00",
    srcfilename='exercises/discrete-probability/prob-00/answer.tex') 
\end{python0}

%-*-latex-*-

\begin{ex} 
  \label{ex:prob-00}
  \tinysidebar{\debug{exercises/{disc-prob-28/question.tex}}}

  \solutionlink{sol:prob-00}
  \qed
\end{ex} 
\begin{python0}
from solutions import *
add(label="ex:prob-00",
    srcfilename='exercises/discrete-probability/prob-00/answer.tex') 
\end{python0}

