%-*-latex-*-
\section{Formal definition of big-$O$}

[USELESS SECTION; SEE SECTION DEFINITION OF BIG-O]

Here's the formal definition of big-$O$:

\begin{defn}
Let $f(n)$ and $g(n)$ be two functions of $n$.
We write
\[ 
f = O(g) \index{$0$}\tinysidebar{$O$}
\]
and say $f(n)$ is \defone{big-$O$} of $g(n)$ if
\[
\text{there exist $C$ and $N$ such that $|f(n)| \leq C|g(n)|$ 
for $n \geq N$}
\]
\end{defn}

So far I've been using graphs to give you an understanding of the 
big-O definition.
However the clause
\[
\text{\lq\lq ... for $n \geq N$''}
\]
requires us to say something about arbitrarily large values for $n$
and of course we can't draw infinitely large graphs.

Now I'm going to use math to show $f(n) = O(g(n))$.
Occasionally I'll be using graphs as a tool to help.
But hard facts must be proven by math.

Let's do some examples.

[EXAMPLES]

