%-*-latex-*-
\section{Recursion with Ceilings}

In the previous I looked at a recursion with  ceiling:
\[
T(n) =
\begin{cases}
A                        & \text{if $n = 0, 1$} \\
2T(\floor{n/2}) + Bn + C & \text{otherwise}
\end{cases} 
\]
What about ceilings?
Let's look at this recursion:
\[
U(n) = 
\begin{cases}
A                          & \text{if $n = 0, 1$} \\
2U(\ceiling{n/2}) + Bn + C & \text{otherwise}
\end{cases}
\]
Let's make a plot.
The blue graph is the graph of $y = U(n)$
while the green is $y = g(n) = n + n \log_2(n) + \log_2(n)$.

\begin{python}
from plot import *
N = 33
STEP = 1
table = {}

def ceiling(x):
    f = x - int(x)
    if f == 0: return int(x)
    else: return int(x) + 1

def t(n):
    n = int(n)
    if n <= 1: return 1
    else:
        if not n in table:
            table[n] = 2 * t(ceiling(n/2.0)) + n + 1
        return table[n]
'''
def s(x):
    x = float(x)
    if x <= 1: return 1
    else:
        return 2 * s(x/2.0) + x + 1
'''

p = Plot(domain="0:%s" % N)

points = [(n, t(n)) for n in range(0, N + 1, STEP)]
p.add(points, style='step', color='blue', python=1)

#points = [(n/50.0, s(n/50.0)) for n in range(0, N*50 + 1, STEP)]
#p.add(points, style='step', color='red', python=1)


p.add("x + x*log(x, 2) + (x - 1)", color='green', python=1, num_points=100)
print(p)

\end{python}

Hmmm .... it looks like in this case $U(n)$ actually beats
$g(n)$.
Of course it could be the case the a constant multiple of $g(n)$
might beat $U(n)$ for all sufficiently large $n$.

Let's use recurrence relation to get rid of the recurrence:
\begin{align*}
U(n) 
&= 2U(\ceiling{n/2}) + Bn + C \\
&= 2(
     2U(
        \ceiling{ 
         \ceiling{n/2} / 2 
       }
      ) 
     + B(\ceiling{n/2}) + C
    ) 
   + Bn + C \\
&= 2^2 U(
         \ceiling{ 
          \ceiling{n
          /2} 
         /2}
       ) 
     + (n + 2\ceiling{n/2})B + (2 + 1)C
    \\
&= 2^2
   (2U(
        \ceiling{
         \ceiling{ 
          \ceiling{n
          /2} 
         /2}
        /2}
      )
    + B(        
        \ceiling{ 
         \ceiling{n
         /2} 
        /2}
      )
    + C
   ) 
     + (n + 2\ceiling{n/2})B + (2 + 1)C
    \\
&= 2^3
   U(
        \ceiling{
         \ceiling{ 
          \ceiling{n
          /2} 
         /2}
        /2}
      )
    +         
       (n 
        + 2\ceiling{n/2}
        +
        2^2
        \ceiling{ 
         \ceiling{n
         /2} 
        /2}
       )B 
     + (2^2 + 2 + 1)C
\end{align*}
To make things neat: write
\[
f(n) = \ceiling{n/2}
\]
and $f^{(0)}(n) = n$,
$f^{(1)}(n) = f(n)$ and $f^{(i+1)}(n) = f(f^{(i)}(n))$. 
The above gives us
\begin{align*}
U(n) 
&= 2^e U(1) \\
&\hspace{0.5cm} + (2^0f^{(0)}(n) + 2^1f^{(1)}(n) + \cdots + 2^{e-1}f^{(e-1)}(n))B \\
&\hspace{0.5cm} + (2^{e-1} + \cdots + 2 + 1)C \\
&= 2^e U(1) \\
&\hspace{0.5cm} + (2^0f^{(0)}(n) + 2^1f^{(1)}(n) + \cdots + 2^{e-1}f^{(e-1)}(n))B \\
&\hspace{0.5cm} + (2^e - 1)C
\end{align*}
Of course $e$ is the number of times I have to use the recurrence relation
to get to $T(1)$ and it definitely depends on $n$.
In other words $d$ is the smallest positive integer such that
\[
f^{(e)}(n) = 1
\]

So the critical mission now is to answer the following question:
\begin{itemize}
\li Given a positive integer $n$, 
what is the smallest positive $e$ such that $f^{(e)}(n) = 1$?
\li Once we have $e$, what is the value of 
\[
2^0f^{(0)}(n) + \cdots + 2^{e-1}f^{(e-1)}(n)  
\]
\end{itemize}

Note that if we can't find exact values, then some upper bound might be good
enough.

There's one other thing that is really helpful in our context:
If $x$ is a real number and $k,\ell$ are positive integers, then
\[
\ceiling{(x/k)/\ell}
=
\ceiling{x / (k \ell)}
\]
This means that in
\begin{align*}
U(n) 
&= 2^e U(1) \\
&\hspace{0.5cm} + (f^{(0)}(n) + f^{(1)}(n) + \cdots + f^{(e-1)}(n))B \\
&\hspace{0.5cm} + (2^e - 1)C
\end{align*}
The expressions $f^{(i)}(n)$ instead of being this monter
\[
\ceiling{
\cdots
\ceiling{
\ceiling{
\ceiling{
n/2
}/2
}/2
}/2
\cdots
/2}
\]
is really just this:
\[
\ceiling{n/2^i}
\]
Woohoo!
So 
\[
f^{(d)}(n) = 1
\]
is the same as
\[
\ceiling{ \frac{n}{2^e} } = 1
\]
Now if you write $n$ as a binary number
\[
n = c_d2^d + \cdots + c_02^0 = (c_d \cdots c_0)
\]
where $d + 1$ is the number of binary bits, then
\[
\ceiling{ \frac{n}{2^k} } = 1
\]
if $k > d$.
To use a more standard function instead of \lq\lq binary length of'',
I look at the $\log_2$ function.
Recall that if
\[
n = c_d 2^d + \cdots c_0 2^0
\]
then
\[
d = \floor{ \log_2(n) }
\]
Therefore
\[
e = d + 1 = \floor{ \log_2(n) } + 1
\]


Now I'm going to put the closed form for $T(n)$ and 
$U(n)$ side by side:
\begin{align*}
T(n) 
&= 2^d T(1) \\
&\hspace{0.5cm} + \left( 
                  2^0 \floor{\frac{n}{2^0}} 
                  +  2^1 \floor{\frac{n}{2^0}} 
                  + \cdots 
                  + 2^{d-1} \floor{\frac{n}{2^{d-1}}}
                  \right) B \\
&\hspace{0.5cm} + (2^{d} - 1)C 
\\
U(n)
&= 2^e U(1) \\
&\hspace{0.5cm} + \left(
                   2^0 \ceiling{\frac{n}{2^0}} 
                   + 2^1 \ceiling{\frac{n}{2^1}} 
                   + \cdots 
                   + 2^{e-1} \ceiling{\frac{n}{2^{e-1}}} 
                  \right) B \\
&\hspace{0.5cm} + (2^e - 1)C
\end{align*}

We want to compute $T(n)$ and $U(n)$.

Note that
\[
\floor{x} \leq x \leq \ceiling{x}
\]
Since I'm trying to trap $\ceiling{x}$ with $\floor{x}$, 
I also need this:
\[
\floor{x} \leq x \leq \ceiling{x} \leq \floor{x} + 1
\]
i.e.,
\[
\floor{x} \leq \ceiling{x} \leq \floor{x} + 1
\]

Hence
\[
2^i \floor{\frac{n}{2^i}}
\leq
2^i \ceiling{\frac{n}{2^i}}
\leq
2^i  \left( \floor{\frac{n}{2^i}} + 1 \right)
= 
2^i  \floor{\frac{n}{2^i}} + 2^i
\]
Therefore
\[
\sum_{i=0}^{d - 1} 2^i \floor{\frac{n}{2^i}} + 2^d \floor{\frac{n}{2^d}}
\leq
\sum_{i=0}^{d} 2^i \ceiling{\frac{n}{2^i}}
\leq
\sum_{i=0}^{d - 1} 2^i  \floor{\frac{n}{2^i}} + 2^d \floor{\frac{n}{2^d}}
+ \sum_{i=0}^{d - 1} 2^i
\]
i.e.,
\[
\sum_{i=0}^{d - 1} 2^i \floor{\frac{n}{2^i}} + 2^d 
\leq
\sum_{i=0}^{e - 1} 2^i \ceiling{\frac{n}{2^i}}
\leq
\sum_{i=0}^{d - 1} 2^i  \floor{\frac{n}{2^i}} + 2^d 
+ (2^d - 1)
\]
Therefore
\[
\sum_{i=0}^{d - 1} 2^i \floor{\frac{n}{2^i}} + 2^d 
\leq
\sum_{i=0}^{e - 1} 2^i \ceiling{\frac{n}{2^i}}
\leq
\sum_{i=0}^{d - 1} 2^i  \floor{\frac{n}{2^i}} + 2^{d+1} - 1
\]
We know that
\[
d = \floor{\log_2(n)}
\]
and hence
\[
\log_2(n) - 1 \leq d = \floor{\log_2(n)} \leq \log_2(n)
\]
Therefore
\[
\sum_{i=0}^{d - 1} 2^i \floor{\frac{n}{2^i}} + \frac{1}{2} n
\leq
\sum_{i=0}^{e - 1} 2^i \ceiling{\frac{n}{2^i}}
\leq
\sum_{i=0}^{d - 1} 2^i  \floor{\frac{n}{2^i}} + 2 n - 1
\]


