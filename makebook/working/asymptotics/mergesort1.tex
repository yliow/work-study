%-*-latex-*-
\section{Mergesort ... 1st Version}

\input{mergesort-pic.tex}

Before I talk about the actual $\mergesort$, I want to 
talk about the version that is easier to understand
but uses a little more memory.

Before I talk about $\mergesort$, I'll talk about $\merge$.
As I mentioned earlier, the $\mergesort$ uses $\merge$.

\begin{ex}
First I want you to write a merge function that 
takes two sorted arrays and put the values, sorted, into another array.
Here's the C++ prototype:
\begin{Verbatim}[frame=single,fontsize=\footnotesize]
void merge(int a[], int size_a;
           int b[], int size_b;
           int c[], int & size_c);
\end{Verbatim}
\qed
\end{ex}

Assuming that the lengths of \verb!a! is $n$ or $n - 1$
and the length of \verb!b! is $n$
and the length of \verb!c! is $2n - 1$ or $2n$, then
the runtime of merge is
\[
T_{\merge}(n) = O(n)
\] 
after all, you need to read all elements of \verb!a!, 
read all elements of \verb!b!, copy
$2n - 1$ or $2n$ values to \verb!c!.

\begin{ex}
Verify the runtime against your implementation of the merge function.
\qed
\end{ex}
 
\begin{ex}
Next I want you to write a split function that 
take array \verb!x! and put 
the first half of the values in \verb!x! into array \verb!y!
and 
the second half the values in array \verb!z!.
If the length of \verb!x! is odd, then 
the length of \verb!y! is one smaller than the length of 
\verb!z!.
Here's the C++ prototype:
\begin{Verbatim}[frame=single,fontsize=\footnotesize]
void split(int x[], int size_x;
           int y[], int & size_y;
           int z[], int & size_z);
\end{Verbatim}
\qed
\end{ex}

The runtime of the split function is
\[
T_{\SPLIT}(n) = O(n)
\]
That's easy to see since you have to read all values in \verb!x!
which is $O(n)$, and copy the values to either \verb!y! 
or \verb!z!.

\begin{ex}
Now complete the version of mergesort that uses new arrays
for each function call.
\begin{Verbatim}[frame=single,fontsize=\footnotesize]
void mergesort(int x[], int size_x)
{
    if (x_size <= 1)
    {
        return;
    }
    else
    {
        int * y = new int[size_x / 2];
        int * z = new int[size_x - size_x / 2];
        split(x, size_x, y, size_y, z, size_z);
        mergesort(y, size_y);
        mergesort(z, size_z);
        merge(y, size_y,
              z, size_y,
              x, size_x);
        return;
    }
}
\end{Verbatim}
Test your code thoroughly.
(Remember to deallocate memory used.)
\qed
\end{ex}

I will assume the time taken to create the array on the heap
is constant.
In that case:
\begin{align*}
T_{\mergesort}(n) 
&= T_{\SPLIT}(n)  \\
&\hspace{0.5cm} + T_{\mergesort}(\floor{n/2})
  + T_{\mergesort}(n - \floor{n/2}) \\
&\hspace{0.5cm} + T_{\merge}(n) + A
\end{align*}
Since $n - \floor{n/2} = \ceiling{n/2}$,
the runtime is 
\begin{align*}
T_{\mergesort}(n) 
&= T_{\SPLIT}(n)  \\
&\hspace{0.5cm} + T_{\mergesort}(\floor{n/2})
  + T_{\mergesort}(\ceiling{n/2}) \\
&\hspace{0.5cm} + T_{\merge}(n) + A
\end{align*}
Hence the runtime $T_{\mergesort}(n)$ is of the form
\begin{align*}
T(n) \\
&=
 \begin{cases}
  A                             & \text{if $n = 0, 1$} \\
  T(\floor{n/2}) 
  + T(\ceiling{n/2}) + Bn + C & \text{otherwise}
 \end{cases}
\end{align*}

Ignoring floors and ceilings, 
the runtime function looks like this:
\[
T(n) = 2T(n/2) + Bn + C
\]
and this is the case if $n$ is a power of $2$ (right?)
Let me try to bring this to a closed form.
If I make the substitution $n = 2^m$, I get
\begin{align*}
T(2^m) 
&= 2T(2^{m-1}) + 2^mB + C \\
&= 2(2T(2^{m-2}) + 2^{m-1}B + C)+ 2^mB + C \\
&= 2^2 T(2^{m-2}) + 2 \cdot 2^mB + (1+2)C \\
&= 2^2 (2T(2^{m-3}) + 2^{m-2}B + C) + 2 \cdot 2^mB + (1+2) C\\
&= 2^3T(2^{m-3}) + 3 \cdot 2^m B + (1+2+2^2)C
\end{align*}
At this point you see that you will get
\begin{align*}
T(2^m) 
&= 2^mT(2^0) + m \cdot 2^m B + (1+2+\cdots+2^{m-1}) C \\
&= 2^mT(2^0) + m \cdot 2^m B + (2^m - 1) C
\end{align*}
Replacing $m$ by $n$ by $n = 2^m$, you get
\begin{align*}
T(n) 
&= nT(1) + (\log_2 n) \cdot n B + (n - 1) C\\
&= nT(1) + B(n\log_2 n) + (n - 1)C \\
&= O(n \log_2 n)
\end{align*}
assuming $B \neq 0$ of course.
AHA! This is better than bubblesort which is $O(n^2)$.

Note that the above is not a proof!!!
All it says is that \textit{if} (a big if) $n$ is a power of $2$, 
then
\[
T(n) = nT(1) + (\log_2 n) \cdot n B + (n - 1) C
\]
There's no reason to believe that above is still true if $n$ is 
not a power of $2$.

Later I'll show you that the computations can be easily
repaired to take into account the case where $n$ is not a power of 2
and the end result is that the big-O is actually as stated above.

\begin{ex}
If the runtime of an algorithm is
\[
T(n) = 3T(n/3) + Bn + C
\]
compute it's big-O. (Use one of the standard functions.)
\qed
\end{ex}

\begin{ex}
If the runtime of an algorithm is
\[
T(n) = 5T(n/3) + Bn + C
\]
compute it's big-O. (Use one of the standard functions.)
\qed
\end{ex}

\begin{ex}
If the runtime of an algorithm is
\[
T(n) = 2T(n/2) + C
\]
compute it's big-O. (Use one of the standard functions.)
\qed
\end{ex}

\begin{ex}
Try this:
\[
T(n) = 2T(n/2) + Bn^2 + Cn + D
\]
\qed
\end{ex}

\begin{ex}
Try this:
\[
T(n) = 2T(n/2) + Bn \ln n + C
\]
\qed
\end{ex}


As I mentioned earlier, I want to use memory carefully, i.e., 
I do not want to create new arrays all the time.

Note that the above already has some savings.
For instance the array \verb!y! and \verb!z! is merged into
the original \verb!x! (not a new array).

Next, I'm going to move on to the usual mergesort which
contains the same idea as above except that it uses memory
more efficiently ...


\begin{ex}
Modify the above mergesort (version 1)
so that instead of splitting up an array into two, split it into \textit{three}.
Implement the algorithm. 
Compute the runtime.
Compare this version against the original one above.
\qed
\end{ex}

