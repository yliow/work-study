%-*-latex-*-
\section{Quicksort: First Correct Partition Strategy (LTR)}


\begin{comment}
I want to now design a partition strategy that works {\it inplace}, i.e.,
directly on the array without creating a new array.

Basically I break up the array into three parts:
the pivot \verb!p!, 
the \verb!left! chunk, the chunk of \verb!TODO!, and the \verb!right!
chunk:
\begin{python}
s = r"""
from quicksort_init import *
p = Plot()
s = chunkedarray(cellwidth=cellwidth, 
                   cellheight=cellheight,
                   arr=[['p'], 4*[''], 7*[''], 5*['']],
                   pipeheight=pipeheight,
                   chunklabels=[('pivot',-1),'left','TODO', 'right'],
                   )
p.add(s)
print(p)
"""
from latextool_basic import makeandincludegraphics
print(makeandincludegraphics(python=s))
\end{python}

Of course those \verb!|! in bold are just to help visualize the 
division of the array into parts.
Later when we design a proper algorithm, we will need 
variables to indicate the positions of these dividers.
Let's not worry about these variables for now.

The \verb!TODO! list initially contains all the values in the array
other than the pivot.
I examine the values in \verb!TODO! (scanning left-to-right),
putting the values into the \verb!left! or \verb!right! pile.
When, the \verb!TODO! list is empty, I would have the 
\verb!left! and \verb!right!.
Of course, initially
the \verb!left! and \verb!right! piles are empty.

Going back to our example,
given the array
\begin{python}
s = r"""
from quicksort_init import *
p = Plot()
s = chunkedarray(cellwidth=cellwidth, 
                   cellheight=cellheight,
                   arr=[[2,1,6,5,3,0,4]],
                   pipeheight=pipeheight,
)
p.add(s)
print(p)
"""
from latextool_basic import makeandincludegraphics
print(makeandincludegraphics(python=s))
\end{python}

I put up dividers to indicate the pivot and the
\verb!left!, \verb!TODO!, and \verb!right! piles:
\begin{python}
s = r"""
from quicksort_init import *
p = Plot()
s = chunkedarray(cellwidth=cellwidth, 
                   cellheight=cellheight,
                   arr=[[2],[],[1,6,5,3,0,4],[]],
                   pipeheight=pipeheight,
                   chunklabels = [('pivot',-1),'left','TODO','right'],
)
p.add(s)
print(p)
"""
from latextool_basic import makeandincludegraphics
print(makeandincludegraphics(python=s))
\end{python}
The first piece of work from the \verb!TODO! pile is \verb!1!.
I put that into the \verb!left! pile by moving the second divider:
\begin{python}
s = r"""
from quicksort_init import *
p = Plot()
s = chunkedarray(cellwidth=cellwidth,
                   cellheight=cellheight,
                   arr=[[2],[1],[6,5,3,0,4],[]],
                   pipeheight=pipeheight,
                   )
p.add(s)
print(p)
"""
from latextool_basic import makeandincludegraphics
print(makeandincludegraphics(python=s))
\end{python}
To take the mystery out of the \lq\lq move the divider''
business, note that later this will basically be achieved by incrementing
a variable that tells us where that divider is.

The next piece of work from the \verb!TODO! pile is \verb!6!.
The problem is that the \verb!right! pile is far away on the right.
How in the world do I get into that slot?
I can shift the values \verb!5,3,0,4! to the left by one
(you know, algorithmically, that can be done, right?)
But it's clear that in the worst case this is going to $O(n)$
where $n$ is the size of this array.
There's actually a faster (and smarter) way of doing this.
We just swap \verb!6! with the value just next to the 
\verb!right! pile:
\begin{python}
s = r"""
from quicksort_init import *
p = Plot()
s = chunkedarray(cellwidth=cellwidth,
                   cellheight=cellheight,
                   arr=[[2],[1],[6,5,3,0,4],[]],
                   pipeheight=pipeheight,
                   swaps=[(2,6)]
                   )
p.add(s)
print(p)
"""
from latextool_basic import makeandincludegraphics
print(makeandincludegraphics(python=s))
\end{python}
to get this:
\begin{python}
s = r"""
from quicksort_init import *
p = Plot()
s = chunkedarray(cellwidth=cellwidth,
                   cellheight=cellheight,
                   arr=[[2],[1],[4,5,3,0,6],[]],
                   pipeheight=pipeheight,
                   )
p.add(s)
print(p)
"""
from latextool_basic import makeandincludegraphics
print(makeandincludegraphics(python=s))
\end{python}
and then move the divider to get \verb!6! into the \verb!right! pile:
\begin{python}
s = r"""
from quicksort_init import *
p = Plot()
s = chunkedarray(cellwidth=cellwidth,
                   cellheight=cellheight,
                   arr=[[2],[1],[4,5,3,0],[6]],
                   pipeheight=pipeheight,
                   chunklabels=[('pivot',-1), 'left', 'TODO', 'right'])
p.add(s)
print(p)
"""
from latextool_basic import makeandincludegraphics
print(makeandincludegraphics(python=s))
\end{python}
(Again, moving the divider is just changing a variable that
tells you the position of the divider.
In this case, we are decrementing a variable that indicates
the position of this divider.)

The next value to process is \verb!4! which should go into the 
\verb!right! pile.
So I do the same swap-and-move-divider deal:
\begin{python}
s = r"""
from quicksort_init import *
p = Plot()
s = chunkedarray(cellwidth=cellwidth,
                   cellheight=cellheight,
                   arr=[[2],[1],[4,5,3,0],[6]],
                   pipeheight=pipeheight,
                   swaps=[(2,5)])
p.add(s)
print(p)
"""
from latextool_basic import makeandincludegraphics
print(makeandincludegraphics(python=s))
\end{python}
\begin{python}
s = r"""
from quicksort_init import *
p = Plot()
s = chunkedarray(cellwidth=cellwidth,
                   cellheight=cellheight,
                   arr=[[2],[1],[0,5,3,4],[6]],
                   pipeheight=pipeheight)
p.add(s)
print(p)
"""
from latextool_basic import makeandincludegraphics
print(makeandincludegraphics(python=s))
\end{python}
\begin{python}
s = r"""
from quicksort_init import *
p = Plot()
s = chunkedarray(cellwidth=cellwidth,
                   cellheight=cellheight,
                   arr=[[2],[1],[0,5,3],[4,6]],
                   pipeheight=pipeheight,
                   chunklabels=[('pivot',-1), 'left', 'TODO', 'right'])
p.add(s)
print(p)
"""
from latextool_basic import makeandincludegraphics
print(makeandincludegraphics(python=s))
\end{python}

The next value \verb!0! is $\leq$ \verb!2! the pivot,
so that goes into the \verb!left! pile:
\begin{python}
s = r"""
from quicksort_init import *
p = Plot()
s = chunkedarray(cellwidth=cellwidth,
                   cellheight=cellheight,
                   arr=[[2],[1,0],[5,3],[4,6]],
                   pipeheight=pipeheight,
                   chunklabels=[('pivot',-1), 'left', 'TODO', 'right'])
p.add(s)
print(p)
"""
from latextool_basic import makeandincludegraphics
print(makeandincludegraphics(python=s))
\end{python}

The next value \verb!5! goes into the \verb!right! pile:
\begin{python}
s = r"""
from quicksort_init import *
p = Plot()
s = chunkedarray(cellwidth=cellwidth,
                   cellheight=cellheight,
                   arr=[[2],[1,0],[5,3],[4,6]],
                   pipeheight=pipeheight,
                   swaps=[(3,4)])
p.add(s)
print(p)
"""
from latextool_basic import makeandincludegraphics
print(makeandincludegraphics(python=s))
\end{python}
\begin{python}
s = r"""
from quicksort_init import *
p = Plot()
s = chunkedarray(cellwidth=cellwidth,
                   cellheight=cellheight,
                   arr=[[2],[1,0],[3,5],[4,6]],
                   pipeheight=pipeheight)
p.add(s)
print(p)
"""
from latextool_basic import makeandincludegraphics
print(makeandincludegraphics(python=s))
\end{python}
\begin{python}
s = r"""
from quicksort_init import *
p = Plot()
s = chunkedarray(cellwidth=cellwidth,
                   cellheight=cellheight,
                   arr=[[2],[1,0],[3],[5,4,6]],
                   pipeheight=pipeheight,
                   chunklabels=[('pivot',-1), 'left', 'TODO', 'right'])
p.add(s)
print(p)
"""
from latextool_basic import makeandincludegraphics
print(makeandincludegraphics(python=s))
\end{python}

The last value in the \verb!TODO! list is \verb!3! which 
should go into the \verb!right! pile.
Note that in this case I don't have to swap \verb!3! to get it to 
be next to the \verb!right! pile  ... it's already next to the
\verb!right! pile! Duh.
So I just move the divider:
\begin{python}
s = r"""
from quicksort_init import *
p = Plot()
s = chunkedarray(cellwidth=cellwidth,
                   cellheight=cellheight,
                   arr=[[2],[1,0],[],[3,5,4,6]],
                   pipeheight=pipeheight,
                   chunklabels=[('pivot',-1), 'left', ('TODO',1.2), 'right'])
p.add(s)
print(p)
"""
from latextool_basic import makeandincludegraphics
print(makeandincludegraphics(python=s))
\end{python}

And now I'm done since the \verb!TODO! pile is empty.
By the way, I hope you notice that the \verb!TODO! list
shrinks by 1 for each step in the above strategy.


There's one last thing to do.
And that is to put the pivot in the middle.
How would I do that?
I can shift the \verb!left! pile to the left by one, 
but that (again) is a bad idea since the time taken
could potentially $O(n)$.
Here's the smart thing to do: I just swap the pivot \verb!2!
with the rightmost value in the \verb!left! pile:
\begin{python}
s = r"""
from quicksort_init import *
p = Plot()
s = chunkedarray(cellwidth=cellwidth,
                   cellheight=cellheight,
                   arr=[[2],[1,0],[],[3,5,4,6]],
                   pipeheight=pipeheight,
                   swaps=[(0,2)])
p.add(s)
print(p)
"""
from latextool_basic import makeandincludegraphics
print(makeandincludegraphics(python=s))
\end{python}
This is the last picture with dividers showing the \verb!left! pile,
the pivot, and the \verb!right! pile:
\begin{python}
s = r"""
from quicksort_init import *
p = Plot()
s = chunkedarray(cellwidth=cellwidth,
                   cellheight=cellheight,
                   arr=[[0,1],[2],[3,5,4,6]],
                   pipeheight=pipeheight,
                   chunklabels=['left', ('pivot',-1), 'right'])
p.add(s)
print(p)
"""
from latextool_basic import makeandincludegraphics
print(makeandincludegraphics(python=s))
\end{python}

TADA!
It looks like a reasonable design experiment.
Once we have variables to help mark the dividers,
it seems reasonable that the above strategy can be translated
into a proper algorithm.
Not only that ... and this is important ...
the above strategy works directly on the array without
creating another array.

If you're sharp, you would 
have noticed a one slight issue here:
Notice that I have to swap the pivot with the rightmost value of the 
\verb!left! pile.
Well ... what if the \verb!left! pile is empty?
I'll let you think about that.
The solution to this problem will appear below.


\begin{ex}
Perform the above partition process on the 
following array \verb!x! where 
the first value is the pivot:
\begin{python}
s = r"""
from quicksort_init import *
p = Plot()
s = chunkedarray(cellwidth=cellwidth,
                   cellheight=cellheight,
                   arr=[[3],[],[1,2,3,4,5],[],],
                   pipeheight=pipeheight,
                   )
p.add(s)
print(p)
"""
from latextool_basic import makeandincludegraphics
print(makeandincludegraphics(python=s))
\end{python}
No code here! Just perform the above process on the array,
showing every step.
Each step should include a picture of the above 
of the array after the swap. 
Indicate the dividers clearly in your diagram.
\qed
\end{ex}

\begin{ex}
Do the same for this array:
\begin{python}
s = r"""
from quicksort_init import *
p = Plot()
s = chunkedarray(cellwidth=cellwidth,
                   cellheight=cellheight,
                   arr=[[3],[],[5,4,3,2,1],[],],
                   pipeheight=pipeheight,
                   )
p.add(s)
print(p)
"""
from latextool_basic import makeandincludegraphics
print(makeandincludegraphics(python=s))
\end{python}
\qed
\end{ex}

\begin{ex}
The following array is the result
of performing the above partition strategy
on an array:
\begin{python}
s = r"""
from quicksort_init import *
p = Plot()
s = chunkedarray(cellwidth=cellwidth,
                   cellheight=cellheight,
                   arr=[[2,1,2],[3],[4,5,6],],
                   pipeheight=pipeheight,
                   )
p.add(s)
print(p)
"""
from latextool_basic import makeandincludegraphics
print(makeandincludegraphics(python=s))
\end{python}
Complete the picture of the original \verb!x!:
\begin{python}
s = r"""
from quicksort_init import *
p = Plot()
s = chunkedarray(cellwidth=cellwidth,
                   cellheight=cellheight,
                   arr=[[3],[],['','','','','',''],[]],
                   pipeheight=pipeheight,
                   )
p.add(s)
print(p)
"""
from latextool_basic import makeandincludegraphics
print(makeandincludegraphics(python=s))
\end{python}
\qed
\end{ex}



Let's move on ...

To keep track of the \verb!left!, \verb!TODO!, and \verb!right! piles,
I need some variables to mark the index positions:
\begin{python}
s = r"""
from quicksort_init import *
numleft = 4
numtodo = 6
numright = 4
p = Plot()
s = chunkedarray(cellwidth=cellwidth, 
                   cellheight=cellheight,
                   arr=[['p'], numleft*[''], numtodo*[''], numright*['']],
                   pipeheight=pipeheight,
                   chunklabels=['','left','TODO', 'right'],
                   celllabels = [('lhead', 1, -1),
                                 ('ltail', 4, -1),
                                 ('thead', 5, -2),
                                 ('ttail', 10, -2),
                                 ('rhead', 11, -1),
                                 ('rtail', 14, -1),
                                ]
)
p.add(s)
print(p)
"""
from latextool_basic import makeandincludegraphics
print(makeandincludegraphics(python=s))
\end{python}




Well ... that's kind of an overkill.
For instance \verb!thead! is just one more than \verb!ltail!.
So I just need one of them and not both.
Also, \verb!rhead! is just one more than \verb!ttail!.
Let me just simplify it to this:




\begin{python}
s = r"""
from quicksort_init import *
numleft = 4
numtodo = 6
numright = 4
p = Plot()
s = chunkedarray(cellwidth=cellwidth, 
                   cellheight=cellheight,
                   arr=[['p'], numleft*[''], numtodo*[''], numright*['']],
                   pipeheight=pipeheight,
                   chunklabels=['','left','TODO', 'right'],
                   celllabels = [('thead', 5, -1),
                                 ('ttail', 10, -1)]
                   )
p.add(s)
print(p)
"""
from latextool_basic import makeandincludegraphics
print(makeandincludegraphics(python=s))
\end{python}




Let me do an example but now with 
variables \verb!thead! and \verb!ttail!.
In this example \verb!2! is the pivot.
We start with this:
\begin{python}
s = r"""
from quicksort_init import *
p = Plot()
s = chunkedarray(cellwidth=cellwidth,
                   cellheight=cellheight,
                   arr=[[2],[],[1,6,5,3,0,4],[]],
                   pipeheight=pipeheight,
                   celllabels = [('thead', 1, 1),
                                 ('ttail', 6, -1)],
                   )
p.add(s)
print(p)
"""
from latextool_basic import makeandincludegraphics
print(makeandincludegraphics(python=s))
\end{python}



We take \verb!1! out of the \verb!TODO! list and put
it into the left pile since \verb!1! $\leq$ \verb!2!.
I do this by incrementing \verb!thead!.
This is the picture after putting \verb!1! into the \verb!left! pile:
\begin{python}
s = r"""
from quicksort_init import *
p = Plot()
s = chunkedarray(cellwidth=cellwidth,
                   cellheight=cellheight,
                   arr=[[2],[1],[6,5,3,0,4],[]],
                   pipeheight=pipeheight,
                   celllabels = [('thead', 2, 1),
                                 ('ttail', 6, -1)],
                   )
p.add(s)
print(p)
"""
from latextool_basic import makeandincludegraphics
print(makeandincludegraphics(python=s))
\end{python}





Now for \verb!6!.
First we swap \verb!6! and \verb!4! to get \verb!6! 
{\it next} to the \verb!right! pile:
\begin{python}
s = r"""
from quicksort_init import *
p = Plot()
s = chunkedarray(cellwidth=cellwidth,
                   cellheight=cellheight,
                   arr=[[2],[1],[4,5,3,0,6],[]],
                   pipeheight=pipeheight,
                   celllabels = [('thead', 2, 1),
                                 ('ttail', 6, -1)],
                   )
p.add(s)
print(p)
"""
from latextool_basic import makeandincludegraphics
print(makeandincludegraphics(python=s))
\end{python}
and then decrement \verb!ttail!
to get \verb!6! {\it into} the \verb!right! pile:
\begin{python}
s = r"""
from quicksort_init import *
p = Plot()
s = chunkedarray(cellwidth=cellwidth,
                   cellheight=cellheight,
                   arr=[[2],[1],[4,5,3,0],[6]],
                   pipeheight=pipeheight,
                   celllabels = [('thead', 2, 1),
                                 ('ttail', 5, -1)],
                   )
p.add(s)
print(p)
"""
from latextool_basic import makeandincludegraphics
print(makeandincludegraphics(python=s))
\end{python}
Notice that \verb!thead! should not be changed. (Right? Think about it.)

Using \verb!thead!,
the next value is \verb!4! which should go into the \verb!right! pile.
I swap \verb!4! and \verb!0!
and then decrement \verb!ttail!:
\begin{python}
s = r"""
from quicksort_init import *
p = Plot()
s = chunkedarray(cellwidth=cellwidth,
                   cellheight=cellheight,
                   arr=[[2],[1],[0,5,3],[4,6]],
                   pipeheight=pipeheight,
                   celllabels = [('thead', 2, 1),
                                 ('ttail', 4, -1)],
                   )
p.add(s)
print(p)
"""
from latextool_basic import makeandincludegraphics
print(makeandincludegraphics(python=s))
\end{python}





The next value in the \verb!TODO! list is \verb!0! 
which should go into the \verb!left! pile since 
\verb!0! $\leq$ \verb|2|.
Again, I can do that by incrementing \verb!thead!:
\begin{python}
s = r"""
from quicksort_init import *
p = Plot()
s = chunkedarray(cellwidth=cellwidth,
                   cellheight=cellheight,
                   arr=[[2],[1,0],[5,3],[4,6]],
                   pipeheight=pipeheight,
                   celllabels = [('thead', 3, 1),
                                 ('ttail', 4, -1)],
                   )
p.add(s)
print(p)
"""
from latextool_basic import makeandincludegraphics
print(makeandincludegraphics(python=s))
\end{python}





The next value to process in the \verb!TODO! list is \verb!5!.
You should know what to do to get this picture:
\begin{python}
s = r"""
from quicksort_init import *
p = Plot()
s = chunkedarray(cellwidth=cellwidth,
                   cellheight=cellheight,
                   arr=[[2],[1,0],[3],[5,4,6]],
                   pipeheight=pipeheight,
                   celllabels = [('thead', 3, 1),
                                 ('ttail', 3, -1)],
                   )
p.add(s)
print(p)
"""
from latextool_basic import makeandincludegraphics
print(makeandincludegraphics(python=s))
\end{python}
Notice that at this point there's only one value left in the 
\verb!TODO! list.


After processing \verb!3!, I get this (what did I do?):
\begin{python}
s = r"""
from quicksort_init import *
p = Plot()
s = chunkedarray(cellwidth=cellwidth,
                   cellheight=cellheight,
                   arr=[[2],[1,0],[],[3,5,4,6]],
                   pipeheight=pipeheight,
                   celllabels = [('thead', 3, 1),
                                 ('ttail', 2, -1)],
                   )
p.add(s)
print(p)
"""
from latextool_basic import makeandincludegraphics
print(makeandincludegraphics(python=s))
\end{python}




Notice that \verb!ttail! has moved past \verb!thead!,
telling us that the there's nothing left in the \verb!TODO! list.




Quick recap ...
We started with this:
\begin{python}
s = r"""
from quicksort_init import *
p = Plot()
s = chunkedarray(cellwidth=cellwidth,
                   cellheight=cellheight,
                   arr=[[2],[],[1,6,5,3,0,4],[]],
                   pipeheight=pipeheight,
                   celllabels = [('thead', 1, 1),
                                 ('ttail', 6, -1)],
                   )
p.add(s)
print(p)
"""
from latextool_basic import makeandincludegraphics
print(makeandincludegraphics(python=s))
\end{python}
and ended with this:
\begin{python}
s = r"""
from quicksort_init import *
p = Plot()
s = chunkedarray(cellwidth=cellwidth,
                   cellheight=cellheight,
                   arr=[[2],[1,0],[],[3,5,4,6]],
                   pipeheight=pipeheight,
                   celllabels = [('thead', 3, 1),
                                 ('ttail', 2, -1)],
                   )
p.add(s)
print(p)
"""
from latextool_basic import makeandincludegraphics
print(makeandincludegraphics(python=s))
\end{python}




At this point, the \verb!TODO! list is empty, 
the \verb!left! pile has values $\leq$ \verb!2!, and the 
\verb!right! pile has value $>$ \verb!2!.




Remember that we want the pivot in the 
middle,
with \verb!left! chunk on the left of the pivot
and the \verb!right! chunk on the right of the pivot.
No big deal. 
As before, 
we just swap the pivot with the {\it rightmost} value in the \verb!left!
pile:
\begin{python}
s = r"""
from quicksort_init import *
p = Plot()
s = chunkedarray(cellwidth=cellwidth,
                   cellheight=cellheight,
                   arr=[[2],[1,0],[],[3,5,4,6]],
                   pipeheight=pipeheight,
                   swaps = [(0,2)],
                   )
p.add(s)
print(p)
"""
from latextool_basic import makeandincludegraphics
print(makeandincludegraphics(python=s))
\end{python}
to {\it finally} (phew!) get this with the pivot in the middle:
\begin{python}
s = r"""
from quicksort_init import *
p = Plot()
s = chunkedarray(cellwidth=cellwidth,
                   cellheight=cellheight,
                   arr=[[0,1],[2],[3,5,4,6]],
                   pipeheight=pipeheight,
                   chunklabels=['left', ('pivot',-1), 'right'],
                   )
p.add(s)
print(p)
"""
from latextool_basic import makeandincludegraphics
print(makeandincludegraphics(python=s))
\end{python}

It should be clear that with variables \verb!thead! and \verb!ttail!
we don't need the visual aid of the dividers.
I just invented them to help you see the partition process.

Note that if the \verb!left! pile is empty, then 
there's no one to swap the pivot with!!!
But hang on there ... that means that the pivot is in the correct place
since all other values are in the \verb!right! pile.
Correct?

Obviously the above works and can be translated into an algorithm.
But the crucial point (again) is this ...
the above partition strategy does not use any extra array.
It operates on the given array itself, i.e., it operates on
the array inplace.

Now what if the pivot is not the first element of the array?
For instance what if I use randomized quicksort?
Or say, just for kicks, I want to always choose the last element in the 
array as pivot.
In that case the pivot need not be the first value.
Again ... no sweat, don't call 911 ... 
just swap the pivot with the first element
of the array to put the pivot in front of everyone.

I'll come back to the \lq\lq inplace'ness'' of the partition
strategy.
Right now here are some exercises ...


Here's the algorithm for the partition strategy in this 
section:
\begin{console}
FUNCTION partition(x):
    choose pivot_index # the index to the pivot
    pivot = x[pivot_index]
    swap x[0] and x[pivot_index]
    thead = 1
    ttail = len(x) - 1
    while ttail <= thead:
        if x[head] <= pivot:
            thead = thead + 1
        else:
            if thead is not ttail:
                swap x[thead] and x[ttail]
            ttail = ttail - 1
    
    if thead > 1:
        swap x[0] and x[ttail]
\end{console}
I'm leaving the choice of the \verb!pivot_index! open.
For basic quicksort, you choose \verb!0!.
For randomized quicksort, you choose a random integer
from 0 to \verb!len(x) - 1! where \verb!len(x)! is the length
of array \verb!x!.
Etc.

There are other partition strategies.
To distinguish between the partition strategies, I'm going to 
call the above PLTR (for \verb!pivot!-\verb!left!-\verb!TODO!-\verb!right!).

Make sure you see that the above ideas (and pictures)
translate to the above algorithm.
I expect you to be able to translate algorithmic experiments
to algorithms.


And why do we want the pivot in front?
So that the algorithm is easier:
The \verb!left!, \verb!TODO!, and \verb!right! are together
and the pivot is out of their way.
In our failed partition strategy (see previous section),
recall that pivot moves around.
Now if you think about it carefully, there's nothing special
about the front.
Another possibility is to have it at the end of the array too.
(We'll see this in action later.)
You can even have the pivot, say, between the \verb!left! pile
and the \verb!TODO! pile.
As long as the position is fixed for a particular partition strategy
is will work.





\begin{ex}
In the previous partition strategy, 
I break up the array into \verb!left!, \verb!TODO!, and \verb!right! piles.
Design a partition algorithm where the array is broken up into
\verb!left!, \verb!right!, \verb!TODO! piles instead:
\begin{python}
s = r"""
from quicksort_init import *
cellwidth = 0.6
cellheight = 0.6
numleft = 4
numtodo = 6
numright = 4
pipewidth=3
pipeheight=1
p = Plot()
s = chunkedarray(cellwidth=cellwidth, 
                   cellheight=cellheight,
                   arr=[['p'], numleft*[''], numright*[''], numtodo*['']],
                   pipeheight=pipeheight,
                   chunklabels=['','left','right', 'TODO'],
)
p.add(s)
print(p)
"""
from latextool_basic import makeandincludegraphics
print(makeandincludegraphics(python=s))
\end{python}
(\verb!p! is the pivot.)
Test your pseudocode thoroughly.
\end{ex}





\begin{ex}
Perform the above process on this array where 
the first value is the pivot:
\begin{python}
s = r"""
from quicksort_init import *
p = Plot()
s = chunkedarray(cellwidth=cellwidth,
                   cellheight=cellheight,
                   arr=[[3],[],[1,2,3,4,5],[],],
                   pipeheight=pipeheight,
                   celllabels = [('thead', 1, 1),
                                 ('ttail', 5, -1),
                                 ],
                   )
p.add(s)
print(p)
"""
from latextool_basic import makeandincludegraphics
print(makeandincludegraphics(python=s))
\end{python}
\end{ex}




\begin{ex}
Perform the above process on this array where 
the first value is the pivot:
\begin{python}
s = r"""
from quicksort_init import *
p = Plot()
s = chunkedarray(cellwidth=cellwidth,
                   cellheight=cellheight,
                   arr=[[1],[],[1,2,3,4,5],[],],
                   pipeheight=pipeheight,
                   celllabels = [('thead', 1, 1),
                                 ('ttail', 5, -1),
                                 ],
                   )
p.add(s)
print(p)
"""
from latextool_basic import makeandincludegraphics
print(makeandincludegraphics(python=s))
\end{python}
\end{ex}




\begin{ex}
Perform the above process on this array where 
the first value is the pivot:
\begin{python}
s = r"""
from quicksort_init import *
p = Plot()
s = chunkedarray(cellwidth=cellwidth,
                   cellheight=cellheight,
                   arr=[[6,1,2,3,4,5]],
                   pipeheight=pipeheight,
                   )
p.add(s)
print(p)
"""
from latextool_basic import makeandincludegraphics
print(makeandincludegraphics(python=s))
\end{python}
\end{ex}




\begin{ex}
Implement the above partition strategy.
Test your code thoroughly.
\qed
\end{ex}




\begin{ex}
Using the above partition procedure {\it once} (choosing the first value
in an array as the pivot), we get the following 
array where the pivot chosen was \verb!3!:
\begin{python}
s = r"""
from quicksort_init import *
p = Plot()
s = chunkedarray(cellwidth=cellwidth,
                   cellheight=cellheight,
                   arr=[[0,2,1,2],[3],[5,7,8,4,6]],
                   pipeheight=pipeheight,
                   chunklabels=['left', ('pivot',-1), 'right'],
                   )
p.add(s)
print(p)
"""
from latextool_basic import makeandincludegraphics
print(makeandincludegraphics(python=s))
\end{python}
What was the original array?
\qed
\end{ex}




\begin{ex}
Create an example where the \verb!thead! moves past \verb!ttail!
instead of \verb!ttail! moving past \verb!thead! using the 
above partition strategy choosing the first value for pivot.
\qed
\end{ex}


\begin{ex}
Describe all arrays \verb!x! such that
after using the above partition strategy,
the values in the \verb!left! pile is in
\begin{tightlist} 
\item ascending order,
\item descending order.
\end{tightlist}
\qed 
\end{ex}

\begin{ex}
Describe all arrays \verb!x! such that
after using the above partition strategy,
the values in the \verb!right! pile is in 
\begin{tightlist}
\item ascending order,
\item descending order.
\end{tightlist}
\qed 
\end{ex}

\begin{ex}
Perform the above partition strategy on this array where 
you always choose the {\it last} value in an array as
pivot.
\begin{python}
s = r"""
from quicksort_init import *
p = Plot()
s = chunkedarray(cellwidth=cellwidth,
                   cellheight=cellheight,
                   arr=[[6,1,2,3,4,5]],
                   pipeheight=pipeheight,
                   )
p.add(s)
print(p)
"""
from latextool_basic import makeandincludegraphics
print(makeandincludegraphics(python=s))
\end{python}
\qed
\end{ex}




\begin{ex}
Note that in the last step in the above partition process,
the pivot moves to the rightmost of the \verb!left! pile.
(a) Can the pivot move to the leftmost of the \verb!right! pile?
(b) How must the above partition process be modified
so that moving the pivot to the leftmost of the \verb!right! pile?
\qed
\end{ex}




\begin{ex}
Develop an algorithm for the following modification of our 
partition strategy:
Instead of \verb!left! pile containing values $\leq$ the pivot
and the \verb!right! pile containing values $>$ the pivot,
create a partition strategy where the 
\verb!left! pile containing values $<$ the pivot
and the \verb!right! pile containing values $\geq$ the pivot.
Test your code thoroughly.
\qed
\end{ex}

\end{comment}

