%-*-latex-*-
\section{$an + b = O(n)$}
Can we show $f(n) = O(g(n))$ if
$f(n) = n$ and 
$g(n) = \frac{1}{2} n + 3$? 
Again to simplify by removing the absolute values, 
I will choose $N > 0$. 
I need to find $C$ and $N > 0$ such that 
\[
\text{$n \leq C \left( \frac{1}{2} n + 3 \right)$ for $n \geq N$}
\]
Well I can choose $C = 2$. (Do you see why?) In that case the above becomes
\[
\text{$n \leq 2 \left( \frac{1}{2} n + 3 \right) = n + 6$ for $n \geq N$}
\]
which is true for $N = 0$. Right?

Here's the problem and the solution:

\begin{eg}
Let $f(n) = n$ and $g(n) = \frac{1}{2}n + 3$.
Show that $f(n) = O(g(n))$.
\end{eg}

Solution.
Let $N = 0$ and $C = 2$. Let $n \geq N$.
We have
\begin{align*}
n &\leq n  + 6\\
\THEREFORE n &\leq n + 6 = 2 \left( \frac{1}{2}n + 3 \right)\\
\THEREFORE |n| &\leq 2 \left| \frac{1}{2}n + 3 \right| \\
\THEREFORE |f(n)| &\leq C |g(n)|
\end{align*}
Therefore $f(n) = O(g(n))$.
\qed

\begin{ex}
Let $f(n) = an + b$ and $g(n) = n$ 
where $a,b,c,d$ are real numbers with $a \neq 0$.
Show that $f(n) = O(g(n))$, i.e., 
\[
an + b = O(n)
\]
\end{ex}

\begin{ex}
Let $f(n) = an + b$ and $g(n) = cn + d$ 
where $a,b,c,d$ are real numbers with $a \neq 0$ and $c \neq 0$.
Show that $f(n) = O(g(n))$.
\end{ex}


\begin{ex}
Let $f(n) = c$ and $g(n) = n$ 
where $c$ is a real number.
Show that $f(n) = O(g(n))$, i.e., 
\[
c = O(n)
\]
\end{ex}

\begin{ex}
Show that $c = O(1)$ for any real number $c$.
\qed
\end{ex}
