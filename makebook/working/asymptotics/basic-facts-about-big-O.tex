%-*-latex-*-
\section{Basic Facts about Big-O}

Here are some basic facts you should know. 

\begin{thm}
Let $f$, $g$, $h$, $f_i$, and $g_i$ be functions 
and $c \neq 0$ be a constant.
\begin{enumerate}
 \item[(a)] If $|f(n)| \leq |g(n)|$ for all $n \geq 1$, then $f = O(g)$.
 \item[(b)] $f = O(f)$
 \item[(c)] If $f = O(g)$ and $g = O(h)$, then $f = O(h)$.
 \item[(d)] $f = O(cg)$ iff $f = O(g)$
 \item[(e)] $cf = O(g)$ iff $f = O(g)$
 \item[(f)] If $f_i = O(g_i)$ $(i=1,2)$ then
 \[
  f_1+f_2 = O \bigl( \max(|g_1|,|g_2|) \bigr), \,\,\, f_1 f_2 \in O(g_1 g_2)
 \]
 \item[(g)] $f_i = O(g)$ $(i=1,2)$ $\implies$ $f_1+f_2 = O(g)$
 \item[(h)] $f \in O(g)$ $\implies$ $O(f) \subseteq O(g)$.
\end{enumerate}
\end{thm}

Here are some examples on how to use the above theorem.

\begin{eg}
Prove that $n^3 + 1 = O(n^3)$.
\end{eg}

\textit{Solution.}
We have
\begin{align*}
n^3 = O(n^3) & & \text{by Theorem 1(b)} \tag{1}
\end{align*}
Also, 
\begin{align*}
           1 &\leq |n^3| \text{ for $n \geq 1$} \\
\THEREFORE 1 &= O(n^3) & \text{by Theorem 1(a)} \tag{2}
\end{align*}
Using (1) and (2),
\begin{align*}
n^3 + 1 = O(n^3) & & \text{by Theorem 1(g)}
\end{align*}
\qed


\begin{eg}
Prove that $7n^5 - 3n = O(n^5)$.
\end{eg}

\textit{Solution.}
We have
\begin{align*}
           7n^5 &= O(7n^3) & & \text{by Theorem 1(a)}  \\
\THEREFORE 7n^5 &= O(n^3) & & \text{by Theorey 1(d)} \tag{1}
\end{align*}
Also, 
\begin{align*}
|-3n| &\leq 3n^3 \text{ for $n \geq 1$} \\ 
\THEREFORE -3n &= O(3n^3)  & & \text{by Theorem 1(a)}  \\
\THEREFORE -3n &= O(n^3)   & & \text{by Theorem 1(d)} \tag{2}
\end{align*}

From (1) and (2), we have
\begin{align*}
7n^5 - 3n 
&= 7n^5 + (-3n) \\
&= O(n^3) & & \text{by Theorem 1(g)} 
\end{align*}
\qed


\begin{ex}
Show that $n^2 = O(n^3)$.
\qed
\end{ex}

\begin{ex}
Show that $5n^7 + 42n^2 - 5 = O(n^7)$.
\qed
\end{ex}

\begin{ex}
Show that $5n^7 + \sin ((2n+1)^2 \pi/4) = O(n^7)$.
\qed
\end{ex}

\begin{ex}
Show that $5n + 0.5^n = O(n)$.
\qed
\end{ex}

\begin{ex}
Show that $(-1)^n n^2 + 5 = O(n^2)$.
\qed
\end{ex}

\begin{ex}
Show that $e^{-n} \sin n = O(1)$.
\qed
\end{ex}


\begin{ex}
Investigate the following question: Suppose $f = O(g)$. Is it
true that $h^f \in O(h^g)$? Of course the function $h^f$ is just
given by $(h^f)(n) = h(n)^{f(n)}$.
\qed
\end{ex}
