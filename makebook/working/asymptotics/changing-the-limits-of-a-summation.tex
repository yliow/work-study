%-*-latex-*-
\section{Bubblesort again: Changing the Limits of a Summation}

In the previous section, I showed you the following big-O formula:
\[
\sum_{i=1}^n i^k = O(n^{k+1})
\]
But if you look at the big-O computation of the worst case runtime
for the bubblesort, you see that you the sum that appear is actually
this:
\[
\sum_{i=0}^{n-2} i
\]

RANT ...  RANT ... RANT:
\begin{itemize}
\li The lower limit of the summation is $0$ and not $1$!!!
\li The upper limit of the summation is $n-2$ and not $n$!!!
\end{itemize}

AARRGHHH!!! OK, don't panic ... let's see what I can do about them ...

\begin{thm} 
From the above, we know that if $k \geq 0$ is an integer, then
\[ \sum_{i=1}^{n} i^k = O(n^{k+1}) \]
The above is still true if the sum 
\[
\sum_{i = 1}^n
\]
is replaced by a sum of the form
\[
\sum_{i = c}^{n - d}
\]
where $c$ and $d$ are constants.
\end{thm}

\begin{eg}
For instance the theorem tells us that 
\begin{align*}
\sum_{i = 0}^{n} i &= O(n^2) \\
\sum_{i = 3}^{n - 1} i^{2} &= O(n^3) \\
\sum_{i = 2}^{n + 2} i^{100} &= O(n^{101})
\end{align*}
\qed
\end{eg}

Now I'm going to prove that changing the lower limit of the summation
does not affect the big-O.
In fact this is true for summation of \textit{any} type of terms, not just $i^k$.
The main idea is that the first few terms of a summation is of course ...
\textit{constant!!!}

For instance for the case where the terms are $i^7$ and the lower
limit of the summation is changed to 4, I have
\begin{align*}
\sum_{i = 4}^n i^7 
&= \sum_{i=1}^n i^7 - 1^7 - 2^7 - 3^7 \\
&= \sum_{i=1}^n i^7 - C
\end{align*}
where $C = - 1^7 - 2^7 - 3^7$ is a constant.
Of course adding a constant to
$\sum_{i = 4}^n i^7$
does not change its big-O.

This gives me the hint that if I should look at an expression of the form
\[
F(n) = G(n) + C
\]
I want to be able to say:
\[
F(n) = G(n) + C \implies F(n) = O \left( G(n) \right)
\]
Can I?
If not, what condition should I put on $F(n)$ and $G(n)$ to be able to 
say that??
Well ... if $G(n)$ is unbounded, it beats 
$C$ after some point, then for $n \geq N$,
then 
\[
G(n) \geq C
\]
and then
\[
F(n) = G(n) + C \leq G(n) + G(n) = 2G(n)
\]
That works!!!
But what if $G(n)$ is bounded?
Well, then $G(n) = O(1)$ and
\[
F(n) = O(G(n) + C) = O(1 + 1) = O(1)
\]
and $F(n)$ and $G(n)$ has the same big-O again.
So whether $G(n)$ is bounded or unbounded, we have the same result.

\begin{thm} If $c$ is a constants then
\[
\sum_{i = c}^{n} f(i) =
O
\left(
\sum_{i = 1}^{n} f(i)
\right)
\]
\end{thm}

\textit{Proof.}
There is nothing to prove is $c = 1$.
If $c > 1$, then
\[
\sum_{i=c}^n f(i) 
= \sum_{i = 1}^{n} f(i) - (f(1) + f(2) + \cdots + f(c - 1))) 
= \sum_{i = 1}^{n} f(i) + C
\]
where $C$ is a constant.
And if $c < 1$, then
\[
\sum_{i=c}^n f(i) 
= \sum_{i = 1}^{n} f(i) + (f(c) + f(c+1) + \cdots + f(0)))
= \sum_{i = 1}^{n} f(i) + D
\]
where $D$ is a constant.
Therefore in both cases, 
$\sum_{i=c}^n f(i)$ and
$\sum_{i=1}^n f(i)$
differ by a constant.
From our observation above, these two functions of $n$ have the 
same big-O (or $n$).
\qed



Now what about the upper limit?
In the case where $f(i) = i^k$ for a constant $k$,
if the upper limit is $n - 2$ instead of $n$:
\[
\sum_{i = 1}^{n - 2} i^k
= \sum_{i=1}^n i^k - (n-1)^k - n^k 
\]
Since $\sum_{i=1}^n i^k = O(n^{k+1})$,
$-(n-1)^k = O(n^k)$, and $-n^k = O(n^k)$,
we have
\[
\sum_{i = 1}^{n - 2} i^k
= O(n^{k+1})
\]


Now let me show you how to use the above to redo the runtime
of our bubblesort ... 

\newpage

\begin{eg}
Compute the big-O of the runtime of the bubblesort:
\begin{Verbatim}[frame=single, fontsize=\footnotesize]
for i = n - 2, ..., 0:
    for j = 0, ..., i:
        if a[i] > a[i + 1]:
            t = a[i]
            a[i] = a[i + 1]
            a[i + 1] = t
\end{Verbatim}
\end{eg}

\textit{Solution.}
The runtime of the body of the inner loop is constant time
say $A = O(1)$.
The runtime of the inner loop is
\[
U(i) = \sum_{j=0}^i A + B(i+1) + C
\]
for constants $B$ and $C$.
Therefore
\begin{align*}
U(i) 
&= O \left( \sum_{j=0}^i A + B(i+1) + C \right)
= O \left( \sum_{j=0}^i 1 + i \right)
= O(i + i)
= O(i)
\end{align*}
For the whole algorithm, the runtime is:
\[
T(n) = \sum_{i=0}^{n-2} U(i) + Cn + D
\]
Therefore
\begin{align*}
T(n) 
&= O \left (
         \sum_{i=0}^{n-2} U(i) + Cn + D
     \right)
&= O \left (
         \sum_{i=0}^{n-2} i + n
     \right)
&= O \left (
         n^2 + n
     \right)
&= O(n^2)
\end{align*}
\qed

\newpage



[TODO: Talk about the case where the limits are 
floor/ceiling of proportions of $n$]
