%-*-latex-*-
\tinysidebar{\debug{exercises/{best-average-worst-0/question.tex}}}
The linear search algorithm searches from index value 0 to the
last index value.
The \textit{reverse} linear search is pretty much the same as the linear
search except that it starts with the last index value and moves
toward the 0 index value.
For the following, assume as before the array is \verb!x!
and the size of the array is \verb!n!. 
\begin{Verbatim}[frame=single, fontsize=\footnotesize]
index = -1
for i = n - 1, n - 2, ..., 1, 0:
    if x[i] is target:
        index = i
        break
\end{Verbatim}
Here's the reverse linear search algorithm with goto statements:
\begin{Verbatim}[frame=single, fontsize=\footnotesize]
                                   time      
         index = -1                t1     
         i = n - 1                 t2     
LOOP:    if i <= -1:               t3      
             goto ENDLOOP          t4     
         if x[i] is not target:    t5        
             goto ELSE             t6      
         index = i                 t7     
         goto ENDLOOP              t8     
ELSE:    i = i - 1                 t9     
         goto LOOP                 t10    
ENDLOOP:
\end{Verbatim}

(a) Assume the best case, i.e., the \verb!target! is at index 
\verb!n - 1!.
Write down the number of times each of the statement is executed.
Compute the total runtime for this case, $T_b(n)$ and then
write down the big-O of this function.

(b) Assume the worst case where the \verb!target! is not found in the 
array.
Write down the number of times each of the statement is executed.
Compute the total runtime for this case, $T_w(n)$ and then
write down the big-O of this function.

(c) Assume the worst case where the \verb!target! is only at index 0.
Write down the number of times each of the statement is executed.
Compute the total runtime for this case, $T_w(n)$ and then
write down the big-O of this function.

(d) Assume now that \verb!target! is at index value $k$.
What is the runtime for this case? 
This should be a formula involving $k$ and the constants 
\verb!t1!, \verb!t2!, ...
(When you set $k$ to $0$, you should get the answer in (c)
and when you set $i$ to $n - 1$, you should get the answer in (a).)
Write it as a polynomial in $k$, given the coefficient of the polynomial
simple constant names $A$, $B$, ...

(e) Part (d) should give you $n$ values, say $T_0$, ..., $T_{n-1}$,
i.e., $T_k$ is the runtime for the case where the \verb!target! is at index 
\verb!k!.
Assume that all the above cases are equally likely.
Compute the average of these $n$ values to obtain
the average runtime $T_a(n)$.
(For now we'll forget about the case where \verb!target! is not in the array.)
