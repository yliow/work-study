\tinysidebar{\debug{exercises/{dac-sum/question.tex}}}
Here's our simple sum to $n$ algorithm:
\begin{Verbatim}[frame=single, fontsize=\footnotesize]
def sum(n):
    s = 0
    for i = 1, 2, 3, ..., n:
        s = s + i
    return s
\end{Verbatim}
Let's do it in a different way.
Here's another function that sums from one point to another:
\begin{Verbatim}[frame=single, fontsize=\footnotesize]
def sum2(m, n):
    s = 0
    for i = m, m + 1, m + 2, ..., n:
        s = s + i
    return s
\end{Verbatim}
Design and write a sum to $n$ function that works recursively like this:
Instead of summing 1 to \verb!n! in a loop, the function
calls \verb!sum2! to sum from \verb!1! to \verb!n/2!, 
calls \verb!sum2! to sum from 
\verb!n/2 + 1! to \verb!n!, 
and finally returns the sum of the two return values.
The base case is when \verb!n! is \verb!1!.
Test your code.
What do you think is the runtime?
