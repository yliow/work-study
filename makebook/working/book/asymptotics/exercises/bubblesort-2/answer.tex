%-*-latex-*-
\tinysidebar{\debug{exercises/bubblesort-2/answer.tex}}
The recursive function calls (without the details) are
\begin{console}
bubblesort(x, n - 1) 
bubblesort(x, n - 2)
bubblesort(x, n - 3)
.
.
.
bubblesort(x, 1)
bubblesort(x, 0)
\end{console}
Let \verb!last_index! be $k$ (just to write less).

If $k = 0$,
\lq\lq \verb!return x;!" has a runtime of $O(n)$
and the space is $O(n)$.

If $0 < k \leq n - 1$.
($k$ begins with $n - 1$ and decrements with each
recursion.)
Let $T(k)$ be the time taken for this recursive
function call, including the time taken for 
\verb!bubblesort(x, k - 1)!.
\begin{console}[fontsize=\scriptsize]
                                                     
std::vector< int > bubblesort(std::vector< int > x, ----+ O(n) 
                              int last_index = -1)  ----+
{
    if (last_index == -1)                           ----+ constant
    {                                                   |
        last_index = x.size() - 1;                      |
    }                                               ----+

    if (last_index == 0)
    {                                               ----+ constant
        return x;                                       |
    }                                               ----+
    else                                            
    {                                               
        // Do one pass:                             ----+ k * O(n) 
        for (int i = 0; i < last_index; ++i)            |
        {                                               |
            if (x[i] > x[i + 1])                        |
            {                                           |
                x = swap(x, i, i + 1);                  |
            }                                           |
        }                                           ----+
        // Recurse:                                 
        x = bubblesort(x, last_index - 1);          ----- T(k - 1) + O(n) 
        return x;                                   ----- O(n)
    }
}
\end{console}
So
\[
T(k) = T(k - 1) + k \cdot O(n)
\]
i.e.,
\[
T(k) = T(k - 1) + k(An + B)
\]
where $A,B$ are constants.
Therefore
\begin{align*}
T(n - 1)
&= T(n - 2) + (n - 1)(An + B) \\
&= T(n - 3) + ((n-2)+(n-1))(An + B) \\
&= T(n - 4) + ((n - 3) + (n-2) + (n-1))(An + B) \\
&= ... \\
&= T(n - n) + ((n - (n-1)) + \cdots + (n - 3) + (n-2) + (n-1))(An + B) \\
&= T(0) + (1 + \cdots + (n - 3) + (n-2) + (n-1))(An + B) \\
&= C + (1 + \cdots + (n-1))(An + B) \\
&= C + \frac{(n-1)(n-2)}{2}(An + B) \\
&= O(n^3)
\end{align*}
where $C = T(0)$ is a constant.
