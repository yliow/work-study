\tinysidebar{\debug{exercises/{recursion-1/question.tex}}}
You are given the following recursive function:
\[
f(n) = 
\begin{cases}
2           & \text{if n = 0} \\
3f(n/2) + 5 &\text{otherwise} 
\end{cases}
\]
Note that \lq\lq $n/2$ '' really 
meant the floor of $n/2$ so technically I should 
really write
\[
f(n) = 
\begin{cases}
  2                 & \text{if n = 0} \\
3f(\floor{n/2}) + 5 & \text{otherwise}
\end{cases}
\]
Compute $f(7)$ performing your computation like a computer,
evaluating one operation of performing only one substitution at a time.
(Recall that floor $\floor{x}$ means the integer before $x$
and ceiling $\ceiling{x}$ means the integer after $x$.
For instance $\floor{3.4} = 3$ and $\ceiling{3.4} = 4$.)
