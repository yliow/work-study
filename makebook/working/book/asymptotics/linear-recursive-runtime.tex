%-*-latex-*-
\sectionthree{Linear recursive runtime}
\begin{python0}
from solutions import *; clear()
\end{python0}

So you see that when you measure the runtime of an algorithm
like our Fibonacci function implemented with linear recursion
will give you a runtime function that looks rather similar.
So the next step is to convert the runtime function
to a \textit{non-recursive} closed form.

Before I show you the closed form for the runtime of Fibonacci, let's just get a feel for the
runtime functions by calculating one.
It turns out that it's pretty bad.

Suppose I need to solve the problem of computing this
function (it's somewhat like my Fibonacci).
Note that $g(n)$ is defined by linear recursion.
\begin{align*}
g(n) = 
\begin{cases}
1                     & \text{ if $n = 0$} \\
3                     & \text{ if $n = 1$} \\
2g(n - 1) + 3g(n - 2) & \text{ if $n \geq 2$}
\end{cases}
\end{align*}
Suppose I choose to write my algorithm like this:
\begin{Verbatim}[frame=single,fontsize=\footnotesize]
def g(n):
    if n == 0: 
        return 1
    elif n == 1:
        return 3
    else:
        return 2 * g(n - 1) + 3 * g(n - 2)
\end{Verbatim}
You know that the runtime function would be like this:
\[
T(n) = 
\begin{cases}
A                     & \text{if $n = 0, 1$} \\
T(n-1) + T(n-2) + B & \text{if $n > 1$}
\end{cases}
\]
for some constants $A$ and $B$.
Later I'll show you how to compute an approximation (in the big-O sense)
for $T(n)$.
For now, suppose we compute $g(5)$ using the above 
program, compute like what a program would, executing one statement
or one operation at a time.
This is what you would see:

\begin{align*}
g(5)
&= 2g(4) + 3g(3) \\
&= 2(2g(3) + 3g(2)) + 3g(3) \\
&= 2(2(2g(2) + 3g(1)) + 3g(2)) + 3g(3) \\
&= 2(2(2(2g(1) + 3g(0)) + 3g(1)) + 3g(2)) + 3g(3) \\
&= 2(2(2(2(3) + 3g(0)) + 3g(1)) + 3g(2)) + 3g(3) \\
&= 2(2(2(6 + 3g(0)) + 3g(1)) + 3g(2)) + 3g(3) \\
&= 2(2(2(6 + 3(1)) + 3g(1)) + 3g(2)) + 3g(3) \\
&= 2(2(2(6 + 3) + 3g(1)) + 3g(2)) + 3g(3) \\
&= 2(2(2(9) + 3g(1)) + 3g(2)) + 3g(3) \\
&= 2(2(18 + 3g(1)) + 3g(2)) + 3g(3) \\
&= 2(2(18 + 3(3)) + 3g(2)) + 3g(3) \\
&= 2(2(18 + 9) + 3g(2)) + 3g(3) \\
&= 2(2(27) + 3g(2)) + 3g(3) \\
&= 2(54 + 3g(2)) + 3g(3) & & \text{will this ever end?!?} \\
&= 2(54 + 3(2g(1) + 3g(0))) + 3g(3) \\
&= 2(54 + 3(2(3) + 3g(0))) + 3g(3) \\
&= 2(54 + 3(6 + 3g(0))) + 3g(3) \\
&= 2(54 + 3(6 + 3(1))) + 3g(3) \\
&= 2(54 + 3(6 + 3)) + 3g(3) \\
&= 2(54 + 3(9)) + 3g(3) \\
&= 2(54 + 27) + 3g(3) \\
&= 2(81) + 3g(3) \\
&= 162 + 3g(3) \\
&= 162 + 3(2g(2) + 3g(1)) & & \text{ ... oh no ... here we go again ...}\\
&= 162 + 3(2(2g(1) + 3g(0)) + 3g(1)) \\
&= 162 + 3(2(2(3) + 3g(0)) + 3g(1)) \\
&= 162 + 3(2(6 + 3g(0)) + 3g(1)) \\
&= 162 + 3(2(6 + 3(1)) + 3g(1)) \\
&= 162 + 3(2(6 + 3) + 3g(1)) \\
&= 162 + 3(2(9) + 3g(1)) \\
&= 162 + 3(18 + 3g(1)) \\
&= 162 + 3(18 + 3(3)) \\
&= 162 + 3(18 + 9) \\
&= 162 + 3(27) \\
&= 162 + 81 \\
&= 243 & & \text{PHEW!!!}
\end{align*}

\begin{ex}
  Show that if \verb!g(n)! is the following
\begin{Verbatim}[frame=single,fontsize=\footnotesize]
def g(n):
    if n == 0: 
        return 1
    elif n == 1:
        return 3
    else:
        return 2 * g(n - 1) + 3 * g(n - 2)
\end{Verbatim}
Then the runtime is $T(n) = O(2^n)$. (This is not the best big-O.)
What is the space complexity?
\end{ex}

\begin{ex}
  Let
\begin{Verbatim}[frame=single,fontsize=\footnotesize]
def h(n):
    if n == 0: 
        return 1
    elif n == 1:
        return 2
    elif n == 2:
        return 42
    else:
        return 7 * h(n - 1) - 3 * h(n - 2) + 4 * h(n - 3)
\end{Verbatim}
What is a big-O of $T(n)$ for \verb!h(n)!?
What is the space complexity?
\end{ex}


So the execution of $g(5)$ using the above algorithm
seems to indicate that $T(5)$ is huge.
You can see even from this simple simulation that the main
problem is that many function calls were actually repeats.
For instance go ahead and count the number of times
$g(2)$ was executed in the above.

It turns out that except for recursion like this:
\begin{Verbatim}[frame=single,fontsize=\footnotesize]
def h(n):
    if n == 0:
        return 42
    else:
        return 7 * h(n - 1) + n
\end{Verbatim}
where in the recursive part only \textit{one} recursive function call was made,
most cases will usually be extremely bad, as in exponential bad.

It turns out that in general if you have a numeric recursive function
like
\[
T(n) = aT(n - 1) + bT(n - 2) + c
\]
where $a$, $b$, and $c$ are constants,
you can always compute a nice formula for $T(n)$.
A \lq\lq nice'' formula for $T(n)$ that does \textit{not} depend on 
$T(i)$ for smaller $i$ but only on $n$, i.e.,
you can compute a closed form formula for $T(n)$.

In fact more generally there are extremely powerful 
tools to compute an \textit{exact} closed form for more complicated
recursion such as 
\[
T(n) = n^2 T(n-1) + (1 + n) T(n - 2) + \frac{2}{3}n^2 - 1 
\]
But I'll stick to the simple case of
\[
T(n) = aT(n - 1) + bT(n - 2) + c
\]
where $a,b,c$ are constants.

Although there are techniques to compute exact closed forms for $T(n)$,
remember that we only need the big-O of $T(n)$.

For the case of a recursive function such as 
\begin{Verbatim}[frame=single,fontsize=\footnotesize]
def g(n):
    if n == 0: 
        return 1
    elif n == 1:
        return 3
    else:
        return 2 * g(n - 1) + 3 * g(n - 2)
\end{Verbatim}
which has a runtime $T(n)$ satisfying
\[
T(n) = T(n - 1) + T(n - 2) + A
\]
where $A$ is a constant, a rough approximation (using the method
in an earlier section) the runtime is
\[
T(n) = O(2^n)
\]
i.e., exponential.
Which means the runtime is bad.
How bad? What exactly is \lq\lq exponential bad"?
You can use the fibonacci function as an example:
\begin{Verbatim}[frame=single,fontsize=\footnotesize]
def fib(n):
    if n == 0: 
        return 1
    elif n == 1:
        return 1
    else:
        return fib(n - 1) + fib(n - 2)
\end{Verbatim}
and print \verb!fib(n)! for \verb!n = 0, 1, 2, ..., 100!.
Remember we did this before in CISS245.
It will grind to a halt before hitting \verb!60!.

In the above, instead of $O(2^n)$, we can be more precise.

First, you ignore the constant $c$ in
\[
T(n) = aT(n - 1) + bT(n - 2) + c \tag{1}
\]
and look at this recursion instead:
\[
T^{(h)}(n) = aT^{(h)}(n - 1) + bT^{(h)}(n - 2) \tag{2}
\]
This is called the \defone{homogeneous part} of $T(n)$.
You need to know that $T^{(h)}(n)$ must have a solution
roughly of the form
\[
T^{(h)}(n) = r^n
\]
where $r$ is a constant.
(Take this on faith for now -- the justification will be
provided in MATH325.)
The immediate goal now is to compute $r$.
You substitute this into (2) to get
\[
r^n = ar^{n-1} + br^{n-2}
\]
You cross out $r^{n - 2}$ to get
\[
r^2 = ar + b
\]
i.e., a quadratic equation.
This then allows you to solve for $r$.

Let's try this technique on our
\[
T(n) = T(n - 1) + T(n - 2) + c
\]
The homogeneous part of this equation is
\[
T^{(h)}(n) = T^{(h)}(n - 1) + T^{(h)}(n - 2)
\]
Let $T^{(h)}(n) = r^n$ and substitute it into
\[
T^{(h)}(n) = T^{(h)}(n - 1) + T^{(h)}(n - 2)
\]
to get
\[
r^n = r^{n - 1} + r^{n - 2}
\]
Cross out $r^{n - 2}$ to get
\[
r^2 = r + 1
\]
which gives us the quadratic
\[
r^2 - r - 1 = 0
\]
Using the quadratic polynomial root formula you get
\[
r = \frac{-(-1) \pm \sqrt{(-1)^2 - 4(1)(-1)} }{2}
\]
which gives us
\[
r = \frac{1 \pm \sqrt{5}}{2}
\]
Let 
\[
r_1 = \frac{1 + \sqrt{5} }{2} 
\]
(... this is the one that is about 1.618)
and 
\[
r_2 = \frac{1 - \sqrt{5} }{2}
\]
Then you know that if the recurrence relation on $T^{(h)}(n)$ is
\[
T^{(h)}(n) = T^{(h)}(n - 1) + T^{(h)}(n - 2)
\]
then $T^{(h)}(n)$ must have the form
\[
T^{(h)}(n) = \alpha \cdot r_1^n + \beta \cdot r_2^n
\]
where $\alpha, \beta$ are constants.
This is the closed form for $T^{(h)}(n)$, not $T(n)$.
Now, $|r_2| = 0.618...$ whereas $r_1 = 1.618...$.
Since $r_1 > |r_2|$, I get 
\[
T^{(h)}(n) = \alpha \cdot r_1^n + \beta \cdot r_2^n = O(r_1^n)
\]
But what about $T(n)$?
Remember that
\[
T(n) = T^{(h)}(n) + c
\]
Since $T^{(h)}(n) = O(r_1^n)$ and $r_1 > 1$,
$T^{(h)}(n)$ is going to climb way faster than the constant $c$.
Therefore the big-O of $T(n)$ is governed by the big-O of
$T^{(h)}(n)$ and not $c$.
Hence
\[
T(n) = O \left( T^{(h)}(n) + c \right) = O \left( T^{(h)}(n) \right) = O(r_1^n)
\]
which means that the algorithm has exponential runtime with base $r_1 > 1$.
Note that 
base of the exponential runtime is $1.618...$ instead of $2$.
$2^n$ climbs much faster than $1.618^n$.
That's because
\[
\frac{2^n}{1.618^n} = 1.236...^n
\]
and $1.236...^n$ grows toward infinity as $n$ grows unboundedly
since $1.236 > 1$.
So the asymptotic
\[
T(n) = O(r_1^n)
\]
is way more accurate than
\[
T(n) = O(2^n)
\]

Note that at the stage where you solve for $r$ in 
the quadratic equation in $r$:
\[
r^2 = ar + b \tag{*}
\]
in the general case, 
you will have \textit{three} cases: $(*)$ has
\begin{enumerate}
\item[(A)] Two distinct real roots
\item[(B)] Two distinct complex (and nonreal) roots
\item[(C)] Two roots with the same real value
\end{enumerate}
The Fibonacci case of 
\[
r^2 = r + 1
\]
gives us two distinct real roots.

For Case (A) and Case (B) above, the $T(n)$ is just
\[
T(n) = \alpha r_1^n + \beta r_2^n
\]
(For Case (B) you would have to use complex
numbers -- obviously you would need to know complex numbers for such cases.)
For Case (C) where there is one value $r_1$ for both roots,
the closed form becomes
\[
T(n) = \alpha r_1^n + \beta nr_1^n
\]
so in this case
\[
T(n) = O(nr_1^n)
\]
Of course as for the specific big-O class, you have to 
compute the exact root values.

[EXERCISES]
