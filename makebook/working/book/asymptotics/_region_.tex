%-*-latex-*-
\message{ !name(roots-of-asymptotics.tex)}
\message{ !name(roots-of-asymptotics.tex) !offset(-2) }
\section{Roots of Asymptotics}

Historically, the use of asymptotics to analyze the
growth of functions started around 1800s and was used extensively
for mathematicians studying analysis (think Calculus) and
especially in the study of analytic number theory. In fact it was
introduced by Paul Bachmann who was an analytic number theorist.
Here's a famous example from analytic number theory, the famous
Prime Number Theorem. The point is to give you a sense of
perspective and history (you do want to be educated right?). The
other reason is to let you read and interpret some mathematical
facts before we go into details.

Number theory is the study of integers. Originally the techniques
used involved only integers. Later on techniques involving number
which are not integers are used to analyze integers. In particular
analytic number theory uses techniques from analysis (this means
topics like real and complex analysis). One big goal of number
theorists is to know everything about prime numbers 2, 3, 5, 7,
11, 13, $\ldots$. In particular they want to know what is the
$n$--th prime $p_n$ for positive $n$. So $p_1 = 2$, $p_5 = 11$.
They want to know if there is a formula for $p_n$ so that for
instance they can work out very quickly what is $p_{1000000}$.
They are interested also in the number of primes less then a fixed
number say $x$; this is called $\pi(x)$. So for instance $\pi(1) =
0$, $\pi(6.2) = 3$. Right? 


$\pi(x)$ for $0 \leq x \leq 10$:
\begin{python}
from plot import *

N = 10
f = file('pi.txt', 'r')
points = []
for i in range(N + 1):
    t = f.readline()
    x, y = t.split(',')
    x, y = int(x), int(y)
    if points==[] or y != points[-1][1]:
        points.append((x, y))
points.append((N, points[-1][1]))

p = Plot(line_width=1, width='5in', height='3in', domain='0:%s' % N)
p.add(points, style='step', color='red', python=1)
print p
\end{python}


$\pi(x)$ for $0 \leq x \leq 100$:
\begin{python}
from plot import *
from math import sqrt

N = 100
f = file('pi.txt', 'r')
points = []
for i in range(N + 1):
    t = f.readline()
    x, y = t.split(',')
    x, y = int(x), int(y)
    if points==[] or y != points[-1][1]:
        points.append((x, y))
points.append((N, points[-1][1]))

p = Plot(line_width=1, width='5in', height='3in', domain='0:%s' % N)
p.add(points, style='step', color='red', python=1)
print p
\end{python}



$\pi(x)$ for $0 \leq x \leq 1000$:
\begin{python}
from plot import *
from math import sqrt

N = 1000
f = file('pi.txt', 'r')
points = []
for i in range(N + 1):
    t = f.readline()
    x, y = t.split(',')
    x, y = int(x), int(y)
    if points==[] or y != points[-1][1]:
        points.append((x, y))
points.append((N, points[-1][1]))

p = Plot(line_width=1, width='5in', height='3in', domain='0:%s' % N)
p.add(points, style='step', color='red', python=1)
print p
\end{python}



$\pi(x)$ for $0 \leq x \leq 100,000$:
\begin{python}
from plot import *
from math import sqrt

N = 100000
f = file('pi.txt', 'r')
points = []
for i in range(N + 1):
    t = f.readline()
    x, y = t.split(',')
    x, y = int(x), int(y)
    if points== [] or y != points[-1][1]:
        points.append((x, y)
points.append((N, points[-1][1]))

points = points[0::100]
p = Plot(line_width=1, width='5in', height='3in', domain='0:100000')
p.add(points, style='step', color='red', python=1)
print p
\end{python}



\begin{ex}
Try to find a \lq\lq nice'' function $f(x)$ such that 
\[
\pi(x) \approx f(x)
\]
for all large values of $x$.
\end{ex}

Here's one \lq\lq nice'' one plotted against the $\pi(x)$ 
for $0 \leq x \leq 10,000,000$:
\begin{python}
from plot import *
from math import sqrt

N = 10000000
f = file('pi.txt', 'r')
points = []
for i in range(N + 1):
    t = f.readline()
    x, y = t.split(',')
    x, y = int(x), int(y)
    if points==[] or y != points[-1][1]:
        points.append((x, y))
points.append((N, points[-1][1]))

points = points[0::10000]
p = Plot(line_width=1, width='6in', height='5in', domain='0:%s' % N)
p.add(points, style='step', color='red', python=1)
p.add('x/log(x, e)', color='blue', python=1, num_points=30)
print p
\end{python}
[HINT: Try $x^a /\ln^b x$ for suitable $a$ and $b$.]

\message{ !name(roots-of-asymptotics.tex) !offset(-155) }
