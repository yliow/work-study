%-*-latex-*-
\section{Mergesort ... 2nd Version}

Now that you understand the basic idea behind mergesort,
let's move on to improve the memory usage of the previous
mergesort.


Now I'm going to use the same idea for the splitting part as well.
Instead of splitting the value of \verb!x! into two different arrays,
I'm just going to use \verb!x! but with extra index variables to tell
me the beginning and ending of the two halves.
As an example, take a look at this:
\[
x = [3, 6, 4, 2, 7, 8, 12, 15, 1, 5]
\]
I know that $\mergesort$ splits $x$
into two halves.
\begin{align*}
x &= [3, 6, 4, 2, 7, 8, 12, 15, 1, 5] \\
i = 0, j = 4 &: \text{to denote the left chunk $x[i],...,x[j]$} \\
j' = 5, k = 9 &: \text{to denote the right chunk $x[j'],...,x[k]$}
\end{align*}
Of course $j'$ is redundant since $j' = j + 1$.
Note that in the mergesort algorithm, the two arrays which are merged
are contiguous, i.e., next to each other.
So it is true that $j' = j + 1$.

Note that the first chunk is empty if
\[
i > j
\]
The second chunk is empty if
\[
j + 1 > k
\]

In terms of C++ the split code would look like this:
\begin{Verbatim}[frame=single,fontsize=\footnotesize]
void split(int x[], int i, int & j, int k)
{
    j = x_start + (x_end - x_start + 1) / 2;
}
\end{Verbatim}


Now let's think about the merge operation ...

\begin{ex} (Merge 1)
The following is a version of merge that does not
use any extra memory.
The idea is simple.
Suppose this is the array
\begin{Verbatim}[frame=single,fontsize=\footnotesize]
x: ..., 1, 3, 5, 7, 2, 4, 6, 8, ...
        i        j           k
\end{Verbatim}
Say $i = 8$, $j = 11$, and $k = 15$.
Notice that 
\verb!x[i],...,x[j]!
and
\verb!x[j+1],...,x[k]! 
are already sorted.
The point is to merge the values in these two chunks.
Now \verb!x[i] < x[j+1]!, we have
\begin{Verbatim}[frame=single,fontsize=\footnotesize]
x: ..., 1, 3, 5, 7, 2, 4, 6, 8, ...
           i     j           k
\end{Verbatim}
Now $x[i] > x[j]$. we keep $x[j+1]$ somewhere temporarily and shift
the $x[i,..,j]$ chunk to the right (note the change in $i,j,k$):
\begin{Verbatim}[frame=single,fontsize=\footnotesize]
x: ..., 1, ?, 3, 5, 7, 4, 6, 8, ...
              i     j        k        t = 2
\end{Verbatim}
and then putting the value of \verb!t! into $x[i-1]$ to get
\begin{Verbatim}[frame=single,fontsize=\footnotesize]
x: ..., 1, 2, 3, 5, 7, 4, 6, 8, ...
              i     j        k       
\end{Verbatim}
Repeat!
Run this function, test it, and then compute the runtime
assuming the two chunk have length $n$.
\end{ex}

\begin{ex} (Merge 2)
In the previous problem (new version of merge), instead of shifting the
left chunk over by one step, 
\begin{Verbatim}[frame=single,fontsize=\footnotesize]
x: ..., 1, 3, 5, 7, 2, 4, 6, 8, ...
           i     j           k
\end{Verbatim}
becomes
\begin{Verbatim}[frame=single,fontsize=\footnotesize]
x: ..., 1, ?, 3, 5, 7, 4, 6, 8, ...
              i     j        k        t = 2
\end{Verbatim}
and then becomes
\begin{Verbatim}[frame=single,fontsize=\footnotesize]
x: ..., 1, 2, 3, 5, 7, 4, 6, 8, ...
              i     j        k       
\end{Verbatim}
Another strategy is to swap the first elements of the chunks like this:
\begin{Verbatim}[frame=single,fontsize=\footnotesize]
x: ..., 1, 3, 5, 7, 2, 4, 6, 8, ...
           i     j           k
\end{Verbatim}
becomes
\begin{Verbatim}[frame=single,fontsize=\footnotesize]
x: ..., 1, 2, 5, 7, 3, 4, 6, 8, ...
           i     j        k        
\end{Verbatim}
and then this
\begin{Verbatim}[frame=single,fontsize=\footnotesize]
x: ..., 1, 2, 5, 7, 3, 4, 6, 8, ...
              i  j           k       
\end{Verbatim}
This avoids the shift.
However moving the first element of the left chunk 
might result in the right chunk being unsorted.
For instance:
\begin{Verbatim}[frame=single,fontsize=\footnotesize]
x: ..., 1, 3, 5, 7, 2, 2, 6, 8, ...
           i     j           k
\end{Verbatim}
becomes
\begin{Verbatim}[frame=single,fontsize=\footnotesize]
x: ..., 1, 2, 5, 7, 3, 2, 6, 8, ...
           i     j        k        
\end{Verbatim}
and then this
\begin{Verbatim}[frame=single,fontsize=\footnotesize]
x: ..., 1, 2, 5, 7, 3, 2, 6, 8, ...
              i  j           k       
\end{Verbatim}
Therefore in this case, one has to \lq\lq move past
smaller values'' in the right chunk:
\begin{Verbatim}[frame=single,fontsize=\footnotesize]
x: ..., 1, 2, 5, 7, 3, 2, 6, 8, ...
              i  j           k       
\end{Verbatim}
becomes
\begin{Verbatim}[frame=single,fontsize=\footnotesize]
x: ..., 1, 2, 5, 7, 2, 3, 6, 8, ...
              i  j           k       
\end{Verbatim}
i.e., $3$ moves past the 2.

Write this function, test it, and compute the runtime.
\qed
\end{ex}

\begin{ex}
In the two version of merge above, 
recall that there's one version that uses \lq\lq shift''
and another uses \lq\lq swap-and-move-past''.
Combine both into one merge function choosing the best
strategy based on the different scenarios.
\qed
\end{ex}

All the above does not required an extra array.
Therefore the space requirement is $O(1)$.
However they require moving values around so that 
the runtime is longer.

\begin{ex} (Merge 3)
If we allow the use of an extra array $t$ of size about $n/2$, we
can do this:
first copy the left array into $t$.
The the space used by left can be used for merging.
Besides the extra space requirement of $n/2$ (in fact the size is
exactly the size of left), 
we also need to copy the left.
Note that if the original array $x$ has size $n$,
we need to allocate memory for $n/2$ and we can use this same array
for every recursive function call to the mergesort.
\end{ex}
