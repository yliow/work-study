%-*-latex-*-


\section{Average Time}

Recall that we have this is our linear search algorithm:

\begin{Verbatim}[frame=single, fontsize=\footnotesize]
index = -1
for i = 0, 1, 2, ..., n - 1:
    if x[i] is target:
        index = i
        break
\end{Verbatim}

Here's the algorithm with times:
\begin{Verbatim}[frame=single,commandchars=\\\{\}, fontsize=\footnotesize]
                                 time   
         index = -1              t1 
         i = 0                   t2 
LOOP:    if i == n:              t3  
             goto ENDLOOP        t4 
         if x[i] != target:      t5
             goto ENDIF          t6
         index = i               t7 
         goto ENDLOOP            t8
ENDIF:   i = i + 1               t9 
         goto LOOP               t10 
ENDLOOP:
\end{Verbatim}

Recall that the best and worst times are
\begin{align*}
T_b(n) &= O(1) \\
T_w(n) &= O(n)
\end{align*}
I have also computed the average runtime for the case where the 
\verb!target! is in the array at index position 
$k = 0, 1, 2, ..., n - 1$ with equal likelihood.
\[
T_a(n) = O(n)
\]
Now I'm going to change the average runtime scenario
a little and compute a
\textit{new} average runtime.

I'm now going to assume the besides the \verb!target! appear
at index position $0, 1, 2, ..., n - 1$,
I also want to include the case where \verb!target! is \textit{not} in the array 
\verb!x! at all.
So altogether, in the new average runtime, we have $n + 1$ cases.
I want to assume that all cases are \textit{equally likely}.
Let's call out new average runtime $T'_a(n)$.

First let me recall that if I write $T_k$ for the runtime for 
the case where \verb!target! is at index position $k$
and $T_n$ is the runtime where the \verb!target! is not in the array
\verb!x!, then they are of the form
\begin{align*}
T_k &= A + Bk, \,\,\,\,\,(k = 0, 1, 2,..., n - 1) \\
T_n &= C + Dn
\end{align*}

Of course you know now that the constants are not that important
when we are computing big-O.
But anyway, referring to an earlier section we have:
\begin{align*}
A &= t_1 + t_2 + t_3 + t_5 + t_6 + t_7 \\
B &= t_3 + t_5 + t_8 + t_9 \\
C &= t_1 + t_2 + t_3 + t_4 \\
D &= t_3 + t_5 + t_9
\end{align*}

Recall that our \textit{old} average runtime is
\[
T_a(n) 
= \frac{1}{n} 
   \left(
     T_0 + T_1 + \cdots + T_{n - 1} 
   \right) 
= O(n)
\]
This averages over $n$ cases.

For our new average runtime function we have
\[
T'_a(n) 
= \frac{1}{n + 1} 
   \left(
     T_0 + T_1 + \cdots + T_{n - 1} + T_n 
   \right)\\
\]
Note that we're now averaging over $n + 1$ cases.
Now I will compute the big-O of our new average runtime:
\begin{align*}
T'_a(n) 
&= \frac{1}{n + 1}
   \left(
     (A + B\cdot 0) + 
     (A + B\cdot 1) + \cdots
     (A + B\cdot (n - 1)) + 
     C + Dn
   \right)\\
&= \frac{1}{n + 1}
   \left(
     nA + B\cdot (0 + 1 + \cdots + (n - 1)) +
     C + Dn
   \right)\\
&= \frac{1}{n + 1}
   \left(
     nA + B\cdot \frac{n(n-1)}{2} +
     C + Dn
   \right)\\
&= \frac{1}{n + 1}
   \left(
   \left( \frac{B}{2} \right) n^2
   + \left( A - \frac{1}{2}B + D \right) n
   + C
   \right)
\\
&= \left( \frac{B}{2} \right) \frac{n^2}{n + 1}
   + \left( A - \frac{1}{2}B + D \right) \frac{n}{n + 1}
   + C \frac{1}{n + 1}
\end{align*}

Clearly, of the 3 functions of $n$, 
\[
\frac{n^2}{n + 1}, \,\,\,\,\,
\frac{n}{n + 1}, \,\,\,\,\,
\frac{1}{n + 1}
\]
the one that grows the fastest is 
\[
\frac{n^2}{n + 1}, \,\,\,\,\,
\]
Here's a plot of the three functions:
\begin{python}
from latextool_basic import *
p = FunctionPlot(domain='0:10', num_points=50, vars=['n'])
p.add('n**2/(n+1)', color='blue', python=1, pin='above')
p.add('n/(n+1)', color='blue', python=1, pin_message='')
p.add('1/(n+1)', color='blue', python=1, pin_message='')
print(p)
\end{python}

Although at this point I can say 
\[
T'_a(n) = O\left( \frac{n^2}{n + 1}\right)
\]
I prefer to bound $n^2/(n + 1)$ by a simpler (and cleaner) function.
In fact it's really simple:
\[
\frac{n^2}{n + 1} = O(n)
\]
There are several ways to prove this.
Here's the simplest proof:
\[
\frac{n^2}{n + 1} 
\leq 
\frac{n^2}{n + 0} = n
\]
Hence 
\[
T'_a(n) = O(n)
\]

Notice that both average runtimes gives the same big-O:
\begin{align*}
T_a(n) &= O(n) \\ 
T'_a(n) &= O(n) \\ 
\end{align*}

\begin{ex}
Consider $n + 2$ cases of the linear search.
\begin{itemize}
\li \verb!target! is at index position $0$ or $1$ or ... or $n - 1$
\li \verb!target! is larger than all values in array \verb!x!
\li \verb!target! is less than all values in array \verb!x!
\end{itemize}
and they are equally likely.
\begin{itemize}
\item[(a)] Compute the average runtime under the above assumptions.
\item[(b)] Compute the big-O of the average runtime of (a) in the form of
$n^k$.
\end{itemize}
\qed
\end{ex}

\begin{ex}
Here are the same cases as the previous problem.
\begin{itemize}
\li \verb!target! is at index position $0$ or $1$ or ... or $n - 1$
\li \verb!target! is larger than all values in array \verb!x!
\li \verb!target! is less than all values in array \verb!x!
\end{itemize}
After running the linear search for a long time on a particular
system
I realize that there is
a particular value that is searched for very frequently.
So I move this value to index 0.
Specifically, the case where the search for \verb!target! at index 0
is twice as frequent other cases.
\begin{itemize}
\item[(a)] Compute the average runtime under the above assumptions.
\item[(b)] Compute the big-O of the average runtime of (a) in the form 
of $n^k$.
\end{itemize}
\qed
\end{ex}

\begin{ex}
For another system, I realized that the following cases are most
helpful:
\begin{itemize}
\li \verb!target! is at index position $0$ or $1$ or ... or $n - 1$.
The case where \verb!target! is at index 0 is twice as likely
as each of the other cases where \verb!target! is at index $i$
for $i = 1, 2, 3, ..., n - 1$.
\li \verb!target! is not in the array \verb!x!.
This case is 10 times more likely that the case
where \verb!target! is at index 1.
\end{itemize}
\begin{itemize}
\item[(a)] Compute the average runtime under the above assumptions.
\item[(b)] Compute the big-O of the average runtime of (a) in the form 
of $n^k$.
\end{itemize}
\qed
\end{ex}
