%-*-latex-*-
\section{Probability distribution function}

A (finite) discrete 
\defone{probability distribution function} is
a 
real-valued function on a finite set $S$ 
\[
p : S \rightarrow \R
\]
such that 
\begin{axioms}
\item[{[DP-1]}] $0 \leq p(x) \leq 1$ for all $x \in S$
\item[{[DP-2]}] $\sum_{x \in S} p(x) = 1$
\end{axioms}
$S$ is a
\defterm{sample space}\index{sample space}\tinysidebar{sample space \\ outcomes}.
The elements of $S$ are called
\defterm{outcomes}\index{outcomes},
Because of [DP-1], I can change the codmain of $p$:
\[
p : S \rightarrow [0,1]
\]
A subset of $S$ is called an \defone{event}.
If $A$ is a subset of $S$, I'll write
\[
p(A) = \sum_{x \in A} p(x)
\]
[DP-2] is the same as saying
\[
p(S) = 1
\]

Here are some basic facts about pdfs.


%-*-latex-*-

\begin{ex} 
  \label{ex:prob-00}
  \tinysidebar{\debug{exercises/{disc-prob-28/question.tex}}}

  \solutionlink{sol:prob-00}
  \qed
\end{ex} 
\begin{python0}
from solutions import *
add(label="ex:prob-00",
    srcfilename='exercises/discrete-probability/prob-00/answer.tex') 
\end{python0}



\begin{defn}
A probability distribution function $p : S \rightarrow [0,1]$ 
is said to be \defone{uniform} if
\[
p(x) = \frac{1}{|S|}
\]
for each $x \in S$.
$|S|$, the size of $S$, denotes the numbers of distinct elements in $S$.
\end{defn}

\begin{eg}
Suppose that I've rigged my coin so that the chance of getting
a head is twice of getting the tail.
This means that the probability function
\[
p : S \rightarrow [0,1]
\]
is
\[
p(\HEAD) = \frac{2}{3}, \,\,\,\,\,
p(\TAIL) = \frac{1}{3} \,\,\,\,\,
\]
Of course if the coin is a fair coin I would have
\[
p(\HEAD) = \frac{1}{2} = p(\TAIL)
\]
\qed
\end{eg}


%-*-latex-*-

\begin{ex} 
  \label{ex:prob-00}
  \tinysidebar{\debug{exercises/{disc-prob-28/question.tex}}}

  \solutionlink{sol:prob-00}
  \qed
\end{ex} 
\begin{python0}
from solutions import *
add(label="ex:prob-00",
    srcfilename='exercises/discrete-probability/prob-00/answer.tex') 
\end{python0}



\begin{eg}
Suppose I have a loaded die such that getting a \lq\lq 1'' 
is 10 times more likely 
than the others and the others are equally likely.
Let $p$ be the probability function of rolling this die.
Then
\begin{align*}
1 
&= p(\ONE) + p(\TWO) + \cdots + p(\SIX) \\
&= p(\ONE) + 5 \cdot p(\TWO) \\
&= p(\ONE) + 5 \cdot \frac{1}{10} p(\ONE) \\
&= \frac{3}{2} \cdot p(\ONE) 
\end{align*}
So 
\[
p(\ONE) = \frac{2}{3}
\]
and 
\[
p(\TWO) = p(\THREE) = ... = p(\SIX) = \frac{2}{30}
\]
And of course
\[
S = \{\ONE, \TWO, \THREE, \FOUR, \FIVE, \SIX\}
\]

Go ahead and write a program to simulate tossing my loaded die.
Run it for a large number of experiments and check that you get the
expected results. 
\begin{Verbatim}[fontsize=\small, frame=single]
import random; random.seed()

ONE = "ONE"
TWO = "TWO"
THREE = "THREE"
FOUR = "FOUR"
FIVE = "FIVE"
SIX = "SIX"

# write a probability function for this experiment

def roll_die():
    r = random.randrange(30)
    if r < 20: return ONE
    elif r < 22: return TWO
    elif r < 24: return THREE
    elif r < 26: return FOUR
    elif r < 28: return FIVE
    else: return SIX

NUM_EXPERIMENTS = 20
for i in range(NUM_EXPERIMENTS):
    print("experiment", i, "... outcome:", roll_die())
\end{Verbatim}
\end{eg}


The following function can be used to produce the different
experiment functions (you should type it up and run it ... if you're
too lazy to do that you can also grab the code from
our class web site ... look for \lq\lq probability library"):
\begin{Verbatim}[frame=single, fontsize=\small]
def build_experiment(xs):
    total = sum([b for (a,b) in xs])
    accumulate = xs[0]
    for (a,b) in xs[1:]:
        accumulate.append((a, accumulate[-1][1] + b)
    def experiment():
        r = random.randrange(total)
        for (a,b) in accumulate:
            if r < b: return a
        raise ValueError("r:%s greater than total:%s" %\
                         (r, total))
    return experiment

HEAD = "HEAD"
TAIL = "TAIL"
toss_coin = build_experiment([(HEAD, 2),
                              (TAIL, 1),
                             ])
ONE = "ONE"
TWO = "TWO"
THREE = "THREE"
FOUR = "FOUR"
FIVE = "FIVE"
SIX = "SIX"
roll_die = build_experiment([(ONE, 20),
                             (TWO, 2),
                             (THREE, 2),
                             (FOUR, 2),
                             (FIVE, 2),
                             (SIX, 2),
                            ])
\end{Verbatim}

%-*-latex-*-

\begin{ex} 
  \label{ex:prob-00}
  \tinysidebar{\debug{exercises/{disc-prob-28/question.tex}}}

  \solutionlink{sol:prob-00}
  \qed
\end{ex} 
\begin{python0}
from solutions import *
add(label="ex:prob-00",
    srcfilename='exercises/discrete-probability/prob-00/answer.tex') 
\end{python0}

%-*-latex-*-

\begin{ex} 
  \label{ex:prob-00}
  \tinysidebar{\debug{exercises/{disc-prob-28/question.tex}}}

  \solutionlink{sol:prob-00}
  \qed
\end{ex} 
\begin{python0}
from solutions import *
add(label="ex:prob-00",
    srcfilename='exercises/discrete-probability/prob-00/answer.tex') 
\end{python0}


Now what is asked is actually related to another
experiment: 
\lq\lq Attend 3 classes in a row''.
(Later you'll see how to view this as a \lq\lq product of
experiments''.)
There are two ways to compute an approximation of the desired
probability.

FIRST METHOD:
You should use the first experiment to generate a sequence of
$\mathsc{Q}$'s and
$\mathsc{N}$'s:
\[
\mathsc{Q},
\mathsc{Q},
\mathsc{N},
\mathsc{Q},
\mathsc{Q},
\mathsc{Q},
\mathsc{Q},
\mathsc{Q},
\mathsc{Q},
\mathsc{N},
\mathsc{Q},
\mathsc{Q},
...
\]
satisfying the given probability that
$p(\mathsc{Q}) = 0.8$ and
$p(\mathsc{N}) = 0.2$.
Next you look at all the consecutive triples.
For instance the above gives
\[
\mathsc{QQN},
\mathsc{QNQ},
\mathsc{NQQ},
\mathsc{QQQ},
\mathsc{QQQ},
\mathsc{QQQ},
\mathsc{QQQ},
\mathsc{QQQ},
\mathsc{QQN},
\mathsc{QNQ},
\mathsc{NQQ}, ...
\]
and then compute the approximate probability of seeing
$\mathsc{QQQ}$ in 
this sampling.
Of course you need to have a reasonbly huge sample.
You can either do this by hand or write a program to simulate the
experiment.

SECOND METHOD:
If you have a function for the first experiment, say
\verb!attend_class! which gives you $\mathsc{Q}$ or
$\mathsc{N}$, then you should have a
function say \verb!attend_3_classes! that uses \verb!attend_class!:
\begin{Verbatim}[frame=single]
Q = 'Q'
N = 'N'

def attend_3_classes():
    return [attend_class(), 
            attend_class(), 
            attend_class(),
           ]
\end{Verbatim}
Of course you can now generate samples and count:
\begin{Verbatim}[frame=single]
MAX = 100000
count = 0
for i in range(MAX):
    if attend_3_classes() == [Q, Q, Q]: 
        count += 1
print(float(count) / MAX)
\end{Verbatim}

%-*-latex-*-

\begin{ex} 
  \label{ex:prob-00}
  \tinysidebar{\debug{exercises/{disc-prob-28/question.tex}}}

  \solutionlink{sol:prob-00}
  \qed
\end{ex} 
\begin{python0}
from solutions import *
add(label="ex:prob-00",
    srcfilename='exercises/discrete-probability/prob-00/answer.tex') 
\end{python0}

