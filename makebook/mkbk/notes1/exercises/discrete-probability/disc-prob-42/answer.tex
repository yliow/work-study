\tinysidebar{\debug{exercises/discrete-probability/{disc-prob-42/answer.tex}}}

(a)
Let $X_{ij} = 1$ if ball $i$ falls into bucket $j$; otherwise it's 0.
Then $X_j = \sum_{i} X_{ij}$
is the number of balls in bucket $j$.
Therefore the number of balls in bucket $j$ is
$\E[X_j] = \sum_{i} \E[X_{ij}] = \sum_{i} (1/n)$ $=$ $m/n$.  

(b)
Let $X_i = 1$ if bucket $i$ is empty.
Let $X_{ij} = 1$ if ball $j$ does not fall into bucket $i$.
Therefore
\[
  X_i = \prod_{j} X_{ij}
\]
Let $X = \sum_i {X_i}$. This is the number of buckets which are empty.
and hence
\begin{align*}
  \E[X]
  &= \E \left[ \sum_{i} X_i \right]
    \\
  &= \E \left[ \sum_i \prod_{j} X_{ij} \right]
    \\
  &= \sum_i \E \left[  \prod_{j} X_{ij} \right]
    \\
  &= \sum_i \prod_{j} \E \left[  X_{ij} \right]
    \\
  &= \sum_i \prod_{j} (1 - 1/n)
    \\
  &= \sum_i (1 - 1/n)^m
    \\
  &= n (1 - 1/n)^m
\end{align*}
(I used the fact that
$\E[\prod_j X_{ij}] = \prod_j \E[X_{ij}]$ if
$X_{ij}$ (all $j$)
are independent.)

By the way
\[
  \lim_{n \rightarrow \infty} (1 + 1/n)^n = e
\]
Therefore
\[
  \lim_{n \rightarrow \infty} (1 - 1/n)^n = e^{-1}
\]
Therefore for large $n$ (number of buckets)),
$n (1 - 1/n)^m = ne^{-m/n}$.
And if $m = n$, 
$n (1 - 1/n)^m = ne^{-m/n} = n/e$
which means that $1/e$ of the buckets are empty.
Note that $1/e$ is independent of $n$.

% p7.py

\textsc{Alternative}.
Let $X_i = 1$ if the bucket $i$ is empty.
Then
\[
  \E[X_i] = \left( \frac{n-1}{n} \right)^m = \left( 1 - \frac{1}{n} \right)^m
\]
Let $X = \sum_{i}X_i$.
Hence
\[
  \E[X] = \sum_{i} \E[X_i] = n \left( \frac{n-1}{n} \right)^m
\]
which is the same as above.


\begin{comment}
(c)
Let $X_i = 1$ if is has more than one ball.
Then
\[
  \E[X_i] = 1 - 1/n - 
\]
\end{comment}

    
