%-*-latex-*-
\section{Quicksort: PLTR partition method}

Now that I've given you one partition strategy, let's talk about
sorting inplace, 
i.e., without creating extra arrays.
The idea is just like mergesort.
In fact when you look at the first partition strategy I've shown,
you notice that I did not use any extra arrays.
Therefore if you have an inplace partition strategy, you apply it to
the whole array, and recursively on the \verb!left! and \verb!right! piles.
For instance if you have an array of integers values:
\begin{python}
from latextool_basic import *
p = Plot()
s = chunkedarray(cellwidth=0.8, 
                 cellheight=0.8,
                 arr=[20*['']],
                )
p.add(s)
print(p)
\end{python}
and you apply a partition strategy to get this, say with a pivot of 
\verb!42!:
\begin{python}
from latextool_basic import *
p = Plot()
s = chunkedarray(cellwidth=0.8,
                 cellheight=0.8,
                 arr=[7*[''], ['42'], 12*['']],
                 chunklabels=['left','','right'],
)
p.add(s)
print(p)
\end{python}

You of course also apply the partition strategy to the \verb!left!,
say with a pivot of \verb!23!:
\begin{python}
from latextool_basic import *
p = Plot()
s = chunkedarray(cellwidth=0.8,
                   cellheight=0.8,
                   arr=[3*[''],['23'],3*[''], ['42'] + 12*['']],
                   chunklabels=['left','','right'],
)
p.add(s)
print(p)
\end{python}
And you keep doing it until you reach base cases.
Etc.
Get it?
Of course if you apply an inplace partition strategy say to the \verb!right!
pile of the above:
it will have its own \verb!left! and \verb!right! of course, say:
\begin{python}
from latextool_basic import *
p = Plot()
s = chunkedarray(cellwidth=0.8,
                   cellheight=0.8,
                   arr=[7*[''] + ['42'], 4*[''], ['83'],7*['']],
                   chunklabels=['','left','','right'],
)
p.add(s)
print(p)
\end{python}

Of course this means that, in general, if we have an 
inplace partition strategy for an \textit{array}, we need to modify it so that
the partition operate on a \textit{subarray}.


Consider this array:

\begin{python}
from latextool_basic import *
p = Plot()
s = chunkedarray(cellwidth=1, 
                   cellheight=0.5,
                   arr=[[3,2,1,3,4,9,6,7,0,2,4,8]],
)
p.add(s)
print(p)
\end{python}

y
The general idea is that first we move the pivot to the front of the
array.
Next, excluding the pivot, we continually maintain three \lq\lq sections''
of the array to be partitioned.
The first section, which I call \verb!left! contains value
$\leq$ the pivot.
This is followed by a section of the array I call \verb!TODO!
which are values not yet processed.
The final section of the array is made up of values
which are $>$ the value of the pivot
I continually analyze the leftmost value (say \verb!v!) from the
\verb!TODO! section.
There are two possibilities:

If \verb!v! is $\leq$ the pivot, then \verb!v! joins \verb!left! --
the size of the \verb!left! section grows by 1.
In this case, \verb!v! joins the rightside of \verb!left!.
I will call this method PLTR (memory aid: pivot-left-todo-right.)

If \verb!v! is $>$ the pivot, then \verb!v! joins \verb!right! --
the size of \verb!right! grows by 1.
You will that \verb!v! joins the leftside of \verb!right!:
this is done by swapping \verb!v! with an appropriate value.

At some point, the \verb!TODO! be empty.
At that point, we have the pivot following by \verb!left!,
the empty \verb!TODO!, and \verb!right!.
We then move the pivot by swapping it with the rightmost
value of \verb!left! or, if \verb!left! is empty,
\verb!v! becomes the leftmost value of \verb!right!.

I will assume for simplicity that the we have chosen the first value,
i.e, \verb!3! to be the pivot.

Let's see the partition strategy in action.
Here we go:

{\small
\begin{python}
from latextool_basic import *
p = Plot()

y = 0
def xyz(xs):
    global y
    y -= 1.8
    t = chunkedarray(x=0, y=y,
                        cellwidth=1, 
                        cellheight=0.5,
                        arr=xs,
                        chunklabels=['','left','TODO','right'])
    return t

def xyz2(xs):
    global y
    y -= 1.1
    t = chunkedarray(x=0, y=y,
                        cellwidth=1, 
                        cellheight=0.5,
                        arr=xs)
    return t

s = xyz([[3],[],[2,1,3,4,9,6,7,0,2,4,8],[]]); p.add(s)
s = xyz([[3],[2],[1,3,4,9,6,7,0,2,4,8],[]]); p.add(s)
s = xyz([[3],[2,1],[3,4,9,6,7,0,2,4,8],[]]); p.add(s)
s = xyz([[3],[2,1,3],[4,9,6,7,0,2,4,8],[]]); p.add(s)
s = xyz([[3],[2,1,3],[8,9,6,7,0,2,4],[4]]); p.add(s)
s = xyz([[3],[2,1,3],[9,6,7,0,2,4],[8,4]]); p.add(s)
s = xyz2([[3],[2,1,3],[6,7,0,2,4],[9,8,4]]); p.add(s)
s = xyz2([[3],[2,1,3],[7,0,2,4],[6,9,8,4]]); p.add(s)
s = xyz2([[3],[2,1,3],[0,2,4],[7,6,9,8,4]]); p.add(s)
s = xyz2([[3],[2,1,3,0],[2,4],[7,6,9,8,4]]); p.add(s)
s = xyz2([[3],[2,1,3,0,2],[4],[7,6,9,8,4]]); p.add(s)
s = xyz2([[3],[2,1,3,0,2],[],[4,7,6,9,8,4]]); p.add(s)

print(p)
\end{python}
}

and swapping the pivot \verb!3! with the rightmost value in the \verb!left!
pile:
\begin{center}
(1)
\end{center}
{\small
\begin{python}
from latextool_basic import *
p = Plot()
s = chunkedarray(cellwidth=1,
                   cellheight=0.5,
                   arr=[[2,1,3,0,2],[3], [4,7,6,9,8,4],],
                   chunklabels=['left','pivot','right'],
)
p.add(s)
print(p)
\end{python}
}

Now I perform partitioning on the \textit{above} \verb!left! pile,
choosing the first value of this section to be the pivot:
{\small
\begin{python}
from latextool_basic import *
p = Plot()

y = 0
def xyz(xs):
    global y
    y -= 1.8
    t = chunkedarray(x=0, y=y,
                        cellwidth=1, 
                        cellheight=0.5,
                        arr=xs,
                        chunklabels=['','left','TODO','right'])
    return t

def xyz2(xs):
    global y
    y -= 1.1
    t = chunkedarray(x=0, y=y,
                        cellwidth=1, 
                        cellheight=0.5,
                        arr=xs)
    return t

s = xyz([[2],[],[1,3,0,2],[],[3,4,7,6,9,8,4]]); p.add(s)
s = xyz2([[2],[1],[3,0,2],[],[3,4,7,6,9,8,4]]); p.add(s)
s = xyz2([[2],[1],[2,0],[3],[3,4,7,6,9,8,4]]); p.add(s)
s = xyz2([[2],[1,2],[0],[3],[3,4,7,6,9,8,4]]); p.add(s)
s = xyz2([[2],[1,2,0],[],[3],[3,4,7,6,9,8,4]]); p.add(s)
print(p)
\end{python}
}

and swapping in the pivot to its right position:
\begin{center}
(2)
\end{center}
{\small
\begin{python}
from latextool_basic import *
p = Plot()
s = chunkedarray(cellwidth=1, 
                   cellheight=0.5,
                   arr=[[0,1,2],[2],[3],[3,4,7,6,9,8,4]],
                   chunklabels=['left',('pivot',1.2),'right']
)
p.add(s)
print(p)
\end{python}
}
For this partitioning, 
the \verb!left! pile is \verb![0,1,2]!, the pivot is \verb!2!, and
the \verb!right! pile is \verb![3]!.

Now I partition the above \verb!left! pile using the first element
\verb!0! as pivot.
{\small
\begin{python}
from latextool_basic import *
p = Plot()

y = 0
def xyz(xs):
    global y
    y -= 1.8
    t = chunkedarray(x=0, y=y,
                        cellwidth=1, 
                        cellheight=0.5,
                        arr=xs,
                        chunklabels=['','left',('TODO', 1.2),'right'])
    return t

def xyz2(xs):
    global y
    y -= 1.1
    t = chunkedarray(x=0, y=y,
                        cellwidth=1, 
                        cellheight=0.5,
                        arr=xs)
    return t

s = xyz([[0],[],[1,2],[],[2,3,3,4,7,6,9,8,4]]); p.add(s)
s = xyz2([[0],[],[2],[1],[2,3,3,4,7,6,9,8,4]]); p.add(s)
s = xyz2([[0],[],[2,1],[2,3,3,4,7,6,9,8,4]]); p.add(s)
print(p)
\end{python}
}
Swapping with the \verb!left! pile is not necessary since the 
\verb!left! pile is empty.
We get
\begin{center}
(3)
\end{center}
\begin{python}
from latextool_basic import *
p = Plot()
s = chunkedarray(cellwidth=1,
                   cellheight=0.5,
                   arr=[[],[0],[2,1],[2,3,3,4,7,6,9,8,4]],
)
p.add(s)
print(p)
\end{python}
Note that there the \verb!left! pile to the left of the pivot \verb!0!
is empty -- there's nothing to do!
We have arrived at a base case!

We continue to execute the partitioning on the \verb!right! pile 
of the pivot \verb!0!,
i.e., on \verb![2,1]!.
The first value \verb!2! is the pivot in this case.
{\small
\begin{python}
from latextool_basic import *
p = Plot()

y = 0
def xyz(xs):
    global y
    y -= 1.8
    t = chunkedarray(x=0, y=y,
                        cellwidth=1, 
                        cellheight=0.5,
                        arr=xs,
                        chunklabels=['','','left',('TODO', 1.2),'right'])
    return t

def xyz2(xs):
    global y
    y -= 1.1
    t = chunkedarray(x=0, y=y,
                        cellwidth=1.0, 
                        cellheight=0.5,
                        arr=xs)
    return t

s = xyz([[0],[2],[],[1],[],[2,3,3,4,7,6,9,8,4]]); p.add(s)
s = xyz2([[0],[2],[1],[],[],[2,3,3,4,7,6,9,8,4]]); p.add(s)
print(p)
\end{python}
}


Swapping the pivot \verb!2! 
with the rightmost value in the \verb!left! pile,
i.e., \verb!1!, we get
\begin{center}
(4)
\end{center}
\begin{python}
from latextool_basic import *
p = Plot()
s = chunkedarray(cellwidth=1, 
                   cellheight=0.5,
                   arr=[[0],[1],[2],[],[],[2,3,3,4,7,6,9,8,4]],
)
p.add(s)
print(p)
\end{python}
I'm now done partitioning the \verb!left! and \verb!right! pile
from (3).
Note that (3) was the partition of the \verb!left! pile from (2).
So I partition the \verb!right! pile at (2).
But this \verb!right! pile is \verb![3]! which has size 1, i.e., 
this is the base case, so nothing is done.
So I'm doing to return from (2) to (1).
Recall that (2) is partitioning the \verb!left! of (1).
So now I need to partition the right of (1).
At (1), the array now looks like this, with the \verb!left! pile 
completely sorted:
\begin{center}
(1)
\end{center}
\begin{python}
from latextool_basic import *
p = Plot()
s = chunkedarray(cellwidth=1,
                   cellheight=0.5,
                   arr=[[0,1,2,2,3],[3], [4,7,6,9,8,4],],
                   chunklabels=['left','pivot','right'],
)
p.add(s)
print(p)
\end{python}
I'll leave it to you to finish the quicksort.

At this point, it's clear that this inplace
strategy (PLTR) works, executing a partition strategy on part of the array.
Of course to write a proper algorithm, we will need to specify
\textit{which} part of the array a recursive call of the quicksort
is support to work on.
That's easy. For our recursive function call,
we just need to carry two index variables
\verb!start! and \verb!end! that tells which the starting
index and ending index that particular quicksort invocation
is suppose to work on.
So during an invocation of quicksort, the picture is something like this:

\begin{python}
from quicksort_init import *
p = Plot()
from quicksort_init import *
s = chunkedarray(cellwidth=0.4, 
                   cellheight=0.4,
                   arr=[8*[''], ['p'],3*[''],6*[''],4*[''],10*['']],
                   chunklabels=['', '', 'left', 'TODO', 'right'],
                   celllabels = [('start', 8, -2),
                                 ('end',   21, -2),
                                 ('thead', 12, -1),
                                 ('ttail', 17, -1),
                                ]
)
p.add(s)
print(p)
\end{python}
(The \verb!p! is the pivot value -- remember that we move the pivot to
the front.)
To describe the beginning and end of the \verb!TODO!, I have
two index variables \verb!thead! and \verb!ttail!
(memory aid:
\verb!thead! = \verb!TODO! head and 
\verb!ttail! = \verb!TODO! tail.)



Here's the algorithm for the PLTR partition
that works on the section of the array 
beginning with index \verb!start! and ending at index
\verb!end!:
\begin{console}
FUNCTION partition(x, start, end):

    choose pivot_index # the index to the pivot
    pivot = x[pivot_index]
    swap x[start] and x[pivot_index]
    thead = start + 1
    ttail = end
    while ttail <= thead:
        if x[thead] <= pivot:
            thead = thead + 1
        else:
            if thead is not ttail:
               swap x[thead] and x[ttail]
            ttail = ttail - 1
    
    if thead > start + 1:
        swap x[start] and x[thead - 1]

    return thead - 1
\end{console}
(Note that we assume \verb!start < end!.
If \verb!start == end!, then there's only one value and
there's nothing to partition.
If \verb!start > end!, that corresponds to the case where
the subarray is empty -- again nothing to partition.)
The value returned is the index of the pivot value after the espivot
value is moved from the front to its final place in this partition.

Note that at the end of the algorithm we have the following:
\begin{tightlist}
  \li \verb!x[start .. thead - 1]! have values $\leq$ \verb!pivot! 
  \li \verb!x[ttail + 1 .. end]! have values $>$ \verb!pivot!
\end{tightlist}

Here's the general (inplace) quicksort algorithm:
\begin{console}
FUNCTION quicksort(x, start, end):

    if start < end:
        pivot_index = partition(x, start, end)
        quicksort(x, start, pivot_index - 1)
        quicksort(x, pivot_index + 1, end)
\end{console}

(Of course to sort a whole array, call the function
with \verb!start = 0! and \verb!end = n - 1! where \verb!n!
is the size of the array.)

Of course the most important thing is the \verb!partition!
function which include the selection of the pivot and the
organization of \verb!left! and \verb!right!.





\newpage
\begin{ex}
\begin{tightlist}
\item Using the partitioning strategy in this section,
execute one single partitioning on the following array
from index \verb!start = 3! to index \verb!end = 8!
choosing the value at index 4 as pivot (i.e., the value \verb!7!).

\begin{python}
from latextool_basic import *
p = Plot()
p += Array2d(0,0,[[3,1,5,2,7,8,4,6,9,0]],width=0.7,height=0.7)
print(p)
\end{python}

Show the state of your array after moving the pivot, at the end of
processing each value in the \verb!TODO! section of the array,
and finally after moving the pivot to the right place.

\item For each of the step from above (except for the first and last),
write down the value of \verb!thead! and \verb!ttail!.
What is the final index of the pivot (i.e., the value returned
by the partition method in this section)? 
\end{tightlist}
\end{ex}



\newpage
\begin{ex}
\begin{tightlist}
\item Using the partitioning strategy in this section,
execute one single partitioning on the following array
from index \verb!start = 2! to index \verb!end = 8!
choosing the value at index 4 (i.e., \verb!4!) as pivot.

\begin{python}
from latextool_basic import *
p = Plot()
p += Array2d(0,0,[[0,1,2,3,4,5,6,7,8,9]],width=0.7,height=0.7)
print(p)
\end{python}

Show the state of your array after moving the pivot, at the end of
processing each value in the \verb!TODO! section of the array,
and after moving the pivot to the right place.

\item For each of the step from above (except for the first and last),
write down the value of \verb!thead! and \verb!ttail!.
What is the final index of the pivot (i.e., the value returned
by the partition method in this section)? 

\end{tightlist}
\qed
\end{ex}


\newpage
\begin{ex}
\begin{tightlist}
\item Using the partitioning strategy in this section,
execute one single partitioning on the following array
from index \verb!start = 2! to index \verb!end = 8!
choosing the value at index 4 as pivot.

\begin{python}
from latextool_basic import *
p = Plot()
p += Array2d(0,0,[[9,8,7,6,5,4,3,2,1,0]],width=0.7,height=0.7)
print(p)
\end{python}

Show the state of your array after moving the pivot, at the end of
processing each value in the \verb!TODO! section of the array,
and after moving the pivot to the right place.

\item For each of the step from above (except for the first and last),
write down the value of \verb!thead! and \verb!ttail!.
What is the final index of the pivot (i.e., the value returned
by the partition method in this section)? 

\end{tightlist}
\qed
\end{ex}

\newpage
\begin{ex}
\begin{tightlist}
\item Using the partitioning strategy in this section,
execute one single partitioning on the following array
from index \verb!start = 2! to index \verb!end = 8!
choosing the value at index 2 as pivot.

\begin{python}
from latextool_basic import *
p = Plot()
p += Array2d(0,0,[[7,8,9,6,0,3,4,2,1,5]],width=0.7,height=0.7)
print(p)
\end{python}

Show the state of your array after moving the pivot, at the end of
processing each value in the \verb!TODO! section of the array,
and after moving the pivot to the right place.

\item For each of the step from (a) (except for the first and last),
write down the value of \verb!thead! and \verb!ttail!.
What is the final index of the pivot (i.e., the value returned
by the partition method in this section)? 

\end{tightlist}
\qed
\end{ex}

\newpage
\begin{ex}
\begin{tightlist}
\item Using the partitioning strategy in this section,
execute one single partitioning on the following array
from index \verb!start = 2! to index \verb!end = 8!
choosing the value at index 4 as pivot.

\begin{python}
from latextool_basic import *
p = Plot()
p += Array2d(0,0,[[7,8,9,6,0,3,4,2,1,5]],width=0.7,height=0.7)
print(p)
\end{python}

Show the state of your array after moving the pivot, at the end of
processing each value in the \verb!TODO! section of the array,
and after moving the pivot to the right place.

\item For each of the step from (a) (except for the first and last),
write down the value of \verb!thead! and \verb!ttail!.
What is the final index of the pivot (i.e., the value returned
by the partition method in this section)? 

\end{tightlist}
\qed
\end{ex}


\newpage
\begin{ex}
\begin{tightlist}
\item Using the partitioning strategy in this section,
execute one single partitioning on the following array
from index \verb!start = 0! to index \verb!end = 2!
choosing the value at index 0 as pivot.

\begin{python}
from latextool_basic import *
p = Plot()
p += Array2d(0,0,[[0,9,9]],width=0.7,height=0.7)
print(p)
\end{python}

\item Is quicksort (using the partition strategy of this section)
a stable sorting algorithm?
\qed
\end{tightlist}
\end{ex}


\newpage
\begin{ex}
Describe arrays and pivot values in the arrays that will become
sorted when you execute one pass of the partition strategy of this section.
\qed
\end{ex}


\newpage
\begin{ex}
Describe sorted arrays and pivot values in the arrays that will become
unsorted when you execute one pass of the partition strategy of this section.
\qed
\end{ex}

