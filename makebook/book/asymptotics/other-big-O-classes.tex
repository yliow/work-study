%-*-latex-*-
\section{Other big-O Classes}

So far we have big-O classes for polynomials, logarithms, and 
exponentials (as well as their products).

There are other big-O classes for functions like
\[
n^n, \,\,\,\,\, n!
\]
Clearly
\[
n! \leq n^n
\]
just by looking
\[
1 \cdot 2 \cdot 3 \cdot \cdots \cdot n
\leq
n \cdot n \cdot n \cdot \cdots \cdot n
\]
So clearly
\[
O(n!) \subseteq O(n^n)
\]
i.e.,
\[
f(n) = O(n!) \implies f(n) = O(n^n)
\]

Recall from CISS240 that $n!$ is the total number of
permutations of $n$ distinct symbols.
(See chapter on recursion.)
Therefore this $O(n!)$ is the runtime of an algorithm
that run through all permutations of $n$ symbols
and perform some constant runtime computation on each permutation.
We'll see examples in a later chapter.


\begin{ex} 
  \label{ex:factorial}
  \tinysidebar{\debug{exercises/{disc-prob-28/question.tex}}}

  \solutionlink{sol:factorial}
  \qed
\end{ex} 
\begin{python0}
from solutions import *
add(label="ex:factorial",
    srcfilename='exercises/factorial/answer.tex') 
\end{python0}


[TODO: subexponential, superexponential, etc.]
