%-*-latex-*-
\sectionthree{Master theorem}
\begin{python0}
from solutions import *; clear()
\end{python0}

So far I have been talking about recurrences like
\[
T(n) = c_1 T(n-1) + c_2 T(n-2) + f(n)
\]
where $c_i$ are constants and $f(n)$ is an expressions in $n$.
More generally,
\[
T(n) = c_1(n) T(n-1) + c_2(n) T(n-2) + f(n)
\]
where $c_i(n)$ are also expressions in $n$.
Then there are the divide-and-conquer type recurrence relations.
Here's one:
\[
T(n) = T(n/2) + A
\]
and here's another
\[
T(n) = 2T(n/2) + An + B
\]
There are also more complex ones like
\[
T(n) = f(n) + \frac{1}{n} \sum_{i=0}^{n-1} T(i)
\]

\begin{comment}
\newpage
\begin{ex}
  \begin{tightlist}
    \item
    Let
    \[
    T(n) = T(\floor{n/2}) + 1
    \]
    and $T(n) = 1$ if $n \leq 1$.
    Compute (by hand) and
    write down the table for $T(n)$ for $n = 0, 1, 2, ..., 9$.
    \item
    Let
    \[
    T'(n) = T'(\ceiling{n/2}) + 1
    \]
    and $T'(n) = 1$ if $n \leq 1$.
    Write down the table for $T'(n)$ for $n = 0, 1, 2, ..., 9$.
  \end{tightlist}
  Compute more values for $T(n)$ and $T'(n)$ ... as in \textit{lots} of $n$'s
  and plot them.
  Compare $T(n)$ and $T'(n)$ when $n$ is large.
  Can you find good approximation for $T(n)$ and $T'(n)$?
  (Here good mean functions such as $n^\alpha$, $\lg^\alpha n$,
  $n^\alpha \lg^\beta n$, $c^n$, etc.
  You'll see that asymptotically (when $n$ is large),
  $T(n)$ and $T'(n)$ are about the same.
  \qed
\end{ex}


\newpage
\begin{ex}
  \begin{tightlist}
    \item
    Let
    \[
    T(n) = 2 T(\floor{n/2}) + 1
    \]
    and $T(n) = 1$ if $n \leq 1$.
    Compute (by hand) and
    write down the table for $T(n)$ for $n = 0, 1, 2, ..., 9$.
    \item
    Let
    \[
    T'(n) = T'(\floor{n/2}) + T'(\ceiling{n/2}) + 1
    \]
    and $T'(n) = 1$ if $n \leq 1$.
    Write down the table for $T'(n)$ for $n = 0, 1, 2, ..., 9$.
  \end{tightlist}
  Plot more values for $T(n)$ and $T'(n)$ ... as in \textit{lots} of $n$'s.
  Compare $T(n)$ and $T'(n)$ when $n$ is large.
  Can you find good approximation for $T(n)$ and $T'(n)$?
  (Here good mean functions such as $n^\alpha$, $\lg^\alpha n$,
  $n^\alpha \lg^\beta n$, $c^n$, etc.
  You'll see that asymptotically (when $n$ is large),
  $T(n)$ and $T'(n)$ are about the same.
  \qed
\end{ex}


%{\small
%\begin{python}
%from latextool_basic import *
%s = r'''
%def a(n):
%    if n in [0, 1]: return 1
%    else:
%        return 2 * a(n/2) + 1
%def b(n):
%    if n in [0, 1]: return 1
%    else:
%        return 2 * b(round(n/2.0)) + 1
%        
%for n in range(0, 9):
%    print(n, a(n), b(n))
%
%for n in range(100000, 100011):
%    print(n, float(a(n))/float(b(n)))
%'''.strip()
%print(console(s))
%f = open('tmp1243525234.py', 'w').write(s)
%print(shell('python tmp1243525234.py'))
%\end{python}
%}
%
\newpage
\begin{ex}
  \begin{tightlist}
    \item
    Let
    \[
    T(n) = T(\floor{n/2}) + T(n - \floor{n/2}) + n
    \]
    and $T(n) = 1$ if $n \leq 1$.
    Compute (by hand) and
    write down the table for $T(n)$ for $n = 0, 1, 2, ..., 9$.
    \item
    Let
    \[
    T'(n) = 2T'(n/2) + n
    \]
    and $T'(n) = 1$ if $n \leq 1$.
    Compute (by hand) and
    write down the table for $T'(n)$ for $n = 0, 1, 2, ..., 9$.
    \item
    Plot more values for $T(n)$ and $T'(n)$ ... as in \textit{lots} of $n$'s.
    Can you find good approximation for $T(n)$?
    (Here good mean functions such as $n^\alpha$, $\ln^\alpha n$,
    $n^\alpha \ln^\beta n$, $c^n$, etc.
  You'll see that asymptotically (when $n$ is large),
  $T(n)$ and $T'(n)$ are about the same.
  \end{tightlist}
  \qed
\end{ex}

The above experiments will show you that for our recurrences,
asymptotically (when $n$ is large),
it doesn't matter if $n/2$ is replaced by $\floor{n/2}$ or
$\ceiling{n/2}$ or
when $n - \floor{n/2}$ or $n - \ceiling{n/2}$ is replaced by
$n/2$.

\end{comment}

For the divide-and-conquer runtime recurrence,
more generally, we are interested in recurrences like this:
\[
T(n) = a \cdot T(n/b) + f(n)
\]
where $a$ and $b$ are constants.
For CISS350,
the two cases you must be familiar with are

\begin{thm}
  \mbox{}
  \begin{enumerate}[nosep]
  \item[\textnormal{(a)}] If \[
    T(n) = T(n/2) + A
    \]
    then
    \[
    T(n) = O(\lg n)
    \]
  \item[\textnormal{(b)}]
    If
    \[
    T(n) = 2 T(n/2) + An + B 
    \]
    then
    \[
    T(n) = O(n \lg n)
    \]
  \end{enumerate}
\end{thm}

For a more general theorem, see the Master Theorem.

\begin{eg}
For instance let's consider the time complexity of the binary search.
\begin{Verbatim}[frame=single,fontsize=\footnotesize]
def binarysearch(x, lower, upper, target):
    if lower > upper: return None
    else:
        mid = (lower + upper) / 2
        if x[mid] == target:
            return i
        elif x[mid] < target:
            return binarysearch(x, mid + 1, upper, target)
        else:
            return binarysearch(x, lower, mid - 1, target)
\end{Verbatim}
If $T(n)$ is the runtime to perform binarysearch on an
array of size $n$ (sorted of course), 
then 
\[
T(n) = T(n/2) + A
\]
Note that to be really precise, the array
\verb!x[0..n-1]! is cut up into
\textit{three} parts:
\verb!x[0..mid - 1]!, \,\,\,
\verb!x[mid]!, \,\,\,
\verb!x[m + 1..n - 1]!.
But note that \verb!mid! is roughly \verb!n/2! and so
\verb!mid! $\pm 1$ is roughly
$n/2$.
\qed
\end{eg}


\begin{eg}
What about mergesort? The list continually split into two equal parts
to be processed by mergesort. On return from mergesort, the two sublists
are combined.
\begin{Verbatim}[frame=single,fontsize=\footnotesize]
MERGESORT
INPUT: x: an array
       start, end: index values

if start >= end:
    return
else:
    mid = (start + end) / 2
    MERGESORT(x, start, mid)
    MERGESORT(x, mid + 1, end)
    MERGE(x, start, mid, mid+1, end)
\end{Verbatim}
The runtime of MERGE is $O(n)$.
The runtime for the computation of \verb!mid!
is constant.
So the runtime for MERGE and the computation of \verb!mid! is
of the form $An + B$.
Therefore
\[
T(n) = 2T(n/2) + An + B
\]
\end{eg}

\begin{comment}
Let's compute a closed form for $T(n)$.
We can perform 
\begin{align*}
T(n)
& = T(n/2) + c \\
& = T(n/2^2) + 2c \\
& = T(n/2^3) + 3c \\
& = \cdots \\
& = T(n/2^m) + mc \\
\end{align*}
At some point $n/2^m = 1$ and we reach
\begin{align*}
T(n) = T(1) + mc \\
\end{align*}
From $n / 2^m = 1$ we get
\[
n = 2^m
\]
Hence $m = \log_2 n$.
Note we have
\begin{align*}
T(n)
& = T(1) + c\log_2 n \\
& = A\log_2 n + B
\end{align*}
for some constants $A$ and $B$.
Therefore $T(n) = O(\log n)$.

Hmmm ... I wonder if this can be generalized ...
Well suppose
\[
T(n) = a(n/c_1) + c_2
\]
Doing the same computation I get
\begin{align*}
T(n) 
&= T(n/c_1) + c_2 \\
&= T(n/c_1^2) + 2c_2 \\
&= T(n/c_1^3) + 3c_2 \\
&= \ldots \\
&= T(n/c_1^m) + mc_2 \\
\end{align*}
At some point $n/c_1^m = 1$ and hence
\[
m = \log_{c_1} n
\]
Therefore
\begin{align*}
T(n) 
&= T(1) + c_2 \lg n\\
&= A \log_{c_1} n + B\\
\end{align*}
for constants $A$ and $B$.

Suppose a divide and conquer algorithm looks like this:
\begin{Verbatim}[frame=single,fontsize=\footnotesize]
ALGORITHM A
INPUT: Some work w of size n

Let w1,w2,...,wa be subwork of w of sizes n1,n2,...,na

Pre-recursion work
Perform A on w1
Perform A on w2
...
Perform A on wa
Post-recursion work
\end{Verbatim}

Let $T(n)$ be the time needed to perform the algorithm on a work of
size $n$.
Suppose the time take for non-recursion work is $f(n)$.
Suppose that the non-recursive parts take $f(n)$ time.
The total time is then
\[
T(n)
= T(n_1) + \cdots + T(n_a) + f(n)
\]
Now I will assume that
\[
n_1 = n_2 = \ldots = n_a = n / b
\]
where $a \geq 1$ and $b > 1$.
(See ??? for divide-and-conquer where the subwork has different sizes.)
The above becomes
\[
T(n)
= a T(n/b) + f(n)
\]
Now I want to know what's the big-O of $T(n)$.
But before that ... here are some examples just to put things into
context.
I'm going to list some runtimes with the corresponding $a$ and $b$.

\begin{eg}
Consider the binary search algorithm on a sorted array.
We have 
\[
T(n) = T(n/2) + A
\]
in the worst case when the target is not found.
Of course when the target is at the middle of the array, the algorithm
terminates right away -- this is of course not the worst case.
So in this case,
\[
a = 1, \,\,\,\,\, b=2, \,\,\,\,\, f(n) = A = \Theta(1)
\]
\qed
\end{eg}

\begin{eg}
Consider the mergesort algorithm on an array.
We have 
\[
T(n) = 2T(n/2) + An + B
\]
So in this case,
\[
a = 2 = b, \,\,\,\,\, f(n) = An + B = \Theta(n)
\]
\qed
\end{eg}


Let's play around with the above runtime recurrence
\[
T(n)
= a T(n/b) + f(n)
\]
It's going to be pretty ugly for a general $n$.
Since we continually have to compute $n/b$, let's try it with $n = b^k$.
Then
\begin{align*}
T(b^k)
&= a T \left( b^{k-1} \right) + f(b^k) \\
&= a \left( a T \left( b^{k-2} \right) + f(b^{k-1}) \right)+ f(b^k) \\ 
&= a^2 T \left( b^{k-2} \right) + a f(b^{k-1}) + f(b^k) \\
&= a^2 \left( a T \left( b^{k-3} \right) + f(b^{k-3}) \right) + a f(b^{k-1})
   + f(b^k) \\ 
&= a^3 T \left( b^{k-3} \right) + a^2 f(b^{k-2}) + a f(b^{k-1}) + f(b^k) \\
&= \cdots \\
&= a^k T \left( b^0 \right)
   + \left( a^{k-1} f(b^1) + \cdots + a f(b^{k-1}) + f(b^k) \right)
\end{align*}
Now don't forget that I introduced $k$ just to make $n$ look like $b^k$.
So when I need to, I have to wipe out $k$.
This is easy:
\[
n = b^k \iff \log_b n = k
\]
In other words, later on, I have to replace $k$ by $\log_b n$.

The resulting expression
\[
T(b^k)
= a^k T \left( b^0 \right)
   + \left( a^{k-1} f(b^1) + \cdots + a f(b^{k-1}) + f(b^k) \right)
\]
is a sum of two terms so it's a matter of
which of the two, i.e.,
\[
a^k T \left( b^0 \right)
\]
or
\[
   a^{k-1} f(b^1) + \cdots + a f(b^{k-1}) + f(b^k) 
\]
is \lq\lq bigger'', asymptotically speaking.

We can simplify the first term a little:
\[
a^k T \left( b^0 \right) = \Theta (a^k) = \Theta (a^{\log_b n})
\]
$a^{\log_b n}$ is kind of weird since it's not one of the
standard representative asymptotic functions like $n^\alpha$
or $\log^\alpha$ or $n^\alpha \log^\beta n$ or $c^n$ or etc.
But note that
\[
a^{\log_b n} = n^{\log_b a}
\]
Why? Look at this:
\begin{align*}
(\log_b n)(\log_b a) &= (\log_b a)(\log_b n) \\
\THEREFORE \log_b a^{\log_b n} &= \log_b n^{\log_b a} \\
\THEREFORE a^{\log_b n} &= n^{\log_b a} 
\end{align*}
So the first term is
\[
a^k T \left( b^0 \right)
= \Theta \left( a^{\log_b n} \right)
= \Theta \left( n^{\log_b a} \right)
\]

The second term
\[
   a^{k-1} f(b^1) + \cdots + a f(b^{k-1}) + f(b^k)
\]
looks like a pain.
But ... hmmm it sure looks like a geometric sum.
Of course I can't say anything more. Why? Well ... there's just nothing to say
because I don't have more information about the function $f(n)$.
The sum clearly adds up the amount of work ($k$ pieces of work)
which takes $f(n)$ time 
for a bunch of values of $n$. 
The question is of course: how much work is involved and therefore
how much time, i.e., how big is $f(n)$?
Now from some examples such as binary search, mergesort, etc.
we can have
\[
f(n) = \Theta(1)
\]
(think binary search)
or
\[
f(n) = \Theta(n)
\]
(think mergesort).
So let's say we impose some condition to make $f(n)$ \lq\lq nice'':
\[
f(n) = \Theta(n^c)
\]
say
\[
f(n) = An^c
\]
to be concrete (ignoring unimportant terms in $f(n)$.)
Then I can at least say that 
\begin{align*}
&a^{k-1} f(b^1) + \cdots + a f(b^{k-1}) + f(b^k) \\
&= a^{k-1} \cdot A(b^1)^c
   + \cdots
   + a \cdot A (b^{k-1})^c
   + a \cdot A (b^{k})^c \\
&= A \left(
   a^{k-1} (b^c)^{1}
   \cdots
   + a (b^c)^{k-1}
   + (b^c)^{k}
   \right)
\end{align*}
Well ... too bad ... it's still pretty yucky.
It still looks somewhat like a geometric sum.
We impose the condition $f(n) = \Theta(n^c)$ to make $f(n)$
nice--looking.
Let's make it even nicer.
When I look at the term
\[
a^{k-1} (b^c)^{1}
\]
in the above sum (and also the rest), I see that if $b^c$ is an $a$--power,
I would get a sum of $a$--powers.
So how about I checkout the case when
\[
b^c = a
\]
This means
\[
c = \log_b a
\]
The above becomes
\begin{align*}
&a^{k-1} f(b^1) + \cdots + a f(b^{k-1}) + f(b^k) \\
&= A \left(
   a^{k-1} (b^c)^{1}
   \cdots
   + a (b^c)^{k-1}
   + (b^c)^{k}
   \right)
   \\
&= A \left(
   a^{k-1} (a)^{1}
   \cdots
   + a (a)^{k-1}
   + (a)^{k}
   \right)
   \\
&= A ka^k
\end{align*}
OK, this is not a geometric sum ... but ... at least it's
simplified!
Since $k = \log_b n$,
\begin{align*}
&a^{k-1} f(b^1) + \cdots + a f(b^{k-1}) + f(b^k) \\
&= A ka^k \\
&= A \log_b n \cdot a^{\log_b n} \\
&= \Theta \left( \log_b n \cdot  a^{\log_b n}\right)
\end{align*}
Don't forget that the awkward looking $a^{\log_b n}$ is
the same as $n^{\log_b a}$.
So
\begin{align*}
&a^{k-1} f(b^1) + \cdots + a f(b^{k-1}) + f(b^k) \\
&= \Theta \left( \log_b n \cdot  a^{\log_b n}\right) \\
&= \Theta \left( \log n \cdot  n^{\log_b a}\right)
\end{align*}
(The $\log_b n$ is replaced by $\log n$, since log bases are not
important in big-O or big-$\Theta$.)

Altogether we have the following.
If $n = b^k$, then
\[
T(n) = a^k T \left( b^0 \right)
 + a^{k-1} f(b^1) + \cdots + a f(b^{k-1}) + f(b^k)
\]
where
\[
a^k T \left( b^0 \right) = \Theta \left( n^{\log_b a} \right)
\]
and if $f(n) = \Theta(n^c)$ where $c = \log_b a$, then
\[
a^{k-1} f(b^1) + \cdots + a f(b^{k-1}) + f(b^k) =
\Theta \left( \log n \cdot  n^{\log_b a}\right)
\]
I conclude that 
\[
T(n)
=
\Theta \left( n^{\log_b a} \log n \right)
\]

Before we celebrate, remember that I made the assumption
\[
f(n) = \Theta(n^c), \,\,\,\,\, c = \log_b a
\]
What if
\[
f(n) = \Theta(n^c), \,\,\,\,\, c \neq \log_b a
\]
We're missing the case $c < \log_b a$ and $c > \log_b a$.
Also, what if $f(n) = O(n^c)$?
Or what if $f(n) = O(\log n)$?
Etc.

Here's the famous \defterm{master theorem} ...

\end{comment}

Here's the Master Theorem which covers the two cases mentioned above:

\begin{thm} \textsc{(Master theorem)}
Let $T : \N \rightarrow \R$ be a function such that
\[
T(n)
= a T \left( \frac{n}{b} \right) + f(n)
\]
where $a \geq 1$ and $b > 1$.
\begin{myenum}
  \item[(a)] If $f(n) = O(n^c)$ where $c <  \log_b a$, then
  \[
  T(n) = \Theta \left( n^{\log_b a} \right)
  \]
  
  \item[(b)] If $f(n) = \Theta(n^c \log^k n)$ where $c = \log_b a$, then
  \[
  T(n) = \Theta \left( n^{\log_b a} \log^{k+1} n \right)
  \]

  \item[(c)] If $f(n) = \Omega(n^c)$ where $c > \log_b a$ and
  there is a constant $k < 1$ such that  
  \[
  a f(n/b) \leq k f(n)
  \]
  for sufficiently large $n$, then
  \[
  T(n) = \Theta(f(n))
  \]
\end{myenum}
\end{thm}

\begin{eg}
  For binary search, if we let $T(n)$ be the runtime, then
  \[
  T(n) = T(n/2) + A
  \]
  \[
  a = 1, \,\,\, b = 2, \,\,\, f(n) = O(1) = O(n^0), \,\,\, c = 0
  \]
  Note that
  \[
  \log_b a = \log_2 1 = 0
  \]
  and
  \[
  c \not< \log_b a
  \]
  So case 1 of the master theorem does not apply.
  Yikes. Now what?
  
  If you look at the second case, you see that
  \[
  0 = c = \log_b a
  \]
  Now note that $f(n)$ is in fact $\Theta(1)$, i.e., $f(n) = \Theta(n^c \log^k n)$
  for $c = 0$ and $k = 0$.
  Therefore,using case (b),
  \[
  T(n) = \Theta(n^0 \log^{0+1} n) = \Theta(\lg n)
  \]
  \qed
\end{eg}
\qed

\begin{eg}
  For mergesort
  \[
  a = 2, \,\,\,\,\, b = 2, \,\,\,\,\, f(n) = \Theta(n) = \Theta(n^c), c = 1
  \]
  Note that
  \[
  \log_b a = \log_2 2 = 1
  \]
  and
  \[
  c = \log_b a
  \]
  So case (b) of Master Theorem applies with $k = 0$.
  This gives us
  \[
  T(n) = \Theta(n \log n)
  \]
  \qed
\end{eg}

\begin{eg}
  Let $T(n) = 2T(n/2) + \Theta(n^2)$.
  Then $a = 2 = b$ and $\log_b a = 1$.
  Using (c) of Master Theorem, with $c = 2 > 1 = \log_b a$.
  With $f(n) = n^2$,
  \[
  af(n/b) = 2 (n/2)^2 = n^2/2 \leq k n^2 = kf(n)
  \]
  for $k = 1/2 < 1$.
  Then $T(n) = \Theta(n^2)$.
  \qed
\end{eg}



\begin{ex} 
  \label{ex:master-theorem-0}
  \tinysidebar{\debug{exercises/{disc-prob-28/question.tex}}}

  \solutionlink{sol:master-theorem-0}
  \qed
\end{ex} 
\begin{python0}
from solutions import *
add(label="ex:master-theorem-0",
    srcfilename='exercises/master-theorem-0/answer.tex') 
\end{python0}



\begin{ex}
We have big-O concept for $a(n)$.
Suppose $a(n) = O(b(n))$.
Let
\[
f(x) = \sum_{n = 0}^\infty a(n) x^n
\]
and
\[
g(x) = \sum_{n = 0}^\infty b(n) x^n
\]
What can we say about the relationship between $f(x)$ and $g(x)$ and
vice versa?
Is it true that
\[
a(n) = O(b(n)) \iff f(x) = O(g(x))
\]
\qed
\end{ex}
