%-*-latex-*-
\tinysidebar{\debug{exercises/{best-average-worst-3/question.tex}}}
  Here's anothering shuffling.
  Here I'm assuming that
the original array \verb!x! does not contain the value \verb!-1!.
I'm going to use another array \verb!y! of the same size \verb!n!.
The idea is very simple: \verb!-1! is used to denote
\lq\lq unoccupied'' in array \verb!y!.
I will put the values from \verb!x! into \verb!y!
at a random index position.
If the index position in \verb!y! is already occupied,
I will move to the next, cycling back to index 0 if necessary.
Once this is done, I copy the values in \verb!y! back to \verb!x!.
\begin{Verbatim}[frame=single, fontsize=\footnotesize]
for i = 0, 1, 2, ..., n - 1:
    y[i] = -1

seed(0)
for i = 0, 1, 2, ..., n - 1:
    j = rand() % n
    while y[j] != -1:
        j = (j + 1) % n
    y[j] = x[i]

for i = 0, 1, 2, ..., n - 1:
    x[i] = y[i]
\end{Verbatim}
\begin{enumerate}
\item[(a)] Compute the best runtime of the above and then the big-O.
\item[(b)] Compute the worst runtime of the above and the big-O.
(Be careful now!!! What exactly is the worst case???)
\end{enumerate}
The translation of a while-loop into
goto and conditional branching statements is similar to the 
for-loop. Here's the translation of the above while-loop:
\begin{Verbatim}[frame=single, fontsize=\footnotesize]
...
    while y[j] != -1:
        j = (j + 1) % n
    y[j] = x[i]
...
\end{Verbatim}
into goto statements:
\begin{Verbatim}[frame=single, fontsize=\footnotesize]
...
LOOP3:     if y[j] == -1:
               goto ENDLOOP3
           j = (j + 1) % n
           goto LOOP3

ENDLOOP3:  y[j] = x[i]
...
\end{Verbatim}
