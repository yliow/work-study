%-*-latex-*-
\tinysidebar{\debug{exercises/40000-1-10-n2-nsinn/answer.tex}}
(a)
For $n$ in $[0,20]$, we have
\begin{python}
from latextool_basic import *
p = FunctionPlot(line_width=1, domain='0:20', vars=['n','y'])
p.add('40000', color='red', python=1, pin='below right', num_points=2)
p.add('n**2/10', color='green', python=1, pin='above left', num_points=8)
p.add('n * sin(n)', color='blue', python=1, pin='above right', num_points=40)
print (p)
\end{python}
For $n$ in $[0,40]$, we have
\begin{python}
from latextool_basic import *
p = FunctionPlot(line_width=1, domain='0:40', vars=['n','y'])
p.add('40000', color='red', python=1, pin='below right', num_points=2)
p.add('n**2/10', color='green', python=1, pin='above left', num_points=8)
p.add('n * sin(n)', color='blue', python=1, pin='above right', num_points=40)
print (p)
\end{python}
For $n$ in $[0,100]$, we have
\begin{python}
from latextool_basic import *
p = FunctionPlot(line_width=1, domain='0:100', vars=['n','y'])
p.add('40000', color='red', python=1, pin='below right', num_points=2)
p.add('n**2/10', color='green', python=1, pin='above left', num_points=8)
p.add('n * sin(n)', color='blue', python=1, pin='above right', num_points=40)
print (p)
\end{python}
For $n$ in $[0,1000]$, we have
\begin{python}
from latextool_basic import *
p = FunctionPlot(line_width=1, domain='0:1000', vars=['n','y'])
p.add('40000', color='red', python=1, pin='above right', num_points=2)
p.add('n**2/10', color='green', python=1, pin='above left', num_points=8)
p.add('n * sin(n)', color='blue', python=1, pin='above right', num_points=40)
print (p)
\end{python}

(b)
Graphically,
for $n \geq 700$, $n^2/10$ dominates $40000$ and $n \sin (n)$.
Therefore the big-O of $f(n)$ is determined by $n^2/10$.

(c)
The big-O of $n^2/10$ is $O(n^2)$ (by replacing $1/10$ by $1$).
Therefore $f(n) = O(n^2)$.
\qed
