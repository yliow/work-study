%-*-latex-*-
\sectionthree{Other types of asymptotic bounds}
\begin{python0}
from solutions import *; clear()
\end{python0}

In this section, all functions are functions of $n \in \N$.
I'll write $f$ instead of $f(n)$, etc.
I'll assume that all functions $f$ are asymptotically nonzero, i.e.,
there is some $N$ such that for all $n \geq N$, I will assume $f(n) \neq 0$.

Recall that $f = O(g)$ is
informally some kind of \lq\lq $\leq$" inequality:
a multiple of $g$ bounds $f$ from above asymptotically.
We also say that $f$ is \defone{asymptotically bounded above} by $g$.

It's also useful to have the concept of asymptotic \textit{lower} bound:

\begin{defn}
  We
  say $f$ is the \defone{big-$\Omega$} of $g$ and 
  write $f = \Omega(g)$ if
  there exist some $C > 0$ and some $N$ such that 
  for $n \geq N$,
  \[
  C|g(n)| \leq |f(n)|
  \]
  In other words a multiple of $g$ (asymptotically)
  bounds $f$ from \textit{below}.
  I will say that $f$ is \defone{asymptotically bounded below} by $g$.
\end{defn}

You can put the two definitions together and get this definition:

\begin{defn}
$f = \Theta(g)$ if $f = O(g)$ and $f = \Omega(g)$, i.e., 
there exist $C_1 > 0, N_1$ and $C_2 > 0, N_2$ such that
for $n \geq N_1$
\[
C_1|g(n)| \leq |f(n)|
\]
and for $n \geq N_2$,
\[
|f(n)| \leq C_2 |g(n)|
\]
If $f(n) = \Theta(g(n))$, I would say that
$f(n)$ has an \defone{asymptotic upper and lower bound} of $g(n)$.
Note that you don't need two cutoff points $N_1$ and $N_2$ for $n$
since if you
set $N = \max \{ N_1, N_2 \}$, then for $n \geq N$,
\[
C_1|g(n)| \leq |f(n)| \leq C_2 |g(n)|
\]
So the above definition is equivalent to this:
$f = \Theta(g)$ if there are constants $C_1 > 0, C_2 > 0$ and $N$ such that
for $n \geq N$,
\[
C_1|g(n)| \leq |f(n)| \leq C_2 |g(n)|
\]
\end{defn}

Informally, you want to think about $O$, $\Omega$, $\Theta$ roughly as:
\begin{align*}
f = O(g)      &\text{ \hspace{1cm} roughly \hspace{1cm} } f \text{\lq\lq$\leq$"} g \\
f = \Omega(g) &\text{ \hspace{1cm} roughly \hspace{1cm} } f \text{\lq\lq$\geq$"} g \\
f = \Theta(g) &\text{ \hspace{1cm} roughly \hspace{1cm} } f \text{\lq\lq$=$"} g
\end{align*}
Recall that you want your asymptotic upper and lower bound be as tight as
possible.
For instance it is true that
\[
n^2 + n + 1 = O(n^{100})
\]
But this is a tighter upper bound:
\[
n^2 + n + 1 = O(n^{2})
\]
This is the same for $\Omega$:
\[
n^2 + n + 1 = \Omega(n^1)
\]
So $O$ and $\Omega$ can be tight.

Here are some basic facts you should know.

\begin{prop}
  $f = O(g)$ iff $g = \Omega(f)$.
\end{prop}

\begin{prop}
  Let $f, g, h, f_i, g_i$ be functions of $n \in \N$ and $c \neq 0$ be a constant.
  \begin{myenum}
  \item
    \textsc{Reflexive:}
    \begin{align*}
    f &= O(f) \\
    f &= \Omega(f) \\
    f &= \Theta(f) 
    \end{align*}
  \item
    \textsc{Symmetric:}
    If $f,g$ are asyptotically nonzero, then
    \begin{align*}
    f = \Theta(g) \implies g = \Theta(f)
    \end{align*}  
  \item
    \textsc{Transitive:}
    \begin{align*}
    f = O(g) \text{, } g = O(h) &\implies f = O(h) \\
    f = \Omega(g) \text{, } g = \Omega(h) &\implies f = \Omega(h) \\
    f = \Theta(g) \text{, } g = \Theta(h) &\implies f = \Theta(h) \\
    \end{align*}
  \item
    \textsc{Antisymmetric:}
    \[
    f = O(g) \text{, } g = O(f) \implies f = \Theta(g)
    \]
  \end{myenum}
  \qed
\end{prop}

\begin{prop}
  \mbox{}
  \begin{myenum}
  \item
    \textsc{Summation:}
    \begin{align*}
    f_1 = O(g_1), \ f_2 = O(g_2) &\implies f_1 + f_2 = O \bigl( \max(|g_1|,|g_2|) \bigr) \\
    f_1 = \Omega(g_1), \ f_2 = \Omega(g_2) &\implies f_1 + f_2 = \Omega \bigl( \min(|g_1|,|g_2|) \bigr)
    \end{align*}
  \item
    \textsc{Product:}
    \begin{align*}
    f_1 = O(g_1), \ f_2 = O(g_2) &\implies f_1 f_2 = O(g_1 g_2) \\
    f_1 = \Omega(g_1), \ f_2 = \Omega(g_2) &\implies f_1 f_2 = \Omega(g_1 g_2) \\
    f_1 = \Theta(g_1), \ f_2 = \Theta(g_2) &\implies f_1 f_2 = \Theta(g_1 g_2) \\
    \end{align*}
  \end{myenum}
\end{prop}


\begin{cor}
    \begin{align*}
    f_1 = O(g), \ f_2 = O(g) &\implies f_1 + f_2 = O(g) \\
    f_1 = \Omega(g), \ f_2 = \Omega(g) &\implies f_1 + f_2 = \Omega(g) \\
    f_1 = \Theta(g), \ f_2 = \Theta(g) &\implies f_1 + f_2 = \Theta(g) 
    \end{align*}
\end{cor}

\begin{cor}
    \textsc{Constant cancellation:}
    \begin{align*}
      f = O(cg) \iff f = O(g)           &\text{ \,\,\, and \,\,\, } cf = O(g) \iff f = O(g) \\
      f = \Omega(cg) \iff f = \Omega(g) &\text{ \,\,\, and \,\,\, } cf = \Omega(g) \iff f = \Omega(g) \\
      f = \Theta(cg) \iff f = \Theta(g) &\text{ \,\,\, and \,\,\, } cf = \Theta(g) \iff f = \Theta(g)
    \end{align*}
\end{cor}

Now I mentioned earlier
\begin{align*}
n^2 + n + 1 &= O(n^2) \\
n^2 + n + 1 &= O(n^3) \\
n^2 + n + 1 &= O(n^{100}) \\
\end{align*}
There are times when I want to say $\lq\lq f$ is asymptotically bounded
above by $g$ but $g$ is \textit{not} a tight upper bound".
That's where  
the \lq\lq little" notations come in: 
the little-o and little-$\omega$.
($\omega$ is the lowercase of Greek $\Omega$.)

\begin{defn}
  Let $f,g$ be functions of $n$.
  \begin{myenum}
  \item
    $f = o(g)$\index{$o$@o}\tinysidebar{$o$}
    if for \textit{every} constant $C > 0$, there is some $N(C)$ such that
    for $n \geq N(C)$,
    \[
    |f(n)| \leq C |g(n)|
    \]
  \item
    $f = \omega(g)$\index{$\omega$@omega}\tinysidebar{$\omega$}
    if for \textit{every} constant $C > 0$, there is some $N(C)$ such that
    for $n \geq N(C)$,
    \[
    |f(n)| \leq C |g(n)|
    \]
  \end{myenum}
   Compare the definitions of little-o and big-O and then
   little-$\omega$ and big-$\Omega$ very carefully.
\end{defn}


What is the difference between little-o and big-O?
Just from the logic behind the two definitions, it's clear that
\[
f = o(g) \implies f = O(g)
\]
But what is the intuition behind the definition of little-o?
For now think of $C$ as a fixed number.
Then if $f = o(g)$, there is some $N(C)$ such that for
\[
n \geq N(C) \implies |f(n)| \leq C|g(n)|
\]
If I change $C$ to a larger number $C'$, I can use the same $N(C)$
for the above implication.
So no big deal.
But what if I replace $C$ by a smaller $C'$?
Then there is some $N(C')$ such that
\[
n \geq N(C') \implies |f(n)| \leq C'|g(n)|
\]
This basically says that no matter how small I \lq\lq shrink" $|g(n)|$
by a very tiny factor $C'$, the $C'|g(n)|$ still bound
$|f(n)|$.
In other words, if $f = o(g)$, then it means that
$g$ is an upper bound of $f$ in such a
way that no matter how I try to \lq\lq shrink $g$ by a constant
multiplier, $f$ can never go beyond $g$ (asymptotically).

Using a concrete example,
\[
100n^2 + n + 1 = O(n^2) \text{ and } 100n^2 + n + 1 = O(n^3)
\]
but
\[
100n^2 + n + 1 \neq o(n^2) \text{ and } 100n^2 + n + 1 = o(n^3)
\]
Therefore, informally, while you think of \lq\lq $f = O(g)$" as
roughly \lq\lq $f \leq g$, you think of \lq\lq $f = o(g)$"
as roughly \lq\lq $f < g$".

This is the same for little-$\omega$.
Informally, you want to think about $O$, $o$, $\Omega$, $\omega$, $\Theta$ roughly as:
\begin{align*}
f = O(g)      &\text{ \hspace{1cm} roughly \hspace{1cm} } f \text{\lq\lq$\leq$"} g \\
f = o(g)      &\text{ \hspace{1cm} roughly \hspace{1cm} } f \text{\lq\lq$<$"} g \\
f = \Omega(g) &\text{ \hspace{1cm} roughly \hspace{1cm} } f \text{\lq\lq$\geq$"} g \\
f = \omega(g) &\text{ \hspace{1cm} roughly \hspace{1cm} } f \text{\lq\lq$>$"} g \\
f = \Theta(g) &\text{ \hspace{1cm} roughly \hspace{1cm} } f \text{\lq\lq$=$"} g \\
\end{align*}

\begin{ex}
  Show that
  \begin{myenum}
  \item $1 \neq o(1)$ and $1 = o(n)$.
  \item $5n \neq o(12n)$ and $5n = o(2n^2)$.
  \item $100n^2 + n + 1 \neq o(n^2)$ and $100n^2 + n + 1 = o(n^3)$.
  \end{myenum}
\end{ex}


\begin{prop}
\mbox{}
\begin{myenum}
\item If $f = o(g)$, then $f = O(g)$. 
\item If $f = \omega(g)$, then $f = \Omega(g)$. 
\end{myenum}
\end{prop}



\begin{prop}
  \mbox{}
  \begin{myenum}
    \item
    \textsc{Transitivity:}
    \begin{align*}
      f = o(g) \text{, } g = o(h) &\implies f = o(h) \\
      f = \omega(g) \text{, } g = \omega(h) &\implies f = \omega(h)
    \end{align*}
  \item
    \textsc{Summation:}
    \begin{align*}
      f_1 = o(g_1), \ f_2 = o(g_2) &\implies f_1 + f_2 = o \bigl( \max(|g_1|,|g_2|) \bigr) \\
      f_1 = \omega(g_1), \ f_2 = \omega(g_2) &\implies f_1 + f_2 = \omega \bigl( \min(|g_1|,|g_2|) \bigr)
    \end{align*}
  \item
    \textsc{Product:}
    \begin{align*}
      f_1 = o(g_1), \ f_2 = o(g_2) &\implies f_1 f_2 = o(g_1 g_2) \\
      f_1 = \omega(g_1), \ f_2 = \omega(g_2) &\implies f_1 f_2 = \omega(g_1 g_2) 
    \end{align*}
  \end{myenum}
\end{prop}

\begin{cor}
    \begin{align*}
    f_1 = o(g), \ f_2 = o(g) &\implies f_1 + f_2 = o(g) \\
    f_1 = \omega(g), \ f_2 = \omega(g) &\implies f_1 + f_2 = \omega(g)
    \end{align*}
\end{cor}

\begin{cor}
    \textsc{Constant cancellation:}
    \begin{align*}
      f = o(cg) \iff f = o(g)           &\text{ \,\,\, and \,\,\, } cf = o(g) \iff f = o(g) \\
      f = \omega(cg) \iff f = \omega(g) &\text{ \,\,\, and \,\,\, } cf = \omega(g) \iff f = \omega(g)
    \end{align*}
\end{cor}


Besides \textit{inequality} types asymptotic bounds, there are also
\textit{limit} types asymptotic properties.
Here's one:

\begin{defn}
  $f$ and $g$ are \defone{asymptotically equivalent},
  and we write
  \[
  f(n) \sim g(n)
  \]
  if
  \[
  \lim_{n \rightarrow \infty} \frac{f(n)}{g(n)} = 1
  \]
\end{defn}

For instance
\[
\lim_{n \rightarrow \infty} \frac{n^2 + 1000\ln n}{n^2 + 10000n\sin(n)}
= 1
\]
This does \textit{not} mean that they actually meet, i.e., 
it does not mean that there is some $N$ such that 
for $n \geq N$, $f(n) = g(n)$.

Note that
\[
f \sim g
\]
is the not the same as
\[
f = \Theta(g)
\]
$f \sim g$ is clearly a lot tighter than $f = \Theta(g)$.
For instance
\[
\sin(n) = \Theta(1)
\]
But
\[
\sin(n) \not\sim (1)
\]

\begin{prop}
\mbox{}
\begin{myenum}
\item \textsc{Reflexive:} $f \sim f$
\item \textsc{Symmetric:} $f \sim g \implies g \sim f$
\item \textsc{Transitive:} $f \sim g, g \sim h \implies f \sim h$
\item \textsc{Summation:} $f_1 \sim g_1, f_2 \sim g_2
  \implies f_1 + f_2 \sim g_1 + g_2$
\item \textsc{Product:}
  $f_1 \sim f_2, g_1 \sim g_2
  \implies f_1 g_1 \sim f_2 g_2$
\item \textsc{Quotient:}
  If $f_2, g_2$ are asymptotically nonzero, then
  \[
  f_1 \sim g_1, f_2 \sim g_2 \implies f_1/f_2 \sim g_1/g_2
  \]
\end{myenum}
(a)-(c) says that asymptotic equivalence is an equivalence
relation.
\end{prop}

Recall that you can view big-$\Theta$ as a asymptotic rough version
of \lq\lq =".
Specifically $C_1|g(n)| \leq f(n) \leq C_2|g(n)|$ for large $n$.
There $C_1,C_2$ are some fixed constants.
Asymptotic equivalence is tighter than big-$\Theta$:

\begin{prop}
If $f \sim g$, then $f = \Theta(g)$. 
\end{prop}

Therefore if $\displaystyle\lim_{n \rightarrow \infty} \frac{f(n)}{g(n)}$ exists,
one can use $\displaystyle\lim_{n \rightarrow \infty} \frac{f(n)}{g(n)} = 0$ as the
definition of $f = o(g)$.
This is in fact done in some books.
But my definition is more general.

\begin{prop}
Assume $f, g$ are asymptotically $> 0$.
Suppose $\displaystyle \lim_{n \rightarrow \infty} \frac{f(n)}{g(n)}$ exists (including the case of $\infty$).
Let $L = \displaystyle \lim_{n \rightarrow \infty} \frac{f(n)}{g(n)}$.
Note that $L \in [0, \infty)$ or $L = \infty$.
For simplicity, I'll write, \lq\lq $L \in (0, \infty]$" as a shorthand for
\lq\lq $L \in (0, \infty)$ or $L = \infty$".
\begin{longtable}{rlcrl}
\textnormal{(a)} & $L \in [0, \infty) \iff f = O(g)$      & \hspace{1cm} & \textnormal{(d)} & $L = 0 \iff f = o(g)$  \\
\textnormal{(b)} & $L \in (0, \infty] \iff f = \Omega(g)$ & \hspace{1cm} & \textnormal{(e)} & $L = \infty \iff f = \omega(g)$ \\
\textnormal{(c)} & $L \in (0, \infty) \iff f = \Theta(g)$ & \hspace{1cm} & \textnormal{(f)} & $L = 1 \iff f \sim g$
\end{longtable}
\end{prop}
\proof
(d)
Let
\begin{myenum}
\item[1] $\lim_{n\rightarrow \infty} f(n)/g(n) = 0$ means:
  For all $\ep > 0$, there is some $N(\ep)$ such that
if $n \geq N(\ep)$, then $|f(n)/g(n) - 0| < \ep$.
The inequality is equivalent to $|f(n)| < \ep |g(n)|$.
\item[2] $f = o(g)$ means: For all $C > 0$, there is some $N(C)$ such that
$|f(n)| \leq C |g(n)|$.
\end{myenum}
Therefore I need to show (1)$\Longleftrightarrow$(2).

(1)$\Longrightarrow$(2):
Let $C > 0$. By (a), there is some $N(C)$ such that
for $n > N(C)$, $|f(n)| < C|g(n)|$ and hence $|f(n)| \leq C|g(n)|$.

(2)$\Longrightarrow$(1):
Let $\ep > 0$.
By (b), there is some $N(\ep/2)$ such that
for $n \geq N(\ep/2)$, $|f(n)| \leq (\ep/2)|g(n)|$, and hence
$|f(n)| \leq (\ep/2)|g(n)| < \ep |g(n)|$.
\qed

Therefore,
provided
$\displaystyle\lim_{n \rightarrow \infty} f(n)/g(n)$
exists, the above provides an alternative definition
for $O, \Omega, \Theta, o, \omega$.
For instance if $\displaystyle\lim_{n \rightarrow \infty} f(n)/g(n)$ exists,
then
\[
f = o(g) \iff \lim_{n \rightarrow} \frac{f(n)}{g(n)} = 0
\]


Therefore when $\displaystyle \lim_{n \rightarrow \infty} f(n)/g(n)$ exists, we can use Calculus to compute the
asymptotic relation between $f$ and $g$.
The following are some results that uses this technique.

\begin{prop}
  \mbox{}
  \begin{myenum}
  \item $\ln^a n = o(n^b)$ where $b/a > 0$ or $a = 0, b > 0$
  \item $n^b = o(c^n)$ where $c > 1$
  \end{myenum}
\end{prop}
\proof
(a)
Suppose $b/a > 0$.
\begin{align*}
  \lim_{n \rightarrow \infty} \frac{\ln^a n}{n^b}
  &= \lim_{n \rightarrow \infty} \left( \frac{\ln n}{n^{b/a}} \right)^a\\
  &= \lim_{n \rightarrow \infty} \left( \frac{1/n}{(b/a)n^{b/a - 1}} \right)^a & & \text{by l'H\^opital's rule}\\
  &= \lim_{n \rightarrow \infty} \left( \frac{1}{(b/a)n^{b/a}} \right)^a\\
  &= 0
\end{align*}
For the case $a = 0, b > 0$, $\ln^a n = 1$ and
$\displaystyle \lim_{n \rightarrow \infty} 1/n^b = 0$.
Hence in both cases $\ln^a n = o(n^b)$.

(b)
Suppose $b > 0$.
\begin{align*}
  \lim_{n \rightarrow \infty} \frac{n^b}{c^n}
  &= \lim_{n \rightarrow \infty} \left( \frac{n}{c^{n/b}} \right)^b\\
  &= \left( \lim_{n \rightarrow \infty} \frac{n}{c^{n/b}} \right)^b\\
  &= \left( \lim_{n \rightarrow \infty} \frac{1}{c^{n/b}(\ln c)(1/b)} \right)^b & & \text{by l'H\^opital's rule}\\
  &= \left( \frac{b}{\ln c} \cdot \lim_{n \rightarrow \infty} \frac{1}{c^{n/b}} \right)^b\\
  &= 0
\end{align*}
If $b \leq 0$,
$\displaystyle\lim_{n \rightarrow \infty} \frac{n^b}{c^n} = \lim_{n \rightarrow \infty} \frac{1}{n^{|b|}c^n} = 0$.
Hence in both cases $n^b = o(c^n)$.
\qed

\begin{prop}
  If $c < d$, then $c^n = o(d^n)$.
\end{prop}

The follow says that in any asymptotic relation, a function can be
replaced by another asymptotic equivalent function:

\begin{prop}
  Suppose $f_1 \sim f_2, g_1 \sim g_2$.
  \begin{myenum}
  \item $f_1 = O(g_1),  \implies f_2 = O(g_2)$
  \item $f_1 = \Omega(g_1),  \implies f_2 = \Omega(g_2)$
  \item $f_1 = \Theta(g_1),  \implies f_2 = \Theta(g_2)$
  \item $f_1 = o(g_1),  \implies f_2 = o(g_2)$
  \item $f_1 = \omega(g_1),  \implies f_2 = \omega(g_2)$
  \end{myenum}
\end{prop}

%\newpage
%\subsection*{Solutions}
%
\newpage
\section*{Solutions}
Solution to Exercise \ref{ex:power-series-11}\labeltext{}{sol:power-series-11}.

\debug{\tinysidebar{exercises/{power-series-11/answer.tex}}}
 
(a) From
\begin{align*}
\sum_{n = 0}^\infty \frac{1}{2^n} x^n \cdot \sum_{n = 0}^\infty \frac{1}{2^n} x^n
&=
\left(
1 + \frac{1}{2}x + \frac{1}{4}x^2 + \frac{1}{8}x^3 + \cdots
\right)
\left(
1 + \frac{1}{2}x + \frac{1}{4}x^2 + \frac{1}{8}x^3 + \cdots
\right)
\end{align*}
the coefficient of $x^3$ is
\[
1 \cdot \frac{1}{8}
+ \frac{1}{2} \cdot \frac{1}{4}
+ \frac{1}{4} \cdot \frac{1}{2}
+ \frac{1}{8} \cdot 1
= 4 \cdot \frac{1}{8} = \frac{1}{2}
\]
The coefficient of $x^n$ is
\[
\sum_{k=0}^n \frac{1}{2^k} \cdot \frac{1}{2^{n-k}}
= \sum_{k=0}^n \frac{1}{2^k \cdot 2^{n-k}}
= \sum_{k=0}^n \frac{1}{2^n}
= \frac{n + 1}{2^n}
\]

(b)
First let's derive the coefficient of $x^n$ in general. 
The coefficient of $x^n$ is
\begin{align*}
\sum_{k=0}^n \frac{1}{2^k} \cdot \frac{1}{3^{n-k}}
&= \sum_{k=0}^n \frac{1}{2^k} \cdot \frac{3^k}{3^n} 
= \frac{1}{3^n} \sum_{k=0}^n \left(\frac{3}{2}\right)^k \\
&= \frac{1}{3^n} \cdot \frac{1 - (3/2)^{n+1}}{1 - 3/2} \\
&= \frac{1}{3^n} \cdot \frac{1 - (3/2)^{n+1}}{-1/2} \\
&= \frac{1}{3^n} \cdot \frac{(3/2)^{n+1} - 1}{1/2} \\
&= \frac{2}{3^n} \cdot \left( \frac{3^{n+1}}{2^{n+1}} - 1 \right) \\
&= 2 \cdot \left( \frac{3^{n+1} - 2^{n+1}}{2^{n+1}3^n} \right) \\
&= \frac{3^{n+1} - 2^{n+1}}{6^n}
\end{align*}
The coefficient of $x^3$ is
\[
\frac{3^4 - 2^{4}}{6^4} = \frac{65}{216}
\]



\newpage

Solution to Exercise \ref{ex:power-series-15}\labeltext{}{sol:power-series-15}.

\debug{\tinysidebar{exercises/{power-series-15/answer.tex}}}
We have
\begin{align*}
\left( \sum_{n=0}^\infty x^n \right)^{100} 
&= \left( \frac{1}{1 - x} \right)^{100} \\
&= \sum_{n=0}^\infty \binom{100 + n - 1}{n} x^n \\
&= \sum_{n=0}^\infty \binom{n + 99}{n} x^n \\
&= \sum_{n=0}^\infty \binom{n + 99}{99} x^n
\end{align*}
Hence the coefficient of $x^n$ is $\binom{n + 99}{99} x^n$ for $n \geq 0$.


\newpage

Solution to Exercise \ref{ex:power-series-16}\labeltext{}{sol:power-series-16}.

\debug{\tinysidebar{exercises/{power-series-16/answer.tex}}}

We have
\begin{align*}
&\left( 
2 + 5x + \frac{7}{1 - x}
\right)
\left( \sum_{n=0}^\infty x^n \right)^{100}
\\
&= \left( 
2 + 5x + \frac{7}{1 - x}
\right)
\left( \frac{1}{1 - x} \right)^{100}
\\
&=
2\left( \frac{1}{1 - x} \right)^{100}
+ 5x \left( \frac{1}{1 - x} \right)^{100}
+ \frac{7}{1 - x} \left( \frac{1}{1 - x} \right)^{100}
\\
&=
2 \sum_{n=0}^\infty \binom{100 + n - 1}{n} x^n
+ 5x \sum_{n=0}^\infty \binom{100 + n - 1}{n} x^n
+ 7 \left( \frac{1}{1 - x} \right)^{101} 
\\
&=
2 \sum_{n=0}^\infty \binom{n + 99}{n} x^n
+ 5x \sum_{n=0}^\infty \binom{n + 99}{n} x^n
+ 7 \sum_{n=0}^\infty \binom{101 + n - 1}{n}
\\
&=
\sum_{n=0}^\infty 2 \binom{n + 99}{99} x^n
+ \sum_{n=0}^\infty 5 \binom{n + 99}{99} x^{n+1}
+ \sum_{n=0}^\infty 7 \binom{n + 100}{n} x^n
\\
&=
\sum_{n=0}^\infty 2 \binom{n + 99}{99} x^n
+ \sum_{p=1}^\infty 5 \binom{p + 98}{99} x^{p}
+ \sum_{n=0}^\infty 7          \binom{n + 100}{n} x^n  \,\,\, \text{(let $p = n + 1$)}
\\
&=
\sum_{n=0}^\infty 2\binom{n + 99}{99} x^n
+ \sum_{n=1}^\infty 5 \binom{n + 98}{99} x^{n}  
+ \sum_{n=0}^\infty 7 \binom{n + 100}{100} x^n \,\,\,\text{(replace $p$ by $n$)}
\\
&=
2 \binom{99}{99} + \sum_{n=1}^\infty 2\binom{n + 99}{99} x^n
+ \sum_{n=1}^\infty 5\binom{n + 98}{99} x^{n} 
+ 7\binom{100}{100}  + \sum_{n=1}^\infty 7 \binom{n + 100}{100}  
\\
&=
9 +
\sum_{n=1}^\infty
\left( 2\binom{n + 99}{99} 
+  5\binom{n + 98}{99} 
+ 7 \binom{n + 100}{100}
\right) x^n
\end{align*}
Hence the coefficient of $x^n$ is
\[
\begin{cases}
9 & \text{ if } n = 0 \\
\displaystyle 2\binom{n + 99}{99} 
+  5\binom{n + 98}{99} 
+ 7 \binom{n + 100}{100} & \text{ if } n > 0
\end{cases}
\]


\newpage

Solution to Exercise \ref{ex:power-series-17}\labeltext{}{sol:power-series-17}.

\debug{\tinysidebar{exercises/{power-series-17/answer.tex}}}

(a)
\begin{align*}
\sum_{n=0}^\infty \frac{1}{2^n} x^n
\cdot
\sum_{n=0}^\infty \frac{1}{2^n} x^n
&=
\left( \sum_{n=0}^\infty \frac{1}{2^n} x^n \right)^2
\\
&=
\left( \sum_{n=0}^\infty \left( \frac{x}{2} \right)^n \right)^2
\\
&=
\left( \sum_{n=0}^\infty \left( \frac{x}{2} \right)^n \right)^2
\\
&=
\left( \frac{1}{1 - (x/2)} \right)^2
\\
&=\sum_{n=0}^\infty \binom{2 + n - 1}{n} \left( \frac{x}{2} \right)^n
\\
&=\sum_{n=0}^\infty \binom{n + 1}{n} \left( \frac{1}{2} \right)^n x^n
\\
\sum_{n=0}^\infty \left( \frac{n + 1}{2^n} \right x^n
\end{align*}

(b)
\begin{align*}
\sum_{n=0}^\infty \frac{1}{2^n} x^n
\cdot
\sum_{n=0}^\infty \frac{1}{3^n} x^n
\end{align*}
    


