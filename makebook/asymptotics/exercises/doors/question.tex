%-*-latex-*-
\tinysidebar{\debug{exercises/{doors/question.tex}}}
* (The doors problem)
Remember this problem from CISS245?
Suppose you are in a mansion with $n$ rooms,
numbered $0, 1, 2, ..., n - 1$.
\begin{enumerate}[nosep]
\li You run from door $0$ to door $n - 1$ and open all of them.
\li Next, you run from door $n - 1$ to door $0$ and close every
other door.
\li Next, you run from door $0$ to door $n - 1$
opening every third door.
\li Next, you run from door $n - 1$ to $0$,
closing every fourth door.
\li Etc. You stop if a run would only open or close one door.
\end{enumerate}
The question is this: When the above process ends, how
many doors are open?
Here’s an example when $n = 10$.
\begin{enumerate}[nosep]
\li You run from door $0$ to door $n - 1 = 9$ and open all of
them, i.e., you open doors $0, 1, 2, 3, 4, 5, 6, 7, 8, 9$.
\li Then you close doors $9, 7, 5, 3, 1$.
\li Then you open doors $0, 3, 6, 9$.
\li Then you close doors $9, 5, 1$.
\li Then you open doors $0, 5$.
\li Then you close doors $9, 3$.
\li Then you open doors $0, 7$.
\li Then you close doors $9, 1$.
\li Then you open doors $0, 9$.
\li And you stop because at this stage if you want to close doors,
you can only close door $9$.
\end{enumerate}
Write a function to do the above where
you use a boolean array \verb!open[0..n-1]! so that
door $i$ is open iff \verb!open[i]! is \verb!true!.
The function returns the number of doors which are open when the above
process ends.
\begin{myenum}
\item
What is the runtime of your code?
Besides using $n^k$ for your big-O, you can
also use $\lg$ (log base 2).
The sharper the bound the better.
\item
Can you design an algorithm that is faster and uses less memory?
\end{myenum}
