%-*-latex-*-
THIS IS DEFUNCT
CHECK IF ANY INFO LEFT SHLD BE MOVED BEFORE DELETING

\chapter{Asymptotics}

\section{Algorithmic Analysis}





\section{Definitions}

In the case of asymptotics for algorithmic analysis, we are
interested in the measurement of complexity for an algorithm when
the size of the problem grows unboundedly, i.e., when $n
\rightarrow \infty$ where $n$ is a measure of the problem size.
For instance for the problem where you want to size a collection
of integers, the number of integers is your collection is the size
of the problem. For the Tower of Hanoi problem, the number of
disks is the size of problem.


There are several different ways of comparing functions: Let
$g(n)$ be a function. We want to define a set $O(g)$ of functions. 
\[ f \in O(g) \]
iff 
\[
\text{there exist $C$ and $N$ such that $|f(n)| \leq C|g(n)|$ 
for $n \geq N$}
\]

Yikes! What's all the mumbo-jumbo??? Let's do an example. 
Suppose $g(n) = n$. Consider $f(n) = n$. I claim that $f(n) \in O(g)$. 
Well, to say that I have to show you that
\[
\text{there exist $C$ and $N$ such that $|f(n)| \leq C|g(n)|$ 
for $n \geq N$}
\]
Putting in the definition of $g(n)$ and $f(n)$, I need to show you
\[
\text{there exist $C$ and $N$ such that $|n| \leq C|n+1|$ 
for $n \geq N$}
\]
In other words I need to find some numbers $C$ and $N$ satisfying
\[
\text{$|n| \leq C|n+1|$ for $n \geq N$}
\]
Well if I choose $C = 5$ and $N = -3$, then I want to know if 
the following is true:
\[
\text{$|n| \leq 5|n+1|$ for $n \geq -3$}
\]
That's a terrible choice! Because I could have chosen a positive $N$ and then for $n > N$, the condition
\[
\text{$|n| \leq C|n+1|$ for $n \geq N$}
\]
becomes
\[
\text{$n \leq C(n+1)$ for $n \geq N$}
\]
since $n > N$ and $N$ is positive implies that $n$ (and therefore $n+1$ as 
well) must be positive. In other words I don't really need the absolute values at all.

OK. So first we insist that we choose a positive $N$ (it can be $N = 0$ or $N = 42$, etc., we be specific later.) So we need to find a $C$ and a positive $N$ such that
\[
\text{$n \leq C(n+1)$ for $n \geq N$}
\]

Duh! Wait a minute ... this statement
\[
\text{$n \leq n+1$}
\]
is true for any positive $n$! So let's choose $C = 1$ and $N = 0$. Then
\[
|n| \leq 1 \cdot |n+1| \text{ for $n > 0$ } 
\]
Hence $f(n) \in O(g)$.

What about the function $f(n) = \frac{1}{2} n + 3$? Again to simplify by removing the absolute values, I will choose $N > 0$. I need to find $C$ and $N > 0$ such that 
\[
\text{$n \leq C \left( \frac{1}{2} n + 3 \right)$ for $n \geq N$}
\]
Well I can choose $C = 2$. (Do you see why?) In that case the above becomes
\[
\text{$n \leq 2 \left( \frac{1}{2} n + 3 \right) = n + 6$ for $n \geq N$}
\]
which is true for $N = 0$. Right?

What about $f(n) = n + 100$? I want to find a $C$ and an $N$ such that
\[
\text{$|n + 100| \leq C|n|$ for $n \geq N$}
\]
In that case the above becomes
\[
\text{$n + 100 \leq Cn$ for $n \geq N$}
\]
In this case choosing $C = 1$ won't work because the above becomes
\[
\text{$100 \leq 0$ for $n \geq N$}
\]
Arghh!

What if I choose $C = 2$? Is the following true?
\[
\text{$n + 100 \leq 2n$ for $n \geq N$}
\]
This is the same as
\[
\text{$100 \leq n$ for $n \geq N$}
\]
Right? 

AHA! If I insist $N = 100$, then this fact must be true. And so we just reverse our steps ... here's the formal proof:

We choose $N = 100$ and $C = 2$. Then for $n \geq N = 100$
we have the following:
\begin{align*}
100 &\leq n \\
\THEREFORE n + 100 &\leq n + n \\
\THEREFORE n + 100 &\leq 2n \\
\THEREFORE n + 100 &\leq Cn \\
\THEREFORE |n + 100| &\leq C|n| \\
\THEREFORE |f(n)| &\leq C|g(n)|
\end{align*}
Hence $f(n) \in O(g)$.
 
\begin{enumerate}
 \item $f \in O(g)$ iff there exist $C$ and $N$ such that $|f(n)| \leq C|g(n)|$ for $n \geq
 N$
 \item $f \in \Omega(g)$ iff there exist $C$ and $N$ such that $C|g(n)| \leq |f(n)|$ for $n \leq N$
 \item $f \in \Theta(g)$ iff $f \in O(g)$ and $f \in \Omega(g)$.
\end{enumerate}

Note that for algorithmic analysis, the functions measure the time
or space required to carry out an algorithm, therefore the
functions are positive. So the absolute value sign is actually
redundant for us.


Another thing to note is that the domain of our functions are
almost always $\N$. Such functions are called
\defterm{arithmetic functions}.
The name come from Number Theory. In this context, the
independent variable is usually written as $n$ instead of $x$.


OK, now for some notational fudging. I will write
\[
  f = O(g)
\]
I simply mean $f \in O(g)$. This is the actual notation used for
equations involving $O$ and $\Theta$. In particular the following
expression
\[
 O(g) = f
\]
does {\it not} make sense.

Actually there is something I did fudge (but I don't feel bad
because this is done by everyone). In general you want to analyze
functions when the value of the independent variable approaches a
value. For instance when I say
\[
 3n^3 + 4 = O(n^3)
\]
I should really say
\[
 3n^3 + 4 = O(n^3) \text{ as } n \rightarrow \infty
\]
For us because we're analyzing algorithms we are only interested
in the growth of functions as $n$ approaches infinity so I will
omit the condition to simplify notation. With the condition, we
can even say things like
\[
 3x^2 + 4x = O(x) \text{ as } x \rightarrow 0
\]
where in this case we're interested in real values of $x$ near to
0. By the way why did I write $O(x)$ and not $O(x^2)$.

\section{More Definitions}

Well, so far for a given function $f$ we defined sets $O(f)$,
$\Omega(f)$ and $\Theta(f)$. However if you read more Math/CS
books you notice that they don't treat these as sets! First of all
you'd see a statement like:
\[
   3n^2 + 4n + 4 = O(n^2)
\]
For us we would write
\[
   3n^2 + 4n + 4 \in O(n^2)
\]
However there's something even more bizarre. You might see
something like this:
\[
   f(n) = 3n^2 + 4n + 5 + 6 \ln n = 3n^2 + O(n)
\]
What does this mean?!? Seems like the author is trying to add a
function $3n^2$ with a set $O(n)$. Basically the author is trying
to say that he cares only about $3n^2$ and the smaller terms are
bunched up as some function in $O(n)$. If he wants to be more
accurate he can also say
\[
  f(n) = 3n^2 + 4n + O(\ln n)
\]


In general let's look at an expression of the form
\[
  f_1(n) + O(g_1(n)) = f_2(n) + O(g_2(n))
\]
First of all let's define $f_1(n) + O(g_1(n))$. This is the sum of
a function and a set. Basically you just add $f_1$ to all the
functions in $O(g_1(n))$. More generally if you have a function
$f$ and a set of functions $S$, then we define
\[
 f + S = \{ f + g \,|\, g \in S \}
\]

With this definition,
\[
 f_1(n) + O(g_1(n)) = f_2(n) + O(g_2(n))
\]
is the same as saying
\[
 \biggl\{ f_1 + g \,\,\biggl|\,\, g \in O(g_1(n)) \biggr\}
 \subseteq
 \biggl\{ f_2 + g \,\,\biggl|\,\, g \in O(g_2(n)) \biggr\}
\]
Remember that when it comes to $O$ and $\Theta$ equations, $=$
always means $\subseteq$!!


If you stop to think about it, not only can you define $f + S$
where $f$ is a function and $S$ is a set of functions, you can
also define $S + f$ as $\{ g + f \,|\, g \in S \}$ too. Of course
since $f+g = g + f$, we have $f + S = S + f$. Right? Furthermore,
you can also define $fS$ and $Sf$.


\begin{ex}
Let $S, T, U$ be sets of functions and $f$ be a function. Equality
here is really equality and not $\in$ or $\subseteq$.
\begin{enumerate}
 \item Define $S + T$ and prove that $S + T = T + S$ (duh) and
 $(S+T)+U = S+(T+U)$ (duh).
 Therefore we can write $S+T+U$ without fear of ambiguity or
 lawsuits or flamemail. Of course with your definition we can just
 define $f+S$ as ${f} + S$.
 \item Define $fS$ and $Sf$. Prove that $fS = fS$ (duh).
 \item Define $ST$. Prove that $ST = TS$ and $(ST)U = S(TU)$
 (two duhs). So again we can write $STU$.
\end{enumerate}
\end{ex}


\begin{ex}
Try this out yourself: What does this means
\[
 f + O(g) + O(h) = a + O(b) + O(c)
\]
where $f,g,h,a,b,c,$ are functions. Don't forget that in this case
$=$ means $\subseteq$.\end{ex}

\section{Basic Facts}

Here are some basic facts you should know. For the following $f$,
$g$, $f_i$, and $g_i$ are functions.
\begin{enumerate}
 \item $O(cf) = O(f)$, $\Omega(cf) = \Omega(f)$, $\Theta(cf) =
 \Theta(f)$
 \item $\Omega(f) \subseteq O(f) \cap \Theta(f)$
 \item If $f_i = O(g_i)$ $(i=1,2)$ then
 \[
  f_1+f_2 = O \bigl( \max(|g_1|,|g_2|) \bigr), \,\,\, f_1 f_2 \in O(g_1 g_2)
 \]
 \item $f_i = O(g)$ $(i=1,2)$ $\implies$ $f_1+f_2 = O(g)$
 \item $f \in O(g)$ $\implies$ $O(f) \subseteq O(g)$.
\end{enumerate}


\begin{thm} \mbox{}
 \begin{enumerate}
  \item If $p$ is a polynomial of degree $d$, then $p = \Theta(n^d)$
  \item $n^d \in O(n^e)$ if $1 \leq d \leq e$.
  \item $n^d \in O(c^n)$ for any constant $c>0$
  \item $\log_a n \in O(n)$
 \end{enumerate}
\end{thm}


Personally, I prefer to view the above as relations. Write $f
\equiv_O g$ iff $f \in O(g)$. Note that $\equiv_O$ and
$\equiv_\Omega$ are not symmetric. $\equiv_O$ and $\equiv_\Omega$
are however reflexive and transitive.
\[
 f \equiv_O f, \,\,\, f \equiv_O g \equiv_O h \implies f \equiv_O
 h
\]



\section{Bubblesort}

Here's bubblesort from CISS240. Suppose $\texttt{a[i]}$ $(i=0,
$\ldots$, n-1)$ is an array of numbers (integers or floats or
doubles, we don't care). The following sorts $\texttt{a[i]}$
$(i=0,\ldots,n-1)$ in ascending order:
\begin{verbatim}
    for i = 0 to n-2
        for j = 0 to i
            if a[i] < a[i+1]:
                t = a[i]
                a[i] = a[j]
                a[j] = t
\end{verbatim}


Analyze the time complexity of the above algorithm and give the
best asymptotics (i.e., give $\Theta$ is possible, if not give
$O$).

\begin{verbatim}
           i = 0
    LOOP1: j = 0
    LOOP2: if a[i] < a[j+1]:
               t = a[i]
               a[i] = a[j]
               a[j] = t
           if j <= i: goto LOOP2
           if i <= n-2: goto LOOP1
\end{verbatim}


If you don't like goto statements and prefer to write for-loops,
then just remember that for a for-loop, if the body is executed
$n$ times, then the initialization happens only once while the
boolean condition run $n+1$ times ($n$ times for each iteration of
the body of the for-loop and once more to kick out of the
for-loop) and the update statement (getting the next $i$) happens
$n$ times. Right?


\section{General Fact on Asymptotics}

Investigate the following question: Suppose $f \in O(g)$. Is it
true that $h^f \in O(h^g)$? Of course the function $h^f$ is just
given by $(h^f)(n) = h(n)^{f(n)}$.

