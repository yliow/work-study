%-*-latex-*-
\section{Some Shorthand Notation and Computational Tools}

The purpose of this section is to give you 
the first bunch of theorems (computational
tools) and a shorthand notation to begin computing big-O.
(There will be more powerful theorems later.)
I'll prove the theorems later. 
Right now, I just want to make sure that you know how to use the tools.

For a chain of inequalities, you have seen that if
\[
a \leq b \text{ and } b \leq c
\] 
then 
\[
a \leq c
\]

You have something similar in big-O:

\begin{thm}
If 
\[
f(n) = O(g(n)) \text{ and } g(n) = O(h(n))
\]
then
\[
f(n) = O(h(n))
\]
\end{thm}

I'll prove this later. 
Informally, you see that if $f(n)$ is bounded above
by a multiple of $g(n)$ (for large $n$) and 
$g(n)$ is bounded above by $h(n)$ (for large $n$),
then $f(n)$ is bounded by a multiple of $h(n)$ (for large $n$).
It's not too shocking.

For instance, if you know that
\[
5n = O(3n^2 + 10) \text{ and }
3n^2 + 10 = O(2n^5 + 1)
\]
then you can conclude
\[
5n = O(2n^5 + 1)
\]
Although the above deduction is true, it's kind of lengthy to write.
Instead of saying:
\[
\text{\lq\lq}
5n = O(3n^2 + 10) \text{ and } 3n^2 + 10 = O(2n^5 + 1)
\text{''}
\]
you can say
\[
\text{\lq\lq}
5n = O(3n^2 + 10) = O(2n^5 + 1)
\text{''}
\]
They mean the same thing.
With this notational shorthand let me repeat what I said earlier ...

If you know that
\[
5n = O(3n^2 + 10) = O(2n^5 + 1)
\]
then you can conclude
\[
5n = O(2n^5 + 1)
\]

This is perfectly legal (mathematically speaking) and is understood
by all mathematicians and computer scientists.

In general something like
\[
f(n) = O(g(n)) = O(h(n)) = O(i(n))
\]
or 
\begin{align*}
f(n) 
&= O(g(n)) \\ 
&= O(h(n)) \\
&= O(i(n))
\end{align*}
is the same as
\begin{align*}
f(n) = O(g(n)) \\
g(n) = O(h(n)) \\
h(n) = O(i(n)) \\
\end{align*}
Note that the above is not a theorem.
It's a notational shorthand.
It's similar to the shorthand that
\[
a \leq b \leq c
\]
means
\[
a \leq b \text{ and } b \leq c
\]

Here's another fact that you can use:
\begin{thm}\mbox{}
\begin{itemize}
\item[(a)] If $f(n) = O(g(n))$, then
\[
f(n) + g(n) = O(g(n))
\]
\item[(b)]
If $f(n) = O(g(n))$
then
\[
f(n) + h(n) = O(g(n) + h(n))
\]
\end{itemize}
\end{thm}

This basically says that if a multiple of 
$g(n)$ grows faster than $f(n)$, then
the growth of $f(n) + g(n)$ is basically determined by $g(n)$.
The second part says this: When computing big-O,
you can replace a term by the big-O of that term.

Here's another fact that should not shock you.
Since we ignore constants, the following must be true:

\begin{thm}
If $c \neq 0$ is a constant then
\[
cf(n) = O(f(n))
\]
This include the obvious fact that
\[
f(n) = O(f(n))
\]
\end{thm}

Let me collect all the above facts in one place ...

\newpage


\begin{thm} \mbox{}
\begin{itemize}
\item[(a)] $f(n) = O(g(n)), g(n) = O(h(n))$ $\implies$ $f(n) = O(h(n))$
\item[(b)] $f(n) = O(g(n))$ $\implies$ $f(n) + g(n) = O(g(n))$
\item[(c)] $f(n) = O(g(n))$ $\implies$ $f(n) + h(n) = O(g(n) + h(n))$
\item[(d)] $cf(n) = O(f(n))$ 
\item[(e)] $f(n) = O(f(n))$ 
\item[(f)] $f(n) = O(h(n))$, $g(n) = O(h(n))$ $\implies$
           $f(n) + g(n) = O(h(n))$ 
\end{itemize}
The above are theorems.
The following is a notational convention:
\begin{itemize}
\item[(e)] The statement
\[ 
f(n) = O(g(n)) = O(h(n))
\]
or
\begin{align*}
f(n) &= O(g(n)) \\
     &= O(h(n))
\end{align*}
means
\[
f(n) = O(g(n)) \text{ and } g(n) = O(h(n))
\]
\end{itemize}
\end{thm}
\qed

That's enough for now.
I'll give you more powerful computational tools.
Let me show you how to use the theorems and notational shorthand.

\newpage

\begin{eg}
Given
\begin{align*}
n       &= O(n^2)  \tag{1}\\
1       &= O(n)    \tag{2}
\end{align*}
prove that
\[
10n^2 + 5n + 100 = O(n^2)
\]
\end{eg}

[Note that the given facts
$n = O(n^2)$ and 
$1 = O(n)$
are in fact true and can be proven easily.
I'll prove all the related facts about polynomial big-O in detail later.]

\textit{Solution.}
We make the following deductions that will be used later:
\begin{align*}
10n^2 &= O(n^2) & & \text{by Theorem 4d} \tag{3} \\
5n    &= O(n)   & & \text{by Theorem 4d} \tag{4} \\
100   &= O(1)   & & \text{by Theorem 4d} \tag{5}
\end{align*}
Now we prove the given statement:
\begin{align*}
&10n^2 + 5n + 100 \\
&= O(10n^2 + 5n + 100) & & \text{by Theorem 4e} \\
&= O(n^2 + 5n + 100)   & & \text{by (3), Theorem 4c} \\
&= O(n^2 + n + 100)    & & \text{by (4), Theorem 4c} \\
&= O(n^2 + n^2 + 100)  & & \text{by (1), Theorem 4c} \\
&= O(2n^2 + 100)       \\
&= O(2n^2 + 1)         & & \text{by (5), Theorem 4c}\\
&= O(2n^2 + n)         & & \text{by (2), Theorem 4c}\\
&= O(2n^2 + n^2)       & & \text{by (1), Theorem 4c}\\
&= O(3n^2)             \\
&= O(n^2)              & & \text{by Theorem 4d}
\end{align*}
By Theorem 4a, we conclude that
\[
10n^2 + 5n + 100 = O(n^2)
\]
\qed

Note that there is no justification here:
\begin{align*}
...\\
&= O(n^2 + n^2 + 100)  & & ... \\
&= O(2n^2 + 100)       \\
...
\end{align*}
because that's just basic algebra.

Note also that there is more than one way to reach the 
the conclusion.
(Just like programming or a game of chess.)
Can you find another proof?
Maybe a shorter (and correct!) one?

Here's another solution to the same problem ...

\newpage

\begin{eg}
Given
\begin{align*}
n       &= O(n^2)  \tag{1}\\
1       &= O(n)    \tag{2}
\end{align*}
prove that
\[
10n^2 + 5n + 100 = O(n^2)
\]
\end{eg}

\textit{Solution.}
First we analyze the terms of the given sum.
The first term gives us
\begin{align*}
10n^2 = O(n^2) & & \text{by Theorem 4d} \tag{3}
\end{align*}
The second term gives us
\begin{align*}
5n 
&= O(n)   & & \text{by Theorem 4d} \\
&= O(n^2) & & \text{by (1)}     
\end{align*}
which implies
\begin{align*}
5n = O(n^2) & & \text{by Theorem 4a} \tag{4}
\end{align*}
And the third term gives us
\begin{align*}
100
&= O(1)   & & \text{by Theorem 4d} \\
&= O(n)   & & \text{by (2)}        \\
&= O(n^2) & & \text{by (1)}        
\end{align*}
which implies
\begin{align*}
100 &= O(n^2) & & \text{by Theorem 4a} \tag{5}
\end{align*}
Using the above results, we have
\begin{align*}
  10n^2 + 5n &= O(n^2) & & \text{by (3), (4), Theorem 4f} \tag{6}
\end{align*}
and
\begin{align*}
10n^2 + 5n + 100 &= O(n^2) & & \text{by (5), (6), Theorem 4f}
\end{align*}
\qed


Make sure you study the above two proofs.

One more example ...

\newpage


\begin{eg}
Given
\begin{align*}
n       &= O(n^2)  \tag{1} \\
n^{1.9} &= O(n^2)  \tag{2} \\
\sin n  &= O(1)    \tag{3} \\
\ln n   &= O(n)    \tag{4} \\
1       &= O(n)    \tag{5}
\end{align*}
show that
\[
n^{1.9} + 10\sin n + 5\ln n = O(n^2)
\]
\end{eg}

\textit{Solution.}
We have
\begin{align*}
10 \sin n 
&= O(\sin n) & & \text{by Theorem 4d} \\
&= O(1)      & & \text{by 3} 
\end{align*}
Therefore by Theorem 4a
\[
10\sin n = O(1) \tag{5}
\] 

We also have
\begin{align*}
5 \ln n = O(\ln n)  & & \text{by Theorem 4d} \tag{6}
\end{align*}

Therefore
\begin{align*}
&n^{1.9} + 10\sin n + 5 \ln n  \\
&= O(n^{1.9} + 10\sin n + 5 \ln n) & & \text{by Theorem 4e} \\
&= O(n^2 + 10\sin n + 5 \ln n)     & & \text{by (2), Theorem 4c} \\
&= O(n^2 + 1 + 5 \ln n)            & & \text{by (2), Theorem 4c}\\ 
&= O(n^2 + n + 5 \ln n)            & & \text{by (5), Theorem 4c}\\ 
&= O(n^2 + n^2 + 5 \ln n)          & & \text{by (1), Theorem 4c} \\
&= O(2n^2 + 5 \ln n)               \\
&= O(2n^2 + n)                     & & \text{by (6), Theorem 4c}\\
&= O(2n^2 + n^2)                   & & \text{by (1), Theorem 4c}\\
&= O(3n^2)                         \\
&= O(n^2)                          & & \text{by Theorem 4d}
\end{align*}
Therefore by Theorem 4a
\[
n^{1.9} + 10\sin n + 5n \ln n = O(n^2)
\]


[EXERCISES]
