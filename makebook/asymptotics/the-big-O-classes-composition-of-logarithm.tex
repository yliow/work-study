%-*-latex-*-
\section{The big-O Classes: Composition of Logarithm}

What about $\ln^{(k)} n$?
Here's a plot of $n, \ln^{(1)} n, \ln^{(2)} n$ 
(you do see quickly just from definition that $\ln^{(1)} n = \ln n$, right?)

\begin{python}
from latextool_basic import *
p = FunctionPlot(line_width=1, width='5in', height='4in', domain='0:50')

p.add('x', color='red', python=1, pin='above', pin_message='', num_points=2)
p.add('log(x,e)', color='blue', python=1, pin='above', pin_message='\ln^{(1)} x')
p.add('log(log(x,e),e)', color='blue', python=1, pin='below', pin_message='\ln^{(2)}', num_points=300)
print(p)
\end{python}

\begin{ex}
Why is it not surprising that $\ln^{(1)} n$ beats $\ln^{(2)} n$?
\qed
\end{ex}

Let's forget about the plot for $n$ and just look at the plots for
$\ln^{(k)} n$ for $k=1, 2, 3, 4$:

\begin{python}
from latextool_basic import *
p = FunctionPlot(line_width=1, width='5in', height='4in', domain='0:50',
                 num_points=500)
p.add('log(x,e)', color='blue', python=1, pin='above', pin_message='')
p.add('log(log(x,e),e)', color='blue', python=1, pin='below', pin_message='')
p.add('log(log(log(x,e),e),e)', color='blue', python=1, pin='below', pin_message='')
p.add('log(log(log(log(x,e),e),e),e)', color='blue', python=1, pin='below', pin_message='')
print(p)
\end{python}

So we suspect that for large values of $n$:
\[
\ln^{(4)} n < 
\ln^{(3)} n < 
\ln^{(2)} n < 
\ln^{(1)} n = \ln n 
\]

Let's plot the functions again but for a domain of up to $100$ just
to visually verify our guess:
\begin{python}
from latextool_basic import *
p = FunctionPlot(line_width=1, width='5in', height='4in', domain='0:100',
                 num_points=500)
p.add('log(x,e)', color='blue', python=1, pin='above', pin_message='')
p.add('log(log(x,e),e)', color='blue', python=1, pin='below', pin_message='')
p.add('log(log(log(x,e),e),e)', color='blue', python=1, pin='below', pin_message='')
p.add('log(log(log(log(x,e),e),e),e)', color='blue', python=1, pin='below', pin_message='')
print(p)
\end{python}

In other words in $O(n^r) \subset O(n^{r+1})$ we expand 
to get
\begin{align*}
\cdots
\subset O(n^r \ln^{(3)})
\subset O(n^r \ln^{(2)})
\subset O(n^r &\ln^{(1)}) \\ 
&\| \\
O(n^r &\ln n) 
\subset  O(n^r\ln^2 n) 
\subset  O(n^r\ln^3 n)
\subset \cdots 
\end{align*}
