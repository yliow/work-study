%-*-latex-*-
\sectionthree{Experimental (or Empirical) Approach}
\begin{python0}
from solutions import *; clear()
\end{python0}

Here's how to think of probability intuitively:
You can of course think of probability as some kind of averaging.
In the case of tossing a particular coin, if you say
\[
p(\HEAD) = \frac{1}{3}
\]
what you meant is that if you toss the coin 3 times, then
it's likely that 
approximately 1 out of the 3 is a head;
if you toss the coin 300 times, then it's very likely that
approximately 100 of the 300 are heads;
if you toss the coin 3000 times, then it's very very likely that
approximately 1000 of the 3000 are heads; etc.
The more tosses you make, the closer you get to \lq\lq one third of the
tosses comes out head''.

In the above coin toss experiment, I have two possible outcomes:
either I get a head or I get a tail.
I create two symbols to represent these two possible outcomes:
$\HEAD$ and $\TAIL$.
The names is arbitrary and it's entirely up to you to come up with
symbols for all the possible outcomes.
For instance you can name the outcomes H and T instead of
$\HEAD$ and $\TAIL$.

How let's formalize the notation for studying probability theory ...

