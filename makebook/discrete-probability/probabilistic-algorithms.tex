%-*-latex-*-
% https://www.cs.umd.edu/users/meesh/cmsc351/mount/lectures/lec15-quicksort.pdf
\sectionthree{Randomized/probabilistic algorithms}
\begin{python0}
from solutions import *; clear()
\end{python0}


A
\defone{randomized algorithm}
or
\defone{probabilistic algorithm}
is an algorithm that uses
randomization.

There are two types of randomized/probabilistic algoroithms.

A \defone{Las Vegas} randomized algorithm uses randomization of input.
The output is correct.
What is changed, because of the randomization of input, is the expected runtime.
An example is randomized quicksort where the choice is pivot is randomized.

A \defone{Monte Carlo} randomized algorithm might produce an incorrect result,
but with a very small probability.
An example is the Miller-Rabin primality testing algorithm.
In the case of primality testing, several rounds of Miller-Rabin will lower
the probability of incorrectness.

Now let's consider quicksort.
Let $T(n)$ be the runtime.
Once a pivot is selected, recall that the
pivot partitions the subarray the quicksort is working on
into two parts (the left and right partitions),
but in general you do not know the sizes of the two partitions.
Therefore the runtime looks like this:
\[
  T(n) = T(0) + T(n - 1) + An + B
\]
or
\[
  T(n) = T(1) + T(n - 2) + An + B
\]
or
\[
  T(n) = T(2) + T(n - 3) + An + B
\]
or
\[
  ...
\]
or
\[
  T(n) = T(n - 1) + T(0) + An + B
\]
depending on how the pivot partitions.
For instance for the first line above, the pivot lands in index 0,
and therefore the left partition has size 0 and the right partition
has size $n - 1$.
Let's assume all the above cases are equally likely.
Adding them up I get
\[
  nT(n) = 2 \cdot \sum_{i = 0}^{n - 1} T(i) + n(An + B)
\]
and therefore
\[
  T(n) = \frac{2}{n} \sum_{i = 0}^{n - 1} T(i) + An + B
\]
Note that technically the $T(n)$ here is the average $T(n)$.

Now I want to show that the above average runtime is $O(n \lg n)$.

Suppose $T(k) \leq k \lg k$ for $1 \leq k < n$.
Then
\begin{align*}
  \frac{2}{n} \sum_{i = 1}^{n - 1} T(i)
  &\leq \frac{2}{n} \sum_{i = 1}^{n - 1} i \lg i 
  \\
  &\leq \frac{2}{n} \sum_{i = 1}^{n - 1} n \lg i 
  \\
  &\leq 2 \sum_{i = 1}^{n - 1} \lg i 
  \\
  &\leq 2 \int_{1}^n \lg x \ dx
  \\
  &\leq 2 C \int_{1}^n x \ln x \ dx \hspace{1cm} \text{change of base}
  \\
  %&\leq 2 C \left( x\ln x \right|_{x=1}^{x=n} - \int_{1}^n x d(\ln x) \right)
  %\\
  %&\leq 2 C \left( n \ln n - \int_{1}^n x (1/x) dx \right)
  %\\
  %&\leq 2 C \left(  n \ln n  - \int_{1}^n 1 dx \right)
  %\\
  %&\leq 2 C \left( n \ln n - (n - 1) \right)
\end{align*}
\begin{comment}
\begin{align*}
  \frac{2}{n} \sum_{i = 1}^{n - 1} T(i)
  &\leq \frac{2}{n} \sum_{i = 1}^{n - 1} i \lg i 
  \\
  &\leq 2 \sum_{i = 1}^{n - 1} \frac{i}{n} \lg i  
  \\
  &= 2 \sum_{i = 1}^{n - 1} \frac{i}{n} \lg \left( \frac{i}{n} \cdot n \right)
  \\
  &= 2 \sum_{i = 1}^{n - 1} \frac{i}{n} \left( \lg \frac{i}{n} + \lg n \right)
  \\
  &= 2 \sum_{i = 1}^{n - 1} \left( \frac{i}{n} \lg \frac{i}{n} + \frac{i}{n} \lg n \right)
  \\
  &= 2 \sum_{i = 1}^{n - 1} \frac{i}{n} \lg \frac{i}{n} + 2\sum_{i=1}^n \frac{i}{n} \lg n
  \\
  &= 2 \sum_{i = 1}^{n - 1} \frac{i}{n} \lg \frac{i}{n}
    +
    \frac{2 \lg n}{n}
    \sum_{i=1}^n i
  \\
  &= 2 \sum_{i = 1}^{n - 1} \frac{i}{n} \lg \frac{i}{n}
    +
    \frac{2 \lg n}{n}
    \cdot
    \frac{n(n - 1)}{2}
  \\
  &= 2 \sum_{i = 1}^{n - 1} \frac{i}{n} \lg \frac{i}{n}
    +
    (\lg n)(n - 1)
  \\
  &= 2 \sum_{i = 1}^{n - 1} \frac{i}{n} \lg \frac{i}{n}
    +
    n\lg n - \lg n
\end{align*}
\end{comment}


\subsection{Average runtime of quicksort using probability}

Assume \verb!x[0..n-1]! values are distinct and I'm going to
perform quicksort on it.
I'm going to assume that the pivot selection is random.

I'm using the partition method mentioned in my notes
where the pivot is placed at the beginning of the
subarray to be sorted.

I'll compute the average runtime this way:
I'm going to count the number of comparisons throughout the \textit{whole} quicksort process,
instead of going through the
stages of recursion.
Again, as before, the above statement means \lq\lq I'm going to count
the \textit{average} number of comparison".

Let $X$ be number of comparisons during quicksort.
So I'm aiming for $\E[X]$.
Let's break it up.
Let $X_{ab}$ be the number of comparison between \verb!a! and \verb!b!
where \verb!a! and \verb!b! are values in the array \verb!x[0..n - 1]!
that I'm sorting.
Then
\[
  X =
  \sum_{a,b \in \texttt{x[0..n-1]}} X_{ab}  
\]
and therefore
\[
  \E[X] =
  \sum_{\texttt{a},\texttt{b} \in \texttt{x[0..n-1]}} \E \left[ X_{\texttt{ab}} \right]  
\]


Note that \verb!a! and \verb!b! is compared at most once.
Why?
Because values are compared \textit{against a pivot} and once a value is
chosen as a pivot, you notice that work is never done on that pivot anymore -- because
the pivot has found its right place during the partitioning algorithm.
Of course it's also possible that \verb!a! and \verb!b! are never chosen as pivot.
In that case the number of comparisong between \verb!a! and \verb!b! is 0.
Therefore
$X_{\texttt{ab}} = 1$ or $0$.
In other words $X_{\texttt{ab}}$ is an indicator random variable.
Therefore
\begin{align*}
  E[X]
  &= \sum_{\texttt{a},\texttt{b} \in \texttt{x[0..n - 1]}} E \left[ X_{\texttt{ab}} \right]
    \\
  &= \sum_{\texttt{a},\texttt{b} \in \texttt{x[0..n - 1]}} \Pr \left[ X_{\texttt{ab}} = 1 \right]
    \\
  &= \sum_{\texttt{a},\texttt{b} \in \texttt{x[0..n - 1]}}
    \text{(probability that \texttt{a} and \texttt{b} are compared)}
\end{align*}

Here's a diagram showing a sample execution of quicksort on \verb!x[0..n - 1]!, focusing on
two elements \verb!x[i]! and \verb!x[j]! where \verb!x[i]! is ultimately
chosen as a pivot:

\begin{python}
from latextool_basic import *
p = Plot()

def cell(label='', background=''):
    return Rect2(x0=0, y0=0, x1=0.8, y1=0.4, label=r'\scriptsize{\texttt{%s}}' % label, background=background)

c = RectContainer(x=1, y=1)
for i in range(10):
    if i == 3:
        c += cell('x[i]')
    elif i == 7:
        c += cell('x[j]')
    else:
        c += cell()

p += c

d = RectContainer(x=1, y=-1)
for i in range(10):
    if i == 8:
        d += cell('x[i]')
    elif i == 6:
        d += cell('x[j]')
    elif i == 2:
        d += cell(r'$p_0$', background='black!10')    
    else:
        d += cell()

p += d

e = RectContainer(x=1, y=-3)
for i in range(10):
    if i == 3:
        e += cell('x[i]')
    elif i == 5:
        e += cell('x[j]')
    elif i == 2:
        e += cell(r'$p_0$', background='black!10')    
    elif i == 8:
        e += cell(r'$p_1$', background='black!10')    
    else:
        e += cell()

p += e

f = RectContainer(x=1, y=-5)
for i in range(10):
    if i == 4:
        f += cell('x[i]', background='black!10')
    elif i == 7:
        f += cell('x[j]')
    elif i == 2:
        f += cell(r'$p_0$', background='black!10')    
    elif i == 8:
        f += cell(r'$p_1$', background='black!10')    
    else:
        f += cell()

p += f

p += Line(points=[c[3].bottom(), d[8].top()], endstyle='>')
p += Line(points=[d[8].bottom(), e[3].top()], endstyle='>')

p += Line(points=[c[7].bottom(), d[6].top()], endstyle='>')
p += Line(points=[d[6].bottom(), e[5].top()], endstyle='>')

p += Line(points=[e[5].bottom(), f[7].top()], endstyle='>')
p += Line(points=[e[3].bottom(), f[4].top()], endstyle='>')

print(p)
\end{python}

After that \verb!x[j]! might change it's position.
But it won't be compared against \verb!x[i]! anymore.
The cells in grey are pivots after the partitioning steps and therefore
will not take part in future partitioning steps.

Here's the diagram of a more complete quicksort on an array of size 10.
I'm drawing the array at the end of each pass, which includes the
pivot selection and the partitioning step.
The pivots are shaded.
\begin{python}
from latextool_basic import *
p = Plot()

def cell(label='', background=''):
    return Rect2(x0=0, y0=0, x1=0.8, y1=0.4, label=r'\scriptsize{\texttt{%s}}' % label, background=background)

c = RectContainer(x=1, y=1)
for i in [5,3,7,2,1,9,4,0,6,8]:
    c += cell(label=i)

p += c

d = RectContainer(x=1, y=0)
for i in [5,3,2,1,4,0,6,7,9,8]:
    if i == 7:
         d += cell(label=i, background='black!10')
    else:
        d += cell(i)

p += d

e = RectContainer(x=1, y=-1)
for i in [2,1,0,3,5,4,6,7,9,8]:
    if i in [7,3]:
        e += cell(i, background='black!10')
    else:
        e += cell(i)

p += e

f = RectContainer(x=1, y=-2)
for i in [0,1,2,3,4,5,6,7,9,8]:
    if i in [7,3,1,4]:
        f += cell(i, background='black!10')
    else:
        f += cell(i)

p += f

g = RectContainer(x=1, y=-3)
for i in [0,1,2,3,4,5,6,7,9,8]:
    if i in [7,3,1,4,5]:
        g += cell(i, background='black!10')
    else:
        g += cell(i)

p += g

h = RectContainer(x=1, y=-4)
for i in [0,1,2,3,4,5,6,7,8,9]:
    if i in [7,3,1,4,5,9]:
        h += cell(i, background='black!10')
    else:
        h += cell(i)

p += h

#p += Line(points=[c[3].bottom(), d[8].top()], endstyle='>')
#p += Line(points=[d[8].bottom(), e[3].top()], endstyle='>')

#p += Line(points=[c[7].bottom(), d[6].top()], endstyle='>')
#p += Line(points=[d[6].bottom(), e[5].top()], endstyle='>')

#p += Line(points=[e[5].bottom(), f[7].top()], endstyle='>')
#p += Line(points=[e[3].bottom(), f[4].top()], endstyle='>')

print(p)
\end{python}
If you look at
\begin{python}
from latextool_basic import *
p = Plot()

def cell(label='', background=''):
    return Rect2(x0=0, y0=0, x1=0.8, y1=0.4, label=r'\scriptsize{\texttt{%s}}' % label, background=background)


h = RectContainer(x=1, y=-4)
for i in [0,1,2,3,4,5,6,7,8,9]:
    if i in [7,3,1,4,5,9]:
        h += cell(i, background='black!10')
    else:
        h += cell(i)

p += h

print(p)
\end{python}
where the pivots are shaded, note that
\verb!0! is compared against the nodes on the path from the root \verb!7!
to \verb!0!, not including \verb!0!.

Here's the binary tree corresponding to the quicksort where the parent nodes
are the pivots:
\begin{python}
from latextool_basic import *
p = Plot()
edges={'7':['3','9'],
 '3':['1','4'],
 '1':['0','2'],
 '4':['5'],
 '5':['','6'],
 '9':['8']}
pos = bintreepositions(edges=edges)

Graph.r = 0.3
for k,(x,y) in pos.items():
    if k in edges.keys():
        p += Graph.node(x=x, y=y, name=k, label=k)
    else:
        p += Graph.node(x=x, y=y, name=k, label=k, background='white')

for k,v in edges.items():
    for x in v:
        if x in [None,'']: continue
        p += Graph.arc(names=[k,x])
print(p)
\end{python}    
The nodes which are shaded are the pivots.
This tree diagram makes it easier to see that \verb!0!
is compared against \verb!1!, \verb!3!, and \verb!7!.

Altogether, the total number of comparisons is the sum of the lengths of all
paths in the above tree.
Therefore \verb!0! is compared against \verb!1!, \verb!3!, and \verb!7!.
Altogether $X$ for this case is
\begin{itemize}
  \li Number of values compared against $7$: 9
  \li Number of values compared against $3$: 6
  \li Number of values compared against $1$: 2
  \li Number of values compared against $4$: 2
  \li Number of values compared against $5$: 1
  \li Number of values compared against $9$: 1
\end{itemize}
Therefore the total number of comparisons is $21$.

Here's a tree where pivots are chosen so that the tree is height-balanced:
\begin{python}
from latextool_basic import *
p = Plot()
edges={'4':['1','7'],
 '1':['0','2'],
 '7':['5','8'],
 '2':['3'],
 '5':['6'],
 '8':['9']}
pos = bintreepositions(edges=edges)

Graph.r = 0.3
for k,(x,y) in pos.items():
    if k in edges.keys():
        p += Graph.node(x=x, y=y, name=k, label=k)
    else:
        p += Graph.node(x=x, y=y, name=k, label=k, background='white')

for k,v in edges.items():
    for x in v:
        if x in [None,'']: continue
        p += Graph.arc(names=[k,x])
print(p)
\end{python}
The number of comparisons is 19.



Analyzing
\begin{align*}
  \E[X]
  = \sum_{\texttt{a},\texttt{b} \in \texttt{x[0..n - 1]}}
    \text{(probability that \texttt{a} and \texttt{b} are compared)}
\end{align*}
where \verb!a! is (say) \verb!x[i]!
and \verb!b! is \verb!x[j]! where \verb!x[0..n - 1]! is before
the execution of quicksort is difficult.
So (and this is important), I'm going to let \verb!x[0..n - 1]! be the
array \textit{after} sorting is done.
Note that
\begin{align*}
  \E[X]
  = \sum_{\texttt{a},\texttt{b} \in \texttt{x[0..n - 1]}}
  \text{(probability that \texttt{a} and \texttt{b} are compared)}
\end{align*}
does not depend on whether
where \verb!a! = \verb!x[i]!
and \verb!b! = \verb!x[j]! are values before or after quicksort.
Remember what I said earlier: for \verb!a! and \verb!b! to be compared,
either \verb!a! or \verb!b! has to be chosen as pivot.
Also, recall that I'm using \verb!x[0..n - 1]! to denote our array
after it is sorted.
Here's the important
observation:


\textsc{Fact}.
\verb!x[i]!, \verb!x[j]! are compared iff the first value in
\verb!x[i..j]! to be chosen as pivot is either \verb!x[i]! or \verb!x[j]!.

$(\impliedby)$:
Easy.
If \verb!x[i]! is the first in \verb!x[i..j]! to be selected as pivot,
then \verb!x[j]! will be compared against \verb!x[i]!.
Likewise if \verb!x[j]! in the first in \verb![i..j]! to be selected as
pivot, then \verb!x[i]! will be compared against \verb!x[j]!.

$(\implies)$:
Assume some \verb!x[k]! where \verb!x[i]! $<$ \verb!x[k]! $<$ \verb!x[j]!
is first selected as pivot among the values \verb!x[i..j]!.
Then \verb!x[i]! and \verb![j]! would have landed in different partitions
after \verb!x[k]! separates them.
After that, they obviously will not be able to compare with each other.

Hence, the probability that \verb!x[i]!
and \verb!x[j]! are compared is the probability
that
the first value in
\verb!x[i..j]! to be chosen as pivot is either \verb!x[i]! or \verb!x[j]!.
There are $j - i + 1$ values in \verb!x[i..j]!.
Assuming equal likelihood of choosing any value
\verb!x[i..j]! to be a pivot,
\begin{align*}
  &\text{probability that \texttt{x[i]} or \texttt{x[j]} is first pivot chosen among \texttt{x[i..j]}}
  \\
  &= \frac{2}{j - i + 1}
\end{align*}
For instance the probability
that \verb!x[i]! or \verb!x[i + 1]!
is first pivot
chosen as pivot along 
\verb!x[i..i + 1]!
is $2/(i + 1 - i + 1) = 2/2 = 1$.
Hence
\begin{align*}
  \E[X]
  &= \sum_{0 \leq i < j \leq n - 1} \frac{2}{j - i + 1}
    \\
  &= 2 \sum_{i = 0}^{n - 2} \sum_{j = i + 1}^{n - 1} \frac{1}{j - i + 1}
    \\
  &= 2 \sum_{i = 0}^{n - 2} \sum_{k = 2}^{n - i} \frac{1}{k} \hspace{1in} \text{(let $k = j - i + 1$)}
    \\
  &\leq 2 \sum_{i = 0}^{n - 2} \sum_{k = 1}^{n} \frac{1}{k}
    \\
  &= 2 \sum_{i = 0}^{n - 2} H_n
    \\
  &= 2 (n - 1) H_n
    \\
  &\leq 2 n H_n
\end{align*}
Since
$\lim_{n \rightarrow \infty} (H_n - \ln n)
= \gamma = 0.577...$,
there is some $N$ such that for $n \geq N$, $H_n \leq \ln n + 1$.
Hence
$H_n = O(\ln n) = O(\lg n)$.
Therefore
\[
\E[X] \leq 2n H_n = O(n \lg n)
\]

To make sure you understand the proof above,
let me look at the quicksort from above:

\begin{python}
from latextool_basic import *
p = Plot()

def cell(label='', background=''):
    return Rect2(x0=0, y0=0, x1=0.8, y1=0.4, label=r'\scriptsize{\texttt{%s}}' % label, background=background)


h = RectContainer(x=1, y=-4)
for i in [0,1,2,3,4,5,6,7,8,9]:
    if i in [7,3,1,4,5,9]:
        h += cell(i, background='black!10')
    else:
        h += cell(i)

p += h

print(p)
\end{python}
and its tree:
\begin{python}
from latextool_basic import *
p = Plot()
edges={'7':['3','9'],
 '3':['1','4'],
 '1':['0','2'],
 '4':['5'],
 '5':['','6'],
 '9':['8']}
pos = bintreepositions(edges=edges)

Graph.r = 0.3
for k,(x,y) in pos.items():
    if k in edges.keys():
        p += Graph.node(x=x, y=y, name=k, label=k)
    else:
        p += Graph.node(x=x, y=y, name=k, label=k, background='white')

for k,v in edges.items():
    for x in v:
        if x in [None,'']: continue
        p += Graph.arc(names=[k,x])
print(p)
\end{python}    

The number of comparison involving \verb!0! is 3.
The \textit{expected} number of comparisons involving \verb!0!
(for all quicksorts) is
given by
\begin{longtable}{|c|c|l|c|}
  \hline
  $i$ & $j$ & \texttt{x[j]} compared $= 2/(i+j-1)$ & $X_{ij}$ \\
  \hline
  0 & 1 & $2/(1 - 0 + 1) = 2/2 = 1$         & 1 \\
  0 & 2 & $2/(2 - 0 + 1) = 2/3 = 0.6666...$ & 0 \\
  0 & 3 & $2/(3 - 0 + 1) = 2/4 = 0.5$       & 1 \\
  0 & 4 & $2/(4 - 0 + 1) = 2/5 = 0.4$       & 0 \\
  0 & 5 & $2/(5 - 0 + 1) = 2/6 = 0.3333...$ & 0 \\
  0 & 6 & $2/(6 - 0 + 1) = 2/7 = 0.2857...$ & 0 \\
  0 & 7 & $2/(7 - 0 + 1) = 2/8 = 0.25$      & 1 \\
  0 & 8 & $2/(8 - 0 + 1) = 2/9 = 0.2222...$ & 0 \\
  0 & 9 & $2/(9 - 0 + 1) = 2/10 = 0.2$      & 0 \\
  \hline \hline
    &   & \textsc{Sum} = $3.8579...$   & \textsc{Sum} = 3 \\
  \hline
\end{longtable}
There are 3 comparisons involving 0 and
the average number of comparisons (over all possible
quicksort) is 3.8579...
For $n = 10$, the expected number of comparisons is
\[
  \E[X] =
  2\sum_{i=0}^{10-2} \sum_{k=2}^{10 - i} \frac{1}{k}  = 32.9
\]

%Mitzenmacher and Upfal
%-*-latex-*-

\begin{ex} 
  \label{ex:prob-00}
  \tinysidebar{\debug{exercises/{disc-prob-28/question.tex}}}

  \solutionlink{sol:prob-00}
  \qed
\end{ex} 
\begin{python0}
from solutions import *
add(label="ex:prob-00",
    srcfilename='exercises/discrete-probability/prob-00/answer.tex') 
\end{python0}


\subsection{Average runtime without probability theory}
Another way to compute the average runtime is to
derive a closed form for $T(n)$ where
\[
  T(n) = \frac{2}{n} \sum_{i = 0}^{n - 1} T(i) + An + B
\]
for sufficiently large $n$.
Here, sufficiently can be, say, $n > 1$, using
$n = 0, 1$ as base case since there's no reason to
use quicksort if the array has size 0 or 1.

There are many ways to proceed from this point.

\subsection{Method A: induction}

The problem with induction is always that you need to know what you need to prove.
In this case, I have to somehow suspect that
$T(n) = O(n \lg n)$.
This can be achieved by doing some timings.
So say I claim that
\[
  T(n) = O(n \lg n)
\]
i.e., there is some $N$ and some $C$ such that
\[
  T(n) \leq Cn \lg n
\]
for $n \geq N$.


Then assuming that
\[
  T(k) \leq C k \lg k
\]
for $k = N, N + 1, ..., n - 1$, I need to prove that
\[
  T(n) \leq C n \lg n
\]
I get
\begin{align*}
  T(n)
  &= \frac{2}{n} \sum_{i = 0}^{n - 1} T(i) + An + B
  \\
  &\leq \frac{2}{n}
    \left( T(0) + B + T(1) + A + B \right)
    +
    \frac{2}{n} \sum_{k = 2}^{n - 1} T(k)
    + An + B
  \\
  &\leq \frac{2}{n}
    \left( T(0) + T(1) + A + 2B \right)
    +
    \frac{2}{n} \sum_{k = 2}^{n - 1} C k \lg k
    + An + B
  \\
  &\leq
    \frac{2C}{n} \sum_{k = 2}^{n - 1} k \lg k
    + \frac{2}{n}
    \left( T(0) + T(1) + A + 2B \right)
    +
    An + B
  \\
\end{align*}
The sum can be bounded this way (converting the sum to an
area computation):
\begin{align*}
  \sum_{k = 2}^{n - 1} k \lg k
  \leq
  \int_{2}^n x \lg x \, dx
\end{align*}
Fortunately we do have a relevant integration formula for this case
which you can find in a calculus book:
\begin{align*}
  \int x \ln x \ dx
  &= \frac{1}{2}x^2 \ln x - \frac{1}{4}x^2 + C 
\end{align*}
Or you can derive the formula using integration-by-parts:
\begin{align*}
  \int x \ln x \ dx
  &= \int \ln x \ d \left( \frac{1}{2} x^2  \right)
  \\
  &= \frac{1}{2}x^2 \ln x - \int \frac{1}{2}x^2 \ d( \ln x )
  \\
  &= \frac{1}{2}x^2 \ln x - \frac{1}{2} \int x^2 \frac{1}{x} \ dx
  \\
  &= \frac{1}{2}x^2 \ln x - \frac{1}{2} \int x \ dx
  \\
  &= \frac{1}{2}x^2 \ln x - \frac{1}{2} \cdot \frac{1}{2} x^2 + C 
  \\
  &= \frac{1}{2}x^2 \ln x - \frac{1}{4}x^2 + C 
\end{align*}
I do have to change the log to base 2:
\begin{align*}
  \int x \lg x \ dx
  &= \frac{1}{2}x^2 \lg x - \frac{\alpha}{4}x^2 + C 
\end{align*}
where $\alpha = 1/\log_e 2$.
Therefore I now have
\begin{align*}
  \sum_{k = 2}^{n - 1} k \lg k
  &\leq
    \int_{2}^n x \lg x \, dx
  \\
  &= \left( \frac{1}{2}x^2 \ln x - \frac{1}{4}x^2 \right)\biggl|_{x=2}^{x=n}
  \\
  &=
    \left( \frac{1}{2}n^2 \ln n - \frac{1}{4}n^2 \right)
    -
    \left( \frac{1}{2}2^2 \lg 2 - \frac{\alpha}{4}2^2 \right) 
  \\
  &=
    \left( \frac{1}{2}n^2 \lg n - \frac{1}{4}n^2 \right)
    -
    \left( 2 \lg 2 - \alpha \right) 
\end{align*}

Going back to $T(n)$:
\begin{align*}
  T(n)
  &\leq
    \frac{2C}{n} \sum_{k = 2}^{n - 1} k \lg k
    + \frac{2}{n}
    \left( T(0) + T(1) + A + 2B \right)
    +
    An + B
  \\
  &\leq
    \frac{2C}{n}
    \left(
        \left( \frac{1}{2}n^2 \lg n - \frac{1}{4}n^2 \right)
    -
    \left( 2 \lg 2 - \alpha \right) 
    \right)
    + \frac{2}{n}
    \left( T(0) + T(1) + A + 2B \right)
    +
    An + B
  \\
  &\leq
    C n \lg n
    + 
    \frac{2C}{n}
    \left(
        \left( - \frac{1}{4}n^2 \right)
    -
    \left( 2 \lg 2 - \alpha \right) 
    \right)
    + \frac{2}{n}
    \left( T(0) + T(1) + A + 2B \right)
    +
    An + B
  \\
\end{align*}
The RHS is
\[
  \leq C n \lg n 
\]
if
\[
  \frac{2C}{n}
    \left(
        \left( - \frac{1}{4}n^2 \right)
    -
    \left( 2 \lg 2 - \alpha \right) 
    \right)
    + \frac{2}{n}
    \left( T(0) + T(1) + A + 2B \right)
    +
    An + B
\]
is $\leq 0$.
In other words, I need
\[
    \frac{2}{n}
    \left( T(0) + T(1) + A + 2B \right)
    +
    An + B
    \leq
    \frac{2C}{n}
    \left(
        \left(\frac{1}{4}n^2 \right)
    +
    \left( 2 \lg 2 - \alpha \right) 
  \right)
\]
i.e.,
\[
    {2}
    \left( T(0) + T(1) + A + 2B \right)
    +
    An^2 + Bn
    \leq
    2C
    \left(
        \left(\frac{1}{4}n^2 \right)
    +
    \left( 2 \lg 2 - \alpha \right) 
  \right)
\]
i.e.,
\[
  0
  \leq
  \left(\frac{C}{2} - A \right) n^2
  - Bn
  +
  \beta
\]
where
$\beta = 2C(2 \lg 2 - \alpha)
- 2 \left( T(0) + T(1) + A + 2B \right)$.
If I choose $C$ such that
$C/2 - A > 0$, then the parabola
function above $(C/2 - A)n^2 - Bn + \beta$
will concave up and
will be $> 0$ for $n > N_0$ for some $N_0$.

Altogether, there is some $N$ and some $C$
such that
\[
  T(n) \leq C n \lg n
\]
for $n \geq N$.

\subsection{Method B: direct derivation}

In the previous section, I have to make a
guess
that $T(n) = O(n \lg n)$.
What if I want to derive the runtime without
guessing?

From the previous section, I have
\[
  T(n) = \frac{2}{n} \sum_{i = 0}^{n - 1} T(i) + An + B
\]
for $n > 1$.

Suppose I want to derive the
a closed-form for $T(n)$.
I'm in trouble because the recursion
is not a linear degree 2 recursion and it does not
fit the form of the master theorem.
Not only is it not degree 2, in fact, it's degree $n$ --
i.e., the degree is not fixed.

Don't panic. First do this:
\[
  nT(n) = 2\sum_{i = 0}^{n - 1} T(i) + An^2 + Bn
\]
Therefore, with $n$ replaced by $n - 1$, I get
\[
  (n-1)T(n-1) = 2\sum_{i = 0}^{n-2} T(i) + A(n-1)^2 + B(n-1)
\]
Subtracting:
\[
  nT(n) - (n-1)T(n-1)= 2T(n-1) + A(n^2 - (n - 1)^2) + B(n - (n - 1))
\]

So what? ... 
well I have removed the $\sum$ so that I now have
\begin{align*}
  nT(n)
  &= (n + 1)T(n - 1) + A(2n + 1) + B \\
  &= (n + 1)T(n - 1) + 2An + (A + B) \\
  &= (n + 1)T(n - 1) + A'n + B' \\
    \THEREFORE
  T(n)
  &= \left( 1 + \frac{1}{n} \right) T(n - 1) + A' + \frac{B'}{n}
\end{align*}
which is a degree 1 recurrence, although unfortunately it's not
linear.
There are many ways to solve this ...

\subsection{Method B1: telescoping trick}

From the above, relation
\begin{align*}
  T(n)
  &= \left( \frac{n + 1}{n} \right) T(n - 1) + A' + \frac{B'}{n}
\end{align*}
I get
\begin{align*}
  \frac{1}{n + 1}T(n)
  &= \frac{1}{n}  T(n - 1) + \frac{A'}{n+1} + \frac{B'}{n(n + 1)}
\end{align*}
and therefore
\begin{align*}
  \frac{1}{n + 1} T(n) - \frac{1}{n}  T(n - 1) = \frac{A'}{n} + \frac{B'}{(n-1)n}
\end{align*}

Therefore I get
\begin{align*}
  \frac{1}{n + 1}T(n - 0) - \frac{1}{n - 0}  T(n - 1) &= \frac{A'}{n + 1} + \frac{B'}{(n-0)(n + 1)} \\
  \frac{1}{n + 0}T(n-1)   - \frac{1}{n - 1}  T(n - 2) &= \frac{A'}{n + 0} + \frac{B'}{(n-1)(n + 0)} \\
  \frac{1}{n + (-1)}T(n-2)  - \frac{1}{n - 2}  T(n - 3) &= \frac{A'}{n + (-1)} + \frac{B'}{(n-2)(n + (-1))} \\
  \frac{1}{n + (-2)}T(n-3)  - \frac{1}{n - 3}  T(n - 4) &= \frac{A'}{n + (-2)} + \frac{B'}{(n-3)(n + (-2))} \\
  &\vdots
\end{align*}
Adding these, I get
\begin{align*}
  \frac{1}{n + 1}T(n - 0) - \frac{1}{2}  T(1) &= A' \sum_{k=3}^{n+1} \frac{1}{k}
                                                + B' \sum_{k=2}^n \frac{1}{k(k+1)}
\end{align*}
i.e.,
\begin{align*}
  T(n)
  =
  A'(n+1)\sum_{k=3}^{n+1} \frac{1}{k}
  + \frac{n+1}{2}  T(1) + B' (n+1)\sum_{k=2}^n \frac{1}{k(k+1)}
\end{align*}
and you get $O(n \lg n)$.

The telescoping trick above depends on the recursion on being degree 1 (but not necessarily
linear) of the form
\[
  f(n) T(n) = f(n - 1) T(n - 1) + g(n)
\]
for $n \geq n_0$ 
and therefore
\[
  f(n) T(n) - f(n - 1) T(n - 1) = g(n)
\]
Therefore
\begin{align*}
  f(n) T(n) - f(n - 1) T(n - 1) &= g(n) \\
  f(n-1) T(n-1) - f(n - 2) T(n - 2) &= g(n-1) \\
  f(n-2) T(n-2) - f(n - 3) T(n - 3) &= g(n-2) 
\end{align*}
etc.
Therefore
\[
  f(n) T(n) - f(n_0 - 1) T(n_0 - 1) = g(n) + \cdots + g(n_0)
\]
and hence
\[
  T(n) = \frac{g(n) + \cdots + g(n_0)}{f(n)} + \frac{C}{f(n)}
\]
where $C = f(n_0 - 1) T(n_0 - 1)$.


\subsection{Method B2: generating function and Bernoulli differential equation}

From the previous section I have
\[
  T(n)
  = \left( 1 + \frac{1}{n} \right) T(n - 1) + A' + \frac{B'}{n}
\]
for $n > 0$.
Not all degree 1 non-linear can be put into a suitable form
for telescoping.
A more general method is the method of generating functions.
Let
\[
  f(x) = \sum_{n = 0}^\infty T(n) x^n
\]

%-*-latex-*-

\begin{ex} 
  \label{ex:prob-00}
  \tinysidebar{\debug{exercises/{disc-prob-28/question.tex}}}

  \solutionlink{sol:prob-00}
  \qed
\end{ex} 
\begin{python0}
from solutions import *
add(label="ex:prob-00",
    srcfilename='exercises/discrete-probability/prob-00/answer.tex') 
\end{python0}


The equation above is an example of an integral equation.
While a differential equation contains the differentiation operators,
an integral equation contains the integration operations.

%-*-latex-*-

\begin{ex} 
  \label{ex:prob-00}
  \tinysidebar{\debug{exercises/{disc-prob-28/question.tex}}}

  \solutionlink{sol:prob-00}
  \qed
\end{ex} 
\begin{python0}
from solutions import *
add(label="ex:prob-00",
    srcfilename='exercises/discrete-probability/prob-00/answer.tex') 
\end{python0}

%-*-latex-*-

\begin{ex} 
  \label{ex:prob-00}
  \tinysidebar{\debug{exercises/{disc-prob-28/question.tex}}}

  \solutionlink{sol:prob-00}
  \qed
\end{ex} 
\begin{python0}
from solutions import *
add(label="ex:prob-00",
    srcfilename='exercises/discrete-probability/prob-00/answer.tex') 
\end{python0}

%-*-latex-*-

\begin{ex} 
  \label{ex:prob-00}
  \tinysidebar{\debug{exercises/{disc-prob-28/question.tex}}}

  \solutionlink{sol:prob-00}
  \qed
\end{ex} 
\begin{python0}
from solutions import *
add(label="ex:prob-00",
    srcfilename='exercises/discrete-probability/prob-00/answer.tex') 
\end{python0}

