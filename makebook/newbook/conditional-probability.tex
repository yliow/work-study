%-*-latex-*-
\sectionthree{Conditional probability}
\begin{python0}
from solutions import *; clear()
\end{python0}

In probability theory, you might see something like 
\lq\lq what is the chance of getting a 1 from this die?''
Another type of probability question looks like this:
\lq\lq What is the chance of getting a 1
\textit{if the result is odd}?''
Or 
\lq\lq Suppose I throw two dice. What is the chance that the second
gives a 5 if the first gives a 1?''

In the real world, you might hear a question like:
\lq\lq What is the likelihood of getting cancer if I'm a smoker?''
(which is not the same as the likelihood of getting cancer in general)
or 
\lq\lq What is the chance of being runned down by a truck if I'm shortsighted
and there's a 50-50 chance that I forgot my glasses?'' 
(which is not the same as the chance of being runned down by a truck in 
general)
or 
\lq\lq What is the probability that my laptop will catch a virus if I only 
update my anti-virus definition once a month?''
(which is not the same as the chance of having a malfunctioning laptop 
for the general person)


Here's how you think of it.

Suppose I say that 
\lq\lq The chance of getting a 1 when I roll this die
\textit{if the result is odd} is $\frac{1}{3}$.''
What I mean is this:
If I roll the die $n$ times where $n$ is huge, then out of all
the results which are odd (i.e., I ignore the even results),
then about $\frac{1}{3}$ of them are 1. 

In notation, I might write
\[
p(A \mid B)
\]
where $A = \{ \ONE \}$ and $B = \{ \ONE, \THREE, \FIVE \}$, i.e.,
\[
p(A \mid B) = \frac{p(A \cap B)}{p(B)}
\]
This assumes that $p(B)$ is not 0.

Note the difference between $p(A \cap B)$ and $p(A \mid B)$.
To understand the difference intuitively, think about the 
difference between \lq\lq What is the chance of rain and thunderstorm today?''
and \lq\lq What is the chance of rain if there is thunderstorm today?''

%-*-latex-*-

\begin{ex} 
  \label{ex:prob-00}
  \tinysidebar{\debug{exercises/{disc-prob-28/question.tex}}}

  \solutionlink{sol:prob-00}
  \qed
\end{ex} 
\begin{python0}
from solutions import *
add(label="ex:prob-00",
    srcfilename='exercises/discrete-probability/prob-00/answer.tex') 
\end{python0}

  
Consider another scenario:
Suppose I ask \lq\lq What is the chance of getting of getting a small
number (i.e., either 1 or 2) if the result is odd?''
In that case $A = \{\ONE, \TWO\}$ and
$B = \{\ONE, \THREE, \FIVE\}$.
The quantity I need is
\begin{align*}
p(A \mid B)
&= \frac{p(A \cap B)}{p(B)} \\
&= \frac{p( \{ \ONE, \TWO \} \cap \{\ONE, \THREE, \FIVE \})}{p(\ONE, \THREE, \FIVE)}
\\
&= \frac{p( \{ \ONE \} ) }{p( \{ \ONE, \THREE, \FIVE\}) }
\\
&= \frac{ 1/6 }{ 1/2 }
\\
&= \frac{1}{3}
\end{align*}

%-*-latex-*-

\begin{ex} 
  \label{ex:prob-00}
  \tinysidebar{\debug{exercises/{disc-prob-28/question.tex}}}

  \solutionlink{sol:prob-00}
  \qed
\end{ex} 
\begin{python0}
from solutions import *
add(label="ex:prob-00",
    srcfilename='exercises/discrete-probability/prob-00/answer.tex') 
\end{python0}


Of course you can also talk about
conditional probabilities using random variables.
Recall we have the random variable $X$:
\[
  X : S \rightarrow \{\textsc{Good}, \textsc{Bad}\}
\]
where
\begin{align*}
  X(\ONE) &= X(\SIX) = \textsc{Good} \\
  X(\TWO) &= X(\THREE) = X(\FOUR) = X(\FIVE) = \textsc{Bad}
\end{align*}
and the random variable $Y$:
\[
  Y : S \rightarrow \{ \textsc{Even}, \textsc{Odd} \}
\]
where
\begin{align*}
  Y(\TWO) &= Y(\FOUR) = Y(\SIX) = \textsc{Even} \\
  Y(\ONE) &= Y(\THREE) = Y(\FIVE) = \textsc{Odd}
\end{align*}

Here's an example of conditional probabilities
using random variables:
\[
  \Pr[X = \textsc{Good} \mid Y = \textsc{Even}]
\]
This is
\[
\Pr[X = \textsc{Good} \mid Y = \textsc{Even}]
=
\frac{ \Pr[X = \textsc{Good} \text{ and } Y = \textsc{Even}] }{ \Pr[Y = \textsc{Even}] }
\]
(look at the word \lq\lq and").
Get it?
The value is:
\begin{align*}
  \Pr[X = \textsc{Good} \mid Y = \textsc{Even}]
  &=
    \frac
    { \Pr[X = \textsc{Good} \text{ and } Y = \textsc{Even}] }
    { \Pr[Y = \textsc{Even}] }
  \\
  &=
    \frac
    { p( \{ s \in S \mid X(s) = \textsc{Good} \text{ and } Y(s) = \textsc{Even} \} ) }
    { p( \{ s \in S \mid Y(s) = \textsc{Even} \}) }
  \\
  &=
    \frac
    { p( \{ \SIX \} ) }
    { p( \{ \TWO, \FOUR, \SIX \}) }
  \\
  &=
    \frac
    { 1/6 }
    { 1/6 + 1/6 + 1/6 }
  \\
  &=
    \frac
    { 1 }
    { 3 }
\end{align*}

%-*-latex-*-

\begin{ex} 
  \label{ex:prob-00}
  \tinysidebar{\debug{exercises/{disc-prob-28/question.tex}}}

  \solutionlink{sol:prob-00}
  \qed
\end{ex} 
\begin{python0}
from solutions import *
add(label="ex:prob-00",
    srcfilename='exercises/discrete-probability/prob-00/answer.tex') 
\end{python0}

%-*-latex-*-

\begin{ex} 
  \label{ex:prob-00}
  \tinysidebar{\debug{exercises/{disc-prob-28/question.tex}}}

  \solutionlink{sol:prob-00}
  \qed
\end{ex} 
\begin{python0}
from solutions import *
add(label="ex:prob-00",
    srcfilename='exercises/discrete-probability/prob-00/answer.tex') 
\end{python0}

%-*-latex-*-

\begin{ex} 
  \label{ex:prob-00}
  \tinysidebar{\debug{exercises/{disc-prob-28/question.tex}}}

  \solutionlink{sol:prob-00}
  \qed
\end{ex} 
\begin{python0}
from solutions import *
add(label="ex:prob-00",
    srcfilename='exercises/discrete-probability/prob-00/answer.tex') 
\end{python0}



