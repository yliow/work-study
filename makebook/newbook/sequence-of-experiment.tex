%-*-latex-*-
\sectionthree{Sequence of experiments}
\begin{python0}
from solutions import *; clear()
\end{python0}


Frequently the random experiment you are interested in
can be broken/decomposed into smaller and simpler
experiments.


\subsection{Sequence of independent experiments}

One way where indicator random variables
occur is when you need some kind of
\lq\lq if-else".
For instance suppose you are
playing a die-rolling gambling game
and $X_0$ and $X_1$ are two \lq\lq gain" functions.
Why two?
Because I'm changing the game
so that
you have to toss a coin.
If your toss gives you a $\HEAD$,
your gain will be
based on $X_0$.
Otherwise your gain is based on
$X_1$.
Instead of doing two separate
analysis based on the following two cases
(based on the coin toss),
you can combine them into one
by defining two random variables:
\[
  X = I_{\HEAD} \cdot X_0 + I_{\TAIL} \cdot X_1
\]
where $I_{\HEAD}$ is the random variable
\[
  I_{\HEAD}(\HEAD) = 1, \,\,\,
  I_{\HEAD}(\TAIL) = 0
\]
and
\[
  I_{\TAIL}(\HEAD) = 0, \,\,\,
  I_{\TAIL}(\TAIL) = 1
\]
This means that in the game,
where you made toss $t$ (of a fair coin) and made roll $r$ (of a fair die),
if your toss $t$ is a head, then
you gain is given by 
\begin{align*}
  X(r)
  &= 1 \cdot X_0(r) +
    0 \cdot X_1(r)
  \\
  &= X_0(r)
\end{align*}
and if your toss is a tail, then your gain is
\begin{align*}
  X(r)
  &= X_1(r)
\end{align*}

Here's a very important warning.
Pay attention.

Your random experiment
involves
tossing a coin \textit{and}
rolling a die.
The sample space should really be
\[
  S = S_0 \times S_1
\]
where
\[
  S_0 = \{ \HEAD, \TAIL \}
\]
and
\[
  S_1 = \{ \ONE, \TWO, \ldots, \SIX \}
\]
Assuming the coin and die are both fair,
we must have
\begin{align*}
  p : S = S_0 \times S_1 &\rightarrow [0,1] \\
  p(t, r) = 1/12
\end{align*}
So even though I wrote
\[
  I_{\HEAD}(\HEAD) = 1, \,\,\,
  I_{\HEAD}(\TAIL) = 0
\]
i.e., $I_{\HEAD}$ is a random variable on
\[
  S_0 = \{ \HEAD, \TAIL \}
\]
in the context of
\[
  X = I_{\HEAD} \cdot X_0 + I_{\TAIL} \cdot X_1
\]
$I_{\HEAD}$ is implicitly extended to
\[
  \widetilde{I}_{\HEAD} : S_0 \times S_1
\]
i.e.,
\[
 \widetilde{I}_{\HEAD}(\HEAD, r) = 1, \,\,\,
  \widetilde{I}_{\HEAD}(\TAIL, r) = 0
\]
for all rolls $r \in S_1$.
In other words
$\widetilde{I}_{\HEAD}$ is the indicator
variable
$I_{\{\HEAD\} \times S_1}$.
In other words
\begin{align*}
  \widetilde{I}_{\HEAD}(\HEAD, r) &= I_{\HEAD}(\HEAD) = 1 \\
  \widetilde{I}_{\HEAD}(\TAIL, r) &= I_{\HEAD}(\TAIL) = 0 
\end{align*}
for all rolls $r$.
Furthermore the gain $X_1$
\[
  X_1 : S_1 \rightarrow \R
\]
should really be extended to
\[
  \widetilde{X}_1 : S_0 \times S_1 \rightarrow \R
\]
where
\[
  \widetilde{X}_1(t, r) = X_1(r)
\]

So to be really accurate,
we extend
$I_{\HEAD}, I_{\TAIL}, S_0, S_1$
to
$\widetilde{I}_{\HEAD}, \widetilde{I}_{\TAIL}, \widetilde{S}_0, \widetilde{S}_1$
and then define $X$ (in the correct way) as
\[
  X
  =
  \widetilde{I}_{\HEAD} \cdot \widetilde{X}_0
  +
  \widetilde{I}_{\TAIL} \cdot \widetilde{X}_1
\]
i.e.,
\[
  X(t, r)
  =
  \widetilde{I}_{\HEAD}(t, r) \cdot \widetilde{X}_0(t, r)
  +
  \widetilde{I}_{\TAIL}(t, r) \cdot \widetilde{X}_1(t, r)
\]
and then we do get
\[
  X(t, r)
  =
  I_{\HEAD}(t) X_0(r)
  +
  I_{\TAIL}(t) X_1(r)
\]

By the way, the events \lq\lq the toss is head"
and \lq\lq the die roll is one" are independent.
In other words
\begin{align*}
  A &= \{ \HEAD \} \times S_1 \\
  B &= S_0 \times \{ \ONE \} 
\end{align*}
are independent.
Here's the check:
\begin{align*}
  p(A)
  &= p(\{ \{ \HEAD \} \times S_1 \}) = 6/12
  \\
  p(B)
  &= p( S_0 \times \{ \ONE \} ) = 2/12
  \\
  \THEREFORE
  p(A) \cdot p(B) &= 1/12
\end{align*}
and
\begin{align*}
  p(A \cap B)
  &= p(\{(\HEAD, \ONE)\}) = 1/12
\end{align*}
Hence
\[
  p(A \cap B) = p(A) \cdot p(B)
\]
In fact $A$ and $B$ are independent
even when
$A$ is replaced by \lq\lq any toss"
and 
$B$ is replaced by \lq\lq any roll".

Another thing to note is that
\[
  p(t, r) = p_0(t) \cdot p_1(r)
\]
where $p_0: \{ \HEAD, \TAIL \} \rightarrow [0,1]$ is the
uniform pdf of a (fair) coin toss
and $p_1 : \{ \ONE, \TWO, \THREE, ..., \SIX \} \rightarrow [0,1]$
is the uniform pdf of a (fair) die roll.

%-*-latex-*-

\begin{ex} 
  \label{ex:prob-00}
  \tinysidebar{\debug{exercises/{disc-prob-28/question.tex}}}

  \solutionlink{sol:prob-00}
  \qed
\end{ex} 
\begin{python0}
from solutions import *
add(label="ex:prob-00",
    srcfilename='exercises/discrete-probability/prob-00/answer.tex') 
\end{python0}


\begin{thm}
If a random experiment involves
carrying out two separate and independent experiments.
Suppose the pdfs are 
$p_0 : S_0 \rightarrow [0,1]$
and
$p_1 : S_1 \rightarrow [0,1]$.
Then
\[
p(\{x\} \times S_2) = p_1(x)
\]
and
\[
p(S_1 \times \{y\}) = p_2(x)
\]
and
\[
  p(x, y) = p_1(x) p_2(y)
\]
\end{thm}


This means that if there is a random experiment
(with pdf $p : S \rightarrow [0,1]$)
which is made up of a sequence of two independent experiments
(with pdf $p_i : S_i \rightarrow [0,1]$),
then $p(x,y) = p_0(x) p_1(y)$, which means that
the computation simplifies to the computation of
$p_0(x)$ and $p_1(y)$ and note that
the computation of $p_0(x)$ for instance allows you to focus on
$p_0 : S_0 \rightarrow [0,1]$ and this
random
experiment has a smaller sample space and is therefore simpler.

The above is also true for a sequence of three independent
experiments, or four, or five, etc.








\subsection{Sequence of dependent experiments}

There are times when a random experiment involves two experiments
and the second experiment depends on the first.
Here's an example.
Suppose you have two boxes: The first box has 1 red ball and 2 green balls
while the second box has 3 red balls and 5 green balls.
Here's the experiment:
I toss a fair coin.
If I get a head, then I pick a ball from the first box.
If I get a tail, then I pick a ball from the second box.

It's clear that the sample space is
\[
S = \{\HEAD, \TAIL\} \times \{\RED, \GREEN\}
\]
The pdf $p$ of the above random experiment is then
\begin{align*}
  p(\HEAD, \RED) &= 1/2 \cdot 1/3 = 1/6 \\
  p(\HEAD, \GREEN) &= 1/2 \cdot 2/3 = 1/3 \\
  p(\TAIL, \RED) &= 1/2 \cdot 3/8 = 3/16 \\
  p(\HEAD, \GREEN) &= 1/2 \cdot 5/8 = 5/16 
\end{align*}

Is the event that I get a head (from the coin toss) independent of the
event that I get a red ball?

Let $A$ be the first event. This means
\[
A = \{(\HEAD, \RED), (\HEAD, \GREEN)\}
\]
Let $B$ be the second event. Then
\[
B = \{(\HEAD, \RED), (\TAIL, \RED)\}
\]
We have
\begin{align*}
  p(A \mid B)
  &= \frac{ p(A \cap B) }{ p(B) } \\
  &= \frac{ p(\{(\HEAD, \RED)\}) }{p(\{(\HEAD,\RED),(\TAIL,\RED)\})} \\
  &= \frac{1/6}{1/6 + 3/16} \\
  &= \frac{8}{17}
\end{align*}
which is not $p(A) = 1/6$.
Therefore $A$ and $B$ are not independent.




\subsection{Multiplication principle}


Suppose that a random experiment $R$ involves
carry out two experiments.
After getting
an outcome $x$ from the first random experiment $R_0$.
Say the pdf of $R_0$ is $p_0 : S_0 \rightarrow [0,1]$.
I perform the second random experiment $R_1$.
The second random experiment might depend on the first,
i.e., I should write $R_1(x)$ --
what I'm going to say deos not depend on
whether the two experiments are dependent or not.
Suppose the pdf for the second experiment is
$p_{1, x} : S_{1, x} \rightarrow [0,1]$.
(Note that it's possible that $p_{1,x}$ depends on the first
outcome $x$.)
Suppose the output for the second random experiment is $y$.
Then the probability for outcome $(x, y)$
is $p_0(x) \cdot p_{1, x}(y)$.

For instance if you roll two dice where the first is fair
and the
second is loaded so that 1/2 the time you will get a six,
then the probability of getting one for the first
and six for the second
is
\[
1/6 \cdot 1/2
\]
Or, if you toss one fair coin and if you
get a head, you roll a fair die and if you get a tail,
you roll a loaded die where the chance of
one is 0.5 and the others are equi-distributed.
Then the probability of head followed by one is (1/2)(1/6)
and the probability of tail followed by one is (1/2)(1/2).

The above fact follows from the multiplication
principle of counting.



\subsection{Variable length}

There are times when you are interested in an
experiment that is made up of a sequence of
experiments where the number of experiments $n$ is not fixed.
In fact the questions you are interested in involves
solving for $n$ when a condition is met.

For instance:
\begin{enumerate}
\item
  How many times do I have a roll a die in order to get a 6?
  (Of course this means \lq\lq What is the average number of times do I have
  to roll a die ...".)
\item
  How many times do I have a roll a die in order to get two 6s?
\item
  How many times do I have a roll a die in order to get two consecutive 6s?
\item
  How times do I have a die in order
  so that the chance of 
  getting two consecutive sixes is greater than 1/2?
\end{enumerate}

See sections on infinite sample space and geometric distribution. 
