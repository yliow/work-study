%-*-latex-*-
\sectionthree{Indicator random variable}
\begin{python0}
from solutions import *; clear()
\end{python0}

Suppose you have a sample space $S$ and $A$ is an event of $S$,
i.e., $A \subseteq S$.
The following is a very useful random variable:
\[
X_A : S \rightarrow \R
\]
where
\[
X_A(s) =
\begin{cases}
  1 & \text{if $s \in A$} \\
  0 & \text{otherwise} \\
\end{cases}
\]
This is like a labeling: values in $A$ are labeled with 1
while values not in $A$ are labeled with 0.
Such a random variable is called an \defone{indicator random variable}.
It's also common to write $I_A$ for such a random variable.

There are times when $A$ has only a single value.
Say $a \in S$.
Then I will write $X_a$ instead of $X_{\{a\}}$.
In other words, 
the indicator random variable for $a$ is
\[
X_a : S \rightarrow \R
\]
where
\[
X_a(s) =
\begin{cases}
  1 & \text{if $s = a$} \\
  0 & \text{otherwise} \\
\end{cases}
\]

You can think of $X_A$ as a boolean function that tells you if
an outcome falls in $A$.
Another way is to think of $X_A$ as a counter that counts (or labels)
an outcome as $1$ if the outcome is in $A$.
This is a very important way to think about the
indicator random variable especially when we do
expected value computations.
For instance, suppose you consider a random experiment
of tossing a coin.
The sample space is $\{ \HEAD, \TAIL \}$.
The indicator variable $X_{\HEAD}$ counts the number of heads:
it's either zero or one.

%-*-latex-*-

\begin{ex} 
  \label{ex:prob-00}
  \tinysidebar{\debug{exercises/{disc-prob-28/question.tex}}}

  \solutionlink{sol:prob-00}
  \qed
\end{ex} 
\begin{python0}
from solutions import *
add(label="ex:prob-00",
    srcfilename='exercises/discrete-probability/prob-00/answer.tex') 
\end{python0}

