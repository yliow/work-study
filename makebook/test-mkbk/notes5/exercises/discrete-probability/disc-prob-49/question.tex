\tinysidebar{\debug{exercises/discrete-probability/{disc-prob-49/question.tex}}}
  Draw a square like this with $n$ points on each side:
  \begin{python}
from latextool_basic import *
p = Plot()
p += Line(points=[(0,0), (5,0), (5, 5), (0, 5), (0,0)])
for i in range(1, 5):
    p += Circle(x=i, y=0, r=0.1, background='black')
    p += Circle(x=i, y=5, r=0.1, background='black')
    p += Circle(x=0, y=i, r=0.1, background='black')
    p += Circle(x=5, y=i, r=0.1, background='black')

print(p)
  \end{python}
  (This picture has $n = 4$.)
  Pairs of points are randomly chosen to be connected
  by a line segment.
  Each point is on exactly one line segment.
  A point can only be paired with a point on a different side of the square.
  Here's an example:
    \begin{python}
from latextool_basic import *
p = Plot()
p += Line(points=[(0,0), (5,0), (5, 5), (0, 5), (0,0)])
for i in range(1, 5):
    p += Circle(x=i, y=0, r=0.1, background='black')
    p += Circle(x=i, y=5, r=0.1, background='black')
    p += Circle(x=0, y=i, r=0.1, background='black')
    p += Circle(x=5, y=i, r=0.1, background='black')

p += Line(points=[(0,1), (3,5)])
p += Line(points=[(1,0), (0,4)])
p += Line(points=[(5,2), (0,3)])
p += Line(points=[(0,2), (4,0)])
p += Line(points=[(1,5), (5,4)])
p += Line(points=[(2,0), (2,5)])
p += Line(points=[(3,0), (5,1)])
p += Line(points=[(4,5), (5,3)])
    
print(p)
  \end{python}
  \begin{itemize}
  \item How many points of intersection are there?
  \item How many polygons are formed?
  \end{itemize}
