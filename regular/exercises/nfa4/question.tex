\tinysidebar{\debug{exercises/{nfa4/question.tex}}}
Refer to our first NFA again:
\begin{center}
\begin{tikzpicture}[shorten >=1pt,node distance=2cm,auto,initial text=,
initial where=above]
\node[state,initial]   (q0) [shape=circle, draw] at ( 0,  0) {$q_0$};
\node[state]           (q1) [shape=circle, draw] at (-1, -4) {$q_1$};
\node[state]           (q3) [shape=circle, draw] at ( 1, -2) {$q_3$};
\node[state,accepting] (q4) [shape=circle, draw] at ( 1, -4) {$q_4$};
\node[state,accepting] (q2) [shape=circle, draw] at (-1, -6) {$q_2$};

\path[->]
(q0) edge [loop right] node {$a,b$} ()
(q0) edge node {$b$} (q1)
(q0) edge node {$\epsilon$} (q3)
(q1) edge node {$b$} (q2)
(q3) edge node {$\epsilon$} (q1)
(q3) edge node {$a$} (q4)
;
\end{tikzpicture}\end{center}
\begin{itemize}
\item What is $\overline{\{\}}$? 
\item What about $\overline{\{q_0\}}$?
\item What about $\overline{\{q_1\}}$?
\item What about $\overline{\{q_2\}}$?
\item What about $\overline{\{q_4\}}$?
\item What about $\overline{\{q_0, q_1\}}$?
\item Write down $\overline{S}$ for all subsets $S$ of $\{q_0, ..., q_4\}$.
\end{itemize}
