\sectionthree{State Diagram}
\begin{python0}
from solutions import *; clear()
\end{python0}

The following is the
\defone{state diagram}
of a
\defone{deterministic finite automata}
\sidebarskip{16pt}\defone{DFA}\sidebarskip{0pt}:

\begin{python}
from latextool_basic import *
print(automata(layout="""
A  B  C
""",
edges="A,$a$,A|A,$b$,B|B,$a$,A|B,$b$,C|C,$a,b$,C",
A="initial|label=$q_0$",
B="label=$q_1$",
C="accept|label=$q_2$",
))
\end{python}

A state diagram is also called a \defone{transition diagram}.

A DFA is also called a
\defone{deterministic finite state machine}
\sidebarskip{16pt}\defone{DFSM}\sidebarskip{0pt}.


A directed edge in the state diagram is called a
\defone{transition}.
A transition labeled with symbol $a$ is called an
\defone{$a$--transition}.
In this example, $a$ and $b$ are the symbols used to label the transitions.
We say that the set $\{a, b\}$ is the \defone{alphabet} of this DFA.
Let's use $\Sigma$ to denote the set $\{a, b\}$.

You think of the above state diagram as the program of the DFA:
it describes how this machine computes.

This is how you run the machine with a given input.
The input will be a string of $a$'s and $b$'s.
For instance the input can be $baaab$; it can even be $\ep$ the empty string.
However $abbc$ is \textit{not} an input for this machine: $c$ does not
appear in the transitions! 
The input $baaab$ is placed on an input tape for the machine read.
\begin{itemize}
 \item Start at the \defterm{start state}. This is the node with an arrow
 pointing to it from nowhere.
 For the above example, the start state is $q_0$.
 Look for it now. Put your finger on it.
 (I'm using $\SPACE$ to denote a blank on the tape.)
\begin{python}
from latextool_basic import *
p = Plot()
dfa(p,
state=r'$q_0$',
body_w=1.5, body_h=1,
tape=['b','a','a','a', 'b', r'\SPACE',r'\SPACE',r'\SPACE'],
head_index=0,
vsep=0.7,
w=0.5, h=0.5
)
print(p)
\end{python}
\begin{python}
from latextool_basic import *
print(automata(layout="""
A  B  C
""",
edges="A,$a$,A|A,$b$,B|B,$a$,A|B,$b$,C|C,$a,b$,C",
A="initial|label=$q_0$",
B="label=$q_1$",
C="accept|label=$q_2$",
))
\end{python}

 \item Read the first character of the input string; you always read the string
 left to right. For the above DFA with input $baaab$, the first character is
 $b$ (the $b$ on the left).

 \item Go from the node you're pointing at to the next node using
 the edge labeled with the character you've just read.
 For our DFA with input $baaab$, we move along the $b$--transition
 and land in state $q_1$.
 \begin{python}
from latextool_basic import *
p = Plot()
dfa(p,
state=r'$q_1$',
body_w=1.5, body_h=1,
tape=['b','a','a','a', 'b', r'\SPACE',r'\SPACE',r'\SPACE'],
head_index=1,
vsep=0.7,
w=0.5, h=0.5
)
print(p)
\end{python}
\begin{python}
from latextool_basic import *
print(automata(layout="""
A  B  C
""",
edges="A,$a$,A|A,$b$,B|B,$a$,A|B,$b$,C|C,$a,b$,C",
A="initial|label=$q_0$",
B="label=$q_1$",
C="accept|label=$q_2$",
))
\end{python}


 \item Repeat the above process, reading the remaining characters
 of the input left-to-right, until you have no more characters.
 The characters of the input are $b, a, a, a, b$ and starting with $q_0$,
 we go through the states $q_1, q_0, q_0, q_0, q_1$.
 The last state after reading the whole input is $q_1$.
 \begin{python}
from latextool_basic import *
p = Plot()
dfa(p,
state=r'$q_1$',
body_w=1.5, body_h=1,
tape=['b','a','a','a', 'b', r'\SPACE',r'\SPACE',r'\SPACE'],
head_index=5,
vsep=0.7,
w=0.5, h=0.5
)
print(p)
\end{python}

You can think of the above as follows.
Given a DFA:
\begin{python}
from latextool_basic import *
print(automata(layout="""
A  B  C
""",
edges="A,$a$,A|A,$b$,B|B,$a$,A|B,$b$,C|C,$a,b$,C",
A="initial|label=$q_0$",
B="label=$q_1$",
C="accept|label=$q_2$",
))
\end{python}
If you have a string $baaab$, then this string traces out a \textit{path}
in the DFA which can be described by this notation:
\[
q_0 \xrightarrow{\,\,\,\,\,b\,\,\,\,\,}
q_1 \xrightarrow{\,\,\,\,\,a\,\,\,\,\,}
q_0 \xrightarrow{\,\,\,\,\,a\,\,\,\,\,}
q_0 \xrightarrow{\,\,\,\,\,a\,\,\,\,\,}
q_0 \xrightarrow{\,\,\,\,\,b\,\,\,\,\,}
q_1
\]
This is an informal notation.
I'll show you two proper
formal notations for \lq\lq computation" in the next section.
\item
  A \defone{accept state} or
  \sidebarskip{16pt}\defone{final state}\sidebarskip{0pt}.

  is a node denoted with a double-circle boundary.
  When you have finished scanning your string left to right, and
  if you are on a final state, then
  the string is \defone{accepted} by the DFA, otherwise the string is
  not accepted.

\end{itemize}

You should think of a DFA as a string recognizer in the sense that
when you feed it a string, it will tell you whether the string is
accepted or not.


\begin{eg}
Write down the list of states you pass through as you run the DFA
with the string $ababb$.
\begin{python}
from latextool_basic import *
print(automata(layout="""
A  B  C
""",
edges="A,$a$,A|A,$b$,B|B,$a$,A|B,$b$,C|C,$a,b$,C",
A="initial|label=$q_0$",
B="label=$q_1$",
C="accept|label=$q_2$",
))
\end{python}
What is the last state? Is it an
accept state? Is the string accepted by the DFA?
The formal way to write down the steps in the computation is as
follows:
\begin{align*}
 (q_0, ababb)
 &\vdash (q_0,babb) \cr
 &\vdash (q_1,abb) \cr
 &\vdash (q_0,bb) \cr
 &\vdash (q_1,b) \cr
 &\vdash (q_2,\ep) \cr
\end{align*}
\end{eg}

The $(q, x)$, where $q$ is a state and $x$ is a word, is called an
\defone{instantaneous description}.

Here's a way to
think of the above:
First of all do you see that instantaneous descriptions are elements of the
set of product of set of states and set of words.
If I write the set of states as $Q$, then the the instantaneous
descriptions are the elements of 
\[
Q \times \Sigma^*
\]
Next, do you see that $\vdash$ is a relation on $Q \times \Sigma^*$?

\begin{defn}
Let $M$ be a DFA. The language \defone{accepted} by $M$, denoted
$L(M)$ is the set of
\textbf{all} strings over $\Sigma$ accepted by
$M$.
\end{defn}

Note that a directed graph is the state diagram of a DFA over
$\Sigma$ if: 
\begin{itemize}

\item There is exactly one special node called the
  \defone{start state} or
  \defone{initial state}.
 This is drawn with a single-line boundary with an arrow pointing to it from nowhere.
 (Some authors use a wedge instead of an arrow.)

 \item There is 0 or more nodes called the
   \textbf{accept states} or
   \textbf{final states}.
 These are drawn with a double-line boundary.

 \item For each node $v$, there are exactly $|\Sigma|$ edges going
 \textbf{out} from $v$ to other nodes. These edges are labeled with the
 elements in $\Sigma$. Note that every alphabet in $\Sigma$ is used
 in labeling the edges of $v$.

 \item In a DFA state diagram, the nodes are called \defone{states} and the
 edges are called \defone{transitions}. If a transition is labeled
 with $a \in \Sigma$, we also call it an \defone{$a$--transition}.

\end{itemize}

Note that given the state diagram of a DFA, say $M$, 
an input can contain only the 
characters in $\Sigma$ where $\Sigma$ is the set of symbols appearing in the
transitions of $M$.

\newpage
\begin{ex}
  Refer to the above DFA
  \begin{python}
from latextool_basic import *
print(automata(layout="""
A  B  C
""",
edges="A,$a$,A|A,$b$,B|B,$a$,A|B,$b$,C|C,$a,b$,C",
A="initial|label=$q_0$",
B="label=$q_1$",
C="accept|label=$q_2$",
))
\end{python}
  \begin{tightlist}
  \item What is the string of shortest length accepted by the above DFA?
  \item  Write down the computation of the string $baba$.
    Is this string accepted by the DFA?
  \item Write down the computation of the string $abab$.
    Is this string accepted by the DFA?
  \item How many strings of length 5 are accepted by the DFA?
  \end{tightlist}
\qed
\end{ex}

\newpage
\begin{ex}
  What is the language accepted by the above DFA:
  \begin{python}
from latextool_basic import *
print(automata(layout="""
A  B  C
""",
edges="A,$a$,A|A,$b$,B|B,$a$,A|B,$b$,C|C,$a,b$,C",
A="initial|label=$q_0$",
B="label=$q_1$",
C="accept|label=$q_2$",
))
\end{python}
\qed
\end{ex}


\newpage
So far we have seen examples where you're given a DFA (or at least the state
diagram), and you look at the string or the full language accepted by the DFA.
In the following diagram, $\DFA_\Sigma$ is the collection (i.e., set)
of all DFAs over $\Sigma$.
$\LANG_\Sigma$ is the collection of languages over $\Sigma$.
Each DFA gives us a language.
(You'll see that indeed, the collection of languages that is,
in some sense, much larger than the collection of DFAs.)

\begin{python}
from latextool_basic import *

p = Plot()
r0 = Rect(0, 0, 2, 2)
x,y = r0.top()
y += 0.25
t0 = Rect(x, y, x, y, label=r'$\DFA_\Sigma$', linewidth=0)
p += r0
p += t0

r1 = Rect(4, 0, 8, 4)
x,y = r1.top()
y += 0.25
t1 = Rect(x, y, x, y, label=r'$\LANG_\Sigma$', linewidth=0)

p += r1
p += t1

x0,y0 = r0.right()
x1,y1 = r1.left()
p += Line(points=[(x0,y0), (x1,y0)], endstyle='>')
"""
DFAs over S                 Languages over S
+---------------+           +---------------+
|               |           |               |
|               | --------> |               |
|               |           |               |
+---------------+           +---------------+
"""

print(p)
\end{python}

Now let's start with a language.

\newpage
\begin{ex}
Draw a state diagram that accepts the following
language of strings over $\Sigma =
\{ a, b \}$ 
\[
L = \{\}
\]
\end{ex}


\newpage
\begin{ex}
Draw a state diagram that accepts the following
language of strings over $\Sigma =
\{ a, b \}$ 
\[
L = \{a\}
\]
\end{ex}


\newpage
\begin{ex}
Draw a state diagram that accepts the following
language of strings over $\Sigma =
\{ a, b \}$ 
\[
L = \{\ep\}
\]
\qed
\end{ex}



\newpage
\begin{ex}
Draw a state diagram that accepts the following
language of strings over $\Sigma =
\{ a, b \}$ 
\[
L = \{\ep, a\}
\]
\qed
\end{ex}

\newpage
\begin{ex}
Draw a state diagram that accepts the following
language of strings over $\Sigma =
\{ a, b \}$ 
\[
L = \{a, b\}
\]
\qed
\end{ex}



\newpage
\begin{ex}
Draw a state diagram that accepts the following
language of strings over $\Sigma =
\{ a, b \}$ 
\[
L = \{a, a^2\}
\]
\qed
\end{ex}

\newpage
\begin{ex}
Draw a state diagram that accepts the following
language of strings over $\Sigma =
\{ a, b \}$ 
\[
L = \{ab, ba\}
\]
\qed
\end{ex}

\newpage
\begin{ex}
Draw a state diagram that accepts the language of strings over $\Sigma =
\{ a, b \}$ that contains $aaa$, i.e.
\[
L = \{ x \in \Sigma^* \,|\, x = ya^3 z \text{ for some } y, z \in \Sigma^*\}
\]
\qed
\end{ex}

\newpage
\begin{ex}
Draw a state diagram that accepts the language of strings over $\Sigma
 = \{a, b\}$ that does not contain $aaa$.
\qed
\end{ex}

\newpage
\begin{ex}
Draw a state diagram that accepts the language of strings over 
$\Sigma = \{a, b\}$ that contains $abab$.
\qed
\end{ex}

\newpage
\begin{ex}
Draw a state diagram that accepts the language of strings over 
$\Sigma = \{a, b\}$ where the number of $a$'s is exactly 
$\equiv 1 \, (\operatorname{mod} 4)$.
So the number of $a$'s can be $1, 5, 9, \ldots$.
\qed
\end{ex}


\newpage
\begin{ex}
  Draw a state diagram that accepts the following language
  \[
  L = \{w \in \Sigma^* \mid |w|_a, |w|_b \text{ both even}\}
  \]
  where $\Sigma = \{a, b\}$.
  Here, 
  $|w|_a$ is the number of $a$'s in $w$
  and
  $|w|_b$ is the number of $b$'s in $w$.
  For example $|abaab|_a = 3$ and $|abaab|_b = 2$.
  Therefore $abaab \not\in L$.
  However $abbaaa \in L$.
  \qed
\end{ex}


\newpage
In the study of languages a very important question to ask is what
is the expressivity (or power) of the devices used in describing
languages. So for a  fixed $\Sigma$, can all languages over $\Sigma$
be the language be accepted by some DFA defined over $\Sigma$?

If so, then we only need to study DFAs.

Unfortunately (or fortunately for those are masochistic or who love
twisted plots) $\ldots$ NO! In fact there are
\textit{many} languages
\textit{not} accepted by DFAs. Here's are a few: 
\begin{itemize}
 \item $\{a^nb^n \,|\, n\geq 0\}$
 \item The palindromic strings over $\Sigma$ if $|\Sigma|>1$.
\end{itemize}
