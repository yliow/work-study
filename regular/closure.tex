\sectionthree{Closure Rules}
\begin{python0}
  from solutions import *; clear()
\end{python0}

Now suppose if you want to show $L$ is regular, you can try to
construct a DFA for it. That would mean that you're always starting
from scratch. In this section we'll adopt a more sophisticated
approach. In particular, we see how to show a language is regular by
looking at other language(s).

We fix a alphabet $\Sigma$.
Every language over $\Sigma$ is a
subset of $\Sigma^*$. Therefore the set of
\textit{all} languages
over $\Sigma$ is the powerset of $\Sigma^*$, i.e., $P(\Sigma^*)$.
In this collection of languages is the set of regular languages
over $\Sigma$. Let's call this set \defone{$\Reg_\Sigma$}. Therefore
$\Reg_\Sigma \subsetneq P(\Sigma^*)$.

Here's why the above are called \defone{closure rules}. Consider all
languages over $\Sigma$. This is $P(\Sigma^*)$. Right? Remember
powerset? You can think of $\cup$ (union) as a binary operation on
$P(\Sigma^*)$. In other words let $L, L' \in P(\Sigma^*)$, then
$L \cup L' \in P(\Sigma^*)$.
This is obvious.

Now suppose $L,L' \in \Reg_\Sigma$.
We want to know if $L \cup L' \in \Reg_\Sigma$.
This is true, but not immediate.
If this is the case, then
we say that $\cup$ is a \defone{closed binary operation} on $\Reg_\Sigma$.
This is similar to saying $+$ is a closed binary operator on $\Z$: adding two integers
will give you an integer.
However, note that the binary operator $/$ division on $\Z$ is not closed:
$5 / 2 = 2.5$ which is \textit{not} in $\Z$.
Obviously working with $\Reg_\Sigma$ is harder than with $\Z$!

Why are closure rules useful? Such rules will allow us to re-use previously
known regular languages. For instance if $\cup$ is a closed
operator on regular languages and
you want to study
\[
  L'' = \{ x \in \{a,b\}^* \,|\, \text{ number of $a$ in $x$ is } \equiv 1 \pmod{5}
  \text{ or $x$ contains $a^3$} \}
\]
then, assuming you see the decomposition
\[
L'' = L \cup L'
\]
where
\begin{align*}
 L &= \{ x \in \{a,b\}^* \,|\, \text{number of $a$ in $x$ is } \equiv 1 \pmod{5}
 \}\\
 L' &= \{ x \in \{a,b\}^* \,|\, x \text{ contains } a^3 \}
\end{align*}
are both regular, then you know immediately that $L$ is also regular.
You can than apply all tools and techniques in regular languages to study $L$.
In fact you can also study $L''$ by studying the components of
its decomposition: $L$ and $L'$.

The above is a general philosophy is used throughout any form of study, 
computer science, math, sciences, etc.
You decompose a problem and you ask which \lq\lq universe" you are in
so that you don't use the wrong formulas or theorems.

We will look at union, intersection, set difference, etc. Note that if $L$
is a language over $\Sigma$, complementation of $L$ will be with
respect to $\Sigma^*$, i.e., $\overline L = \Sigma^* - L$.
Here are some more operators:

\begin{defn}
  Let $L$ and $L'$ be languages over $\Sigma$. 
  \begin{itemize}
  \item
    The
    \defone{concatenation} of $L$ and $L'$ is defined as follows:
    \[
    LL' = \{ z \in \Sigma^* \,|\, z = xx' \text{ for some } x\in L \text{ and } x' \in L'\}
    \]
  \item We can define $L^n$ for $n \geq 0$ as follows:
    \[
    L^n =
    \begin{cases}
      \{ \ep \} & \text{ if } n = 0 \\
      L^{n-1} L  & \text{ if } n > 0 \\
    \end{cases}
    \]
  \item The \defone{Kleene star} of $L$, written $L^*$ is defined by
    \[
    L^* = \bigcup_{n \geq 0} L^n = \{x_1\cdots x_n \,|\, x_i \in L  \text{ for } i=1,\ldots,n \}
    \]
  \end{itemize}
\end{defn}

You can easily make up your own operators.
For instance here's one that you have seen before:
If $L$ is a language $L^R$ is the reversal of $L$, i.e.,
the words in $L^R$ are the reversed of $L$.
For instance if $abaa \in L$, then $aaba \in L^R$.
You can then ask this question: if $L$ is regular, is $L^R$ also regular?
Note that ${-}^R$ is a unary operator.
I'll give you another one:
Let $L$ be a language. Define $\frac{1}{2}L$ be the language
of half-words of $L$ where a half-word is just the \textit{first half} of the word.
For instance if $w = aaba$, then $\frac{1}{2}w$ is $aa$.
For the case when $w$ has odd length,
$\frac{1}{2}w$ is the first $\floor{|w|/2}$ characters.
For instance $\frac{1}{2}aabbbba = aab$.
You can ask this question: If $L$ is regular, is $\frac{1}{2}L$ also regular?

For binary language operator, here's an example:
Suppose $L$ and $L'$ are two languages, define $\operatorname{Zip}(L,L')$
to be the language of words $w$ such that
\[
w = x_1 y_1 x_2 y_2 x_3 y_3 \cdots
\]
where
$x = x_1 x_2 x_3 \cdots$ is a word in $L$
and
$y = y_1 y_2 y_3 \cdots$ is a word in $L'$.
(If $x$ and $y$ has different lengths, you just add $\ep$'s so that they have the same length.)
You can then ask if 
$\operatorname{Zip}(L,L')$ is regular
if
$L,L'$ are regular.


You can create your oan unary and binary operators on languages are ask
if your operator is closed in the collection of regular languages.

 
We will prove the following closure rules. In the proofs you will
see the power of NFAs.



\begin{thm}
 Let $L,L'$ be languages.
 \begin{enumerate}[label=\textnormal{(\alph*)},itemsep=0pt,nosep,noitemsep,partopsep=0pt,topsep=0pt,parsep=0pt]
  \item $L,L'$ regular $\implies$ $L \cup L'$ regular
  \item $L,L'$ regular $\implies$ $L \cap L'$ regular
  \item $L$ regular $\implies$ $\overline{L}$ regular
  \item $L,L'$ regular $\implies$ $L - L'$ regular
  \item $L,L'$ regular $\implies$ $LL'$ regular
  \item $L$ regular $\implies$ $L^*$ regular
 \end{enumerate}
 In other words, regular languages
 are closed under union, intersection, complement, set difference,
 concatenation, and Kleene star.
 \end{thm}





\newpage
\section{Closure: Intersection}

Let's prove closure of $\cap$ on $\Reg_\Sigma$.

Since we assumed that $L$ and $L'$ are regular, by definition,
there must be DFAs $M$ and $M'$ such that $L = L(M)$ and $L' =
L(M')$. To show that $L \cap L'$ is regular, we must construct a
DFA $M''$ such that $L(M'') = L\cap L'$. Right? This is all by
definition.

Of course $M''$ must be constructed from $M$ and $M'$. What do you
want $M''$ to do? Suppose you're given a string $x \in \Sigma^*$. By
definition a string $x$ is in $L''$ if and only if $x$ is in both
$L$ and $L'$. This means that $M''$ must be a machine that will
accept $x$ if and only if $x$ is accepted by both $M$ and $M'$. This
means that $M''$ must be able to \lq\lq run" $M$ and $M'$ separately
and must ``know" if $M$ and $M'$ accepted the $x$ or not. So in
other words suppose that when you run $M$ and $M'$ with $x$, you get
the following sequence of states:
\begin{align*}
M: \,\,\, &q_0 \rightarrow s_1 \rightarrow s_2 \rightarrow \cdots \rightarrow s_n \\
M': \,\,\, &q_0' \rightarrow s'_1 \rightarrow s'_2 \rightarrow
\cdots \rightarrow s'_n
\end{align*}
where $s_i$ are states in $M$ and $s_i'$ are states in $M'$. Of
course $M''$ is a single DFA and so as you run $M''$ you should
only see a single sequence of states, not two. How would you
create a single sequence of states from the above? What if we do
this:
\[
M'' \,\,\, (???): \,\,\, q_0,q_0' \rightarrow s_1,s_1' \rightarrow
s_2, s'_2 \rightarrow \cdots \rightarrow s_n,s'_n
\]
So we want ``$q_0,q_0'$'' to be a single ``thing''. AHA!!!
Tuples!!!
\[
M'': \,\,\, (q_0,q_0') \rightarrow (s_1,s_1') \rightarrow (s_2,
s'_2) \rightarrow \cdots \rightarrow (s_n,s'_n)
\]
Note in particular that $(s_i,s'_i)$ is not the same as
$(s_i',s_i)$.
This is the first hint that the states of $M''$ must
be
\[
 Q'' = Q \times Q' = \{(q,q') \,|\, q\in Q, q'\in Q'\}
\]
i.e., the product of $Q$ and $Q'$, the set of states of $M$ and
$M'$.

How should we initialize our $M''$ engine? Obviously $q''_0 =
(q_0,q_0')$ where $q_0$ and $q_0'$ are the initial states of $M$
and $M'$ respectively.

When is $x$ accepted? Obviously if the final state after running
$x$ through the $M''$ is $(q,q')$, then
\[
 x \in L'' \iff x \in L \text{ and } x \in L'
\]
forces us to say that $(q,q')$ is an accepting state exactly when
$q \in F$ and $q'\in F'$ where $F,F'$ are sets of accepting states
of $M$,$M'$ respectively:
\[
   (q,q') \in F'' \iff q \in F, \,\,\, q' \in F'
\]
This means that we must have
\[
 F'' = F \times F'
\]

Furthermore it's now clear how to define $\delta''$ the transition
function of $M''$ in terms of $\delta, \delta'$ the transition
functions of $M,M'$ respectively. If the symbol is $a \in \Sigma$
and $\delta$ and $\delta'$ have the following effect:
\begin{align*}
 q &\mapsto r \\
 q' &\mapsto r' \\
\end{align*}
for an $a$--transition, then $\delta''$, the transition function
of $M''$ (if $M''$ exists) must have the following effect
\[
(q,q') \mapsto (r,r')
\] In other words
\[
 \delta'' \biggl( (q,q'),a \biggr) = \biggl( \delta(q,a), \,\,\, \delta'(q',a) \biggr)
\]

The above gives the intuition on how to construct $M''$. Now to
present the formal proof:

\begin{proof}
Let $M = (\Sigma, Q, q_0, F, \delta)$ and $M' = (\Sigma,Q', q_0',
F', \delta')$ such that $L = L(M)$ and $L' = L(M')$.

Define $M'' = (\Sigma, Q'', q_0'', F'', \delta'')$ as follows:
 \begin{itemize}
 \item $Q'' = Q \times Q'$
 \item $q_0'' = (q_0,q_0')$. Note that $q_0'' \in Q''$.
 \item $F'' = F \times F'$. Note that $F'' \subseteq Q''$. [Prove
 this yourself.]
 \item $\delta'' : Q'' \times \Sigma \rightarrow Q''$ by
 \[
   \delta''((q,q'),a) = (\delta(q,a), \delta'(q',a))
 \]
 where $a \in \Sigma$ and $(q,q') \in Q''$.
 Note that $\delta''$ is well-defined since $(\delta(q,a),
 \delta'(q',a)) \in Q''$. Recall that $\delta''$ induces the function
 $\delta''^* : \Sigma^* \times Q'' \rightarrow Q''$
 \end{itemize}
By definition, $M''$ is a DFA.

We now want to show that $L(M'') = L(M) \cap L(M')$. Let $x \in
\Sigma^*$. Then
\begin{align*}
x \in L(M'')
&\iff \delta''^*(q_0'',x) \in F'' \\
&\iff \delta''^*((q_0,q'_0),x) \in F \times F' \\
&\iff (\delta^*(q_0,x), \delta'^*(q',x)) \in F \times F' \\
&\iff \delta^*(q_0,x) \in F, \,\,\, \delta'^*(q_0',x) \in F' \\
&\iff x \in L(M), \,\,\, x \in L(M') \\
&\iff x \in L(M) \cap L(M')
\end{align*}
Hence $L(M'') = L(M) \cap L(M')$.
\end{proof}

So the proof is complete, right? Are you sure?
\\

\begin{lem}
Assume the notation in the above proof. Then
\[
 \delta''^*((q,q'),x) = (\delta^*(q,x), \delta^*(q',x))
\]
for any $x \in \Sigma^*$, $q \in Q$ and $q' \in Q'$.
\end{lem}

\begin{proof}
We will prove this by Mathematical Induction on $|x|$.
In other words let $P(n)$ be the statement that for any
string $x \in \Sigma^*$ with $|x| = n$, we have
\[
 \delta''^*((q,q'),x) = (\delta^*(q,x), \delta^*(q',x))
\]

When $|x| = 0$, i.e., $x = \ep$, we have, by definition
\begin{align*}
 \delta''^*((q,q'),x) &= (q,q') \\
 \delta^*(q,x) &= q \\
 \delta^*(q',x) &= q' \\
\end{align*}
Hence $\delta^*((q,q'),x) = (\delta^*(q,x), \delta^*(q',x))$.
Therefore $P(0)$ holds.

Now we want to prove that for $n \geq 0$, if $P(n)$ is true,
then $P(n+1)$ is also true.
Now assume that
\[
 \delta^*((q,q'),x) = (\delta^*(q,x), \delta^*(q',x))
\]
for any string $x \in \Sigma^*$ with $|x| = n
\geq 0$. Consider a string $y\in\Sigma^*$ of length $n+1$. Then $y
= ax$ where $a \in \Sigma$ and $x \in \Sigma^*$ with $|x| = n$.
Then by the recursive definition of $\delta''^*$
\[
\delta''^*((q,q'),ax)
 = \delta''^*\biggl( \delta''((q,q'),a), \,\,\, x\biggr)
\]
For clarity, since $\delta''((q,q'),a) \in Q'' = Q \times Q'$, we
can write $\delta''((q,q'),a) = (\delta(q,a), \delta(q',a)) =
(r,r')$ for some $r \in Q$ and $r' \in Q'$. Continuing the above,
\begin{align*}
\delta''^*((q,q'),ax)
 = \delta''^* ( (r,r'), x )
 = (\delta^*(r,x), \,\,\, \delta'^*(r',x) )
\end{align*}
using the induction hypothesis since $|x| = n$.
 Therefore substituting back the expression for $r$ and $r'$:
\begin{align*}
\delta''^*((q,q'),y)
 &= \biggl(\delta^*(r,x), \,\,\, \delta'^*(r',x) \biggr) \\
 &= \biggl( \delta^*(\delta(q,a),x), \,\,\, \delta'^*(\delta(q',a),x) \biggr) \\
 &= ( \delta^*(q,ax), \delta'^*(q',ax) ) \\
 &= ( \delta^*(q,y),\delta(q',y) )
\end{align*}
Therefore the statement holds for any string $y$ of length $n+1$.

By Mathematical Induction, the statement holds for any string.
\end{proof}

I will call this either the \defone{product construction} 
(because we're using cross product to form the new states) or the 
\defone{intersection construction}.
Note that the above theorem does not just tell you that
if $L$ and $L'$ are regular, then $L \cap L'$ is regular.
The theorem actually includes the recipe for constructing
the DFA for $L \cap L'$ from the DFAs of $L$ and $L'$.
In other words the proof is constructive.
So the proof is very handy.

%-*-latex-*-

\begin{ex} 
  \label{ex:prob-00}
  \tinysidebar{\debug{exercises/{disc-prob-28/question.tex}}}

  \solutionlink{sol:prob-00}
  \qed
\end{ex} 
\begin{python0}
from solutions import *
add(label="ex:prob-00",
    srcfilename='exercises/discrete-probability/prob-00/answer.tex') 
\end{python0}


\newpage
\section{Closure: Union}

In this section, we will prove the closure rule for union.
The idea is actually very simple.
Suppose you have two DFAs $N$ and $N'$ (or DFAs):

\begin{center}
\begin{tikzpicture}[shorten >=1pt,node distance=2cm,auto,initial text=,
    initial where=left,
    state/.style={circle, draw, minimum size=0.4cm},
    accepting/.style={double distance=2pt}
  ]
\node[]                (q0) [shape=circle]       at (-0.5,    0) {$$};
\node[state,initial]   (q1) [shape=circle, draw] at (   0,    0) {$$};
\node[state]           (q2) [shape=circle, draw] at (   1,    1) {$$};
\node[state]           (q3) [shape=circle, draw] at (   1,   -1) {$$};
\node[state          ] (q4) [shape=circle, draw] at (   2, -0.8) {$$};
\node[state          ] (q5) [shape=circle, draw] at (   2,  0.5) {$$};
\node[state,accepting] (q6) [shape=circle, draw] at (   3,    1) {$$};
\node[state,accepting] (q7) [shape=circle, draw] at (   3,   -1) {$$};
\node[thick,draw=black,rounded corners=2mm] (N) [fit=(q0) (q1) (q2) (q3) (q4) (q5) (q6) (q7)] {};
\node[above right] at (N.north west) {$N$};

\node[]                (r0) [shape=circle]       at (4.5,    0) {$$};
\node[state,initial]   (r1) [shape=circle, draw] at (  5,    0) {$$};
\node[state]           (r2) [shape=circle, draw] at (  6,    1) {$$};
\node[state]           (r3) [shape=circle, draw] at (5.7, -0.6) {$$};
\node[state          ] (r4) [shape=circle, draw] at (  7,  0.7) {$$};
\node[state          ] (r5) [shape=circle, draw] at (5.6,  0.5) {$$};
\node[state,accepting] (r6) [shape=circle, draw] at (  8,    1) {$$};
\node[state,accepting] (r7) [shape=circle, draw] at (7.8, -0.4) {$$};
\node[state,accepting] (r8) [shape=circle, draw] at (  8,   -1) {$$};
\node[thick,draw=black,rounded corners=2mm] (Np) [fit=(r0) (r1) (r2) (r3) (r4) (r5) (r6) (r7) (r8)] {};
\node[above right] at (Np.north west) {$N'$};

\end{tikzpicture}\end{center}

\begin{center}
\begin{tikzpicture}[shorten >=1pt,node distance=2cm,auto,initial text=,
    initial where=left,
    state/.style={circle, draw, minimum size=0.4cm},
    accepting/.style={double distance=2pt}
  ]
\node[]                (q0) [shape=circle]       at (-0.5,    0) {$$}; % dummy 
\node[state]           (q1) [shape=circle, draw] at (   0,    0) {$$};
\node[state]           (q2) [shape=circle, draw] at (   1,    1) {$$};
\node[state]           (q3) [shape=circle, draw] at (   1,   -1) {$$};
\node[state          ] (q4) [shape=circle, draw] at (   2, -0.8) {$$};
\node[state          ] (q5) [shape=circle, draw] at (   2,  0.5) {$$};
\node[state,accepting] (q6) [shape=circle, draw] at (   3,    1) {$$};
\node[state,accepting] (q7) [shape=circle, draw] at (   3,   -1) {$$};
\node[thick,draw=black,rounded corners=2mm] (N) [fit=(q0) (q1) (q2) (q3) (q4) (q5) (q6) (q7)] {};
\node[above right] (Nname) at (N.north west) {$N$};

\node[]                (r0) [shape=circle]       at (4.5,    0) {$$}; % dummy
\node[state]           (r1) [shape=circle, draw] at (  5,    0) {$$};
\node[state]           (r2) [shape=circle, draw] at (  6,    1) {$$};
\node[state]           (r3) [shape=circle, draw] at (5.7, -0.6) {$$};
\node[state          ] (r4) [shape=circle, draw] at (  7,  0.7) {$$};
\node[state          ] (r5) [shape=circle, draw] at (5.6,  0.5) {$$};
\node[state,accepting] (r6) [shape=circle, draw] at (  8,    1) {$$};
\node[state,accepting] (r7) [shape=circle, draw] at (7.8, -0.4) {$$};
\node[state,accepting] (r8) [shape=circle, draw] at (  8,   -1) {$$};
\node[thick,draw=black,rounded corners=2mm] (Np) [fit=(r0) (r1) (r2) (r3) (r4) (r5) (r6) (r7) (r8)] {};
\node[above right] (Npname) at (Np.north west) {$N'$};

% N'':
\node[]                (s0) [shape=circle]       at (-2,     0) {$$}; % dummy
\node[state,initial]   (s1) [shape=circle, draw] at (-1.5,   0) {$$};
\path[->] (s1) edge node[pos=0.25] {$\epsilon$} (q1);
\draw[->, line width=0.5pt] (s1) -- node[below right,pos=0.15]{$\epsilon$} ++(0,-1.75) -| (r1);
\node[below = 1.25cm of s1] (dummy) [shape=circle] {}; % dummy
\node[thick,draw=black,rounded corners=2mm] (Npp) [fit=(s0) (N) (Nname) (Np) (Npname) (dummy)] {};
\node[above right] at (Npp.north west) {$N''$};

\end{tikzpicture}\end{center}

Let $L = L(N)$ and $L' = L(N')$ where $N = (\Sigma, Q, q_0, F,
\delta)$ and $N' = (\Sigma, Q', q'_0, F', \delta')$.

Define $N'' = (\Sigma, Q'', q_0'', F'', \delta'')$ as follows:
\begin{itemize}
\item $Q'' = Q \cup Q' \cup\{q_0''\}$. Note that technically
  speaking $Q$ and $Q'$ are sets in different universes. We need to
  assume that we can put them both into the same universe in such a
  way that if $q \in Q$ and $q' \in Q'$, then $q \neq q'$ in this new
  universe. Furthermore $q''_0$ must obviously be an element in this
  universe distinct from any element in $Q$ and $Q'$.
\item $q_0''$ is already explained above
\item $F'' = F \cup F'$
\item Define $\delta'' : \Sigma \times Q'' \rightarrow Q''$ as
  follows: Let $q \in Q''$ and $a \in \Sigma$. Note that since $Q'' = Q \cup Q'$, either
  $q \in Q$ or $q \in Q'$. Furthermore since $Q \cap Q' = \empty$, $q$
  cannot be in both $Q$ and $Q'$. We define
  \[
  \delta''(q,a) =
  \begin{cases}
    \delta(q,a) & \text{ if } q \in Q \\
    \delta'(q,a) & \text{ if } q \in Q'
  \end{cases}
  \]
\end{itemize}
Note that by definition, $N''$ is an NFA.

Now prove that $L(N'') = L \cup L'$. (Exercise).

I'll call this the \defone{union construction}.

%-*-latex-*-

\begin{ex} 
  \label{ex:prob-00}
  \tinysidebar{\debug{exercises/{disc-prob-28/question.tex}}}

  \solutionlink{sol:prob-00}
  \qed
\end{ex} 
\begin{python0}
from solutions import *
add(label="ex:prob-00",
    srcfilename='exercises/discrete-probability/prob-00/answer.tex') 
\end{python0}


\newpage
\section{Closure: Complement}

Given a DFA $M$, you can easily construct another DFA $\overline M$
such that 
\[
L(\overline{M}) = \overline{L(M)}
\]
Here's what you do: you simply change the accept state to non--accept state
and non--accept state to accept state.
Diagrammatically, if this is $M$:
\begin{center}
\begin{tikzpicture}[shorten >=1pt,node distance=2cm,auto,initial text=,
    initial where=left,
    state/.style={circle, draw, minimum size=0.4cm},
    accepting/.style={double distance=2pt}
  ]
\node[]                (q0) [shape=circle]       at (-0.5,    0) {$$};
\node[state,initial]   (q1) [shape=circle, draw] at (   0,    0) {$$};
\node[state]           (q2) [shape=circle, draw] at (   1,    1) {$$};
\node[state]           (q3) [shape=circle, draw] at (   1,   -1) {$$};
\node[state          ] (q4) [shape=circle, draw] at (   2, -0.8) {$$};
\node[state          ] (q5) [shape=circle, draw] at (   2,  0.5) {$$};
\node[state,accepting] (q6) [shape=circle, draw] at (   3,    1) {$$};
\node[state,accepting] (q7) [shape=circle, draw] at (   3,   -1) {$$};
\node[thick,draw=black,rounded corners=2mm] (N) [fit=(q0) (q1) (q2) (q3) (q4) (q5) (q6) (q7)] {};
\node[above right] at (N.north west) {$M$};

\end{tikzpicture}\end{center}

then $M'$ is

\begin{center}
\begin{tikzpicture}[shorten >=1pt,node distance=2cm,auto,initial text=,
    initial where=left,
    state/.style={circle, draw, minimum size=0.4cm},
    accepting/.style={double distance=2pt}
  ]
\node[]                (q0) [shape=circle]       at (-0.5,    0) {$$}; % dummy 
\node[state,initial,accepting]   (q1) [shape=circle, draw] at (   0,    0) {$$};
\node[state,accepting]           (q2) [shape=circle, draw] at (   1,    1) {$$};
\node[state,accepting]           (q3) [shape=circle, draw] at (   1,   -1) {$$};
\node[state,accepting] (q4) [shape=circle, draw] at (   2, -0.8) {$$};
\node[state,accepting] (q5) [shape=circle, draw] at (   2,  0.5) {$$};
\node[state] (q6) [shape=circle, draw] at (   3,    1) {$$};
\node[state] (q7) [shape=circle, draw] at (   3,   -1) {$$};
\node[thick,draw=black,rounded corners=2mm] (N) [fit=(q0) (q1) (q2) (q3) (q4) (q5) (q6) (q7)] {};
\node[above right] (Nname) at (N.north west) {$M$};

% N'':
\node[below = 1.25cm of s1] (dummy) [shape=circle] {}; % dummy
\node[thick,draw=black,rounded corners=2mm] (Npp) [fit=(N) (Nname)] {};
\node[above right] at (Npp.north west) {$M'$};

\end{tikzpicture}\end{center}




WARNING: This method does not work when the construction is applied to an
NFA.

%-*-latex-*-

\begin{ex} 
  \label{ex:prob-00}
  \tinysidebar{\debug{exercises/{disc-prob-28/question.tex}}}

  \solutionlink{sol:prob-00}
  \qed
\end{ex} 
\begin{python0}
from solutions import *
add(label="ex:prob-00",
    srcfilename='exercises/discrete-probability/prob-00/answer.tex') 
\end{python0}


\newpage
\section{Closure: Difference}

This will be an exercise in the future.

Let $L, L'$ be languages.
Then
\[
L - L' = L \cap \overline{L'}
\]
In other words, you use the complement construction and the product construction.
That's it.

%-*-latex-*-

\begin{ex} 
  \label{ex:prob-00}
  \tinysidebar{\debug{exercises/{disc-prob-28/question.tex}}}

  \solutionlink{sol:prob-00}
  \qed
\end{ex} 
\begin{python0}
from solutions import *
add(label="ex:prob-00",
    srcfilename='exercises/discrete-probability/prob-00/answer.tex') 
\end{python0}


\newpage
\section{Closure: Concatenation}

Let $L = L(N)$ and $L' = L(N')$ where $N, N'$ are NFAs (or DFAs).
I'm going to construct asn NFA $N''$ that accepts $LL'$.
(The construction also works when $N$ and $N'$ are DFAs.)

\begin{center}
\begin{tikzpicture}[shorten >=1pt,node distance=2cm,auto,initial text=,
    initial where=left,
    state/.style={circle, draw, minimum size=0.4cm},
    accepting/.style={double distance=2pt}
  ]
\node[]                (q0) [shape=circle]       at (-0.5,    0) {$$};
\node[state,initial]   (q1) [shape=circle, draw] at (   0,    0) {$$};
\node[state]           (q2) [shape=circle, draw] at (   1,    1) {$$};
\node[state]           (q3) [shape=circle, draw] at (   1,   -1) {$$};
\node[state          ] (q4) [shape=circle, draw] at (   2, -0.8) {$$};
\node[state          ] (q5) [shape=circle, draw] at (   2,  0.5) {$$};
\node[state,accepting] (q6) [shape=circle, draw] at (   3,    1) {$$};
\node[state,accepting] (q7) [shape=circle, draw] at (   3,   -1) {$$};
\node[thick,draw=black,rounded corners=2mm] (N) [fit=(q0) (q1) (q2) (q3) (q4) (q5) (q6) (q7)] {};
\node[above right] at (N.north west) {$N$};

\node[]                (r0) [shape=circle]       at (4.5,    0) {$$};
\node[state,initial]   (r1) [shape=circle, draw] at (  5,    0) {$$};
\node[state]           (r2) [shape=circle, draw] at (  6,    1) {$$};
\node[state]           (r3) [shape=circle, draw] at (5.7, -0.6) {$$};
\node[state          ] (r4) [shape=circle, draw] at (  7,  0.7) {$$};
\node[state          ] (r5) [shape=circle, draw] at (5.6,  0.5) {$$};
\node[state,accepting] (r6) [shape=circle, draw] at (  8,    1) {$$};
\node[state,accepting] (r7) [shape=circle, draw] at (7.8, -0.4) {$$};
\node[state,accepting] (r8) [shape=circle, draw] at (  8,   -1) {$$};
\node[thick,draw=black,rounded corners=2mm] (Np) [fit=(r0) (r1) (r2) (r3) (r4) (r5) (r6) (r7) (r8)] {};
\node[above right] at (Np.north west) {$N'$};

\end{tikzpicture}\end{center}

The intuition is as follows.
For a string $x \in \Sigma^*$, $x \in LL'$
if and only if $x$ is accepted by $N$ and after that if
you run it on $N'$, it is again accepted.

But how do you do "\ldots after that if you run it on $N'$\ldots"?
Easy. Join up the two machines using $\ep$-transitions.
Just add $\epsilon$-transitions from the accepting states of $N$ to
the starting state of $N'$.
Of course the accept states of $N$
are \textit{not} accept states in $N''$.
Where do you start? At the starting state
of $N$.
Therefore the start state of $N''$ is the start state of $N$.
When do you finally accept? When the string reaches the
an accept state of $N'$.
Therfore the accept states of $N''$ are the accept states of $N'$. 
Here's $N''$:

\begin{center}
\begin{tikzpicture}[shorten >=1pt,node distance=2cm,auto,initial text=,
    initial where=left,
    state/.style={circle, draw, minimum size=0.4cm},
    accepting/.style={double distance=2pt}
  ]
\node[]                (q0) [shape=circle]       at (-0.5,    0) {$$}; % dummy 
\node[state,initial]   (q1) [shape=circle, draw] at (   0,    0) {$$};
\node[state]           (q2) [shape=circle, draw] at (   1,    1) {$$};
\node[state]           (q3) [shape=circle, draw] at (   1,   -1) {$$};
\node[state          ] (q4) [shape=circle, draw] at (   2, -0.8) {$$};
\node[state          ] (q5) [shape=circle, draw] at (   2,  0.5) {$$};
%\node[state,accepting] (q6) [shape=circle, draw] at (   3,    1) {$$};
\node[state] (q6) [shape=circle, draw] at (   3,    1) {$$};
%\node[state,accepting] (q7) [shape=circle, draw] at (   3,   -1) {$$};
\node[state] (q7) [shape=circle, draw] at (   3,   -1) {$$};
\node[thick,draw=black,rounded corners=2mm] (N) [fit=(q0) (q1) (q2) (q3) (q4) (q5) (q6) (q7)] {};
\node[above right] (Nname) at (N.north west) {$N$};

\node[]                (r0) [shape=circle]       at (4.5,    0) {$$}; % dummy
\node[state]           (r1) [shape=circle, draw] at (  5,    0) {$$};
\node[state]           (r2) [shape=circle, draw] at (  6,    1) {$$};
\node[state]           (r3) [shape=circle, draw] at (5.7, -0.6) {$$};
\node[state          ] (r4) [shape=circle, draw] at (  7,  0.7) {$$};
\node[state          ] (r5) [shape=circle, draw] at (5.6,  0.5) {$$};
\node[state,accepting] (r6) [shape=circle, draw] at (  8,    1) {$$};
\node[state,accepting] (r7) [shape=circle, draw] at (7.8, -0.4) {$$};
\node[state,accepting] (r8) [shape=circle, draw] at (  8,   -1) {$$};
\node[thick,draw=black,rounded corners=2mm] (Np) [fit=(r0) (r1) (r2) (r3) (r4) (r5) (r6) (r7) (r8)] {};
\node[above right] (Npname) at (Np.north west) {$N'$};

% N'':
\node[]                (s0) [shape=circle]       at (-2,     0) {$$}; % dummy

% concat:
\path[->] (q6) edge node[pos=0.27] {$\epsilon$} (r1);
\path[->] (q7) edge node[pos=0.5] {$\epsilon$} (r1);

%\node[state,initial]   (s1) [shape=circle, draw] at (-1.5,   0) {$$};
%\path[->] (s1) edge node[pos=0.25] {$\epsilon$} (q1);
%\draw[->, line width=0.5pt] (s1) -- node[below right,pos=0.15]{$\epsilon$} ++(0,-1.75) -| (r1);
\node[below = 1.25cm of s1] (dummy) [shape=circle] {}; % dummy
%\node[thick,draw=black,rounded corners=2mm] (Npp) [fit=(s0) (N) (Nname) (Np) (Npname) (dummy)] {};
\node[thick,draw=black,rounded corners=2mm] (Npp) [fit=(N) (Nname) (Np) (Npname)] {};
\node[above right] at (Npp.north west) {$N''$};

\end{tikzpicture}\end{center}


It's clear that $N''$ constructed above is an NFA.
Furthermore, it's clear that $L(N'') = LL'$.
Since we know that any language accepted by an NFA is also accepted by a DFA,
we're done.

Don't you think NFAs are nice?

I'll call this the \defone{concatenation construction}.

%-*-latex-*-

\begin{ex} 
  \label{ex:prob-00}
  \tinysidebar{\debug{exercises/{disc-prob-28/question.tex}}}

  \solutionlink{sol:prob-00}
  \qed
\end{ex} 
\begin{python0}
from solutions import *
add(label="ex:prob-00",
    srcfilename='exercises/discrete-probability/prob-00/answer.tex') 
\end{python0}


\newpage
\section{Closure: Kleene Star}

The construction of a DFA of $L^*$ given a DFA of $L$
is simple once you have seen the
construction of the machine for concatenation of languages.

Given an NFA $N$ (or DFA)

\begin{center}
\begin{tikzpicture}[shorten >=1pt,node distance=2cm,auto,initial text=,
    initial where=left,
    state/.style={circle, draw, minimum size=0.4cm},
    accepting/.style={double distance=2pt}
  ]
\node[]                (q0) [shape=circle]       at (-0.5,    0) {$$};
\node[state,initial]   (q1) [shape=circle, draw] at (   0,    0) {$$};
\node[state]           (q2) [shape=circle, draw] at (   1,    1) {$$};
\node[state]           (q3) [shape=circle, draw] at (   1,   -1) {$$};
\node[state          ] (q4) [shape=circle, draw] at (   2, -0.8) {$$};
\node[state          ] (q5) [shape=circle, draw] at (   2,  0.5) {$$};
\node[state,accepting] (q6) [shape=circle, draw] at (   3,    1) {$$};
\node[state,accepting] (q7) [shape=circle, draw] at (   3,   -1) {$$};
\node[thick,draw=black,rounded corners=2mm] (N) [fit=(q0) (q1) (q2) (q3) (q4) (q5) (q6) (q7)] {};
\node[above right] at (N.north west) {$N$};
\end{tikzpicture}\end{center}

we will be constructing a new NFA $N'$ such will accept $L(N)^*$.

First $N'$ contains $N$.
Next you add a new state to $N'$ and make it the initial state of
the new machine and you make this new state an accepting state.
Why?
Because, then $N'$ will accept $\ep$.
Of course the initial state of $N$ is now not an initial state in $N'$.
Now join the new initial state to the old initial state of $N$ with an $\epsilon$--transition.
Next you join the accept states of $N$, if there are any, to the 
old initial state of $N$ with
$\ep$-transitions.
Why? Because we are actually imitating the concatenation construction: if a string is accepted by $N$, then
the string is still accepted by $N'$. But if the string is made up of $xx'$ where $x$ and $x'$ are both
accepted by $N$, then the $\epsilon$--transitions we just added will allow $N'$ to test acceptance on $x'$ starting
from the beginning of $N$. Right?

\begin{center}
\begin{tikzpicture}[shorten >=1pt,node distance=2cm,auto,initial text=,
    initial where=left,
    state/.style={circle, draw, minimum size=0.4cm},
    accepting/.style={double distance=2pt}
  ]
\node[]                (q0) [shape=circle]       at (-0.5,    0) {$$}; % dummy 
\node[state,initial]   (q1) [shape=circle, draw] at (   0,    0) {$$};
\node[state]           (q2) [shape=circle, draw] at (   1,    1) {$$};
\node[state]           (q3) [shape=circle, draw] at (   1,   -1) {$$};
\node[state] (q4) [shape=circle, draw] at (   2, -0.8) {$$};
\node[state] (q5) [shape=circle, draw] at (   2,  0.5) {$$};
\node[state,accepting] (q6) [shape=circle, draw] at (   3,    1) {$$};
\node[state,accepting] (q7) [shape=circle, draw] at (   3,   -1) {$$};
\node[thick,draw=black,rounded corners=2mm] (N) [fit=(q0) (q1) (q2) (q3) (q4) (q5) (q6) (q7)] {};
\node[above right] (Nname) at (N.north west) {$N$};

% N'':
\node[]                (s0) [shape=circle]       at (-2,     0) {$$}; % dummy
\node[]                (s98) [shape=circle]       at (-2,     2) {$$}; % dummy
\node[]                (s99) [shape=circle]       at (-2,     -2) {$$}; % dummy
\node[state,initial,accepting]   (s1) [shape=circle, draw] at (-1.5,   0) {$$};
\path[->] (s1) edge[] node[pos=0.25] {$\epsilon$} (q1);
%\path[->]          (q2)  edge   [bend left=20]   node {b} (q3);
\path[->] (q6) edge[bend right=100] node[pos=0.5, above] {$\epsilon$} (q1);
\path[->] (q7) edge[bend left=100] node[pos=0.5] {$\epsilon$} (q1);

\node[below = 1.25cm of s1] (dummy) [shape=circle] {}; % dummy
\node[thick,draw=black,rounded corners=2mm] (Npp) [fit=(N) (Nname) (s0) (s98) (s99)] {};
\node[above right] at (Npp.north west) {$N'$};

\end{tikzpicture}\end{center}

That's it. 
I'll call this the \defone{Kleene star construction}.

%-*-latex-*-

\begin{ex} 
  \label{ex:prob-00}
  \tinysidebar{\debug{exercises/{disc-prob-28/question.tex}}}

  \solutionlink{sol:prob-00}
  \qed
\end{ex} 
\begin{python0}
from solutions import *
add(label="ex:prob-00",
    srcfilename='exercises/discrete-probability/prob-00/answer.tex') 
\end{python0}


\begin{thm}
Let $N$ be an NFA and $N'$ be the NFA obtained by the Kleene star 
construction.
Then
\[
L(N') = L(N)^*
\]
\end{thm}

%-*-latex-*-

\begin{ex} 
  \label{ex:prob-00}
  \tinysidebar{\debug{exercises/{disc-prob-28/question.tex}}}

  \solutionlink{sol:prob-00}
  \qed
\end{ex} 
\begin{python0}
from solutions import *
add(label="ex:prob-00",
    srcfilename='exercises/discrete-probability/prob-00/answer.tex') 
\end{python0}


%-*-latex-*-

\begin{ex} 
  \label{ex:prob-00}
  \tinysidebar{\debug{exercises/{disc-prob-28/question.tex}}}

  \solutionlink{sol:prob-00}
  \qed
\end{ex} 
\begin{python0}
from solutions import *
add(label="ex:prob-00",
    srcfilename='exercises/discrete-probability/prob-00/answer.tex') 
\end{python0}


%-*-latex-*-

\begin{ex} 
  \label{ex:prob-00}
  \tinysidebar{\debug{exercises/{disc-prob-28/question.tex}}}

  \solutionlink{sol:prob-00}
  \qed
\end{ex} 
\begin{python0}
from solutions import *
add(label="ex:prob-00",
    srcfilename='exercises/discrete-probability/prob-00/answer.tex') 
\end{python0}


%-*-latex-*-

\begin{ex} 
  \label{ex:prob-00}
  \tinysidebar{\debug{exercises/{disc-prob-28/question.tex}}}

  \solutionlink{sol:prob-00}
  \qed
\end{ex} 
\begin{python0}
from solutions import *
add(label="ex:prob-00",
    srcfilename='exercises/discrete-probability/prob-00/answer.tex') 
\end{python0}


\newpage
\section{Closure: Miscellaneous}

There are a lot more closure rules/operations.
Here one:

%-*-latex-*-

\begin{ex} 
  \label{ex:prob-00}
  \tinysidebar{\debug{exercises/{disc-prob-28/question.tex}}}

  \solutionlink{sol:prob-00}
  \qed
\end{ex} 
\begin{python0}
from solutions import *
add(label="ex:prob-00",
    srcfilename='exercises/discrete-probability/prob-00/answer.tex') 
\end{python0}


\newpage
\section{Closure: Homomorphism}

WATCHOUT! The following is not in the textbook!

\begin{defn}
  Let $f : \Sigma \rightarrow \Sigma'$ be a map. Note that $f$
  induces a map $f^* : \Sigma^* \rightarrow \Sigma'^*$ which is
  defined as follows: First of all for $x \in \Sigma$,
  \[
    f^*(x) =
    \begin{cases}
      f(x') \cdot f(x'') & \text{ if } x = x'x'' \neq \ep \text{ where } x' \in \Sigma, x'' \in \Sigma^*\cr
      \ep & \text{ if } x = \ep \cr
    \end{cases}
  \]
\end{defn}

(For the math experts: $f^*$ is a semigroup homomorphism. You can
also extend it to be group homomorphism on corresponding free
group. The $(\cdot)^*$ is a function from sets to semigroups or to
groups).

[CHECK]

%-*-latex-*-

\begin{ex} 
  \label{ex:prob-00}
  \tinysidebar{\debug{exercises/{disc-prob-28/question.tex}}}

  \solutionlink{sol:prob-00}
  \qed
\end{ex} 
\begin{python0}
from solutions import *
add(label="ex:prob-00",
    srcfilename='exercises/discrete-probability/prob-00/answer.tex') 
\end{python0}


%-*-latex-*-

\begin{ex} 
  \label{ex:prob-00}
  \tinysidebar{\debug{exercises/{disc-prob-28/question.tex}}}

  \solutionlink{sol:prob-00}
  \qed
\end{ex} 
\begin{python0}
from solutions import *
add(label="ex:prob-00",
    srcfilename='exercises/discrete-probability/prob-00/answer.tex') 
\end{python0}

