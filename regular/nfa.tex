\sectionthree{Nondeterministic Finite Automata}
\begin{python0}
  from solutions import *; clear()
\end{python0}

Now I'll give you the formal definition of an NFA.
Read the following
definition \textit{very} carefully and make sure the definition models
the above examples of NFA state diagrams.

\begin{defn}
  $N$ is a
  \defone{nondeterministic finite automata NFA}
  if $N =
  (\Sigma,Q,q_0,F,\delta)$ where
  \begin{tightlist}
  \item $\Sigma$ is a finite set. This is called the
    \defone{alphabets}.
  \item $Q$ is a finite set. The elements are called
    \defone{states}.
  \item $q_0 \in Q$ is called the
    \defone{initial state}
    or
    \tinysidebarskip\defone{start state}\sidebarskip{0pt}.
  \item $F \subseteq Q$ and elements of $F$ are called
    \defone{accept states}
    or
    \tinysidebarskip\defone{final states}\sidebarskip{0pt}.
  \item $\delta : Q \times \Sigma_\ep \rightarrow P(Q)$.
    This is called the \defone{transition function}
    of this $N$.
  \end{tightlist}
  In the above, $\Sigma_\ep = \Sigma \cup \{\epsilon\}$
  and $P(Q)$ is the powerset of $Q$.
\end{defn}

%-*-latex-*-

\begin{ex} 
  \label{ex:prob-00}
  \tinysidebar{\debug{exercises/{disc-prob-28/question.tex}}}

  \solutionlink{sol:prob-00}
  \qed
\end{ex} 
\begin{python0}
from solutions import *
add(label="ex:prob-00",
    srcfilename='exercises/discrete-probability/prob-00/answer.tex') 
\end{python0}


%-*-latex-*-

\begin{ex} 
  \label{ex:prob-00}
  \tinysidebar{\debug{exercises/{disc-prob-28/question.tex}}}

  \solutionlink{sol:prob-00}
  \qed
\end{ex} 
\begin{python0}
from solutions import *
add(label="ex:prob-00",
    srcfilename='exercises/discrete-probability/prob-00/answer.tex') 
\end{python0}


%-*-latex-*-

\begin{ex} 
  \label{ex:prob-00}
  \tinysidebar{\debug{exercises/{disc-prob-28/question.tex}}}

  \solutionlink{sol:prob-00}
  \qed
\end{ex} 
\begin{python0}
from solutions import *
add(label="ex:prob-00",
    srcfilename='exercises/discrete-probability/prob-00/answer.tex') 
\end{python0}


%-*-latex-*-

\begin{ex} 
  \label{ex:prob-00}
  \tinysidebar{\debug{exercises/{disc-prob-28/question.tex}}}

  \solutionlink{sol:prob-00}
  \qed
\end{ex} 
\begin{python0}
from solutions import *
add(label="ex:prob-00",
    srcfilename='exercises/discrete-probability/prob-00/answer.tex') 
\end{python0}


\newpage
\begin{defn}
Let $S \subseteq Q$ and $a \in \Sigma_\ep$. Note that we can
define
\[
 \delta(S,a) := \bigcup_{q \in S} \delta(q,a)
\]
The following is a computation: Let $a \in \Sigma_\ep$.
\[ (S,ax) \vdash (\overline{\delta(S,a)}, x) \]
where the notation $\overline{\textwhite{123}}$ means the following: If $S$ is a
set of states, then we want to include those states reachable from
$S$ through an $\ep$--transition, i.e.,
 \begin{tightlist}
  \item $q \in S$ $\implies$ $q \in \overline{S}$
  \item $q \in \overline{S}$ $\implies$ $\delta(q,\ep) \subseteq
  \overline{S}$
 \end{tightlist}
\end{defn}
Informally, you think of it this way:
\begin{enumerate}
\item Let $S_0$ be the set of all
  state $S$.
\item
  Let $S_1$ be the set of states in $S_0$
  as well as the states $q' \in \delta(q, \epsilon)$
  where $q \in S_0$.
\item
  In general,
  let $S_i$ be the set of states in $S_{i-1}$
  as well as the states $q' \in \delta(q, \epsilon)$
  where $q \in S_{i-1}$.
\end{enumerate}
At some point $S_i$ will stop changing, say $S_n$.
Then $S_n$ is the set of all states that are
\defone{reachable} from $S$, i.e.,
viewing the state diagram as a directed graph,
$S_n$ are the states $q'$ such that there is a path
from some $q \in S$ to $q'$.
In other words the computation of $\overline{S}$ is a
\defone{reachability problem} in directed graph.
If you remove from the graph where the transitions are
not labeled $\ep$ (only $\epsilon$--transitions are retained), then you are
also computing the \defone{transitive closure} of $S$.
(This is from discrete 1.)

Easy, right?
This is sometimes called the \defone{$\epsilon$--closure} of the
set $S$.

As an algorithm, here's how you compute the $\epsilon$--closure of $S$:
\begin{Verbatim}[frame=single, fontsize=\small]
ALGORITHM: EPSILON-CLOSURE
INPUT: S - a set of states
       delta - a transition function

let DONE be an empty set
put all the states in S into BOUNDARY
let NEW be an empty set       
while 1:
    for q in BOUNDARY:
        put states in delta(q, epsilon) which are not in DONE
        and not in BOUNDARY into NEW
    put all the states in BOUNDARY into DONE
    if NEW is empty:
        break the loop
    else:
        put all the states in NEW into BOUNDARY

return DONE
\end{Verbatim}
For states not in $S$, each of them is placed in NEW and then in
BOUNDARY and then DONE.
Assume that adding and removing from NEW, BOUNDARY, DONE
can be done in $O(1)$.
%-*-latex-*-

\begin{ex} 
  \label{ex:prob-00}
  \tinysidebar{\debug{exercises/{disc-prob-28/question.tex}}}

  \solutionlink{sol:prob-00}
  \qed
\end{ex} 
\begin{python0}
from solutions import *
add(label="ex:prob-00",
    srcfilename='exercises/discrete-probability/prob-00/answer.tex') 
\end{python0}


\newpage
\begin{defn}
  Let $N$ be an NFA.
  The definition of $\vdash^*$ from $\vdash$ is just like the case of the
  DFA.
  A string $x \in \Sigma^*$ is accepted by the above $N$ if
  \[
  (\overline{\{q_0\}}, x) \vdash^* (S, \ep)
  \]
  and $S \subseteq Q$ contains at least one accepting state, i.e.,
  $S \cap F \neq \emptyset$.
\end{defn}

Here's an example to prove the power of the NFA ...

%-*-latex-*-

\begin{ex} 
  \label{ex:prob-00}
  \tinysidebar{\debug{exercises/{disc-prob-28/question.tex}}}

  \solutionlink{sol:prob-00}
  \qed
\end{ex} 
\begin{python0}
from solutions import *
add(label="ex:prob-00",
    srcfilename='exercises/discrete-probability/prob-00/answer.tex') 
\end{python0}

