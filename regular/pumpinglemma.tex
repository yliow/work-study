\sectionthree{Pumping Lemma}
\begin{python0}
  from solutions import *; clear()
\end{python0}

So far we have to been working 
with regular languages.
The question now to ask is this:
Are there languages which are not regular?
If the answer is no, then in a sense we know how to compute with language:
we simply use DFAs (or NFAs or regex.)
However if the answer is YES, then we would need to ask if there are other
devices which are more \lq\lq powerful'' than
DFA, NFA and regex?

It turns out that there are in fact non-regular languages.
Recall that Myhill-Nerode's theorem can tell you exactly
when a language is not regular.
However Myhill--Nerode is hard to use.
It's simple to use this theorem:

\begin{thm}\textbf{\defone{\textnormal{Pumping Lemma}}.}
Let $L$ be a regular language over $\Sigma$. There is a number $n$
such that for each $x \in L$ with $|x| \geq n$, there exists $u,v,w
\in \Sigma^*$ such that
\begin{tightlist}
  \item[\textnormal{(a)}] $x = uvw$
  \item[\textnormal{(b)}] $|uv| \leq n$
  \item[\textnormal{(c)}] $|v| > 0$
  \item[\textnormal{(d)}] $uv^iw \in L$ for all $i \geq 0$.
\end{tightlist}
\end{thm}

That's quite a long theorem!
By the way, do you realize that the conclusion of the theorem
says something like this:
\[
\exists n \left( \forall x \left( \exists u,v,w (...) \right) \right)
\]
Yikes!
Sure looks a little like the $\epsilon$--$\delta$ definition
of limits in calculus:
$\forall \epsilon ( \exists \delta (...))$

\begin{cor}
Let $L$ be a language over $\Sigma$. Then $L$ is non-regular if for
any $n \geq 0$, there exists $x \in L$ with $|x| \geq n$ such
that if $x=uvw$ for any $u,v,w \in \Sigma^*$
 \begin{tightlist}
  \item[\textnormal{(a)}] $|uv| \leq n$
  \item[\textnormal{(b)}] $|v| > 0$
  \item[\textnormal{(c)}] $uv^{i_0}w \notin L$ for some $i_0 \geq 0$.
 \end{tightlist}
\end{cor}

I repeat: If you want to prove $L$ is not regular, then you need to
show the following. For each positive integer $n \geq 0$, you must
write down a string $x \in L$ such that
 \begin{enumerate}[label=\textnormal{(\alph*)},itemsep=0pt,nosep,noitemsep,partopsep=0pt,topsep=0pt,parsep=0pt]
  \item[(a)] $|x| \geq n$
  \item[(b)]$x$ satisfied the following property: Let $u,v,w \in
  \Sigma^*$ be any strings such that $x = uvw$, $|uv| \leq n$ and
  $|v| > 0$, you must find some integer $i_0 \geq 0$ such that
  $uv^{i_0}w \notin L$.
 \end{enumerate}

So here's a template of the proof that a language $L$ is not
regular.

\boxpar{ We want to show $L$ is not regular.
\\

Let $n \geq 0$ be a positive integer. Let $x = \underline{\hspace
{1cm}}$. $x$ is in $L$ because \underline{\hspace{1cm}}
\textsc{(explain only if not obvious)}.
\\

Let $u,v,w$ be any string in $\Sigma^*$ such that $x = uvw$
and
\begin{tightlist}
  \item[(a)] $|uv| \leq n$
  \item[(b)] $|v| > 0$
  \item[(c)] $uv^{i_0}w \notin L$ for some $i_0 \geq 0$.
\end{tightlist}

\[
\underline{\hspace{1cm}}\textsc{(analyze
$u,v,w$)}\underline{\hspace{1cm}}
\]

Let $i_0 = \underline{\hspace{1cm}}$ and consider $uv^{i_0}w$.

\[
\underline{\hspace{1cm}}\textsc{(analyze
$uv^{i_0}w$)}\underline{\hspace{1cm}}
\]

Therefore $uv^{i_0}w \notin L$.
\\

Hence by the Pumping Lemma for regular languages, $L$ is not
regular.
}


\begin{lem}
 Let $G = (V,E)$ be a graph with $n$ nodes.
 If $P$ is a path with $n$ node, then $P$ must contain a cycle.
 Specifically if $P = (e_1,\ldots,e_{n-1})$ where $e_i \in E$ for
 $1 \leq n \leq n-1$,
 then there exists $k,\ell$ such that $1\leq k \leq \ell \leq n-2$
 if $P_0 = (e_1,\ldots,e_k)$,
 $P_1 = (e_{k+1},\ldots,e_\ell)$,
 $P_2 = (e_{\ell+1},\ldots,e_{n-1})$
 then $P_0P_1^iP_2$ is a path in $G$ for all $i \geq 0$.
\end{lem}

%-*-latex-*-

\begin{ex} 
  \label{ex:prob-00}
  \tinysidebar{\debug{exercises/{disc-prob-28/question.tex}}}

  \solutionlink{sol:prob-00}
  \qed
\end{ex} 
\begin{python0}
from solutions import *
add(label="ex:prob-00",
    srcfilename='exercises/discrete-probability/prob-00/answer.tex') 
\end{python0}


%-*-latex-*-

\begin{ex} 
  \label{ex:prob-00}
  \tinysidebar{\debug{exercises/{disc-prob-28/question.tex}}}

  \solutionlink{sol:prob-00}
  \qed
\end{ex} 
\begin{python0}
from solutions import *
add(label="ex:prob-00",
    srcfilename='exercises/discrete-probability/prob-00/answer.tex') 
\end{python0}


\newpage
When we prove certain languages are non-regular, 
we will frequently look at substrings within a larger strings. 
For instance suppose $x$ is a string such that
\[
x = x_1 x_2 = y_1 y_2
\]
where $x_1, x_2, y_1, y_2$ are strings.
Suppose further that $|x_1| = 5$ and $|y_1| = 10$,
then $x_1$ must clearly be a substring of $y_1$.
Right?

The general (obvious) fact is this:
If 
\[
x_1 x_2 = y_1 y_2
\]
where $x_1, x_2, y_1, y_2$ are strings and 
\[
|x_1| \leq |y_1|
\]
then $x_1$ is a substring of $y_1$.
(In fact it's a left substring, i.e. a prefix, of $y_1$.)


In particular, if
\[
x_1 x_2 = a^n b^n
\]
and $|x_1| \leq n$, then $x_1$ must be a substring of $a^n$.
This means that $x_1$ can only be made up of $a$'s.
Right?
Since the length of $x_1$ is written $|x_1|$, we must have
\[
x_1 = a^{|x_1|}
\]
Note that since $x_1$ takes up $|x_1|$ number of 
$a$'s, the amount of $a$'s left is $n - |x_1|$.
Therefore the above also imply that
\[
x_2 = a^{n - |x_1|} b^n
\]

Here's another example.
Let 
\[
x_1 x_2 = a^m b^{2m} 
\]
where $|x_1| < m$.
Do you see that
\[
x_1 = a^{|x_1|}
\]
and
\[
x_2 = a^{m - |x_1|} b^{2m}
\]

\newpage
\begin{eg}
 $L = \{a^ib^i \,|\, i\geq 0 \}$ is not regular.

\textbf{Solution.}
Let $n \geq 0$. We choose $x = a^nb^n$. Note that $x \in
L$ and $|x| = 2n \geq n \geq 0$. 
Let $u,v,w$ be strings in $\Sigma^*$ such that
\begin{enumerate}[label=\textnormal{(\alph*)},itemsep=0pt,nosep,noitemsep,partopsep=0pt,topsep=0pt,parsep=0pt]
 \item[(a)] $x = uvw$
 \item[(b)] $|uv| \leq n$
 \item[(c)] $|v| > 0$
\end{enumerate}

Since $|uv| \leq n$, and $uv$ is a prefix of $x = a^n b^n$.
This implies that $uv$ is made up of only $a$'s.
Therefore 
\[
u = a^{|u|}, \,\,\, v = a^{|v|}
\]
We also have 
\[
w = a^{n-|u|-|v|}b^n
\]

Let $i_0 = 2$. 
Note that
\[
u v^{i_0} w 
= uv^2w 
= a^{|u|} (a^{|v|})^2 a^{n - |u| - |v|}b^n 
= a^{|u|+2|v| + |n| - |u|- |v|}b^n = a^{n+|v|}b^n
\]
Since
$|v|>0$, we have
\[
n+|v| > n
\]
Therefore 
\[
uv^{i_0}w = a^{n+|v|}b^n \notin L
\]

Therefore by the Pumping Lemma for regular languages, $L$ is not
regular.
\end{eg}



[Informally,
since $|uv| \leq n$, $uv$ must be a substring of $a^n$ in $x = a^nb^n$.
Also note that since $|v| > 0$, $uv^2w$ would increase
the number of $a$'s in in $a^nb^n$ without changing the number of
$b$'s.]


%-*-latex-*-

\begin{ex} 
  \label{ex:prob-00}
  \tinysidebar{\debug{exercises/{disc-prob-28/question.tex}}}

  \solutionlink{sol:prob-00}
  \qed
\end{ex} 
\begin{python0}
from solutions import *
add(label="ex:prob-00",
    srcfilename='exercises/discrete-probability/prob-00/answer.tex') 
\end{python0}


%-*-latex-*-

\begin{ex} 
  \label{ex:prob-00}
  \tinysidebar{\debug{exercises/{disc-prob-28/question.tex}}}

  \solutionlink{sol:prob-00}
  \qed
\end{ex} 
\begin{python0}
from solutions import *
add(label="ex:prob-00",
    srcfilename='exercises/discrete-probability/prob-00/answer.tex') 
\end{python0}


%-*-latex-*-

\begin{ex} 
  \label{ex:prob-00}
  \tinysidebar{\debug{exercises/{disc-prob-28/question.tex}}}

  \solutionlink{sol:prob-00}
  \qed
\end{ex} 
\begin{python0}
from solutions import *
add(label="ex:prob-00",
    srcfilename='exercises/discrete-probability/prob-00/answer.tex') 
\end{python0}


%-*-latex-*-

\begin{ex} 
  \label{ex:prob-00}
  \tinysidebar{\debug{exercises/{disc-prob-28/question.tex}}}

  \solutionlink{sol:prob-00}
  \qed
\end{ex} 
\begin{python0}
from solutions import *
add(label="ex:prob-00",
    srcfilename='exercises/discrete-probability/prob-00/answer.tex') 
\end{python0}


%-*-latex-*-

\begin{ex} 
  \label{ex:prob-00}
  \tinysidebar{\debug{exercises/{disc-prob-28/question.tex}}}

  \solutionlink{sol:prob-00}
  \qed
\end{ex} 
\begin{python0}
from solutions import *
add(label="ex:prob-00",
    srcfilename='exercises/discrete-probability/prob-00/answer.tex') 
\end{python0}


%-*-latex-*-

\begin{ex} 
  \label{ex:prob-00}
  \tinysidebar{\debug{exercises/{disc-prob-28/question.tex}}}

  \solutionlink{sol:prob-00}
  \qed
\end{ex} 
\begin{python0}
from solutions import *
add(label="ex:prob-00",
    srcfilename='exercises/discrete-probability/prob-00/answer.tex') 
\end{python0}


%-*-latex-*-

\begin{ex} 
  \label{ex:prob-00}
  \tinysidebar{\debug{exercises/{disc-prob-28/question.tex}}}

  \solutionlink{sol:prob-00}
  \qed
\end{ex} 
\begin{python0}
from solutions import *
add(label="ex:prob-00",
    srcfilename='exercises/discrete-probability/prob-00/answer.tex') 
\end{python0}

\vspace{0.1in}

%-*-latex-*-

\begin{ex} 
  \label{ex:prob-00}
  \tinysidebar{\debug{exercises/{disc-prob-28/question.tex}}}

  \solutionlink{sol:prob-00}
  \qed
\end{ex} 
\begin{python0}
from solutions import *
add(label="ex:prob-00",
    srcfilename='exercises/discrete-probability/prob-00/answer.tex') 
\end{python0}


\newpage
\begin{eg}
Prove that $L = \{a^{i^2} \,|\, i \geq 0\}$ is not regular.

\textbf{Solution.} Let $n \geq 0$ and $x = a^{n^2}$. Note that $x \in
L$ and $|x| \geq n$.

Let $x = uvw$ where $u,v,w \in \Sigma^*$, $|uv| \leq n$ and $|v|>0$. We
will show that $uv^2w \notin L$.

Consider $uv^2w = a^{n^2+|v|}$. We now analyze $n^2+|v|$. Note
that $|v| \leq |uv|\leq n$. By assumption $|v|>0$. Therefore
altogether we have $0 < |v| \leq n$ and hence
\[
 n^2 < n^2 + |v| \leq n^2 + n
\]
Now we note that
\[
 n^2 + n < n^2 + 2n + 1 = (n+1)^2
\]
Altogether we have shown that
\[
 n^2 < n^2+|v| < (n+1)^2
\]
There are obviously no squares of integers in the interval
$(n^2,(n+1)^2)$ (why?). Hence $n^2+|v|$ cannot be a square.
Therefore $wv^2w = a^{n^2+|v|} \notin L$.

Therefore by the Pumping Lemma for regular languages, $L$ is not
regular.
\\
\end{eg}


%-*-latex-*-

\begin{ex} 
  \label{ex:prob-00}
  \tinysidebar{\debug{exercises/{disc-prob-28/question.tex}}}

  \solutionlink{sol:prob-00}
  \qed
\end{ex} 
\begin{python0}
from solutions import *
add(label="ex:prob-00",
    srcfilename='exercises/discrete-probability/prob-00/answer.tex') 
\end{python0}


%-*-latex-*-

\begin{ex} 
  \label{ex:prob-00}
  \tinysidebar{\debug{exercises/{disc-prob-28/question.tex}}}

  \solutionlink{sol:prob-00}
  \qed
\end{ex} 
\begin{python0}
from solutions import *
add(label="ex:prob-00",
    srcfilename='exercises/discrete-probability/prob-00/answer.tex') 
\end{python0}


%-*-latex-*-

\begin{ex} 
  \label{ex:prob-00}
  \tinysidebar{\debug{exercises/{disc-prob-28/question.tex}}}

  \solutionlink{sol:prob-00}
  \qed
\end{ex} 
\begin{python0}
from solutions import *
add(label="ex:prob-00",
    srcfilename='exercises/discrete-probability/prob-00/answer.tex') 
\end{python0}


\newpage
\begin{eg}
Prove $L = \{a^p \,|\, p \text{ is prime } \}$ is not regular.

\textbf{Solution.} Let $n \geq 0$. Let $p \geq n$ be a prime; this is
possible since there are infinitely many primes. Consider $x = a^p$.
Note that $x \in L$ and $|x| = p \geq n$.

Let $u,w,v \in \Sigma$ such that $x = uvw$. Now we consider
$uv^{k+1}w$; we will choose $k\geq 0$ in a minute so that the
length of $uv^{k+1}w$ is not a prime. Note that
\[
 |uv^{k+1}w| = |uvw| + k|v|
\]
Choose $k = |uvw|$ and we get
\[
|uv^{k+1}w| = |uvw| + |uvw||v| = |uvw|(1 + |v|)
\]
which is not a prime and so $uv^{1+|uvw|}w \notin L$.

By the Pumping Lemma for regular languages, $L$ is not regular.
\\
\end{eg}

%-*-latex-*-

\begin{ex} 
  \label{ex:prob-00}
  \tinysidebar{\debug{exercises/{disc-prob-28/question.tex}}}

  \solutionlink{sol:prob-00}
  \qed
\end{ex} 
\begin{python0}
from solutions import *
add(label="ex:prob-00",
    srcfilename='exercises/discrete-probability/prob-00/answer.tex') 
\end{python0}


%-*-latex-*-

\begin{ex} 
  \label{ex:prob-00}
  \tinysidebar{\debug{exercises/{disc-prob-28/question.tex}}}

  \solutionlink{sol:prob-00}
  \qed
\end{ex} 
\begin{python0}
from solutions import *
add(label="ex:prob-00",
    srcfilename='exercises/discrete-probability/prob-00/answer.tex') 
\end{python0}


\newpage
You can also combine regular operators with PL to prove that a language
is not regular.
Here, a regular operators means a set operator which is closed
for regular languages, meaning to say that regular languages remains
closed for such operators.
You already know that the following closed:
\begin{enumerate}[topsep=0mm]
\item If $L,L'$ are regular, then $L \cup L'$ is also regular
\item If $L,L'$ are regular, then $L \cap L'$ is also regular
\item If $L,L'$ are regular, then $LL'$ is also regular
\item If $L$ are regular, then $L^*$ is also regular
\item If $L$ is regular, then $\overline{L}$ is also regular
\end{enumerate}


\begin{eg} Let $\Sigma = \{a,b\}$.
Is 
\[
L = \{ x \in \Sigma^* \,|\, x \text{ has same number of $a$'s
and $b$'s} \}
\]
regular?
\end{eg}

\textbf{Solution.} Assume $L$ is regular.

$L(a^*b^*) = \{a^mb^n\,|\, m\geq 0, \, n \geq 0\}$ is 
regular since languages generates by regex's are also regular.
Now note that
\[
L \cap L(a^*b^*) = \{a^nb^n \,|\, n \geq 0\}
\]
$L(a^*b^*)$ is regular since it is the language generated by a regex.
Since intersection is a closed regular operator and we assumed that
$L$ is regular, we conclude that $L \cap L(a^*b^*)$ is also regular.

However we have already shown that $\{a^nb^n \,|\, n \geq 0\}$ is
\textit{not} regular.

This gives us a contradiction.
Therefore our assumption that $L$ is regular cannot hold.
Hence $L$ is not regular.

%-*-latex-*-

\begin{ex} 
  \label{ex:prob-00}
  \tinysidebar{\debug{exercises/{disc-prob-28/question.tex}}}

  \solutionlink{sol:prob-00}
  \qed
\end{ex} 
\begin{python0}
from solutions import *
add(label="ex:prob-00",
    srcfilename='exercises/discrete-probability/prob-00/answer.tex') 
\end{python0}


%-*-latex-*-

\begin{ex} 
  \label{ex:prob-00}
  \tinysidebar{\debug{exercises/{disc-prob-28/question.tex}}}

  \solutionlink{sol:prob-00}
  \qed
\end{ex} 
\begin{python0}
from solutions import *
add(label="ex:prob-00",
    srcfilename='exercises/discrete-probability/prob-00/answer.tex') 
\end{python0}


%-*-latex-*-

\begin{ex} 
  \label{ex:prob-00}
  \tinysidebar{\debug{exercises/{disc-prob-28/question.tex}}}

  \solutionlink{sol:prob-00}
  \qed
\end{ex} 
\begin{python0}
from solutions import *
add(label="ex:prob-00",
    srcfilename='exercises/discrete-probability/prob-00/answer.tex') 
\end{python0}

\vspace{0.1in}

%-*-latex-*-

\begin{ex} 
  \label{ex:prob-00}
  \tinysidebar{\debug{exercises/{disc-prob-28/question.tex}}}

  \solutionlink{sol:prob-00}
  \qed
\end{ex} 
\begin{python0}
from solutions import *
add(label="ex:prob-00",
    srcfilename='exercises/discrete-probability/prob-00/answer.tex') 
\end{python0}

\vspace{0.1in}

%-*-latex-*-

\begin{ex} 
  \label{ex:prob-00}
  \tinysidebar{\debug{exercises/{disc-prob-28/question.tex}}}

  \solutionlink{sol:prob-00}
  \qed
\end{ex} 
\begin{python0}
from solutions import *
add(label="ex:prob-00",
    srcfilename='exercises/discrete-probability/prob-00/answer.tex') 
\end{python0}

