\section{Nondeterministic Finite Automata}

Now I'll give you the formal definition of an NFA.
Read the following
definition \textit{very} carefully and make sure the definition models
the above examples of NFA state diagrams.

\begin{defn}
  $N$ is a
  \defone{nondeterministic finite automata NFA}
  if $N =
  (\Sigma,Q,q_0,F,\delta)$ where
  \begin{tightlist}
  \item $\Sigma$ is a finite set. This is called the
    \defone{alphabets}.
  \item $Q$ is a finite set. The elements are called
    \defone{states}.
  \item $q_0 \in Q$ is called the
    \defone{initial state}
    or
    \tinysidebarskip\defone{start state}\sidebarskip{0pt}.
  \item $F \subseteq Q$ and elements of $F$ are called
    \defone{accept states}
    or
    \tinysidebarskip\defone{final states}\sidebarskip{0pt}.
  \item $\delta : Q \times \Sigma_\ep \rightarrow P(Q)$.
    This is called the \defone{transition function}
    of this $N$.
  \end{tightlist}
  In the above, $\Sigma_\ep = \Sigma \cup \{\epsilon\}$
  and $P(Q)$ is the powerset of $Q$.
\end{defn}

\newpage
\begin{ex}
  Here's an NFA state diagram
  for $\emptyset$ for $\Sigma = \{a,b\}$:

\begin{python}
from latextool_basic import *
print(automata(layout="""
A
""",
edges="",
A="initial|label=$q_0$",
))
\end{python}

Write down the formal definition of this NFA.
\end{ex}

\SOLUTION

The formal definition of this NFA is $(\Sigma, Q, q_0, \delta, F)$ where
\begin{tightlist}
\li $\Sigma = \{a,b\}$
\li $Q = \{q_0\}$
\li $\delta$ is the function
\[
\delta : Q \times \Sigma_\epsilon \rightarrow P(Q)
\]
given by
\begin{align*}
  \delta(q_0, \epsilon) &= \{\} \\
  \delta(q_0, a) &= \{\} \\
  \delta(q_0, b) &= \{\} 
\end{align*}
\end{tightlist}


\newpage
\begin{ex}
Draw the (obvious) NFA $N$ for the words of $\{a,b\}^*$ containing $aba$.
Write down the formal definition of $N$.
\qed
\end{ex}

\newpage
\begin{ex}
Draw the (obvious) NFA $N$ for the words of $\{a,b\}^*$ containing $abab$
or $baba$.
Write down the formal definition of $N$.
\end{ex}
\qed

\newpage
\begin{ex}
  Write down the formal definition of this NFA:
\begin{python}
from latextool_basic import *
print(automata(layout="""
A   B   D

    C
""",
edges="A,$\epsilon$,B|A,$a$,A|A,$a$,C|C,$a,\epsilon$,B|B,$b$,D|D,$a,b$,C",
A="initial|label=$q_0$",
B="label=$q_1$",
C="label=$q_2$",
D="accept|label=$q_3$",
))
\end{python}
Compute all the states reached by the following words:
\begin{enumerate}[label=(\alph*)]
\item $\ep$
\item $a$
\item $b$
\item $aa$
\item $ab$
\item $ab$
\item $ba$
\item $bb$
\item Can you find a \lq\lq simpler" NFA that accepts the same words as the above NFA?
  \lq\lq Simpler" means \lq\lq fewer states".
\end{enumerate}
\end{ex}


\newpage
\begin{defn}
Let $S \subseteq Q$ and $a \in \Sigma_\ep$. Note that we can
define
\[
 \delta(S,a) := \bigcup_{q \in S} \delta(q,a)
\]
The following is a computation: Let $a \in \Sigma_\ep$.
\[ (S,ax) \vdash (\overline{\delta(S,a)}, x) \]
where the notation $\overline{\textwhite{123}}$ means the following: If $S$ is a
set of states, then we want to include those states reachable from
$S$ through an $\ep$--transition, i.e.,
 \begin{tightlist}
  \item $q \in S$ $\implies$ $q \in \overline{S}$
  \item $q \in \overline{S}$ $\implies$ $\delta(q,\ep) \subseteq
  \overline{S}$
 \end{tightlist}
\end{defn}
Informally, you think of it this way:
\begin{enumerate}
\item Let $S_0$ be the set of all
  state $S$.
\item
  Let $S_1$ be the set of states in $S_0$
  as well as the states $q' \in \delta(q, \epsilon)$
  where $q \in S_0$.
\item
  In general,
  let $S_i$ be the set of states in $S_{i-1}$
  as well as the states $q' \in \delta(q, \epsilon)$
  where $q \in S_{i-1}$.
\end{enumerate}
At some point $S_i$ will stop changing, say $S_n$.
Then $S_n$ is the set of all states that are
\defone{reachable} from $S$, i.e.,
viewing the state diagram as a directed graph,
$S_n$ are the states $q'$ such that there is a path
from some $q \in S$ to $q'$.
In other words the computation of $\overline{S}$ is a
\defone{reachability problem} in directed graph.
If you remove from the graph where the transitions are
not labeled $\ep$ (only $\epsilon$--transitions are retained), then you are
also computing the \defone{transitive closure} of $S$.
(This is from discrete 1.)

Easy, right?
This is sometimes called the \defone{$\epsilon$--closure} of the
set $S$.

As an algorithm, here's how you compute the $\epsilon$--closure of $S$:
\begin{Verbatim}[frame=single, fontsize=\small]
ALGORITHM: EPSILON-CLOSURE
INPUT: S - a set of states
       delta - a transition function

let DONE be an empty set
put all the states in S into BOUNDARY
let NEW be an empty set       
while 1:
    for q in BOUNDARY:
        put states in delta(q, epsilon) which are not in DONE
        and not in BOUNDARY into NEW
    put all the states in BOUNDARY into DONE
    if NEW is empty:
        break the loop
    else:
        put all the states in NEW into BOUNDARY

return DONE
\end{Verbatim}
For states not in $S$, each of them is placed in NEW and then in
BOUNDARY and then DONE.
Assume that adding and removing from NEW, BOUNDARY, DONE
can be done in $O(1)$.
\newpage
\begin{ex}
Refer to our first NFA again:
\begin{center}
\begin{tikzpicture}[shorten >=1pt,node distance=2cm,auto,initial text=,
initial where=above]
\node[state,initial]   (q0) [shape=circle, draw] at ( 0,  0) {$q_0$};
\node[state]           (q1) [shape=circle, draw] at (-1, -4) {$q_1$};
\node[state]           (q3) [shape=circle, draw] at ( 1, -2) {$q_3$};
\node[state,accepting] (q4) [shape=circle, draw] at ( 1, -4) {$q_4$};
\node[state,accepting] (q2) [shape=circle, draw] at (-1, -6) {$q_2$};

\path[->]
(q0) edge [loop right] node {$a,b$} ()
(q0) edge node {$b$} (q1)
(q0) edge node {$\epsilon$} (q3)
(q1) edge node {$b$} (q2)
(q3) edge node {$\epsilon$} (q1)
(q3) edge node {$a$} (q4)
;
\end{tikzpicture}\end{center}
\begin{itemize}
\item What is $\overline{\{\}}$? 
\item What about $\overline{\{q_0\}}$?
\item What about $\overline{\{q_1\}}$?
\item What about $\overline{\{q_2\}}$?
\item What about $\overline{\{q_4\}}$?
\item What about $\overline{\{q_0, q_1\}}$?
\item Write down $\overline{S}$ for all subsets $S$ of $\{q_0, ..., q_4\}$.
\end{itemize}
\qed
\end{ex}




\newpage
\begin{defn}
  Let $N$ be an NFA.
  The definition of $\vdash^*$ from $\vdash$ is just like the case of the
  DFA.
  A string $x \in \Sigma^*$ is accepted by the above $N$ if
  \[
  (\overline{\{q_0\}}, x) \vdash^* (S, \ep)
  \]
  and $S \subseteq Q$ contains at least one accepting state, i.e.,
  $S \cap F \neq \emptyset$.
\end{defn}



Here's an example to prove the power of the NFA ...


\newpage
\begin{ex}
Design an NFA $N$ that accepts words in $\{a, b\}^*$ which ends in 
either $aba$ or $aba$.
Write down the formal definition of $N$. 
\qed
\end{ex}
