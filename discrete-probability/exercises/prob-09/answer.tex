%-*-latex-*-
\tinysidebar{\debug{exercises/{prob-09/answer.tex}}}

(a)
One way to get 2 sixes is $(6, 6, a, b, c)$ where $a, b, c$ are not sixes.
There are $6^5$ possible outcomes for rolling 5 dice.
For $a,b,c$ all not sixes,
the probability of getting $(6, 6, a, b, c)$ is $1/6^5$.
Therefore the probability of getting $(6, 6, a, b, c)$ for all possible
cases where $a,b,c$ are not sixes is $5^3/6^5$.
I can also get exactly two sixes with $(6, a, 6, b, c)$ or
$(6, a, b, 6, c)$, etc.

The total number of ways to permute $6, 6, a, b, c$ is
$\binom{5}{2} \cdot 5^3$
Therefore the probability of getting exactly two sixes is
\[
\binom{5}{2} \cdot 5^3 \cdot \frac{1}{6^5} \approx 0.16075102880658437
\]

Here a program to check the above computation:
0.160396 0.16075102880658437




(b)
The probability of getting at least two sixes
and the probability of getting at most one six adds up to 1.
Therefore
the probability of getting at least two sixes is
\[
1 - \text{probabilty of getting at most one six}
\]
The probability of getting no sixes is $5^5/6^5$.
The probability of getting exactly one six is $\binom{5}{1} 5^4/6^5$.
Therefore
the probability of getting at least two sixes is
\[
1 - 5^5/6^5 - \binom{5}{1} 5^4/6^5
=
1 - 2 \cdot 5^5/6^5
\]
Here a program to check the above computation:
%-*-latex-*-
\begin{console}[frame=single,fontsize=\footnotesize]
# file: main09b.py
import random; random.seed()
n = 1000000
count = 0
for _ in range(n):
    a = random.randrange(1, 7)
    b = random.randrange(1, 7)
    c = random.randrange(1, 7)
    d = random.randrange(1, 7)
    e = random.randrange(1, 7)
    xs = [a,b,c,d,e]
    if xs.count(6) >= 2: 
        count += 1

print(count / n, 1 - 2 * 5**5/6**5)
\end{console}
The output is
\begin{console}[frame=single,fontsize=\footnotesize]
[student@localhost discrete-probability] python main09b.py
0.19618 0.1962448559670782
\end{console}



