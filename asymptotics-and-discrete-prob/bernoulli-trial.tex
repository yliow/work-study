%-*-latex-*-
\sectionthree{Bernoulli trials and Binomial distribution}
\begin{python0}
from solutions import *; clear()
\end{python0}

At this point, we have the basic concept of the probability attached to
a random experiment.
I have also talked about an experiment that is broken up into two
independent random experiment -- this is when the pdf is a product of
two pdfs.

Now I want to talk about the case where
a random experiment involves
performing a \textit{sequence} of the \textit{same} random experiment.
The sequence need not be made up of two experiments.
Usually there's no limit on the size of the sequence.
In fact, I am usually interested in questions like
\lq\lq
How many times do I need to execute this random experiment
until a goal is reached?''

To be specific, consider the following question:

\lq\lq If there's a 32\% chance of making \$1 when I buy
one google stock on Monday at 10:42AM and sell it on the same day at 2:45PM,
what is the chance of me making \$100 if I buy and sell one google
stock according to the above day and times for 200 consecutive
Mondays when the stock market is open''.
Or:
\lq\lq How many consecutive Mondays of trading do I need to execute before
I made 20 days of gains?''


A 
\defterm{Bernoulli trial}
is a random experiment with two outcomes.
This term is used especially when the Bernoulli trials
are performed in a sequence such that the trials are mutually
independent.

For instance suppose I have a biased coin: 
the probability of getting head is 1/3.
What then is the probability of getting 4 heads when the coin 
is tossed 10 times?

Let me put everything into proper mathematical notation.
Let pdf $p_i : S_i \rightarrow [0,1]$
denote the $i$--th time you are tossing the coin.
Note that 
\[
S_i = \{\HEAD, \TAIL\}
\]
and
\[
p_i(\HEAD) = 1/3, \,\,\,\, p_i(\TAIL) = 2/3
\]
(they are all the same, i.s., $S_i = S_j$ and $p_i = p_j$).
The experiments are mutually independent.
Therefore if $p : S \rightarrow [0,1]$ is the pdf
for the experiment of tossing the coin 10 times 
where $S = S_1 \times \cdots \times S_{10}$,
then
\[
p(x_1, \ldots, x_{10}) = p_1(x_1) \cdots p_{10}(x_{10})
\]
(Note that the probability is a product since
the $i$--th toss is independent of the $j$--th toss for $i \neq j$.)
I am interested in the event
\[
A = \{(x_1, \ldots, x_{10}) \in S \mid 
\text{exactly four of the $x_i$'s are $\HEAD$}\}
\]
Note that 
\[
|A| = \binom{10}{4}
\]
and each element of $A$ has the same probability as
the case where the first four tosses are heads:
\begin{align*}
&p(\HEAD, \HEAD, \HEAD, \HEAD, \TAIL, \ldots, \TAIL) \hspace{1cm} \text{(4 heads, 6 tails)}\\
&= p_1(\HEAD) \cdot p_2(\HEAD) \cdot p_3(\HEAD) \cdot p_4(\TAIL) \cdots p_{10}(\TAIL) \\
&= (1/3)^4 (2/3)^4
\end{align*}
Therefore the required probability is
\[
p(A) = \binom{10}{4} (1/3)^4 (2/3)^6 
\]


Because a Bernoulli trial has two outcomes,
it's common to call one of the outcomes a
\defterm{success}\index{success}\index{success \\ failure}
and the other a
\defterm{failure}\index{failure}.

In general, you see right away that:

\begin{thm}
If the probability of the success of a Bernoulli trial is $p$,
then the probability of having $k$ successes when performing
$n$ of the Bernoulli trials is given by 
\[
\binom{n}{k} p^k (1-p)^{n-k}
\]
\qed
\end{thm}

Frequently you will see the following notations:
\[
B_{n,p}(k) = \binom{n}{k} p^k (1-p)^{n-k}
\]
Other notations include
\[
B(n,p; k)
\text{ \,\,\, and \,\,\, }
B(n,p,k)
\]

Frequently the $n$ and $p$ are fixed and $k$ is considered the
variable of the pdf $B(n,p)$.

Formally, define the pdf of a \defone{Bernoulli trial} as
\[
p_{\textsc{Bernoulli}}: \{\textsc{Success}, \textsc{Failure}\} \rightarrow [0, 1]
\]
Instead of using the outcomes \textsc{Success}
and \textsc{Failure}, it's useful to label them as $1$ and $0$.
In other words, it's useful to have a Bernoulli trial random variable
defined as
\[
X_{\textsc{Bernoulli}} : \{\textsc{Success}, \textsc{Failure}\} \rightarrow \{1, 0\}
\]
where
\[
X_{\textsc{Bernoulli}}(\textsc{Success}) = 1, \,\,\,\,\,
X_{\textsc{Bernoulli}}(\textsc{Failure}) = 0
\]
especially since for the corresponding Binomial distribution I'll
need to count the number of successes.
In other words $X_{\textsc{Bernoulli}}$
is an indicator random variable of \textsc{Success}.
All the above notations are for a single Bernoulli trial.
Now we define the \defone{Binomial distribution}.

Associated with a given Bernoulli trial
$p_{\textsc{Bernoulli}}$, 
we define the $n$--fold product pdf,
the \defone{Binomial distribution} 
\[
p_{B(n,p)}:
\{\textsc{Success}, \textsc{Failure}\}^n \rightarrow [0, 1]
\]
as
\[
p_{B(n,p)}(x_0, x_1, \ldots, x_{n - 1}) =
p_0(x_0) \cdots p_{n-1}(x_{n-1})
\]
where $p_i$ is the pdf of the $i$--th Bernoulli trial
where the \lq\lq $p$" is the probability of success of
the Bernoulli trail.
Of course $p_i = p_{\textsc{Bernoulli}}$, i.e.,
\[
p_{B(n,p)}(x_0, x_1, \ldots, x_{n - 1}) =
p_{\textsc{Bernoulli}}(x_0) \cdots p_{\textsc{Bernoulli}}(x_{n-1})
\]
Let $X_i$ be the indicator random variable for a success
for the $i$--th Bernoulli trial, i.e.,
\[
X_i(x_0, ..., x_{n-1})
=
\begin{cases}
  1 & \text{ if } x_i = \textsc{Success}\\
  0 & \text{ otherwise}\\
\end{cases}
\]
In other words
\[
X_i(x_0, ..., x_{n-1}) = X_{\textsc{Bernoulli}}(x_i)
\]
Next define the random variable
\[
X_{B(n,p)} = \sum_{i=0}^{n-1} X_i
\]
Then
\[
B_{n,p}(k) = \Pr[X_{B(n,p)} = k]
\]
With these notation, the above theorem says
\[
B_{n,p}(k) = \Pr[X_{B(n,p)} = k] = \binom{n}{k} p^k (1 - p)^{n-k}
\]

Here's a plot of $B_{30, 1/10}$:
%-*-latex-*-

\begin{center}
\begin{tikzpicture}[line width=1]
\begin{axis}[width=5in, height=3in,
             scatter/classes={a={mark=*,draw=black}},
             xlabel={\mbox{}},
             xlabel style={name=xlabel}, 
             ylabel={\mbox{}}, 
             legend style={
                at={(xlabel.south)},
                yshift=-1ex,
                anchor=north,
                legend cell align=left,
                },
        ]
]
\addplot[draw=black, line width=1] coordinates {(0, 0.004212720233087431)
(1, 0.025276321398524582)
(2, 0.07330133205572129)
(3, 0.13682915317067967)
(4, 0.1847193567804176)
(5, 0.19210813105163424)
(6, 0.16009010920969519)
(7, 0.10977607488664813)
(8, 0.06312124305982265)
(9, 0.030859274384802193)
(10, 0.01296089524161692)
(11, 0.004713052815133424)
(12, 0.001492466724792251)
(13, 0.00041329847763477715)
(14, 0.00010037248742558872)
(15, 2.141279731745893e-05)
(16, 4.014899497023548e-06)
(17, 6.612775642156433e-07)
(18, 9.551787038670401e-08)
(19, 1.2065415206741558e-08)
(20, 1.327195672741571e-09)
(21, 1.263995878801496e-10)
(22, 1.0341784462921333e-11)
(23, 7.19428484377136e-13)
(24, 4.1966661588666263e-14)
(25, 2.0143997562559807e-15)
(26, 7.747691370215309e-17)
(27, 2.2956122578415726e-18)
(28, 4.919169123946227e-20)
(29, 6.785060860615486e-22)
(30, 4.523373907076989e-24)};
\end{axis}\end{tikzpicture}\end{center}

and here's a plot of $B_{30, 5/6}$::
%-*-latex-*-

\begin{center}
\begin{tikzpicture}[line width=1]
\begin{axis}[width=5in, height=3in,
             scatter/classes={a={mark=*,draw=black}},
             xlabel={\mbox{}},
             xlabel style={name=xlabel}, 
             ylabel={\mbox{}}, 
             legend style={
                at={(xlabel.south)},
                yshift=-1ex,
                anchor=north,
                legend cell align=left,
                },
        ]
]
\addplot[draw=black, line width=1] coordinates {(0, 4.523373907076967e-24)
(1, 6.785060860615451e-22)
(2, 4.919169123946204e-20)
(3, 2.2956122578415626e-18)
(4, 7.747691370215276e-17)
(5, 2.0143997562559725e-15)
(6, 4.196666158866609e-14)
(7, 7.194284843771332e-13)
(8, 1.0341784462921294e-11)
(9, 1.263995878801492e-10)
(10, 1.3271956727415666e-09)
(11, 1.2065415206741518e-08)
(12, 9.551787038670372e-08)
(13, 6.612775642156413e-07)
(14, 4.014899497023538e-06)
(15, 2.1412797317458876e-05)
(16, 0.0001003724874255885)
(17, 0.0004132984776347763)
(18, 0.0014924667247922477)
(19, 0.004713052815133416)
(20, 0.012960895241616897)
(21, 0.030859274384802148)
(22, 0.06312124305982258)
(23, 0.109776074886648)
(24, 0.16009010920969502)
(25, 0.1921081310516341)
(26, 0.18471935678041745)
(27, 0.13682915317067965)
(28, 0.07330133205572126)
(29, 0.02527632139852458)
(30, 0.004212720233087431)};
\end{axis}\end{tikzpicture}\end{center}


\newpage
%-*-latex-*-

\begin{ex} 
  \label{ex:bernoulli-00}
  \tinysidebar{\debug{exercises/{disc-prob-28/question.tex}}}

  \solutionlink{sol:bernoulli-00}
  \qed
\end{ex} 
\begin{python0}
from solutions import *
add(label="ex:bernoulli-00",
    srcfilename='exercises/bernoulli-00/answer.tex') 
\end{python0}

%-*-latex-*-

\begin{ex} 
  \label{ex:bernoulli-01}
  \tinysidebar{\debug{exercises/{disc-prob-28/question.tex}}}

  \solutionlink{sol:bernoulli-01}
  \qed
\end{ex} 
\begin{python0}
from solutions import *
add(label="ex:bernoulli-01",
    srcfilename='exercises/discrete-probability/bernoulli-01/answer.tex') 
\end{python0}

%-*-latex-*-

\begin{ex} 
  \label{ex:bernoulli-02}
  \tinysidebar{\debug{exercises/{disc-prob-28/question.tex}}}

  \solutionlink{sol:bernoulli-02}
  \qed
\end{ex} 
\begin{python0}
from solutions import *
add(label="ex:bernoulli-02",
    srcfilename='exercises/bernoulli-02/answer.tex') 
\end{python0}

%-*-latex-*-

\begin{ex} 
  \label{ex:bernoulli-03}
  \tinysidebar{\debug{exercises/{disc-prob-28/question.tex}}}

  \solutionlink{sol:bernoulli-03}
  \qed
\end{ex} 
\begin{python0}
from solutions import *
add(label="ex:bernoulli-03",
    srcfilename='exercises/bernoulli-03/answer.tex') 
\end{python0}

%-*-latex-*-

\begin{ex} 
  \label{ex:bernoulli-04}
  \tinysidebar{\debug{exercises/{disc-prob-28/question.tex}}}

  \solutionlink{sol:bernoulli-04}
  \qed
\end{ex} 
\begin{python0}
from solutions import *
add(label="ex:bernoulli-04",
    srcfilename='exercises/discrete-probability/bernoulli-04/answer.tex') 
\end{python0}

%-*-latex-*-

\begin{ex} 
  \label{ex:bernoulli-05}
  \tinysidebar{\debug{exercises/{disc-prob-28/question.tex}}}

  \solutionlink{sol:bernoulli-05}
  \qed
\end{ex} 
\begin{python0}
from solutions import *
add(label="ex:bernoulli-05",
    srcfilename='exercises/discrete-probability/bernoulli-05/answer.tex') 
\end{python0}


