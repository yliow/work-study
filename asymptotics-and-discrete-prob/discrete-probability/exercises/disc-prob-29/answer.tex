\tinysidebar{\debug{discrete-probability/exercises/{disc-prob-29/answer.tex}}}

Clearly $p_A(x) = p(\{x\} \cap A) / p(A) \leq 1$ and is $\geq 0$.
since $\{x\} \cap A \subseteq A$.
Also,
\begin{align*}
\sum_{x \in S} p_A(x)
&= \sum_{x \in A} p_A(x) + \sum_{x \in S - A} p_A(x)\\
&= \sum_{x \in A} \frac{p(\{x\} \cap A)}{p(A)}
   + \sum_{x \in S - A} \frac{p(\{x\} \cap A)}{p(A)} \\
&= \sum_{x \in A} \frac{p(\{x\})}{p(A)}
   + \sum_{x \in S - A} \frac{p(\emptyset)}{p(A)} \\
&= \sum_{x \in A} \frac{p(x)}{p(A)} + \sum_{x \in S - A} \frac{0}{p(A)}\\
&= \frac{\sum_{x \in A} p(x)}{p(A)} + 0 \\
&= \frac{p(A)}{p(A)} \\
&= 1
\end{align*}

For instance if $S = \{\ONE, \TWO, ..., \SIX\}$ and $A = \{\ONE, \TWO\}$,
then $p_A$ is given by
\[
p_A(\ONE) = p_A(\TWO) = \frac{1/6}{2/6} = \frac{1}{2}
\]
and $p_A(\THREE) = p_A(\FOUR) = p_A(\FIVE) = p_A(\SIX) = 0$.
    
