\tinysidebar{\debug{exercises/{nonarray-implementation-2/question.tex}}}
Write a class \texttt{MaxHeap} that works like this:
\begin{console}
int x;
MaxHeap< int > heap;
x = heap.size();       // x = 0 
heap.insert(5);        // [5]
heap.insert(7);        // [7, 5]
heap.insert(9);        // [9, 5, 7]
x = heap.size();       // x = 3
std::cout << heap;     // Prints "[9, 5, 7]".

int a = heap.delete(); // [7, 5], deleting at index 0
                       // a = 9
x = heap.max();        // x = 9. The root is not deleted.

heap[0] = 1;           // [1, 5], heap is also like a
                       // std::vector.
heap.heapify_down(0);  // [5, 1]

heap[1] := 10;         // [5, 10]
heap.heapify_up(1);    // [10, 5]

heap.resize(5);
heap[0] = 5;
heap[1] = 7;
heap[2] = 8;
heap[3] = 10;
heap[4] = 2;
std::cout << heap;     // Prints "[5, 7, 8, 10, 2]".
heap.build();          // build-max-heap with 5 values
                       // already in the heap.
std::cout << heap;     // Prints "[10, 7, 8, 5, 2]".

heap.clear();          // []
\end{console}
