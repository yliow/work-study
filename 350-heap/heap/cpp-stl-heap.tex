%-*-latex-*-
\section{C++ STL heap functions}

C++ provides heap functions
which operate on sequential container such as \verb!std::vector!
or arrays.
By default the functions assume the heap is a maxheap.

To use it do this at the top of your cpp file:
\begin{Verbatim}[frame=single, fontsize=\small]
#include <algorithm>
\end{Verbatim}

Let \verb!v! be a \verb!std::vector< T >! object (for some type \verb!T!).
\begin{enumerate}[nosep]
  \li \verb!std::make_heap(v.begin(), v.end())!: make \verb!v! into a maxheap
  \li \verb!std::is_heap(v.begin(), v.end())!: returns \verb!true! if \verb!v! is maxheap
  \li \verb!std::is_heap_until(v.begin(), v.end())!:
  return iterator pointing to first value that violates maxheap property.
  If \verb!v! is maxheap, !v.end()! is returned.
  \li \verb!std::push_heap(v.begin(), v.end())!: Performs heapify-up on
  the last entry of \verb!v!.
  So to perform a maxheap insert with \verb!x!, you would first
  do \verb!v.push_back(x)! and then execute
  \verb!std::push_heap(v.begin(), v.end())!.
  \li \verb!std::pop_heap(v.begin(), v.end())!: Performs extract-root,
  but instead of removing the root,
  the root is placed at the last value entry of \verb!v!.
  So to perform a maxheap delete, you would first
  do \verb!std::pop_heap(v.begin(), v.end())!
  and then \verb!v.pop_back(x)!.
  \li \verb!std::sort_heap(v.begin(), v.end()!: heapsort assuming the
  vector is already a maxheap.
\end{enumerate}
All the functions above can also accept an extra comparator functor.
\begin{python}
s = r'''
#include <iostream>
#include <algorithm>
#include <vector>
#include <string>

std::ostream & operator<<(std::ostream & cout, 
                          const std::vector< int > & v)
{
    std::string sep = "";
    cout << '{';
    for (auto & x: v)
    {
        cout << sep << x;
        sep = ", ";
    }
    cout << '}';
    return cout;
}

int main()
{
    std::vector< int > v {5,3,0,1,2,6,7,4};
    std::cout << v << '\n';

    // Default is maxheap
    std::make_heap(v.begin(), v.end());
    std::cout << "maxheap: " << v << '\n';

    // Use std::greater to get a minheap
    std::make_heap(v.begin(), v.end(), std::greater< int >());
    std::cout << "minheap: " << v << '\n';

    // Make maxheap
    std::make_heap(v.begin(), v.end());
    std::cout << "maxheap: " << v << '\n';
    
    // Insert: push_back and push_heap
    std::cout << "insert 99\n";
    v.push_back(99);
    std::cout << "maxheap: " << v << '\n';
    std::push_heap(v.begin(), v.end());
    std::cout << "maxheap: " << v << '\n';

    // Extract root: pop_heap and pop_back
    std::cout << "extract-root\n";
    std::pop_heap(v.begin(), v.end());
    std::cout << "maxheap: " << v << '\n';
    v.pop_back();
    std::cout << "maxheap: " << v << '\n';

    // Heapsort
    std::sort_heap(v.begin(), v.end());
    std::cout << "heapsorted: " << v << '\n';
    
    return 0;
}
'''.strip()
print(r'''
\begin{Verbatim}[frame=single, fontsize=\small]
%s
\end{Verbatim}
''' % s)
from latextool_basic import *
f = open("heapsort.cpp", "w")
f.write(s)
f.close()
print(r'{\footnotesize %s}' % shell('g++ heapsort.cpp; ./a.out'))
\end{python}
Of course you can also use an array:
\begin{python}
s = r'''
#include <iostream>
#include <algorithm>
#include <vector>
#include <string>

void println(int v[], int size)
{
    std::string delim = "";
    std::cout << '{';
    for (int i = 0; i < size; ++i)
    {
        std::cout << delim << v[i];
        delim = ", ";
    }
    std::cout << "}\n";
}

int main()
{
    int x[] = {5,3,0,1,2,6,7,4};
    println(x, 8);
    std::make_heap(x, x + 8);
    println(x, 8);
    return 0;
}
'''.strip()

print(r'''
\begin{Verbatim}[frame=single, fontsize=\small]
%s
\end{Verbatim}
''' % s)

from latextool_basic import *
f = open("heapsort2.cpp", "w")
f.write(s)
f.close()
print(shell(['g++ heapsort2.cpp -std=c++11', './a.out']))
#print(shell('./a.out'))
\end{python}

\begin{ex}
  Create a \verb!priority_queue! class.
  Here are some methods:
  \begin{console}
priority_queue< int > pq; // maxheap by default
std::cout << pq << '\n';
std::cout << pq.size() << '\n';
std::cout << pq.empty() << '\n';
pq.insert(5);
pq.insert(3);
pq.insert(1);
pq.insert(2);
pq.insert(4);
std::cout << pq << '\n';
std::cout << pq.root() << '\n';
pq.delete();
std::cout << pq << '\n';
ps.heapsort();  // heapsort in ascending order
ps.buildheap(); // make into maxheap

priority_queue< int, std::greater< int > > pq1; // minheap
  \end{console}
\end{ex}
