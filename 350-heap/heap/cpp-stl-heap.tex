%-*-latex-*-
\sectionthree{C++ STL heap functions}
\begin{python0}
from solutions import *; clear()
\end{python0}

C++ provides heap functions
which operate on sequential container such as \verb!std::vector!
or arrays.
By default the functions assume the heap is a maxheap.

To use it do this at the top of your cpp file:
\begin{Verbatim}[frame=single, fontsize=\small]
#include <algorithm>
\end{Verbatim}

Let \verb!v! be a \verb!std::vector< T >! object (for some type \verb!T!).
\begin{enumerate}[nosep]
  \li \verb!std::make_heap(v.begin(), v.end())!: make \verb!v! into a maxheap
  \li \verb!std::is_heap(v.begin(), v.end())!: returns \verb!true! if \verb!v! is maxheap
  \li \verb!std::is_heap_until(v.begin(), v.end())!:
  return iterator pointing to first value that violates maxheap property.
  If \verb!v! is maxheap, !v.end()! is returned.
  \li \verb!std::push_heap(v.begin(), v.end())!: Performs heapify-up on
  the last entry of \verb!v!.
  So to perform a maxheap insert with \verb!x!, you would first
  do \verb!v.push_back(x)! and then execute
  \verb!std::push_heap(v.begin(), v.end())!.
  \li \verb!std::pop_heap(v.begin(), v.end())!: Performs extract-root,
  but instead of removing the root,
  the root is placed at the last value entry of \verb!v!.
  So to perform a maxheap delete, you would first
  do \verb!std::pop_heap(v.begin(), v.end())!
  and then \verb!v.pop_back(x)!.
  \li \verb!std::sort_heap(v.begin(), v.end()!: heapsort assuming the
  vector is already a maxheap.
\end{enumerate}
All the functions above can also accept an extra comparator functor.
\begin{center}
\begin{tikzpicture}

\fill[white] (0.0, 0.0) circle (0.3);
\node [line width=0.03cm,black,minimum size=0.57cm,draw,circle] at (0.0,0.0)(10){};\draw (0.0, 0.0) node[color=black] {\texttt{10}};
\fill[white] (1.4, 0.0) circle (0.3);
\node [draw=none,line width=0cm,black,minimum size=0.6cm,circle] at (1.4,0.0){};\draw (1.4, 0.0) node[color=black] {$b \rightarrow b + 1$};
\fill[white] (0.75, -1.0) circle (0.3);
\node [line width=0.03cm,black,minimum size=0.57cm,draw,circle] at (0.75,-1.0)(5){};\draw (0.75, -1.0) node[color=black] {\texttt{5}};
\fill[white] (1.35, -1.0) circle (0.3);
\node [draw=none,line width=0cm,black,minimum size=0.6cm,circle] at (1.35,-1.0){};\draw (1.35, -1.0) node[color=black] {0};\draw[line width=0.03cm,black,->,>=triangle 60] (10) to  (5);
\end{tikzpicture}

\end{center}


Of course you can also use an array:

\begin{Verbatim}[frame=single, fontsize=\small]
#include <iostream>
#include <algorithm>
#include <vector>
#include <string>

void println(int v[], int size)
{
    std::string delim = "";
    std::cout << '{';
    for (int i = 0; i < size; ++i)
    {
        std::cout << delim << v[i];
        delim = ", ";
    }
    std::cout << "}\n";
}

int main()
{
    int x[] = {5,3,0,1,2,6,7,4};
    println(x, 8);
    std::make_heap(x, x + 8);
    println(x, 8);
    return 0;
}
\end{Verbatim}

\begin{console}[frame=single,fontsize=\small]
[student@localhost heap] g++ heapsort2.cpp -std=c++11
[student@localhost heap] ./a.out
{5, 3, 0, 1, 2, 6, 7, 4}
{7, 4, 6, 3, 2, 5, 0, 1}
\end{console}



%-*-latex-*-

\begin{ex} 
  \label{ex:prob-00}
  \tinysidebar{\debug{exercises/{disc-prob-28/question.tex}}}

  \solutionlink{sol:prob-00}
  \qed
\end{ex} 
\begin{python0}
from solutions import *
add(label="ex:prob-00",
    srcfilename='exercises/discrete-probability/prob-00/answer.tex') 
\end{python0}

