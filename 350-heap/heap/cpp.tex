%-*-latex-*-
\section{Implementing min- and maxheaps at the same time}

(You want to review \verb!std::vector!, iterators,
function pointers and functors.)

With function pointers or functors, you can write a heapify-up (and heapify-down)
function that works for both maxheap and minheap
simply by passing in a comparison function or functor.

\begin{Verbatim}[frame=single, fontsize=\small]
#include <iostream>

bool less(int x, int y)
{
    return (x < y);
}
  
bool greater(int x, int y) 
{
    return (x > y);
}

void heapify_up(int x[], int n, int i, bool (*cmp)(int, int))
{
    int v = x[i];
    while (1)
    {
        int p = (i - 1) / 2;
        if (cmp(x[i], x[p]))
        {
            x[p] = x[i];
            i = p;
        }
        else
        {
            x[i] = v;
            break;
        }
    }
}

int main()
{
    f(less);
    return 0;
}
\end{Verbatim}
The next thing is to make heapify up into a template:
\begin{Verbatim}[frame=single, fontsize=\small]
template < typename T >
bool less(const T & x, const T & y)
{
    return (x < y);
}
  
template < typename T >
bool greater(const T & x, const T & y)
{
    return (x > y);
}

template < typename T >
void heapify_up(T x[], int n, int i, bool (*cmp)(T, T))
{
    T v = x[i];
    ...
}
\end{Verbatim}

Another way to achieve this is through functors.
\begin{Verbatim}[frame=single, fontsize=\small]
template < typename T >
struct less
{
    bool operator()(const T & x, const T & y)
    {
        return (x < y);
    }
};

template < typename T >
struct greater
{
    bool operator()(const T & x, const T & y)
    {
        return (x > y);
    }
};

template < typename T, typename Comparator >
void heapify_up(T x[], int n, int i)
{
    T v = x[i];
    ...
}
\end{Verbatim}





