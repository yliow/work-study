%-*-latex-*-
\sectionthree{Heapify-down}
\begin{python0}
from solutions import *; clear()
\end{python0}

Heapify-down is the opposite of heapify-up:
you keep pushing a value \verb!v! down, swapping with the largest child
if that largest child is greater than \verb!v!. 
Here's an example.

Suppose I have this tree:

\begin{python}
from latextool_basic import *
from drawheap import *
p = Plot()
edges={'10':['0','8'],
       '0':['5','2'],
       '8':['4','6'],
       '5':['1','3']
       }
drawheap(p, edges, include_array=False)
print(p)
\end{python}

This is almost a maxheap except that \verb!0!
violates the maxheap property.
I perform heapify-down at \verb!0!, swapping with
the larger of the children, i.e., \verb!5!


\begin{python}
from latextool_basic import *
from drawheap import *
p = Plot()
edges={'10':['0','8'],
       '0':['5','2'],
       '8':['4','6'],
       '5':['1','3']
       }
drawheap(p, edges, include_array=False)
p += Line(names=['0', '5'], linewidth=0.03, linecolor='red',
          bend_right=60, endstyle='>', startstyle='>')       
print(p)
\end{python}

to get


\begin{python}
from latextool_basic import *
from drawheap import *
p = Plot()
edges={'10':['5','8'],
       '5':['0','2'],
       '8':['4','6'],
       '0':['1','3']
       }
drawheap(p, edges, include_array=False)
print(p)
\end{python}

I do it again, swapping \verb!0! with \verb!3! to get

\begin{python}
from latextool_basic import *
from drawheap import *
p = Plot()
edges={'10':['5','8'],
       '5':['3','2'],
       '8':['4','6'],
       '3':['1','0']
       }
drawheap(p, edges, include_array=False)
print(p)
\end{python}

In this case, I stop at a leaf.
In general, it's possible that the heapify-down stops before
arriving at a leaf.

Note that I can heapify-down at any value even if
the tree has multiple spots that violate the
maxheap property.

\begin{Verbatim}[frame=single]
ALGORITHM: heapify-down (for maxheap)
INPUTS: x[0..n-1] -- array
        i         -- index of x

while i < n and x[i] has a child larger than x[i]:
    swap x[i] with the largest child, say x[j]
    set i to j
x[i] = val
\end{Verbatim}

In more details:

\begin{Verbatim}[frame=single]
ALGORITHM: heapify-down (for maxheap)
INPUTS: x[0..n-1] -- array
        i         -- index of x

v = x[i]
while 1:

    # Set j to index of the larger of the left and right
    # child of v. If there is no left and right child,
    # j is set to -1.
    l = left(i)
    r = right(i)
    if l == -1:
        # left child does no exist
        # (therefore right child does not exist)
        j = -1
    else:
        # left child exists
        if r == -1:
            # right child does not exists
            j = l
        else:
            # right child exists
            if x[l] > x[r]:
                j = l
            else:
                j = r    

    if j != -1 and x[j] > v:
        # v heapify-down                
        x[i] = x[j]
        i = j
    else:
        # v arrives at final index
        x[i] = v
        break
\end{Verbatim}

[REDUNDANT PARAGRAPH]
You can heapify-up or heapify-down at any index position.
There are times when heapifying-up or heapifying-down depends on the
value at the given index.
Let's call the function \verb!heapify!.
In such cases, if it's possible to heapify-up, then \verb!heapify! will
heapify-up and if it's possible to heapify-down, then \verb!heapify!
will heapify-down.

[What if there's a value that can heapify up and can heapify down?]


\begin{ex}
  In terms if actual real time, which cost more, one step of
  heapify-up or one step of heapify-down?
  \qed
\end{ex}
