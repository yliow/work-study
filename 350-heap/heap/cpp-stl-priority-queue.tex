%-*-latex-*-
\sectionthree{C++ STL \texttt{std::priority\_queue}}
\begin{python0}
from solutions import *; clear()
\end{python0}

C++ provides an STL container \defone{\texttt{std::priority\_queue}}
for ...
priority queues (duh).

Let \verb!priqueue! be a \verb!priority_queue< T >! object
for some type \verb!T!.
Here are some of the methods:
\begin{enumerate}[nosep]
  \li \texttt{priqueue.size()}: number of values in the \verb!priqueue! 
  \li \texttt{priqueue.empty()}: true iff \verb!priqueue! is empty
  \li \texttt{priqueue.top()}: the root of \verb!priqueue!
  \li \texttt{priqueue.pop()}: extract root from \verb!priqueue!
\end{enumerate}

\begin{python}
s = r'''
#include <iostream>
#include <queue> // for std::priority_queue
#include <vector>
#include <string>

std::ostream & operator<<(std::ostream & cout, 
                          const std::vector< int > & v)
{
    std::string sep = "";
    cout << '{';
    for (auto & x: v)
    {
        cout << sep << x;
        sep = ", ";
    }
    cout << '}';
    return cout;
}

int main()
{
    std::priority_queue< int > pq {5,3,0,1,2,6,7,4};
    std::cout << pq << '\n';

    // Default is maxheap
    std::cout << "pq: " << pq << '\n';
    std::cout << "size: " << pq.size() << '\n';
    pq.push(99);
    std::cout << "after push(99): " << pq << '\n';
    std::cout << "top: " << pq.top() << '\n';
    pq.pop();
    std::cout << "after pop: " << pq << '\n';
    
    return 0;
}
'''.strip()
print(r'''
\begin{Verbatim}[frame=single, fontsize=\small]
%s
\end{Verbatim}
''' % s)
from latextool_basic import *
f = open("priqueue.cpp", "w")
f.write(s)
f.close()
print(r'{\footnotesize %s}' % shell('g++ priqueue.cpp; ./a.out'))
\end{python}
