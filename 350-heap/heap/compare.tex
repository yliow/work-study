%-*-latex-*-
\section{Compare heap against queue and BST/AVL}

As mentioned earlier, the point of the heap is to
organized data so that you want remove an item
with the highest priority (use maxheap or minheap depending
on whether \lq\lq higher prioriy" mean \lq\lq larger priority number"
or \lq\lq smaller priority number".

Here is the runtimes of the heap (min or max):
\begin{enumerate}[nosep]
\item Insert: $O(\lg n)$
\item Delete: $O(\lg n)$ (i.e., delete the root)
\end{enumerate}
Of course if you just want to peek at the highest priority value without deleting it,
it's $O(1)$.

If you use a queue (using double linked list) to implement a 
priority queue,
\begin{enumerate}[nosep]
\item Insert: $O(n)$
\item Delete: $O(1)$ (i.e., delete the root)
\end{enumerate}
You might think this is OK since the delete is fast.
But hang on cowboy ... if you perform $n$ inserts and deletes 
say alternativing them, the average is going to be $O(n)$ for each.
The average for the heap is going to be $O(\lg n)$.
So using a queue is a bad idea.

What about a BST? Say we use the AVL tree.
\begin{enumerate}[nosep]
\item Insert: $O(\lg n)$
\item Delete: $O(\lg n)$ (i.e., delete the minimum)
\end{enumerate}
If you want to peek at the minimum without removing it,
the runtime is $O(\lg n)$.
Excluding the peek, the insert and delete are the 
same as the heap.
But wait ...

When you insert a value into an AVL, what is the runtime?
The value is inserted as a leaf. The runtime is not just worst
runtime of $O(\lg n)$. It is in fact exactly $\Theta(\lg n)$.

But what about a heap? Say a minheap?
While in the case of the AVL tree, you have to walk from the root
to a leaf, in the case of the insert into a minheap,
you insert at the leaf and you heapify-up -- but the
heapify-up might stop \textit{anywhere} along the way to the root.

Assuming the tree is full.
How many leaves are there? If the height is $h$,
then there are $2^{h + 1}$ leaves.
And how many nonleaves are there?
Well it's $1 + 2 + \cdots + 2^{h} = 2^{h+1} - 1$.
So there's a 50\% chance that the inserted value is a leaf!
Which in the case of a heap, it means that there's not
a whole lot of heapify up at all!!!
In fact you can show (but you would need probability theory)
that the \textit{average} runtime of heap insert is $O(1)$.

Wonderful! 
