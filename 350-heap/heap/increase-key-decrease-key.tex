%-*-latex-*-
\section{Increase-key and decrease-key}

So suppose I already have a priority queue (as a heap):

\begin{python}
from latextool_basic import *
p = Plot()
edges={'10': ['9','8'],
       '9': ['5', '3'],
       '8': ['0', '7'],
       '5': ['2', '1'],
     }
drawheap(p, edges, include_array=False)
print(p)
\end{python}


Suppose \texttt{1} is increased to \texttt{12}:

\begin{python}
from latextool_basic import *
p = Plot()
edges={'10': ['9','8'],
       '9': ['5', '3'],
       '8': ['0', '7'],
       '5': ['2', '12'],
     }
drawheap(p, edges, include_array=False)
print(p)
\end{python}


I'm sure you can see very quickly that in order to make this back to
a maxheap, I need to heapify-up \texttt{12}.
In this case I need to swap 3 times to get this:

\begin{python}
from latextool_basic import *
p = Plot()
edges={'12': ['10','8'],
       '10': ['9', '3'],
       '8': ['0', '7'],
       '9': ['2', '5'],
     }
drawheap(p, edges, include_array=False)
print(p)
\end{python}

\begin{console}
ALGORITHM: increase_key (for max heap)
INPUT: x - heap (using an array)
       n - length of heap in x
       i - index where key will increase
       k - new key value (k > x[i])

x[i] = k # originally x[i] < k
Perform heapify-up on x starting at index i.
\end{console}
So if the above represent processes with priorities, the low priority
process moves up, in fact to the top, so it will be the next process to
be executed.
The worse runtime of increase-key for max heap is
\[
O(\log n)
\]

Using the above tree

\begin{python}
from latextool_basic import *
from drawheap import *  
p = Plot()
edges={'12': ['10','8'],
       '10': ['9', '3'],
       '8': ['0', '7'],
       '9': ['2', '5'],
     }
drawheap(p, edges, include_array=False)
print(p)
\end{python}

suppose \texttt{10} is decreased to \texttt{1}:

\begin{python}
from latextool_basic import *
p = Plot()
edges={'12': ['1','8'],
       '1': ['9', '3'],
       '8': ['0', '7'],
       '9': ['2', '5'],
     }
drawheap(p, edges, include_array=False)
print(p)
\end{python}

Of course this is not a maxheap anymore.
To make this back to a maxheap, clearly the simplest thing to do is
to heapify-down.
In this case I need two swaps.
First I do this swap

\begin{python}
from latextool_basic import *
p = Plot()
edges={'12': ['1','8'],
       '1': ['9', '3'],
       '8': ['0', '7'],
       '9': ['2', '5'],
     }
drawheap(p, edges, include_array=False)
p += Line(names=['1', '9'], startstyle='>', endstyle='>', linecolor='red', linewidth=0.03, bend_right=60)     
print(p)
\end{python}

to get this:

\begin{python}
from latextool_basic import *
p = Plot()
edges={'12': ['9','8'],
       '9': ['1', '3'],
       '8': ['0', '7'],
       '1': ['2', '5'],
     }
drawheap(p, edges, include_array=False)
print(p)
\end{python}

Next I do this swap 

\begin{python}
from latextool_basic import *
p = Plot()
edges={'12': ['9','8'],
       '9': ['1', '3'],
       '8': ['0', '7'],
       '1': ['2', '5'],
     }
drawheap(p, edges, include_array=False)
p += Line(names=['1', '5'], endstyle='>', startstyle='>', linewidth=0.03, linecolor='red', bend_left=90)     
print(p)
\end{python}

to get this:

\begin{python}
from latextool_basic import *
p = Plot()
edges={'12': ['9','8'],
       '9': ['5', '3'],
       '8': ['0', '7'],
       '5': ['2', '1'],
     }
drawheap(p, edges, include_array=False)
print(p)
\end{python}

Here's decrease key:

\begin{Verbatim}[frame=single]
ALGORITHM: decrease_key (for maxheap)
INPUT: x - heap (using an array)
       n - length of heap in x
       i - index where key will increase
       k - new key value (k < x[i])

x[i] = k # originally x[i] > k
Perform heapify_down on x at starting at index i.
\end{Verbatim}

The worse runtime of decrease-key for maxheap is clearly
\[
O(\log n)
\]

It's clear that given a heap (max or min), 
you can heapify-up and heapify-down.
I prefer to use 
\lq\lq heapify-up"
and 
\lq\lq heapify-down"
because the important thing is whether a node moves up or down.
In the case of maxheap, 
heapify-up and heapify-down
are respectively
increase-key and decrease-key
whereas
in the case of minheap
heapify-up and heapify-down
are respectively
decrease-key and increase-key.




\newpage
\begin{ex}
Let $x$ be the array
\[
4, 9, 2, 6, 3, 8, 0, 1, 5, 7
\]
First heapify it.
Next do the following:
\begin{tightlist}
  \item Perform increase-key from on the key with value \texttt{2}
  and change it to \texttt{11}.
  \item Perform decrease-key from on the key with value
  \texttt{7} and change it to \texttt{-1}.
\end{tightlist}
\qed
\end{ex}



\newpage
\begin{ex}
Continuing the implementation of the heap ADT using functions,
implement the increase-key and decrease-key
functions:
\begin{console}
int x;
std::vector< int > heap;

heap.resize(5);
heap[0] = 5;
heap[1] = 7;
heap[2] = 8;
heap[3] = 10;
heap[4] = 2;
maxheap_build(heap);      // [10, 7, 8, 5, 2]

maxheap_increasekey(heap, 2, 12); // heap[2] is changed
                          // to 12. heap has to be
                          // reorganized to become
                          // maxheap again.
                          
maxheap_decreasekey(heap, 2, 0); // heap[2] is changed
                          // to 0. heap has to be
                          // reorganized to become
                          // maxheap again.
\end{console}
\end{ex}
