%-*-latex-*-
\section{Priority queues}

Recall a queue is a data structure that supports
enqueue and dequeue (entering and leaving a queue).
Think of people lining up at a ticket booth.

A \defone{priority queue} is queue where people
join a queue \textit{but}
they have priority numbers.
So someone with a higher priority can jump the queue
by going ahead of someone in front of him/her
who has a lower priority.

Priority queues are very important.
For instance modern computers
run multiple processes at the \lq\lq same time", i.e,
the OS maintains a queue of processes and
allow each process to execute a small time slice
before that process pauses and goes back to the
end of the queue.
But this is not just a regular queue:
the OS process queue is a priority queue.
Why?
Because some processes are more important than others.
Try the following ...

Linux allows user to display running processes using the
\verb!ps! command:
priority of processes.
The following is a list of processes that I'm running:
{\scriptsize
\begin{console}
[student@localhost n]$ ps -l
F S   UID   PID  PPID  C PRI  NI ADDR SZ WCHAN  TTY          TIME CMD
0 S  1000  3148 26588  0  80   0 -  8250 poll_s pts/0    00:07:46 emacs
0 R  1000  7239 26588  0  80   0 -  1253 -      pts/0    00:00:00 ps
0 S  1000 26588 20828  0  80   0 -  1574 wait   pts/0    00:00:01 bash
0 S  1000 28473     1  1  80   0 - 75939 poll_s pts/0    00:00:34 atril
\end{console}
}
The priority is under the column labeled \texttt{NI}.
Notice that \texttt{emacs} is running with priority \texttt{0}.
I can change the priority of my \texttt{emacs} to this:
{\scriptsize
\begin{console}
F S   UID   PID  PPID  C PRI  NI ADDR SZ WCHAN  TTY          TIME CMD
0 S  1000  3148 26588  0  81   1 -  8250 poll_s pts/0    00:07:47 emacs
0 R  1000  8093 26588  0  80   0 -  1253 -      pts/0    00:00:00 ps
0 S  1000 26588 20828  0  80   0 -  1574 wait   pts/0    00:00:01 bash
0 S  1000 28473     1  1  80   0 - 76240 poll_s pts/0    00:00:35 atril
\end{console}
}
In Linux 
lower priority number
means
a higher priority.

Besides the prioritized use of resources 
(in OS or networks or web servers etc.)
priority queues are also very 
important in algorithms.
You have already seen examples where
containers are used to hold \lq\lq work to do".
For instance in breadth-first traversals, queues are used to 
hold nodes to be processed. 
In depth-first traversals, stacks are used instead.

Many algorithms use priority queues to hold 
\lq\lq work to do".
For instance when you visit a graph or a tree,
you might see a node that might provide
a shorter processing time to reach a solution
than the nodes (in a container) 
that are waiting to be processed.
In that case, the new node you just discovered
after being placed in the container (of nodes to 
processed)
need to jump over some nodes.
For instance suppose you are in an unknown maze
and your goal is to locate a pot of
curry. As you walk, you draw the maze as much as
you can because you don't want to walk in a
cycle forever and you want to know how to
backtrack if you hit a deadend.
You might want to be systematic and explore all
possible passage ways.
\textit{But}
if you can smell curry coming from a specific
direction, you will probably explore
that promising passageway right away, putting
other passageways aside for the time being.
In a nutshell, that's the main idea 
behind Dijstra's shortest path algorithm, 
probably the most famous algorithm that uses
a priority queue.
(Make sure you google for how to pronounce
Dijkstra!)
But there are many others.

Obviously, you can implement a priority
queue, say implemented using a doubly linked list.
After a node enters a queue at the tail, you 
compare that node with the node in front of it
and swap them if necessary (based on their
priorities).
The runtimes are then as follows:
\begin{tightlist}
  \item[1.] insert: $O(n)$
  \item[2.] delete: $O(1)$
\end{tightlist}
Can we do better?
Yes we can. But we'll need something different.
These are called heaps.
And there are many different types of heaps.

There are some other operations that might be helpful
for a priority queue.
Although a priority queue is a self-organizing container,
there are times when you want to access a particular
item in the queue and make some modifications. 
\begin{tightlist}
  \item[3.] find a node in a priority queue and returning a pointer or index or location
\end{tightlist}
Using the pointer or index or location of a node in the priority queue, you can make the
following modifications:
\begin{tightlist}
  \item[4.] delete an item in the priority queue 
  \item[5.] modifying the priority of an item in the priority queue 
\end{tightlist}
Another operations is 
\begin{tightlist}
  \item[6.] merging two priority queues into one
\end{tightlist}
Of course there are other basic operations such as 
checking if the priority queue is empty, getting the size,
clearing it, etc.
For debugging it would be helpful if you can print it.
Like any self-organizing container, 
sometimes it's helpful to look at the next item that
will be removed (i.e., peek), without actually removing it.


