%-*-latex-*-
[Spiral]
ASCII art.
(Do not use arrays.)

The test cases explains what you need to do.
Make sure the drawing is done by a function called with the
following prototype:
\[
\texttt{void draw\_spiral(int);}
\]
In other words, the skeleton code is
\begin{Verbatim}[frame=single]
#include <iostream>

void draw_spiral(int n)
{
}

int main()
{
    int n;
    std::cin >> n;
    draw_spiral(n);

    return 0;
}
\end{Verbatim}

\textsc{Test 1}
\begin{console}[commandchars=\\\{\}]
\userinput{1}
+--------+
|        |
|  +--+  |
|  |     |
|  +-----+
\end{console}

\textsc{Test 2}
\begin{console}[commandchars=\\\{\}]
\userinput{2}
+--------------+
|              |
|  +--------+  |
|  |        |  |
|  |  +--+  |  |
|  |  |     |  |
|  |  +-----+  |
|  |           |
|  +-----------+
\end{console}

\textsc{Test 3}
\begin{console}[commandchars=\\\{\}]
\userinput{3}
+--------------------+
|                    |
|  +--------------+  |
|  |              |  |
|  |  +--------+  |  |
|  |  |        |  |  |
|  |  |  +--+  |  |  |
|  |  |  |     |  |  |
|  |  |  +-----+  |  |
|  |  |           |  |
|  |  +-----------+  |
|  |                 |
|  +-----------------+
\end{console}

\textsc{Test 4}
\begin{console}[commandchars=\\\{\}]
\userinput{4}
+--------------------------+
|                          |
|  +--------------------+  |
|  |                    |  |
|  |  +--------------+  |  |
|  |  |              |  |  |
|  |  |  +--------+  |  |  |
|  |  |  |        |  |  |  |
|  |  |  |  +--+  |  |  |  |
|  |  |  |  |     |  |  |  |
|  |  |  |  +-----+  |  |  |
|  |  |  |           |  |  |
|  |  |  +-----------+  |  |
|  |  |                 |  |
|  |  +-----------------+  |
|  |                       |
|  +-----------------------+
\end{console}

WARNING: ... INCOMING SPOILERS ...
Hints on next page.
Use only if necessary.

\newpage
\textsc{Hints}

Look at the case when the input is \verb!4!:
{\small
\begin{console}[commandchars=\\\{\}]
\userinput{4}
+--------------------------+
|                          |
|  +--------------------+  |
|  |                    |  |
|  |  +--------------+  |  |
|  |  |              |  |  |
|  |  |  +--------+  |  |  |
|  |  |  |        |  |  |  |
|  |  |  |  +--+  |  |  |  |
|  |  |  |  |     |  |  |  |
|  |  |  |  +-----+  |  |  |
|  |  |  |           |  |  |
|  |  |  +-----------+  |  |
|  |  |                 |  |
|  |  +-----------------+  |
|  |                       |
|  +-----------------------+
\end{console}
}
Notice that the first line of output is \lq\lq more similar''
to the third and fifth than the second and fourth.
So if you look at the 1-st, 3-rd, 5-th, ... lines of output
the picture looks like this:
{\small
\begin{console}[commandchars=\\\{\}]
\userinput{4}
+--------------------------+
|  +--------------------+  |
|  |  +--------------+  |  |
|  |  |  +--------+  |  |  |
|  |  |  |  +--+  |  |  |  |
|  |  |  |  +-----+  |  |  |
|  |  |  +-----------+  |  |
|  |  +-----------------+  |
|  +-----------------------+
\end{console}
}
In other words the pseudocode should look like this:
{\small
\begin{console}
for i = 1, 2, 3, 4, ..., 17:
    if i is odd:
        draw it in a certain way
    else:
        draw it in another way
\end{console}
}
This is similar to the problem of drawing alternativing characters,
something like
{\small
\begin{console}[commandchars=\\\{\}]
\underline{5}
*
@
*
@
*
\end{console}
}

Going back to our problem,
I suggest you focus on the odd case first, i.e.,:
{\small
\begin{console}
for i = 1, 2, 3, 4, ..., 17:
    if i is odd:
        draw it in a certain way
\end{console}
}
Of course later you have to make the program work for any input $n$.
When the input is \verb!4!, the number of lines printed is \verb!17!.
You have to experiment to see what happens when the input is 1, 2, 3, 5.
You should be able to figure out a formula for the number of lines
printed in terms of the user input $n$ (look at the \verb!??! below):
{\small
\begin{console}
for i = 1, 2, 3, 4, ..., ??:
    if i is odd:
        draw it in a certain way
\end{console}
}
If you look at Test 1, Test 2, Test 3, Test 4, you'll see that
\begin{tightlist}
  \li when $n = 1$, the number of lines printed is $5$.
  \li when $n = 2$, the number of lines printed is $9$.
  \li when $n = 3$, the number of lines printed is $13$.
  \li when $n = 4$, the number of lines printed is $17$.
\end{tightlist}
So what is \verb!??! in terms of $n$?

The output lines of the top half are similar
but slightly different from the bottom half.
The top half looks like this:
{\small
\begin{console}[commandchars=\\\{\}]
\userinput{4}
+--------------------------+
|  +--------------------+  |
|  |  +--------------+  |  |
|  |  |  +--------+  |  |  |
|  |  |  |  +--+  |  |  |  |
\end{console}
}
and the bottom half looks like this:
{\small
\begin{console}[commandchars=\\\{\}]
\userinput{4}
|  |  |  |  +-----+  |  |  |
|  |  |  +-----------+  |  |
|  |  +-----------------+  |
|  +-----------------------+
\end{console}
}
So the pseudocode should look like this:
{\small
\begin{console}
for i = 1, 2, 3, 4, ..., ??:
    if i is odd:
        if i < 10: 
            draw it in a certain way (for top half)
        else:
            draw it in a certain way (for bottom half)
\end{console}
}
I suggest you focus on the top half first, i.e.,
{\small
\begin{console}
for i = 1, 2, 3, 4, ..., ??:
    if i is odd:
        if i < 10: 
            draw it in a certain way (for top half)
\end{console}
}
Here's the output again for the top half when \verb!i! is odd --
I've included the value of \verb!i! on the left:
{\small
\begin{console}
i
1  +--------------------------+
3  |  +--------------------+  |
5  |  |  +--------------+  |  |
7  |  |  |  +--------+  |  |  |
9  |  |  |  |  +--+  |  |  |  |
\end{console}
}
You see that for each value of \verb!i! in the above,
you have to
\begin{tightlist}
  \li print a bunch of \verb!"|  "!,
  \li followed by \verb!'+'!,
  \li followed by a bunch of \verb!'-'!,
  \li followed by \verb!'+'!,
  \li followed by a bunch of \verb!"  |"!.
\end{tightlist}
So the pseudocode now becomes:
{\small
\begin{console}
for i = 1, 2, 3, 4, ..., ??:
    if i is odd:
        if i < 10: 
            draw a bunch of "|  "
            draw '+'
            draw a bunch of '-'
            draw '+'
            draw a bunch of "  |"
            draw newline
\end{console}
}
In fact it's easy to count the number of times to draw
the things in the bunches:
{\small
\begin{console}
i
1  +--------------------------+    0"|  ", 1'+', 26'-', 1'+', 0"  |", 1'\n'
3  |  +--------------------+  |    1"|  ", 1'+', 20'-', 1'+', 1"  1", 1'\n'
5  |  |  +--------------+  |  |    2"|  ", 1'+', 14'-', 1'+', 2"  1", 1'\n'
7  |  |  |  +--------+  |  |  |    3"|  ", 1'+',  8'-', 1'+', 3"  1", 1'\n'
9  |  |  |  |  +--+  |  |  |  |    4"|  ", 1'+',  2'-', 1'+', 4"  1", 1'\n'
\end{console}
}
You now have to relate the values of
\verb!i! = 1,3,5,7,9 to the number of \verb!"|  "! to draw:
{\small
\begin{console}
i
1  +--------------------------+    0"|  "
3  |  +--------------------+  |    1"|  "
5  |  |  +--------------+  |  |    2"|  "
7  |  |  |  +--------+  |  |  |    3"|  "
9  |  |  |  |  +--+  |  |  |  |    4"|  "
\end{console}
}
Do you see that: (9 - 1)/2 = 4, (7 - 1)/2 = 3, etc.
{\small
\begin{console}
for i = 1, 2, 3, 4, ..., ??:
    if i is odd:
        if i < 10:
            draw (i - 1)/2 "|  "
            draw '+'
            draw a bunch of '-'
            draw '+'
            draw a bunch of "  |"
\end{console}
{
In other words
{\small
\begin{console}
for i = 1, 2, 3, 4, ..., ??:
    if i is odd:
        if i < 10:
            for k = 1, 2, 3, ..., (i - 1)/2:
                draw "|  "
            draw '+'
            draw a bunch of '-'
            draw '+'
            draw a bunch of "  |"
\end{console}
}

The above should get you going.
