We will prove this by weak mathematical induction.
Let $P(n)$ be the statement
\[
P(n) =
\left(
1 + 2 + \cdots + n = \frac{n(n + 1)}{2}
\right)
\]
for $n \geq 0$.

\textsc{Base case}.
When $n = 0$,
\begin{align*}
1 + 2 + \cdots + n
&= 0 \\
&= \frac{0(0 + 1)}{2} \\
&= \frac{n(n + 1)}{2} 
\end{align*}
Hence $P(0)$ holds.

\textsc{Inductive case}.
Let $n \geq 0$ and assume $P(n)$ holds, i.e.,
\[
1 + 2 + \cdots + n = \frac{n(n + 1)}{2}
\]
Therefore
\begin{align*}
1 + 2 + \cdots + n + (n + 1)
&= \frac{n(n + 1)}{2} + (n + 1) \\
&= (n + 1) \left( \frac{n}{2} + 1 \right) \\
&= (n + 1) \left( \frac{n + 2}{2} \right) \\
&= \frac{(n + 1)((n + 1) + 1)}{2} 
\end{align*}
i.e., $P(n + 1)$ holds.

Therefore by weak mathematical induction, $P(n)$ holds for all
$n \geq 0$, i.e.,
\[
1 + 2 + \cdots + n 
= \frac{n(n + 1)}{2} 
\]
for all $n \geq 0$.
\qed

