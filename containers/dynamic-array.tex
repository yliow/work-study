%-*-latex-*-
\sectionthree{Dynamic array}
\begin{python0}
from solutions import *; clear()
\end{python0}

By a dynamic array, I mean an array in the heap or free store.
Recall that a dynamic array can have variable size:
\begin{console}[fontsize=\footnotesize]
T * p;
...
p = new T[size]; // size is a not-necessarily const variable
...
delete [] p;
...
\end{console}

All the runtimes are the same except when the array needs to be reallocated.
For instance let's think about the insert tail.
In the case of static array, if the array is not full,
then the runtime is $O(1)$.
If it's full, you abort mission -- either you thrown
an exception right away or you do nothing to the array.
What about the dynamic array?

If the dynamic array is not full, then the runtime is
again $O(1)$.
If it's full, you request for a large array.
Then you need to copy your values over to the new array
and perform insert tail and (don't forget!) deallocate the
memory used by the smaller array.
So in this case, the runtime is $O(n)$, not $O(1)$.
(We're ignoring the time taken for the memory request.)
Of course the memory allocation for the new array can fail --
but let's ignore that.

What about delete?
If the delete is delete tail?
Then the runtime is $O(1)$ just like the case of static array.
However if you want to make sure your dynamic array is not
wasting too much memory, then you want to use a smaller array
when the length of the array is less than 1/3 of the capacity.
You allocate an array of size say $2n$, copy the values over to the
new array, deallocate the old array.
In that case, you again have a runtime of $O(n)$.
