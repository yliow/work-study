%-*-latex-*-
%https://stackoverflow.com/questions/29859796/c-auto-vs-auto

\sectionthree{\cpp : Automatic type deduction}
\begin{python0}
from solutions import *; clear()
\end{python0}

[NOTE: See also CISS362 notes.]

Later versions of \gpp\ (\cpp 11 and later) support automatic type deductions and
range-based for-loops.
Here's an easy example on automatic type deduction\tinysidebar{auto}\index{auto@\texttt{auto}}: 
\begin{python}
s = r'''
#include <iostream>
#include <set>

int main()
{
    auto i = 42;
    auto x = 3.14;
    auto j = i;
    auto & k = i;
    std::cout << i << ' ' << x << ' ' << j << ' ' << k << '\n';
    i = -1;
    std::cout << i << ' ' << x << ' ' << j << ' ' << k << '\n';
    
    return 0;
}
'''.strip()
f = open('main.cpp', 'w')
f.write(s)
f.close()
\end{python}
\VerbatimInput[frame=single, fontsize=\footnotesize]{main.cpp}
\begin{python}
from latextool_basic import *
print(r'{\footnotesize %s}' % shell('g++ main.cpp; ./a.out'))  
\end{python}
\begin{python}
import os; os.system('rm main.cpp')
\end{python}
In this case \gpp\
figured out that \verb!i! should have type \verb!int!
and \verb!x! should be type \verb!double!.

(To understand how type deductions work, see CISS445 where I'll talk
about type inferencing for the OCAML language.)

Just because you can use \verb!auto!, it does \textit{not} mean
you can forget about what is the actual type you want.
Using something blindly is dangerous.
Frequently going back to explicit code is helpful.
You should learn the concept of \cpp\ type deduction and
ranged-based for-loop.
You might want to use it to quickly write code for assignments.
But once you are done with the assignment, you should
replace the type deduction code and range-based for-loops
with the explicit version so that you don't forget what
is the \cpp\ code generated.
Of course you can use it in your personal projects.
