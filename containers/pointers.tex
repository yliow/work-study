%-*-latex-*-
\sectionthree{Accessing values in a container: pointers}
\begin{python0}
from solutions import *; clear()
\end{python0}

You usually want to traverse a container, i.e.,
\lq\lq travel through'' a container.
In the case of an array, you can traverse the container
by running through the index values of the array.
(Like I said in the previous section, think of the index
as a mechanism for accessing a value in the array.)
For instance:
\begin{console}[frame=single,fontsize=\footnotesize]
for (int i = 0; i < n; ++i)
{
    std::cout << student[i] << std::endl;
}
\end{console}
Note that when you're doing a search, you are in fact doing
a traversal.
When doing a search in an array, when the target key value
is found, an index value is returned (with -1 indicating search failure).
Then index value is then used to access the value in the array
for some work (such as printing, modification, etc.)

In any case, an index value can be used to access a value in an array.

Note that there's another way to locate a particular student
in the \verb!student! array: by using pointers.
So instead of
\begin{console}[frame=single,fontsize=\footnotesize]
...

int find(const Student & student[], const std::string & id)
{
     ...
}

int main()
{
    ...
    std::string id;
    std::cin >> id;
    int index = find(student, id);
    std::cout << student[index] << std::endl;
    ...
}
\end{console}
we can do this:
\begin{console}[frame=single,fontsize=\footnotesize]
...

Student * find(Student student[], std::string id)
{
     ...
}

int main()
{
    ...    
    std::string id;
    std::cin >> id;
    Student * p = find(student, id);
    std::cout << (*p) << std::endl;
    ...
}
\end{console}


Compare the following traversal (using index values):
\begin{console}[frame=single,fontsize=\footnotesize]
for (int i = 0; i < n; ++i)
{
    std::cout << student[i] << std::endl;
}
\end{console}
with this (using pointers):
\begin{console}[frame=single,fontsize=\footnotesize]
for (const Student * p = &student[0]; p != &student[n]; ++p)
{
    std::cout << (*p) << std::endl;
}
\end{console}
Note that \verb!&student[n]! is the address
\textit{just past} the \textit{last} address occupied by the last student value
\verb!&student[n - 1]!.
Think about the above very carefully.
Drawing a picture of the computer's memory helps.

So what exactly is the difference between the two?!?
And which is better?
Note that the index version looks like this
\begin{console}[frame=single,fontsize=\footnotesize]
...
    std::cout << student[i] << std::endl;
...
\end{console}
and it uses \verb!student! and \verb!i!
whereas the pointer version
\begin{console}[frame=single,fontsize=\footnotesize]
...
    std::cout << (*p) << std::endl;
...
\end{console}
uses \verb!p!.
Without going into assembly/machine code,
it should already be clear that the index version
is slower since it requires memory access to more
variables (if nothing else), i.e., two variables
\texttt{student} and \texttt{i}
compared against one variable \texttt{p} for the pointer version.
In fact that \textit{is} the case at the assembly code level:
compilers actually translate array traversal by index
into array traversal by pointer.

Another reason why the pointer version is better is
because, if you think about the meaning of
\verb!student[i]!, you see that you start at the
memory location of
\verb!student[0]! and go to the
memory address of the $i$--th value in order to get to the
$i$--value in \verb!student!.
In the case of the array, this can be done rather quickly.
You will see later that there are other containers where
the computation to get to the
$i$--th value from the first value is very slow.
For some containers, it's a lot easier to compute the location
of the $i$--th value from the $(i-1)$--st value.
In other words the method of using index values is not
appropriate in some cases.
I'll give you specific examples later.
For now just remember that index values might
not make sense for some containers.

Note that the pointer method requires us to know the
beginning address of the container and the
\lq\lq end of address'' of the container or rather
the address that is just outside the address space occupied
by the container's values.
Nonetheless I'm going to call this \lq\lq outside end of address''
the end address.
