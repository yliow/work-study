%-*-latex-*-
\sectionthree{Accessing values in the container: index values}
\begin{python0}
from solutions import *; clear()
\end{python0}

Note that in the case of an array, you have the concept of
an index value of a particular value in the array.
The index is of course the \lq\lq position'' or
\lq\lq location"
of that value.
If the array is \texttt{x}, and \verb!i! contains a valid index value,
then \texttt{x[i]} arrives at a value in \verb!x!.
You can (and \textit{should}) think of an index value as
a mechanism for referring to or reaching a value in your array.

In the case of an array, ZZZ
\begin{myenum}
  \li You want to add a value to a container \textit{at an index position}.
  This case potentially overwrites a value.
  So for instance in the case of an array, values at that
  index position up to the last index value of the array is moved to the
  right by one index position.
  \li You want to delete a value from the container
  \textit{at an index position}.
  \li You want to search for a student in the container,
  \textit{returning the index position if it's found}.
  If the student is not found, a special index value is returned to
  indicate failure.
\end{myenum}

Again, the index represents a position.
In the case of the search, if a student is found and the index position
is returned, the index value can then be used to
locate the student (so as to modify satellite data, print
the satellite data, etc.)

You'll see very soon that a more uniform way to
access a value in a container is through
a pointer.
And to make the pointer more flexible,
we wrap the pointer in an object.
In fact sometimes, depending on what you want to do to the value
the pointer is pointing to, this object can contain more than just
a single pointer.
Such an object (that contains a pointer -- and maybe more -- to access a value
in a container) is called an iterator.
But first let's compare the index variable and the pointer ...
