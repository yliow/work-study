%-*-latex-*-
\sectionthree{\cpp : typedefs}
\begin{python0}
from solutions import *; clear()
\end{python0}

Besides the fact that automatic type deductions and
range-based for-loops makes container code easier to write,
\texttt{typedef}s help too.
You have already seen \texttt{typedef}s from CISS240, CISS245.
(You should check my CISS240, CISS245 notes on typedefs.)

Suppose you want to work with a pair of \verb!double!s
and also an array of pairs of \verb!double!s.
You can do this:
\begin{Verbatim}[frame=single,fontsize=\footnotesize]
#include <iostream>
#include <vector>

std::ostream & operator<<(std::ostream & cout,
                          const std::pair< double, double > & x)
{
    cout << '(' << x.first << ", " << x.second << ')';
    return cout;
}

std::ostream & operator<<(std::ostream & cout,
                    const std::vector< std::pair< double, double > > & v)
{
    cout << '{';
    std::string delim = "";
    for (auto & x: v)
    {
        cout << delim << x;
        delim = ", ";
    }
    cout << '}';
    return cout;
}

int main()
{
    std::pair< double, double > pair;
    std::vector< std::pair< double, double > > pairs;
      
    return 0;
}
\end{Verbatim}
Or you can use \texttt{typedef}s:
\begin{Verbatim}[frame=single,fontsize=\footnotesize]
#include <iostream>
#include <vector>

typedef std::pair< double, double > twodouble;
typedef std::vector< std::pair< double, double > > twodoubles;

std::ostream & operator<<(std::ostream & cout, twodouble & x)
{
    cout << '(' << x.first << ", " << x.second << ')';
    return cout;
}

std::ostream & operator<<(std::ostream & cout, twodoubles & v)
{
    cout << '{';
    std::string delim = "";
    for (auto & x: v)
    {
        cout << delim << x;
        delim = ", ";
    }
    cout << '}';
    return cout;
}

int main()
{
    twodouble pair;
    twodoubles pairs;
      
    return 0;
}
\end{Verbatim}

Clearly the \texttt{typedef}s help!
