%-*-latex-*-
\sectionthree{Memory fragmentation}
\begin{python0}
from solutions import *; clear()
\end{python0}

From the above, it's clear that dynamic arrays (arrays in the heap)
is better than static arrays (arrays in an area of your computer
memory called stack segment).

One extremely important fact about dynamic arrays
is this:
because of the unpredictable calls to the heap for
allocation and deallocation of memory, after some time
your heap becomes highly \lq\lq fragmented".
What is this \defone{memory fragmentation}?

Suppose when you first run your program, your heap looks like this
(i.e., no memory allocation yet so the whole heap is available):
\begin{python}
from latextool_basic import *
p = Plot()
p += Rect(x0=0, y0=0, x1=4, y1=6)
print(p)
\end{python}
Suppose you program requests for a block of 2KB of memory
for pointer \verb!p0!.
Then you memory might look like this:
\begin{python}
from latextool_basic import *
p = Plot()
p += Rect(x0=0, y0=0, x1=4, y1=6)
p += Rect(x0=0, y0=0, x1=4, y1=1, background='blue!10')
print(p)
\end{python}
And then your program asks for another block of 1KB for \verb!p1!
and then 3KB for \verb!p2!
\begin{python}
from latextool_basic import *
p = Plot()
p += Rect(x0=0, y0=0, x1=4, y1=6)
p += Rect(x0=0, y0=0, x1=4, y1=1, background='blue!10')
p += Rect(x0=0, y0=1, x1=4, y1=1.5, background='blue!10')
p += Rect(x0=0, y0=1.5, x1=4, y1=3, background='blue!10')
print(p)
\end{python}
And if your program deallocates the memory used by \verb!p1!, then
the heap looks like this:
\begin{python}
from latextool_basic import *
p = Plot()
p += Rect(x0=0, y0=0, x1=4, y1=6)
p += Rect(x0=0, y0=0, x1=4, y1=1, background='blue!10')
#p += Rect(x0=0, y0=1, x1=4, y1=1.5, background='blue!10')
p += Rect(x0=0, y0=1.5, x1=4, y1=3, background='blue!10')
print(p)
\end{python}
After some time, your heap might look like this:
\begin{python}
from latextool_basic import *
p = Plot()
p += Rect(x0=0, y0=0, x1=4, y1=6)
p += Rect(x0=0, y0=0, x1=4, y1=1, background='blue!10')
#p += Rect(x0=0, y0=1, x1=4, y1=1.5, background='blue!10')
p += Rect(x0=0, y0=1.5, x1=4, y1=3, background='blue!10')
p += Rect(x0=0, y0=3.5, x1=4, y1=3.7, background='blue!10')
p += Rect(x0=0, y0=4, x1=4, y1=4.25, background='blue!10')
p += Rect(x0=0, y0=4.25, x1=4, y1=4.5, background='blue!10')
p += Rect(x0=0, y0=4.6, x1=4, y1=4.75, background='blue!10')
p += Rect(x0=0, y0=4.9, x1=4, y1=5.1, background='blue!10')
p += Rect(x0=0, y0=5.5, x1=4, y1=5.8, background='blue!10')
print(p)
\end{python}
due to a huge number of memory allocations and deallocations.

Your heap memory becomes highly fragmented if you have
many blocks of free memory that are not continguous.
This is a problem.
Why? ...

Because you might actually have a total of 10KB of memory
which is scattered all over the place and the largest
\textit{contiguous} free block of memory might only be
1KB.
In that case, you cannot allocate \textit{contiguous} free
memory for a 2KB dynamic
array even if you actually have a 10 KB of \textit{total} free memory.

The fact that the memory usage of an array must be contiguous
should be clear from the facts found in my chapters on pointers from CISS245:
If \verb!p! points to an array of values,
note that the address in \verb!p + i! is very close to the
value of \verb!p + i + 1!.
(More about low level details about memory usage in CISS360.)

We will be studying many data structures (i.e. containers of values)
that will hold values in a \lq\lq scattered" way to combat this
very serious shortcoming of the array containers.
