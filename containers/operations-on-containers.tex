%-*-latex-*-
\sectionthree{Operations on containers}
\begin{python0}
from solutions import *; clear()
\end{python0}

What are some of the common operations you want to have on a container,
such as for instance an array of students where the key
is the student's id and the satellite data include
the student's first name, last name, etc.?
\begin{tightlist}
  \li You want to \textbf{add} a value
  (example: a student object) to a container
  (example: a container of students).
  \li You want to \textbf{delete} a value from the container.
  \li You want to \textbf{search} for a student in the container.
\end{tightlist}
The above are probably the most common and most important.
Here are some more:
\begin{tightlist}
  \li
  When the container has a sense of ordering of values,
  you might want to do this:
  find the $k$--th smallest value 
  or $k$--largest value in the container.
  \li
  When you have two containers of the same kind,
  you might want to merge both into one.
\end{tightlist}
Etc.

In the case of an array, there \textit{is}  a concept of ordering.
You have the concept of the \lq\lq third student'',
the \lq\lq $105$--th student'', etc.
If you prefer to call students
the  \lq\lq zeroth'' student,
the  \lq\lq first'' student,
the  \lq\lq second'' student,
the  \lq\lq third'' student, etc
then
the \lq\lq third'' student in the container is
\verb!student[3]!.
Of course what's meant by \lq\lq third'' depends on
how you order the values.
In the case where the order is
\[
\texttt{student[0], student[1], student[2], ..., student[n - 1]}
\]
the $k$--th value is \verb!student[k]!.
However if I prefer to
view the values like this (i.e., in reverse direction):
\[
\texttt{student[n - 1], student[n - 2], student[n - 3], ..., student[0]}
\]
then the $k$ value means \texttt{student[n - k]}.

There are many, many, many other possible operations
on the container, depending on the \lq\lq structure'' of the contains.

For instance if the container is a graph, i.e., think of this
as a bunch of dots and lines where data is stored at the dots.
Then one operation is \lq\lq find the shortest path from one dot
to another.''
