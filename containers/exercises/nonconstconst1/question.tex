\tinysidebar{\debug{exercises/{nonconstconst1/question.tex}}}
Add a \verb!const_iterator! class inside your
\verb!vector! class.
Note that if \verb!p! is a constant iterator to (say)
a \cpp\ STL vector object, then
\verb!*p! refers to a value in that vector object.
\verb!*p! has read access but not write access, i.e.,:
\begin{console}[frame=single]
int a = *p; // OK
*p = 42;    // BAD!!!
\end{console}
This means that in your constant iterator class inside your vector class
you have
\begin{console}[frame=single,fontsize=\footnotesize]
template < typename T >
class vector
{
    class const_iterator
    {
    public:
        T operator*() const
        { ... }
    };
};
\end{console}
It's true that the actual value in the vector is not changed since
you're returning a copy of that value.
But this might be misleading to the person using your constant iterator class
(including yourself if you're not careful).
It's better to do this:
\begin{console}[frame=single,fontsize=\footnotesize]
template < typename T >
class vector
{
    class const_iterator
    {
    public:
        const T & operator*() const
        { ... }
    };
};
\end{console}
Why? (You ought to know.)
