%-*-latex-*-
% -*-latex-*-
\sectionthree{\cpp\ integer types, size type, and midpoints}
\begin{python0}
from solutions import *; clear()
\end{python0}

\subsection{Integer types}

You should already know (from CISS240, CISS245) that there
are several integer types.
\begin{myenum}
  \li \verb!int!
  \li \verb!long int!
  \li \verb!long long int!
  \li \verb!unsigned int!
  \li \verb!unsigned long int!
  \li \verb!unsigned long long int!
\end{myenum}
\verb!char! is actually also an integer type,
which is why \verb!char! values can be case labels.
There's also \verb!unsigned char! type.
\verb!bool! is also an integer type.

The problem with the \verb!int!, \verb!long int!,
\verb!long long int! is that the
number of bits used by them is not standardized.
For me (\cpp\ code compiled with \gpp\ on F31 vm),
\verb!int! is 32 bits and an \verb!int!
value lies in the range $-2^{31}, ..., 2^{31} - 1$
and \verb!unsigned int! is 32 bits and an \verb!unsigned int!
value lies in the range $0, ..., 2^{32} - 1$.
On such systems, the \verb!long long int! is 64 bits
and the \verb!long int! can be either 32 bits or 64 bits.
In the future, an \verb!int! might be 64 bits wide.
(When I was in high school, my PC had an
Intel 80286 CPU which had a 16-bit
architecture, i.e., an \verb!int! then had 16 bits.)
This can be a problem:
if you bring your \cpp\ code from a machine where an \verb!int!
is 32 bits in length and recompile it on another machine where
the \verb!int! is 16 bits in length,
your program might not work correctly when it processes
an integer such as $2^{16}$.
(Why?)

To make your program portable (i.e., can be compiled and
run on any machine that has a C/\cpp\ compiler),
C/\cpp\ provides integer types with fixed bit lengths.
\begin{myenum}
\li \verb!int8_t!, \verb!uint8_t!
\li \verb!int16_t!, \verb!uint16_t!
\li \verb!int32_t!, \verb!uint32_t!
\li \verb!int64_t!, \verb!uint64_t!
\end{myenum}
The \lq\lq \verb!_t"! in the above means \lq\lq type".
There are constants for the limits of the above types.
For instance:
\begin{myenum}
\li 
  For \verb!int32_t!,
  the minimum is \verb!INT32_MIN! and
  the maximum is \verb!INT32_MAX!.
\li
  For \verb!uint32_t!,
  the minimum is \verb!0! and
  the maximum is \verb!UINT32_MAX!.
\end{myenum}
Professionally written \cpp\ code tends to use these types.
If you want to use the above, you need to \lq\lq\verb!#include <cstdint>!".


\subsection{Size type}

Another thing to know is that
for containers (including \verb!std::vector!),
we have the concept of the size of a container.
Recall that if \verb!v! is a \verb!std::vector< int >! object,
the size of \verb!v! is
\[
\texttt{v.size()}
\]
The type of \verb!v.size()! and \verb!v.capacity()!
is \verb!size_t! (\lq\lq size type").
\verb!size_t! is an unsigned integer type and is
either \verb!uint32_t! or \verb!uint64_t!,
depending on the platform.
This makes sense since you don't expect a size to be negative!!!
(It doesn't make sense to say a vector has -5 values.)

When you write something like this:
\begin{console}[fontsize=\footnotesize]
for (int i = 0; i < v.size(); ++i)
{
    ...  
}
\end{console}
you'll get a warning since \verb!i! is an \verb!int!
and \verb!v.size()! is an \verb!unsigned int!:
\begin{console}[fontsize=\footnotesize]
main.cpp:5:23: warning: comparison of integer expressions of different
signedness: ‘int’ and ‘std::vector<int>::size_type’ {aka ‘long unsigned
int’} [-Wsign-compare]
    5 |     for (int i = 0; i < v.size(); ++i)
      |                     ~~^~~~~~~~~~
\end{console}
To be absolutely correct, one would write
\begin{console}[fontsize=\footnotesize]
for (unsigned int i = 0; i < v.size(); ++i)
{
    ...  
}
\end{console}
or even better
\begin{console}[fontsize=\footnotesize]
for (size_t i = 0; i < v.size(); ++i)
{
    ...  
}
\end{console}
If you want to use \verb!size_t!
you might have to do \lq\lq\verb!#include <cstddef>!":
\begin{console}[fontsize=\footnotesize]
#include <cstddef> // remove this and you'll get an error
int main()
{
    size_t i = 0;
    return 0;
}
\end{console}
However if you are already using any \cpp\ STL container classes such as
\verb!std::vector!, the container class has already done
\lq\lq\verb!#include <stddef>!" since \verb!std::vector! class uses
\verb!size_t!.

In some cases,
it won't make a difference whether you are using \verb!int! values
or \verb!unsigned int! values.
In particular for 32-bit \verb!int! and 32-bit \verb!unsigned int!,
their overlap is $0,...,2^{31}-1$.
Because of the math of binary representations of integers and
unsigned integers (see CISS360),
it doesn't matter if you are using \verb!int! or \verb!unsigned int!
if you stay in $0,...,2^{31}-1$.
So it's perfectly OK to write
\begin{console}[fontsize=\footnotesize]
for (int i = 0; i < v.size(); ++i)
{
    std::cout << x[i] << '\n';
}
\end{console}
if your computations involving \verb!i! and \verb!v.size()! stays
in $0,...,2^{31}-1$, except that you might get compiler warnings.

One final point about working with size type and type for index values.
Recall from CISS240 when we use \verb!int! for index type,
we frequently use \verb!-1! to indicate \lq\lq invalid index value".
For instance during a search (linear search or binary search) when a
target value is not found in an array, we return \verb!-1!.
What will happen if you want to handle such scenarios and you are
using \verb!size_t! or \verb!unsigned int! for your index type?

When I run this:
\begin{console}
#include <iostream>

int main()
{
    std::cout << sizeof(size_t) << '\n';
    size_t x = -1;
    std::cout << x << '\n';
    return 0;
}
\end{console}
I get
\begin{console}
8
18446744073709551615
\end{console}
(This is on our f31 vm).
What's happening?
The above tells me that a \verb!size_t! value consumes 8 bytes, i.e., 64 bits.
That huge positive integer \verb!18446744073709551615! is of course
then $2^{64} - 1$. (Check this yourself).
The \verb!-1! (as an \verb!int!) value is the same as
\verb!18446744073709551615! (as a \verb!size_t! value).
This is due to \lq\lq clock arithmetic", the way math
works in our CPU.
I already talked about this in CISS240 when I talked about
the \lq\lq clock arithmetic" of the \verb!int! type.
And \lq\lq clock arithmetic" is the informal term for
\lq\lq modular arithmetic"
(discrete math, CISS360, CISS451) which is the mathematical model used in
designing CPUs.

This means that if you have a really huge array \verb!x!
and \verb!x[18446744073709551615]! is actually a value in the array
and some search algorithm returns \verb!-1!
to a \verb!size_t! variable, you'll have a problem:
does the algorithm mean that value is not found
or does it mean the value is found at index \verb!18446744073709551615!?

These are things (among others) that a serious computer scientist
have to be aware of.

By the way since the \cpp\ string class has a definition
of the constant \verb!std::string::npos! which is \verb!size_t (-1)!.
The name of this constant \verb!npos! reads
\lq\lq not a position", \lq\lq no position", \lq\lq non-position".

\subsection{Midpoints}

Frequently you need to compute the midpoint of two indices.
We saw that in binary search in CISS240.
You will see a lot more of that.
Given two index values say \verb!left! and \verb!right!,
the midpoint index value can be computed as
\[
\verb!mid = (left + right) / 2!
\]
However there's problem with this if \verb!left! and \verb!right! are
huge -- \verb!left + right! will give you an arithmetic overflow.
If \verb!left! and \verb!right! are unsigned int of 32-bits,
then \verb!left + right! overflows if
\verb!left!
and
\verb!right! are both slightly smaller than $2^{32} - 1$.
A better mid point calculation is
\[
\verb!mid = left + (right - left) / 2!
\]
Make sure you run this program:
\begin{console}[fontsize=\footnotesize]
#include <iostream>

int main()
{
    unsigned int left = 0 - 3;  // 2^32 - 3
    unsigned int right = 0 - 1; // 2^32 - 1
    unsigned int mid = 0 - 2;                                // correct
    std::cout << left << ' ' << mid << ' ' << right << '\n';
    mid = (left + right) / 2;                                // wrong
    std::cout << left << ' ' << mid << ' ' << right << '\n';
    mid = left + (right - left) / 2;                         // correct
    std::cout << left << ' ' << mid << ' ' << right << '\n';
    return 0;
}
\end{console}

We also need to compute the midpoint between two addresses.
Here's how you do it (similar to the above):
\begin{console}[fontsize=\footnotesize]
#include <iostream>

int main()
{
    int x[8] = {2, 3, 5, 7};
    int * left = &x[0];
    int * right = &x[4];
    int * mid = left + (right - left) / 2; // ERROR: (left + right) / 2 !!!
    std::cout << sizeof(int) << '\n';
    std::cout << left << ' ' << mid << ' ' << right << '\n';
    std::cout << (unsigned long long) left << ' '
              << (unsigned long long) mid << ' '
              << (unsigned long long) right << '\n'; 
    std::cout << *left << ' ' << *mid << ' ' << *right << '\n';
    return 0;
}
\end{console}
Note that for pointers, you cannot compute
\lq\lq\verb!(left + right) / 2!" because you cannot add addresses.
Try it out yourself and read the error message.

In case you want to store the difference of addresses,
the type for difference of addresses is \verb!std::ptrdiff_t!
and if you want to use it do \lq\lq\verb!#include <cstddef>!":
\begin{console}[fontsize=\footnotesize]
#include <iostream>
#include <cstddef>
  
int main()
{
    int x[8] = {2, 3, 5, 7};
    int * left = &x[0];
    int * right = &x[4];
    std::ptrdiff_t diff = right - left;
    return 0;
}
\end{console}
