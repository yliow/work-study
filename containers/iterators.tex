%-*-latex-*-
\sectionthree{Accessing values in a container: iterators}
\begin{python0}
from solutions import *; clear()
\end{python0}

In general, lots of \cpp\ STL containers come with
features to access values in the containers using
iterators (which are more or less pointers).
For instance in the case of \verb!std::vector! class (see CISS245),
you can do the following to traverse a vector of integers \verb!v!:
\begin{console}[fontsize=\footnotesize]
for (typename std::vector< int >::iterator p = v.begin();
     p != v.end(); ++p)
{
     ... do something with (*p) ...
}
\end{console}
In the above
\[
\texttt{std::vector< int >::iterator}
\]
is a class:
it's a class inside the \verb!std::vector! class.
(Yes you can put a class inside a class.)
So you would expect the \verb!std::vector! class to look like this:
\begin{console}[frame=single,fontsize=\footnotesize]
template < typename T >
class vector
{
public:
    class iterator
    {
    };
};
\end{console}
Objects of this class, \verb!iterator!, are iterators to
for instance \verb!std::vector< int >! objects
(and
remember that iterators are like pointers).

\verb!v.begin()! will \lq\lq essentially''
return a pointer to \verb!v[0]!.
OK ... it's not a pointer ... it's an iterator, i.e.,
an object \textit{containing
a pointer} that points to \verb!v[0]!.
\verb!v.end()! also returns an iterator that contains
a pointer that points to \verb!&v[n]! where
\verb!n! is \verb!v.size()!.
Again, \verb!v.end()! does not return a pointer value,
but rather an \textit{iterator} that \textit{contains a pointer} that points to \verb!&v[n]!.
And remember that \verb!&v[n]! is not an address of one of the values of
\verb!v! -- it's an address that is \textit{just beyond} or
\textit{just outside} of \verb!v!.
Here's a picture that represents these two iterators:

\begin{python}
from latextool_basic import *
from latexcircuit import *

p = Plot()
c = RectContainer(x=0, y=0)
for i,x in enumerate('2357'):
        c += Rect2(x0=0, y0=0, x1=0.6, y1=0.6, label=r'{\texttt{%s}}' % x)
                 
p += c

x,y = c[0].bottomleft(); y = y + 2
s = Rect2(x0=x, y0=y, x1=x+0.6, y1=y+0.6, label='')
p += s
p += Line(points=[s.center(), c[0].top()], endstyle='>')
x,y = s.left()
X = POINT(x=x, y=y, r=0, label=r'\texttt{v.begin()}')
p += str(X)

x,y = c[-1].bottomright(); y = y + 2
s = Rect2(x0=x, y0=y, x1=x+0.6, y1=y+0.6, label='')
p += s
x, y = c[-1].top(); x += 0.6
p += Line(points=[s.center(), (x,y)], endstyle='>')
x,y = s.right()
X = POINT(x=x, y=y, r=0, anchor='west', label=r'\texttt{v.end()}')
p += str(X)

x,y = c[0].left()
X = POINT(x=x, y=y, r=0, label=r'\texttt{v}')
p += str(X)

print(p)
\end{python}

\cpp\ STL has a container called the \verb!std::list!.
If \verb!x! is a list (i.e., STL list) of integers,
then you can do this:
\begin{console}[fontsize=\footnotesize]
for (typename std::list< int >::iterator p = x.begin();
     p != x.end(); ++p)
{
     ... do something with (*p) ...
}
\end{console}
A list is a kind of container that I will talk about soon: it's a
doubly-linked list.
There's another \cpp\ STL contains called the \verb!std::set!.
If \verb!x! is a set of integers, then you can do this:
\begin{console}[fontsize=\footnotesize]
for (typename std::set< int >::iterator p = x.begin();
     p != x.end(); ++p)
{
     ... do something with (*p) ...
}
\end{console}
As you can see, the code to run through the values in the
above three containers look almost exactly the same.

Get it? This is why knowing iterators is useful, because of the
fact that they unify the access to values in \cpp\ STL containers.

By the way, as I said above, the code
\begin{console}[fontsize=\footnotesize]
... typename std::vector< int >::iterator p ...
\end{console}
tells you that \textit{inside} the class \texttt{std::vector< int >},
there's a class called \texttt{iterator}.
In other words the \texttt{std::vector} class looks like this:
\begin{console}[frame=single, fontsize=\footnotesize]
template < typename T >
class vector
{
public:
    class iterator
    {
    };
private:
};
\end{console}
Now you might ask ... what's this \verb!typename! keyword thing for:
\begin{console}[frame=single,commandchars=\\\{\},fontsize=\footnotesize]
... \underline{typename} std::vector< int >::iterator p ...
\end{console}
The \verb!typename! tells your compiler that
\verb!std::vector< int >::iterator! is a type.
Why must you do this?
Because remember that from CISS245, when \cpp\ sees \verb![class]::[something]!,
it could mean that \verb![something]! is a static instance member.
The default is static instance member.
The \verb!typename! tells your \cpp\ compiler that it's a class,
not a static member.

Note that for the case of \verb!std::vector!, you do have
\verb!operator[]! so that you can do this:
\begin{console}[fontsize=\footnotesize]
for (size_t i = 0; i < v.size(); ++i)
{
     ... v[i] ...
}
\end{console}
i.e., you can run through a vector container using index values
instead of iterators.
The \verb!size_t!\index{sizet@\texttt{size\_t}}, \lq\lq size type", is basically \verb!unsigned int!
and is always $\geq 0$.

The problem is that \verb!std::list! does \textit{not} have
\verb!operator[]!.
So the index method of traversal does \textit{not} apply to
STL lists.
And why doesn't STL list have \verb!operator[]!?
Because the list container has a certain structure that makes
this operator, i.e. \verb!operator[]!,
very slow if this was ever implemented.
In other words, you shouldn't ever want or think
about getting the $i$--th value
of a list using \verb!operator[]!.
That's why it's not included in the list library.

However arrays can go to the $i$--th value
extremely fast.
(This will be explained in CISS360.)
That's why arrays (and \verb!std::vector! objects)
have \verb!operator[]!.

Wrapping a pointer inside an object (an iterator)
is more flexible simply because there are times
when the \lq\lq move pointer to the next object in the container''
is very complex and needs extra data.
For instance suppose for some bizarre reason you
have a vector \verb!v! of 10 values and you want to access the
values in this order:
\[
\texttt{v[1], v[2], v[2], v[3], v[3], v[3], v[4], v[4], ...}
\]
i.e., you want to process \verb!v[i]! \verb!i! times.
If you do this frequently, you might want to create an iterator
class that does that so that doing \verb!++p! moves
the iterator like the above.
The iterator object must then remember the index value
 and the number of times it has processes the value at the index value.
 Right?
Get it??
If you put all that information into the iterator object,
it will simplify your (crazy) algorithm.
