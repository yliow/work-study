%-*-latex-*-
\sectionthree{Containers}
\begin{python0}
from solutions import *; clear()
\end{python0}

From your programming experience, you see that your programming languages
provide some basic types (\verb!int!, \verb!double!, \verb!bool!, etc.)
with operations/operators and functions.
You then build a 
software using these types.
On the way to building your software system, you will find that 
you usually have to build new types 
(such as structures) and functions to be used by your software.
Some times, you will find that after some thought, certain functions
can be placed inside structures, i.e., you combine data and functions,
to get classes.

For instance if you want to develop an RSA crypto system,
you will need to perform computations on 
integers with arbitrary number of
digits. 
Well ... it cannot be arbitrary since you are limited by 
your computer system resources:
you would need to work with integers with hundreds of digits.
C/\cpp\ does not allow you to do that.
To model integers with arbitrary number of digits, you would probably use
heap-based arrays, i.e., you would probably want to use pointers to 
\verb!int! arrays in the free store.
Since you expect to perform the usual arithmetic operations on 
such integers, you would probably want to develop a class
called, say, \verb!LongInt!.

Each \verb!LongInt! object is essentially a container of digits.
(Technically speaking it contains a pointer that points to 
a container -- i.e., array -- of digits in the heap.
But you get the point.)

In your arithmetic operators (addition, subtraction, blah-blah-blah)
you see that you frequently need to \textit{travel through} the container, 
i.e., you will see something like this frequently in your code:
\begin{console}[fontsize=\footnotesize]
for (int i = 0; i < n; i++)
{
    ... do something with p[i] ...
}
\end{console}
if \verb!p! is the pointer in your class and \verb!n! is the number of
digits \verb!p! points to.

The technical CS term is that you have to \defone{traverse} the container.

Since the heap-based array is linear, i.e., the layout of the data
in the container is along a straight line,
you see at least two different ways to travel through (traverse) the 
container.
It's either
\begin{console}[fontsize=\footnotesize]
for (int i = 0; i < n; i++) // forward traversal
{
    ... do something with p[i] ...
}
\end{console}
or
\begin{console}[fontsize=\footnotesize]
for (int i = n - 1; i >= 0; i--) // reverse traversal
{
    ... do something with p[i] ...
}
\end{console}
So in this case, there are two obvious traversals.

You can also fantasy about something like this:
\begin{console}[fontsize=\footnotesize]
for (int i = 0; i < n; i += 2) // forward, even index
{
    ... do something with p[i] ...
}
\end{console}

Anyhow, you see that your \verb!LongInt! class will need to 
have a container and you need to traverse the container.

Other things you want to do with a container is to put things
into the container, remove things from the container, 
search for a value in the container, etc.

If the above idea of a container applies only to your \verb!LongInt!
class, then there's no reason to make a big deal out of it and create
a whole theory or body of knowledge of such thingies.
But it turns out that containers of different shapes (i.e. not 
necessarily along a straight line) is actually very common in nature
(i.e., in computations).

When I say that containers are common in nature I really mean it.
Think of your bookshelves.
As a whole, it's a container of books.
When you need a book, you have a search for it and 
then remove it from your shelves.
Of course you probably want to add books to your shelves too.

In the same way, a database is a container of data.
For scientific purposes, it could be a database of genomic data;
for business applications, it could be a database of customer, 
product, and sales transaction data.
You also want to add data, remove data, and search for data in databases. 

In a computer game, you might have a container of all spaceships attacking you.
In a fantasy role playing game, you might have an inventory list
(weapons and what-nots) which is a container.

So I'm not lying nor exaggerating
when I say that the world is full of containers.

By the way, although beginning algorithms courses focus on the
algorithms operating on containers, you should know that the study of
algorithms is broader than that.
