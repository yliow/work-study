%-*-latex-*-
\sectionthree{\cpp : Range-based loops}
\begin{python0}
from solutions import *; clear()
\end{python0}

Here's an example of
\defone{range-based for-loop}
where \verb!x! is a
\verb!std::vector< int >! object:
\begin{console}[fontsize=\footnotesize]
for (int i: x)
{
    std::cout << i << '\n';
}
\end{console}
This is frequently used together with automatic type deduction:
\index{auto@\texttt{auto}}\tinysidebar{\texttt{auto}}
\begin{console}[fontsize=\footnotesize]
for (auto i: x)
{
    std::cout << i << '\n';
}
\end{console}

Here's an example:
\begin{python}
s = r'''
#include <iostream>
#include <string>
#include <vector>

template< typename T >
std::ostream & operator<<(std::ostream & cout,
                          const std::vector< T > & x)
{
    std::string delim = "";                            
    cout << '{';                            
    for (auto i: x)
    {
        cout << delim << i;
        delim = ", ";
    }
    cout << '}';
    return cout;
}

int main()
{
    int w[] = {1,2,3};
    for (int i: w)
    {
        std::cout << i << ' ';
    }
    std::vector< int > v = {2, 3, 5, 7, 11, 13};
    std::cout << v << '\n';

    auto p = v.begin();
    ++p;
    std::cout << (*p) << '\n';
    auto q = v.end();
    --q;
    v.erase(p, q);
    std::cout << v << '\n';
    
    return 0;
}
'''.strip()
f = open('main.cpp', 'w')
f.write(s)
f.close()
\end{python}
\VerbatimInput[frame=single, fontsize=\footnotesize]{main.cpp}
\begin{python}
from latextool_basic import *
print(r'{\footnotesize %s}' % shell('g++ main.cpp; ./a.out'))
import os; os.system('rm main.cpp a.out')
\end{python}
If you want to modify a value in a container, you should use this in
the declaration of \verb!v!:
\begin{console}[fontsize=\footnotesize]
for (auto & v: x)
{
    v = 0;
}
\end{console}
so that \verb!v! is a reference to a value in \verb!x!.
Of course you cannot change the values of \verb!x! if \verb!x! is
constant.
So in the above I'm assuming \verb!x! is not constant.
That's because for
\begin{console}[fontsize=\footnotesize]
for (auto v: x)
{
    v = 0;
}
\end{console}
a separate local copy of a value in \verb!x! is created for \verb!v!
and changing \verb!v! to \verb!0! will not
change the corresponding value in \verb!x!.
And of course it's
slow especially if the values in container \verb!x! are complex
and time consuming to construct.
There are some cases where you have to do \textit{this} instead\index{&&@\texttt{\&\&}}\tinysidebar{\texttt{\&\&}}
\begin{console}[fontsize=\footnotesize]
for (auto && v: x)
{
    v = 0;
}
\end{console}
if you want to change the values in \verb!x!.
So I suggest you use this \verb!&&! method anyway
if you want to change the values in \verb!x!.

Usually you should make \verb!v! a reference to a value in \verb!x!
and not make copies all the time.
If \verb!x! is constant and your loop does not change the values in \verb!x!,
you can do this when the values in \verb!x! are complex (example: objects):
\begin{console}[fontsize=\footnotesize]
for (const auto & v: x)
{
    ...
}
\end{console}
Note that if your \verb!x! is constant and you do
\begin{console}[fontsize=\footnotesize]
for (auto & v: x)
{
    ...
}
\end{console}
the type deduction will be smart
enough to make the \verb!v! a constant reference.
There are many other variations.

In summary if you do not want to change the values in \verb!x!
and the values are simple and you don't mind copying
the values of \verb!x! to \verb!v!, do this:
\begin{console}[fontsize=\footnotesize]
for (auto v: x)
{
    ...
}
\end{console}
If you do not want to change the values in \verb!x!
and you want \verb!v! to reference values in \verb!x!, you should do this:
\begin{console}[fontsize=\footnotesize]
for (const auto & v: x)
{
    ...
}
\end{console}
(the \verb!const! is redundant if \verb!x! is constant).
If \verb!x! is not constant and you want to change the
values in \verb!x!, you can do
\begin{console}[fontsize=\footnotesize]
for (auto & v: x)
{
    ...
}
\end{console}
or
\begin{console}[fontsize=\footnotesize]
for (auto && v: x)
{
    ...
}
\end{console}
where the second version is probably better.

The following is an example involving \verb!std::vector!:
\begin{python}
s = r'''
#include <iostream>
#include <string>
#include <vector>

template< typename T >
std::ostream & operator<<(std::ostream & cout,
                          const std::vector< T > & x)
{
    std::string delim = "";
    cout << '{';
    for (auto & v: x)
    {
        cout << delim << v;
        delim = ", ";
    }
    cout << '}';
    return cout;
}

int main()
{
    std::vector< int > v = {2, 3, 5, 7, 11, 13};
    std::cout << v << '\n';

    for (auto x: v)
    {
        x = 0;
    }
    std::cout << v << '\n';
    
    for (auto & x: v)
    {
        x = 1;
    }
    std::cout << v << '\n';
    
    for (auto && x: v)
    {
        x = 2;
    }
    std::cout << v << '\n';
    
    return 0;
}
'''.strip()
f = open('main.cpp', 'w'); f.write(s); f.close()
\end{python}
\VerbatimInput[frame=single,fontsize=\footnotesize]{main.cpp}
\begin{python}
from latextool_basic import *
print(r'{\footnotesize %s}' % shell('g++ main.cpp; ./a.out'))
import os; os.system('rm main.cpp')
\end{python}
And again these are not the only possibilities.
And if the compiler does not deduce the right type
or your code does not allow a suitable type deduction,
I suggest you go back to basics and type in the actual type you really want.

In some STL containers, the values cannot be changed. 
For instance the
values in a \verb!std::set! object cannot be changed -- they are immutable.
Of course you know that
values in a \verb!std::vector! can be changed if it's not constant.

Again be aware of the above syntax.
Try out the above examples.
But avoid using it in assignments
until you really know your iterators without using shortcuts
like \verb!auto! and range-based for-loops.

