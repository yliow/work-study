%-*-latex-*-
\sectionthree{\cpp : initializer lists}
\begin{python0}
from solutions import *; clear()
\end{python0}

The \lq\lq initializer list" refers to the array initializer from
CISS240:
\begin{console}[fontsize=\small]
int x[] = {2, 3, 5};
\end{console}
and not the constructor initializer list from CISS245:
\begin{console}[fontsize=\small]
class C
{
public:
    C()  
    : x_(2), y_(3), z(5)
    {}
};
\end{console}
Note that the original notation for (array) initializer list
can only be used during declaration.

In the newer \cpp\ (since \cpp 11), the \defone{initializer list} notation can be used
when you work with \cpp\ STL containers and it can be used for assignments.

Before the introduction of initializer lists for STL template classes,
it was very tedious to initialize an STL container.
You can first initialize it to an empty
container and then you put values into the
container one value at a time:
\begin{console}[fontsize=\footnotesize]
std::vector< int > v;
v.push_back(2);
v.push_back(3);
v.push_back(5);
v.push_back(7);
\end{console}
You can also do this:
\begin{console}[fontsize=\footnotesize]
int x[] = {2, 3, 5, 7};
std::vector< int > v(x, x + 4);
\end{console}
But you do have to create an array of the values and the unnecessary name
of \verb!x!.
Both cases are tedious.

With the new initializer lists for STL template classes, you can do
the following during constructor call with initialization:
\begin{console}[fontsize=\footnotesize]
std::vector< int > v = {2, 3, 5, 7};
\end{console}
or
\begin{console}[fontsize=\footnotesize]
std::vector< int > v {2, 3, 5, 7};
\end{console}
You can also use the initializer list notation with the
assignment operator
\begin{console}[fontsize=\footnotesize]
std::vector< int > v;
v = {2, 3, 5, 7};
\end{console}

Run and study the following:
\begin{python}
s = r'''
#include <iostream>
#include <vector>

typedef std::pair< double, double > twodouble;
typedef std::vector< twodouble > twodoubles;

std::ostream & operator<<(std::ostream & cout, const twodouble & x)
{
    cout << '(' << x.first << ", " << x.second << ')';
    return cout;
}

std::ostream & operator<<(std::ostream & cout, const twodoubles & v)
{
    cout << '{';
    std::string delim = "";
    for (auto & x: v)
    {
        cout << delim << x;
        delim = ", ";
    }
    cout << '}';
    return cout;
}

int main()
{
    twodoubles v = {{0.0, 0.1}, {0.2, 0.3}, {0.4, 0.5}};
    std::cout << v << '\n';

    v = {{1.1, 2.2}};
    std::cout << v << '\n';

    std::cout << (twodoubles {{0.0, 0.1}, {0.2, 0.3}, {0.4, 0.5}}) << '\n';
    
    return 0;
}'''.strip()
f = open('main.cpp', 'w'); f.write(s); f.close()
\end{python}
\VerbatimInput[frame=single,fontsize=\footnotesize]{main.cpp}
Here's the output:
\begin{python}
from latextool_basic import *
print(r'{\footnotesize %s}' % shell('g++ main.cpp; ./a.out'))
\end{python}
For
\begin{console}[fontsize=\footnotesize,frame=single]
twodoubles v = {{0.0, 0.1}, {0.2, 0.3}, {0.4, 0.5}};
\end{console}
without using typedefs, it would become
\begin{console}[fontsize=\footnotesize,frame=single]
std::vector< std::pair< double, double > > v
    = {{0.0, 0.1}, {0.2, 0.3}, {0.4, 0.5}};
\end{console}
Without the inner initializer list, it becomes
\begin{console}[fontsize=\footnotesize,frame=single]
std::vector< std::pair< double, double > > v
    = {std::pair< double, double >(0.0, 0.1),
       std::pair< double, double >(0.2, 0.3),
       std::pair< double, double >(0.4, 0.5)};
\end{console}
i.e., initializer lists have to be replaced by explicit
constructor calls.
And without the outer initializer list it becomes
\begin{console}[fontsize=\footnotesize,frame=single]
std::vector< std::pair< double, double > > v;
v.push_back(std::pair< double, double >(0.0, 0.1));
v.push_back(std::pair< double, double >(0.2, 0.3)),
v.push_back(std::pair< double, double >(0.4, 0.5));
\end{console}
which is truly horrific.

By the way in the above, just like the case for \verb!std::vector!
I did this:
\begin{console}[frame=single, fontsize=\footnotesize]
twodoubles v = {{0.0, 0.1}, {0.2, 0.3}, {0.4, 0.5}};
\end{console}
I can also do this:
\begin{console}[frame=single, fontsize=\footnotesize]
twodoubles v {{0.0, 0.1}, {0.2, 0.3}, {0.4, 0.5}};
\end{console}

Finally the initializer notation can be used in a constructor call
to create an object value without assigning it to a name:
\begin{console}[frame=single, fontsize=\footnotesize]
std::cout << (twodoubles {{0.0, 0.1}, {0.2, 0.3}, {0.4, 0.5}}) << '\n';
\end{console}
In the same way you can do
\begin{console}[frame=single, fontsize=\footnotesize]
std::cout << (std::vector< int > {2, 3, 5}) << '\n';
\end{console}
which is very convenient for short tests:
\begin{console}[frame=single, fontsize=\footnotesize]
std::cout << somesort(std::vector< int > {2, 5, 3}) << '\n';
\end{console}
