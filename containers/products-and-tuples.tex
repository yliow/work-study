%-*-latex-*-
\sectionthree{\cpp\ STL: \texttt{std::pair} and \texttt{std::tuple}}
\begin{python0}
from solutions import *; clear()
\end{python0}

\cpp\ provides two STL classes for tuples:
\verb!std::pair!\index{std::pair@\texttt{std::pair}}
for 2-tuples
and
\verb!std::tuple!\index{std::tuple@\texttt{std::tuple}}
for general $k$--tuples.

Each \verb!std::pair! object is made up of two values which can be of
different types.
If \verb!x! is a \verb!std::pair! object, then the two values are
\verb!x.first!\tinysidebar{\texttt{first \\ second}}{\index{first@\texttt{first}}
and
\verb!x.second!{\index{second@\texttt{second}}.
In other words, \verb!std::pair! is basically just this:
\begin{console}[fontsize=\footnotesize]
template < typename S, typename T >
class pair
{
public:
    ...
    S first;
    T second;
};
\end{console}
(Actually \verb!std::pair! is a \verb!struct!.)

Each \verb!std::tuple! object is made up of two or more
values which can be of different types.
The number of value is fixed.
If \verb!x! is a \verb!std::tuple! object,
then the values are
\verb!std::get<0>(x)!\tinysidebar{\texttt{std::get}}\index{std::get@\texttt{std::get}},
\verb!std::get<1>(x)!,
\verb!std::get<2>(x)!, etc.

Run the following example code:
\begin{python}
s = r'''
#include <iostream>
#include <set>
#include <utility> // for std::pair
#include <tuple>

template< typename S, typename T >
std::ostream & operator<<(std::ostream & cout,
                          const std::pair< S, T > & x)
{
    cout << '(' << x.first << ", " << x.second << ')';
    return cout;
}

template< typename S, typename T, typename U >
std::ostream & operator<<(std::ostream & cout,
                          const std::tuple< S, T, U > & x)
{
    cout << '('
         << std::get<0>(x) << ", "
         << std::get<1>(x) << ", "
         << std::get<2>(x) << ')';
    return cout;
}

int main()
{
    std::pair< int, double > u = {2, 3.14159};
    std::cout << u << '\n';
    u.first = 1;
    u.second = 2.718281;
    std::cout << u << '\n';
    
    std::tuple< int, double, char > v = {-2, 0.01, 'A'};
    std::cout << v << '\n';
    std::get<0>(v) = -1;
    std::get<1>(v) = -4.2;
    std::get<2>(v) = 'B';
    std::cout << v << '\n';
    
    return 0;
}
'''.strip()
f = open('main.cpp', 'w'); f.write(s); f.close()
\end{python}
\VerbatimInput[frame=single, fontsize=\small]{main.cpp}
\begin{python}
from latextool_basic import *
print(r'{\small %s}' % shell('g++ main.cpp; ./a.out'))  
\end{python}
\begin{python}
import os; os.system('rm main.cpp')
\end{python}
Of course a \verb!std::pair! is just a special case of
\verb!std::tuple!.

It's obvious that the length of the tuple in the above
example is fixed at 3.
If you want to have a tuple of arbitrary length you
should not use \verb!std::tuple!.
If the values in your tuple have the same type but you
want the length to be arbitrary, then you should use
a \verb!std::vector!.
Of course a \verb!std::vector! is homogeneous:
the type of the values must be the same.
If you want to have a \verb!std::vector! of \lq\lq different types"
of objects, say from class
\verb!A!,
\verb!B!,
\verb!C!,
you can create a parent class \verb!P! for 
\verb!A!,
\verb!B!,
\verb!C!,
and use
\verb!std::vector< P * >! and polymorphism (see CISS245).
