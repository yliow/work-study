%-*-latex-*-
\sectionthree{Static unsorted array}
\begin{python0}
from solutions import *; clear()
\end{python0}

I will say \lq\lq static array'' to mean arrays with memory
allocated in the frame, i.e. not in the heap,
because the amount of memory used for such things are fixed
while your array is in scope.

Here's an example:
\begin{console}[fontsize=\footnotesize]
const int N = 1000;
int x[N];
\end{console}

If \verb!T! is a type (\verb!int!, \verb!double!, \verb!bool!,
structure, class, ...), you do the same thing:
\begin{console}[fontsize=\footnotesize]
const int N = 1000;
T x[N];
\end{console}
The size of the array (see \verb!N!) is usually a constant
although some C/\cpp\ compilers allow you to use a variable.
But in any case, after the array is requested, you cannot
change the size of your \verb!x!:
\begin{console}[fontsize=\footnotesize]
const int N = 1000;
T x[N];
...
gimme_more(x, 1000000); // NOPE!!!
\end{console}

This is probably the simplest container.
Here are some of the things you can do with your array.

If you want to simulate a container containing different number of
items in the container, you have to include a variable that
records the number of things already placed in the container:
\begin{console}[fontsize=\footnotesize]
const int N = 1000;
T x[N];
int n = 0; // number of things in x, i.e., only x[0],...,
           // x[n-1] are considered values in the 
           // container. The values x[n],...,x[N-1] 
           // should be considered extra unused space.
\end{console}

When you package that into a class, it would look like this:
\begin{console}[frame=single,fontsize=\footnotesize]
template < typename T >
class Array
{
private:
    static const int N = 1000;
    T x[N];
    int n;
};
\end{console}


\newpage
\subsection{Insert}

If you need to insert a value at index position \verb!i!,
then you have to move the values at positions
\verb!i!, ..., \verb!n - 1! to positions
\verb!i + 1!, ..., \verb!n! 
to make an \lq\lq empty space'' for the new value.
If \verb!i! is 0, that requires moving lots of values.
If \verb!i! is \verb!n - 1!, then only one value has to be moved.
If \verb!i! is \verb!n!, the value is to be placed just one step past the
last value in the array -- you don't have to move anything.
Here's the picture of insert at index 2 where \verb!n! is 5
and the capacity (size) is 10:

\begin{python}
from latextool_basic import *
p = Plot()
c = RectContainer(x=0, y=0)
for i,x in enumerate('26453?????'):
        c += Rect2(x0=0, y0=0, x1=0.6, y1=0.6, label=r'{\texttt{%s}}' % x)
                 
p += c

x,y = c[2].bottomleft(); y = y + 2
s = Rect2(x0=x, y0=y, x1=x+0.6, y1=y+0.6, label=r'{\texttt{%s}}' % '8')
p += s
p += Line(points=[s.bottom(), c[2].top()], endstyle='>')
print(p)
\end{python}

The values from index values \verb!2! to \verb!4!
are moved (i.e., copied) to the right by one step:

\begin{python}
from latextool_basic import *
p = Plot()
c = RectContainer(x=0, y=0)
for i,x in enumerate('264453????'):
        c += Rect2(x0=0, y0=0, x1=0.6, y1=0.6, label=r'{\texttt{%s}}' % x)
                 
p += c

x,y = c[2].bottomleft(); y = y + 2
s = Rect2(x0=x, y0=y, x1=x+0.6, y1=y+0.6, label=r'{\texttt{%s}}' % '8')
p += s
p += Line(points=[s.bottom(), c[2].top()], endstyle='>')
print(p)
\end{python}

and then \verb!8! (the new value) is placed at index \verb!2!:


\begin{python}
from latextool_basic import *
p = Plot()
c = RectContainer(x=0, y=0)
for i,x in enumerate('268453????'):
        c += Rect2(x0=0, y0=0, x1=0.6, y1=0.6, label=r'{\texttt{%s}}' % x)
                 
p += c

print(p)
\end{python}

Of course after the above is done, you have to increment \verb!n!,
the count of number of things in your array.
\begin{console}
for (int j = n - 1; j >= i; j--)
{ 
    x[j + 1] = x[j];
}
x[i] = newvalue;
n++;
\end{console}
The time taken is
\[
A + B(n - i)
\]
for constants $A$ and $B$.

Let me call the first value of the array the \defone{head}.
If a value is to be inserted at index 0, I will say that
I'm \lq\lq inserting at the head'' (to be precise,
I'm inserting before the current head -- the new value becomes
the head of the new array).
Pictorially, this is what happens:
I have an array of values $2,6,4,5,3$ and I want to insert an 8
at the head:

\begin{python}
from latextool_basic import *
p = Plot()
c = RectContainer(x=0, y=0)
for i,x in enumerate('126453'):
    if i == 0:
        c += Rect2(x0=0, y0=0, x1=0.6, y1=0.6, label=r'{\texttt{%s}}' % '',
             linecolor='white')
    else:
        c += Rect2(x0=0, y0=0, x1=0.6, y1=0.6, label=r'{\texttt{%s}}' % x)
                 
p += c

s = Rect2(x0=0, y0=-2, x1=0.6, y1=-2+0.6, label=r'{\texttt{%s}}' % '8')
p += s
p += Line(points=[s.top(), c[0].bottom()], endstyle='>')
print(p)
\end{python}

After doing that I get this:

\begin{python}
from latextool_basic import *
p = Plot()
c = RectContainer(x=0, y=0)
for i,x in enumerate('826453'):
        c += Rect2(x0=0, y0=0, x1=0.6, y1=0.6, label=r'{\texttt{%s}}' % x)        
p += c
print(p)
\end{python}

Let me call the last value of the array the \defone{tail}, i.e.
at index $n-1$.
If a value is to be inserted at index $n-1$, I will say that
I'm \lq\lq inserting a tail'' (to be precise, I mean
inserting after the current tail -- it becomes the tail of the new array).
Pictorially, this is what happens:
I have an array of values $4,2,3,1,5$ and I want to insert an 8
at the head:

\begin{python}
from latextool_basic import *
p = Plot()
c = RectContainer(x=0, y=0)
for i,x in enumerate('42315'):
    c += Rect2(x0=0, y0=0, x1=0.6, y1=0.6, label=r'{\texttt{%s}}' % x)
                 
p += c

x0,y0 = c[-1].bottomright()
s = Rect2(x0=x0, y0=y0-2, x1=x0 + 0.6, y1=y0-2+0.6, label=r'{\texttt{%s}}' % '8')
p += s
x,y = s.top()
p += Line(points=[(x,y),(x,y+1)], endstyle='>')
print(p)

\end{python}

After that I get

\begin{python}
from latextool_basic import *
p = Plot()
c = RectContainer(x=0, y=0)
for i,x in enumerate('423158'):
    c += Rect2(x0=0, y0=0, x1=0.6, y1=0.6, label=r'{\texttt{%s}}' % x)          
p += c
print(p)
\end{python}

Inserting at the head takes the most time.
Big-O-wise, insert at the head has a runtime of
\[
O(n)
\]

Inserting at the tail is the fastest since the runtime is
\[
O(1)
\]


Averaging over all possible $i = 0, 1, ..., n - 1$ and $i = n$ (the case
where you're inserting just beyond the array)
the average runtime is
\begin{align*} 
&\frac{1}{n + 1}  \left(
                    (A + Bn) + (A + B(n-1)) + \cdots + (A + B) + A  
             \right) \\
&= 
\frac{1}{n + 1} \left(
                   A(n + 1) + B(n + \cdots + 1)
            \right) \\
&= \frac{1}{n + 1} \left( A(n + 1) + B\frac{n(n+1)}{2} \right) \\
&= A + \frac{B}{2}n \\
&= \frac{B}{2} n + A
\end{align*}
I've used the formula:
\[
1 + 2 + \cdots + n = \frac{n(n + 1)}{2}
\]
OK ... now we know that the average time taken to insert is $O(n)$.
(Don't recall the big-O notation and 
all that algorithmic analysis goodies? 
Don't panic. 
Just check the relevant notes/chapters.)
Of course I'm assuming in the above that 
all index values are equally likely.

Here are runtime complexities for insert:
\begin{tightlist}
  \li Insert head: $O(n)$
  \li Insert tail: $O(1)$
  \li Average insert: $O(n)$
\end{tightlist}
(assuming the array is not full.)

\newpage
\subsection{Delete}

If you want to remove a value, say it's at index position \verb!i!,
you have to move the value at \verb!i + 1!, ..., \verb!n - 1!
to the \lq\lq left'' by one position to index positions
\verb!i!, ..., \verb!n - 2!.
After that's done, you have to decrement your \verb!n!.
The average and worse runtime is again $O(n)$.
The best case runtime is $O(1)$.

Here are the runtime complexities for delete:
\begin{tightlist}
  \li Delete head: $O(n)$
  \li Delete tail: $O(1)$
  \li Average delete: $O(n)$
\end{tightlist}



\newpage
\subsection{Search}

What about search (or find)?
In the worse case, you have to scan the whole array.
Which means that the time taken is $O(n)$.
The average case is also $O(n)$.
The best case runtime is $O(1)$.

Now suppose we sort the values in 
\verb!x[0]!,...,
\verb!x[n - 1]!
(say in ascending order).
Depending on the sorting algorithm used
(see relevant notes)
the best time for sorting is 
$O(n \lg n)$.
The benefit of having a sorted array is that
search time using binary search takes
\[
O(\lg n)
\]
(see chapter on algorithmic analysis)
which is better than $O(n)$.
If you want to ensure that the array is always sorted,
when you insert a value in the array, you (of course)
don't have to specify where to insert the value but make sure 
that the value is inserted so that the list remains sorted.
How much time do you need?
Well ... first you do a binary search.
If it's found, you can just insert at that place.
You would have a duplicate value - if that's allowed in your list.
If it's not found, a $-1$ is returned.
Now you call a version of the binary search that returns
the index of where a new value should go instead of $-1$.
(This modified binary search is not difficult to write.)
You performed two binary searches.
So the time taken is $O(2\lg n) = O(\lg n)$.
Then you perform the insert, taking $O(n)$ time.
So all in all, insert also takes $O(n)$ time to insert.
Of course instead of performing two binary searches, you 
can just do a linear search and perform the insert.
All in all, you would still get $O(n)$ for runtime.

For delete, say you specify the value to delete
and not the index.
In this case, you again need to search for the index
where the value occurs 
(taking $O(\lg n)$ time using binary search).

\begin{longtable}{|l|c|c|c|}
\hline
Unsorted array & Best   & Average      & Worse  \\*
\hline\hline
Insert         & $O(1)$ & $O(n)$       & $O(n)$ \\*
Delete         & $O(1)$ & $O(n)$       & $O(n)$ \\*
Search         & $O(1)$ & $O(n)$       & $O(n)$ \\* 
\hline
\end{longtable}


Note that in many cases the best case runtime is not that useful.
The average and worse is more important.
When you build a system, you want to focus on the worse case
scenario, right?
The best case scenario probably does not occur frequently
and it would be overly optimistic to focus only on that.

But what if we're in a situation where the best case always occurs?
Maybe the program you're trying to build requires only insert
and delete at one end of the array.
In the case of the (unsorted) array, the best case for insert and delete
is pretty fast if you insert and delete at the tail end
of the array.
Remember that!!!
(See a later chapter on stacks, queues, etc. implemented using arrays.)

There's one thing to note about the array:
You can very quickly get to the $i$-th value in the array.
The time taken is the same (pretty much) regardless of the value of $i$.
You will see later that there are containers where going to the 
$i$--th value of the container is pretty costly.
So this is a plus point for arrays.



\newpage
\subsection{Capacity issue}

The \textit{BAD} thing about the static array (and you absolutely have to 
remember this) is that the maximum size 
(or capacity if you like) of the container is fixed
and cannot be changed during runtime.
If I declare an array with \verb!N! $= 1000$ values
and I keep adding things to this array, ultimately
my \verb!n! will reach $1000$ and I won't be able to 
add more stuff into the array.
You might think ... \lq\lq Well ... I'm going to start with 100000
then!''
True, true, true.
But what if 1000000000 things were intended to be put into the array?
Another thing is this:
What if there's a scenario where the number of things needed
is actually $100$?
You would be wasting lots of memory.
It's just difficult to write flexible software to handle different situations
if you're using static arrays.
So the static'ness of the memory usage of static arrays is a very
serious disadvantage.


\begin{ex}
Write a template \verb!Array! class with the following 
(public! ... right?) methods:
\begin{console}[fontsize=\footnotesize]
void      insert(int index, T key);
void      remove(int index);
int       find(const T & key); // returns index or -1 (if not found)
const T & operator[](int index) const;
T &  operator[](int index);
\end{console}
Allow duplicate values in an \verb!Array! object.
Also, write a template function to print the contents of your object:
\begin{console}[fontsize=\footnotesize]
template < class T >
std::ostream & operator<<(std::ostream & cout,  const Array &);
\end{console}
If the \verb!Array< int >! object contains 1, 2, 3, then printing it will
give you
\begin{console}[fontsize=\footnotesize]
[1, 2, 3]
\end{console}
Set the maximum capacity of your \verb!Array! objects to 1000000.
Remember my advice in class:
Always write a non-template version first!!!
Collect timing data for the following scenarios for 
\verb!Array< int >! objects:
\begin{itemize}
\item What is the time taken to insert a value when
$n = 100000$, $200000$, $300000$, $400000$, ..., $1000000$?
\item What is the time taken to delete the value at a given index position
when
$n = 100000$, $200000$, $300000$, $400000$, ..., $1000000$?
\item What is the time taken to search for a value when
$n = 100000$, $200000$, $300000$, $400000$, ..., $1000000$?
\item What is the time taken to access a value at an index position when
$n = 100000$, $200000$, $300000$, $400000$, ..., $1000000$?
\end{itemize}
Make a plot.
For your experiments, make sure your array contains distinct
values.
\qed
\end{ex}



\begin{ex}
Write methods to return the size of the array, boolean functions
to tell you if the array is empty or full (i.e., \verb!n! has
reached the maximum capacity.)
Also, throw some exceptions objects when performing invalid deletes
or inserts.
\end{ex}


We can abstract the situation and think of an \verb!Array< T >!
object as containing \verb!T! values laid out on a straight line.
The supporting methods are
\begin{console}[fontsize=\footnotesize]
list.insert(i, v): insert v into list at index position i
list.remove(v)   : remove one value v from list
list.remove(i)   : remove the value at index position i
list.find(v)     : return the index where v occurs (left-to-right)
                   or return -1 if not found
list[i]          : return the value at index i
list[i] = v      : change value at index i to v
list.size()      : return the size of list
list.is_empty()  : return true iff list is empty
list.is_full()   : return true iff list is full
\end{console}

Note that when you describe a container and the operations available
on the container
(or using OO-speak ... what the container can do)
without referring to the implementation (or even the programming language
used), you're describing an \defone{abstract data type (ADT)}.

Technically speaking, the above description uses OO syntax
and there are language out there which are non-OO.
We would have to be even more free-form to call our description
the description of an ADT.
However there's no standardization ADT description
just like there's no standardization of pseudocode language (!!!)

If I have to describe the above using a non-OO language say a 
procedural/functional
language I would have to write:
\begin{console}[fontsize=\footnotesize]
insert(list, i, v): insert v into list at index position i
remove(list, v)   : remove one value v from list
remove(list, i)   : remove the value at index position i
find(list, v)     : return the index where v occurs (left-to-right)
                    or return -1 if not found
get(list, i)      : return the value at index i
set(list, i, v)   : change value at index i to v
size(list)        : return the size of list
is_empty(list)    : return true iff list is empty
is_full(list)     : return true iff list is full
\end{console}
Unfortunately there are others who feel the above is not
free-form enough.

I'm not going to argue with the language lawyers if one should
describe ADT using a pseudo-OO language or
pseudo-procedural/functional
language. This is how I'm going to describe our unordered list ADT
... and that's that:
\begin{console}[fontsize=\footnotesize]
ADT: Unordered list
list.insert(i, v): insert v into list at index position i
list.remove(v)   : remove one value v from list
list.remove(i)   : remove the value at index i from list
list.find(v)     : return the index where v occurs (left-to-right)
                   or return -1 if not found
list[i]          : return the value at index i
list[i] = v      : change value at index i to v
list.size()      : return the size of list
list.is_empty()  : return true iff list is empty
list.is_full()   : return true iff list is full
\end{console}

In general when you have a container where the value
is abstractly laid out in a straight line and you have operations
to insert, delete, etc on the list, you have 
the ADT called a \defone{list}.
Specifically, a list is either empty or among the values in the list
there is one that is the head of the list and there is one that is the tail.
(The head can be the tail.)
Associated to every value in the list that is not a tail, there is the
concept
\lq\lq next" value.

If the values in the list is not ordered in any way we say that
it is
an \defone{unordered list}.

Our \verb!Array< T >! is an implementation of the unordered list.
The language used is of course \cpp\ and the implementation uses
the
C/\cpp\ static array.

If the list is ordered, we say that the ADT is an \defone{ordered list}.
Here's the ADT. In the next section, I'll talk about
the sorted array which is an example of a sorted list.
\begin{console}[fontsize=\footnotesize]
ADT: Ordered list
list.insert(v)   : insert v into list
list.remove(v)   : remove one value v from list
list.remove(i)   : remove the value at index i from list
list.find(v)     : return the index where v occurs (left-to-right)
                   or return -1 if not found
list[i]          : return the value at index i
list[i] = v      : change value at index i to v
list.size()      : return the size of list
list.is_emspty()  : return true iff list is empty
list.is_full()   : return true iff list is full
\end{console}

Note that the insert method for a sorted list does not require an index value since
the list will insert \verb!v! in the right place since
the list is ordered.
    
The \verb!SortedArray< T >! is an implementation of the order list.
The language used is again \cpp\ and the static array is again used.
