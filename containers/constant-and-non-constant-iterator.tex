%-*-latex-*-
\sectionthree{\cpp\ STL: iterators and constant iterators}
\begin{python0}
from solutions import *; clear()
\end{python0}

Implementing an iterator class is easy: refer to CISS245 if you need help.
In particular, look at the notes on the \verb!IntPointer! class.
Specifically, your personal vector class should look like this:
\begin{console}[frame=single,fontsize=\footnotesize]
template < typename T >
class vector
{
public:
    class iterator
    {
    private:
        T * q_; // points to a T value in the array that p_
                // points to
    };
private:
    T * p_; // points to an array in the free store
};
\end{console}

In the case of
\begin{console}[fontsize=\footnotesize]
for (typename std::vector< int >::iterator p = v.begin();
     p != v.end(); ++p)
{
     ... do something with (*p) ...
}
\end{console}
The iterator \verb!p! can modify the value that it points to:
\begin{console}[fontsize=\footnotesize]
for (std::vector< int >::iterator p = v.begin();
     p != v.end(); ++p)
{
     (*p) = 0;
}
\end{console}
In case you do \textit{not} want that to happen, or that you cannot do this
(for instance \verb!v! is a constant and therefore you cannot
change the values in \verb!v!), you do this:
\begin{console}[fontsize=\footnotesize]
for (typename std::vector< int >::const_iterator p = v.begin();
     p != v.end(); ++p)
{
     ... only read access to (*p), not write access ...
}
\end{console}
i.e., \verb!p! is a \defone{constant iterator}.


\begin{ex} 
  \label{ex:some-decision1}
  \tinysidebar{\debug{exercises/{empty0/question.tex}}}
  \solutionlink{sol:some-decision1}
  \qed
\end{ex} 
\begin{python0}
from solutions import *
add(label="ex:some-decision1",
    srcfilename='exercises/some-decision1/answer.tex') 
\end{python0}



\begin{ex} 
  \label{ex:some-decision1}
  \tinysidebar{\debug{exercises/{empty0/question.tex}}}
  \solutionlink{sol:some-decision1}
  \qed
\end{ex} 
\begin{python0}
from solutions import *
add(label="ex:some-decision1",
    srcfilename='exercises/some-decision1/answer.tex') 
\end{python0}


You \textit{should} write a complete
template vector class that has supporting
iterators (constant and nonconstant).
