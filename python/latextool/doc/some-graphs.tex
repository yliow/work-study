\subsection{\texttt{Graph} class}


\begin{python}
s = r'''
from latextool_basic import *
p = Plot()
for i in [1,2,3]:
    p += Graph.node(p, x=i, y=1, name='a%s' % i)
    p += Graph.node(p, x=i, y=-1, name='c%s' % i)

for i in range(5):
    p += Graph.node(p, x=i, y=0, name='b%s' % i)

p += Graph.edge(names=['b%s' % i for i in range(5)])
for i in [1,2,3]:
    p += Graph.edge(names=['a%s' % i, 'b%s' % i])
    p += Graph.edge(names=['c%s' % i, 'b%s' % i])

p += Graph.arc(names=['c3', 'b4'])

print(p)
'''.strip()
from latextool_basic import *
print(console(s))
exec(s)
\end{python}




\newpage
\subsection{Edge label anchor}
\begin{python}
s = r'''
from latextool_basic import *
p = Plot()
p += Graph.node(x=0, y=0, label='a', name='0')
p += Graph.node(x=5, y=0, label='b', name='1')
p += Graph.edge(names=['0','1'], label='d', anchor='above')

print(p)
'''.strip()

from latextool_basic import *
print(console(s))
execute(s, print_source=True)
\end{python}




\newpage
\subsection{Cycle graphs}

Odd Cycle.

The node name is of the form \verb![radius]_[i]!
where \verb!i! is the index of node going in an anticlockwise
direction, starting at degree \verb!startdegree!.

By default, the first node is at \verb!startdegree=90! degrees.

\begin{python}
s = r'''
from latextool_basic import *
p = Plot()
cyclegraph(p, num=3, radius=1)
print(p)'''.strip()

from latextool_basic import *
print(console(s))
exec(s)
\end{python}


\begin{python}
s = r'''
from latextool_basic import *
p = Plot()
cyclegraph(p, num=7, radius=2, drawline=False)
print(p)'''.strip()

from latextool_basic import *
print(console(s))
exec(s)
\end{python}




\newpage
\subsection{Name of nodes}

Name of the nodes:

The nodes are names
\verb!'0'!,
\verb!'1'!, etc. in the anticlockwise direction where \verb!'0'!
is the node with angle \verb!startdegree!.
You can specify the \verb!startdegree! in the \verb!cyclegraph!.
By default, when the number of nodes is even, the \verb!startdegree! is 0
and when the number of nodes is odd, the \verb!startdegree! is 90.

\begin{python}
s = r'''
from latextool_basic import *
p = Plot()
cyclegraph(p, num=7, radius=1, drawline=True)
p += Circle(x=5, y=2, r=0.25, background='red', name='A')
p += Circle(x=0, y=-2, r=0.25, background='blue', name='B')
p += Line(names=['A', '0'])
p += Line(names=['B', '3'])
print(p)'''.strip()

from latextool_basic import *
print(console(s))
exec(s)
\end{python}


You can change the pgf/tikz name of the nodes by
setting \verb!names! dictionary in the \verb!cyclegraph!.
For instance if you want the first node to be named \verb!'a'!
instead of \verb!0!, then \verb!0:'a'! should be a key-value pair
in \verb!names!.

\begin{python}
s = r'''
from latextool_basic import *
p = Plot()
cyclegraph(p, num=7, radius=1, drawline=True,
           names={0:'a',
                  3:'d',
                  })
p += Circle(x=5, y=2, r=0.25, background='red', name='A')
p += Circle(x=0, y=-2, r=0.25, background='blue', name='B')
p += Line(names=['A', 'a'])
p += Line(names=['B', 'd'])
print(p)'''.strip()

from latextool_basic import *
print(console(s))
exec(s)
\end{python}


\newpage
Even Cycle.
By default, the starting degree is 0.

\begin{python}
s = r'''from latextool_basic import *
p = Plot()
cyclegraph(p, num=4, radius=1)
print(p)
'''.strip()

from latextool_basic import *
print(console(s))
exec(s)
\end{python}
  
\begin{python}
s = r'''from latextool_basic import *
p = Plot()
cyclegraph(p, num=4, radius=1, startdegree=45)
print(p)
'''.strip()

from latextool_basic import *
print(console(s))
exec(s)
\end{python}
  

\begin{python}
s = r'''from latextool_basic import *
p = Plot()
cyclegraph(p, num=6, radius=2)
print(p)
'''.strip()

from latextool_basic import *
print(console(s))
exec(s)
\end{python}




\newpage
\subsection{Petersen graph}
\begin{python}
s = r'''
from latextool_basic import *
p = Plot()
petersen(p)
print(p)
'''.strip()

from latextool_basic import *
print(console(s))
exec(s)
\end{python}




\newpage
\subsection{Complete graphs}

$K_4$:
\begin{python}
s = r'''
from latextool_basic import *
p = Plot()
completegraph(p, num=4)
print(p)
'''.strip()
from latextool_basic import *
print(console(s))
exec(s)
\end{python}

$K_4$:
\begin{python}
s = r'''from latextool_basic import *
p = Plot()
completegraph(p, num=4, startdegree=45)
print(p)
'''.strip()
from latextool_basic import *
print(console(s))
exec(s)
\end{python}


$K_5$:
\begin{python}
s = r'''
from latextool_basic import *
p = Plot()
completegraph(p, num=5)
print(p)
'''.strip()
from latextool_basic import *
print(console(s))
exec(s)
\end{python}
  
$K_6$:
\begin{python}
s = r'''
from latextool_basic import *
p = Plot()
completegraph(p, num=6, radius=1.5)
print(p)
'''.strip()
from latextool_basic import *
print(console(s))
exec(s)
\end{python}
  



\newpage
\subsection{Star graphs}

The name of the center node is \verb!'center'!.

\begin{python}
s = r'''
from latextool_basic import *
p = Plot()
stargraph(p, num=5)
print(p)
'''.strip()
from latextool_basic import *
print(console(s))
exec(s)
\end{python}


\begin{python}
s = r'''
from latextool_basic import *
p = Plot()
stargraph(p, num=4)
print(p)
'''.strip()
from latextool_basic import *
print(console(s))
exec(s)
\end{python}


When there are two stars, you need to give the centers different names:
\begin{python}
s = r'''
from latextool_basic import *
p = Plot()
stargraph(p, x=0, y=0, num=4, radius=1, names={'center':'c0'})
stargraph(p, x=2, y=0, num=4, radius=1, names={'center':'c1'})
print(p)
'''.strip()
from latextool_basic import *
print(console(s))
exec(s)
\end{python}





\newpage
\subsection{Bipartite graphs}

$K_{3,3}$:
\begin{python}
s = r'''
from latextool_basic import *
p = Plot()
completebipartite(p=p, num1=3, num2=3)
print(p)
'''.strip()

from latextool_basic import *
exec(s)
print(console(s))
\end{python}


$K_{3,3}$ without edges:
\begin{python}
s = r'''
from latextool_basic import *
p = Plot()
completebipartite(p=p, num1=3, num2=3, drawline=False)
print(p)
'''.strip()

from latextool_basic import *
print(console(s))
exec(s)
\end{python}

$K_{3,4}$:
\begin{python}
s = r'''
from latextool_basic import *
p = Plot()
completebipartite(p=p, num1=3, num2=4, horsep=2)
print(p)
'''.strip()

from latextool_basic import *
print(console(s))
exec(s)
\end{python}


$K_{3,4}$:
\begin{python}
s = r'''
from latextool_basic import *
p = Plot()
completebipartite(p=p, num1=3, num2=4, versep=2)
print(p)
'''.strip()

from latextool_basic import *
print(console(s))
exec(s)
\end{python}

