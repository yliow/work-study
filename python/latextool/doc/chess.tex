\subsection{Chess}

The chess function read image data from image files stored at
\verb!data/chess/png/!.
The names of the files are \verb!bp.png! for black pawn
and \verb!wp.png! for white pawn. Etc.
The directory \verb!data/! must be in the same directory as the directory
containing \verb!latextool_basic.py!.

[Note that the chess images for pawns are slightly to the right.]


\begin{python}
s = r'''
from latextool_basic import *
p = Plot()

xs = [list('rnbqkbnr'),
      list('pppppppp'),
      list('        '),
      list('        '),
      list('        '),
      list('        '),
      list('PPPPPPPP'),
      list('RNBQKBNR'),
      ]
C = chess(p, x=0, y=0, xs=xs)

p += Line(points=[C[6][4].center(), C[4][4].center()], linecolor='red',
          linewidth=0.1, endstyle='>')

print(p)
'''.strip()
from latextool_basic import *
print(console(s))
execute(s)
\end{python}



\begin{python}
from latextool_basic import *
p = Plot()

xs = [list('        '),
      list('        '),
      list('        '),
      list('        '),
      list('        '),
      list('        '),
      list('    P   '),
      list('        '),
      ]
      
C = chess(p, x=0, y=0, xs=xs, WIDTH=0.5)

p += Line(points=[C[6][4].center(), C[4][4].center()], linecolor='red',
          linewidth=0.05, endstyle='>')

print(p)
\end{python}


\begin{python}
from latextool_basic import *
p = Plot()

xs = {(5,7):'b',
      (5,5):'R',
      }
      
C = chess(p, x=0, y=0, xs=xs, WIDTH=0.3)

p += Line(points=[C[5][5].center(), C[5][7].center()], linecolor='red',
          linewidth=0.04, endstyle='>')

print(p)
\end{python}

\begin{python}
from latextool_basic import *
p = Plot()

xs = {(5,7):'b',
      (5,5):'R',
      }
      
C = chess(p, x=0, y=0, xs=xs, WIDTH=0.2)

p += Line(points=[C[5][5].center(), C[5][7].center()], linecolor='red',
          linewidth=0.02, endstyle='>')

print(p)
\end{python}
