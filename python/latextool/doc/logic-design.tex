\newpage
\section{Logic design}

\subsection{AND gate}
\begin{python}
s = r'''
from latextool_basic import *
from latexcircuit import *

p = Plot()
g = AND_GATE(x=0, y=0, inputs=2)
p += str(g)

for x,y in g.inputs():
    p += Line(points=[(x - 1, y), (x, y)])

x,y = g.output()
p += Line(points=[(x, y), (x + 1,y)])

p += Grid(x0=-2, y0=-1, x1=2, y1=1)
print(p)
'''

from latextool_basic import *
exec(s)
print(console(s.strip()))
\end{python}





\newpage
\subsection{OR gate}
\begin{python}
s = r'''
from latextool_basic import *
from latexcircuit import *

p = Plot()
g = OR_GATE()
p += str(g)

for x,y in g.inputs():
    p += Line(points=[(x - 1, y), (x, y)])

x,y = g.output()
p += Line(points=[(x, y),(x+1,y)])

p += Grid(x0=-2, y0=-1, x1=2, y1=1)
print(p)
'''

from latextool_basic import *
exec(s)
print((console(s).strip()))
\end{python}


\newpage
\subsection{NOT gate}
\begin{python}
from latextool_basic import *
s = r'''
from latextool_basic import *
from latexcircuit import *

p = Plot()
g = NOT_GATE()
p += str(g)

x,y = g.input()
p += Line(points=[(x,y),(x - 1,y)])

x,y = g.output()
p += Line(points=[(x,y),(x + 1,y)])

p += Grid(x0=-2, y0=-1, x1=2, y1=1)
print(p)
'''

from latextool_basic import *
execute(s)
print((console(s).strip()))
\end{python}





\newpage
\subsection{NAND gate}

\begin{python}
s = r'''
from latextool_basic import *
from latexcircuit import *

p = Plot()
g = NAND_GATE()
p += str(g)

for x,y in g.inputs():
    p += Line(points=[(x-1,y),(x,y)])

x,y = g.output()
p += Line(points=[(x,y),(x+1,y)])

p += Grid(x0=-2, y0=-1, x1=2, y1=1)
print(p)
'''

from latextool_basic import *
execute(s)
print(console(s).strip())
\end{python}



\newpage
\subsection{NOR gate}
\begin{python}
s = '''
from latextool_basic import *
from latexcircuit import *

p = Plot()
g = NOR_GATE()
p += str(g)

for x,y in g.inputs():
    p += Line(points=[(x-1,y),(x,y)])

x,y = g.output()
p += Line(points=[(x,y),(x+1,y)])

p += Grid(x0=-2, y0=-1, x1=2, y1=1)
print(p)
'''

from latextool_basic import *
execute(s)
print(console(s).strip())
\end{python}



\newpage
\subsection{XOR gate}
\begin{python}
s = r'''
from latextool_basic import *
from latexcircuit import *

p = Plot()
g = XOR_GATE()
p += str(g)

for x,y in g.inputs():
    p += Line(points=[(x-1,y),(x,y)])

x,y = g.output()
p += Line(points=[(x,y),(x+1,y)])

p += Grid(x0=-2, y0=-1, x1=2, y1=1)
print(p)
'''
from latextool_basic import *
execute(s)
print(console(s).strip())
\end{python}


\newpage
\subsection{Orthogonal paths}
For drawing orthogonal paths between two points.
\begin{python}
s = r'''
from latextool_basic import *
from latexcircuit import *
p = Plot()
p += str(OrthogonalPath([(0,0), (5,1)]))
p += Grid(x0=-1, y0=-1, x1=6, y1=1)
print(p)
'''

from latextool_basic import *
execute(s)
print(console(s).strip())
\end{python}


Orthogonal path with arrow head:
\begin{python}
s = r'''
from latextool_basic import *
from latexcircuit import *
p = Plot()
p += str(OrthogonalPath([(0,0), (5,1)], endstyle='>', arrowstyle='triangle'))
p += Grid(x0=-1, y0=-1, x1=6, y1=1)
print(p)
'''

from latextool_basic import *
execute(s)
print(console(s).strip())
\end{python}



\newpage
Orthogonal paths -- no bend case:
\begin{python}
s = r'''
from latextool_basic import *
from latexcircuit import *
p = Plot()
p += str(OrthogonalPath([(0,0), (5,0)]))
p += Grid(x0=-1, y0=-1, x1=6, y1=1)
print(p)
'''

from latextool_basic import *
execute(s)
print(console(s).strip())
\end{python}


\newpage
with a positive horizontal shift of the bend:
\begin{python}
s = r'''
from latextool_basic import *
from latexcircuit import *
p = Plot()
p += str(OrthogonalPath([(0,0), (5,1)], shifts=[1]))
p += Grid(x0=-1, y0=-1, x1=6, y1=1)
print(p)
'''

from latextool_basic import *
execute(s)
print(console(s).strip())
\end{python}


with a negative horizontal shift of the bend:
\begin{python}
from latextool_basic import *
s = r'''
from latextool_basic import *
from latexcircuit import *
p = Plot()
p += str(OrthogonalPath([(0,0), (5,1)], shifts=[-1]))
p += Grid(x0=-1, y0=-1, x1=6, y1=1)
print(p)
'''

from latextool_basic import *
execute(s)
print((console(s).strip()))
\end{python}

TODO: shifts for vertical case


\newpage
orthogonal path, direction='vh' (vertical-horizontal):
\begin{python}
s = r'''
from latextool_basic import *
from latexcircuit import *
p = Plot()
p += str(OrthogonalPath(points=[(0,0), (5,1)], direction='vh'))
p += Grid(x0=-1, y0=0, x1=6, y1=1)
print(p)
'''

from latextool_basic import *
execute(s)
print(console(s).strip())
\end{python}



orthogonal path, direction='vh':
\begin{python}
s = r'''
from latextool_basic import *
from latexcircuit import *
p = Plot()
p += str(OrthogonalPath(points=[(0,1), (5,0)], direction='vh'))
p += Grid(x0=-1, y0=0, x1=6, y1=1)
print(p)
'''

from latextool_basic import *
execute(s)
print(console(s).strip())
\end{python}


\newpage
\subsection{Orthogonal path with gates}
\begin{python}
s = r'''
from latextool_basic import *
from latexcircuit import *
p = Plot()
AND = AND_GATE(x=0, y=0); p += str(AND)
OR = OR_GATE(x=5, y=1); p += str(OR)
opath = OrthogonalPath(gate0=AND, gate1=OR, input_index=0)
p += str(opath)
print(p)
'''

from latextool_basic import *
execute(s)
print(console(s).strip())
\end{python}




\newpage
\subsection{POINT}
POINT:
\begin{python}
s = r'''
from latextool_basic import *
from latexcircuit import *
p = Plot()
X = POINT(x=0, y=0, label='$x$')
p += str(X)
p += Grid(x0=-2, y0=-1, x1=4, y1=1)
print(p)
'''

from latextool_basic import *
execute(s)
print(console(s).strip())
\end{python}



\begin{python}
s = r'''
from latextool_basic import *
from latexcircuit import *
p = Plot()
X = POINT(x=0, y=0, label='$x$', anchor='north')
p += str(X)
p += str(OrthogonalPath([X.output(), (3,1)]))
p += Grid(x0=-2, y0=-1, x1=4, y1=1)
print(p)
'''
from latextool_basic import *
execute(s)
print(console(s).strip())
\end{python}




\newpage
\subsection{Linecolor for gate}
\begin{python}
s = r'''
from latextool_basic import *
from latexcircuit import *

p = Plot()
g0 = AND_GATE(linewidth=5, linecolor='blue')
p += str(g0)

print(p)
'''
from latextool_basic import *
execute(s)
print(console(s).strip())
\end{python}


\newpage
\subsection{Label for gate}
\begin{python}
s = r'''
from latextool_basic import *
from latexcircuit import *

p = Plot()
g0 = AND_GATE(linewidth=5, linecolor='blue', label='$z$')
p += str(g0)

print(p)
'''

from latextool_basic import *
execute(s)
print(console(s)().strip())
\end{python}






Test 2 inputs and output
\begin{python}
s = r'''
from latextool_basic import *
from latexcircuit import *

p = Plot()
g0 = AND_GATE(linewidth=5, linecolor='blue'); p += str(g0)

inputs = g0.inputs()

# Test input 0
x0,y0 = inputs[0]; p0 = (x0, y0)
x1,y1 = x0 - 2, y0; p1 = (x1, y1)
p += Line(points=[p0,p1], linewidth=0.1, linecolor='red')
x0 = x1 - 0.3
p += Rect(x0=x0, y0=y0, x1=x0, y1=y0, linewidth=0, s = '$x$')

# test input 1
x0,y0 = inputs[1]; p0 = (x0, y0)
x1,y1 = x0 - 1, y0; p1 = (x1, y1)
p += Line(points=[p0,p1], linewidth=0.1, linecolor='green')

# test output
x0,y0 = g0.output(); p0 = x0,y0
x1,y1 = x0 + 3, y0; p1 = x1,y1
p += Line(points=[p0,p1], linewidth=0.1, linecolor='black')

p += Grid()
print(p)
'''.strip()
from latextool_basic import *
execute(s)
print(console(s))
\end{python}


Test 3 inputs and output
\begin{python}
s = r'''
from latextool_basic import *
from latexcircuit import *

p = Plot()
g0 = AND_GATE(linewidth=5, linecolor='blue', inputs=3)
p += str(g0)

for x0,y0 in g0.inputs():
    p0 = (x0, y0)
    x1,y1 = x0 - 1, y0
    p1 = (x1, y1)
    p += Line(points=[p0,p1], linewidth=0.1, linecolor='red')

x0,y0 = g0.output()
p0 = (x0, y0)
x1,y1 = x0 + 1, y0
p1 = (x1, y1)
p += Line(points=[p0,p1], linewidth=0.1, linecolor='green')

p += Grid()
print(p)
'''
from latextool_basic import *
execute(s)
print(console(s).strip())
\end{python}
  





Test different size (height = 2)
\begin{python}
s = r'''
from latextool_basic import *
from latexcircuit import *

p = Plot()
g0 = AND_GATE(x=0, y=0, h=2,
              linewidth=5, linecolor='magenta', inputs=3)
p += str(g0)

for x0,y0 in g0.inputs():
    p0 = (x0, y0)
    x1,y1 = x0 - 1, y0
    p1 = (x1, y1)
    p += Line(points=[p0,p1], linewidth=0.1, linecolor='red')

x0,y0 = g0.output()
p0 = (x0, y0)
x1,y1 = x0 + 1, y0
p1 = (x1, y1)
p += Line(points=[p0,p1], linewidth=0.1, linecolor='green')

p += Grid()
print(p)
'''
from latextool_basic import *
execute(s)
print(console(s).strip())
\end{python}





Test height = 2, 4 inputs:
\begin{python}
s = r'''
from latextool_basic import *
from latexcircuit import *

p = Plot()
g0 = AND_GATE(x=0, y=0, h=2,
              linewidth=5, linecolor='blue', inputs=4)
p += str(g0)

for x0,y0 in g0.inputs():
    p0 = (x0, y0)
    x1,y1 = x0 - 1, y0
    p1 = (x1, y1)
    p += Line(points=[p0,p1], linewidth=0.1, linecolor='red')

x0,y0 = g0.output()
p0 = (x0, y0)
x1,y1 = x0 + 1, y0
p1 = (x1, y1)
p += Line(points=[p0,p1], linewidth=0.1, linecolor='green')

p += Grid()
print(p)
'''.strip()
from latextool_basic import *
execute(s)
print(console(s))
\end{python}


Test height = 2, 5 inputs:
\begin{python}
s = r'''
from latextool_basic import *
from latexcircuit import *

p = Plot()
g0 = AND_GATE(x=0, y=0, h=2,
              linewidth=5, linecolor='blue', inputs=5)
p += str(g0)

for x0,y0 in g0.inputs():
    p0 = (x0, y0)
    x1,y1 = x0 - 1, y0
    p1 = (x1, y1)
    p += Line(points=[p0,p1], linewidth=0.1, linecolor='red')

x0,y0 = g0.output()
p0 = (x0, y0)
x1,y1 = x0 + 1, y0
p1 = (x1, y1)
p += Line(points=[p0,p1], linewidth=0.1, linecolor='green')

p += Grid()
print(p)
'''.strip()
from latextool_basic import *
execute(s)
print(console(s))
#from latextool_basic import *
#execute(s, print_source=True)
\end{python}
  


OR gate:
\begin{python}
s = r'''
from latextool_basic import *
from latexcircuit import *

p = Plot()
g0 = OR_GATE(linewidth=2, linecolor='blue')
p += str(g0)

p += Grid(x0=-1,y0=-1,x1=3,y1=1)

print(p)
'''
from latextool_basic import *
execute(s)
print(console(s).strip())
\end{python}


\begin{python}
s = r'''
from latextool_basic import *
from latexcircuit import *

p = Plot()
g0 = OR_GATE(linewidth=2, linecolor='blue', label='$z$')
p += str(g0)

p += Grid(x0=-1,y0=-1,x1=3,y1=1)

print(p)
'''
from latextool_basic import *
execute(s)
print(console(s).strip())
\end{python}




\newpage
OR gate testing 2 inputs
\begin{python}
s = r'''
from latextool_basic import *
from latexcircuit import *

p = Plot()
g0 = OR_GATE(linewidth=2, linecolor='blue')
p += str(g0)

for x0,y0 in g0.inputs():
    p0 = (x0, y0)
    x1,y1 = x0 - 1, y0
    p1 = (x1, y1)
    p += Line(points=[p0,p1], linewidth=0.1, linecolor='red')

x0,y0 = g0.output()
p0 = (x0, y0)
x1,y1 = x0 + 1, y0
p1 = (x1, y1)
p += Line(points=[p0,p1], linewidth=0.1, linecolor='green')

p += Grid(x0=-1,y0=-1,x1=3,y1=1)

print(p)
'''
from latextool_basic import *
execute(s)
print(console(s).strip())
\end{python}


\newpage
OR gate testing 2 inputs
\begin{python}
s = r'''
from latextool_basic import *
from latexcircuit import *

p = Plot()
g0 = OR_GATE(linewidth=2, x=0, y=0, linecolor='blue')
p += str(g0)

for x0,y0 in g0.inputs():
    p0 = (x0, y0)
    x1, y1 = - 1, y0
    p1 = (x1, y1)
    p += Line(points=[p0,p1], linewidth=0.1, linecolor='red')

x0, y0 = g0.output(); p0 = (x0, y0)
x1, y1 = x0 + 1, y0; p1 = (x1, y1)
p += Line(points=[p0,p1], linewidth=0.1, linecolor='green')

p += Grid(x0=-1,y0=-2,x1=3,y1=2)

print(p)
'''.strip()
from latextool_basic import *
execute(s)
print(console(s).strip())
\end{python}



\newpage
OR gate testing 3 inputs
\begin{python}
s = r'''
from latextool_basic import *
from latexcircuit import *

p = Plot()
g0 = OR_GATE(linewidth=2, x=0,y=0,h=3, inputs=3, linecolor='blue')
p += str(g0)

for x0,y0 in g0.inputs():
    p0 = (x0, y0)
    p1 = (-1, y0)
    p += Line(points=[p0,p1], linewidth=0.1, linecolor='red')

x0,y0 = g0.output(); p0 = (x0, y0)
x1,y1 = x0 + 1, y0; p1 = (x1, y1)
p += Line(points=[p0, p1], linewidth=0.1, linecolor='green')

p += Grid(x0=-1,y0=-2,x1=3,y1=2)

print(p)
'''.strip()
from latextool_basic import *
execute(s)
print(console(s).strip())
\end{python}


\newpage
OR gate testing 4 inputs
\begin{python}
s = r'''
from latextool_basic import *
from latexcircuit import *

p = Plot()
g0 = OR_GATE(linewidth=2, x=0,y=0,h=3, inputs=4,linecolor='blue')
p += str(g0)

for x0,y0 in g0.inputs():
    p0 = (x0, y0); x1, y1 = -1, y0; p1 = (x1, y1)
    p += Line(points=[p0,p1], linewidth=0.1, linecolor='red')

x0,y0 = g0.output()
p0 = (x0, y0); p1 = (x0 + 1, y0)
p += Line(points=[p0,p1], linewidth=0.1, linecolor='green')

p += Grid(x0=-1, y0=-2, x1=5, y1=2)

print(p)
'''
from latextool_basic import *
execute(s)
print(console(s).strip())
\end{python}


\newpage
\subsection{Rotation}
AND gate rotate by $\pi/4$:
\begin{python}
s = r'''
from math import pi
from latextool_basic import *
from latexcircuit import *
p = Plot()
g = AND_GATE(x=0, y=0, inputs=2, angle=pi/2)
p += str(g)

for x,y in g.inputs():
    p += Line(points=[(x, y - 1), (x, y)])

x,y = g.output()
p += Line(points=[(x, y),(x, y + 1)])

p += Grid(x0=-1, y0=-1, x1=1, y1=2)
print(p)
'''.strip()
from latextool_basic import *
execute(s)
print(console(s))
#from latextool_basic import *
#execute(s, print_source=True)
\end{python}

\newpage
AND gate rotate by $\pi$:
\begin{python}
s = r'''
from math import pi
from latextool_basic import *
from latexcircuit import *
p = Plot()
g = AND_GATE(x=0, y=0, inputs=2, angle=pi)
p += str(g)

for x,y in g.inputs():
    p += Line(points=[(x + 1, y), (x, y)])

x,y = g.output()
p += Line(points=[(x, y),(x - 1, y)])

p += Grid(x0=-2, y0=-1, x1=1, y1=1)
print(p)
'''.strip()
from latextool_basic import *
execute(s)
print(console(s))
#from latextool_basic import *
#execute(s, print_source=True)
\end{python}

\newpage
AND gate rotate by $3\pi/2$:
\begin{python}
s = r'''
from math import pi
from latextool_basic import *
from latexcircuit import *
p = Plot()
g = AND_GATE(x=0, y=0, inputs=2, angle=3*pi/2)
p += str(g)

for x,y in g.inputs():
    p += Line(points=[(x, y + 1), (x, y)])

x,y = g.output()
p += Line(points=[(x, y),(x, y - 1)])

p += Grid(x0=-1, y0=-2, x1=1, y1=1)
print(p)
'''.strip()
from latextool_basic import *
execute(s)
print(console(s))
#from latextool_basic import *
#execute(s, print_source=True)
\end{python}


\newpage
OR gate rotate by $\pi/4$:
\begin{python}
s = r'''
from math import pi
from latextool_basic import *
from latexcircuit import *
p = Plot()

g = OR_GATE(x=0, y=0, inputs=2, angle=pi/2)
p += str(g)

for x,y in g.inputs():
    p += Line(points=[(x, y - 1), (x, y)])

x,y = g.output()
p += Line(points=[(x, y),(x, y + 1)])

p += Grid(x0=-1, y0=-1, x1=1, y1=2)
print(p)
'''.strip()
from latextool_basic import *
execute(s)
print(console(s))
#from latextool_basic import *
#execute(s, print_source=True)
\end{python}

\newpage
OR gate rotate by $\pi$:
\begin{python}
s = r'''
from math import pi
from latextool_basic import *
from latexcircuit import *
p = Plot()
g = OR_GATE(x=0, y=0, inputs=2, angle=pi)
p += str(g)

for x,y in g.inputs():
    p += Line(points=[(x + 1, y), (x, y)])

x,y = g.output()
p += Line(points=[(x, y),(x - 1, y)])

p += Grid(x0=-2, y0=-1, x1=1, y1=1)
print(p)
'''.strip()
from latextool_basic import *
execute(s)
print(console(s))
#from latextool_basic import *
#execute(s, print_source=True)
\end{python}


\newpage
OR gate rotate by $3\pi/2$:
\begin{python}
s = r'''
from math import pi
from latextool_basic import *
from latexcircuit import *
p = Plot()
g = OR_GATE(x=0, y=0, inputs=2, angle=3*pi/2)
p += str(g)

for x,y in g.inputs():
    p += Line(points=[(x, y + 1), (x, y)])

x,y = g.output()
p += Line(points=[(x, y),(x, y - 1)])

p += Grid(x0=-1, y0=-2, x1=1, y1=1)
print(p)
'''.strip()
from latextool_basic import *
execute(s)
print(console(s))
#execute(s, print_source=True)
\end{python}





\newpage
NOT gate rotate by $\pi/4$:
\begin{python}
s = r'''
from math import pi
from latextool_basic import *
from latexcircuit import *
p = Plot()
g = NOT_GATE(x=0, y=0, inputs=2, angle=pi/2)
p += str(g)

x,y = g.input()
p += Line(points=[(x, y - 1), (x, y)])

x,y = g.output()
p += Line(points=[(x, y),(x, y + 1)])

p += Grid(x0=-1, y0=-1, x1=1, y1=2)
print(p)
'''.strip()
from latextool_basic import *
execute(s)
print(console(s))
#from latextool_basic import *
#execute(s, print_source=True)
\end{python}


\newpage
NOT gate rotate by $\pi$:
\begin{python}
s = r'''
from math import pi
from latextool_basic import *
from latexcircuit import *
p = Plot()
g = NOT_GATE(x=0, y=0, inputs=2, angle=pi)
p += str(g)

x,y = g.input()
p += Line(points=[(x + 1, y), (x, y)])

x,y = g.output()
p += Line(points=[(x, y),(x - 1, y)])

p += Grid(x0=-2, y0=-1, x1=1, y1=1)
print(p)
'''.strip()
from latextool_basic import *
execute(s)
print(console(s))
#from latextool_basic import *
#execute(s, print_source=True)
\end{python}



NOT gate rotate by $3\pi/2$:
\begin{python}
s = r'''
from math import pi
from latextool_basic import *
from latexcircuit import *
p = Plot()
g = NOT_GATE(x=0, y=0, inputs=2, angle=3*pi/2)
p += str(g)

x,y = g.input()
p += Line(points=[(x, y + 1), (x, y)])

x,y = g.output()
p += Line(points=[(x, y),(x, y - 1)])

p += Grid(x0=-1, y0=-2, x1=1, y1=1)
print(p)
'''.strip()
from latextool_basic import *
execute(s)
print(console(s))
#from latextool_basic import *
#execute(s, print_source=True)
\end{python}



\newpage
Test OUTPUT POINT
\begin{python}
s = r'''
from latextool_basic import *
from latexcircuit import *
p = Plot()

NOR = NOR_GATE(x=0, y=3)
P = OUTPUT_POINT(gate=NOR)

p += str(NOR)
p += str(P)
print(p)
'''.strip()
from latextool_basic import *
execute(s)
print(console(s))
#from latextool_basic import *
#execute(s, print_source=True)
\end{python}

\begin{python}
s = r'''
from latextool_basic import *
from latexcircuit import *
p = Plot()

AND = AND_GATE(x=0, y=3)
P = OUTPUT_POINT(gate=AND, output_length=4, label='z', anchor='west')

p += str(AND)
p += str(P)
print(p)
'''.strip()
from latextool_basic import *
execute(s)
print(console(s))
#from latextool_basic import *
#execute(s, print_source)
\end{python}






\newpage
Test INPUT POINT
\begin{python}
s = r'''
from latextool_basic import *
from latexcircuit import *
p = Plot()

NOR = NOR_GATE(x=0, y=3)
P = INPUT_POINT(gate=NOR, input_index=0)

p += str(NOR)
p += str(P)
print(p)
'''
from latextool_basic import *
execute(s)
#print(r'{\small %s }' % console(s.strip()))

#from latextool_basic import *
#print(r'{\small %s }' % console(s.strip()))
\end{python}
\begin{python}
from latextool_basic import *
from latexcircuit import *
p = Plot()

NOR = NOR_GATE(x=0, y=3)
P = INPUT_POINT(gate=NOR, input_index=0)

p += str(NOR)
p += str(P)
print(p)
\end{python}


Example:
\begin{python}
s = r'''
from latextool_basic import *
from latexcircuit import *
p = Plot()

AND = AND_GATE(x=0, y=3, inputs=3)
P0 = INPUT_POINT(gate=AND, input_index=2, label='$x$', input_length=1)
P1 = INPUT_POINT(gate=AND, input_index=1, label='$y$', input_length=1)
P2 = INPUT_POINT(gate=AND, input_index=0, label='$z$', input_length=1)
P3 = OUTPUT_POINT(gate=AND, label='$xyz$', anchor='west', output_length=1)

p += str(AND)
p += str(P0)
p += str(P1)
p += str(P2)
p += str(P3)
print(p)
'''
from latextool_basic import *
print(r'{\small %s }' % console(s.strip()))
\end{python}
\begin{python}
from latextool_basic import *
from latexcircuit import *
p = Plot()

AND = AND_GATE(x=0, y=3, inputs=3)
P0 = INPUT_POINT(gate=AND, input_index=2, label='$x$', input_length=1)
P1 = INPUT_POINT(gate=AND, input_index=1, label='$y$', input_length=1)
P2 = INPUT_POINT(gate=AND, input_index=0, label='$z$', input_length=1)
P3 = OUTPUT_POINT(gate=AND, label='$xyz$', anchor='west', output_length=1)

p += str(AND)
p += str(P0)
p += str(P1)
p += str(P2)
p += str(P3)
print(p)
\end{python}




\newpage
Example:
\begin{python}
s = r'''
from latextool_basic import *
from latexcircuit import *
p = Plot()

and0 = AND_GATE(x=0, y=0, h=1, inputs=3)

x, y = and0.inputs()[2]; x -= 1
X = POINT(x=x, y=y, label='$x$')

x, y = and0.inputs()[1]; x -= 1
Y = POINT(x=x, y=y, label='$y$')

x, y = and0.inputs()[0]; x -= 1
Z = POINT(x=x, y=y, label='$z$')

x, y = and0.output(); x += 1
XYZ = POINT(x=x, y=y, label='$xyz$', anchor='west')

p += str(and0)
p += str(X); p += str(Y); p += str(Z)
p += str(XYZ)

p += '%s' % OrthogonalPath([X.output(), and0.inputs()[2]])
p += '%s' % OrthogonalPath([Y.output(), and0.inputs()[1]])
p += '%s' % OrthogonalPath([Z.output(), and0.inputs()[0]])

p += '%s' % OrthogonalPath([XYZ.output(), and0.output()])

p += Grid(x0=-2, y0=-1, x1=4, y1=1)
print(p)
'''
from latextool_basic import *
print('{\small %s }' % console(s.strip()))
\end{python}
\begin{python}
from latextool_basic import *
from latexcircuit import *
p = Plot()

and0 = AND_GATE(x=0, y=0, h=1, inputs=3)

x, y = and0.inputs()[2]; x -= 1
X = POINT(x=x, y=y, label='$x$')

x, y = and0.inputs()[1]; x -= 1
Y = POINT(x=x, y=y, label='$y$')

x, y = and0.inputs()[0]; x -= 1
Z = POINT(x=x, y=y, label='$z$')

x, y = and0.output(); x += 1
XYZ = POINT(x=x, y=y, label='$xyz$', anchor='west')

p += str(and0)
p += str(X); p += str(Y); p += str(Z)
p += str(XYZ)

p += '%s' % OrthogonalPath([inX.output(), and0.inputs()[2]])
p += '%s' % OrthogonalPath([inY.output(), and0.inputs()[1]])
p += '%s' % OrthogonalPath([inZ.output(), and0.inputs()[0]])

p += '%s' % OrthogonalPath([D.output(), and0.output()])

p += Grid(x0=-2, y0=-1, x1=4, y1=1)
print(p)
\end{python}


\newpage
Example:
\begin{python}
s = r'''
from latextool_basic import *
from latexcircuit import *
p = Plot()

and0 = AND_GATE(x=0,y=0,h=1, inputs=3)
and1 = AND_GATE(x=0,y=2,h=1, inputs=3)
and2 = AND_GATE(x=0,y=4,h=1, inputs=3)

x,y = and1.output(); x += 4
or0 = OR_GATE(x=x, y=y,h=1, inputs=3)

x, y = and1.inputs()[1]; x -= 4
A = POINT(x=x, y=y + 1, label='$A$')
B = POINT(x=x, y=y + 0, label='$B$')
C = POINT(x=x, y=y + -1, label='$C$')
notA = NOT_GATE(x=A.x()+1, y=A.y()-0.5)
notB = NOT_GATE(x=B.x()+1, y=B.y()-0.5)
notC = NOT_GATE(x=C.x()+1, y=C.y()-0.5)

x0,y0 = or0.output()
Z = POINT(x=x0+1, y=y0, label='$z$', anchor='west')

p += str(A); p += str(B); p += str(C)
p += str(notA); p += str(notB); p += str(notC)
p += str(and0); p += str(and1); p += str(and2)
p += str(or0)

p += str(Z)

OP = OrthogonalPath
# Join A to notA, ...
p += '%s' % OP([A.output(), notA.input()])
p += '%s' % OP([B.output(), notB.input()])
p += '%s' % OP([C.output(), notC.input()])

# Join inputs to AND gates
p += '%s' % OP([A.output(), and0.inputs()[2]])
p += '%s' % OP([A.output(), and1.inputs()[2]], shifts=[0.3])
p += '%s' % OP([A.output(), and2.inputs()[2]])
p += '%s' % OP([notB.output(), and2.inputs()[0]])

# Join AND gates to OR
p += '%s' % OP([and0.output(), or0.inputs()[0]])
p += '%s' % OP([and1.output(), or0.inputs()[1]])
p += '%s' % OP([and2.output(), or0.inputs()[2]])

# Join OR to z
p += '%s' % OP([or0.output(), Z.input()])

p += Grid(x0=-6, y0=-1, x1=9, y1=5)
print(p)
'''.strip()

from latextool_basic import *
execute(s)
print('{\small %s}' % console(s.strip()))
\end{python}




\newpage
\subsection{Example: SR latch}

SR latch:
\begin{python}
s = r'''
from latextool_basic import *
from latexcircuit import *
p = Plot()

NOR1 = NOR_GATE(x=0, y=3)
R = INPUT_POINT(gate=NOR1, input_index=1, label='$R$', input_length=1)
R0 = INPUT_POINT(gate=NOR1, input_index=0, input_length=1)
OUT1 = OUTPUT_POINT(gate=NOR1, output_length=1)
x,y = OUT1.output()
OUT11 = POINT(x=x+2, y=y, label='$Q$', anchor='west')

NOR2 = NOR_GATE(x=0, y=0)
S = INPUT_POINT(gate=NOR2, input_index=0, label='$S$', input_length=1)
S0 = INPUT_POINT(gate=NOR2, input_index=1, input_length=1)
OUT2 = OUTPUT_POINT(gate=NOR2, output_length=1)
x,y = OUT2.output()
OUT22 = POINT(x=x+2, y=y, label='$\overline{Q}$', anchor='west')

OP = OrthogonalPath

p += str(NOR1)
p += str(R)
p += str(R0)
p += str(OUT1); p += str(OUT11)

p += str(NOR2)
p += str(S)
p += str(S0)
p += str(OUT2); p += str(OUT22)

p += str(OP([OUT1.output(), OUT11.input()]))
p += str(OP([OUT2.output(), OUT22.input()]))

# Cross
x0,y0 = R0.input()
x1,y1 = OUT2.input()
p += Line(points=[(x0,y0),(x0,y0-1),(x1,y1+1),(x1,y1)])
p += str(POINT(x=x0,y=y0-1))
p += str(POINT(x=x1,y=y1+1))

x0,y0 = S0.input()
x1,y1 = OUT1.input()
p += Line(points=[(x0,y0),(x0,y0+1),(x1,y1-1),(x1,y1)])
p += str(POINT(x=x0,y=y0+1))
p += str(POINT(x=x1,y=y1-1))

print(p)
'''.strip()

from latextool_basic import *
execute(s)
print('{\small %s}' % console(s.strip()))
\end{python}




Test angle = 3.14159/4:
\begin{python}
s = r'''
from latextool_basic import *
from latexcircuit import *
p = Plot()

NOR1 = NOR_GATE(x=0, y=3, angle=3.14159/4)
R = INPUT_POINT(gate=NOR1, input_index=1, label='$R$', input_length=1)
R0 = INPUT_POINT(gate=NOR1, input_index=0, input_length=1)
OUT1 = OUTPUT_POINT(gate=NOR1, output_length=1)
x,y = OUT1.output()
OUT11 = POINT(x=x+2, y=y, label='$Q$', anchor='west')

NOR2 = NOR_GATE(x=0, y=0)
S = INPUT_POINT(gate=NOR2, input_index=0, label='$S$', input_length=1)
S0 = INPUT_POINT(gate=NOR2, input_index=1, input_length=1)
OUT2 = OUTPUT_POINT(gate=NOR2, output_length=1)
x,y = OUT2.output()
OUT22 = POINT(x=x+2, y=y, label='$\overline{Q}$', anchor='west')

OP = OrthogonalPath

p += str(NOR1)
p += str(R)
p += str(R0)
p += str(OUT1); p += str(OUT11)

p += str(NOR2)
p += str(S)
p += str(S0)
p += str(OUT2); p += str(OUT22)

p += str(OP([OUT1.output(), OUT11.input()]))
p += str(OP([OUT2.output(), OUT22.input()]))

# Cross
x0,y0 = R0.input()
x1,y1 = OUT2.input()
p += Line(points=[(x0,y0),(x0,y0-1),(x1,y1+1),(x1,y1)])
p += str(POINT(x=x0,y=y0-1))
p += str(POINT(x=x1,y=y1+1))

x0,y0 = S0.input()
x1,y1 = OUT1.input()
p += Line(points=[(x0,y0),(x0,y0+1),(x1,y1-1),(x1,y1)])
p += str(POINT(x=x0,y=y0+1))
p += str(POINT(x=x1,y=y1-1))

print(p)
'''.strip()

from latextool_basic import *
execute(s)
print('{\small %s}' % console(s.strip()))
\end{python}




\newpage
\subsection{layout function}

SOP example:
\begin{python}
s = r'''
from latextool_basic import *
from latexcircuit import *
p = Plot()
expr = [["a", "b"], ["a'", "b'"], ["a", "c"], ["a", "b'"]]
layout(p, expr, AND_GATE, OR_GATE)
print(p)
'''.strip()

from latextool_basic import *
execute(s)
print('{\small %s}' % console(s.strip()))
\end{python}


\newpage
layout with scale and font for label:
\begin{python}
s = r'''
from latextool_basic import *
from latexcircuit import *
p = Plot(scale=0.5)
expr = [["a", "b"], ["a'", "b'"], ["a", "c"], ["a", "b'"]]
layout(p, expr, AND_GATE, OR_GATE, font='footnotesize')
print(p)
'''.strip()

from latextool_basic import *
execute(s)
print('{\small %s}' % console(s.strip()))
\end{python}




\newpage
POS example:
\begin{python}
s = r'''
from latextool_basic import *
from latexcircuit import *
p = Plot()
expr = [["a", "b", "c"], ["a'", "b'", "c'"], ["a", "c"], ["a", "b'"]]
layout(p, expr, OR_GATE, AND_GATE)
print(p)
'''.strip()

from latextool_basic import *
execute(s)
print(r'{\small %s}' % console(s.strip()))
\end{python}




\newpage
POS example:
\begin{python}
s = r'''
from latextool_basic import *
from latexcircuit import *
p = Plot()
expr = [["a", "b", "c", "d"], ["a'", "b'", "c'", "d'"],
        ["a", "b"], ["a", "c"], ["b","c"], ["c","d"]]
layout(p, expr, OR_GATE, AND_GATE)
print(p)
'''.strip()

from latextool_basic import *
execute(s)
print(r'{\small %s}' % console(s) )
\end{python}




\newpage
POS example:
\begin{python}
s = r'''
from latextool_basic import *
from latexcircuit import *
p = Plot()
expr = [["A_1'", "A_0", "B_1'", "B_0"],
        ["A_1'", "A_0", "B_1", "B_0"],
        ["A_1", "A_0", "B_1'", "B_0"],
        ["A_1", "A_0", "B_1", "B_0"],
        ]
layout(p, expr, AND_GATE, OR_GATE)
print(p)
'''.strip()

from latextool_basic import *
execute(s)
print(r'{\small %s}' % console(s) )
\end{python}


\newpage
SOP example:
\begin{python}
s = r'''
from latextool_basic import console
from latexcircuit import SOP2
s = SOP("a'bcd + a'bc'd + abc'd + abcd")
print(s)
'''.strip()

from latextool_basic import *
execute(s)
print(r'{\small %s}' % console(s) )
\end{python}



\newpage
\subsection{Logic blocks}

\verb!points! argument is a dictionary.

\verb!points[k]! is \verb!(label, coordinates)!.
\verb!label! is used to label the point in the block.
\verb!coordinate! is the coordinates of the point related to the block.

You can get the coordinates of a point by calling the
\verb!block.point(k)!.

\begin{python}
s = r'''
from latextool_basic import *
from latexcircuit import *
p = Plot()
g = LOGIC_BLOCK(x=0, y=0, w=2, h=3, s='half adder', 
                points={'s':('$s$', (2,2.5)),
                        'c':('$c$', (2,0.5)),
                        'x':('', (0,2.5)),
                        'y':('', (0,0.5))})
p += str(g)

x,y = g.point('x'); P0 = POINT(x=x-2, y=y, label='$x$', anchor='east')
p += str(OrthogonalPath([P0.output(), (x,y)]))

x,y = g.point('y'); P1 = POINT(x=x-2, y=y, label='$y$', anchor='east')
p += str(OrthogonalPath([P1.output(), (x,y)]))

x,y = g.point('s'); P2 = POINT(x=x+2, y=y, label='$s$', anchor='east')
p += str(OrthogonalPath([P2.output(), (x,y)]))

x,y = g.point('c'); P2 = POINT(x=x+2, y=y, label='$c$', anchor='east')
p += str(OrthogonalPath([P2.output(), (x,y)]))

p += Grid(-3, -1, 5, 4)
print(p)
'''
from latextool_basic import *
execute(s)
print(r'{\small %s }' % console(s.strip()))
\end{python}




\newpage

\begin{python}
s = r'''
from latextool_basic import *
from latexcircuit import *
p = Plot()
g = LOGIC_BLOCK(x=0, y=0, w=2, h=3, s='half adder', 
                points={'s':('$s$', (2,2.5)),
                        'c':('$c$', (2,0.5)),
                        'x':('', (0,2.5)),
                        'y':('', (0,0.5))})
p += str(g)

x,y = g.point('x'); P0 = POINT(x=x-2, y=y, label='$x$', anchor='east')
p += str(OrthogonalPath([P0.output(), (x,y)]))

x,y = g.point('y'); P1 = POINT(x=x-2, y=y, label='$y$', anchor='east')
p += str(OrthogonalPath([P1.output(), (x,y)]))

x,y = g.point('s'); P2 = POINT(x=x+2, y=y, label='$s$', anchor='east')
p += str(OrthogonalPath([P2.output(), (x,y)]))

x,y = g.point('c'); P2 = POINT(x=x+2, y=y, label='$c$', anchor='east')
p += str(OrthogonalPath([P2.output(), (x,y)]))

p += '%s' % LOGIC_BLOCK(x=6, y=1, h=3, s='half adder',
                        points={'s':('$s$', (2,2.5)),
                                'c':('$c$', (2,0.5)),
                                'x':('', (0,2.5)),
                                'y':('', (0,0.5))})
p += Grid(-3, -1, 5, 4)
print(p)
'''
s = s.strip()
from latextool_basic import *
execute(s)
print(r'{\footnotesize %s}' % console(s))
print(p)
\end{python}


\newpage
\subsection{Logic blocks: full adder}
\begin{python}
from latextool_basic import *
from latexcircuit import *
p = Plot()

g0 = HALF_ADDER(x=0, y=-1.5)
g0_in0 = g0.point('in0')
g0_in1 = g0.point('in1')
g0_out0 = g0.point('out0')
g0_out1 = g0.point('out1')

g1 = HALF_ADDER(x=4, y=0)
g1_in0 = g1.point('in0')
g1_in1 = g1.point('in1')
g1_out0 = g1.point('out0')
g1_out1 = g1.point('out1')

g2 = OR_GATE(x=8, y=-0.5)
g2_in0 = g2.inputs()[1]
g2_in1 = g2.inputs()[0]
g2_out = g2.output()

p += str(g0); p += str(g1); p += str(g2)

# x0,y0 = position for A 
x,y = g0_in0; x0 = x - 2; y0 = y
A = POINT(x=x0, y=y0, label='$A$'); p += str(A)
B = POINT(x=A.input()[0], y=A.input()[1] - 1.5, label='$B$'); p += str(B)
C0 = POINT(x=A.input()[0], y=A.input()[1] + 1.5, label='$C_0$'); p += str(C0)

x,y = g2_out; x2 = x + 1
C1 = POINT(x=x2, y=y, label='$C_1$', anchor='west'); p += str(C1)

x,y = C1.input()[0], g1_out0[1]
S = POINT(x=x, y=y, label='$S$', anchor='west'); p += str(S)

opath = OrthogonalPath

p += '%s' % opath(points=[A.output(), g0_in0])                # A  -> g0
p += '%s' % opath(points=[B.output(), g0_in1], shifts=[-0.5]) # B -> g0
p += '%s' % opath(points=[C0.output(), g1_in0])               # C0 -> g1

p += '%s' % opath(points=[g0_out0, g1_in1])                   # g0 -> g1
p += '%s' % opath(points=[g0_out1, g2_in1])                   # g0 -> g2
p += '%s' % opath(points=[g1_out1, g2_in0])                   # g1 -> g2
p += '%s' % opath(points=[g1_out0, S.input()])                # g1 -> S
p += '%s' % opath(points=[g2_out, C1.input()])                # g2 -> C1

print(p)
\end{python}


\newpage
\subsection{Two NOTs}
\begin{python}
s = r'''
from latextool_basic import *
from latexcircuit import *
p = Plot()

g0 = NOT_GATE(x=0, y=0)
x,y = g0.output()
g1 = NOT_GATE(x=2, y=y)

p += str(g0)
p += str(g1)

p += Line(points=[g0.output(), g1.input()], label="$Q'$", anchor='below')

x0, y0 = g1.output()
x5, y5 = g0.input()

dx = 0.25
dy = 0.75

x1, y1 = x0+dx, y0
x2, y2 = x1, y1+dy
x4, y4 = x5-dx, y5
x3, y3 = x4, y2
p += Line(points=[(x0, y0),
                  (x1, y1), 
                  (x2, y2),
                  (x3, y3),
                  (x4, y4),
                  (x5, y5)])
X = POINT(x=x1, y=y1, r=0, label='$Q$', anchor='west')
p += str(X)
print(p)
'''.strip()

from latextool_basic import *
execute(s)
print(r'{\footnotesize %s}' % console(s))
print(p)
\end{python}



