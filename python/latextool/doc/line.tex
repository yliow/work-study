\subsection{linecolor and linewidth}

\begin{python}
s = r"""
from latextool_basic import *
p = Plot()
p += Line(x0=0, y0=0, x1=1, y1=1)
p += Line(x0=3, y0=0, x1=6, y1=2, linecolor='red', linewidth=0.1)
p += Line(x0=6, y0=0, x1=9, y1=2, linecolor='red', linewidth=0.2, 
                                  linestyle='dashed')
p += Grid()
print(p)
""".strip()

from latextool_basic import console, makeandincludegraphics
print(console(s.strip()))

exec(s)
\end{python}




\newpage
\subsection{startstyle and endstyle}

\begin{python}
s = r"""
from latextool_basic import *
p = Plot()
p += Line(x0=0, y0=0, x1=1, y1=1, startstyle='>')
p += Line(x0=3, y0=0, x1=6, y1=2, linecolor='red', linewidth=0.1, 
                                  endstyle='>')
p += Line(x0=6, y0=0, x1=9, y1=2, linecolor='red', linewidth=0.2, 
                                  linestyle='dashed', 
                                  startstyle='>', endstyle='>')
p += Line(x0=9, y0=0, x1=12, y1=2, linecolor='blue', linewidth=0.1,  
                                   startstyle='>|', endstyle='>|')
p += Line(x0=12, y0=0, x1=15, y1=2, linecolor='blue', linewidth=0.05,  
                                   startstyle='>>', endstyle='>>')
p += Grid()
print(p)
""".strip()

from latextool_basic import console, makeandincludegraphics
print(console(s))
exec(s)
\end{python}




\newpage
\verb!arrowstyle=triangle!.

\begin{python}
s = r"""
from latextool_basic import *
p = Plot()
p += Line(x0=0, y0=0, x1=1, y1=1, startstyle='->', arrowstyle='triangle')
p += Line(x0=3, y0=0, x1=6, y1=2, linecolor='red', linewidth=0.1, 
                                  endstyle='->', arrowstyle='triangle')
p += Line(x0=6, y0=0, x1=9, y1=2, linecolor='red', linewidth=0.2, 
                                  linestyle='dashed', 
                                  startstyle='->', endstyle='->', 
                                  arrowstyle='triangle')
p += Grid()
print(p)
""".strip()

from latextool_basic import console, makeandincludegraphics
print(console(s))
exec(s)
\end{python}




\newpage
\verb!startstyle='dot'!.
\begin{python}
s = r"""
from latextool_basic import *
p = Plot()
p += Line(x0=0, y0=0, x1=1, y1=1, startstyle='->', 
                                  arrowstyle='triangle', endstyle='dot')
p += Line(x0=3, y0=0, x1=6, y1=2, linecolor='red', linewidth=0.02, 
                                  endstyle='->', arrowstyle='triangle',
                                  startstyle='dot', r=0.2)
p += Line(x0=6, y0=0, x1=9, y1=2, linecolor='black', linewidth=0.05, 
                                  linestyle='dashed', startstyle='dot', 
                                  endstyle='->',  arrowstyle='triangle')
p += Grid()
print(p)
""".strip()
from latextool_basic import console, makeandincludegraphics
print(console(s))
exec(s)
\end{python}


\newpage
\subsection{\texttt{points} parameter}

\begin{python}
s = r"""
from latextool_basic import *
p = Plot()
p += Line(points=[(1,4), (0,0), (5,0), (6,3)], 
          linecolor='blue', linewidth=0.1, 
          startstyle='->', arrowstyle='triangle', endstyle='dot')
p += Grid()
print(p)
""".strip()

from latextool_basic import console, makeandincludegraphics
print(console(s))
exec(s)
\end{python}


\newpage
\subsection{boundary points}

\begin{python}
s = r"""
from latextool_basic import *
p = Plot()

aline = Line(points=[(1,4), (0,0), (5,0), (6,3)], 
             linecolor='blue', linewidth=0.1, 
             startstyle='->', arrowstyle='triangle', endstyle='dot')
p += aline
p += Circle(center=aline.top(), r=0.25, background='red')
p += Circle(center=aline.bottom(), r=0.25, background='red')
p += Circle(center=aline.left(), r=0.25, background='red')
p += Circle(center=aline.right(), r=0.25, background='red')
p += Circle(center=aline.topleft(), r=0.25, background='red')
p += Circle(center=aline.topright(), r=0.25, background='red')
p += Circle(center=aline.bottomleft(), r=0.25, background='red')
p += Circle(center=aline.bottomright(), r=0.25, background='red')
p += Grid()
print(p)
""".strip()

from latextool_basic import console, makeandincludegraphics
print(console(s))
exec(s)
\end{python}


\newpage
Test points
\begin{python}
s = r"""
from latextool_basic import *
p = Plot()

aline = Line(points=[(1,4), (0,0), (5,0), (6,3)], 
             linecolor='blue', linewidth=0.1, 
             startstyle='->', arrowstyle='triangle', endstyle='dot')
for point in aline.points:
    p += Circle(center=point, r=0.3, background='red')

p += aline
p += Grid()
print(p)
""".strip()

from latextool_basic import console
print(console(s))
exec(s)
\end{python}


\newpage
\subsection{Midpoints}

Test midpoint (can be useful for weighted graphs and network flows.
\begin{python}
s = r"""
from latextool_basic import *
p = Plot()
points = [(0,-1), (10,3)]
aline = Line(points=points, endstyle='>', linewidth=0.1)    
p += aline

p += Circle(center=aline.midpoint(), r=0.4, 
            background='white', label=r'{\texttt{%s}}' % 42)
p += Grid()
print(p)
""".strip()

from latextool_basic import console
print(console(s))
exec(s)
\end{python}





\newpage
Test midpoint with different ratios.
\begin{python}
s = r"""
from latextool_basic import *
p = Plot()
def linewithpoints(p, points, color1, color2):
    aline = Line(points=points, linecolor=color1, linewidth=0.2, 
                 startstyle='dot', arrowstyle='triangle', endstyle='->')
    p += aline
    for i in [0, 0.2, 0.4, 0.6, 0.8, 1.0]:
        p += Circle(center=aline.midpoint(ratio=i), r=0.4, 
                    background=color2, label=r'{\texttt{%s}}' % i)
linewithpoints(p, [(0, 0), (10, 0)], 'red', 'red!20')
linewithpoints(p, [(10, -1), (0, -1)], 'red', 'red!20')
linewithpoints(p, [(0, -2), (10, -3)], 'blue', 'red!20')
linewithpoints(p, [(10, -4), (0, -3)], 'blue', 'red!20')
linewithpoints(p, [(0, -6), (10, -5)], 'blue', 'red!20')
linewithpoints(p, [(10, -6), (0, -7)], 'blue', 'red!20')
linewithpoints(p, [(11, -7), (11, 0)], 'green', 'red!20')
linewithpoints(p, [(12, 0), (12, -7)], 'green', 'red!20')
p += Grid()
print(p)
""".strip()

from latextool_basic import console
print(console(s))
exec(s)
\end{python}





\newpage

Test midpoint for 1 segment.
\begin{python}
s = r"""
from latextool_basic import *
p = Plot()
points = [(0,-1), (10,3), (12,-2), (5,-1), (5,-2), (1,-2)]
aline = Line(points=points, linecolor='blue', linewidth=0.2, 
             startstyle='dot', arrowstyle='triangle', endstyle='->')    
p += aline
for i in [0, 0.1, 0.2, 0.3, 0.4, 0.5, 0.6, 0.7, 0.8, 0.9, 1.0]:
    p += Circle(center=aline.midpoint(ratio=i), r=0.4, 
                background='red!20', label=r'{\texttt{%s}}' % i)
p += Grid()
print(p)
""".strip()

from latextool_basic import console
print(console(s))
exec(s)
\end{python}


\newpage
Test short line with arrow tip (basically to test arrow tip).
\begin{python}
s = r"""
from latextool_basic import *
p = Plot()
for i, x in enumerate([0.2, 0.15, 0.1, 0.05, 0.01]):
    p += Line(points=[(1-x, i), (1, i)], linecolor='blue', linewidth=x, 
              endstyle='->')    
    p += Line(points=[(4-x, i), (4, i)], linecolor='blue', linewidth=x, 
              endstyle='->', arrowstyle='triangle')    
#p += Grid()
print(p)
""".strip()

from latextool_basic import console
print(console(s.strip()))
exec(s)
\end{python}

For arrows with line width x, it seems enough to have a length of x.



\newpage
\subsection{tikz names}
line from tikz name to tikz name:

\begin{python}
s = r"""
from latextool_basic import *
p = Plot()
p += Rect(x0=1, y0=0, x1=2, y1=1, label='a', name='a')
p += Rect(x0=5, y0=1, x1=6, y1=2, label='b', name='b')

p += Line(names=['a', 'b'], endstyle='>')

print(p)
""".strip()

from latextool_basic import console, makeandincludegraphics
print(console(s))
exec(s)
\end{python}



\newpage
\subsection{bend}

\begin{python}
s = r'''
from latextool_basic import *
p = Plot()

p += Circle(x=0, y=0, r=1, name='a', label='a')
p += Circle(x=4, y=0, r=1, name='b', label='b')
p += Circle(x=8, y=3, r=1, name='c', label='c')

p += Line(names=['a','b','c'], bend_left=30, endstyle='>', linewidth=0.1)

print(p)
'''.strip()

from latextool_basic import *
print(console(s))

exec(s)
\end{python}




\newpage
\subsection{label and anchor}

anchor = below
\begin{python}
s = r'''
from latextool_basic import *
p = Plot()
p += Circle(x=0, y=0, r=0.3, name='a', label='a')
p += Circle(x=10, y=0, r=0.3, name='b', label='b')
p += Line(names=['a', 'b'], anchor='below', label='hello')
print(p)
'''.strip()

from latextool_basic import *
execute(s, print_source=True)
\end{python}





anchor = below with bend
\begin{python}
s = r'''
from latextool_basic import *
p = Plot()
p += Circle(x=0, y=0, r=0.3, name='a', label='a')
p += Circle(x=10, y=0, r=0.3, name='b', label='b')
p += Line(names=['a', 'b'], bend_left=30, anchor='below', label='hello')
print(p)
'''.strip()

from latextool_basic import *
execute(s, print_source=True)
\end{python}




\newpage
anchor = above
\begin{python}
s = r'''
from latextool_basic import *
p = Plot()
p += Circle(x=0, y=0, r=0.3, name='a', label='a')
p += Circle(x=10, y=0, r=0.3, name='b', label='b')
p += Line(names=['a', 'b'], bend_left=30, anchor='above', label='hello')
print(p)
'''.strip()

from latextool_basic import *
execute(s, print_source=True)
\end{python}


anchor = above with bend
\begin{python}
s = r'''
from latextool_basic import *
p = Plot()
p += Circle(x=0, y=0, r=0.3, name='a', label='a')
p += Circle(x=10, y=0, r=0.3, name='b', label='b')
p += Line(names=['a', 'b'], bend_left=30, anchor='above', label='hello')
print(p)
'''.strip()

from latextool_basic import *
execute(s, print_source=True)
\end{python}



\newpage
anchor = left with bend
\begin{python}
s = r'''
from latextool_basic import *
p = Plot()
p += Circle(x=0, y=0, r=0.3, name='a', label='a')
p += Circle(x=0, y=3, r=0.3, name='b', label='b')
p += Line(names=['a', 'b'], bend_left=30, anchor='left', label='hello')
print(p)
'''.strip()

from latextool_basic import *
execute(s, print_source=True)
\end{python}

anchor = right with bend
\begin{python}
s = r'''
from latextool_basic import *
p = Plot()
p += Circle(x=0, y=0, r=0.3, name='a', label='a')
p += Circle(x=0, y=3, r=0.3, name='b', label='b')
p += Line(names=['a', 'b'], bend_left=30, anchor='right', label='hello')
print(p)
'''.strip()

from latextool_basic import *
execute(s, print_source=True)
\end{python}





\newpage
\subsection{Using midpoint to place labels}

\begin{python}
s = r'''
from latextool_basic import *
p = Plot()
aline = Line(points=[(0,0), (3,0)])
p += aline
x,y = aline.midpoint(ratio=0.2)

from latexcircuit import *
p += Circle(x=x, y=y, r=0.05, background='red')
X = POINT(x=x, y=y, label='$abcde$', anchor='south')
p += str(X)

print(p)
'''.strip()

from latextool_basic import *
execute(s, print_source=True)
\end{python}





\newpage
\subsection{Two control points}

\begin{python}
s = r'''
from latextool_basic import *
p = Plot()
L = 0.03
C = 'blue!50'

p += Circle(x=2, y=2, r=0.2, background=C, name='a', linewidth=L)
p += Circle(x=1, y=1.5, r=0.05, background='black')
p += Circle(x=1.5, y=1, r=0.05, background='black')

p += Line(names=['a','a'], controls=[(1,1.5),(1.5,1)], linewidth=L)
p += Grid(0, 0, 3, 3)
print(p)
'''.strip()
from latextool_basic import *
print(console(s))
exec(s)
\end{python}


\newpage
\subsection{One control point}

\begin{python}
s = r'''
from latextool_basic import *
p = Plot()
C = 'blue!50'

p += Circle(x=2, y=2, r=0.2, background=C, name='22', linewidth=L)
p += Circle(x=0, y=0, r=0.2, background=C, name='00', linewidth=L)

p += Circle(x=0, y=2, r=0.05, background='black')

p += Line(names=['22','00'], controls=[(0,2)], linewidth=0.03)
p += Grid(x0=-1,y0=-1,x1=3,y1=3)

print(p)
'''.strip()
from latextool_basic import *
print(console(s))
exec(s)
\end{python}



\newpage
1 control point and label
\begin{python}
s = r'''
from latextool_basic import *
p = Plot()
L = 0.03
C = 'blue!50'

p += Circle(x=2, y=2, r=0.2, background=C, name='22', linewidth=L)
p += Circle(x=0, y=0, r=0.2, background=C, name='00', linewidth=L)

p += Circle(x=0, y=2, r=0.05, background='black')

p += Line(names=['22','00'], controls=[(0,2)], linewidth=L, label='test', anchor='below')

p += Grid(x0=-1,y0=-1,x1=3,y1=3)
print(p)
'''.strip()
from latextool_basic import *
print(console(s))
exec(s)
\end{python}



\subsection{loop}

Example uses Graph.

xxx
\begin{python}
s = r'''
from latextool_basic import *
p = Plot()
L = 0.03
C = 'blue!50'

p += Graph.node(x=0, y=0, name='a')
p += Line(names=['a','a'], label='test', anchor='below', loop='loop above')

p += Grid(x0=-1,y0=-1,x1=3,y1=3)
print(p)
'''.strip()
from latextool_basic import *
print(console(s))
exec(s)
\end{python}
