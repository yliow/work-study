\section{Code}

The \verb!code! function creates a table based on the string passed in.
The linewidths of the cells are set to 0.

\begin{python}
s = '''
from latextool_basic import *
p = Plot()

M = r"""
twice (fun x -> x * x) 42
(fun f -> (fun x -> f(f x))) (fun x -> x * x) 42
""".strip()
    
code(p, M)

print(p)
'''.strip()

from latextool_basic import *
execute(s, print_source=True)
\end{python}

\newpage
Position:
\begin{python}
s = '''
from latextool_basic import *
p = Plot()

M = r"""
twice (fun x -> x * x) 42
(fun f -> (fun x -> f(f x))) (fun x -> x * x) 42
""".strip()

p += Circle(x=0, y=0, r=0.5)
code(p, M, x=0, y=0)

p += Circle(x=2, y=2, r=0.5, linecolor='red')
code(p, M, x=2, y=2)

print(p)
'''.strip()

from latextool_basic import *
execute(s, print_source=True)
\end{python}



\newpage
\subsection{Cell size}
\begin{python}
s = '''
from latextool_basic import *
p = Plot()

M = r"""
twice (fun x -> x * x) 42
(fun f -> (fun x -> f(f x))) (fun x -> x * x) 42
""".strip()
    
code(p, M, width=0.3, height=1)

print(p)
'''.strip()

from latextool_basic import *
execute(s, print_source=True)
\end{python}




\newpage
\subsection{border linewidth}

\begin{python}
s = '''
from latextool_basic import *
p = Plot()

M = r"""
twice (fun x -> x * x) 42
(fun f -> (fun x -> f(f x))) (fun x -> x * x) 42
""".strip()
    
code(p, M, border_linewidth=0.05)

print(p)
'''.strip()

from latextool_basic import *
execute(s, print_source=True)
\end{python}




\newpage
\subsection{distance of border to rect}

\begin{python}
s = '''
from latextool_basic import *
p = Plot()

M = r"""
twice (fun x -> x * x) 42
(fun f -> (fun x -> f(f x))) (fun x -> x * x) 42
""".strip()
    
code(p, M, border_linewidth=0.05, innersep=0.8)

print(p)
'''.strip()

from latextool_basic import *
execute(s, print_source=True)
\end{python}




\newpage
\subsection{coderect}

The return of \verb!code! is just a rect.
\verb!coderect! takes in a \verb!code! rect and
starting row and column indices and
ending row and column indices and returns a rect.
The linecolor of this rect is red.

\begin{python}
s = '''
from latextool_basic import *
p = Plot()

M = r"""
twice (fun x -> x * x) 42


(fun f -> (fun x -> f(f x))) (fun x -> x * x) 42
""".strip()
    
N = code(p, M)
r0 = coderect(N, 0, 0, 0, 4)
p += r0
r1 = coderect(N, 3, 0, 3, 27)
r1.linecolor='blue'
p += r1

p0 = r0.bottom()
p1 = r1.top(); p1 = (p0[0], p1[1])
p += Line(points=[p0,p1], endstyle='>', linecolor='red')
print(p)
'''.strip()

from latextool_basic import *
execute(s, print_source=True)
\end{python}




\newpage
\subsection{Substitution}

If you want to print $\rho$ which takes more than
one character, you can use an unused character in the string and then
pass in a dictionary to replace the character to the actual
string that you need.

\begin{python}
s = r'''
from latextool_basic import *
p = Plot()

M = r"""
twice (fun x -> x * x) 42    #


(fun f -> (fun x -> f(f x))) (fun x -> x * x) 42
""".strip()

code(p, M, d={'#':r'\textred{$\rho$}'})
print(p)
'''.strip()

from latextool_basic import *
execute(s, print_source=True)
\end{python}

\newpage
\subsection{Long division example}

\verb!linebelow! is used to draw lines in code.
\verb!divlinebelow! is used for long division line.

\begin{Verbatim}[frame=single]
from latextool_basic import *
p = Plot()
    
M = r"""
.    01
1101 101010101010
    -0000
     10101
     -1101
      10000
       1101
""".strip()

N = code(p, M)
divlinebelow(p, N, 0, 5, 16)
linebelow(p, N, 2, 4, 8)
linebelow(p, N, 4, 5, 9)
print(p)
\end{Verbatim}

\begin{python}
s = r'''
from latextool_basic import *
p = Plot()

    
M = r"""
.    01
1101 101010101010
    -0000
     10101
     -1101
      10000
       1101
""".strip()

N = code(p, M)
divlinebelow(p, N, 0, 5, 16)
linebelow(p, N, 2, 4, 8)
linebelow(p, N, 4, 5, 9)
print(p)
'''.strip()
from latextool_basic import *
execute(s, print_source=False)
\end{python}
