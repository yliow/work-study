\sectionthree{Informal Introduction to Languages}
\begin{python0}
from solutions import *; clear()
\end{python0}

We will now introduce some terms used in the study of Languages in
CS. Most of it is just giving terms in set theory some new names.

Let $\Sigma = \{a,b,c\}$. The following is a \defone{string} of
length 5:
\[ x = bccab \]
Two strings are \defone{equal} if elements of $\Sigma$ appears in the
string in the same order. You can \defone{concatenate} strings: For
instance
\[ x \cdot (aaab) = bccabaaab \]
We include an \defone{empty string} written $\ep$. The following
holds for $\ep$:
\[ \ep bccab = bcca\ep\ep b = bccab\ep\ep\ep \]
Given a string $x$ we can define $|x|$, the \defone{length} of $x$.
For instance:
\[
  |\ep| = 0, \quad |ab| = 2, \quad |a\ep\ep\ep b| = 2
\]
If $L$ is a set of strings (not necessarily all the strings) is a
\defone{language} over $\Sigma$. $\Sigma$ is said to be the
\defone{alphabet} of
$L$.
Elements of $L$ are called \defone{words}.
(I'll sometimes use strings since as programmers, you are more used to that terms.)
If a word has
length $k$, it's also called a \defone{$k$--word}.

For instance take your favorite programming language, say C++. A
string of the form:
\[
\text{\texttt{ if ( x == 0 ) x = 1; }}
\]
is a string in the C++ language.
You can already see words such as
\[
\texttt{if}
\]
and
\[
\texttt{==}
\]
and even
\[
\texttt{(}
\]
In this case the alphabet is pretty much all the
visible character on your keyboard.
%Formally
%\[
% \token{ \text{ \texttt{IF}   }} \
% \token{ \text{ \texttt{(}    }} \
% \token{ \text{ \texttt{VAR}  }} \
% \token{ \text{ \texttt{==}   }} \
% \token{ \text{ \texttt{INT}  }} \
% \token{ \text{ \texttt{)}    }} \
% \token{ \text{ \texttt{VAR}  }} \
% \token{ \text{ \texttt{=}    }} \
% \token{ \text{ \texttt{INT}  }} \
% \token{ \text{ \texttt{;}    }}
%\]
%So you see the the alphabet $\Sigma$ include keywords like
%$\token{
%  \text{ \texttt{IF} }
%}$.
%Variables are changed to
%$\token{
%  \text{ \texttt{VAR} }
%}$. Etc.

The language therefore specifies all possible in \lq\lq words''
C++.
See how important the concept is now?

The above tells us that language theory (and automata theory ... later),
although it seems abstract, can be very practical and useful.

In the chapter on sets, I mentioned that
you might want to solve the following problem:
\[
\text{\textsc{TelephoneNumberProblem}(s) : Is the string $s$ a valid telephone number?}
\]
One thing you can do is to construct a set
to solve this problem.
In this context, you have a language for the
\textsc{TelephoneNumberProblem}:
\[
\textsc{TelephoneNumberLanguage} = \{s \mid s \text{ is a valid telephone number}\}
\]
Then answering
the question
\text{\textsc{TelephoneNumberProblem}(1112223333)} is just a  membership check:
\[
1112223333 \in \textsc{TelephoneNumberLanguage}
\]
If you don't care about efficiency,
of course this problem can be solved since you
can store all telephone numbers in 
\textsc{TelephoneNumberLanguage}.
This is possible because the
\textsc{TelephoneNumberLanguage}
is finite!
So it's no big deal.

But in fact the above is way more important than for the sake of
a telephone number validation program.

The above example converts a question (that requires a YES or NO answer)
into a set membership problem.
This abstraction is a very important step
to the heart of
\defone{TCS}
\sidebarskip{12pt}
(\defone{Theoretical Computer Science}\sidebarskip{0pt}).

By the way a question that asks for a YES or NO
is called a \defone{decision problem}.
We'll come back this concept during the later half of this course.

The the above problem hides something more complex.
  
Consider the following problem.
Suppose we consider the symbols $\Sigma$ so be the set
\[
\Sigma = \{\texttt{0},\texttt{1},\texttt{2},\texttt{3},\texttt{4},\texttt{5},\texttt{6},\texttt{7},\texttt{8},\texttt{9},\texttt{+},\texttt{=}\}
\]
Right now just think of 0,1,2,...,9 as symbols.
Forget about the fact that they look like numbers.
Also forget the fact that $+$ looks like addition: it's just a symbol.
Same for $=$: it's just a symbol.
Now consider the following problem:
\[
\textsc{AdditionProblem}: \text{Is $x + y = z$?}
\]
where $x,y,z$ are integers.
I can convert this to a language:
\begin{align*}
  \textsc{AdditionLanguage}
  &= \{x\texttt{+}y\texttt{=}z \mid x,y,z \text{ are numeric strings} \\
  &\hspace{0.7cm}\text{ and when viewed as integers, } x + y = z \}
\end{align*}
For instance the word
\[
\texttt{1+1=2}
\]
is word over $\Sigma$. And when we re-think this word (or string) as
representing the \textit{mathematical} fact
\[
1 + 1 = 2
\]
we see that the mathematical fact is true.
Hence
\[
\texttt{1+1=2} \in \textsc{AdditionLanguage}
\]
On the other hand although
\[
\texttt{2+3=7}
\]
is a word over $\Sigma$,
but it definitely does not represent a valid mathematical fact.
Hence
\[
\texttt{2+3=7} \notin \textsc{AdditionLanguage}
\]
And of course the word
\[
\texttt{2+52=} \notin \textsc{AdditionLanguage}
\]
But there's fundamentally something very different between
\textsc{AdditionLanguage} and
\textsc{TelephoneNumberLanguage}.
\textsc{AdditionLanguage} is infinite.
You cannot possible store an infinite collection of string
from the set \textsc{AdditionLanguage} in a database then query it!!!

So then how do we answer the membership problem
\[
w \in \textsc{AdditionLanguage}
\]
when \textsc{AdditionLanguage}?

It turns out that it can be done.
But for sure the algorithm used has to be finite.
This is achieved through the mathematical construction of
computational models called \defone{automata}.
Automata are finite in nature in terms of the mathematical
description of how they work.
The description is provided by a finite directed graph
where the edges are decorated in a certain way.

Although automata were first discovered abstractly as mathematical objects,
we now know that they can be implemented in software and hardware.
In fact the modern day computer is essentially the realization of
Turing machines, which are a type of automatas.

You will see that there are many classes of automatas.
For simple languages, we only need to define \lq\lq simple'' automatas.
For complex languages, we will need \lq\lq complex'' automatas.
We will study the following classes of automatas:
\begin{enumerate}
\item Finite state automatas 
\item Pushdown automatas
\item Deterministic pushdown automatas (Optional)
\item Turing machines
\end{enumerate}

These computational models have given rise to
the study of computational complexity 
which studies how to classify the complexity of a problem
in terms of resource usage such as time and space.
Probably the most important problem is TCS and in math
is the so--called \lq\lq P = NP" or \lq\lq P vs NP" problem.
This is one of the seven Millennium Problem.
You can read about it at
\url{https://www.claymath.org/millennium-problems/p-vs-np-problem}.

The classification of problems into complexity classes
gives us a catalog of problems that we know either cannot be solved
or can only be solves with exponential runtime.
Automata theory provides a methods to compare problems and
help predict if a problem is easy or hard.
This is obviously very helpful:
If you know your theory of automata well, you can tell your boss
that a problem he wants you to solve either cannot be solved or
is know to have an exponential runtime.
