\sectionthree{Concatenation of languages and Kleene Star}
\begin{python0}
from solutions import *; clear()
\end{python0}

I already talked about exponentiation of a word, 
i.e., $x^n$ where $x$ is a string.
Now I want to talk about the exponentiation of a set of words (i.e.
exponentiation of languages).

\begin{defn}
Let $L, L'$ be languages over $\Sigma$.
The
\defterm{concatenation}\index{concatenation}\tinysidebar{concatenation}
of $L$ and $L'$ is defined to be
\[
L \cdot L' = \{x \cdot x' \mid x \in L, \, x' \in L'\}
\index{$\cdot$}\tinysidebar{$\cdot$}
\]
We will usually write
\[
LL'
\]
instead of 
\[
L \cdot L'
\]
\end{defn}

Let's do a quick example:
Suppose $L_1 = \{a, ba, bab, baba\}$ 
and $L_2 = \{b, ab, aab, aaab\}$.
The product $L_1 L_2$ is
\begin{align*}
L_1L_2 
&=   \{a\cdot b, a\cdot ab, a \cdot aab, a \cdot aaab, \\
& \hspace{0.8cm} ba \cdot b, ba \cdot ab, ba \cdot aab, ba \cdot aaab, \\ 
& \\
& \\
&=   \{ab, aab, aaab, aaaab, \\
& \hspace{0.8cm} bab, baab, baaab, baaaab, \\ 
& \\
& \\
&=   \{ab, a^2b, a^3b, a^4b, \\
& \hspace{0.8cm} bab, ba^2b, ba^3b, ba^4b, \\ 
& \\
& \\
\end{align*}
(finish it!)
Now check which of the following strings are in $L_1L_2$:
\begin{tightlist}
\item $\ep$
\item $a$
\item $b$
\item $aa$
\item $ab$
\item $ba$
\item $bb$
\item $aaa$
\item $aab$
\item $aba$
\item $baa$
\item $abb$
\item $bab$
\item $bba$
\item $bbb$
\end{tightlist}



%-*-latex-*-

\begin{ex} 
  \label{ex:prob-00}
  \tinysidebar{\debug{exercises/{disc-prob-28/question.tex}}}

  \solutionlink{sol:prob-00}
  \qed
\end{ex} 
\begin{python0}
from solutions import *
add(label="ex:prob-00",
    srcfilename='exercises/discrete-probability/prob-00/answer.tex') 
\end{python0}


%-*-latex-*-

\begin{ex} 
  \label{ex:prob-00}
  \tinysidebar{\debug{exercises/{disc-prob-28/question.tex}}}

  \solutionlink{sol:prob-00}
  \qed
\end{ex} 
\begin{python0}
from solutions import *
add(label="ex:prob-00",
    srcfilename='exercises/discrete-probability/prob-00/answer.tex') 
\end{python0}


%-*-latex-*-

\begin{ex} 
  \label{ex:prob-00}
  \tinysidebar{\debug{exercises/{disc-prob-28/question.tex}}}

  \solutionlink{sol:prob-00}
  \qed
\end{ex} 
\begin{python0}
from solutions import *
add(label="ex:prob-00",
    srcfilename='exercises/discrete-probability/prob-00/answer.tex') 
\end{python0}


%-*-latex-*-

\begin{ex} 
  \label{ex:prob-00}
  \tinysidebar{\debug{exercises/{disc-prob-28/question.tex}}}

  \solutionlink{sol:prob-00}
  \qed
\end{ex} 
\begin{python0}
from solutions import *
add(label="ex:prob-00",
    srcfilename='exercises/discrete-probability/prob-00/answer.tex') 
\end{python0}


\newpage
It's not shocking that if I have three languages $L$, $L'$, and $L''$,
then $LL'L''$ is defined as
\[
LL'L'' = (LL')L''
\]
This is just like for real value variables $x,y,z$, when I write
$x + y + z$, I mean $(x + y) + z$.
Because $(x + y) + z = x + (y + z)$, it doesn't really matter
which + you perform first which is why we omit parentheses
from either $(x + y) + z$ or from $x + (y + z)$.

It's not too surprising that

\begin{prop}
$(LL')L'' = L(L'L'')$.
\end{prop}

\proof
DIY.
[Hint: The proof depends on the fact that $(xy)z = x(yz)$ if
$x,y,z$ are words in $L,L',L''$ respectively.]
\qed


\newpage
Frequently, we want to concatenate the same language.
So let's have a shorthand for that ...

We will write 
\[
L^2
\]
instead of $LL$.
In general we define
\begin{align*}
L^0 &= \{\ep\} \\
L^{n+1} &= L L^n \,\,\, \text{ for } n \geq 0
\end{align*}
We also define
\begin{align*}
L^* &= \bigcup_{n = 0}^\infty L^n = L^0 \cup L^1 \cup L^2 \cup \cdots \\
L^+ &= \bigcup_{n = 1}^\infty L^n = L^1 \cup L^2 \cup \cdots \\
\end{align*}


%-*-latex-*-

\begin{ex} 
  \label{ex:prob-00}
  \tinysidebar{\debug{exercises/{disc-prob-28/question.tex}}}

  \solutionlink{sol:prob-00}
  \qed
\end{ex} 
\begin{python0}
from solutions import *
add(label="ex:prob-00",
    srcfilename='exercises/discrete-probability/prob-00/answer.tex') 
\end{python0}


%-*-latex-*-

\begin{ex} 
  \label{ex:prob-00}
  \tinysidebar{\debug{exercises/{disc-prob-28/question.tex}}}

  \solutionlink{sol:prob-00}
  \qed
\end{ex} 
\begin{python0}
from solutions import *
add(label="ex:prob-00",
    srcfilename='exercises/discrete-probability/prob-00/answer.tex') 
\end{python0}


%-*-latex-*-

\begin{ex} 
  \label{ex:prob-00}
  \tinysidebar{\debug{exercises/{disc-prob-28/question.tex}}}

  \solutionlink{sol:prob-00}
  \qed
\end{ex} 
\begin{python0}
from solutions import *
add(label="ex:prob-00",
    srcfilename='exercises/discrete-probability/prob-00/answer.tex') 
\end{python0}


%-*-latex-*-

\begin{ex} 
  \label{ex:prob-00}
  \tinysidebar{\debug{exercises/{disc-prob-28/question.tex}}}

  \solutionlink{sol:prob-00}
  \qed
\end{ex} 
\begin{python0}
from solutions import *
add(label="ex:prob-00",
    srcfilename='exercises/discrete-probability/prob-00/answer.tex') 
\end{python0}

