\section{Concatenation Operation}

\begin{defn}
  Define the \defone{concatenation} operation as following. If
  $x_0 \cdots x_{m-1}$ and $y_0 \cdots y_{n-1}$ are strings over $\Sigma$ where
  $x_i,y_j \in \Sigma_\ep$, then
  \[
  (x_0 \cdots x_{m-1}) \cdot (y_0 \cdots y_{n-1})
  \defeq x_0 \cdots x_{m-1} y_0 \cdots y_{n-1}
  \]
  We will also write $xy$ instead of $x\cdot y$.
\end{defn}

(Some authors write $x||y$ instead of $xy$, but I don't like it.
This is very common in cryptography.)

Sanity check: Are you really sure that all the above are
consistent? No circular definitions? Sure?

Let's check the definition very carefully.
First of all, of course intuitively we know that we want 
$(x_0 \cdots x_{m-1}) \cdot (y_0 \cdots y_{n-1})$
to be a string.
We then need to check the formal definition of a string.
We have accepted the definition of a string (check the definition on previous pages) to be a sequence
\[
x_0 x_1 \cdots x_{m-1}
\]
where each $x_i \in \Sigma_\ep$.
And $x_0 \cdots x_{m-1} y_0 \cdots y_{n-1}$is clearly a sequence where each $x_i, y_j$ are in $\Sigma_\ep$.
To make it even more explicit, I can define it as:

\begin{defn}
  Define the \defone{concatenation} operation as following. If
  $x_0 \cdots x_{m-1}$ and $y_0 \cdots y_{n-1}$ are strings over $\Sigma$ where
  $x_i,y_j \in \Sigma_\ep$, then
  \[
  (x_0 \cdots x_{m-1}) \cdot (y_0 \cdots y_{n-1})
  \defeq z_0 z_1 z_2 \cdots z_{m + n - 1}
  \]
  where
  \[
  z_i =
  \begin{cases}
    x_i & \text{ if $0 \leq i \leq m - 1$} \\
    y_{i - m} & \text{ if $m \leq j \leq m + n - 1$}
  \end{cases}
  \]
  We will also write $xy$ instead of $x\cdot y$.
\end{defn}

Compare the two definitions and make sure you can see the difference.
In this case, the earlier definition is actually correct.
The second definition is equivalent to the first.
But the second is a little bit more explicit.


\newpage
\begin{prop}
Let $x,y,z\in \Sigma^*$. Then
\[ (xy) z = x(yz) \quad (Associativity) \]
This means that we can be lazy (or efficient) and write $xyz$.
\end{prop}


\begin{ex}
Prove
\[ xyz = x'yz' \iff xz = x'z' \]
[Hint: Induction on $|y|$.]
\end{ex}

\newpage
\begin{ex}
Given any string $x = y_0\ldots y_m$, either
\begin{itemize}
 \item $x = \ep$, or
 \item $x = x_0 \ldots x_n$ where $x_i \in \Sigma$. Specifically,
       there exists integers $0 \leq i_0 < i_1 < \ldots < i_n \leq m$ such that
       \[
        y_j =
        \ep \iff j \in \{0,\ldots,m\} - \{i_0, \ldots, \}
       \]
      and
      \[
       x = y_{i_0} y_{i_1} \ldots y_{i_n}
      \]
      We will call this the simplified expression for $x$. Furthermore,
      the simplified expression for $x$ is unique in the sense that if $x
      = z_0 \ldots z_p = z'_0 \ldots z'_q$, then $p=q$, and $z_i = z'_i$
      for $0\leq i \leq p$.
\end{itemize}
\end{ex}

\begin{defn}
Let $x,y \in \Sigma^*$. Then $x \neq y$ iff their simplied
expression are not the same, i.e., if their simplified expressions
are $x = x_0\ldots x_m$ and $y=y_0 \ldots y_n$, then either $n
\neq m$, or there is some $i$ such that $x_i \neq y_i$.
\end{defn}


(For you math experts out there, $\Sigma^*$ is a free semigroup on
$\Sigma$ where concatenation is the semigroup operation.)
%In general
%$G$ is a semigroup with binary operation $\cdot$ if
%\begin{itemize}
% \item $(x \cdot y) \cdot z = x\cdot (y\cdot z)$ for $x,y,z \in G$
% \item There exists some element $\ep \in G$ such that $\ep \cdot
%        x = x = x\ep$ for $x \in G$.
%\end{itemize}
%Besides including the symbol $\ep$, if you also include the symbol
%$a^{-1}$ for each symbol $a \in \Sigma$ and extend $\cdot$ so that
%$aa^{-1}=\ep=a^{-1}a$, then the set of strings with all these
%symbols ($\ep$, $a \in \Sigma$, and new symbols $a^{-1}$ for $a \in
%\Sigma$) form a free group over $\Sigma$. In general $G$ is a group
%with binary group operation $\cdot$ if
%\begin{itemize}
% \item $(x \cdot y) \cdot z = x \cdot ( y \cdot z )$
% \item There is some $\ep \in G$ such that $\ep \cdot x = x = x
%       \cdot \ep$ for $x \in G$
% \item For each $x \in G$, there is some $x^{-1}$ such that $x
%       \cdot x^{-1} = x^{-1} \cdot x = \ep$
%\end{itemize}
