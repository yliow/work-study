\sectionthree{The equals relation}
\begin{python0}
from solutions import *; clear()
\end{python0}

Most authors define $=$ using intuition: $x=y$ if after omitting
$\ep$ from $x$ and $y$, the alphabets in $x$ and $y$ occur in the
same order. Here's a more formal definition. First we will define
$=$ using recursion. We will then note that $=$ is an equivalence
relation. Also, we will then show that every string is equivalence
to either $\ep$ or one without $\ep$. So formally, you can think of
$\ep$ and strings without $\ep$ as a complete set of representatves
for $\Sigma^*$.

\begin{defn}
Given a string $x$ over $\Sigma$. We define the relation $=$ on
$\Sigma^*$ inductively as follows.
Let $x, y \in \Sigma^*$.
\begin{myenum}
\item  $\ep = \ep$
\item  $a = a$ for all $a \in \Sigma$
\item  If $x = y$, then $x = \ep y$
\item  If $x = y$, then $x = y \ep$
\item  If $x = y$, then $\ep x = y$
\item  If $x = y$, then $x \ep = y$
\item  If $x = y$, then $ax = ay$ for any $a \in \Sigma$.
\item  If $x = y$, then $xa = ya$ for any $a \in \Sigma$.
\end{myenum}
We say $x$ \defone{equals} $y$
if $x=y$.
\end{defn}

%-*-latex-*-

\begin{ex} 
  \label{ex:prob-00}
  \tinysidebar{\debug{exercises/{disc-prob-28/question.tex}}}

  \solutionlink{sol:prob-00}
  \qed
\end{ex} 
\begin{python0}
from solutions import *
add(label="ex:prob-00",
    srcfilename='exercises/discrete-probability/prob-00/answer.tex') 
\end{python0}

%-*-latex-*-

\begin{ex} 
  \label{ex:prob-00}
  \tinysidebar{\debug{exercises/{disc-prob-28/question.tex}}}

  \solutionlink{sol:prob-00}
  \qed
\end{ex} 
\begin{python0}
from solutions import *
add(label="ex:prob-00",
    srcfilename='exercises/discrete-probability/prob-00/answer.tex') 
\end{python0}


\begin{prop}
$=$ is an equivalence relation, i.e.,
\begin{myenum}
\item \textsc{Reflexive:} $x = x$
\item \textsc{Symmetric:} If $x = y$, then $y = x$.
\item \textsc{Transitive:} If $x = y$ and $y = z$, then $x = z$.
\end{myenum}
\qed
\end{prop}

%-*-latex-*-

\begin{ex} 
  \label{ex:prob-00}
  \tinysidebar{\debug{exercises/{disc-prob-28/question.tex}}}

  \solutionlink{sol:prob-00}
  \qed
\end{ex} 
\begin{python0}
from solutions import *
add(label="ex:prob-00",
    srcfilename='exercises/discrete-probability/prob-00/answer.tex') 
\end{python0}


\begin{defn}
  A
  \defone{language}
  $L$ over $\Sigma$ is just a subset of $\Sigma^*$.
  $\Sigma$ is the
  \defone{alphabet}
  of $L$.
  Elements of $L$ are called
  \sidebarskip{12pt}\defone{words}
  in $L$.
\end{defn}
