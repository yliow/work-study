\section{The Equals Relation}

Most authors define $=$ using intuition: $x=y$ if after omitting
$\ep$ from $x$ and $y$, the alphabets in $x$ and $y$ occur in the
same order. Here's a more formal definition. First we will define
$=$ using recursion. We will then note that $=$ is an equivalence
relation. Also, we will then show that every string is equivalence
to either $\ep$ or one without $\ep$. So formally, you can think of
$\ep$ and strings without $\ep$ as a complete set of representatves
for $\Sigma^*$.

\begin{defn}
Given a string $x$ over $\Sigma$. We define the relation $=$ on
$\Sigma^*$ inductively as follows.
\begin{itemize}
\item[(1)] $\ep = \ep$
\item[(2)] $a = a$ for all $a \in \Sigma$
\item[(3)] If $x = y$ where $x,y \in \Sigma^*$, then $x = \ep y$
\item[(4)] If $x = y$ where $x,y \in \Sigma^*$, then $x = y \ep$
\item[(5)] If $x = y$ where $x,y \in \Sigma^*$, then $\ep x = y$
\item[(6)] If $x = y$ where $x,y \in \Sigma^*$, then $x \ep = y$
\item[(7)] If $x = y$ where $x,y \in \Sigma^*$, then $ax = ay$ for any $a \in \Sigma$.
\item[(8)] If $x = y$ where $x,y \in \Sigma^*$, then $xa = ya$ for any $a \in \Sigma$.
\end{itemize}
We say $x$ \defone{equals} $y$
if $x=y$.
\end{defn}



\newpage
\begin{ex}
Let $\Sigma = {0,1}$. Verify that $\ep 011 \ep 1 = 0 \ep\ep 111
\ep$ using the above definition of $=$.
\end{ex}

\SOLUTION
\begin{align*}
 1 &= 1 \cr
\THEREFORE         1 &= 1\ep                  & & \text{by \underline{\textwhite{AAA}}} \cr
\THEREFORE     \ep 1 &= 1\ep                  & & \text{by \underline{\textwhite{AAA}}} \cr
\THEREFORE   1 \ep 1 &= 11 \ep                & & \text{by \underline{\textwhite{AAA}}} \cr
\THEREFORE  11 \ep 1 &= 111 \ep                 \cr
\THEREFORE  11 \ep 1 &= \ep 111 \ep             \cr
\THEREFORE  11 \ep 1 &= \ep\ep 111 \ep          \cr
\THEREFORE 011 \ep 1 &= 0 \ep\ep 111 \ep        \cr
\THEREFORE \ep 011 \ep 1 &= 0 \ep\ep 111 \ep    \cr
\end{align*}
Hence $\ep 011 \ep 1 = 0 \ep\ep 111 \ep$.
\qed

\newpage
\begin{ex} Prove or disprove: Let $x, y \in \Sigma^*$. Then $xy = yx$.
\end{ex}


\newpage
\begin{prop}
$=$ is an equivalence relation.
\qed
\end{prop}

\begin{ex}
You know this is coming: Prove the above statement.
\end{ex}





\newpage
\begin{defn}
  A
  \defone{language}
  $L$ over $\Sigma$ is just a subset of $\Sigma^*$.
  $\Sigma$ is the
  \defone{alphabet}
  of $L$.
  Elements of $L$ are called
  \sidebarskip{12pt}\defone{words}
  in $L$.
\end{defn}
