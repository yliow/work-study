%-*-latex-*-
%
% https://www.fluentcpp.com/2019/10/29/stdless-and-its-modern-evolution/
%

\sectionthree{API}
\begin{python0}
from solutions import *; clear()
\end{python0}

Here I'm assuming the implementation is an array-based implementation
(for us, this means C array or C++ \verb!std::vector! objects).
Here are some very common operations:

\begin{longtable}{l|l}
\hline
\texttt{is\_heap}                   & true if it's a heap \\
\texttt{heap\_insert}               & insert a value into a heap \\
\texttt{heap\_delete/extract-root}  & delete max (min) in a maxheap (minheap) \\
\texttt{build\_heap}           & create a heap from the array) \\
\texttt{root}                  & return reference to the root of the heap \\
\texttt{heapsort}              & heapsort on array \\
\hline
\end{longtable}
(Another important operation is to merge two heaps into one.)
Since I am using an array or \verb!std::vector!, the function \verb!root!
is redundant.

It's kind of annoying to have to use max/min in the names above.
The simplest thing to do is to define a comparison function that the
heap functions/class uses.
The boolean function basically compares two nodes and tells you
which node should be \lq\lq higher" in the heap tree.
