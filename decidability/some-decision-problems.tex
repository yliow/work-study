\sectionthree{Some decision problems}
\begin{python0}
from solutions import *; clear()
\end{python0}

Decision problems need not be about problems on automatas to be solved
(decided) by Turing deciders.
They can be problems
about pretty much any discrete mathematical structures
that we hope to solve using Turing deciders.
Just so you see that the idea of decision can be
very general, here are some examples.

\newpage
\subsection{Some decision problems for undirected graphs}

  An (undirected) \defone{graph} is a finite bunch of dots and a bunch of lines where
  each line joins two dots.
  The dots are called \defone{vertices} and the lines are called
  \defone{edges}.
  (Refer to your discrete math textbook for more details.)
Here's the picture of a graph (probably the most famous of all graphs):
\begin{center}
\begin{tikzpicture}[-,>=stealth',auto,node distance=2cm,
  thick,main node/.style={circle,fill=blue!20,draw}]

  \node[main node] (1) {$n$};
  \node[main node] (2) [below of=1] {$e$};
  \node[main node] (3) [right of=2] {$w$};
  \node[main node] (4) [below of=2] {$s$};

  \path[every node/.style={font=\sffamily\small}]
    (1) edge [bend left]   node[right]  {$e_2$} (2)
        edge [bend right]  node[left]  {$e_1$} (2)
        edge [bend left]   node[right] {$e_3$} (3)
    (2) edge [bend left]   node[right]  {$e_6$} (4)
        edge [bend right]  node[left]  {$e_5$} (4)
        edge               node[above] {$e_4$} (3)
    (4) edge [bend right]  node[right] {$e_7$} (3)
  ;
\end{tikzpicture}
\end{center}
If you like, you can think of the vertices as places and
the edges as roads.
If the edges of a graph have arrows, then you think of
the roads as enforcing directions and you can only travel on the road
in the direction specified by the arrow.

Technically speaking when one say a \lq\lq graph'', one is
refering to graphs where there is at most one edge joining
two vertices and furthermore a edge cannot join the same vertex.
The above is an example of \defone{multigraph}
where there can be multiple edges joining two vertices
and where there can be loops (a \defone{loop}
is an edge joining the same vertex).
The important thing about a graph
is not the way the vertices and edges are named,
but rather the structure of the graph, i.e.,
how the vertices are connected.

Suppose I rename the vertices like this and I ignore names
for the edges:
\begin{center}
\begin{tikzpicture}[-,>=stealth',auto,node distance=2cm,
  thick,main node/.style={circle,fill=blue!20,draw}]

  \node[main node] (1) {{\small $1^1$}};
  \node[main node] (2) [below of=1] {{\small $1^2$}};
  \node[main node] (3) [right of=2] {{\small $1^3$}};
  \node[main node] (4) [below of=2] {{\small $1^4$}};

  \path[every node/.style={font=\sffamily\small}]
    (1) edge [bend left]   node[right]  {} (2)
        edge [bend right]  node[left]  {} (2)
        edge [bend left]   node[right] {} (3)
    (2) edge [bend left]   node[right]  {} (4)
        edge [bend right]  node[left]  {} (4)
        edge               node[above] {} (3)
    (4) edge [bend right]  node[right] {} (3)
  ;
\end{tikzpicture}
\end{center}
Mathematically, the definition of a multigraph
looks like this:
\[
G = (V, E)
\]
where
\[
V = \{1^1, 1^2, 1^3, 1^4\}
\]
and
\begin{align*}
E = \{2(1^1, 1^2),\, 2(1^2, 1^1),\, 2(1^2,1^4),\, 2(1^4,1^2),\, 1(1^1,1^3),\, \\
\hspace{1cm}     1(1^3,1^1),\, 1(1^2,1^3),\, 1(1^3,1^2),\, 1(1^3,1^4),\, 1(1^4,1^3) \}
\end{align*}
The value
\[
2(1^1, 1^2)
\]
in $E$ tells us that you can go from $1^1$ to $1^2$ in 2 ways.
(Aside: Something like $\{a, a, b, c, c\}$ is a set
and repetitions are not significant, i.e.,
$\{a, a, b, c, c\} = \{a, b, c\}$.
Something like
$\{2 \cdot a, 1 \cdot b, 2 \cdot\}$
is called a \defone{multiset}
-- it's like a set except that
repetitions \textit{are} significant.)

To encode $G$, we encode $V$ and $E$.
For $V$ we can do this:
\[
\langle V \rangle = 1^1 0 1^2 0 1^3 0 1^4
\]
and for each value in $E$, there are three values and we encode them
in the obvious way and separate the three values by 0.
For instance
\[
\langle 2(1^1, 1^2) \rangle
=
1^2 0 1^1 0 1^2
\]
To encode $E$, say $E = \{t_1, t_2, t_3, \cdots\}$, we encode each value
$t_i$ in $E$
and separate them by $00$:
\begin{align*}
\langle E \rangle = 1^201^101^2 00 1^201^201^1 00 \cdots 00 1^1 0 1^4 0 1^3
\end{align*}

We then encode $G$ by concatenating the encoding of $V$, 000,
and the encoding of $E$:
\begin{align*}
\langle G \rangle
&= \langle V \rangle 000 \langle E \rangle \\
&= 1^1 0 1^2 0 1^3 0 1^4 \cdot 
   000 \cdot
1^201^101^2 00 1^201^201^1 00 \cdots 00 1^1 0 1^4 0 1^3
\end{align*}
With the above encoding, Turing machines can now analyze and compute with graphs.

Here are some very famous decision problems for graphs:
\begin{tightlist}
  \item \textsc{EulerianPath}: Given a multigraph $G$, is there an Euler path in $G$?
  (i.e., a path that visits every edge exactly once).
  \item \textsc{EulerianCycle}: Given a multigraph $G$, is there a closed Euler path in $G$?
  (i.e., a path that visited every edge exactly once and the first vertex in the path
  is also the last).
  \item \textsc{HamiltonianPath}: Given a graph $G$, is there a
  Hamilton path in $G$?
  (i.e., a path that visits every node exactly once).
  \item \textsc{HamiltonianCycle}: Given a multigraph $G$, is there a closed Euler path in $G$?
  (i.e., a path that visited every node exactly once and the first vertex in the path
  is also the last).
\end{tightlist}
In fact they are all decidable problems.

%-*-latex-*-

\begin{ex} 
  \label{ex:prob-00}
  \tinysidebar{\debug{exercises/{disc-prob-28/question.tex}}}

  \solutionlink{sol:prob-00}
  \qed
\end{ex} 
\begin{python0}
from solutions import *
add(label="ex:prob-00",
    srcfilename='exercises/discrete-probability/prob-00/answer.tex') 
\end{python0}


But that's not the end of the story -- there's more.

You can divide Turing machines into deciders and non-deciders.
Among Turing deciders, you can further divide them in different ways.
One way to classify deciders is by their \lq\lq runtime performance''.
Meaning to say the following:
Suppose $M$ is a decider and $w$ is an input for $M$.
You can measure the \lq\lq runtime performance'' of $M$ when it runs $w$ by
counting the number of transitions to a halting state (acccept or reject state).
So the \lq\lq number of transitions'' to reach a halting state can be used as a
measurement of time used.
Based on this concept of time, one can divide problems into different
classes.
This gives a measure of complexity of a problem.
For details see notes on time complexity classes.

The interesting thing about the
\textsc{HamiltonianCycle}
and
\textsc{EulerianCycle}
is that despite the fact that they are so similar,
the Eulerian cycle problem can be decided in a way that
is very different from the way a Hamiltonian cycle problem is decided:

There is a decider of the Eulerian cycle that has a polynomial
runtime while nobody knows if there Hamiltonian cycle can be
decided in polynomial runtime.
When I say \lq\lq polynomial runtime'', I mean
that there is a decider $M$ and a polynomial function $p(n)$
such that for any Eulerian problem, i.e., $\langle G \rangle$ where $G$ is a multigraph,
$M$ will run with $O(p(n))$ number of transitions
where $n$ is the length $\langle G \rangle$.
The class of problems that can be solved by deciders with polynomial runtime
is denoted by $\textsc{P}$.
Therefore we would write
\[
\textsc{EulerianCycle} \in \textsc{P}
\]
You can and should think of problems
(or rather their languages) in $\textsc{P}$ as
problems that can be solved
quickly.

However we do not know if there's a decider for the
Hamiltonian cycle that
runtimes in polynomial runtime.
$\textsc{HamiltonianCycle}$ belongs to a class of problems
denoted by $\textsc{NP}$.
You can and should think of the problems in
$\textsc{NP}$ as being
problems which are probably difficult and needs
more time to solve.
The class $\textsc{NP}$ contains $\textsc{P}$.
One of the biggest unsolved problems in computer science
and mathematics
is whether $\textsc{P}$ is actually $\textsc{NP}$.

For details see notes on time complexity classes.


\newpage
\subsection{More decision problems}

%-*-latex-*-

\begin{ex} 
  \label{ex:prob-00}
  \tinysidebar{\debug{exercises/{disc-prob-28/question.tex}}}

  \solutionlink{sol:prob-00}
  \qed
\end{ex} 
\begin{python0}
from solutions import *
add(label="ex:prob-00",
    srcfilename='exercises/discrete-probability/prob-00/answer.tex') 
\end{python0}


%-*-latex-*-

\begin{ex} 
  \label{ex:prob-00}
  \tinysidebar{\debug{exercises/{disc-prob-28/question.tex}}}

  \solutionlink{sol:prob-00}
  \qed
\end{ex} 
\begin{python0}
from solutions import *
add(label="ex:prob-00",
    srcfilename='exercises/discrete-probability/prob-00/answer.tex') 
\end{python0}


%-*-latex-*-

\begin{ex} 
  \label{ex:prob-00}
  \tinysidebar{\debug{exercises/{disc-prob-28/question.tex}}}

  \solutionlink{sol:prob-00}
  \qed
\end{ex} 
\begin{python0}
from solutions import *
add(label="ex:prob-00",
    srcfilename='exercises/discrete-probability/prob-00/answer.tex') 
\end{python0}


%-*-latex-*-

\begin{ex} 
  \label{ex:prob-00}
  \tinysidebar{\debug{exercises/{disc-prob-28/question.tex}}}

  \solutionlink{sol:prob-00}
  \qed
\end{ex} 
\begin{python0}
from solutions import *
add(label="ex:prob-00",
    srcfilename='exercises/discrete-probability/prob-00/answer.tex') 
\end{python0}

