\tinysidebar{\debug{exercises/{decision-tm0/answer.tex}}}
We will reduce to this the acceptance problem, i.e.,
\[
\ACCEPT_\TM \leq \SIZEONE_\TM
\]
Suppose $\SIZEONE_\TM$ is decidable, say it's decided by $S$.
Then I claim that I can construct a decider for $\ACCEPT_\TM$.

Construct $A$ as follows.
Suppose $\langle M \rangle \# \langle w \rangle$ is given
as an input to $A$ where
$M$ is a TM and $w$ is an input for $M$.
(If the input is not of the form $\langle M \rangle \# \langle w \rangle$,
$A$ rejects immediately.)
$A$ will first construct $M_w$, a Turing machine
that works exactly like $M$ if the input is $w$
but if the input is not $w$, $M$ will reject after checking that the
input is not $w$.
In other words $L(M_w) = \{\}$ or $\{w\}$.
$A$ then does the following:
\begin{tightlist}
  \li $A$ will first simulate
  $S$ with $M_w$.
  Note that $S$ is a decider and therefore must halt.
  \li If while $A$ is simulating
  $S$, $A$ sees that $S$ halts in the accept state, then
  $A$ will enter its accept state.
  [Note that $S$ accepts iff $M_w$
  accept $w$ since by definition $M_w$ can only accept $w$
  and nothing else.]
  \li
  If while $A$ is simulating $S$, $A$ sees that $S$
  halts in the reject state, then $A$ will enter its reject state.
  [Note that $S$ rejects iff $M_w$ rejects
  which means that $M$ does not accept $w$.]
  \li Again, note that $S$ is a decider and therefore must enter
  the accept or reject state at some point -- i.e., $S$
  cannot run forever.
\end{tightlist}
The above means that $A$ accepts $\langle M \rangle \# \langle w \rangle$
iff $M$ accepts $w$. Furthermore $A$ is a decider.
We have shown that
But this means that $\ACCEPT_\TM$ is decidable (by the decider $A$)
which is a contradiction.
