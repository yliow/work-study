\tinysidebar{\debug{exercises/{decision-dfa0/answer.tex}}}
1.
If $M$ is a DFA, then $L(M) = \{\}$ exactly when there is no
path (in the DFA diagram) from the initial state to any of its
accept state. Therefore we do the following:

Construct a TM $D$ that accepts $\langle M \rangle$
(where $M$ is a DFA) and does the following
(if the input is not the encoding of a DFA, $D$ rejects right away).
The tape for $A$ is of the form
\[
\$ \langle M \rangle \# x \# y
\]
First $A$ writes the encoding of the initial state $q_0$:
\[
\$ \langle M \rangle \# \# \langle q_0 \rangle
\]
$y$ collects the states which are \lq\lq not processed''
while $x$ collects the states which are already \lq\lq processed''.
$A$ will process each state $q$ by computing what states
$q$ can go to via transitions.
The states that $q$ can go to and which are not stored in $x$ and $y$
and stored in $y$.
Once a state is processed, the state is removed from $y$ and
moved to $x$.
If an accept state is stored in $x$, then this means that
there's a path from the initial state of $M$ to 
an accept state of $M$.
Therefore if $A$ sees an accept state in $x$, $A$ enters its reject state.
If $y$ is empty (and no accept state of $M$ appears in $x$),
then $A$ enters its accept state.

3. Construct a TM $A$ such that
when given $\langle M \rangle$, $A$ checks if the initial state is an accept
state. If it is so, $A$ enters its accept state.
Otherwise $A$ enters its reject state.
Clearly $L(A) = \{ \langle M \rangle \mid M \text{ is a DFA},
\ep \in L(M)\}$.

4. HINT:
If $M$ is a DFA, $L(M)$ is infinite iff in the DFA diagram
there is a path from the start state to an accept that contains a loop.


6. HINT: $w$ is in $L(M)$ is $w$ is a path in the DFA diagram of $M$
from $M$'s initial state to one of its accept states.

7. Design a TM $A$ that checks that the input is valid and then
accept. Otherwise, reject. (Right?)

8. HINT: Let $L_i = L(M_i)$.
Let $L = ((L_1 \cap \overline{L_2}) \cup (L_2 \cap \overline{L_1}))$
(the symmetric difference of $L_1$ and $L_2$).
Then $L_1 = L_2$ iff $L = \{\}$.

