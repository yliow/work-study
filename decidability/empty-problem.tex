\sectionthree{The Empty Problem}
\begin{python0}
from solutions import *; clear()
\end{python0}

Here's another problem:

$\EMPTYPROBLEM_\TM$\tinysidebar{$\EMPTYPROBLEM_\TM$}\index{emptyproblem@$\EMPTYPROBLEM_\TM$}: 
\text{Given Turing machine $M$, decide if $L(M) = \{\}$ }

Let me give you the corresponding language:
\[
\EMPTY_\TM
= \{ \langle M \rangle \mid L(M) = \emptyset \}
\]
I will show you that $\EMPTY_\TM$ is not decidable.

Suppose $\EMPTY_\TM$ is decided by Turing decider $E$.
I'm going to design a Turing machine $E'$ that will decide
$\ACCEPT_\TM$.

For an input of $\langle M \rangle \# \langle w \rangle$
to $E'$, $E'$ will simulate 
$\langle M_w \rangle$ on $E$.
Now what is this $M_w$?!?
This is a slight modification to $M$.
Basically, $M_w$ first checks if an input is $w$.
If the input is not $w$, $M_w$ rejects right away.
If the input is $w$, $M_w$ behaves just like $M$ on the input of $w$.
It's clear that given $\langle M \rangle$,
you can easily modify it to become $\langle M_w \rangle$.
The important thing is either
\[
M_w = \{\}
\]
or 
\[
M_w = \{w\}
\]
There are no other cases.
Note that the case of 
\[
M_w = \{w\}
\]
implies that $M_w$ will of course halt (in its accept state)
on input $w$.
On the other hand, the case of
\[
M_w = \{\}
\]
might be the case of either 
\[
M_w(w) = 0
\]
i.e., $M_w$ halts in its reject state or 
\[
M_w(w) = \infty
\]
i.e., $M_w$ runs forever.
Ahh .... but that's not a problem!!!
Why? ... Because we can use $E$ to determine if 
\[
M_w = \{\}
\]
or not
and therefore we can avoid running $M_w$ on $w$ which 
can potentially run forever!!! 
See it!!!

So this is how I'm going to construct $E'$.
On input $\langle M \rangle \# \langle w \rangle$,
$E'$ will simulate 
$E$ with input $\langle M_w \rangle$.
(Remember that it's easy to construct 
$\langle M_w \rangle$ from $\langle M \rangle$.)

\begin{tightlist}

\item If the simulation of $E$ with input 
$\langle M_w \rangle$ results in an accept state, i.e.,
\[
E(\langle M_w \rangle) = 1
\]
then 
\[
M_{w} = \{\}
\]
and we know that $M$ does not accept $w$.
In this case, I make $E'$ go to its reject state, 
i.e.,
\[
E'(\langle M \rangle \# \langle w \rangle) = 0
\]
(You'll see why I have to invert 0 and 1 later.)

\item On the other hand if the simulation goes into the reject state,
\[
E(\langle M_w \rangle) = 0
\]
then
\[
M_{w} \neq \{\}
\]
But if $M_w \neq \{\}$, then $M_w$ must be $\{w\}$.
which means that $M$ must accept $w$.
At this point, I make $E'$ goes into its accept state, 
i.e.,
\[
E'(\langle M \rangle \# \langle w \rangle) = 1
\]
\end{tightlist}
Note that since $E$ is a decider, the above are the only cases,
i.e., we cannot possibly have $E(\langle M_w \rangle) = \infty$.
By design, 
\begin{align*}
E(\langle M_w \rangle) = 0 
&\implies E'(\langle M \rangle \# \langle w \rangle) = 1 \\
E(\langle M_w \rangle) = 1 
&\implies E'(\langle M \rangle \# \langle w \rangle) = 0
\end{align*}
$E$ and $E'$ cannot have $\infty$ as results.
Therefore
\begin{align*}
E(\langle M_w \rangle) = 0
&\iff E'(\langle M \rangle \# \langle w \rangle) = 1
\end{align*}

We also have the following based on the definition of $E$, 
i.e., that it's a decider for $\EMPTY_\TM$
and the definition of $M_w$:
\begin{align*}
E(\langle M_w \rangle) = 1 &\implies M_w = \{\} \implies M(w) = 0, \infty 
\implies M \text{ does not accept } w \\
E(\langle M_w \rangle) = 0 &\implies M_w \neq \{\} \implies M(w) = 1
\implies M \text{ accepts } w
\end{align*}
(Note that we actually did not run or simulate $M$ on $w$:
this is key because the execution of $M$ with input $w$ might go on
forever!)
In other words
\begin{align*}
E(\langle M_w \rangle) = 0
&\iff 
M \text{ accepts } w 
\end{align*}
Putting everything together we have
\begin{align*}
E'(\langle M \rangle \# \langle w \rangle) = 1
\iff
E(\langle M_w \rangle) = 0
\iff
M \text{ accepts } w
\end{align*}

Altogether, we have constructed a decider $E'$ which can 
decide the acceptance problem, i.e.,
\[
L(E') = \ACCEPT_\TM
\]
But this contradicts the fact that $\ACCEPT_\TM$ is not a 
Turing--decidable language from an earlier section.

Done!

\begin{thm}
$\EMPTY_\TM$ is not a Turing--decidable language.
\end{thm}

In the above proof, I reduce the $\EMPTY_\TM$ to $\ACCEPT_\TM$:
\[
\ACCEPT_\TM \leq \EMPTY_\TM
\]




\begin{ex} 
  \label{ex:some-decision1}
  \tinysidebar{\debug{exercises/{empty0/question.tex}}}
  \solutionlink{sol:some-decision1}
  \qed
\end{ex} 
\begin{python0}
from solutions import *
add(label="ex:some-decision1",
    srcfilename='exercises/some-decision1/answer.tex') 
\end{python0}


(Hint: Given $M$ and $w$.
You want to know if there's an algorithmic (i.e., by a decider) way
to tell if $M$ halts on input $w$.
If $M$ accepts $w$, then of course it does halt.
However if $M$ does not accept $w$, it can either 
go to its reject state and halt or it can run forever.
Therefore you want to prevent running $M$ on $w$
if it runs on forever.
Not a problem!
We simply run $M_w$ through a purported decider for 
$\EMPTY$.
If $M_w$ is in $\EMPTY$, then $M$ does not accept $w$.
If $M_w$ is not in $\EMPTY$, then $M$ accepts $w$.
Therefore $M$ must halt on $w$.)


