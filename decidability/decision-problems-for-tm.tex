\section{More decision problems for TM}

The acceptance problem and halting problem earlier are examples
of what we call \defone{decision problems for Turing machines}.
You can of course rephrase them for NFAs,
context-free grammars (or PDAs)
and DFAs.

Here are more decision problems for TMs.

\newpage
\newcommand\SIZEONE{\mathsc{Size-One}}
\begin{ex}
Is the following language decidable:
\[
\SIZEONE_\TM = \{ \langle M \rangle \mid |L(M)| = 1 \}
\]
\end{ex}

Do not peek ... the solution is on the next page.


\newpage
\SOLUTION

We will reduce to this the acceptance problem, i.e.,
\[
\ACCEPT_\TM \leq \SIZEONE_\TM
\]
Suppose $\SIZEONE_\TM$ is decidable, say it's decided by $S$.
Then I claim that I can construct a decider for $\ACCEPT_\TM$.

Construct $A$ as follows.
Suppose $\langle M \rangle \# \langle w \rangle$ is given
as an input to $A$ where
$M$ is a TM and $w$ is an input for $M$.
(If the input is not of the form $\langle M \rangle \# \langle w \rangle$,
$A$ rejects immediately.)
$A$ will first construct $M_w$, a Turing machine
that works exactly like $M$ if the input is $w$
but if the input is not $w$, $M$ will reject after checking that the
input is not $w$.
In other words $L(M_w) = \{\}$ or $\{w\}$.
$A$ then does the following:
\begin{tightlist}
  \li $A$ will first simulate
  $S$ with $M_w$.
  Note that $S$ is a decider and therefore must halt.
  \li If while $A$ is simulating
  $S$, $A$ sees that $S$ halts in the accept state, then
  $A$ will enter its accept state.
  [Note that $S$ accepts iff $M_w$
  accept $w$ since by definition $M_w$ can only accept $w$
  and nothing else.]
  \li
  If while $A$ is simulating $S$, $A$ sees that $S$
  halts in the reject state, then $A$ will enter its reject state.
  [Note that $S$ rejects iff $M_w$ rejects
  which means that $M$ does not accept $w$.]
  \li Again, note that $S$ is a decider and therefore must enter
  the accept or reject state at some point -- i.e., $S$
  cannot run forever.
\end{tightlist}
The above means that $A$ accepts $\langle M \rangle \# \langle w \rangle$
iff $M$ accepts $w$. Furthermore $A$ is a decider.
We have shown that
But this means that $\ACCEPT_\TM$ is decidable (by the decider $A$)
which is a contradiction.



\newpage
\begin{ex}
Let $k \geq 0$ be an integer.
Is the following language decidable:
\[
\{ \langle M \rangle \mid |L(M)| = k \}
\]
\end{ex}


\newpage
\begin{ex}
Is the following language decidable:
\[
\{ \langle M \rangle \mid |L(M)| \text{ is finite} \}
\]
\end{ex}




\newpage
\newcommand\INFINITE{\mathsc{Infinite}}
\begin{ex}
Is the following language decidable:
\[
\INFINITE_\TM = \{ \langle M \rangle \mid L(M) \text{ is infinite } \}
\]
\end{ex}


\newpage
\newcommand\Null{\mathsc{Null}}
\begin{ex}
Is the following language decidable:
\[
\Null_\TM = \{ \langle M \rangle \mid \ep \in L(M) \}
\]
\end{ex}


\newpage
\SOLUTION

Assume that $\Null_\TM$ is decidable, say it is decided by $D$.

Construct a TM $A$ that does this:
Let $\langle M \rangle \# \langle w \rangle$ be an input for $A$.
\begin{tightlist}
  \item $A$ construct a TM $M'$ (which depends on $M$ and $w$)
  which does the following:
  When $M'$ executes, it replaced the input with $w$,
  and then proceeds to
  work exactly like $M$.
  Note that if we run $M'$ on $\ep$, then it would be the
  same as running $M$ on $w$.
  Of course when I say $A$ \lq\lq constructs'' $M'$, I  mean $A$ write
  the encoding, $\langle M'\rangle$ somewhere on its tape.
  \item
  $A$ then simulates $D$ on $\langle M' \rangle$.
  Note that $D$ is decidable therefore $D$ must either accept or reject
  ($D$ cannot run forever).
  This means that as $A$ simulates $D$ one step at a time
  $A$ will at some point in time see $D$ entering the
  accept or the reject state.
  \item If while $A$ is simulating $D$ running on $\langle M'\rangle$,
  $A$ sees that $D$ accepts $\langle M' \rangle$ (i.e., 
  $M'$ accepts $\ep$, which means that $M$ accepts $w$), then
  $A$ enters its accept state.
  \item If while $A$ is simulating $D$ running on $\langle M'\rangle$,
  $A$ sees that $D$ rejects $\langle M' \rangle$ (i.e., 
  $M'$ does not accepts $\ep$, which means that $M$ does not
  accepts $w$), then
  $A$ enters its reject state.
  \end{tightlist}
We have shown that
\[
\ACCEPT_\TM \leq \Null_\TM
\]
This means that $A$ is a decider for $\ACCEPT_\TM$ which is a contradiction.
Therefore $\Null_\TM$ cannot be a decidable language.
\qed

As exercise, try to write a proof that shows
\[
\mathsc{HALT}_{\TM} \leq \Null_\TM
\]

\newpage
\newcommand\NULL{\mathsc{Null}}
\begin{ex}
Is the following language decidable:
\[
\{ \langle M \rangle \mid M \text{ has 42 states } \}
\]
\end{ex}


\newpage
\SOLUTION

Yes.
Depending on the way TMs are encoded,
design a TM, $D$, such that when $D$ is given $\langle M \rangle$,
$D$ moves the
read-write head to
the section of $\langle M \rangle$ that encodes the set of states
and begin counting the number of states.
If the counting stops before 42, $D$ rejects.
If the counting goes up to 43, $D$ rejects.
Otherwise $D$ accepts.


\newpage
\newcommand\ALL{\mathsc{All}}
\begin{ex}
  Consider the following problem:

  $\mathsc{AllProblem}_\TM$\tinysidebar{$\mathsc{AllProblem}_\TM$}\index{allproblem$\mathsc{AllProblem}_\TM$}:
  Decide if a TM $M$ accepts all strings.
  
Is the following language decidable:
\[
\ALL_\TM\tinysidebar{$\mathsc{All}_\TM$}\index{all$\mathsc{All}_\TM$}
= \{ \langle M \rangle  \,|\, L(M) = \Sigma^*
\text{ where $\Sigma$ is the input alphabet of $M$} \}
\]
\end{ex}


\newpage
\begin{ex}
Is the following language decidable:
\[
\mathsc{Equal}_\TM
\tinysidebar{$\mathsc{Equal}_\TM$}\index{equal$\mathsc{Equal}_\TM$}
=
\{
\langle M_1 \rangle
\#
\langle M_2 \rangle
\,|\, L(M_1) =  L(M_2)
\}
\]
This is related to the problem of determining if
two given TMs accept the same language.
\end{ex}


\newpage
\begin{ex}
Is the following language decidable:
\[
\mathsc{Regular}_\TM
\tinysidebar{$\mathsc{Regular}_\TM$}\index{regular$\mathsc{Regular}_\TM$}
=
\{
\langle M \rangle
\,|\, \exists DFA M' \text{ such that } L(M') =  L(M)
\}
\]
This is the problem of determining if a Turing machine
accepts the same language as a DFA, i.e., the problem of
determining if the language of a Turing machine is regular.
\end{ex}

I'll show
\[
\ACCEPT_\TM \leq \mathsc{Regular}_\TM
\]
Suppose I'm given $\langle M\rangle\#\langle w \rangle$.
Suppose I assume 
$\mathsc{Regular}_\TM$
is decided by a TM $R$.
Basically I then need to construct (using $R$), a decider, say $A$ for
$\ACCEPT_\TM$.
I would expect something like this:
\begin{align*}
M(w) = 1 &\iff R(M) = 1 \iff A(Mw) = 1 \\
M(w) = 0/\infty &\iff R(M) = 0 \iff A(Mw) = 0 
\end{align*}
Now it turns out that for $R$, I cannot use $M$ directly.
I have to modify $M$ to say another TM say I call it $M_w$ so that the
above desired implications become
\begin{align*}
M(w) = 1 &\iff R(M_w) = 1 \iff A(Mw) = 1 \\
M(w) = 0/\infty &\iff R(M_w) = 0 \iff A(Mw) = 0 
\end{align*}

$M_w$ works as follows: on input $x$, if $x = 0^n1^n$, $M_w$ accepts.
If $x \neq 0^n1^n$, $M_w$ runs on $w$ like $M$ on $w$.
In other words
\[
M_w(x) =
\begin{cases}
  1 &\text{ if } x = 0^n1^n \text{ or } M(w) = 1 \\
  ? &\text{ otherwise}
\end{cases}
\]
Now the problem is that in the otherwise case, and this is
important, you can't say it's $0$. It might be $\infty$.
In the \lq\lq if" case, the fact that $x = 0^n1^n$ can be
checked without running forever. But the \lq\lq $M(x) = 1$"
if not true, might run forever, i.e., it can be
\lq\lq $M(x) = 0$" or $\infty$.
In other words, the complete description of $M_w(x)$ is
\[
M_w(x) =
\begin{cases}
  1 &\text{ if } x = 0^n1^n \text{ or } M(w) = 1 \\
  ? &\text{ if } x \neq 0^n1^n \text{ and } M(w) \neq 0 \\
\end{cases}
\]
i.e.,
\[
M_w(x) =
\begin{cases}
  1 &\text{ if } x = 0^n1^n \text{ or } M(w) = 1 \\
  0 &\text{ if } x \neq 0^n1^n \text{ and } M(w) = 0 \\
  \infty &\text{ if } x \neq 0^n1^n \text{ and } M(w) = \infty \\
\end{cases}
\]
Get it? This is very important. Read the above again very carefully.

If $M(w) = 1$, let $M_w$ be a TM that accepts $\Sigma^*$.
If $M(w) = 0$ or $\infty$, let $M_w$ be a TM that accepts
$\{0^n 1^n \mid n \geq 0\}$.
(The pair language $\Sigma^*$ and $\{0^n 1^n \mid n \geq 0\}$
can be replaced a pair $L, L'$ where $L$ is regular
and $L'$ is not regular.)

Basically we want
\[
R(M_w) = 1 \iff M(w) = 1
\]

Let $M,w$ be a turing machine and input.
$M_w$ is defined to the TM that works this way:
\begin{tightlist}
\item If input $x$ is $0^n1^n$, $M_w$ accepts
\item Otherwise, $M_w$ runs $M$ on $w$ and accepts $x$ iff $M$ accepts $w$.
  (and
  $M_w$ rejects $w$ iff $M$ rejects $w$ and
  $M_w$ runs forever on $w$ iff $M$ runs forever on $w$.) 
\end{tightlist}
Note that
\begin{align*}
M(w) = 1 &\implies L(M_w) = \Sigma^* \\
M(w) = 0 \text{ or } \infty &\implies L(M_w) = \{ 0^n1^n \mid n \geq 0 \}
\end{align*}
In other words
\begin{align*}
M(w) = 1 &\iff L(M_w) = \Sigma^* \\
M(w) = 0 \text{ or } \infty &\iff L(M_w) = \{ 0^n1^n \mid n \geq 0 \}
\end{align*}

Why is this important?
Suppose we have $M_w$.
Now I run $R$ on $\langle M_w \rangle$.
What will happen?
Note that $R$ is a decider.
So we have only two (not three) possibilities:
$R(\langle M_w \rangle) = 0$ or $1$.
But hang on ...
$R(\langle M_w \rangle) = 0$ means that $M_w$ is regular and 
$R(\langle M_w \rangle) = 1$ means that $M_w$ is not regular.
The way I have it set up, it's not just that $M_w$ is not
regular, it's not regular in the specific sense that
$L(M_w) = \{0^n 1^n \mid n \geq 0\}$!!!
In other words,
I can add the following to the above:
\begin{align*}
M(w) = 1 &\iff L(M_w) = \Sigma^* \iff R(\langle M_w\rangle) = 1\\
M(w) = 0 \text{ or } \infty &\iff L(M_w) = \{ 0^n1^n \mid n \geq 0 \}
\iff R(\langle M_w\rangle) = 0
\end{align*}

What's the point of \lq\lq $x = 0^1 1^n$"?
Nothing other than the fact that $\{0^n 1^n \mid n \geq 0\}$
is not regular and $\Sigma^*$ is regular.

Now I'm going to build a TM $A$ that does the following:
Let $x$ be an input.
\begin{tightlist}
\item If $x$ is not of the form $M,w$ (NOTE: this check does not run
  forever)
  \begin{tightlist}
  \item $A$ rejects.
  \end{tightlist}
\item Otherwise (i.e., $x = M,w$) $A$ continues as follows:
  \begin{tightlist}
  \item $A$ constructs $M_w$ on its input tape
  \item $A$ runs $R$ on $M_w$
  \item If $R(M_w) = 1$, $A$ accepts, i.e., $A(x) = A(\langle M, w\rangle) = 1$
  \item Otherwise (i.e., $R(M_w) = 0$), $A$ rejects, i.e.,
    $A(x) = A(\langle M, w\rangle) = 0$.
    (NOTE: It's $0$ and not $\infty$ since
    every step does not run forever.)
  \end{tightlist}
\end{tightlist}





\SOLUTION.
Suppose the above problem is decidable,
say $D$ is a Turing decider such that $L(D) = \mathsc{Regular}_\TM$.
Construct a TM $D'$ as follows. $D'$ takes $\langle M \rangle$ as inputs.
$D'$ will run $D$ on $M$.
Since $D$ is a decider, $D(\langle M \rangle)$ can only be $0$ or $1$.
If $D(\langle M \rangle) = 1$, then $D'$ runs $M$ on $\langle M \rangle$
and accept if $M$ accepts $\langle M \rangle$
and rejects if $M$ rejects $\langle M \rangle$.
If $D(\langle M \rangle) = 0$
