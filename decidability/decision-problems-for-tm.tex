\sectionthree{More decision problems for TM}
\begin{python0}
from solutions import *; clear()
\end{python0}

The acceptance problem and halting problem earlier are examples
of what we call \defone{decision problems for Turing machines}.
You can of course rephrase them for NFAs,
context-free grammars (or PDAs)
and DFAs.

Here are more decision problems for TMs.

\newcommand\SIZEONE{\mathsc{Size-One}}
%-*-latex-*-

\begin{ex} 
  \label{ex:prob-00}
  \tinysidebar{\debug{exercises/{disc-prob-28/question.tex}}}

  \solutionlink{sol:prob-00}
  \qed
\end{ex} 
\begin{python0}
from solutions import *
add(label="ex:prob-00",
    srcfilename='exercises/discrete-probability/prob-00/answer.tex') 
\end{python0}


%-*-latex-*-

\begin{ex} 
  \label{ex:prob-00}
  \tinysidebar{\debug{exercises/{disc-prob-28/question.tex}}}

  \solutionlink{sol:prob-00}
  \qed
\end{ex} 
\begin{python0}
from solutions import *
add(label="ex:prob-00",
    srcfilename='exercises/discrete-probability/prob-00/answer.tex') 
\end{python0}


%-*-latex-*-

\begin{ex} 
  \label{ex:prob-00}
  \tinysidebar{\debug{exercises/{disc-prob-28/question.tex}}}

  \solutionlink{sol:prob-00}
  \qed
\end{ex} 
\begin{python0}
from solutions import *
add(label="ex:prob-00",
    srcfilename='exercises/discrete-probability/prob-00/answer.tex') 
\end{python0}


\newcommand\INFINITE{\mathsc{Infinite}}
%-*-latex-*-

\begin{ex} 
  \label{ex:prob-00}
  \tinysidebar{\debug{exercises/{disc-prob-28/question.tex}}}

  \solutionlink{sol:prob-00}
  \qed
\end{ex} 
\begin{python0}
from solutions import *
add(label="ex:prob-00",
    srcfilename='exercises/discrete-probability/prob-00/answer.tex') 
\end{python0}


\newcommand\Null{\mathsc{Null}}
%-*-latex-*-

\begin{ex} 
  \label{ex:prob-00}
  \tinysidebar{\debug{exercises/{disc-prob-28/question.tex}}}

  \solutionlink{sol:prob-00}
  \qed
\end{ex} 
\begin{python0}
from solutions import *
add(label="ex:prob-00",
    srcfilename='exercises/discrete-probability/prob-00/answer.tex') 
\end{python0}


As exercise, try to write a proof that shows
\[
\mathsc{HALT}_{\TM} \leq \Null_\TM
\]

\newcommand\NULL{\mathsc{Null}}
%-*-latex-*-

\begin{ex} 
  \label{ex:prob-00}
  \tinysidebar{\debug{exercises/{disc-prob-28/question.tex}}}

  \solutionlink{sol:prob-00}
  \qed
\end{ex} 
\begin{python0}
from solutions import *
add(label="ex:prob-00",
    srcfilename='exercises/discrete-probability/prob-00/answer.tex') 
\end{python0}


\newcommand\ALL{\mathsc{All}}
%-*-latex-*-

\begin{ex} 
  \label{ex:prob-00}
  \tinysidebar{\debug{exercises/{disc-prob-28/question.tex}}}

  \solutionlink{sol:prob-00}
  \qed
\end{ex} 
\begin{python0}
from solutions import *
add(label="ex:prob-00",
    srcfilename='exercises/discrete-probability/prob-00/answer.tex') 
\end{python0}


%-*-latex-*-

\begin{ex} 
  \label{ex:prob-00}
  \tinysidebar{\debug{exercises/{disc-prob-28/question.tex}}}

  \solutionlink{sol:prob-00}
  \qed
\end{ex} 
\begin{python0}
from solutions import *
add(label="ex:prob-00",
    srcfilename='exercises/discrete-probability/prob-00/answer.tex') 
\end{python0}


%-*-latex-*-

\begin{ex} 
  \label{ex:prob-00}
  \tinysidebar{\debug{exercises/{disc-prob-28/question.tex}}}

  \solutionlink{sol:prob-00}
  \qed
\end{ex} 
\begin{python0}
from solutions import *
add(label="ex:prob-00",
    srcfilename='exercises/discrete-probability/prob-00/answer.tex') 
\end{python0}


I'll show
\[
\ACCEPT_\TM \leq \mathsc{Regular}_\TM
\]
Suppose I'm given $\langle M\rangle\#\langle w \rangle$.
Suppose I assume 
$\mathsc{Regular}_\TM$
is decided by a TM $R$.
Basically I then need to construct (using $R$), a decider, say $A$ for
$\ACCEPT_\TM$.
I would expect something like this:
\begin{align*}
M(w) = 1 &\iff R(M) = 1 \iff A(Mw) = 1 \\
M(w) = 0/\infty &\iff R(M) = 0 \iff A(Mw) = 0 
\end{align*}
Now it turns out that for $R$, I cannot use $M$ directly.
I have to modify $M$ to say another TM say I call it $M_w$ so that the
above desired implications become
\begin{align*}
M(w) = 1 &\iff R(M_w) = 1 \iff A(Mw) = 1 \\
M(w) = 0/\infty &\iff R(M_w) = 0 \iff A(Mw) = 0 
\end{align*}

$M_w$ works as follows: on input $x$, if $x = 0^n1^n$, $M_w$ accepts.
If $x \neq 0^n1^n$, $M_w$ runs on $w$ like $M$ on $w$.
In other words
\[
M_w(x) =
\begin{cases}
  1 &\text{ if } x = 0^n1^n \text{ or } M(w) = 1 \\
  ? &\text{ otherwise}
\end{cases}
\]
Now the problem is that in the otherwise case, and this is
important, you can't say it's $0$. It might be $\infty$.
In the \lq\lq if" case, the fact that $x = 0^n1^n$ can be
checked without running forever. But the \lq\lq $M(x) = 1$"
if not true, might run forever, i.e., it can be
\lq\lq $M(x) = 0$" or $\infty$.
In other words, the complete description of $M_w(x)$ is
\[
M_w(x) =
\begin{cases}
  1 &\text{ if } x = 0^n1^n \text{ or } M(w) = 1 \\
  ? &\text{ if } x \neq 0^n1^n \text{ and } M(w) \neq 0 \\
\end{cases}
\]
i.e.,
\[
M_w(x) =
\begin{cases}
  1 &\text{ if } x = 0^n1^n \text{ or } M(w) = 1 \\
  0 &\text{ if } x \neq 0^n1^n \text{ and } M(w) = 0 \\
  \infty &\text{ if } x \neq 0^n1^n \text{ and } M(w) = \infty \\
\end{cases}
\]
Get it? This is very important. Read the above again very carefully.

If $M(w) = 1$, let $M_w$ be a TM that accepts $\Sigma^*$.
If $M(w) = 0$ or $\infty$, let $M_w$ be a TM that accepts
$\{0^n 1^n \mid n \geq 0\}$.
(The pair language $\Sigma^*$ and $\{0^n 1^n \mid n \geq 0\}$
can be replaced a pair $L, L'$ where $L$ is regular
and $L'$ is not regular.)

Basically we want
\[
R(M_w) = 1 \iff M(w) = 1
\]

Let $M,w$ be a turing machine and input.
$M_w$ is defined to the TM that works this way:
\begin{tightlist}
\item If input $x$ is $0^n1^n$, $M_w$ accepts
\item Otherwise, $M_w$ runs $M$ on $w$ and accepts $x$ iff $M$ accepts $w$.
  (and
  $M_w$ rejects $w$ iff $M$ rejects $w$ and
  $M_w$ runs forever on $w$ iff $M$ runs forever on $w$.) 
\end{tightlist}
Note that
\begin{align*}
M(w) = 1 &\implies L(M_w) = \Sigma^* \\
M(w) = 0 \text{ or } \infty &\implies L(M_w) = \{ 0^n1^n \mid n \geq 0 \}
\end{align*}
In other words
\begin{align*}
M(w) = 1 &\iff L(M_w) = \Sigma^* \\
M(w) = 0 \text{ or } \infty &\iff L(M_w) = \{ 0^n1^n \mid n \geq 0 \}
\end{align*}

Why is this important?
Suppose we have $M_w$.
Now I run $R$ on $\langle M_w \rangle$.
What will happen?
Note that $R$ is a decider.
So we have only two (not three) possibilities:
$R(\langle M_w \rangle) = 0$ or $1$.
But hang on ...
$R(\langle M_w \rangle) = 0$ means that $M_w$ is regular and 
$R(\langle M_w \rangle) = 1$ means that $M_w$ is not regular.
The way I have it set up, it's not just that $M_w$ is not
regular, it's not regular in the specific sense that
$L(M_w) = \{0^n 1^n \mid n \geq 0\}$!!!
In other words,
I can add the following to the above:
\begin{align*}
M(w) = 1 &\iff L(M_w) = \Sigma^* \iff R(\langle M_w\rangle) = 1\\
M(w) = 0 \text{ or } \infty &\iff L(M_w) = \{ 0^n1^n \mid n \geq 0 \}
\iff R(\langle M_w\rangle) = 0
\end{align*}

What's the point of \lq\lq $x = 0^1 1^n$"?
Nothing other than the fact that $\{0^n 1^n \mid n \geq 0\}$
is not regular and $\Sigma^*$ is regular.

Now I'm going to build a TM $A$ that does the following:
Let $x$ be an input.
\begin{tightlist}
\item If $x$ is not of the form $M,w$ (NOTE: this check does not run
  forever)
  \begin{tightlist}
  \item $A$ rejects.
  \end{tightlist}
\item Otherwise (i.e., $x = M,w$) $A$ continues as follows:
  \begin{tightlist}
  \item $A$ constructs $M_w$ on its input tape
  \item $A$ runs $R$ on $M_w$
  \item If $R(M_w) = 1$, $A$ accepts, i.e., $A(x) = A(\langle M, w\rangle) = 1$
  \item Otherwise (i.e., $R(M_w) = 0$), $A$ rejects, i.e.,
    $A(x) = A(\langle M, w\rangle) = 0$.
    (NOTE: It's $0$ and not $\infty$ since
    every step does not run forever.)
  \end{tightlist}
\end{tightlist}

\SOLUTION.
Suppose the above problem is decidable,
say $D$ is a Turing decider such that $L(D) = \mathsc{Regular}_\TM$.
Construct a TM $D'$ as follows. $D'$ takes $\langle M \rangle$ as inputs.
$D'$ will run $D$ on $M$.
Since $D$ is a decider, $D(\langle M \rangle)$ can only be $0$ or $1$.
If $D(\langle M \rangle) = 1$, then $D'$ runs $M$ on $\langle M \rangle$
and accept if $M$ accepts $\langle M \rangle$
and rejects if $M$ rejects $\langle M \rangle$.
If $D(\langle M \rangle) = 0$
