% Options for packages loaded elsewhere
\PassOptionsToPackage{unicode}{hyperref}
\PassOptionsToPackage{hyphens}{url}
%
\documentclass[
]{article}
\usepackage{lmodern}
\usepackage{amssymb,amsmath}
\usepackage{ifxetex,ifluatex}
\ifnum 0\ifxetex 1\fi\ifluatex 1\fi=0 % if pdftex
  \usepackage[T1]{fontenc}
  \usepackage[utf8]{inputenc}
  \usepackage{textcomp} % provide euro and other symbols
\else % if luatex or xetex
  \usepackage{unicode-math}
  \defaultfontfeatures{Scale=MatchLowercase}
  \defaultfontfeatures[\rmfamily]{Ligatures=TeX,Scale=1}
\fi
% Use upquote if available, for straight quotes in verbatim environments
\IfFileExists{upquote.sty}{\usepackage{upquote}}{}
\IfFileExists{microtype.sty}{% use microtype if available
  \usepackage[]{microtype}
  \UseMicrotypeSet[protrusion]{basicmath} % disable protrusion for tt fonts
}{}
\makeatletter
\@ifundefined{KOMAClassName}{% if non-KOMA class
  \IfFileExists{parskip.sty}{%
    \usepackage{parskip}
  }{% else
    \setlength{\parindent}{0pt}
    \setlength{\parskip}{6pt plus 2pt minus 1pt}}
}{% if KOMA class
  \KOMAoptions{parskip=half}}
\makeatother
\usepackage{xcolor}
\IfFileExists{xurl.sty}{\usepackage{xurl}}{} % add URL line breaks if available
\IfFileExists{bookmark.sty}{\usepackage{bookmark}}{\usepackage{hyperref}}
\hypersetup{
  hidelinks,
  pdfcreator={LaTeX via pandoc}}
\urlstyle{same} % disable monospaced font for URLs
\usepackage{longtable,booktabs}
% Correct order of tables after \paragraph or \subparagraph
\usepackage{etoolbox}
\makeatletter
\patchcmd\longtable{\par}{\if@noskipsec\mbox{}\fi\par}{}{}
\makeatother
% Allow footnotes in longtable head/foot
\IfFileExists{footnotehyper.sty}{\usepackage{footnotehyper}}{\usepackage{footnote}}
\makesavenoteenv{longtable}
\setlength{\emergencystretch}{3em} % prevent overfull lines
\providecommand{\tightlist}{%
  \setlength{\itemsep}{0pt}\setlength{\parskip}{0pt}}
\setcounter{secnumdepth}{-\maxdimen} % remove section numbering

\author{}
\date{}

\begin{document}

­68. Array of Objects

Objectives

\begin{itemize}
\tightlist
\item
  Review Array and pointer
\item
  Use Arrays to store objects
\item
  Build static sized Array Objects
\item
  Build static sized Array Object Pointers
\item
  Build dynamic sized Array Object
\item
  Build dynamic sized Array Object Pointers
\end{itemize}

Review

As you know, the size of the array is fixed:

\begin{longtable}[]{@{}l@{}}
\toprule
\endhead
\begin{minipage}[t]{0.97\columnwidth}\raggedright
const int SIZE = 5;

int x{[}2 * SIZE{]}; // OK

int y{[}42{]}; // OK

int i = 5;

int z{[}i{]}; // WRONG!!!\strut
\end{minipage}\tabularnewline
\bottomrule
\end{longtable}

For a ``dynamic'' array, i.e., an array whose size can be variable, you
can use a pointer.

\begin{longtable}[]{@{}l@{}}
\toprule
\endhead
\begin{minipage}[t]{0.97\columnwidth}\raggedright
int * p = new int{[}5{]}; // OK

const int SIZE = 5;

int * q = new int {[}2 * SIZE{]}; // OK

int i = 5;

int * r = new int{[}i{]}; // OK!!!\strut
\end{minipage}\tabularnewline
\bottomrule
\end{longtable}

Memory allocated to a pointer must be de-allocated by the programmer:

\begin{longtable}[]{@{}l@{}}
\toprule
\endhead
\begin{minipage}[t]{0.97\columnwidth}\raggedright
int i = 5;

int * r = new int{[}i{]};

// do something with r{[}0{]},...,r{[}i -- 1{]}

delete {[}{]} r;\strut
\end{minipage}\tabularnewline
\bottomrule
\end{longtable}

Memory can be allocated to a pointer more than once. The sizes can be
different.

\begin{longtable}[]{@{}l@{}}
\toprule
\endhead
\begin{minipage}[t]{0.97\columnwidth}\raggedright
int * r = new int{[}5{]};

// ... do something with r{[}0{]},...,r{[}4{]}

delete {[}{]} r;

...

r = new int;

// ... do something with *r;

delete r;

...

r = new int{[}10{]};

// ... do something with r{[}0{]},...,r{[}9{]}

delete {[}{]} r;\strut
\end{minipage}\tabularnewline
\bottomrule
\end{longtable}

Like a pointer to a structure variable, if p points to an object with
instance variable \emph{x}, then

(*p).x is the same as p-\textgreater x

Similarly if the object has method \emph{m()}, then invoking

\emph{(*p).m()} is the same as \emph{p-\textgreater m()}

Use \emph{-\textgreater{} }instead of \emph{*.}

Array of Objects

So far objects are created individually.

What if you want to create an array of objects

C obj{[}10{]};

where \emph{C} is the class?

Try the following example:

\begin{longtable}[]{@{}l@{}}
\toprule
\endhead
\begin{minipage}[t]{0.97\columnwidth}\raggedright
// Int.h

\#include \textless iostream\textgreater{}

class Int

\{

public:

Int(int x=0)

: x\_(x)

\{

std::cout \textless\textless{} "Int ... x\_: " \textless\textless{} x\_

\textless\textless{} std::endl;

\}

private:

int x\_;

\};\strut
\end{minipage}\tabularnewline
\bottomrule
\end{longtable}

\begin{longtable}[]{@{}l@{}}
\toprule
\endhead
\begin{minipage}[t]{0.97\columnwidth}\raggedright
\#include ``Int.h''

int main()

\{

Int a{[}10{]};

return 0;

\}\strut
\end{minipage}\tabularnewline
\bottomrule
\end{longtable}

The Point: Each \emph{a{[}i{]} (i=0,...,9)} is constructed using the
default constructor. When you declare an array of objects each object in
the array calls the default constructor.

You \textbf{must} have a default constructor if you want to have an
array of objects. (Don't forget that if you don't specify any
constructor, C++ will give you a default constructor.)

Of course you can still call the constructor one at a time like this if
you wish:

Int a{[}3{]} = \{Int(3), Int(-1), Int(42)\};

Now try to run this (why does it not work):

\begin{longtable}[]{@{}l@{}}
\toprule
\endhead
\begin{minipage}[t]{0.97\columnwidth}\raggedright
// vec2d.h

\#include \textless iostream\textgreater{}

class vec2d

\{

public:

vec2d(double x, double y)

: x\_(x), y\_(y)

\{\}

private:

double x\_, y\_;

\};\strut
\end{minipage}\tabularnewline
\bottomrule
\end{longtable}

\begin{longtable}[]{@{}l@{}}
\toprule
\endhead
\begin{minipage}[t]{0.97\columnwidth}\raggedright
\#include ``vec2d.h''

int main()

\{

vec2d v{[}10{]};

return 0;

\}\strut
\end{minipage}\tabularnewline
\bottomrule
\end{longtable}

The point: Each of the \emph{v{[}0{]},...,v{[}9{]}} calls \emph{vec2d()}
(the default constructor) to initialize itself. But there is no default
constructor in \emph{vec2d}. There's only one constructor and it must
receive two doubles.

Remember that! If you want to declare an array of objects without an
initializer list, you \textbf{MUST} have the default constructor used to
construct the objects in the array.

\textbf{Exercise. }Fix the following class so that an array of
\emph{WeatherControl} objects can be created with all instance variables
initialized to 10.

\begin{longtable}[]{@{}l@{}}
\toprule
\endhead
\begin{minipage}[t]{0.97\columnwidth}\raggedright
// WeatherCtrl.h

\#ifndef WEATHERCTRL\_H

\#define WEATHERCTRL\_H

class WeatherCtrl

\{

public:

WeatherCtrl(double temp, double pressure)

: temp\_(temp), pressure\_(pressure)

\{\}

double get\_temp() const \{ return temp\_; \}

double get\_pressure() const \{ return pressure\_; \}

void set\_temp(double);

void set\_pressure(double);

private:

double temp\_;

double pressure\_;

\};

\#endif\strut
\end{minipage}\tabularnewline
\bottomrule
\end{longtable}

Suppose we have the following class:

\begin{longtable}[]{@{}l@{}}
\toprule
\endhead
\begin{minipage}[t]{0.97\columnwidth}\raggedright
class Int

\{

public:

Int(int x)

: x\_(x)

\{\}

private:

int x\_;

\};\strut
\end{minipage}\tabularnewline
\bottomrule
\end{longtable}

Now suppose you want to create an array of \emph{Int} objects with

Int a{[}10{]}; // want a{[}i{]}.x = i

First Method:

\begin{longtable}[]{@{}l@{}}
\toprule
\endhead
\begin{minipage}[t]{0.97\columnwidth}\raggedright
class Int

\{

public:

Int (int x = 0)

: x\_(x) \{\}

void set(int a) \{x\_ = a;\}

private:

int x\_;

\};

int main()

\{

Int a{[}10{]};

for (int i = 0; i \textless{} 10; i++)

a{[}i{]}.set(i);

...

\}\strut
\end{minipage}\tabularnewline
\bottomrule
\end{longtable}

So note that in the above each \emph{a{[}i{]}.x} was first set to
\emph{0} by the constructor ... then \emph{a{[}i{]}.x} was set to
\emph{i}.

\textbf{The Point}: \emph{a{[}i{]}.x} was set two values.

This is no big deal for the \emph{Int} class because not a lot of work
was done in the constructor. But... what if a certain class \emph{C}
does a lot of work in the constructor? Example: \emph{C} has
\emph{1000000} member variables and the default constructor sets all to
\emph{0} and then you need to set them all to \emph{i}?

Second method:

\begin{longtable}[]{@{}l@{}}
\toprule
\endhead
\begin{minipage}[t]{0.97\columnwidth}\raggedright
int main()

\{

Int * a{[}10{]};

for (int i = 0; i \textless{} 10; ++i)

a{[}i{]} = new Int(i);

// do something with *(a{[}i{]})

...

\}\strut
\end{minipage}\tabularnewline
\bottomrule
\end{longtable}

\textbf{The point:} Each \emph{a{[}i{]}} is not an object, but a pointer
to an object. We can then construct the object in the heap with whatever
constructor call that we choose and then point \emph{a{[}i{]}} to the
object.

\textbf{The point:}There is no constructor call when you do

Int * a{[}10{]};

or when you do

Int * p;

\textbf{The point:} There is no ``default constructor call'' for
pointers. Pointers are not objects.

Let's compare the two methods.

In the first method, when we create an array of objects the default
constructor is called for each object.

In the second method, for the array of pointers to objects, you can call
a constructor for each object in the heap that each pointer of the array
points to.

How do you deallocate memory correctly for the second method?

\begin{longtable}[]{@{}l@{}}
\toprule
\endhead
\begin{minipage}[t]{0.97\columnwidth}\raggedright
Int * a{[}10{]};

for (int i = 0; i \textless{} 10; ++i)

a{[}i{]} = new Int(i);

// do something with *(a{[}i{]})

for (int i = 0; i \textless{} 10; ++i)

delete a{[}i{]};\strut
\end{minipage}\tabularnewline
\bottomrule
\end{longtable}

Compare the above with this:

\begin{longtable}[]{@{}l@{}}
\toprule
\endhead
\begin{minipage}[t]{0.97\columnwidth}\raggedright
Int * a = new Int{[}10{]};

for (int i = 0; i \textless{} 10; ++i)

a{[}i{]}.set(i);

delete {[}{]} a;\strut
\end{minipage}\tabularnewline
\bottomrule
\end{longtable}

Make sure you see the difference between

\begin{itemize}
\tightlist
\item
  Array of Pointers
\item
  Pointer to an Array
\end{itemize}

Note that for the second method:

\begin{longtable}[]{@{}l@{}}
\toprule
\endhead
\begin{minipage}[t]{0.97\columnwidth}\raggedright
int main()

\{

Int * a{[}10{]};

for (int i = 0; i \textless{} 10; ++i)

a{[}i{]} = new Int(i);

// do something with *(a{[}i{]})

\}\strut
\end{minipage}\tabularnewline
\bottomrule
\end{longtable}

The size of the array is fixed. To have a ``dynamic'' array where the
size can change, you can use a pointer.

\begin{longtable}[]{@{}l@{}}
\toprule
\endhead
\begin{minipage}[t]{0.97\columnwidth}\raggedright
int size;

std::cin \textgreater\textgreater{} size; // enter integer
\textgreater{} 0

Int ** a = new Int*{[}size{]};

for (int i = 0; i \textless{} 10; ++i)

a{[}i{]} = new Int(i);

...

// Exercise on pointers to objects:

// Assign 43 to x of Int object a{[}0{]}

// is point to. \strut
\end{minipage}\tabularnewline
\bottomrule
\end{longtable}

The above is analogous to this:

\begin{longtable}[]{@{}l@{}}
\toprule
\endhead
\begin{minipage}[t]{0.97\columnwidth}\raggedright
int size;

std::cin \textgreater\textgreater{} size; // enter integer
\textgreater{} 0

Int ** a = new Int*{[}size{]};

for (int i = 0; i \textless{} 10; ++i)

a{[}i{]} = new Int(i);

// ... do something with *a{[}i{]}

for (int i = 0; i \textless{} size; ++i)

delete a{[}i{]};\strut
\end{minipage}\tabularnewline
\bottomrule
\end{longtable}

Note:

Int ** a = new Int*{[}size{]};

Int ** b = new (Int*){[}size{]}; \textbf{//WRONG!!!}

Comparison:

Int a{[}10{]}; //Fixed size, default constructor

Int * b{[}10{]}; //Fixed, can call constructor

Int ** c = new Int*{[}size{]}; // variable size and can

// call constructor.

What if you want to do this:

Date a{[}10{]}; // all 10 initialized to 12/25/2009

Date b{[}20{]}; // all 20 initialized to 12/25/2010

Keep default year, month, day values in Date class for initialization
purposes. Allow defaults to be changed. But default values for the
constructor is fixed.

See notes on Static for details. Such values are more associated with
the class than with individual objects.

\textbf{Exercise. }Write a program that does this:

\begin{itemize}
\tightlist
\item
  Prompt the user for s.
\item
  Create a dynamic array of size s of Int objects initializing them so
  that you get a Fibonacci sequence.
\item
  For instance if you enter 7 the integers modeled in the array are:
\end{itemize}

1, 1, 2, 3, 5, 8, 13

You can also have 2-dim array of objects.

Example:

\begin{longtable}[]{@{}l@{}}
\toprule
\endhead
\begin{minipage}[t]{0.97\columnwidth}\raggedright
Int a{[}5{]}{[}6{]}; // default constructor called on each

// a{[}i{]}{[}j{]}

for (int i = 0; i \textless{} 5; i++)

for (int j = 0; j \textless{} 6; j++)

a{[}i{]}{[}j{]}.set(42);\strut
\end{minipage}\tabularnewline
\bottomrule
\end{longtable}

In this example, the array has fixed sizes, and you loose control over
the constructor call.

Use an array of object pointers to control constructor call

\begin{longtable}[]{@{}l@{}}
\toprule
\endhead
\begin{minipage}[t]{0.97\columnwidth}\raggedright
Int * a{[}5{]}{[}6{]}; // 5-by-6 array of Int *

for (int i = 0; i \textless{} 5; i++)

for (int j = 0; j \textless{} 6; j++)

a{[}i{]}{[}j{]} = new Int(42);

// do something useful with *(a{[}i{]}{[}j{]})

for (int i = 0; i \textless{} 5; i++)

for (int j = 0; j \textless{} 6; j++)

delete a{[}i{]}{[}j{]};\strut
\end{minipage}\tabularnewline
\bottomrule
\end{longtable}

The above is an array of \textbf{fixed} sizes.

So... what if you want the sizes to be dynamic?

\begin{longtable}[]{@{}l@{}}
\toprule
\endhead
\begin{minipage}[t]{0.97\columnwidth}\raggedright
int rows, columns;

// ... prompt user for rows columns

Int * a{[}rows{]}{[}columns{]}; // WRONG!!!\strut
\end{minipage}\tabularnewline
\bottomrule
\end{longtable}

The following is a 2-dimensional dynamic array of objects, i.e., the row
and column size can be variables. However, the value in the array are
objects (not pointers yet \ldots{} we'll get to that in a bit...)

\begin{longtable}[]{@{}l@{}}
\toprule
\endhead
\begin{minipage}[t]{0.97\columnwidth}\raggedright
int rows = 5, columns = 6;

Int ** a = new Int*{[}rows{]}; // a{[}0{]}, ..., a{[}rows-1{]}

// are all Int * values

for (int i = 0; i \textless{} rows; ++i)

\{

a{[}i{]} = new Int{[}columns{]}; // a{[}i{]}{[}j{]} for j = 0,...,

// columns-1 are Int

for (int j = 0; j \textless{} columns; ++j)

a{[}i{]}{[}j{]} = 42;

\}\strut
\end{minipage}\tabularnewline
\bottomrule
\end{longtable}

\begin{longtable}[]{@{}l@{}}
\toprule
\endhead
\begin{minipage}[t]{0.97\columnwidth}\raggedright
int rows = 5, columns = 6;

Int ** a = new Int*{[}rows{]}; // a{[}0{]}, ..., a{[}rows-1{]}

// are all Int * values

for (int i = 0; i \textless{} rows; ++i)

\{

a{[}i{]} = new Int{[}columns{]}; // a{[}i{]}{[}j{]} for j = 0,...,

// columns-1 are Int

for (int j = 0; j \textless{} columns; ++j)

a{[}i{]}{[}j{]} = 42;

\}

// do something with a{[}i{]}{[}j{]}

// De-allocate memory used by a{[}0{]}, \ldots, a{[}rows-1{]} and a

for (int i = 0; i \textless{} rows; i++)

delete {[}{]} a{[}i{]};

delete {[}{]} a;\strut
\end{minipage}\tabularnewline
\bottomrule
\end{longtable}

Exercise. Write a program that does this:

\begin{itemize}
\tightlist
\item
  prompts the user for the row and column size of a 2-dim array of
  integer values. (Just the regular int).
\end{itemize}

\begin{itemize}
\tightlist
\item
  Fill the 2-dim array with random integers.
\item
  Print the values in the array in a 2-dim grid.
\item
  Compute the row, column, and overall average of all the values (up to
  1 decimal place).
\item
  Deallocate Memory used.
\item
  Here is what a 3 by 3 case looks like
\end{itemize}

\begin{longtable}[]{@{}l@{}}
\toprule
\endhead
\begin{minipage}[t]{0.97\columnwidth}\raggedright
0 3 3 \textbar2.0

2 9 0 \textbar3.6

8 2 6 \textbar5.3

-\/-\/-\/-\/-\/-\/-\/-\/-\/-\/-+

3.3 4.6 3.0 3.6\strut
\end{minipage}\tabularnewline
\bottomrule
\end{longtable}

\textbf{Exercise. }Run this program. Explain why it does not work.

\begin{longtable}[]{@{}l@{}}
\toprule
\endhead
\begin{minipage}[t]{0.97\columnwidth}\raggedright
int floors = 3, rooms\_per\_floor = 10;

WeatherControl ** wc = new WeatherControl*{[}floors{]};

for (int floor = 0; floor \textless{} floors; ++floor)

\{

wc{[}floor{]} = new WeatherControl{[}rooms\_per\_floor{]};

\}

delete {[}{]} wc;

for (int floor = 0; floor \textless{} floors; ++floor)

\{

delete {[}{]} wc{[}floor{]};

\}\strut
\end{minipage}\tabularnewline
\bottomrule
\end{longtable}

From previous example, each \emph{a{[}i{]}{[}j{]}} is an \emph{Int}
object. You lose control over calling constructor.

Now we want the row and column sizes to be dynamic and the values in the
array to be pointers so that we can call the constructor for each
pointer\ldots{}

\begin{longtable}[]{@{}l@{}}
\toprule
\endhead
\begin{minipage}[t]{0.97\columnwidth}\raggedright
int rows = 5, columns = 6;

Int *** a = new Int**{[}rows{]};

for (int i = 0; i \textless{} rows; i++)

\{

a{[}i{]} = new Int*{[}columns{]};

for (int j = 0; j \textless{} columns; j++)

a{[}i{]}{[}j{]} = new Int(42);

\}

// do something with (*a){[}i{]}{[}j{]}

for (int i = 0; i \textless{} rows; i++)

\{

for (int j = 0; j \textless{} columns; j++)

delete a{[}i{]}{[}j{]};

delete {[}{]} a{[}i{]};

\}

delete {[}{]} a;\strut
\end{minipage}\tabularnewline
\bottomrule
\end{longtable}

Here a summary:

\begin{itemize}
\tightlist
\item
  For a static sized 2-dim array of \emph{Int} objects:
\end{itemize}

Int a{[}3{]}{[}5{]}

\begin{itemize}
\tightlist
\item
  For a static sized 2-dim array of \emph{Int} pointers:
\end{itemize}

Int * a{[}3{]}{[}5{]}

\begin{itemize}
\tightlist
\item
  For a dynamic sized 2-dim array of \emph{Int} objects:
\end{itemize}

Int ** a

\begin{itemize}
\tightlist
\item
  For a dynamic sized 2-dim array of \emph{Int} pointers:
\end{itemize}

Int *** a

The point:

Int *** a

Exercise.

\begin{itemize}
\tightlist
\item
  Create a 2-dim (dynamic) array of \emph{Int} object pointers. The size
  of the first and second dim is randomly chosen from integer
  1,2,...,10.
\item
  Initialize the pointers by allocating memory for \emph{Int} objects
  and initializing the \emph{Int} objects with integers randomly chosen
  from 1,2,...,10
\item
  Print all the integer values ``in'' this array in a 2-dim grid.
\end{itemize}

All the arrays up to this point have been rectangular, i.e., the arrays
as a container values look like this.

\begin{longtable}[]{@{}lll@{}}
\toprule
\endhead
& &\tabularnewline
& &\tabularnewline
& &\tabularnewline
\bottomrule
\end{longtable}

Create an array that is triangular, i.e. for row 0

there are 3 columns, for row 1 there are 2 columns, etc. (They are
called jagged arrays.)

\begin{longtable}[]{@{}lll@{}}
\toprule
\endhead
& &\tabularnewline
& &\tabularnewline
& &\tabularnewline
\bottomrule
\end{longtable}

For the static size 3-dim array of \emph{Int} objects:

Int a{[}3{]}{[}6{]}{[}4{]}

For the static size 3-dim array of \emph{Int} pointers:

Int * a{[}3{]}{[}6{]}{[}4{]}

For the dynamic size 3-dim array of \emph{Int} objects:

Int *** a

For the dynamic size 3-dim array of \emph{Int} pointers:

Int **** a

Phew!!!!

\textbf{Exercise.} In a function \emph{f(a, b, c)} create a dynamic
array of \emph{Int} pointers with sizes \emph{a, b, c}. Initialize all
the pointers by allocating the memory and initializing the objects the
pointer points to with a random integer. Make sure you de-allocate
memory used.

In your \emph{main()}, write a \emph{for} loop that calls \emph{f(10,
10, 10)} and print a message for each iteration of the loop. If the
printing of the messages slows down, it tells you that you probably have
a memory leak.

\textbf{Exercise. }Write a class that models a dynamic 2-dim array of
integers. The following usage will tell you how it works:

\begin{longtable}[]{@{}l@{}}
\toprule
\endhead
\begin{minipage}[t]{0.97\columnwidth}\raggedright
IntDynArr a(5,6); // 5-by-6 int array

a(1, 2) = 42; // set ``a{[}1{]}{[}2{]}'' to 42

std::cout \textless\textless{} a(1, 2) \textless\textless{}
std::endl;\strut
\end{minipage}\tabularnewline
\bottomrule
\end{longtable}

Hint: \emph{a(1, 2)} is \emph{a.operator()(1, 2)}. You need to define
\emph{int \& operator(int, int)}.

\textbf{Exercise.} Rewrite your class so you can execute this:

\begin{longtable}[]{@{}l@{}}
\toprule
\endhead
\begin{minipage}[t]{0.97\columnwidth}\raggedright
IntDynArr a(5, 6);

a{[}1{]}{[}2{]} = 42;\strut
\end{minipage}\tabularnewline
\bottomrule
\end{longtable}

Hint: Note that

\begin{longtable}[]{@{}l@{}}
\toprule
\endhead
a{[}1{]}{[}2{]}\tabularnewline
\bottomrule
\end{longtable}

Is not \emph{a.operator{[}{]}{[}{]}(1,2)} because there is no
\emph{{[}{]}{[}{]}} operator!!! Instead \emph{a{[}1{]}{[}2{]}} is
\emph{a.operator{[}{]}(1).operator{[}{]}(2)}

Summary

The objects in an array are initialized using the default constructor.

If you want to get away from using the default constructor (i.e., you
want to explicitly call the constructor) but still want a static sized
array, you can use an array of pointers.

If you want to get away from using the default constructor (i.e., you
want to explicitly call the constructor) and you want to get away from a
static sized array, you can use a pointer to an array of pointers.

\end{document}
