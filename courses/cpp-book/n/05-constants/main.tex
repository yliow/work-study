% Options for packages loaded elsewhere
\PassOptionsToPackage{unicode}{hyperref}
\PassOptionsToPackage{hyphens}{url}
%
\documentclass[
]{article}
\usepackage{lmodern}
\usepackage{amssymb,amsmath}
\usepackage{ifxetex,ifluatex}
\ifnum 0\ifxetex 1\fi\ifluatex 1\fi=0 % if pdftex
  \usepackage[T1]{fontenc}
  \usepackage[utf8]{inputenc}
  \usepackage{textcomp} % provide euro and other symbols
\else % if luatex or xetex
  \usepackage{unicode-math}
  \defaultfontfeatures{Scale=MatchLowercase}
  \defaultfontfeatures[\rmfamily]{Ligatures=TeX,Scale=1}
\fi
% Use upquote if available, for straight quotes in verbatim environments
\IfFileExists{upquote.sty}{\usepackage{upquote}}{}
\IfFileExists{microtype.sty}{% use microtype if available
  \usepackage[]{microtype}
  \UseMicrotypeSet[protrusion]{basicmath} % disable protrusion for tt fonts
}{}
\makeatletter
\@ifundefined{KOMAClassName}{% if non-KOMA class
  \IfFileExists{parskip.sty}{%
    \usepackage{parskip}
  }{% else
    \setlength{\parindent}{0pt}
    \setlength{\parskip}{6pt plus 2pt minus 1pt}}
}{% if KOMA class
  \KOMAoptions{parskip=half}}
\makeatother
\usepackage{xcolor}
\IfFileExists{xurl.sty}{\usepackage{xurl}}{} % add URL line breaks if available
\IfFileExists{bookmark.sty}{\usepackage{bookmark}}{\usepackage{hyperref}}
\hypersetup{
  hidelinks,
  pdfcreator={LaTeX via pandoc}}
\urlstyle{same} % disable monospaced font for URLs
\usepackage{longtable,booktabs}
% Correct order of tables after \paragraph or \subparagraph
\usepackage{etoolbox}
\makeatletter
\patchcmd\longtable{\par}{\if@noskipsec\mbox{}\fi\par}{}{}
\makeatother
% Allow footnotes in longtable head/foot
\IfFileExists{footnotehyper.sty}{\usepackage{footnotehyper}}{\usepackage{footnote}}
\makesavenoteenv{longtable}
\usepackage[normalem]{ulem}
% Avoid problems with \sout in headers with hyperref
\pdfstringdefDisableCommands{\renewcommand{\sout}{}}
\setlength{\emergencystretch}{3em} % prevent overfull lines
\providecommand{\tightlist}{%
  \setlength{\itemsep}{0pt}\setlength{\parskip}{0pt}}
\setcounter{secnumdepth}{-\maxdimen} % remove section numbering

\author{}
\date{}

\begin{document}

05. Constants

Objectives

\begin{itemize}
\tightlist
\item
  Choose appropriate names for constants.
\item
  Declare constants
\item
  Use constants in expressions
\end{itemize}

This is a pretty short section. We want to talk about constants. A
constant is just a variable whose value cannot be changed once it's
initialized.

Constants

A \textbf{constant} is just a variable whose \textbf{value cannot
change}.

A constant must be initialized.

This example will show you how to create constant:

\begin{longtable}[]{@{}@{}}
\toprule
\endhead
\bottomrule
\end{longtable}

\textbf{Exercise.} What if we don't initialize but assign after
declaration? Does it work?

\begin{longtable}[]{@{}@{}}
\toprule
\endhead
\bottomrule
\end{longtable}

\textbf{Exercise.} What if we assign after the initialization? Does it
work?

\begin{longtable}[]{@{}@{}}
\toprule
\endhead
\bottomrule
\end{longtable}

The format for the declaration of a constant in C++ looks like this:

const {[}type{]} {[}var name{]} = {[}value{]};

Of course the \emph{{[}value{]}} can be an expression that can evaluate
to a value. The expression must be a \textbf{constant expression}. This
means that the expression cannot have a nonconstant variable. Try these
example:

\textbf{Exercise.} Try this:

\begin{longtable}[]{@{}@{}}
\toprule
\endhead
\bottomrule
\end{longtable}

\textbf{Exercise.} Now try this:

\begin{longtable}[]{@{}@{}}
\toprule
\endhead
\bottomrule
\end{longtable}

The standard practice for naming constants is that you use uppercase
letters and you separate words in the variable name with the underscore.
Here are some examples:

\begin{longtable}[]{@{}l@{}}
\toprule
\endhead
\begin{minipage}[t]{0.97\columnwidth}\raggedright
const double MAX = 41.41;

const double MAX\_POWER = 123.41;

const int MAX\_LIVES = 3;

const int MONTHS\_PER\_YEAR = 12;

const int HOURS\_PER\_WEEK = 40;\strut
\end{minipage}\tabularnewline
\bottomrule
\end{longtable}

The only exception is when you're writing scientific applications where
there are lots of constants and their names have already been
standardized for a long time. Remember that a programmer always strives
to write code that's easy to read. For instance

\begin{longtable}[]{@{}l@{}}
\toprule
\endhead
const double e = 2.718; // GOOD\tabularnewline
\bottomrule
\end{longtable}

and not

\begin{longtable}[]{@{}l@{}}
\toprule
\endhead
const double E = 2.718; // BAD, BAD, BAD!!!\tabularnewline
\bottomrule
\end{longtable}

Remember that a constant is a variable. Therefore you can print it and
you can use all the relevant operators on it. The only operators you
cannot use are those that attempt to change the value of the constant
(I've already mentioned that you cannot use the assignment after a
constant has been initialized.)

\textbf{Exercise.} Does this work?

\begin{longtable}[]{@{}@{}}
\toprule
\endhead
\bottomrule
\end{longtable}

(Duh)

Why Constants?

Let's say you work in a company where the boss wants a program to
compute the annual salary of an employee. First run this program:

\begin{longtable}[]{@{}l@{}}
\toprule
\endhead
\begin{minipage}[t]{0.97\columnwidth}\raggedright
int monthly\_salary;

std::cout \textless\textless{} "Enter salary per month: ";

std::cin \textgreater\textgreater{} monthly\_salary;

std::cout \textless\textless{} "In 12 months the employee earns \$"

\textless\textless{} monthly\_salary * 12 \textless\textless{}
std::endl;\strut
\end{minipage}\tabularnewline
\bottomrule
\end{longtable}

Now suppose the boss decided to have 11 paid months. (Yes, he's a slave
driver.) Modify the program. Run your program and make sure it works.

But he changed his mind. He's going to pay 10 months of salary. Change
the program. (Grrrrr...) Make sure it works.

Now I want you to modify your program to get this:

\begin{longtable}[]{@{}l@{}}
\toprule
\endhead
\begin{minipage}[t]{0.97\columnwidth}\raggedright
\textbf{const int PAID\_MTHS\_PER\_YR = 10};

int monthly\_salary;

std::cout \textless\textless{} "Enter salary per month: ";

std::cin \textgreater\textgreater{} monthly\_salary;

std::cout \textless\textless{} "In " \textless\textless{}
\textbf{PAID\_MTHS\_PER\_YR}

\textless\textless{} " months the employee earn \$"

\textless\textless{} monthly\_salary * \textbf{PAID\_MTHS\_PER\_YR }

\textbf{ \textless\textless{} }'\textbackslash n';\strut
\end{minipage}\tabularnewline
\bottomrule
\end{longtable}

After a few days, your boss felt real bad. He decided to pay everyone 12
months. Change your program accordingly.

There are at least two reasons why we use constants:

Constants give \textbf{meaningful names to values so that the code is
more readable}. Plain-jane values have no meanings. They are just
values. Constants make C++ code easier to read. For instance this is
easier to comprehend:

\begin{longtable}[]{@{}l@{}}
\toprule
\endhead
std::cout \textless\textless{} pi * r * r \textless\textless{}
std::endl;\tabularnewline
\bottomrule
\end{longtable}

than this:

\begin{longtable}[]{@{}l@{}}
\toprule
\endhead
std::cout \textless\textless{} 3.1415926535897 * r * r
\textless\textless{} std::endl;\tabularnewline
\bottomrule
\end{longtable}

If the constant value needs to be changed, just change at one spot - in
the declaration of the constant. Therefore using \textbf{constants make
code easier to maintain in the event of changes.}

Of course you can create variables for the above two reasons too like
this:

\begin{longtable}[]{@{}l@{}}
\toprule
\endhead
\begin{minipage}[t]{0.97\columnwidth}\raggedright
\textbf{\sout{const} int PAID\_MTHS\_PER\_YR = 10};

int monthly\_salary;

std::cout \textless\textless{} "Enter salary per month: ";

std::cin \textgreater\textgreater{} monthly\_salary;

std::cout \textless\textless{} "In " \textless\textless{}
PAID\_MTHS\_PER\_YR

\textless\textless{} " months the employee earn \$"

\textless\textless{} monthly\_salary * PAID\_MTHS\_PER\_YR

\textless\textless{} std::endl;\strut
\end{minipage}\tabularnewline
\bottomrule
\end{longtable}

i.e., \emph{PAID\_MTHS\_PER\_YR} is not a constant.

Why do we want the value of the constant variable not to change? It's a
safety measure. If one day you (or your colleague) accidentally assign a
value to a constant, C++ will yell at you (or your colleague). Therefore
\textbf{constants prevent accidental changes to the value of a variable
that should not be modified}.

Writing values directly into the code instead of using constants is
called ``hard-coding constants''.

Of course you only create constants to make a program easier to read and
maintain. You don't create constants for every single value that can
appear in your program! For instance don't change the following program:

\begin{longtable}[]{@{}l@{}}
\toprule
\endhead
\begin{minipage}[t]{0.97\columnwidth}\raggedright
std::cout \textless\textless{} "triangle area:"

\textless\textless{} 0.5 * base * height \textless\textless{}
'\textbackslash n';\strut
\end{minipage}\tabularnewline
\bottomrule
\end{longtable}

to this:

\begin{longtable}[]{@{}l@{}}
\toprule
\endhead
\begin{minipage}[t]{0.97\columnwidth}\raggedright
const double HALF = 0.5; // a silly constant!!!

std::cout \textless\textless{} "triangle area:"

\textless\textless{} HALF * base * height \textless\textless{}
'\textbackslash n';\strut
\end{minipage}\tabularnewline
\bottomrule
\end{longtable}

This doesn't help! ``Half'' \emph{\textbf{is}} 0.5!!!

\textbf{Exercise.} Write a program that computes the area of a circle
when the user enters a value for the radius. There should be a constant.

\textbf{Exercise.} According to Elbert Ainstain the energy of a body of
mass m is given by

E = mc\textsuperscript{\emph{3}}

where c is the speed of light which is 123.45 meter per second. Write a
program that prompts the user for m and prints E up to 4 decimal places.
There should be a constant in your code.

\textbf{Exercise.} The CEO of VyHee has asked you to write program to
print some salary data. Employees on hourly wages are paid at a fixed
rate of \$123.89 per hour. Such employees are required to work 120 per
week. All other employees are paid \$4242 per month. There are 50 work
weeks at VyHee. You need to write a program that prompts the user for
relevant data and print the total amount that the CEO must pay for
salaries. Include the fact the CEO himself/herself makes \$1,000,000 per
year. Use as many constants as you should.

C-style constants

This is a DIY section.

There's another way to create constants in C++ programs. This is in fact
from the C language. (C++ is a descendant of C and therefore this method
of creating constants are carried forward to C++.) Run this:

\begin{longtable}[]{@{}l@{}}
\toprule
\endhead
\begin{minipage}[t]{0.97\columnwidth}\raggedright
\#include \textless iostream\textgreater{}

\#define NUM\_ARMS 3

\#define HELLO\_WORLD "HELLO... WORLD... !!!"

int main()

\{

std::cout \textless\textless{} NUM\_ARMS \textless\textless{}
'\textbackslash n'

\textless\textless{} HELLO\_WORLD \textless\textless{}
'\textbackslash n';

return 0;

\}\strut
\end{minipage}\tabularnewline
\bottomrule
\end{longtable}

Note the syntax. There's no = sign next to NUM\_ARMS:

\begin{longtable}[]{@{}l@{}}
\toprule
\endhead
\#define NUM\_ARMS = 3\tabularnewline
\bottomrule
\end{longtable}

You can do this if you like (but it's not so common):

\begin{longtable}[]{@{}l@{}}
\toprule
\endhead
\begin{minipage}[t]{0.97\columnwidth}\raggedright
\#include \textless iostream\textgreater{}

int main()

\{

\#define NUM\_ARMS 3

\#define HELLO\_WORLD "HELLO... WORLD... !!!"

std::cout \textless\textless{} NUM\_ARMS \textless\textless{}
'\textbackslash n'

\textless\textless{} HELLO\_WORLD \textless\textless{}
'\textbackslash n';

return 0;

\}\strut
\end{minipage}\tabularnewline
\bottomrule
\end{longtable}

The \emph{\#define} is not part of the C or C++ language. Briefly, your
MS VS .NET will actually replace NUM\_ARMS with 3 even before the
program is compiled. In other words before compiling (and running the
program), there's a \emph{\textbf{textual replacement}} of
\emph{NUM\_ARMS} by \emph{3}. This textual replacement process will
actually take the above, and give you this:

\begin{longtable}[]{@{}l@{}}
\toprule
\endhead
\begin{minipage}[t]{0.97\columnwidth}\raggedright
\#include \textless iostream\textgreater{}

int main()

\{

std::cout \textless\textless{} 3 \textless\textless{} '\textbackslash n'

\textless\textless{} "HELLO... WORLD... !!!" \textless\textless{}
'\textbackslash n';

return 0;

\}\strut
\end{minipage}\tabularnewline
\bottomrule
\end{longtable}

and \emph{\textbf{then}} \ldots{} this above program is sent to the
compiler.

As much as possible you should avoid using C-style constants.

You should be aware of this method of creating constants since

\begin{enumerate}
\def\labelenumi{\arabic{enumi}.}
\tightlist
\item
  many C++ programmers (unfortunately) use it occasionally
\item
  you might one day inherit C legacy code.
\end{enumerate}

The \emph{\#define} however can be used in another context. We will talk
about that later.

Summary

A constant is a variable whose value cannot change. The declaration of
constants look like this:

const {[}type{]} {[}var name{]} = {[}value{]};

Constants must be initialized. You can initialize a constant with a
value or an expression that involves constants. They cannot be assigned
values after their initialization.

The name of a constant is usually in uppercase letters with words
separated by the underscore. The only exception is when the constant it
represents already has an established name.

Exercises

Q1. Modify the following code by creating meaningful constants in place
of integer values and variables with meaningful names:

\begin{longtable}[]{@{}l@{}}
\toprule
\endhead
\begin{minipage}[t]{0.97\columnwidth}\raggedright
int i = 0;

std::cout \textless\textless{} "how many toyota priuses do u want?";

std::cin \textgreater\textgreater{} i;

int j = 0;

std::cout \textless\textless{} "how many ipods do u want?";

std::cin \textgreater\textgreater{} j;

double k = 0;

std::cout \textless\textless{} "how many more yrs in college?";

std::cin \textgreater\textgreater{} k;

double l = 0;

std::cout \textless\textless{} "how much do u have?";

std::cin \textgreater\textgreater{} l;

double m = 0;

std::cout \textless\textless{} "how much can u borrow from mom?";

std::cin \textgreater\textgreater{} m;

std::cout \textless\textless{} "you need this much:"

\textless\textless{} l + m -- 30000 * i -- 290 * j -- 15000 * k

\textless\textless{} '\textbackslash n';\strut
\end{minipage}\tabularnewline
\bottomrule
\end{longtable}

Q2. Declare the following constants for a tic-tac-toe program:

\begin{itemize}
\tightlist
\item
  A constant representing the number of rows in the tic-tac-toe board.
\item
  A constant representing the number of columns in the tic-tac-toe
  board.
\item
  A constant representing the number of squares in the tic-tac-toe
  board.
\end{itemize}

Print all the constants. The third constant should depend on the first
two. This means that if there are changes to board size, you only need
to change the first two constants.

Q3. List down all the benefits for using constants -- don't look at the
notes! Once you're done check against the notes.

\end{document}
