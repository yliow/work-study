% Options for packages loaded elsewhere
\PassOptionsToPackage{unicode}{hyperref}
\PassOptionsToPackage{hyphens}{url}
%
\documentclass[
]{article}
\usepackage{lmodern}
\usepackage{amssymb,amsmath}
\usepackage{ifxetex,ifluatex}
\ifnum 0\ifxetex 1\fi\ifluatex 1\fi=0 % if pdftex
  \usepackage[T1]{fontenc}
  \usepackage[utf8]{inputenc}
  \usepackage{textcomp} % provide euro and other symbols
\else % if luatex or xetex
  \usepackage{unicode-math}
  \defaultfontfeatures{Scale=MatchLowercase}
  \defaultfontfeatures[\rmfamily]{Ligatures=TeX,Scale=1}
\fi
% Use upquote if available, for straight quotes in verbatim environments
\IfFileExists{upquote.sty}{\usepackage{upquote}}{}
\IfFileExists{microtype.sty}{% use microtype if available
  \usepackage[]{microtype}
  \UseMicrotypeSet[protrusion]{basicmath} % disable protrusion for tt fonts
}{}
\makeatletter
\@ifundefined{KOMAClassName}{% if non-KOMA class
  \IfFileExists{parskip.sty}{%
    \usepackage{parskip}
  }{% else
    \setlength{\parindent}{0pt}
    \setlength{\parskip}{6pt plus 2pt minus 1pt}}
}{% if KOMA class
  \KOMAoptions{parskip=half}}
\makeatother
\usepackage{xcolor}
\IfFileExists{xurl.sty}{\usepackage{xurl}}{} % add URL line breaks if available
\IfFileExists{bookmark.sty}{\usepackage{bookmark}}{\usepackage{hyperref}}
\hypersetup{
  hidelinks,
  pdfcreator={LaTeX via pandoc}}
\urlstyle{same} % disable monospaced font for URLs
\usepackage{longtable,booktabs}
% Correct order of tables after \paragraph or \subparagraph
\usepackage{etoolbox}
\makeatletter
\patchcmd\longtable{\par}{\if@noskipsec\mbox{}\fi\par}{}{}
\makeatother
% Allow footnotes in longtable head/foot
\IfFileExists{footnotehyper.sty}{\usepackage{footnotehyper}}{\usepackage{footnote}}
\makesavenoteenv{longtable}
\setlength{\emergencystretch}{3em} % prevent overfull lines
\providecommand{\tightlist}{%
  \setlength{\itemsep}{0pt}\setlength{\parskip}{0pt}}
\setcounter{secnumdepth}{-\maxdimen} % remove section numbering

\author{}
\date{}

\begin{document}

04. Floating Point Numbers

Objectives

\begin{itemize}
\item
  Use floating point literals
\item
  Use floating point operators
\item
  Declare floating point variables
\item
  Declare and initialize floating point variables
\item
  Use assignment operator for floating variables
\item
  Use type casting
\item
  Understand automatic type conversion and type promotion
\end{itemize}

If you've finished studying the set of notes on integers (both integer
values and integer variables) this one should be easy.

Floats

No, not the A\&W kind.

A floating point number is just a number with decimal places (well ...
almost). I usually call them floats.

Run this program:

\begin{longtable}[]{@{}@{}}
\toprule
\endhead
\bottomrule
\end{longtable}

Phew! It's good to know that C++ can do decimal places ...

Recall that a \textbf{type} is just a collection of values. For instance
\emph{int} is a type; it includes values such as 0, 5, -3, and 41.

(WARNING: The word ``type'' has several meanings in Computer Science.
The above is one of them. You can usually discern the intended meaning
from the context.)

The name of the type of the above values (1.2345,...) is
\emph{\textbf{double}}.

C++ actually understands two different kinds for floats:
\emph{\textbf{float}} and \emph{\textbf{double}}. Of course
mathematically you can have infinitely many decimal places. For instance
your math instructor might write this on the board:

3.333333...

But hey you can't put infinitely many values (the three dots ...) into a
computer right?

In the first program above 1.2345 is actually a \emph{double}. If you
want to use a \emph{float} you write this:

\begin{longtable}[]{@{}@{}}
\toprule
\endhead
\bottomrule
\end{longtable}

But why are there two kinds of floating point numbers, the \emph{double}
and the \emph{float}? Good question!!!

You can be more precise with a \emph{double} than a \emph{float}.
Furthermore \emph{double}s represent a wider range of value than the
\emph{float}s. It's because of this that a \emph{double} value takes up
more memory than a \emph{float }and operations with \emph{double}s might
be slower than operations with \emph{float}s.

For this course we will be using mainly the \emph{double} type.

The notation for doubles using e is called the \textbf{scientific
notation}. Although you can write this in this notation

\emph{123.45e3}

it's usually written so it looks like this:

\emph{1.2345e5}

So ...

1.2345e5

The notation not using e is usually called \textbf{fixed-point
notation}.

In these notes, to distinguish between the integer five from the
floating point number five I will write

\begin{itemize}
\item
  5 for the integer five
\item
  5.0 for the floating point five
\end{itemize}

\textbf{Exercises.} Convert the following to scientific notation. You
should check your work with a calculator.

\begin{itemize}
\tightlist
\item
  12345.0
\item
  -12345.6789
\item
  0.00001234
\item
  -0.00004564
\end{itemize}

\textbf{Exercise.} Convert the following numbers from scientific
notation to fixed-point notation:

\begin{itemize}
\tightlist
\item
  1.23E-3
\item
  123.32e5
\item
  -123.32e-5
\end{itemize}

Operators

\textbf{Exercise.} Write a program that displays that sum, difference,
product and quotient of 1.1 and 2.2.

AHA! You can perform +, --, *, / on floats just like the \emph{int}
type.

\textbf{Exercise.} Run this program.

\begin{longtable}[]{@{}@{}}
\toprule
\endhead
\bottomrule
\end{longtable}

What's the point of this exercise?

Associativity and precedence rules

The associativity and precedence rules are exactly the same as those for
integer values (with the caveat that \emph{\%} is not defined for
floats).

Done! Yeah! :)

\textbf{Exercise.} Which operator is the second to be used for
evaluating the following expression:

\emph{1.234 + (3.4562 + 234.23 * 1.2345) / 1.23}

\textbf{Exercise.} Which operator is the third to be used for evaluating
the following expression:

(1.234 - 3.4562 + 234.23) - 1.2345 / 1.23

\textbf{Output: }\emph{\textbf{std::setw()}}

The next few sections are some tricks for controlling output.

\textbf{Exercise.} Try the following which has nothing to do with
floats:

\begin{longtable}[]{@{}l@{}}
\toprule
\endhead
\begin{minipage}[t]{0.97\columnwidth}\raggedright
\#include \textless iostream\textgreater{}

\#include \textless iomanip\textgreater{}

int main()

\{

std::cout \textless\textless{} "ham" \textless\textless{} std::endl;

std::cout \textless\textless{} \textbf{std::setw(5)}
\textless\textless{} "and" \textless\textless{} std::endl;

std::cout \textless\textless{} "eggs" \textless\textless{} std::endl;

return 0;

\}\strut
\end{minipage}\tabularnewline
\bottomrule
\end{longtable}

Change the 5 to 6 and run the program. Change the 6 to 10 and run it.
Change the 6 to 2 and run it. So what does \emph{std::setw()} do? Remove
the line \emph{\#include \textless iomanip\textgreater{}} and run it. So
what must you remember to do whenever you use \emph{std::setw()}?

An important thing to note is that \emph{std::setw()} only affects one
output value. So in the above example, the width for the
\emph{\textbf{next}} output (i.e. \emph{"eggs"}) is not affected.

\textbf{Exercise.} Change the string \emph{"and"} in the above program
to your favorite integer (example: maybe 42?) and run your program.
Change the integer value to a double (say 3.14). Does \emph{std::setw()}
work with integers and doubles?

Try this:

\begin{longtable}[]{@{}l@{}}
\toprule
\endhead
\begin{minipage}[t]{0.97\columnwidth}\raggedright
std::cout \textless\textless{} std::setw(5) \textless\textless{} 2

\textless\textless{} std::setw(5) \textless\textless{} 3
\textless\textless{} std::endl;

\textless\textless{} std::setw(10) \textless\textless{} 1
\textless\textless{} std::endl;\strut
\end{minipage}\tabularnewline
\bottomrule
\end{longtable}

(And what must you do whenever you use \emph{std::setw()}?) This shows
you that you can have multiple width controls in a single
\emph{std::cout} statement.

\textbf{Exercise.} Write a program that produces the following output.
Do not use the space or the tab character.

\begin{longtable}[]{@{}l@{}}
\toprule
\endhead
\begin{minipage}[t]{0.97\columnwidth}\raggedright
n n\^{}3

- -\/-\/-

0 0

1 1

2 8

3 27\strut
\end{minipage}\tabularnewline
\bottomrule
\end{longtable}

\textbf{Exercise.} Can the integer value used in \emph{std::setw()} be
an integer \emph{\textbf{variable}} instead? Write a program that works
as follows.

\begin{longtable}[]{@{}l@{}}
\toprule
\endhead
\begin{minipage}[t]{0.97\columnwidth}\raggedright
\#include \textless iostream\textgreater{}

\#include \textless iomanip\textgreater{}

int main()

\{

int w = 0;

std::cout \textless\textless{} "width: ";

std::cin \textgreater\textgreater{} w;

std::cout \textless\textless{} std::setw(\textbf{w})
\textless\textless{} 1 \textless\textless{} std::endl;

return 0;

\}\strut
\end{minipage}\tabularnewline
\bottomrule
\end{longtable}

Run this program entering different values for w.

\textbf{Exercise.} What happens when the width set is too small for the
value to be printed? Does C++ chop off (truncate) the value so that it
fits? Do this experiment: set the width to 3 and print \emph{"hello"}.

Output: \emph{std::setfill()}

The next set of exercises have to do with \emph{std::setfill()}.
Remember that the h character is written \emph{'h'}.

\textbf{Exercise.} Run this program:

\begin{longtable}[]{@{}l@{}}
\toprule
\endhead
\begin{minipage}[t]{0.97\columnwidth}\raggedright
\#include \textless iostream\textgreater{}

\#include \textless iomanip\textgreater{}

int main()

\{

std::cout \textless\textless{} std::setfill('*')

\textless\textless{} std::setw(3) \textless\textless{} 1
\textless\textless{} '\textbackslash n'

\textless\textless{} std::setw(5) \textless\textless{} 1
\textless\textless{} std::endl;

std::cout \textless\textless{} std::setfill('\$')

\textless\textless{} std::setw(3) \textless\textless{} 1
\textless\textless{} '\textbackslash n'

\textless\textless{} std::setw(5) \textless\textless{} 1
\textless\textless{} std::endl;

std::cout \textless\textless{} std::setfill(' ')

\textless\textless{} std::setw(3) \textless\textless{} 1
\textless\textless{} '\textbackslash n'

\textless\textless{} std::setw(5) \textless\textless{} 1
\textless\textless{} std::endl;

return 0;

\}\strut
\end{minipage}\tabularnewline
\bottomrule
\end{longtable}

What does \emph{std::setfill()} do? (Duh)

\textbf{Exercise.} Write a program that produces the following output.
You are only allowed to press the \# key once in your program!!!

\begin{longtable}[]{@{}l@{}}
\toprule
\endhead
\begin{minipage}[t]{0.97\columnwidth}\raggedright
\#\#\#\#\#\#\#\#\#\#

\#\#\#\#\#\#\#\#\#\#

\#\#\#\#\#\#\#\#\#\#

\#\#\#\#\#\#\#\#\#\#\strut
\end{minipage}\tabularnewline
\bottomrule
\end{longtable}

Output: justification

Try this:

\begin{longtable}[]{@{}l@{}}
\toprule
\endhead
\begin{minipage}[t]{0.97\columnwidth}\raggedright
\#include \textless iostream\textgreater{}

\#include \textless iomanip\textgreater{}

int main()

\{

std::cout \textless\textless{} std::left

\textless\textless{} std::setw(3) \textless\textless{} 1
\textless\textless{} '\textbackslash n'

\textless\textless{} std::setw(5) \textless\textless{} 1
\textless\textless{} std::endl;

return 0;

\}\strut
\end{minipage}\tabularnewline
\bottomrule
\end{longtable}

Now try this:

\begin{longtable}[]{@{}l@{}}
\toprule
\endhead
\begin{minipage}[t]{0.97\columnwidth}\raggedright
\#include \textless iostream\textgreater{}

\#include \textless iomanip\textgreater{}

int main()

\{

std::cout \textless\textless{} std::left

\textless\textless{} std::setw(3) \textless\textless{} 1
\textless\textless{} '\textbackslash n'

\textless\textless{} std::setw(5) \textless\textless{} 1
\textless\textless{} std::endl;

std::cout \textless\textless{} std::right

\textless\textless{} std::setw(3) \textless\textless{} 1
\textless\textless{} '\textbackslash n'

\textless\textless{} std::setw(5) \textless\textless{} 1
\textless\textless{} std::endl;

return 0;

\}\strut
\end{minipage}\tabularnewline
\bottomrule
\end{longtable}

Enough said.

Notice that once you do one \emph{std::right} for right justification
like this:

\begin{longtable}[]{@{}l@{}}
\toprule
\endhead
\begin{minipage}[t]{0.97\columnwidth}\raggedright
std::cout \textless\textless{} std::right;

\textless\textless{} std::setw(3) \textless\textless{} 1
\textless\textless{} '\textbackslash n'

\textless\textless{} std::setw(5) \textless\textless{} 1
\textless\textless{} std::endl;\strut
\end{minipage}\tabularnewline
\bottomrule
\end{longtable}

it will be in effect until you change to left justification.

\textbf{Exercise. }Run this program

\begin{longtable}[]{@{}l@{}}
\toprule
\endhead
\begin{minipage}[t]{0.97\columnwidth}\raggedright
\#include \textless iostream\textgreater{}

\#include \textless iomanip\textgreater{}

int main()

\{

std::cout \textless\textless{} std::left;

std::cout \textless\textless{} '\textless' \textless\textless{} 1
\textless\textless{} '\textgreater' \textless\textless{} std::endl;

std::cout \textless\textless{} std::right;

std::cout \textless\textless{} '\textless' \textless\textless{} 1
\textless\textless{} '\textgreater' \textless\textless{} std::endl;

return 0;

\}\strut
\end{minipage}\tabularnewline
\bottomrule
\end{longtable}

and you get

\begin{longtable}[]{@{}l@{}}
\toprule
\endhead
\begin{minipage}[t]{0.97\columnwidth}\raggedright
\textless1\textgreater{}

\textless1\textgreater{}\strut
\end{minipage}\tabularnewline
\bottomrule
\end{longtable}

Insert two \emph{\textless\textless{} std::setw(10)} into the code so
the you get the following output:

\begin{longtable}[]{@{}l@{}}
\toprule
\endhead
\begin{minipage}[t]{0.97\columnwidth}\raggedright
\textless1 \textgreater{}

\textless{} 1\textgreater{}\strut
\end{minipage}\tabularnewline
\bottomrule
\end{longtable}

Here's another example. Run this program and study the code and the
output yourself:

\begin{longtable}[]{@{}l@{}}
\toprule
\endhead
\begin{minipage}[t]{0.97\columnwidth}\raggedright
\#include \textless iostream\textgreater{}

\#include \textless iomanip\textgreater{}

int main()

\{

std::cout \textless\textless{} std::left \textless\textless{}
std::setw(12) \textless\textless{} "January"

\textless\textless{} '\$'

\textless\textless{} std::right \textless\textless{} std::setw(10)
\textless\textless{} 123.45

\textless\textless{} std::endl;

std::cout \textless\textless{} std::left \textless\textless{}
std::setw(12) \textless\textless{} "February"

\textless\textless{} '\$'

\textless\textless{} std::right \textless\textless{} std::setw(10)
\textless\textless{} 23.45

\textless\textless{} std::endl;

std::cout \textless\textless{} std::left \textless\textless{}
std::setw(12) \textless\textless{} "March"

\textless\textless{} '\$'

\textless\textless{} std::right \textless\textless{} std::setw(10)
\textless\textless{} 3.45

\textless\textless{} std::endl;

return 0;

\}\strut
\end{minipage}\tabularnewline
\bottomrule
\end{longtable}

\textbf{Exercise. }Modify the program\textbf{ }

\begin{longtable}[]{@{}l@{}}
\toprule
\endhead
\begin{minipage}[t]{0.97\columnwidth}\raggedright
\#include \textless iostream\textgreater{}

\#include \textless iomanip\textgreater{}

int main()

\{

std::cout \textless\textless{} std::left \textless\textless{}
std::setw(12) \textless\textless{} "January"

\textless\textless{} '\$'

\textless\textless{} std::right \textless\textless{} std::setw(10)
\textless\textless{} 123.45

\textless\textless{} std::endl;

std::cout \textless\textless{} std::left \textless\textless{}
std::setw(12) \textless\textless{} "February"

\textless\textless{} '\$'

\textless\textless{} std::right \textless\textless{} std::setw(10)
\textless\textless{} 23.45

\textless\textless{} std::endl;

std::cout \textless\textless{} std::left \textless\textless{}
std::setw(12) \textless\textless{} "March"

\textless\textless{} '\$'

\textless\textless{} std::right \textless\textless{} std::setw(10)
\textless\textless{} 3.45

\textless\textless{} std::endl;

return 0;

\}\strut
\end{minipage}\tabularnewline
\bottomrule
\end{longtable}

so that you get this output:

\begin{longtable}[]{@{}l@{}}
\toprule
\endhead
\begin{minipage}[t]{0.97\columnwidth}\raggedright
January \$....123.45

February \$.....23.45

March \$......3.45\strut
\end{minipage}\tabularnewline
\bottomrule
\end{longtable}

Output: std::fixed and std::scientific

Because there are two formats for \emph{double}s (example: 123.456 is
the same as 1.23456e2) sometimes you want to force C++ to choose a
particular format.

Try this:

\begin{longtable}[]{@{}l@{}}
\toprule
\endhead
\begin{minipage}[t]{0.97\columnwidth}\raggedright
\#include \textless iostream\textgreater{}

\#include \textless iomanip\textgreater{}

int main()

\{

std::cout \textless\textless{} 0.00001 \textless\textless{} std::endl;

std::cout \textless\textless{} 0.00002 \textless\textless{} std::endl;

std::cout \textless\textless{} std::fixed;

std::cout \textless\textless{} 0.00001 \textless\textless{} std::endl;

std::cout \textless\textless{} 0.00002 \textless\textless{} std::endl;

std::cout \textless\textless{} std::scientific;

std::cout \textless\textless{} 0.00001 \textless\textless{} std::endl;

std::cout \textless\textless{} 0.00002 \textless\textless{} std::endl;

return 0;

\}\strut
\end{minipage}\tabularnewline
\bottomrule
\end{longtable}

\textbf{Output: }\emph{\textbf{std::setprecision()}}

First run this:

\begin{longtable}[]{@{}l@{}}
\toprule
\endhead
\begin{minipage}[t]{0.97\columnwidth}\raggedright
\#include \textless iostream\textgreater{}

\#include \textless iomanip\textgreater{}

int main()

\{

std::cout \textless\textless{} std::setprecision(3);

std::cout \textless\textless{} std::setw(10) \textless\textless{} 6789
\textless\textless{} std::endl;

std::cout \textless\textless{} std::setw(10) \textless\textless{} 67.8
\textless\textless{} std::endl;

std::cout \textless\textless{} std::setw(10) \textless\textless{} 67.89
\textless\textless{} std::endl;

std::cout \textless\textless{} std::setw(10) \textless\textless{} 678.9
\textless\textless{} std::endl;

return 0;

\}\strut
\end{minipage}\tabularnewline
\bottomrule
\end{longtable}

And also this:

\begin{longtable}[]{@{}l@{}}
\toprule
\endhead
\begin{minipage}[t]{0.97\columnwidth}\raggedright
\#include \textless iostream\textgreater{}

\#include \textless iomanip\textgreater{}

int main()

\{

std::cout \textless\textless{} std::setprecision(3) \textless\textless{}
std::fixed;

std::cout \textless\textless{} std::setw(10) \textless\textless{} 6789
\textless\textless{} std::endl;

std::cout \textless\textless{} std::setw(10) \textless\textless{} 67.8
\textless\textless{} std::endl;

std::cout \textless\textless{} std::setw(10) \textless\textless{} 67.89
\textless\textless{} std::endl;

std::cout \textless\textless{} std::setw(10) \textless\textless{} 678.9
\textless\textless{} std::endl;

return 0;

\}\strut
\end{minipage}\tabularnewline
\bottomrule
\end{longtable}

Change the 3 to 6 in the above programs and run the programs again. So
... what does \emph{std::setprecision()} do?

Note that whereas you need to call \emph{std::setw()} for each value
displayed, note that calling \emph{std::setprecision()} will affect all
the displayed value until you call it again to set the precision to
another number.

\textbf{Exercise.} Modify the above example by print different doubles
of different number of decimal places.

\textbf{Exercise.} Do the same experiment as above but with printing set
to scientific mode.

\begin{longtable}[]{@{}l@{}}
\toprule
\endhead
\begin{minipage}[t]{0.97\columnwidth}\raggedright
\#include \textless iostream\textgreater{}

\#include \textless iomanip\textgreater{}

int main()

\{

std::cout \textless\textless{} std::setprecision(3)

\textless\textless{} std::scientific;

std::cout \textless\textless{} std::setw(10) \textless\textless{} 6789
\textless\textless{} std::endl;

std::cout \textless\textless{} std::setw(10) \textless\textless{} 67.8
\textless\textless{} std::endl;

std::cout \textless\textless{} std::setw(10) \textless\textless{} 67.89
\textless\textless{} std::endl;

std::cout \textless\textless{} std::setw(10) \textless\textless{} 678.9
\textless\textless{} std::endl;

return 0;

\}\strut
\end{minipage}\tabularnewline
\bottomrule
\end{longtable}

Get it? You might want to modify the precision and the \emph{double}s
being printed.

\textbf{Exercise.} Can you use a variable when specifying the precision?

Double variables

Warning: Incoming warm-up ...

\textbf{Exercise.} Declare a \emph{double} variable \emph{x} with an
initial value of 0.0. Print the value of \emph{x}. Run the program and
make sure it works. Now fill in the blank:

\_\_\_\_\_\_\_\_\_ x = \_\_\_\_\_\_\_\_\_\_;

\textbf{Exercise.} Using the previous program, declare a \emph{double}
variable called \emph{pi} with an initial value of 3.14159. Print the
value of \emph{pi}.

It's always a good practice to initialize your variables. If you do not
have a value in mind, the default choice is \emph{0.0}.

\textbf{Exercise.} Declare two \emph{double} variables \emph{x} and
\emph{y}, initializing their values to 1.2 and 3.4 respectively. Print
their sum, difference, product and quotient. (Yes, I haven't shown you
how to get C++ to compute the sum, difference, product and quotient of
\emph{\textbf{double}} variables. But you do know how to do the sum,
difference, product and quotient of \emph{\textbf{int}}s. So make a
guess.)

Double input (console window)

\textbf{Exercise.} Be brave. You can do it. Write a program that prompts
the user for two floating point numbers (use \emph{double}, not
\emph{float}) and then print their sum.

That's it for input to floating point variables!!!

\textbf{Exercise.} Complete the program above so that it prompts the
user for two \emph{double}s, prompts the user for a display width,
prompts the user for the precision, and finally prints the sum with the
given width and precision in fixed point format.

\textbf{Exercise.} Write a mileage calculator. Here's an execution of
the program.

\begin{longtable}[]{@{}l@{}}
\toprule
\endhead
\begin{minipage}[t]{0.97\columnwidth}\raggedright
***** Mileage calculator *****

Enter distance travelled (in miles): \textbf{1234}

Enter gas used (in gallons): \textbf{12}

Mileage (miles/gallon): 102.833\strut
\end{minipage}\tabularnewline
\bottomrule
\end{longtable}

(Make sure you choose good variable names.)

\textbf{Exercise.} Write an hourly wage calculator. Here's an execution
of the program.

\begin{longtable}[]{@{}l@{}}
\toprule
\endhead
\begin{minipage}[t]{0.97\columnwidth}\raggedright
Pay calculator (version 0.1)

Enter hours: \textbf{12.3}

Enter pay-per-hour (in dollars): \textbf{99.88}

Pay: \$1228.52\strut
\end{minipage}\tabularnewline
\bottomrule
\end{longtable}

\textbf{Exercise.} Write a program that compute the area of a right
triangle. Make sure the output is in fixed point mode and with 4 decimal
places. Here's an execution of the program:

\begin{longtable}[]{@{}l@{}}
\toprule
\endhead
\begin{minipage}[t]{0.97\columnwidth}\raggedright
Right triangle area calculator !!!

Base: \textbf{12.345}

Height: \textbf{23.456}

The area is 144.7822\strut
\end{minipage}\tabularnewline
\bottomrule
\end{longtable}

Here's another execution:

\begin{longtable}[]{@{}l@{}}
\toprule
\endhead
\begin{minipage}[t]{0.97\columnwidth}\raggedright
Right triangle area calculator !!!

Base: \textbf{1}

Height: \textbf{2}

The area is 1.000\strut
\end{minipage}\tabularnewline
\bottomrule
\end{longtable}

Types

So far you have seen \textbf{five} types of values:

The first is the \textbf{C-string}. For instance \emph{"hello world"} is
a string. The empty string (or null string) is just \emph{"".}

The second type is the \textbf{character}. For instance \emph{'h'} is a
character, so is

\emph{' '} (the space)\emph{. }There are special character such as
\emph{'\textbackslash n'}, \emph{'\textbackslash t'},
\emph{'\textbackslash"'}, etc. A C-string is in fact made up of
characters.

The third is the \textbf{integer}. For instance 41 is an integer. The
C++ name of this type is int.

The fourth and fifth type are both \textbf{floating point types}. Their
C++ names are double and the float.

I haven't given you the C++ names of the \textbf{character} and
\textbf{C-string} type yet. We will talk about them later. But just to
give you something to play with, try this program:

\begin{longtable}[]{@{}l@{}}
\toprule
\endhead
\begin{minipage}[t]{0.97\columnwidth}\raggedright
char x = '\$';

char y = '\textbackslash n';

char s{[}100{]} = "Frodo lives!";

char t{[}100{]} = "C++ rules!";

std::cout \textless\textless{} x \textless\textless{} std::endl

\textless\textless{} y \textless\textless{} std::endl

\textless\textless{} s \textless\textless{} std::endl

\textless\textless{} t \textless\textless{} std::endl;\strut
\end{minipage}\tabularnewline
\bottomrule
\end{longtable}

\textbf{Exercise.} You are entirely on your own \ldots{} be brave
\ldots{} write a program that does this when you run it with this input:

\begin{longtable}[]{@{}l@{}}
\toprule
\endhead
\begin{minipage}[t]{0.97\columnwidth}\raggedright
what's you fav character? \textbf{a}

a? ... not bad\strut
\end{minipage}\tabularnewline
\bottomrule
\end{longtable}

and this when you run it with this input:

\begin{longtable}[]{@{}l@{}}
\toprule
\endhead
\begin{minipage}[t]{0.97\columnwidth}\raggedright
what's you fav character? \textbf{T}

T? ... not bad\strut
\end{minipage}\tabularnewline
\bottomrule
\end{longtable}

\textbf{Exercise.} And now do this \ldots{} write a program that does
this when you run it with this input:

\begin{longtable}[]{@{}l@{}}
\toprule
\endhead
\begin{minipage}[t]{0.97\columnwidth}\raggedright
i'm c++ ... what's your name? \textbf{arthur}

hi arthur\strut
\end{minipage}\tabularnewline
\bottomrule
\end{longtable}

and this when you run it with this input:

\begin{longtable}[]{@{}l@{}}
\toprule
\endhead
\begin{minipage}[t]{0.97\columnwidth}\raggedright
i'm c++ ... what's your name? \textbf{zaphod}

hi zaphod\strut
\end{minipage}\tabularnewline
\bottomrule
\end{longtable}

What happens when you entered \emph{zaphod beeblebrox} instead?

Important warning

It's extremely important to remember that floating point numbers are
approximations.

\textbf{Exercise.} Run this program.

\begin{longtable}[]{@{}@{}}
\toprule
\endhead
\bottomrule
\end{longtable}

No surprises. Now increase the precision to 15 and then to 20 and run
the program ...\\
~\\
Got it?

The mystery of what's happening here will be revealed in CISS360
(Assembly language and computer systems).

Typecasting

\begin{longtable}[]{@{}l@{}}
\toprule
\endhead
std::cout \textless\textless{} 5 \textless\textless{}
std::endl;\tabularnewline
\bottomrule
\end{longtable}

And try this

\begin{longtable}[]{@{}l@{}}
\toprule
\endhead
\begin{minipage}[t]{0.97\columnwidth}\raggedright
std::cout \textless\textless{} std::setprecision(10)

\textless\textless{} \textbf{double(5)} \textless\textless{}
std::endl;\strut
\end{minipage}\tabularnewline
\bottomrule
\end{longtable}

\emph{double(5)} returns the \emph{double} value of the integer five.

What do I mean by ``return''? It's just like evaluating an expression.

Here's another way to do it:

\begin{longtable}[]{@{}l@{}}
\toprule
\endhead
\begin{minipage}[t]{0.97\columnwidth}\raggedright
std::cout \textless\textless{} std::setprecision(10)

\textless\textless{} \textbf{(double) 5} \textless\textless{}
std::endl;\strut
\end{minipage}\tabularnewline
\bottomrule
\end{longtable}

\textbf{Exercise.} What does this do?

\begin{longtable}[]{@{}l@{}}
\toprule
\endhead
\begin{minipage}[t]{0.97\columnwidth}\raggedright
std::cout \textless\textless{} \textbf{int(12.89)} \textless\textless{}
std::endl;

std::cout \textless\textless{} \textbf{(int) 12.89} \textless\textless{}
std::endl;\strut
\end{minipage}\tabularnewline
\bottomrule
\end{longtable}

\textbf{Exercise.} Can you type cast the value of a variable? Try this:

\begin{longtable}[]{@{}l@{}}
\toprule
\endhead
\begin{minipage}[t]{0.97\columnwidth}\raggedright
int i = 42;

double j = 3.14;

std::cout \textless\textless{} (double) i \textless\textless{}
std::endl;

std::cout \textless\textless{} (int) j \textless\textless{}
std::endl;\strut
\end{minipage}\tabularnewline
\bottomrule
\end{longtable}

Do you get an error?

Sometimes C++ will automatically do typecasting for you. Run this
program:

\begin{longtable}[]{@{}l@{}}
\toprule
\endhead
\begin{minipage}[t]{0.97\columnwidth}\raggedright
int i = 7.89;

float x = 5;

double y = 42;

std::cout \textless\textless{} i \textless\textless{} std::endl;

std::cout \textless\textless{} std::fixed \textless\textless{} x
\textless\textless{} '\textbackslash n'

\textless\textless{} y \textless\textless{} std::endl;\strut
\end{minipage}\tabularnewline
\bottomrule
\end{longtable}

Then there's \textbf{type promotion}. I've already told you that there's
the integer addition operator and the double addition operator. Try
this:

\begin{longtable}[]{@{}l@{}}
\toprule
\endhead
std::cout \textless\textless{} 2 + 3.1 \textless\textless{}
std::endl;\tabularnewline
\bottomrule
\end{longtable}

This is not the ``integer-double'' addition operator -- there is no such
operator!

What happens is that 2 is actually converted to the \emph{double} of 2
before the addition. Of course, the addition is the \emph{double}
addition operator.

In general if you apply an operator to an integer and a double (or
float), the integer will first be converted to a double (or float)
before the operator is evaluated. That's all there is. Remember that the
integer is converted to a double and not the other way round. Why?
Because C++ will try to be as accurate as possible.

\textbf{Exercise.} Is the value of this expression an \emph{int} or a
\emph{double}?

\begin{longtable}[]{@{}l@{}}
\toprule
\endhead
1 + 2 -- 3 / 4 * 5.1 + 2\tabularnewline
\bottomrule
\end{longtable}

\textbf{Exercise.} What integer value is first type promoted to a
\emph{double}?

\begin{longtable}[]{@{}l@{}}
\toprule
\endhead
4 + 2 -- 13 / 4 * 5.1 + 2\tabularnewline
\bottomrule
\end{longtable}

If you have an integer expression and you really want the output to be a
\emph{double}, you can typecast the variables. Try this:

\begin{longtable}[]{@{}l@{}}
\toprule
\endhead
\begin{minipage}[t]{0.97\columnwidth}\raggedright
int i = 1;

int j = 2;

std::cout \textless\textless{} i / j \textless\textless{}
std::endl;\strut
\end{minipage}\tabularnewline
\bottomrule
\end{longtable}

You get zero right? Because the / is the integer division operator. Now
change it to this:

\begin{longtable}[]{@{}l@{}}
\toprule
\endhead
\begin{minipage}[t]{0.97\columnwidth}\raggedright
int i = 1;

int j = 2;

std::cout \textless\textless{} double(i) / double(j)
\textless\textless{} std::endl;\strut
\end{minipage}\tabularnewline
\bottomrule
\end{longtable}

Run it again.

\textbf{Exercise.} Run this and explain why the output is the same as
the previous program.

\begin{longtable}[]{@{}l@{}}
\toprule
\endhead
\begin{minipage}[t]{0.97\columnwidth}\raggedright
int i = 1;

int j = 2;

std::cout \textless\textless{} double(i) / j \textless\textless{}
std::endl;\strut
\end{minipage}\tabularnewline
\bottomrule
\end{longtable}

\textbf{Exercise.} Run this \emph{version}:

\begin{longtable}[]{@{}l@{}}
\toprule
\endhead
\begin{minipage}[t]{0.97\columnwidth}\raggedright
int i = 1;

int j = 2;

std::cout \textless\textless{} double(i / j) \textless\textless{}
std::endl;\strut
\end{minipage}\tabularnewline
\bottomrule
\end{longtable}

Explain! Pay attention: This is a \emph{\textbf{very common mistake!!!}}

\textbf{Exercise.} Run this:

\begin{longtable}[]{@{}l@{}}
\toprule
\endhead
\begin{minipage}[t]{0.97\columnwidth}\raggedright
double x;

std::cin \textgreater\textgreater{} x;

int i = x + 0.5;

std::cout \textless\textless{} x \textless\textless{} ','
\textless\textless{} i \textless\textless{} std::endl;\strut
\end{minipage}\tabularnewline
\bottomrule
\end{longtable}

With several inputs for x (example: 3, 3.1, 3.5, 3.8, etc.) What does
this program do?

A function quickie

So what is a function? A function in C++ is like a function in math but
it's lot more. There is a complete set of notes for functions. Right now
I just want to give you enough understanding to move on. Now in math, a
function like

f(x) = 2x

means that

f(5) = 2(5) = 10

In other words a function in math is something that has an input and an
output. For f(x) above, when you put in 5, f(5) gives you 10.

Likewise for the function

g(x,y) = xy

when you put 4 for x and 7 for y you get

g(4,7) = (4)(7) = 28

Again the 4 and 7 are the inputs and 28 is the output.

The above examples are functions that performs numeric computations:
they take input(s), compute, and return a value. In programming,
functions can do more than just computations. Later you will learn that
there are functions that open a file for read/write, open a network
socket for network communication, open a GUI (graphical user interface)
window, etc. Right now I just want to get back to functions that perform
numeric computations.

Try this:

\begin{longtable}[]{@{}l@{}}
\toprule
\endhead
\begin{minipage}[t]{0.97\columnwidth}\raggedright
\#include \textless iostream\textgreater{}

\#include \textless cmath\textgreater{}

int main()

\{

std::cout \textless\textless{} pow(2.0, -1.0) \textless\textless{}
'\textbackslash n'

\textless\textless{} pow(2.0, 1.0) \textless\textless{}
'\textbackslash n'

\textless\textless{} pow(2.0, 2.0) \textless\textless{}
'\textbackslash n'

\textless\textless{} pow(2.0, 3.0) \textless\textless{}
'\textbackslash n';

return 0;

\}\strut
\end{minipage}\tabularnewline
\bottomrule
\end{longtable}

So \ldots{} what does the function \emph{pow(a, b)} give you?

Note that to use the \emph{pow()} function you must have this in the
program:

\begin{longtable}[]{@{}l@{}}
\toprule
\endhead
\begin{minipage}[t]{0.97\columnwidth}\raggedright
...

\#include \textless cmath\textgreater{}

...\strut
\end{minipage}\tabularnewline
\bottomrule
\end{longtable}

Of course you can use expressions:

\begin{longtable}[]{@{}l@{}}
\toprule
\endhead
\begin{minipage}[t]{0.97\columnwidth}\raggedright
\#include \textless iostream\textgreater{}

\#include \textless cmath\textgreater{}

int main()

\{

double x = 2.0, y = -1.0;

std::cout \textless\textless{} pow(x, y) \textless\textless{}
'\textbackslash n'

\textless\textless{} pow(x, y + 1.0) \textless\textless{}
'\textbackslash n'

\textless\textless{} pow(x * x, y + 5 * x) \textless\textless{}
'\textbackslash n';

return 0;

\}\strut
\end{minipage}\tabularnewline
\bottomrule
\end{longtable}

WARNING: MS VS .NET will give you an error if you do this:

\emph{pow(1, 2)}

For \emph{pow()} to work correctly, at least one of the two values you
pass in to the \emph{pow()} function must be a \emph{double}; the return
value is a \emph{double}.

\textbf{Exercise.} How would you correct this program without changing
the type of \emph{x} and \emph{y}?

\begin{longtable}[]{@{}l@{}}
\toprule
\endhead
\begin{minipage}[t]{0.97\columnwidth}\raggedright
\#include \textless iostream\textgreater{}

\#include \textless cmath\textgreater{}

int main()

\{

int x, y;

std::cout \textless\textless{} pow(x, y) \textless\textless{}
'\textbackslash n';

return 0;

\}\strut
\end{minipage}\tabularnewline
\bottomrule
\end{longtable}

\textbf{Exercise.} The Pythagoras' theorem states if x, y, z are the
lengths of a right angle triangle where z is the hypothenuse, then

x\textsuperscript{2} + y\textsuperscript{2 } = z\textsuperscript{2 }

Write a program that prompts the user for x and y, and print the value
of z.

The Pythagoras' theorem can be used to compute the distance between two
points. This can then be used in, for instance, computing if two things
collide in a computer game.

Now, you know that \emph{x * x} is the same as \emph{pow(x, 2)}. However
you should not overuse \emph{pow()} since the evaluation of \emph{x * x}
is probably much faster than \emph{pow()} especially if you're trying to
compute the square of integers.

In general, a more general function tends to be slower than a more
specific/specialized function. Remember that.

\textbf{Exercise.} Write a program that prompts the user for a
\emph{double} and prints the square root of the \emph{double}.

\textbf{Exercise.} Write a program that get an integer n from the user
prints 10 to the n-th power (as an integer). For instance if your input
is 0, the program prints 1. If your input is 3, the program prints 1000.

\textbf{Exercise.} If n is the integer 860713524, then the 0th digit of
n is 4 (the rightmost digit), the 1st digit of n is 2, the 2nd digit of
n is 5, the 3rd digit of n is 3, the 4th digit of n is 1, the 5th digit
of n is 7, the 6\textsuperscript{th} digit of n is 0, the 7th digit of n
is 6, etc. Write a program that get n and k from the user and prints the
k-th digit of n.

\textbf{Exercise.} Here the famous simple interest formula: If you have
P in a savings account at a bank and the bank gives you an interest rate
of r (for instance r might be 0.015, i.e. 1.5\%), then after t years you
will have

P(1 + r)\textsuperscript{t}

(P is called the ``principal''.) Write a program that computes this.
Here's an execution of the program:

\begin{longtable}[]{@{}l@{}}
\toprule
\endhead
\begin{minipage}[t]{0.97\columnwidth}\raggedright
Savings account calculator!!!

Enter P (principal): \$\textbf{1000}

Enter r (interest rate in percent): \textbf{1.5}

Enter t (number of years): \textbf{30}

Final balance: \$1563.08\strut
\end{minipage}\tabularnewline
\bottomrule
\end{longtable}

Here's another execution:

\begin{longtable}[]{@{}l@{}}
\toprule
\endhead
\begin{minipage}[t]{0.97\columnwidth}\raggedright
Savings account calculator!!!

Enter P (principal): \$\textbf{2000}

Enter r (interest rate in percent): \textbf{1.75}

Enter t (number of years): \textbf{40}

Final balance: \$4003.19\strut
\end{minipage}\tabularnewline
\bottomrule
\end{longtable}

A bunch of functions available in C++

There are many functions available in C++. Here are some (and don't
forget to use \emph{\#include \textless cmath\textgreater{}} when you
want to use any of these functions.)

\emph{pow(x, y)}Returns the power of x to y i.e. x\textsuperscript{y}

\textsuperscript{}x and y are \emph{double}s and the return value is a
\emph{double}

\emph{sqrt(x)}Returns the square root of x

\textsuperscript{}x is a \emph{double}, the return value is a
\emph{double}

\emph{exp(x)}Returns the power of the natural number e to x,
e\textsuperscript{x}. Here e = 2.718...

\textsuperscript{}x is a \emph{double}, the return value is a
\emph{double}

\emph{log(x)}Returns the natural logarithm of x, log\textsubscript{e}(x)
which is also

written ln(x) in math

\textsuperscript{}x is a \emph{double}, the return value is a
\emph{double}

\emph{log10(x)}Returns the logarithm of x to base 10,
log\textsubscript{10}(x)

\textsuperscript{}x is a \emph{double}, the return value is a
\emph{double}

\emph{sin(x)}Returns the sine of x, sin(x) where x is in radians

x is a \emph{double}, the return value is a \emph{double}

\emph{cos(x)}Returns the cosine of x, cos(x) where x is in radians

x is a \emph{double}, the return value is a \emph{double}

\emph{tan(x)}Returns the tangent of x, tan(x) where x is in radians

x is a \emph{double}, the return value is a \emph{double}

For all the above functions you can also pass in \emph{float}s instead
of \emph{doubles}. If you pass in \emph{float}s the return value is
usually a \emph{float}.

(The above list of functions is not exhaustive.)

\textbf{Exercise.} You should know from math classes that

log\textsubscript{10}(10\textsuperscript{x}) = x

Verify this for the case x = 3 and x = 4.5 using C++.

\textbf{Exercise.} Another version of the Pythagoras' theorem says that

sin\textsuperscript{2}(x) + cos\textsuperscript{2}(x) = 1

{[}Here sin\textsuperscript{2}(x) means (sin(x))\textsuperscript{2.}{]}
Verify this for the case x = 1 and x = 2.34 using C++.

Summary

There are two floating point types: \emph{double} and \emph{float}. A
double is capable of more precision than a float.

The following are binary operators available for floating point types
(both double and float): +, -, *, /. Note that \% is not defined for
floating point types. The same precedence and associative rules for
integers hold.

The declaration of double or float variables follows the same rules as
that or integer variables.

Keyboard input of double variables is similar to integer variables.

\emph{double()} returns the double of a value. \emph{int()} returns the
integer of a value.

Assignment and initialization operator (i.e. =) can be used between
integers and doubles. For instance you can assign an integer variable a
\emph{double} value. An automatic type casting occurs.

When a binary operator operates on an integer and a double, the integer
is replaced by its double (it's automatically type casted to a double)
before the operator is evaluated.

Exercises\\

\textbf{Q1.} Write a console program that does the following. It prompts
for the user's height, weight, length of his/her thumb, radius of
his/her skull and the number college classes the he/she has taken and
displays the user's IQ using the following formula

\emph{iq = PI * skullRadius}\textsuperscript{\emph{2 }}\emph{+ x
classes}

Your program should use the following for variable names.

\begin{itemize}
\tightlist
\item
  height
\item
  weight
\item
  thumbLength
\item
  skullRadius
\item
  classes
\item
  iq
\end{itemize}

PI is the constant 3.14159. Your code must of course contain the
definition of PI.

Q2. What's wrong with this program:

\begin{longtable}[]{@{}l@{}}
\toprule
\endhead
\begin{minipage}[t]{0.97\columnwidth}\raggedright
std::cout \textless\textless{} "Area of triangle\textbackslash n";

double height = 0.0; base = 0.0;

std::cout \textless\textless{} "Base: ";

std::cin \textgreater\textgreater{} based;

std::cout \textless\textless{} "Height: ";

std::cin \textgreater\textgreater{} height;

std::cout \textless\textless{} "Area of triangle with height "

\textless\textless{} base \textless\textless{} " and base "
\textless\textless{} height

\textless\textless{} " is "

\textless\textless{} (1 / 2) * base * height \textless\textless{}
std::end;\strut
\end{minipage}\tabularnewline
\bottomrule
\end{longtable}

Correct it by hand. Verify with C++.

Q3. Do you get the same output for the two print statements?

\begin{longtable}[]{@{}l@{}}
\toprule
\endhead
\begin{minipage}[t]{0.97\columnwidth}\raggedright
std::cout \textless\textless{} (1.2 + 3.4 + 5.6) / 3
\textless\textless{} std::endl;

std::cout \textless\textless{} (1.2 + 3.4 + 5.6) / 3.0
\textless\textless{} std::endl;\strut
\end{minipage}\tabularnewline
\bottomrule
\end{longtable}

Verify with C++

Q4. The following should print the average of three integers but does
not. Correct it by hand.

\begin{longtable}[]{@{}l@{}}
\toprule
\endhead
\begin{minipage}[t]{0.97\columnwidth}\raggedright
int x = 0, y = 0, z = 0;

std::cin \textgreater\textgreater{} x \textgreater\textgreater{} y
\textgreater\textgreater{} z \textgreater\textgreater{} std::endl;

std::cout \textless\textless{} (x + y + z) / 3 \textless\textless{}
std::endl;\strut
\end{minipage}\tabularnewline
\bottomrule
\end{longtable}

Verify with C++.

\textbf{Q5.} The formula for converting temperature in Fahrenheit to
Celsius is given by this formula:

5

c = --- (f - 32)

9

(where f is the temperature in Fahrenheit and c is the temperature in
Celsius.) Here's an execution of the program:

\begin{longtable}[]{@{}l@{}}
\toprule
\endhead
\begin{minipage}[t]{0.97\columnwidth}\raggedright
35

1.66667\strut
\end{minipage}\tabularnewline
\bottomrule
\end{longtable}

(Of course the input on the first line is the temperature in Fahrenheit
and the second line gives the temperature in Celsius.)

Q6. Write a program that does the following:

\begin{longtable}[]{@{}l@{}}
\toprule
\endhead
\begin{minipage}[t]{0.97\columnwidth}\raggedright
Differentiator!!!

Only for monomials right now ... :(

Enter c and a to differentiate c * x\^{}a: \textbf{1.2 5.6}

d

-\/- (1.2 * x \^{} 5.6) = 6.72 * x \^{} 4.6

dx\strut
\end{minipage}\tabularnewline
\bottomrule
\end{longtable}

Mathematically,

\begin{longtable}[]{@{}ll@{}}
\toprule
\endhead
d &\tabularnewline
& ( c x\textsuperscript{a }) = (ca) x \textsuperscript{a -
1}\tabularnewline
dx &\tabularnewline
\bottomrule
\end{longtable}

You do not need to know calculus to do this problem.

Q7. Write a program gets integers n and d from the user and prints the
d-th digit of n. If n is 13524, the 0-th digit is 4, the 1-st digit is
2, the 2\textsuperscript{nd} digit is 5, etc. Here's a test run:

\begin{longtable}[]{@{}l@{}}
\toprule
\endhead
\begin{minipage}[t]{0.97\columnwidth}\raggedright
13524 0

4\strut
\end{minipage}\tabularnewline
\bottomrule
\end{longtable}

Here's another test run:

\begin{longtable}[]{@{}l@{}}
\toprule
\endhead
\begin{minipage}[t]{0.97\columnwidth}\raggedright
13524 1

2\strut
\end{minipage}\tabularnewline
\bottomrule
\end{longtable}

\end{document}
