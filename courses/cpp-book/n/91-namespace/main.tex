% Options for packages loaded elsewhere
\PassOptionsToPackage{unicode}{hyperref}
\PassOptionsToPackage{hyphens}{url}
%
\documentclass[
]{article}
\usepackage{lmodern}
\usepackage{amssymb,amsmath}
\usepackage{ifxetex,ifluatex}
\ifnum 0\ifxetex 1\fi\ifluatex 1\fi=0 % if pdftex
  \usepackage[T1]{fontenc}
  \usepackage[utf8]{inputenc}
  \usepackage{textcomp} % provide euro and other symbols
\else % if luatex or xetex
  \usepackage{unicode-math}
  \defaultfontfeatures{Scale=MatchLowercase}
  \defaultfontfeatures[\rmfamily]{Ligatures=TeX,Scale=1}
\fi
% Use upquote if available, for straight quotes in verbatim environments
\IfFileExists{upquote.sty}{\usepackage{upquote}}{}
\IfFileExists{microtype.sty}{% use microtype if available
  \usepackage[]{microtype}
  \UseMicrotypeSet[protrusion]{basicmath} % disable protrusion for tt fonts
}{}
\makeatletter
\@ifundefined{KOMAClassName}{% if non-KOMA class
  \IfFileExists{parskip.sty}{%
    \usepackage{parskip}
  }{% else
    \setlength{\parindent}{0pt}
    \setlength{\parskip}{6pt plus 2pt minus 1pt}}
}{% if KOMA class
  \KOMAoptions{parskip=half}}
\makeatother
\usepackage{xcolor}
\IfFileExists{xurl.sty}{\usepackage{xurl}}{} % add URL line breaks if available
\IfFileExists{bookmark.sty}{\usepackage{bookmark}}{\usepackage{hyperref}}
\hypersetup{
  hidelinks,
  pdfcreator={LaTeX via pandoc}}
\urlstyle{same} % disable monospaced font for URLs
\usepackage{longtable,booktabs}
% Correct order of tables after \paragraph or \subparagraph
\usepackage{etoolbox}
\makeatletter
\patchcmd\longtable{\par}{\if@noskipsec\mbox{}\fi\par}{}{}
\makeatother
% Allow footnotes in longtable head/foot
\IfFileExists{footnotehyper.sty}{\usepackage{footnotehyper}}{\usepackage{footnote}}
\makesavenoteenv{longtable}
\setlength{\emergencystretch}{3em} % prevent overfull lines
\providecommand{\tightlist}{%
  \setlength{\itemsep}{0pt}\setlength{\parskip}{0pt}}
\setcounter{secnumdepth}{-\maxdimen} % remove section numbering

\author{}
\date{}

\begin{document}

91. Namespace

Objectives

\begin{itemize}
\tightlist
\item
  Create a namespace
\item
  Place identifiers in a namespace
\end{itemize}

Namespace

Run this:

\begin{longtable}[]{@{}l@{}}
\toprule
\endhead
\begin{minipage}[t]{0.97\columnwidth}\raggedright
\#include \textless iostream\textgreater{}

namespace yliow

\{

int x;

void f()

\{

std::cout \textless\textless{} "hello world" \textless\textless{}
std::endl;

\}

class C

\{\};

\}

int main()

\{

yliow::x = 42;

yliow::f();

yliow::C obj;

return 0;

\}\strut
\end{minipage}\tabularnewline
\bottomrule
\end{longtable}

A \textbf{namespace} is just a container of names (i.e., variables,
functions, structs, classes.)

Why?

Allows you to put names into a namespace to avoid naming conflicts.

For instance suppose John and Mary work on the same project. They can
choose different namespaces for their own things:

\begin{itemize}
\tightlist
\item
  john::print\_helloworld()
\item
  mary::print\_helloworld()
\end{itemize}

Openness of namespace

Namespaces are \textbf{open}: you can open it after it's closed.

\begin{longtable}[]{@{}l@{}}
\toprule
\endhead
\begin{minipage}[t]{0.97\columnwidth}\raggedright
namespace yliow

\{

class C

\{\};

\}

namespace yliow // open again!!!

\{

class D

\{\};

\}

int main()

\{

yliow::C obj1;

yliow::D obj2;

return 0;

\}\strut
\end{minipage}\tabularnewline
\bottomrule
\end{longtable}

The most common usage of namespace is probably to hold classes and
function prototypes:

\begin{longtable}[]{@{}l@{}}
\toprule
\endhead
\begin{minipage}[t]{0.97\columnwidth}\raggedright
// GameLib.h

namespace GameLib

\{

class vec2d \{ ... \};

class graphics \{ ... \};

class sound \{ ... \};

class physics \{ ... \};

\}\strut
\end{minipage}\tabularnewline
\bottomrule
\end{longtable}

If you're writing games and scientific simulations. You can split the
above.

\begin{longtable}[]{@{}l@{}}
\toprule
\endhead
\begin{minipage}[t]{0.97\columnwidth}\raggedright
// Physics.h

namespace GameLib

\{

class vec2d \{ ... \};

class physics \{ ... \};

\}\strut
\end{minipage}\tabularnewline
\bottomrule
\end{longtable}

\begin{longtable}[]{@{}l@{}}
\toprule
\endhead
\begin{minipage}[t]{0.97\columnwidth}\raggedright
// GameLib.h

\#include "Physics.h"

namespace GameLib

\{

class graphics \{ ... \};

class sound \{ ... \};

\}\strut
\end{minipage}\tabularnewline
\bottomrule
\end{longtable}

In this way, when you're writing a text-based sci simulations, you don't
need to compile with the graphics and sound class.

Using

You can avoid typing a namespace when you use it:

\begin{longtable}[]{@{}l@{}}
\toprule
\endhead
\begin{minipage}[t]{0.97\columnwidth}\raggedright
namespace yliow

\{

class X \{\};

class Y \{\};

\}

using namespace yliow;

int main()

\{

X obj; // do not need yliow::X

return 0;

\}\strut
\end{minipage}\tabularnewline
\bottomrule
\end{longtable}

\end{document}
