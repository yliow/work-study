% Options for packages loaded elsewhere
\PassOptionsToPackage{unicode}{hyperref}
\PassOptionsToPackage{hyphens}{url}
%
\documentclass[
]{article}
\usepackage{lmodern}
\usepackage{amssymb,amsmath}
\usepackage{ifxetex,ifluatex}
\ifnum 0\ifxetex 1\fi\ifluatex 1\fi=0 % if pdftex
  \usepackage[T1]{fontenc}
  \usepackage[utf8]{inputenc}
  \usepackage{textcomp} % provide euro and other symbols
\else % if luatex or xetex
  \usepackage{unicode-math}
  \defaultfontfeatures{Scale=MatchLowercase}
  \defaultfontfeatures[\rmfamily]{Ligatures=TeX,Scale=1}
\fi
% Use upquote if available, for straight quotes in verbatim environments
\IfFileExists{upquote.sty}{\usepackage{upquote}}{}
\IfFileExists{microtype.sty}{% use microtype if available
  \usepackage[]{microtype}
  \UseMicrotypeSet[protrusion]{basicmath} % disable protrusion for tt fonts
}{}
\makeatletter
\@ifundefined{KOMAClassName}{% if non-KOMA class
  \IfFileExists{parskip.sty}{%
    \usepackage{parskip}
  }{% else
    \setlength{\parindent}{0pt}
    \setlength{\parskip}{6pt plus 2pt minus 1pt}}
}{% if KOMA class
  \KOMAoptions{parskip=half}}
\makeatother
\usepackage{xcolor}
\IfFileExists{xurl.sty}{\usepackage{xurl}}{} % add URL line breaks if available
\IfFileExists{bookmark.sty}{\usepackage{bookmark}}{\usepackage{hyperref}}
\hypersetup{
  hidelinks,
  pdfcreator={LaTeX via pandoc}}
\urlstyle{same} % disable monospaced font for URLs
\usepackage{longtable,booktabs}
% Correct order of tables after \paragraph or \subparagraph
\usepackage{etoolbox}
\makeatletter
\patchcmd\longtable{\par}{\if@noskipsec\mbox{}\fi\par}{}{}
\makeatother
% Allow footnotes in longtable head/foot
\IfFileExists{footnotehyper.sty}{\usepackage{footnotehyper}}{\usepackage{footnote}}
\makesavenoteenv{longtable}
\setlength{\emergencystretch}{3em} % prevent overfull lines
\providecommand{\tightlist}{%
  \setlength{\itemsep}{0pt}\setlength{\parskip}{0pt}}
\setcounter{secnumdepth}{-\maxdimen} % remove section numbering

\author{}
\date{}

\begin{document}

12. while Loops

Objectives

\begin{itemize}
\tightlist
\item
  Write while-loops
\item
  Use sentinel values to exit a while-loop
\end{itemize}

Like the \emph{for}-loop, the purpose of the \emph{while}-loop is to
repeat a block of statements (or just a statement). While there is a
difference, once you know how the \emph{for}-loop works, the while-loop
is easy: the \emph{while}-loop is a simplification of the
\emph{for}-loop.

\emph{while}-loop

First run this. I'm not going to explain it since this is old stuff.

\begin{longtable}[]{@{}@{}}
\toprule
\endhead
\bottomrule
\end{longtable}

Now try this:

\begin{longtable}[]{@{}@{}}
\toprule
\endhead
\bottomrule
\end{longtable}

The program works the same right? One final modification:

\begin{longtable}[]{@{}@{}}
\toprule
\endhead
\bottomrule
\end{longtable}

That's all there is to it.

The \emph{while}-loop is just the \emph{for}-loop without the
initialization and the update part; the \emph{while}-loop only has the
boolean condition from the \emph{for}-loop. So if a \emph{while}-loop
must perform an initialization and/or an update, you can put them in the
following spots:

\begin{longtable}[]{@{}l@{}}
\toprule
\endhead
\begin{minipage}[t]{0.97\columnwidth}\raggedright
int sum = 0;

int i = 1; // INITIALIZATION

while (\textbf{i \textless= 10}) \textbf{// BOOLEAN CONDITION}

\{

sum += i;

i++; // UPDATE

\}\strut
\end{minipage}\tabularnewline
\bottomrule
\end{longtable}

By the way, there is a slight difference between the \emph{while}-loop
and \emph{for}-loop. The difference has nothing to do with the flow of
execution. It has to do with where we declare variables in loops. (This
was already mentioned earlier in the notes on \emph{for}-loops).

Look at these:

\begin{longtable}[]{@{}@{}}
\toprule
\endhead
\bottomrule
\end{longtable}

\begin{longtable}[]{@{}@{}}
\toprule
\endhead
\bottomrule
\end{longtable}

Because the variable \emph{i} in the second example is declared outside
the \emph{while}-loop, it is still available after exiting the
\emph{while}-loop. In the case of the \emph{for}-loop, the variable
\emph{i} is destroyed when you exit the \emph{for}-loop.

Flow of execution in a \emph{while} loop

Recall that the flow of execution for the for-loop looks like this:

for ( ; ; )

true

false

Here's the flow of execution of the while-loop:

while ( )

true

false

More Examples

Here's a typical example that uses the \emph{while}-loop.

\begin{longtable}[]{@{}l@{}}
\toprule
\endhead
\begin{minipage}[t]{0.97\columnwidth}\raggedright
int i = 0;

std::cout \textless\textless{} "gimme a number ...";

std::cin \textgreater\textgreater{} i;

while (i != 42)

\{

std::cout \textless\textless{} "so 42 is not your fav number?";

\textless\textless{} std::endl;

std::cout \textless\textless{} "gimme a number ...";

std::cin \textgreater\textgreater{} i;\\
\} \strut
\end{minipage}\tabularnewline
\bottomrule
\end{longtable}

Notice something about this \emph{while}-loop?

Hmmmm ....there is \textbf{no control variable} unlike our
\emph{for}-loop examples. Now try this:

\begin{longtable}[]{@{}l@{}}
\toprule
\endhead
\begin{minipage}[t]{0.97\columnwidth}\raggedright
int sum = 0;

int i = 0;

std::cout \textless\textless{} "gimme a number ...";

std::cin \textgreater\textgreater{} i;

while (i != 42)

\{

sum += i;

std::cout \textless\textless{} "sum: " \textless\textless{} sum
\textless\textless{} std::endl;

std::cout \textless\textless{} "gimme a number ...";

std::cin \textgreater\textgreater{} i;\\
\}

std::cout \textless\textless{} "final sum: " \textless\textless{} sum
\textless\textless{} std::endl;\strut
\end{minipage}\tabularnewline
\bottomrule
\end{longtable}

In general, if you want to process a bunch of numbers, you can run over
these numbers in two different ways: Either the program knows how to
compute them (example: the numbers are 1, 2, 3, 4, 5, 6, 7, 8, 9, 10) or
the program does not (example: it's based on user input).

For the first case, you should use the \textbf{for-loop} with a
\textbf{control or index or counter variable} running over the values.

For the second case, you should use the \textbf{while-loop}. To let the
user stop the data processing, you have to pick a special value for the
user to use to indicate end-of-data. This special value is called the
\textbf{sentinel value}.

Now I want you to step back and look at the ``structure'' of the above
program:

initialization

prompt user for data

while value of data is not sentinel value:

process data

prompt user for data

Note that there is code duplication:

initialization

prompt user for data

while value of data is not sentinel value:

process data

prompt user for data

Exercise. Write a program that continually ``rolls a die'' and prints
the face value of the die until 6 is reached. (Use rand() to generate a
random integer between 1 and 6.)

Exercise. Write a program that computes the product of integers entered
by the user. Use a sentinel value of 0 to indicate the end of data.
Here's the skeleton code:

\begin{longtable}[]{@{}l@{}}
\toprule
\endhead
\begin{minipage}[t]{0.97\columnwidth}\raggedright
int product = \_\_\_\_\_;

int i = 0;

std::cout \textless\textless{} "gimme a number ...";

std::cin \textgreater\textgreater{} i;

while (\_\_\_\_\_\_\_\_\_\_)

\{

\_\_\_\_\_\_\_\_\_\_\_\_\_\_\_\_\_\_\_\_\_;

std::cout \textless\textless{} "product: " \textless\textless{} product
\textless\textless{} std::endl;

std::cout \textless\textless{} "gimme a number ...";

std::cin \textgreater\textgreater{} i;\\
\}

std::cout \textless\textless{} "final product: " \textless\textless{}
product

\textless\textless{} std::endl;\strut
\end{minipage}\tabularnewline
\bottomrule
\end{longtable}

Exercise. Write a program that computes the cost of head grafting
surgery. Each head costs \$100.42. Display the total number of heads
requested. The program exits the loop when the number of heads requested
is zero. Here is the skeleton code:

\begin{longtable}[]{@{}l@{}}
\toprule
\endhead
\begin{minipage}[t]{0.97\columnwidth}\raggedright
int heads = 0;

double cost = 0.0;

int totalHeads = \_\_\_\_\_;

double totalCost = \_\_\_\_\_;

std::cout \textless\textless{} "how many heads? ";

std::cin \textgreater\textgreater{} heads;

while (\_\_\_\_\_\_\_\_\_\_)

\{

cost = \_\_\_\_\_\_\_\_\_\_\_\_\_\_\_\_\_\_\_\_\_\_\_\_\_\_\_\_\_\_\_\_;

totalHeads =
\_\_\_\_\_\_\_\_\_\_\_\_\_\_\_\_\_\_\_\_\_\_\_\_\_\_\_\_\_\_;

totalCost =
\_\_\_\_\_\_\_\_\_\_\_\_\_\_\_\_\_\_\_\_\_\_\_\_\_\_\_\_\_\_;

std::cout \textless\textless{} "total heads requested: "

\textless\textless{} totalHeads \textless\textless{} std::endl

\textless\textless{} "total cost: "

\textless\textless{} totalHeads \textless\textless{} '\textbackslash n'
\textless\textless{} std::endl;

// prompt for heads

\}

std::cout \textless\textless{} "your final bill:" \textless\textless{}
totalCost \textless\textless{} '\textbackslash n'

\textless\textless{} "i hope you enjoy your extra "

\textless\textless{} totalHeads \textless\textless{} " heads"
\textless\textless{} std::endl;\strut
\end{minipage}\tabularnewline
\bottomrule
\end{longtable}

\textbf{Exercise. }Due to popular demand and lack of staff and surgeons
(and donors), we have to limit the number of extra heads per visit.
Modify the above program so that the program exits the while-loop when
the number of heads requested is zero or when the total number of heads
requested exceeds 42.

\textbf{Exercise.} Write a simple number guessing game. The program
generates a random integer from 1 to 10 (inclusive), and continually
prompts the user to guess the number until the correct answer is given.
When you're done with that, get the program to print the number of
tries.

\textbf{Exercise.} Modify our multiplication game so that it keeps
posing questions until the user answers 5 questions correctly. While
prompting the user, print the number of correct answers. (Once you're
done with that, make a slight modification to prompt the user for the
number of correct answers before terminating the program.)

\textbf{Exercise.} Modify the above program so that the program
terminates when 5 \emph{\textbf{consecutive}} correct answers are given.

break

Fortunately the \emph{break} statement works the same way for the
\emph{while}-loop as it does for the for-loop.

\begin{longtable}[]{@{}l@{}}
\toprule
\endhead
\begin{minipage}[t]{0.97\columnwidth}\raggedright
int i = 0;

while (i \textless{} 1000)

\{

std::cout \textless\textless{} i \textless\textless{} std::endl;

if (i == 100) break;

i++;

\}

std::cout \textless\textless{} "out ... " \textless\textless{} i
\textless\textless{} std::endl;\strut
\end{minipage}\tabularnewline
\bottomrule
\end{longtable}

Again, the same advice for using the \emph{break} statement in the
\emph{for}-loop applies: Don't overuse it.

\textbf{Exercise.} What is the output? Work it out by hand and then
check it by running the program.

\begin{longtable}[]{@{}@{}}
\toprule
\endhead
\bottomrule
\end{longtable}

continue

Recall that this is the flow of execution for \emph{continue} in the
\emph{for}-loop:

The \emph{continue} statement works like this for the \emph{while}-loop:

\textbf{Exercise.} What is the output? Work it out by hand and then
check it by running the program.

\begin{longtable}[]{@{}@{}}
\toprule
\endhead
\bottomrule
\end{longtable}

\textbf{Exercise.} Here's an example. Run it.

\begin{longtable}[]{@{}l@{}}
\toprule
\endhead
\begin{minipage}[t]{0.97\columnwidth}\raggedright
int numHeads = 0;

std::cout \textless\textless{} "how many heads do you have? ";

std::cin \textgreater\textgreater{} numHeads;

while (numHeads != 0)

\{

std::cout \textless\textless{} "cool!" \textless\textless{} std::endl;\\

std::cout \textless\textless{} "how many heads do you have?";

std::cin \textgreater\textgreater{} numHeads;

\}\strut
\end{minipage}\tabularnewline
\bottomrule
\end{longtable}

Now insert code that prints ``rubbish'' and \emph{continue} to the next
iteration of the loop if \emph{numHeads} is negative:

\begin{longtable}[]{@{}l@{}}
\toprule
\endhead
\begin{minipage}[t]{0.97\columnwidth}\raggedright
int numHeads = 0;

std::cout \textless\textless{} "how many heads do you have? ";

std::cin \textgreater\textgreater{} numHeads;

while (numHeads != 0)

\{

// INSERT CODE HERE

std::cout \textless\textless{} "cool!" \textless\textless{} std::endl;\\

std::cout \textless\textless{} "how many heads do you have?";

std::cin \textgreater\textgreater{} numHeads;

\}\strut
\end{minipage}\tabularnewline
\bottomrule
\end{longtable}

More examples

Remember our example on computing the maximum of a list of values?

\begin{longtable}[]{@{}l@{}}
\toprule
\endhead
\begin{minipage}[t]{0.97\columnwidth}\raggedright
int x, y, z;

std::cin \textgreater\textgreater{} x \textgreater\textgreater{} y
\textgreater\textgreater{} z;

int max = x;

if (max \textless{} y) max = y;

if (max \textless{} z) max = z;

std::cout \textless\textless{} max \textless\textless{} std::endl;\strut
\end{minipage}\tabularnewline
\bottomrule
\end{longtable}

Let's write a program that prompts the user for integers and computes
the maximum of all integers entered. The program stops prompting when
the user enters -99999. Here's the skeleton code:

\begin{longtable}[]{@{}l@{}}
\toprule
\endhead
\begin{minipage}[t]{0.97\columnwidth}\raggedright
int x = 0;

std::cin \textgreater\textgreater{} x;

int max = x;

while (x != -99999)

\{

// process x

std::cout \textless\textless{} max \textless\textless{} std::endl;

std::cin \textgreater\textgreater{} x;

\}

std::cout \textless\textless{} "final max: " \textless\textless{} max
\textless\textless{} std::endl;\strut
\end{minipage}\tabularnewline
\bottomrule
\end{longtable}

Note that we need a ``special value'' of -99999 to stop the loop. What
do we do with x? We check if it's larger than what we have in max:

\begin{longtable}[]{@{}l@{}}
\toprule
\endhead
\begin{minipage}[t]{0.97\columnwidth}\raggedright
int x = 0;

std::cin \textgreater\textgreater{} x;

int max = x;

while (x != -99999)

\{

\textbf{if (max \textless{} x) max = x;}

std::cout \textless\textless{} max \textless\textless{} std::endl;

std::cin \textgreater\textgreater{} x;

\}

std::cout \textless\textless{} "final max: " \textless\textless{} max
\textless\textless{} std::endl;\strut
\end{minipage}\tabularnewline
\bottomrule
\end{longtable}

\textbf{Exercise.} Modify the above program so that it computes the
minimum instead of the maximum.

\textbf{Exercise.} Modify the above program so that it computes the sum
of all values entered by the user until 0 is entered. When 0 is entered,
the program prints the final sum and stops.

\textbf{Exercises.} Write a program that continually prompts the user
for numbers and prints the running average. The program stops prompting
when the user enters \emph{-99999}.

Summary

The format of the while-loop looks like this:

while (\emph{{[}bool expr{]}})

\emph{{[}stmt{]}}

where {[}bool expr{]} is a boolean expression and {[}stmt{]} is a single
statement or a block of statements.

The \emph{while}-loop is just the \emph{for}-loop without the
initialization and the update parts.

In general, if your loop requires a variable that runs over a sequence
of numbers that can be computed by the program, you should use the
\emph{for}-loop. Otherwise you should use the \emph{while}-loop. (There
are exceptions. But this advice will do for now).

The \emph{break} statement, if executed in the body of the
\emph{while}-loop, will exit the \emph{while}-loop and continues with
the execution of the statement immediately after the \emph{while}-loop.

The \emph{continue} statement, if executed in the body of the
\emph{while}-loop, will immediately bring the execution to the boolean
condition of the \emph{while}-loop.

Exercises

\hfill\break

Q1. Write a program that computes the gas mileage of a car. The program
continually prompts the user for the relevant data. In each iteration of
the loop, the program prompts the user for the miles traveled and the
gas used. When the user entered 0 for miles, the program stops prompting
the user, and displays the gas mileage (i.e. total miles divided by the
total gas usage).

Q2. Modify our multiplication game so that it keeps posing questions
until the user answers 80\% of the questions correctly. While prompting
the user, print the percentage of his/her correct answers.

Q3. Investigate the following problem: How many die rolls do you need to
get the value 6? What about two 6's? What about three 6's? ... What is
the likelihood of getting a 6? What is the likelihood of getting two
6's? ... What about the case of two consecutive 6's? Three consecutive
6's?

\end{document}
