% Options for packages loaded elsewhere
\PassOptionsToPackage{unicode}{hyperref}
\PassOptionsToPackage{hyphens}{url}
%
\documentclass[
]{article}
\usepackage{lmodern}
\usepackage{amssymb,amsmath}
\usepackage{ifxetex,ifluatex}
\ifnum 0\ifxetex 1\fi\ifluatex 1\fi=0 % if pdftex
  \usepackage[T1]{fontenc}
  \usepackage[utf8]{inputenc}
  \usepackage{textcomp} % provide euro and other symbols
\else % if luatex or xetex
  \usepackage{unicode-math}
  \defaultfontfeatures{Scale=MatchLowercase}
  \defaultfontfeatures[\rmfamily]{Ligatures=TeX,Scale=1}
\fi
% Use upquote if available, for straight quotes in verbatim environments
\IfFileExists{upquote.sty}{\usepackage{upquote}}{}
\IfFileExists{microtype.sty}{% use microtype if available
  \usepackage[]{microtype}
  \UseMicrotypeSet[protrusion]{basicmath} % disable protrusion for tt fonts
}{}
\makeatletter
\@ifundefined{KOMAClassName}{% if non-KOMA class
  \IfFileExists{parskip.sty}{%
    \usepackage{parskip}
  }{% else
    \setlength{\parindent}{0pt}
    \setlength{\parskip}{6pt plus 2pt minus 1pt}}
}{% if KOMA class
  \KOMAoptions{parskip=half}}
\makeatother
\usepackage{xcolor}
\IfFileExists{xurl.sty}{\usepackage{xurl}}{} % add URL line breaks if available
\IfFileExists{bookmark.sty}{\usepackage{bookmark}}{\usepackage{hyperref}}
\hypersetup{
  hidelinks,
  pdfcreator={LaTeX via pandoc}}
\urlstyle{same} % disable monospaced font for URLs
\usepackage{longtable,booktabs}
% Correct order of tables after \paragraph or \subparagraph
\usepackage{etoolbox}
\makeatletter
\patchcmd\longtable{\par}{\if@noskipsec\mbox{}\fi\par}{}{}
\makeatother
% Allow footnotes in longtable head/foot
\IfFileExists{footnotehyper.sty}{\usepackage{footnotehyper}}{\usepackage{footnote}}
\makesavenoteenv{longtable}
\setlength{\emergencystretch}{3em} % prevent overfull lines
\providecommand{\tightlist}{%
  \setlength{\itemsep}{0pt}\setlength{\parskip}{0pt}}
\setcounter{secnumdepth}{-\maxdimen} % remove section numbering

\author{}
\date{}

\begin{document}

03. Integer Variables

Objectives

\begin{itemize}
\tightlist
\item
  Choose valid and appropriate names for variables
\item
  Declare integer variable without initialization
\item
  Declare integer variable with initialization
\item
  Print the value of an integer variable
\item
  Using assignment operator to assign an integer value to an integer
  variable.
\item
  Assign an integer value entered from the keyboard to an integer
  variable
\item
  Use integer operators with integer variables
\end{itemize}

In this set of notes, we will learn to use variables. We will start with
integer variables. Once you know how to use integer variables, variables
of other types should be easy.

Variables!

We're now ready to put integer values into variables.

The concept of a variable in C++ is similar to the notion of a variable
from your math classes.

Once again instead of writing

\begin{longtable}[]{@{}l@{}}
\toprule
\endhead
\begin{minipage}[t]{0.97\columnwidth}\raggedright
// Name: John Doe

// File: hameggs.cpp

\#include \textless iostream\textgreater{}

int main()

\{

std::cout \textless\textless{} "ham\textbackslash n";

std::cout \textless\textless{} "eggs\textbackslash n";

return 0;

\}\strut
\end{minipage}\tabularnewline
\bottomrule
\end{longtable}

I will write

\begin{longtable}[]{@{}l@{}}
\toprule
\endhead
\begin{minipage}[t]{0.97\columnwidth}\raggedright
std::cout \textless\textless{} "ham\textbackslash n";

std::cout \textless\textless{} "eggs\textbackslash n";\strut
\end{minipage}\tabularnewline
\bottomrule
\end{longtable}

Creating and printing an integer variable

Run this program:

\begin{longtable}[]{@{}@{}}
\toprule
\endhead
\bottomrule
\end{longtable}

Modify the above program to get this:

\begin{longtable}[]{@{}@{}}
\toprule
\endhead
\bottomrule
\end{longtable}

Run it.

This is how you should think of the declaration of an integer variable.
Think of your program as creating a box that can contain one integer
value and giving the box a name. That's all there is to it.

There are two things you can do with this box called x. You can either

\begin{itemize}
\tightlist
\item
  \textbf{read} the value in the box
\end{itemize}

or you can

\begin{itemize}
\tightlist
\item
  \textbf{put} a new value into the box.
\end{itemize}

When you put a new value in, the old value is wiped out.

Let's look at the statement to declare our variable \emph{x}.

int x = 42;

The word \emph{\textbf{int}} tells C++ the kind of values that can be
placed into the box of \emph{x}. In this case it tells C++ that \emph{x}
can hold \textbf{integer values}. \emph{int} is a \textbf{type}. We'll
see more types later.

So the format of the statement to declare a variable is either:

\emph{{[}type{]} {[}variable name{]}};

or

\emph{\textbf{ }\textbf{{[}type{]} {[}var name{]}}}\textbf{ =
}\emph{\textbf{{[}init val{]}}}\textbf{;}

where the second format includes an initialization. If there is an
initialization, then the initial value is placed in the box. What if
there's no initialization?

Try this:

\begin{longtable}[]{@{}@{}}
\toprule
\endhead
\bottomrule
\end{longtable}

In general, it's a good practice to initialize variables.

\textbf{Exercise.} This won't work:

\begin{longtable}[]{@{}@{}}
\toprule
\endhead
\bottomrule
\end{longtable}

Run it and read the error message. What's the problem?

\textbf{Exercise.} This won't work either:

\begin{longtable}[]{@{}@{}}
\toprule
\endhead
\bottomrule
\end{longtable}

Try to run it. Read the error message. What's the problem?

You can assign values to your variable as many times as you like. Try
this

\begin{longtable}[]{@{}@{}}
\toprule
\endhead
\bottomrule
\end{longtable}

It's important to note that once you assign a value to \emph{x}, the old
value is overwritten. \emph{x} can hold only one value at one time.

Make sure you see the difference between the following two print
statements!!!

\begin{longtable}[]{@{}l@{}}
\toprule
\endhead
\begin{minipage}[t]{0.97\columnwidth}\raggedright
int x = 42;

std::cout \textless\textless{} "x" \textless\textless{} std::endl;

std::cout \textless\textless{} x \textless\textless{} std::endl;\strut
\end{minipage}\tabularnewline
\bottomrule
\end{longtable}

\textbf{Exercise.} Here's an easy warm up. Write a program that does the
following:

\begin{itemize}
\tightlist
\item
  Create an integer variable with the name \emph{i} with initial value
  of 0
\item
  Print the value of \emph{i}.
\end{itemize}

\textbf{Exercise.} Complete this program by writing two statements. The
first statement declares a variable. Here's what you need:

\begin{itemize}
\tightlist
\item
  The type of the variable is \emph{int}.
\item
  The name of the variable is \emph{num\_arms}
\item
  The initial value of the variable is 3.
\end{itemize}

Don't forget the underscore \emph{\_} in the variable name! The second
statement is a print statement. Refer to the print out and you should be
able to finish the program. The print statement must involve your
variable.

\begin{longtable}[]{@{}l@{}}
\toprule
\endhead
\begin{minipage}[t]{0.97\columnwidth}\raggedright
std::cout \textless\textless{} \strut
\end{minipage}\tabularnewline
\bottomrule
\end{longtable}

When you run the program you should get this output:

\begin{longtable}[]{@{}l@{}}
\toprule
\endhead
I have 3 arms.\tabularnewline
\bottomrule
\end{longtable}

\textbf{Exercise.} 42 is an integer value. You have already seen another
type of value: the C-string. Can you initialize an integer variable a
C-string value? Try this:

\begin{longtable}[]{@{}l@{}}
\toprule
\endhead
int x = "ham";\tabularnewline
\bottomrule
\end{longtable}

Or how about assignment:

\begin{longtable}[]{@{}@{}}
\toprule
\endhead
\bottomrule
\end{longtable}

Integer variables and operators

Try this:

\begin{longtable}[]{@{}@{}}
\toprule
\endhead
\bottomrule
\end{longtable}

Not too surprising, right?

Now let's try something with two variables. (It's not too surprising
that you can have two variables -- I hope!)

\begin{longtable}[]{@{}@{}}
\toprule
\endhead
\bottomrule
\end{longtable}

Now look at the above declaration of the variables. How many statements
are there?

You can also actually declare \emph{\textbf{several}} variables in
\emph{\textbf{one}} statement:

\begin{longtable}[]{@{}@{}}
\toprule
\endhead
\bottomrule
\end{longtable}

You can also choose either to initialize or not initialize the variables
you're declaring:

\begin{longtable}[]{@{}@{}}
\toprule
\endhead
\bottomrule
\end{longtable}

\textbf{Exercise.} Refer to the previous exercise. Is it possible to
initialize the second variable and not the first? Verify yourself with a
C++ program!

\textbf{Exercise.} Spot the syntax error(s):

\begin{longtable}[]{@{}@{}}
\toprule
\endhead
\bottomrule
\end{longtable}

Exercise. What is the output of this program:

\begin{longtable}[]{@{}l@{}}
\toprule
\endhead
\begin{minipage}[t]{0.97\columnwidth}\raggedright
int x = 3,y = 2, z = -2;

std::cout \textless\textless{} x + y * 3 / x -- z * 4 -- (2 -- y) \% x

\textless\textless{} std::endl;\strut
\end{minipage}\tabularnewline
\bottomrule
\end{longtable}

The assignment operator

OK. Pay attention to this.

In programming, = is not exactly the same as in Math. In Math, = can be
used to denote an \textbf{equation}. For instance:

5x = 3 + x

When you're told (in a Math class of course) to ``solve this equation'',
it means that you are to find ``the value(s) of x satisfying the
equation 5x = 3 + x''. (Of course in this case the solution for x is
3/4).

In most programming languages, = denotes means \textbf{assignment} or
\textbf{initialization}. When you have a program statement that looks
like:

\emph{x = y + z + 3;}

it means:

\begin{itemize}
\tightlist
\item
  Evaluate the expression on the right, and
\item
  Give the value on the right to the variable on the left
\end{itemize}

\textbf{Exercise.} Using the above recipe for =, ``execute'' this
program by hand.

\begin{longtable}[]{@{}@{}}
\toprule
\endhead
\bottomrule
\end{longtable}

Now verify your work by running the program.

Of course in math an \textbf{equation} like

x = x + 1

becomes

0 = 1

which is gibberish. In other words you cannot solve this equation (at
least not in real numbers). In C++

\emph{x = x + 1;}

makes sense because = is an assignment and has nothing to do with
equations.

So remember this: In math = means equation. In C++ = means assignment or
initialization.

\textbf{Exercise. }First figure out the output by hand. Next run the
program

\begin{longtable}[]{@{}@{}}
\toprule
\endhead
\bottomrule
\end{longtable}

\textbf{Exercise.} True or false: The output of this

\begin{longtable}[]{@{}@{}}
\toprule
\endhead
\bottomrule
\end{longtable}

is

\begin{longtable}[]{@{}@{}}
\toprule
\endhead
\bottomrule
\end{longtable}

Check with your C++ compiler.

\textbf{Exercise.} Try this:

\begin{longtable}[]{@{}@{}}
\toprule
\endhead
\bottomrule
\end{longtable}

Run this. Can you explain why there's an error?

Keyboard input (console window)

Run this program.

\begin{longtable}[]{@{}@{}}
\toprule
\endhead
\bottomrule
\end{longtable}

Enter an integer value (example: 42) and press the enter key. Run the
program again, entering a different value (example: 167).

Notice that the value of \emph{x} printed is the value entered. And
who's the culprit that gave the value entered to x? Clearly it's

std::cin \textgreater\textgreater{} x;

In other words the above statement puts the value entered through the
keyboard into the variable \emph{x}.

Get it?

Exercise. Does this work? Run it to verify.

\begin{longtable}[]{@{}@{}}
\toprule
\endhead
\bottomrule
\end{longtable}

Why?

Exercise. Write a program that prompts you for your age, and then prints
it. The following is an execution of the program when the user enters 15
(the value is in bold).

\begin{longtable}[]{@{}l@{}}
\toprule
\endhead
\begin{minipage}[t]{0.97\columnwidth}\raggedright
How old are you? \textbf{15}

You are 15 years old. Of course you could be lying.\strut
\end{minipage}\tabularnewline
\bottomrule
\end{longtable}

Here's another execution of the same program where the user enters 142:

\begin{longtable}[]{@{}l@{}}
\toprule
\endhead
\begin{minipage}[t]{0.97\columnwidth}\raggedright
How old are you? \textbf{142}

You are 142 years old. Of course you could be lying.\strut
\end{minipage}\tabularnewline
\bottomrule
\end{longtable}

Now let's try an example with two inputs:

\begin{longtable}[]{@{}@{}}
\toprule
\endhead
\bottomrule
\end{longtable}

Run this twice, entering different values for \emph{x} and \emph{y}. As
you run the program, try to match what you see in the console window
with the corresponding statement in the program.

Exercise. Write a program that does the following. Here's one execution
of the program:

\begin{longtable}[]{@{}l@{}}
\toprule
\endhead
\begin{minipage}[t]{0.97\columnwidth}\raggedright
Multiplicator !!!

Enter an integer: \textbf{5}

Enter another integer: \textbf{7}

5 * 7 = 35\strut
\end{minipage}\tabularnewline
\bottomrule
\end{longtable}

Here is another execution of the same program:

\begin{longtable}[]{@{}l@{}}
\toprule
\endhead
\begin{minipage}[t]{0.97\columnwidth}\raggedright
Multiplicator !!!

Enter an integer: --\textbf{3}

Enter another integer: \textbf{4}

--3 * 4 = --12\strut
\end{minipage}\tabularnewline
\bottomrule
\end{longtable}

Recall that for output you can print several things:

std::cout \textless\textless{} x \textless\textless{} ", "
\textless\textless{} y \textless\textless{} std::endl;

It turns out that input is very similar. You can have one input
statement that handles more than one variable:

int x = 0, y = 0;

std::cout \textless\textless{} "Give me x and y: ";

std::cin \textgreater\textgreater{} x \textgreater\textgreater{} y;

std::cout \textless\textless{} x \textless\textless{} ", "
\textless\textless{} y \textless\textless{} std::endl;

Run it. When you're prompted for x and y, you can either enter the
values separated by spaces (1 or more):

\begin{longtable}[]{@{}l@{}}
\toprule
\endhead
\begin{minipage}[t]{0.97\columnwidth}\raggedright
Give me x and y: \textbf{2 5}

2, 5\strut
\end{minipage}\tabularnewline
\bottomrule
\end{longtable}

Or you can separate the values with tabs (1 or more):

\begin{longtable}[]{@{}l@{}}
\toprule
\endhead
\begin{minipage}[t]{0.97\columnwidth}\raggedright
Give me x and y: \textbf{2 5}

2, 5\strut
\end{minipage}\tabularnewline
\bottomrule
\end{longtable}

Or you can separate the values with newlines (1 or more, using the Enter
key):

\begin{longtable}[]{@{}l@{}}
\toprule
\endhead
\begin{minipage}[t]{0.97\columnwidth}\raggedright
Give me x and y: \textbf{2}

5

2, 5\strut
\end{minipage}\tabularnewline
\bottomrule
\end{longtable}

Or any combination of spaces, newlines, or tabs:

\begin{longtable}[]{@{}l@{}}
\toprule
\endhead
\begin{minipage}[t]{0.97\columnwidth}\raggedright
Give me x and y:

2

5

2, 5\strut
\end{minipage}\tabularnewline
\bottomrule
\end{longtable}

Basically your program will scan through all the data entered, skipping
whitespaces (spaces, tabs, newlines) if necessary, to find the relevant
integer values to be placed into variables \emph{x} and \emph{y}.

Computational model

Sometimes it's useful to ``visualize'' the execution of a program. The
following is a model of the computer. We will execute a program on it.
Remember that \textbf{this is only a model}. In fact, later I will need
to increase the power of this ``computer'' because this model is not
powerful enough.

To illustrate the computation in this computer, let's run it on the
following program:

\begin{longtable}[]{@{}@{}}
\toprule
\endhead
\bottomrule
\end{longtable}

So here's your computer:

The speech bubble is the output. Everything printed appears in it. The
CPU performs math. Now when you run the program, the computer loads the
program:

\begin{longtable}[]{@{}@{}}
\toprule
\endhead
\bottomrule
\end{longtable}

And when it runs, the first thing that happens is that it creates a
place for keeping variables:

\begin{longtable}[]{@{}@{}}
\toprule
\endhead
\bottomrule
\end{longtable}

You can think of the box named main as an area of putting variables.

It then goes on to the next statement, a declaration and initialization:

\begin{longtable}[]{@{}@{}}
\toprule
\endhead
\bottomrule
\end{longtable}

Now for an assignment:

\begin{longtable}[]{@{}@{}}
\toprule
\endhead
\bottomrule
\end{longtable}

\begin{longtable}[]{@{}@{}}
\toprule
\endhead
\bottomrule
\end{longtable}

\begin{longtable}[]{@{}@{}}
\toprule
\endhead
\bottomrule
\end{longtable}

\begin{longtable}[]{@{}@{}}
\toprule
\endhead
\bottomrule
\end{longtable}

Done! The next statement is the declaration and initialization of y:

\begin{longtable}[]{@{}@{}}
\toprule
\endhead
\bottomrule
\end{longtable}

This is followed by another assignment:

\begin{longtable}[]{@{}@{}}
\toprule
\endhead
\bottomrule
\end{longtable}

\begin{longtable}[]{@{}@{}}
\toprule
\endhead
\bottomrule
\end{longtable}

\begin{longtable}[]{@{}@{}}
\toprule
\endhead
\bottomrule
\end{longtable}

\begin{longtable}[]{@{}@{}}
\toprule
\endhead
\bottomrule
\end{longtable}

\begin{longtable}[]{@{}@{}}
\toprule
\endhead
\bottomrule
\end{longtable}

The important thing to remember is that right now the computer executes
one statement at a time from top to bottom.

Again, this is only a model.

Exercise. Complete the main box at the point when C++ finish executing
the statement in red.

When you're done, you can check if you're correct by adding a statement
to print the value of \emph{a} and \emph{b} at the right place and run
the program with your C++ compiler to see if you're correct.

Exercise. Take a sheet of paper and perform a trace like the above with
the following code:

\begin{longtable}[]{@{}@{}}
\toprule
\endhead
\bottomrule
\end{longtable}

What is the output? Write it down here using one square for each output
character or digit:

\begin{longtable}[]{@{}lllllllllllllll@{}}
\toprule
\endhead
& & & & & & & & & & & & & &\tabularnewline
& & & & & & & & & & & & & &\tabularnewline
& & & & & & & & & & & & & &\tabularnewline
& & & & & & & & & & & & & &\tabularnewline
\bottomrule
\end{longtable}

Run this program with your C++ compiler and compare.

Tracing programs with repeating chunks of statements

The following exercises are simple but extremely important. They all
involve tracing programs \ldots{} with repeating chunks of code.

Exercise. This is an important exercise! Take a sheet of paper and
perform a trace like the above with the follow code:

\begin{longtable}[]{@{}@{}}
\toprule
\endhead
\bottomrule
\end{longtable}

What is the output? Write it down here using one square for each output
character or digit:

\begin{longtable}[]{@{}lllllllllllllll@{}}
\toprule
\endhead
& & & & & & & & & & & & & &\tabularnewline
& & & & & & & & & & & & & &\tabularnewline
& & & & & & & & & & & & & &\tabularnewline
& & & & & & & & & & & & & &\tabularnewline
& & & & & & & & & & & & & &\tabularnewline
\bottomrule
\end{longtable}

Next run this program and compare your trace with the program's output.

Exercise. This is an important exercise! Take a sheet of paper and
perform a trace like the above with the follow code:

\begin{longtable}[]{@{}@{}}
\toprule
\endhead
\bottomrule
\end{longtable}

What is the output? Write it down here using one square for each output
character or digit:

\begin{longtable}[]{@{}lllllllllllllll@{}}
\toprule
\endhead
& & & & & & & & & & & & & &\tabularnewline
& & & & & & & & & & & & & &\tabularnewline
& & & & & & & & & & & & & &\tabularnewline
& & & & & & & & & & & & & &\tabularnewline
& & & & & & & & & & & & & &\tabularnewline
\bottomrule
\end{longtable}

Next run this program and compare your trace with the program's output.

Exercise. This is an important exercise! Take a sheet of paper and
perform a trace like the above with the follow code:

\begin{longtable}[]{@{}@{}}
\toprule
\endhead
\bottomrule
\end{longtable}

What is the output? Write it down here using one square for each output
character or digit:

\begin{longtable}[]{@{}lllllllllllllll@{}}
\toprule
\endhead
& & & & & & & & & & & & & &\tabularnewline
& & & & & & & & & & & & & &\tabularnewline
& & & & & & & & & & & & & &\tabularnewline
& & & & & & & & & & & & & &\tabularnewline
& & & & & & & & & & & & & &\tabularnewline
\bottomrule
\end{longtable}

Next run this program and compare your trace with the program's output.

Exercise. The following is a \emph{very} important exercise!!! Take a
sheet of paper and perform a trace like the above with the follow code:

\begin{longtable}[]{@{}@{}}
\toprule
\endhead
\bottomrule
\end{longtable}

What is the output? Write it down here using one square for each output
character or digit:

\begin{longtable}[]{@{}lllllllllllllll@{}}
\toprule
\endhead
& & & & & & & & & & & & & &\tabularnewline
& & & & & & & & & & & & & &\tabularnewline
& & & & & & & & & & & & & &\tabularnewline
& & & & & & & & & & & & & &\tabularnewline
& & & & & & & & & & & & & &\tabularnewline
\bottomrule
\end{longtable}

Next run this program and compare your trace with the program's output.

What has the above got to do with 0 + 1 + 2 + 3 + 4?

Exercise. The following is a \emph{very} important exercise!!! Take a
sheet of paper and perform a trace like the above with the follow code.
WARNING: This is very similar to the above.

\begin{longtable}[]{@{}@{}}
\toprule
\endhead
\bottomrule
\end{longtable}

What is the output? Write it down here using one square for each output
character or digit:

\begin{longtable}[]{@{}lllllllllllllll@{}}
\toprule
\endhead
& & & & & & & & & & & & & &\tabularnewline
& & & & & & & & & & & & & &\tabularnewline
& & & & & & & & & & & & & &\tabularnewline
& & & & & & & & & & & & & &\tabularnewline
& & & & & & & & & & & & & &\tabularnewline
\bottomrule
\end{longtable}

Exercise. The following is also a \emph{very} important example! Take a
sheet of paper and perform a trace like the above with the follow code:

\begin{longtable}[]{@{}l@{}}
\toprule
\endhead
\begin{minipage}[t]{0.97\columnwidth}\raggedright
\#include \textless iostream\textgreater{}

int main()

\{

int i = 1, p = 1;

std::cout \textless\textless{} i \textless\textless{} ' '
\textless\textless{} p \textless\textless{} '\textbackslash n';

p = p * i;

i = i + 1;

std::cout \textless\textless{} i \textless\textless{} ' '
\textless\textless{} p \textless\textless{} '\textbackslash n';

p = p + i;

i = i + 1;

std::cout \textless\textless{} i \textless\textless{} ' '
\textless\textless{} p \textless\textless{} '\textbackslash n';

p = p + i;

i = i + 1;

std::cout \textless\textless{} i \textless\textless{} ' '
\textless\textless{} p \textless\textless{} '\textbackslash n';

p = p + i;

i = i + 1;

std::cout \textless\textless{} i \textless\textless{} ' '
\textless\textless{} p \textless\textless{} '\textbackslash n';

return 0;

\}\strut
\end{minipage}\tabularnewline
\bottomrule
\end{longtable}

What is the output? Write it down here using one square for each output
character or digit:

\begin{longtable}[]{@{}lllllllllllllll@{}}
\toprule
\endhead
& & & & & & & & & & & & & &\tabularnewline
& & & & & & & & & & & & & &\tabularnewline
& & & & & & & & & & & & & &\tabularnewline
& & & & & & & & & & & & & &\tabularnewline
& & & & & & & & & & & & & &\tabularnewline
\bottomrule
\end{longtable}

Make sure your trace-by-hand matches the output of the program when you
run it. What the above got to do with 1 x 2 x 3 x 4? What if I changed
the initial value \emph{p} to 0? Redo the trace and see what you get.

Exercise. The following is also important. Take a sheet of paper and
perform a trace \ldots{} WAIT!!! Look at the code:

\begin{longtable}[]{@{}@{}}
\toprule
\endhead
\bottomrule
\end{longtable}

Can you very quickly figure out the output without a detail trace? (If
you've understood the above exercises completely you should be able to
do it. Next, run this program and compare it with your trace. Make sure
your trace-by-hand matches the output of the program when you run it.

Note that in the above exercises, a chunk of code is repeated. For
instance for this program:

\begin{longtable}[]{@{}l@{}}
\toprule
\endhead
\begin{minipage}[t]{0.97\columnwidth}\raggedright
\#include \textless iostream\textgreater{}

int main()

\{

int i = 0, s = 0;

std::cout \textless\textless{} i \textless\textless{} ' '
\textless\textless{} s \textless\textless{} '\textbackslash n';

s = s + i;

i = i + 1;

std::cout \textless\textless{} i \textless\textless{} ' '
\textless\textless{} s \textless\textless{} '\textbackslash n';

s = s + i;

i = i + 1;

std::cout \textless\textless{} i \textless\textless{} ' '
\textless\textless{} s \textless\textless{} '\textbackslash n';

s = s + i;

i = i + 1;

std::cout \textless\textless{} i \textless\textless{} ' '
\textless\textless{} s \textless\textless{} '\textbackslash n';

s = s + i;

i = i + 1;

std::cout \textless\textless{} i \textless\textless{} ' '
\textless\textless{} s \textless\textless{} '\textbackslash n';

return 0;

\}\strut
\end{minipage}\tabularnewline
\bottomrule
\end{longtable}

notice that this code segment is repeated:

\begin{longtable}[]{@{}@{}}
\toprule
\endhead
\bottomrule
\end{longtable}

Frequently, after tracing a few chunks of repeated code, once you see
the pattern, you can guess the output for the rest.

\textbf{Exercises.} Go through all the examples above and locate
repeated code segments. Can you figure out quickly the output without
tracing every line slowing?

Rules for creating names

Try this

\begin{longtable}[]{@{}@{}}
\toprule
\endhead
\bottomrule
\end{longtable}

(Don't panic. Don't call 911. You \emph{should} get an error.)

Try this:

\begin{longtable}[]{@{}@{}}
\toprule
\endhead
\bottomrule
\end{longtable}

Again you should get an error.

Rules

There are rules that you must follow for creating variable names in C++.
If you don't follow them, your C++ compiler will yell at you and refuse
to compile and run your code.

\begin{itemize}
\item
  You can only use alphanumerics (i.e. a-z, A-Z, 0-9), underscore (i.e.
  \_).
\item
  The first character of the name cannot be numeric.
\item
  You cannot use a keyword as a name. A \textbf{keyword} is just a word
  that C++ has already taken. For instance the word ``int'' and
  ``return'', etc. are already used by C++.
\end{itemize}

Again \ldots{} these rules MUST be followed otherwise C++ will shout at
you.

Common Practices

The following are \emph{\textbf{not}} rules but common practices in
naming C++ variables. This means that the following are advice and your
compiler will probably run your code even if you ignore the advice.
However you should still follow the advice because all good programers
should.

\begin{enumerate}
\def\labelenumi{\arabic{enumi}.}
\tightlist
\item
  A name can be as long as you like (depending on the compiler). But be
  reasonable. Something like
\end{enumerate}

numberOfSpaceInvadersStillAliveAndWellAndCausingTrouble

is hard to read and even harder to type \ldots{} \emph{correctly!!!}

\begin{enumerate}
\def\labelenumi{\arabic{enumi}.}
\setcounter{enumi}{1}
\item
  The first character is a lowercase letter. (You'll see exceptions
  later.)
\item
  If a variable represents something in the real world you're modeling,
  you should use a meaningful name for it.
\item
  It's a convention that if your variable name is made up of words, you
  should either
\end{enumerate}

\begin{enumerate}
\def\labelenumi{\arabic{enumi}.}
\tightlist
\item
  capitalize the first letter of each word in the variable name (except
  for the first word), or
\item
  insert an underscore between the words.
\end{enumerate}

Example:

\emph{int energylevel = 10000; // BAD!}

int energyLevel = 10000; // Good

int energy\_level = 10000; // Good

In general you choose either style 1 or 2 but not both in the same
program.

\begin{enumerate}
\def\labelenumi{\arabic{enumi}.}
\setcounter{enumi}{4}
\tightlist
\item
  Although you can start with an underscore, you should not. This is
  because your C++ compiler actually includes extra variables into the
  final program. These special variables usually have names starting
  with either one or maybe even two underscores.
\end{enumerate}

\begin{enumerate}
\def\labelenumi{\arabic{enumi}.}
\setcounter{enumi}{4}
\tightlist
\item
  Of course variable names are case-sensitive:
\end{enumerate}

\emph{int energylevel = 10000;}

int energyLevel = 10000;

Make sure you know the difference between rules and advice for choosing
variable names.

If you work for a company, you might find that they have their own rules
in order to make code more readable according to \emph{their} practices.

Jargon

In programming languages, an \textbf{identifier} is just a technical
term for ``name''. So a variable name is an identifier.

You have in fact seen another name: the word \emph{main} is an
identifier. I'll come to this later.\\

\textbf{Exercise.} Which of the following variable names are valid?

\begin{itemize}
\tightlist
\item
  gpa
\item
  \_gpa
\item
  gradePointAverage
\item
  gradepointaverage
\item
  greatpointavarag
\item
  grade\_point\_average
\item
  grade\$point\$average
\item
  gradePointAverage3
\item
  3gradePointAverage
\item
  grade Point Average
\end{itemize}

Check with your C++ compiler.

(Note that you cannot use \$ in variable names. Some programming
languages allow it, but not C++.)

\textbf{Exercise.} True or false? The output of this code

\begin{longtable}[]{@{}l@{}}
\toprule
\endhead
\begin{minipage}[t]{0.97\columnwidth}\raggedright
int num arms = 3;

int num legs = 4;

std::cout \textless\textless{} "number of limbs: "

\textless\textless{} num arms + num legs

\textless\textless{} std::endl;\strut
\end{minipage}\tabularnewline
\bottomrule
\end{longtable}

is

\begin{longtable}[]{@{}l@{}}
\toprule
\endhead
7\tabularnewline
\bottomrule
\end{longtable}

\textbf{Exercise.} Rename the variables using more descriptive variable
names:

\begin{longtable}[]{@{}l@{}}
\toprule
\endhead
\begin{minipage}[t]{0.97\columnwidth}\raggedright
int x = 0;

std::cout \textless\textless{} "Number of martians: ";

std::cin \textgreater\textgreater{} x;

int y = 0;

std::cout \textless\textless{} "Fingers on each martian: ";

std::cin \textgreater\textgreater{} y;

std::cout \textless\textless{} "Fingers altogether: "
\textless\textless{} x * y

\textless\textless{} std::endl;\strut
\end{minipage}\tabularnewline
\bottomrule
\end{longtable}

Swapping values in variables

Pay attention!!!

Using the computational model in the notes, trace the following
extremely important trick. It ``swaps'' the values of variables \emph{x}
and \emph{y}.

\begin{longtable}[]{@{}l@{}}
\toprule
\endhead
\begin{minipage}[t]{0.97\columnwidth}\raggedright
\#include \textless iostream\textgreater{}

int main()

\{

int x = 4, y = 2;

std::cout \textless\textless{} x \textless\textless{} ", "
\textless\textless{} y \textless\textless{} std::endl;\\

int t = x;

x = y;

y = t;

std::cout \textless\textless{} x \textless\textless{} ", "
\textless\textless{} y \textless\textless{} std::endl;

return 0;\\
\}\strut
\end{minipage}\tabularnewline
\bottomrule
\end{longtable}

Exercise. Complete the following program so that it swaps the variables
\emph{x}, \emph{y}, \emph{z} ``in a circle'':

\begin{longtable}[]{@{}l@{}}
\toprule
\endhead
\begin{minipage}[t]{0.97\columnwidth}\raggedright
\#include \textless iostream\textgreater{}

int main()

\{

int x = 4, y = 2, z = -6;

std::cout \textless\textless{} x \textless\textless{} ' '
\textless\textless{} y \textless\textless{} ' ' \textless\textless{} z

\textless\textless{} std::endl;\\

std::cout \textless\textless{} x \textless\textless{} ' '
\textless\textless{} y \textless\textless{} ' ' \textless\textless{} z

\textless\textless{} std::endl;\\

return 0;\\
\}\strut
\end{minipage}\tabularnewline
\bottomrule
\end{longtable}

The expected output is

\begin{longtable}[]{@{}l@{}}
\toprule
\endhead
-6 4 2\tabularnewline
\bottomrule
\end{longtable}

Limit On C++ Integer Value

In math, there is no limit on integer values. There's nothing wrong with
your math teacher writing

1232332498530453485345

on the board. One thing to watch when writing C++ programs is that C++
integers have limits. On a 32-bit machine, an integer value
\emph{usually} ranges from -2\textsuperscript{31} to
2\textsuperscript{31 } -- 1, i.e.

-2,147,483,648 to 2,147,483,647

What happens when you go beyond? The numbers ``cycles around''. In other
words if you declare:

int x = 2147483647;

x = x + 1;

std::cout \textless\textless{} x \textless\textless{} std::endl;

you'll find that x becomes -2147483648. Yikes!!! Furthermore

int y = -2147483648;

y = y - 1;

will give y the value 2147483647.

You know what that means right? If you write a program for a bank and
you have the following:

int total\_assets = 2147483647;

int deposit;

std::cin \textgreater\textgreater{} deposit;

total\_assets = total\_assets + deposit;

you could very well end up with a negative number!!!

At this point you need not worry about about how programmers handle this
problem (as well as lawsuits). You just need to know that most

C++ integers in a 32-bit machine is between roughly -2 billion to 2
billion.

Summary

An integer variable has a name and a space for an integer value. You can
think of the space as a box. You can only put integer values into this
box. In C++ the integer type is written \emph{int}.

You can declare an integer variable like this:

int\emph{ {[}name of variable{]}};

or with an initial value like this:

int\emph{ {[}name of variable{]}} = \emph{{[}initial value{]}};

For instance:

int x;

int y = 1;

You can refer to a variable only after it's declared.

The name of a variable is an example of an identifier. A identifier is
just a name. Identifiers in C++ can be made up of alphanumerics and the
underscore. The first character of an identifier cannot be a digit. An
identifier should not begin with an underscore. To make a variable
readable, you can either capitalize the first character of each word in
the variable's name or insert an underscore between words. For instance:

int numLives = 3;

int num\_lives = 3;

Using an uninitialized integer variable will either cause an error or a
random value is given. (This depends on your compiler.)

You can declare more than one variable in one statement with or without
initialization:

int x = 0, y, z = 5000;

An expression can contain variables:

x + y + 2 * z -- a

Such an expression is evaluated by referring to the values of the
relevant variables at the time of evaluation. C++ evaluates according to
the usual precedence and associative rules.

You can assign a variable the value of an expression. For instance

i = x + y + 2 * z -- a;

An assignment statement like the above means:

\begin{enumerate}
\def\labelenumi{\arabic{enumi}.}
\tightlist
\item
  Evaluate the expression on the right
\item
  Give the value on the right to the variable on the left.
\end{enumerate}

Printing a variable results in printing the value of the variable.
Printing an expression results in printing the value of the expression
after the expression is evaluated.

You can give an integer value to an integer variable by entering an
integer at the keyboard:

int x;

std::cin \textgreater\textgreater{} x;

You can have an input statement for two variables. For instance:

int x, y;

std::cin \textgreater\textgreater{} x \textgreater\textgreater{} y;

In this case the integer values entered by the user from the keyboard
can be separated any amount of whitespace.

On a 32-bit machine, a C++ \emph{int} value ranges from
-2\textsuperscript{31} to 2\textsuperscript{31 } -- 1, i.e.

-2,147,483,648 to 2,147,483,647

Note: At this point you cannot declare a variable twice, i.e. the
following is an error:

int x = 0;

int x = 42;

Later when we talk about scopes, I'll show it can be done, if variables
(with the same name) are declared in different scopes.

Exercises.

1. Write a program that does the following when you execute it:

\begin{longtable}[]{@{}l@{}}
\toprule
\endhead
\begin{minipage}[t]{0.97\columnwidth}\raggedright
Enter an integer: \textbf{135}

The rightmost digit is 5.\strut
\end{minipage}\tabularnewline
\bottomrule
\end{longtable}

Here's another execution of the same program:

\begin{longtable}[]{@{}l@{}}
\toprule
\endhead
\begin{minipage}[t]{0.97\columnwidth}\raggedright
Enter an integer: \textbf{42}

The rightmost digit is 2.\strut
\end{minipage}\tabularnewline
\bottomrule
\end{longtable}

2. According to Albert Einstein, E = mc\textsuperscript{2}. Write a
program that prompts the user for an integer value for m, an integer
value for c, and prints the value of E according to the formula.

3. According to Elbert Ainstein, your IQ is given by the following
formula:

IQ = 3 * w / h + (3 + f) / 42

where w is your waist (in inches), h is your height (in inches) and f is
the number of fingers on your hands. The divisions are all integer
divisions (i.e. quotients). Write a program that prompts the user for w,
h, and f and print his/her IQ. (No ... the software won't sell.)

4. The following is a very important formula that appears frequently in
math and computer science:

\[s=\frac{n{({n-1})}}{2}\]

It's the sum of integer from 1 to n; it's also the number of ways to
choose two out of a total of n toppings when you're at an ice cream
shop. First work out the value of s for n = 1, 5, and 10 by hand. Write
a program that prompts the user for a value of n and prints the value of
s using the above formula. Test your program against the values you
computed earlier.

5. What's wrong with this program?

\begin{longtable}[]{@{}l@{}}
\toprule
\endhead
\begin{minipage}[t]{0.97\columnwidth}\raggedright
\#include \textless iostream\textgreater{}

int main()

\{

std::cout \textless\textless{} "rectangle area!!!" \textless\textless{}
std::endl;

std::cout \textless\textless{} "length: ";

std::cin \textgreater\textgreater{} w;

std::cout \textless\textless{} "width: ";

std::cin \textgreater\textgreater{} h;

std::cout \textless\textless{} "area = " \textless\textless{} wh
\textless\textless{} std::cout;

return 0;\\
\}\strut
\end{minipage}\tabularnewline
\bottomrule
\end{longtable}

Now verify your correction with C++.

6. Write down the output of this program or explain why it won't run:

\begin{longtable}[]{@{}l@{}}
\toprule
\endhead
\begin{minipage}[t]{0.97\columnwidth}\raggedright
\#include \textless iostream\textgreater{}

int main()

\{

int x = 1;

int y = 2;

int z = x + y;

std::cout \textless\textless{} x \textless\textless{} y
\textless\textless{} z \textless\textless{} std::endl;

x = x + z;

std::cout \textless\textless{} y \textless\textless{} x
\textless\textless{} z \textless\textless{} std::endl;

y = x * z;

std::cout \textless\textless{} z \textless\textless{} x
\textless\textless{} y \textless\textless{} std::endl;

return 0;\\
\} \strut
\end{minipage}\tabularnewline
\bottomrule
\end{longtable}

Verify with your C++ compiler.

7. Write down the output of this program or explain why it won't run:

\begin{longtable}[]{@{}l@{}}
\toprule
\endhead
\begin{minipage}[t]{0.97\columnwidth}\raggedright
\#include \textless iostream\textgreater{}

int main()

\{

int age;

std::cin \textgreater\textgreater{} age;

std::cout \textless\textless{} "age:" \textless\textless{} age
\textless\textless{} std::endl;

int age = age + 1;

std::cout \textless\textless{} "next year:" \textless\textless{} age
\textless\textless{} std::endl;

return 0;\\
\} \strut
\end{minipage}\tabularnewline
\bottomrule
\end{longtable}

Verify with your C++ compiler.

8. Write down the output of this program or explain why it won't run:

\begin{longtable}[]{@{}l@{}}
\toprule
\endhead
\begin{minipage}[t]{0.97\columnwidth}\raggedright
\#include \textless iostream\textgreater{}

int main()

\{

int 1st\_prize = 300000;

int 2nd\_prize = 200000;

int 3rd\_prize = 100000;

std::cout \textless\textless{} 1st\_prize \textless\textless{} std::endl

\textless\textless{} 2nd\_prize \textless\textless{} std::endl

\textless\textless{} 3rd\_prize \textless\textless{} std::endl

return 0;\\
\} \strut
\end{minipage}\tabularnewline
\bottomrule
\end{longtable}

Verify with your C++ compiler.

9. Write down the output of this program or explain why it won't run:

\begin{longtable}[]{@{}l@{}}
\toprule
\endhead
\begin{minipage}[t]{0.97\columnwidth}\raggedright
\#include \textless iostream\textgreater{}

int main()

\{

int a = 0, b = 1;

std::cout \textless\textless{} a \textless\textless{} ' '
\textless\textless{} b \textless\textless{} std::endl;\\

int t = a;

b = a;

a = t;

std::cout \textless\textless{} a \textless\textless{} ' '
\textless\textless{} b \textless\textless{} std::endl;

return 0;\\
\}\strut
\end{minipage}\tabularnewline
\bottomrule
\end{longtable}

Verify with your compiler.

10. What is the output of this program?

\begin{longtable}[]{@{}@{}}
\toprule
\endhead
\bottomrule
\end{longtable}

11. We have been initializing variables with constant integer values.
What if we initialize an integer variable with the value of an
expression? Can we do that? Circle one: YES NO. Now run this program
where we try to initialize \emph{k} with the value of an expression:

\begin{longtable}[]{@{}l@{}}
\toprule
\endhead
\begin{minipage}[t]{0.97\columnwidth}\raggedright
\#include \textless iostream\textgreater{}

int main()

\{

int i = 0, j = 1;

int k = i * j + j + 5;

return 0;

\}\strut
\end{minipage}\tabularnewline
\bottomrule
\end{longtable}

12. What is the output of this program?

\begin{longtable}[]{@{}@{}}
\toprule
\endhead
\bottomrule
\end{longtable}

13. Here's some math: The derivative of x\textsuperscript{n} is
nx\textsuperscript{n-1}. For instance the derivative of
x\textsuperscript{3} is 3x\textsuperscript{2}. You don't have to
understand the math behind it or what ithis ``derivative thingy'' is
good for. You just need to write a program that prompts for n and then
prints the derivative. Here's an execution of the program where the user
enters 3:

\begin{longtable}[]{@{}l@{}}
\toprule
\endhead
\begin{minipage}[t]{0.97\columnwidth}\raggedright
n = \textbf{3}

d 3 2

-\/- x = 3x

dx\strut
\end{minipage}\tabularnewline
\bottomrule
\end{longtable}

and if the user enters 7 for n you get:

\begin{longtable}[]{@{}l@{}}
\toprule
\endhead
\begin{minipage}[t]{0.97\columnwidth}\raggedright
n = \textbf{7}

d 7 6

-\/- x = 7x

dx\strut
\end{minipage}\tabularnewline
\bottomrule
\end{longtable}

And if the user enters 1324 you get

\begin{longtable}[]{@{}l@{}}
\toprule
\endhead
\begin{minipage}[t]{0.97\columnwidth}\raggedright
n = \textbf{1324}

d 1324 1323

-\/- x = 1324x

dx\strut
\end{minipage}\tabularnewline
\bottomrule
\end{longtable}

14. What's wrong with this program?

\begin{longtable}[]{@{}l@{}}
\toprule
\endhead
\begin{minipage}[t]{0.97\columnwidth}\raggedright
\#include \textless iostream\textgreater{}

int main()

\{

std::cout \textless\textless{} "triangle area calculator"

\textless\textless{} std::endl;

int base;

std::cout \textless\textless{} "base: ";

std::cin \textgreater\textgreater{} base;

std::cout \textless\textless{} "height: ";

std::cin \textgreater\textgreater{} height;

std::cout \textless\textless{} "area = "

\textless\textless{} 1/2 * base * height

\textless\textless{} std::cout;

return 0;\\
\}\strut
\end{minipage}\tabularnewline
\bottomrule
\end{longtable}

Now verify your correction with C++.

\end{document}
