% Options for packages loaded elsewhere
\PassOptionsToPackage{unicode}{hyperref}
\PassOptionsToPackage{hyphens}{url}
%
\documentclass[
]{article}
\usepackage{lmodern}
\usepackage{amssymb,amsmath}
\usepackage{ifxetex,ifluatex}
\ifnum 0\ifxetex 1\fi\ifluatex 1\fi=0 % if pdftex
  \usepackage[T1]{fontenc}
  \usepackage[utf8]{inputenc}
  \usepackage{textcomp} % provide euro and other symbols
\else % if luatex or xetex
  \usepackage{unicode-math}
  \defaultfontfeatures{Scale=MatchLowercase}
  \defaultfontfeatures[\rmfamily]{Ligatures=TeX,Scale=1}
\fi
% Use upquote if available, for straight quotes in verbatim environments
\IfFileExists{upquote.sty}{\usepackage{upquote}}{}
\IfFileExists{microtype.sty}{% use microtype if available
  \usepackage[]{microtype}
  \UseMicrotypeSet[protrusion]{basicmath} % disable protrusion for tt fonts
}{}
\makeatletter
\@ifundefined{KOMAClassName}{% if non-KOMA class
  \IfFileExists{parskip.sty}{%
    \usepackage{parskip}
  }{% else
    \setlength{\parindent}{0pt}
    \setlength{\parskip}{6pt plus 2pt minus 1pt}}
}{% if KOMA class
  \KOMAoptions{parskip=half}}
\makeatother
\usepackage{xcolor}
\IfFileExists{xurl.sty}{\usepackage{xurl}}{} % add URL line breaks if available
\IfFileExists{bookmark.sty}{\usepackage{bookmark}}{\usepackage{hyperref}}
\hypersetup{
  hidelinks,
  pdfcreator={LaTeX via pandoc}}
\urlstyle{same} % disable monospaced font for URLs
\usepackage{longtable,booktabs}
% Correct order of tables after \paragraph or \subparagraph
\usepackage{etoolbox}
\makeatletter
\patchcmd\longtable{\par}{\if@noskipsec\mbox{}\fi\par}{}{}
\makeatother
% Allow footnotes in longtable head/foot
\IfFileExists{footnotehyper.sty}{\usepackage{footnotehyper}}{\usepackage{footnote}}
\makesavenoteenv{longtable}
\setlength{\emergencystretch}{3em} % prevent overfull lines
\providecommand{\tightlist}{%
  \setlength{\itemsep}{0pt}\setlength{\parskip}{0pt}}
\setcounter{secnumdepth}{-\maxdimen} % remove section numbering

\author{}
\date{}

\begin{document}

25. Function Overloading

Objectives

\begin{itemize}
\tightlist
\item
  Understand function signatures
\item
  Write overloaded functions
\end{itemize}

Too many function names to remember!

This is an easy:

\begin{longtable}[]{@{}l@{}}
\toprule
\endhead
\begin{minipage}[t]{0.97\columnwidth}\raggedright
int max(int x, int y)

\{

if (x \textless{} y) return y;

else return x;\\
\}\strut
\end{minipage}\tabularnewline
\bottomrule
\end{longtable}

But what if I want to compute the max of two doubles? If I use the
above:

\begin{longtable}[]{@{}l@{}}
\toprule
\endhead
\begin{minipage}[t]{0.97\columnwidth}\raggedright
int max(int, int);

int main()

\{

std::cout \textless\textless{} max(3, 42) \textless\textless{} ' '

\textless\textless{} max(3.14, 2.718) \textless\textless{} std::endl;

return 0;

\}

\hfill\break
int max(int x, int y)

\{

if (x \textless{} y) return y;

else return x;\\
\}\strut
\end{minipage}\tabularnewline
\bottomrule
\end{longtable}

I will lose the fractional part of the true maximum (because when the
function is called, \emph{x} will have a value of 3 and y will have a
value of 2.)

In other words 3.14 is typecasted to an integer before it's given to
\emph{x}.

So I have to write another max function for doubles. To make sure I
don't have two functions with the same name I do this:

\begin{longtable}[]{@{}l@{}}
\toprule
\endhead
\begin{minipage}[t]{0.97\columnwidth}\raggedright
int max(int, int);

double max2(double, double);

int main()

\{

std::cout \textless\textless{} max(3, 42) \textless\textless{} ' '

\textless\textless{} max2(3.14, 2.718) \textless\textless{} std::endl;

return 0;

\}

\hfill\break
int max(int x, int y)

\{

if (x \textless{} y) return y;

else return x;\\
\}

double max2(double x, double y)

\{

if (x \textless{} y) return y;

else return x;\\
\}\strut
\end{minipage}\tabularnewline
\bottomrule
\end{longtable}

Now the nightmare begins ...

What if I want the max of an array of integers? An array of doubles?
Looks like I need:

\begin{longtable}[]{@{}l@{}}
\toprule
\endhead
\begin{minipage}[t]{0.97\columnwidth}\raggedright
int max(int, int);

double max2(double, double);

int max3(int{[}{]}, int len);

double max4(double{[}{]}, int len);

...\strut
\end{minipage}\tabularnewline
\bottomrule
\end{longtable}

YIKES!!! HELP ME MAMA!!!

MY HEAD IS GOING TO BLOW UP JUST REMEMBERING WHICH IS WHICH!!!

Function Overloading

This is from the previous section:

\begin{longtable}[]{@{}l@{}}
\toprule
\endhead
\begin{minipage}[t]{0.97\columnwidth}\raggedright
int max(int, int);

double max2(double, double);

int main()

\{

std::cout \textless\textless{} max(3, 42) \textless\textless{} ' '

\textless\textless{} max2(3.14, 2.718) \textless\textless{} std::endl;

return 0;\\
\}

\hfill\break
int max(int x, int y)

\{

if (x \textless{} y) return y;

else return x;\\
\}

double max2(double x, double y)

\{

if (x \textless{} y) return y;

else return x;\\
\}\strut
\end{minipage}\tabularnewline
\bottomrule
\end{longtable}

Make this change:

\begin{longtable}[]{@{}l@{}}
\toprule
\endhead
\begin{minipage}[t]{0.97\columnwidth}\raggedright
int max(int, int);

double max(double, double);

int main()

\{

std::cout \textless\textless{} max(3, 42) \textless\textless{} ' '

\textless\textless{} max(3.14, 2.718) \textless\textless{} std::endl;

return 0;\\
\}

\hfill\break
int max(int x, int y)

\{

if (x \textless{} y) return y;

else return x;\\
\}

double max(double x, double y)

\{

if (x \textless{} y) return y;

else return x;\\
\} \strut
\end{minipage}\tabularnewline
\bottomrule
\end{longtable}

Run it.

:O

How come C++ is not confused by two functions with the
\emph{\textbf{same name}}?

The reason is because C++ recognizes functions by their \textbf{names
}\emph{\textbf{and}}\textbf{ the types of the parameters}. Here are the
prototypes of the above function:

\begin{longtable}[]{@{}l@{}}
\toprule
\endhead
\begin{minipage}[t]{0.97\columnwidth}\raggedright
int max(int, int);

double max(double, double);\strut
\end{minipage}\tabularnewline
\bottomrule
\end{longtable}

You can think of it this way: the integer \emph{max()} function is
identified by \emph{(max, int, int)} and the double \emph{max()}
function is identified by \emph{(max, double, double)}.

You see what happens is that the actual function name of \emph{max(int,
int)} might be \emph{max\_int\_int()} and the name of \emph{max(double,
double) }might be \emph{max\_double\_double()}. I say ``might'' because
it depends on the compiler. For instance another compiler might use
\emph{max\_\_\_\_int\_\_\_\_int()} instead of \emph{max\_int\_int()}.
This change of function name is called name mangling.

The types of the parameters is called the \textbf{signature} of the
function. So the signature of

\begin{longtable}[]{@{}l@{}}
\toprule
\endhead
int max(int, int);\tabularnewline
\bottomrule
\end{longtable}

is \emph{(int, int)}.

The important thing to remember that a function is identified by its
\textbf{name} \emph{and} \textbf{signature}.

Of course now that a function is determined by its name and signature,
this tells you that you cannot have two different functions with the
same name and signature. Therefore this is BAD:

\begin{longtable}[]{@{}l@{}}
\toprule
\endhead
\begin{minipage}[t]{0.97\columnwidth}\raggedright
void bad(int x, double y, char z)

\{

std::cout \textless\textless{} "first bad ..." \textless\textless{}
'\textbackslash n';

...\\
\}

int bad(int x, double y, char z)

\{

std::cout \textless\textless{} "second bad ..." \textless\textless{}
'\textbackslash n';

...\\
\}\strut
\end{minipage}\tabularnewline
\bottomrule
\end{longtable}

Yes, yes, yes \ldots{} the return type is different \ldots{} but C++
only looks at the name and the signature when choosing a function to
execute.

So we revise the way C++ searches for identifiers (i.e. names):

\begin{itemize}
\tightlist
\item
  The search for variables is still the same
\item
  The search for functions when they are called is different.
\item
  If the function name can be found and the types of the arguments
  passed in matches the function signature (the type of the parameters),
  then that function is used.
\item
  Otherwise, C++ will try to typecast the arguments to values that will
  match a signature.
\end{itemize}

Here's one more example. Run it.

\begin{longtable}[]{@{}l@{}}
\toprule
\endhead
\begin{minipage}[t]{0.97\columnwidth}\raggedright
\#include \textless iostream\textgreater{}

void f(int, int, int);

void f(double, double, double);

int main()

\{

f(1, 2, 3);

f(1.1, 2.2, 3.3);

return 0;

\}

void f(int x, int y, int z)

\{

std::cout \textless\textless{} "f(int, int, int)" \textless\textless{}
std::endl;

\}

void f(double x, double y, double z)

\{

std::cout \textless\textless{} "f(double, double, double)"

\textless\textless{} std::endl;

\}\strut
\end{minipage}\tabularnewline
\bottomrule
\end{longtable}

When you have several functions with the same name (but different
signatures!), we say that the function is \textbf{overloaded}.

Exercise. Here are two function prototypes:

\begin{longtable}[]{@{}l@{}}
\toprule
\endhead
\begin{minipage}[t]{0.97\columnwidth}\raggedright
// computes the maximum value of the array passed in

int max(int{[}{]}, int len);

double max(double{[}{]}, int len);\strut
\end{minipage}\tabularnewline
\bottomrule
\end{longtable}

Implement the function bodies and test them in a \emph{main()}.

Exercise. Implement the following and test your code.

\begin{longtable}[]{@{}l@{}}
\toprule
\endhead
\begin{minipage}[t]{0.97\columnwidth}\raggedright
// Functions return the minimum of values passed in

// either through parameters or values in an array

int min(int, int);

double min(double, double);

int min(int{[}{]}, int len);

double min(double{[}{]}, int len);\strut
\end{minipage}\tabularnewline
\bottomrule
\end{longtable}

Exercise. Implement and test your functions for the following
prototypes:

\begin{longtable}[]{@{}l@{}}
\toprule
\endhead
\begin{minipage}[t]{0.97\columnwidth}\raggedright
// area functions

double area(double, double); // area of rectangle

// with given sides

double area(double); // area of circle with

// given radius\strut
\end{minipage}\tabularnewline
\bottomrule
\end{longtable}

Remember earlier in the semester when I said that there are two +
operators. Example:

1 + 2integer addition operator

1.2 + 3.4double addition operator

At that point I told you to remember this important distinction. This is
very similar to function overloading, except that these are operators
and not functions. We say that + is an \textbf{overloaded operator}. In
CISS245 you will learn to define your own operators so for instance you
can define your own operators including +.

WARNING: \emph{const int} and \emph{int}

Here's a warning: ``const int'' is the same type as ``int''. The
``const'' is about the nature of the value that a variable refers to,
i.e. for

const int x = 42;

the const simply means that x is a constant variable. The type of value
that x has is still an \emph{int}. Now why is this important in the
context of function overloading? Well \ldots{} the above implies that

\begin{longtable}[]{@{}l@{}}
\toprule
\endhead
\begin{minipage}[t]{0.97\columnwidth}\raggedright
void f(int x)

\{

\}

void f(const int x)

\{

\}

int main()

\{

return 0;

\}\strut
\end{minipage}\tabularnewline
\bottomrule
\end{longtable}

will not compile because the two functions \emph{f} above have the same
signature. Make sure you at least compile the above and read the error
message from the compiler so that when this error bites you, you'll know
why.

\textbf{Exercise.} What about references? Are \emph{int\&} and
\emph{int} considered the same too? In other words does the following
compile:

\begin{longtable}[]{@{}l@{}}
\toprule
\endhead
\begin{minipage}[t]{0.97\columnwidth}\raggedright
void f(int x)

\{

\}

void f(int \& x)

\{

\}

int main()

\{

return 0;

\}\strut
\end{minipage}\tabularnewline
\bottomrule
\end{longtable}

Why Overloading?

The reason why we have function (and operator) overloading is clear:
Many functions perform similar operations and we want to remember fewer
names.

This is also the case in Math. Not only is + used for addition of real
numbers, it's also used for things like addition of functions, addition
of vectors, addition of matrices, etc. So having function/operator
overloading in a programming language allows us to design software to
match the mathematical language/symbols that we are used to.

Even in daily use of the English language, we see the same word used
similar but different situations: open a door, open a jar, open my
wallet, open your mind, etc.

Overloading is a software technique for controlling software
complexities.

(Some older programming languages do not have function overloading. Some
don't even allow you to define your own operators. Also, there are
programming languages where overloading is not allowed. In fact there
are programming languages where the symbol for adding integers is
different from the symbol for adding doubles.)

Return type warning

I've already mentioned this \ldots{} but's it's such a common error that
it's a good idea to repeat it \ldots{}

\ldots{} the \textbf{return type is }\emph{\textbf{not}}\textbf{ part of
the identification of a function}. In other words the following is
\textbf{incorrect}:

\begin{longtable}[]{@{}l@{}}
\toprule
\endhead
\begin{minipage}[t]{0.97\columnwidth}\raggedright
int f(int); // identified as f, int

double f(int); // identified as f, int\strut
\end{minipage}\tabularnewline
\bottomrule
\end{longtable}

Why? Because of this ...

Recall that it's OK \emph{\textbf{not}} to use the return value of a
function. For instance this is a valid piece of code:

\begin{longtable}[]{@{}l@{}}
\toprule
\endhead
\begin{minipage}[t]{0.97\columnwidth}\raggedright
int f()

\{

return 42;

\}

int main()

\{

f(); // return value is not used

return 0;\\
\}\strut
\end{minipage}\tabularnewline
\bottomrule
\end{longtable}

In fact even this is OK:

\begin{longtable}[]{@{}l@{}}
\toprule
\endhead
\begin{minipage}[t]{0.97\columnwidth}\raggedright
int main()

\{

42;

return 0;\\
\}\strut
\end{minipage}\tabularnewline
\bottomrule
\end{longtable}

Now suppose you have a C++ compiler (called Crazy C++) that identifies
function by their names, signatures \textbf{and return types}. Then you
have this \textbf{problem}:

\begin{longtable}[]{@{}l@{}}
\toprule
\endhead
\begin{minipage}[t]{0.97\columnwidth}\raggedright
int f(int)

\{\}

double f(int) // should be different from above f()

\{\}

int main()

\{

f(42); // Crazy C++: ... which one to use???

// the return type is not indicated!!!

return 0;\\
\}\strut
\end{minipage}\tabularnewline
\bottomrule
\end{longtable}

Make sure you run this program.

Type Casting and Ambiguous Invocation

Recall that this is how C++ searches for functions (see above):

\begin{itemize}
\tightlist
\item
  If the function name can be found and the types of the arguments
  passed in matches the function signature (the type of the parameters),
  then that function is used.
\item
  Otherwise, C++ will try to typecast to something suitable.
\end{itemize}

Try this:

\begin{longtable}[]{@{}l@{}}
\toprule
\endhead
\begin{minipage}[t]{0.97\columnwidth}\raggedright
\#include \textless iostream\textgreater{}

void f(int, int, int);

int main()

\{

f(1, 2, 3);

f(4.4, 5.5, 6.6);

return 0;

\}

void f(int x, int y, int z)

\{

std::cout \textless\textless{} "f(int, int, int)" \textless\textless{}
std::endl;

\}\strut
\end{minipage}\tabularnewline
\bottomrule
\end{longtable}

Now this:

\begin{longtable}[]{@{}l@{}}
\toprule
\endhead
\begin{minipage}[t]{0.97\columnwidth}\raggedright
\#include \textless iostream\textgreater{}

void f(int, int, int);

void f(double, double, double);

int main()

\{

f(1, 2, 3);

f(1.1, 2.2, 3.3);

return 0;

\}

void f(int x, int y, int z)

\{

std::cout \textless\textless{} "f(int, int, int)" \textless\textless{}
std::endl;

\}

void f(double x, double y, double z)

\{

std::cout \textless\textless{} "f(double, double, double)"

\textless\textless{} std::endl;

\}\strut
\end{minipage}\tabularnewline
\bottomrule
\end{longtable}

So far so good \ldots{}

But now ...

\textbf{Exercise.} Add \emph{f(1, 2, 3.4);} to the main of the program
above and run it.

\begin{longtable}[]{@{}l@{}}
\toprule
\endhead
\begin{minipage}[t]{0.97\columnwidth}\raggedright
...

int main()

\{

f(1, 2, 3);

f(1.1, 2.2, 3.3);

f(1, 2, 3.4);

return 0;

\}

...\strut
\end{minipage}\tabularnewline
\bottomrule
\end{longtable}

Does it work?

There are two possibilities: C++ can call

\emph{void f(int, int, int)} type cast 3.4 to 3

\emph{void f(double, double, double) }type cast 1 to 1.0, 2 to 2.0

If there are two or more possible function calls after attempts to type
cast, C++ will choke and yell at you. This type of error is called
\textbf{ambiguous invocation}.

As mentioned before: Always be explicit in your code -\/- as much as
possible avoid implicit typecasts that are done behind your back.
Something like:

int x = 0;

double y = x;

should be written

int x = 0;

double y = double(x);

\textbf{Exercise.} Are there any ambiguous function calls? Find and
circle if any.

\begin{longtable}[]{@{}l@{}}
\toprule
\endhead
\begin{minipage}[t]{0.97\columnwidth}\raggedright
void f(int x, int y, int z)

\{\}

void f(int x)

\{\}

void f()

\{\}

void f(int x, int y, double z)

\{\}

int main()

\{

f(1);

f(1, 2, 3);

f(1.1, 3.3, 4.4);

return 0;\\
\}\strut
\end{minipage}\tabularnewline
\bottomrule
\end{longtable}

Verify with your C++.

\end{document}
