% Options for packages loaded elsewhere
\PassOptionsToPackage{unicode}{hyperref}
\PassOptionsToPackage{hyphens}{url}
%
\documentclass[
]{article}
\usepackage{lmodern}
\usepackage{amssymb,amsmath}
\usepackage{ifxetex,ifluatex}
\ifnum 0\ifxetex 1\fi\ifluatex 1\fi=0 % if pdftex
  \usepackage[T1]{fontenc}
  \usepackage[utf8]{inputenc}
  \usepackage{textcomp} % provide euro and other symbols
\else % if luatex or xetex
  \usepackage{unicode-math}
  \defaultfontfeatures{Scale=MatchLowercase}
  \defaultfontfeatures[\rmfamily]{Ligatures=TeX,Scale=1}
\fi
% Use upquote if available, for straight quotes in verbatim environments
\IfFileExists{upquote.sty}{\usepackage{upquote}}{}
\IfFileExists{microtype.sty}{% use microtype if available
  \usepackage[]{microtype}
  \UseMicrotypeSet[protrusion]{basicmath} % disable protrusion for tt fonts
}{}
\makeatletter
\@ifundefined{KOMAClassName}{% if non-KOMA class
  \IfFileExists{parskip.sty}{%
    \usepackage{parskip}
  }{% else
    \setlength{\parindent}{0pt}
    \setlength{\parskip}{6pt plus 2pt minus 1pt}}
}{% if KOMA class
  \KOMAoptions{parskip=half}}
\makeatother
\usepackage{xcolor}
\IfFileExists{xurl.sty}{\usepackage{xurl}}{} % add URL line breaks if available
\IfFileExists{bookmark.sty}{\usepackage{bookmark}}{\usepackage{hyperref}}
\hypersetup{
  hidelinks,
  pdfcreator={LaTeX via pandoc}}
\urlstyle{same} % disable monospaced font for URLs
\usepackage{longtable,booktabs}
% Correct order of tables after \paragraph or \subparagraph
\usepackage{etoolbox}
\makeatletter
\patchcmd\longtable{\par}{\if@noskipsec\mbox{}\fi\par}{}{}
\makeatother
% Allow footnotes in longtable head/foot
\IfFileExists{footnotehyper.sty}{\usepackage{footnotehyper}}{\usepackage{footnote}}
\makesavenoteenv{longtable}
\setlength{\emergencystretch}{3em} % prevent overfull lines
\providecommand{\tightlist}{%
  \setlength{\itemsep}{0pt}\setlength{\parskip}{0pt}}
\setcounter{secnumdepth}{-\maxdimen} % remove section numbering

\author{}
\date{}

\begin{document}

72. Class Templates

Objectives

\begin{itemize}
\tightlist
\item
  Write class templates
\item
  Instantiate classes from class templates
\end{itemize}

The problem

What if we want to have a class where the only instance variable is an
integer, and then we want another where the only instance variable is a
character?

They both have the same kind of ``behavior''. The constructor sets the
value of the instance variable and there are get and set methods. Here's
an example. Make sure you run it.

\begin{longtable}[]{@{}l@{}}
\toprule
\endhead
\begin{minipage}[t]{0.97\columnwidth}\raggedright
class Int

\{

public:

Int(int x0)

: x(x0)

\{\}

int get() const \{ return x; \}

void set(int x0) \{ x\_ = x0; \}

private:

int x;

\};

class Char

\{

public:

Char(char x0): x(x0) \{\}

char get() const \{ return x; \}

void set(char x0) \{ x = x0; \}

private:

char x;

\};\strut
\end{minipage}\tabularnewline
\bottomrule
\end{longtable}

The test code is:

\begin{longtable}[]{@{}l@{}}
\toprule
\endhead
\begin{minipage}[t]{0.97\columnwidth}\raggedright
int main()

\{

Int d(5);

Char e('a');

std::cout \textless\textless{} d.get() \textless\textless{} ' '

\textless\textless{} e.get() \textless\textless{} '\textbackslash n';

return 0;

\}\strut
\end{minipage}\tabularnewline
\bottomrule
\end{longtable}

Note that the two classes are very similar. In fact, within the class
the only difference is the type associated with the instance variable.

\begin{longtable}[]{@{}l@{}}
\toprule
\endhead
\begin{minipage}[t]{0.97\columnwidth}\raggedright
class Int

\{

public:

Int(\textbf{int} x0)

: x(x0)

\{\}

\textbf{int} get() const \{ return x; \}

void set(\textbf{int} x0) \{ x\_ = x0; \}

private:

\textbf{int} x;

\};

class Char

\{

public:

Char(\textbf{char} x0): x(x0) \{\}

\textbf{char} get() const \{ return x; \}

void set(\textbf{char} x0) \{ x = x0; \}

private:

\textbf{char} x;

\};\strut
\end{minipage}\tabularnewline
\bottomrule
\end{longtable}

What a pain to type two chunks of code which are almost the same. It's
like typing unrolled for-loops!

Class templates

Now we allow types to vary.

Generic Programming refers to the style of programming using parameters
for types.

\begin{longtable}[]{@{}l@{}}
\toprule
\endhead
\begin{minipage}[t]{0.97\columnwidth}\raggedright
template\textless{} typename T \textgreater{}

class D \{

public:

D()\{\}

D(T c) : x(c) \{\}

T get() const \{ return x; \}

void set(const T \& c) \{ x = c; \}

private:

T x;

\};

int main()

\{

\textbf{ D\textless{} int \textgreater{}} d(5);

\textbf{ D\textless{} char \textgreater{}} e('a');

std::cout \textless\textless{} d.get() \textless\textless{} ","
\textless\textless{} e.get() \textless\textless{} "\textbackslash n";

return 0;

\}\strut
\end{minipage}\tabularnewline
\bottomrule
\end{longtable}

A class is generated from a class template by the compiler when you
specify the parameters.

Example:

D\textless{} int \textgreater{} d(5);

If \emph{D\textless{} char \textgreater{}} does not appear, then the
class \emph{D\textless{} char \textgreater{}} is not generated by the
compiler. Here are some terms:

\begin{itemize}
\tightlist
\item
  \textbf{Template instantiation} = generating class from class template
  and template arguments.
\end{itemize}

\begin{itemize}
\tightlist
\item
  \textbf{Specialization} = template with parameters specified
\end{itemize}

\begin{itemize}
\tightlist
\item
  \textbf{Template members} = members of a class template.
\end{itemize}

Template members are parameterized by the templates. So if a template
member is defined outside the class, then the class template must be
specified

WARNING: Everything must be placed in the header file!!! No .cpp for
class template!!!

\begin{longtable}[]{@{}l@{}}
\toprule
\endhead
\begin{minipage}[t]{0.97\columnwidth}\raggedright
\#include \textless iostream\textgreater{}

template\textless typename T\textgreater{}

class D \{

public:

D()\{\}

D(T c) : x(c) \{\}

T get() const;

void set(const T \&);

private:

T x;

\};

template\textless class T\textgreater{}

T D\textless{} T \textgreater::get() const

\{

return x;

\}

template\textless typename T\textgreater{}

void D\textless{} T \textgreater::set(const T \& c)

\{

x = c;

\}

int main()

\{

D\textless{} int \textgreater{} d(5);

D\textless{} char \textgreater{} e('a');

std::cout \textless\textless{} d.get() \textless\textless{} ","
\textless\textless{} e.get() \textless\textless{} "\textbackslash n";

return 0;

\}\strut
\end{minipage}\tabularnewline
\bottomrule
\end{longtable}

Note that

\begin{longtable}[]{@{}l@{}}
\toprule
\endhead
\begin{minipage}[t]{0.97\columnwidth}\raggedright
template\textless typename T\textgreater{}

T D\textless{} T \textgreater::get\textless{} T \textgreater() const

\{

return x;

\}\strut
\end{minipage}\tabularnewline
\bottomrule
\end{longtable}

is the same as:

\begin{longtable}[]{@{}l@{}}
\toprule
\endhead
\begin{minipage}[t]{0.97\columnwidth}\raggedright
template\textless typename T\textgreater{}

T D\textless{} T \textgreater::get() const

\{

return x;

\}\strut
\end{minipage}\tabularnewline
\bottomrule
\end{longtable}

i.e., within the scope of \emph{D\textless{} T \textgreater{}}, T is
already known.

Constants for template parameters

There are three things you can specify as template parameters. We'll
talk about only two. The template parameters can be for

\begin{itemize}
\tightlist
\item
  types
\end{itemize}

or

\begin{itemize}
\tightlist
\item
  basic type values
\end{itemize}

In the above examples, the template parameters are for types:

\begin{longtable}[]{@{}l@{}}
\toprule
\endhead
\begin{minipage}[t]{0.97\columnwidth}\raggedright
template \textless{} typename T \textgreater{}

class X

\{

...

private:

T ...

...

\};\strut
\end{minipage}\tabularnewline
\bottomrule
\end{longtable}

Here's an example where the template parameter is a value. Run this and
study it very carefully.

\begin{longtable}[]{@{}l@{}}
\toprule
\endhead
\begin{minipage}[t]{0.97\columnwidth}\raggedright
\#include \textless iostream\textgreater{}

template \textless typename T, \textbf{int} MAX\textgreater{}

class D

\{

public:

T get(int) const;

void set(int, const T \&);

private:

T x{[}MAX{]};

\};

template \textless typename T, int MAX\textgreater{}

T D\textless{} T, MAX \textgreater::get(int index) const

\{

return x{[}index{]};

\}

template \textless typename T, int MAX\textgreater{}

void D\textless{} T, MAX \textgreater::set(int index, const T \& c)

\{

x{[}index{]} = c;

\};

int main()

\{

D\textless{} int, \textbf{10} \textgreater{} d; // d is an int array of
size 10

D\textless char, \textbf{5}\textgreater{} e; // d is a char array of
size 5

d.set(1,5);

e.set(2,'a');

std::cout \textless\textless{} d.get(1) \textless\textless{} ","
\textless\textless{} e.get(2) \textless\textless{} "\textbackslash n";

return 0;

\}\strut
\end{minipage}\tabularnewline
\bottomrule
\end{longtable}

D is a class for fixed size arrays of different types. One of the
template parameter is for the type of the array and the other template
parameter is for the size of the array.

When instantiating the value must be a \textbf{constant expression.}

Run this:

\begin{longtable}[]{@{}l@{}}
\toprule
\endhead
\begin{minipage}[t]{0.97\columnwidth}\raggedright
const int x = 4;

int y = 5;

D\textless int, 5\textgreater{} d1;

D\textless int, x - 2\textgreater{} d2;

D\textless int, y\textgreater{} d3; // \textbf{WRONG!!! y is a
variable.}\strut
\end{minipage}\tabularnewline
\bottomrule
\end{longtable}

Type parameters are:

\begin{itemize}
\tightlist
\item
  Any type including basic types (int, bool) or classes.
\end{itemize}

If D is a class template

\begin{longtable}[]{@{}l@{}}
\toprule
\endhead
\begin{minipage}[t]{0.97\columnwidth}\raggedright
template\textless{} typename T1, typename T2,

int x, bool b \textgreater{}

class D\{\ldots\}\strut
\end{minipage}\tabularnewline
\bottomrule
\end{longtable}

Then

\begin{longtable}[]{@{}l@{}}
\toprule
\endhead
D\textless{} X, Y, 10, true \textgreater{}\tabularnewline
\bottomrule
\end{longtable}

and

\begin{longtable}[]{@{}l@{}}
\toprule
\endhead
\begin{minipage}[t]{0.97\columnwidth}\raggedright
const int x = 12;

D\textless{} X, Y, x - 2, !false \textgreater{}\strut
\end{minipage}\tabularnewline
\bottomrule
\end{longtable}

are considered the same type (if \emph{X} and \emph{Y} are valid types).

Recall that typedef is only an alias and does not create a new type. So
if

\begin{longtable}[]{@{}l@{}}
\toprule
\endhead
\begin{minipage}[t]{0.97\columnwidth}\raggedright
template\textless class X\textgreater{}

class D\{...\};

typedef int Salary;

int main()

\{

D\textless{} int \textgreater{} x;

D\textless{} Salary \textgreater{} y;

\}\strut
\end{minipage}\tabularnewline
\bottomrule
\end{longtable}

then, \emph{D\textless{} int \textgreater{}}, \emph{D\textless{} Salary
\textgreater{}} are of the same type and \emph{x}, \emph{y} have the
same type.

Error checking can be done with two different levels: syntax errors in
template and when the template parameters are being used.

In the definition of the template and class template, the template
parameters are not specified yet. So the definition of

\emph{Array\textless{} T \textgreater::print} is OK

\emph{Array\textless{} int \textgreater::print} is OK

\begin{longtable}[]{@{}l@{}}
\toprule
\endhead
\begin{minipage}[t]{0.97\columnwidth}\raggedright
class vec2d

\{

public:

vec2d(float x0, float y0)

: x(x0),y(y0)

\{\}

private:

float x, y;

\};

template\textless{} typename T \textgreater{}

class Array \{

public:

Array(int s)

: size(s), arr(new T{[}s{]})

\{\}

void print() const;

private:

int size; T * arr;

\};

template\textless{} typename T \textgreater{}

void Array\textless T\textgreater::print()

\{

for (int i=0; i\textless size; i++)

std::cout \textless\textless{} arr{[}i{]} \textless\textless{}
'\textbackslash n';

\}

int main() \{

Array\textless int\textgreater{} a(10); a.print();

\textbf{ Array\textless{} vec2d \textgreater{} b(10); }

b.print();

return 0;

\}\strut
\end{minipage}\tabularnewline
\bottomrule
\end{longtable}

You can specify default values for template parameters.

The rules for default parameters are like those for default values for
function parameters.

\begin{longtable}[]{@{}l@{}}
\toprule
\endhead
\begin{minipage}[t]{0.97\columnwidth}\raggedright
\#include \textless iostream\textgreater{}

class C\{\};

template\textless{} typename T0, \textbf{typename T1=int} \textgreater{}

class D\{

public: void f(T1 a) \{ std::cout \textless\textless{} a
\textless\textless{} "\textbackslash n"; \}

\};

int main()

\{

\textbf{D\textless{} C \textgreater{}} obj1; obj1.f(1.1);

\textbf{D\textless{} C, double \textgreater{}} obj2; obj2.f(1.1);

return 0;

\}\strut
\end{minipage}\tabularnewline
\bottomrule
\end{longtable}

\begin{longtable}[]{@{}l@{}}
\toprule
\endhead
\begin{minipage}[t]{0.97\columnwidth}\raggedright
\#include \textless iostream\textgreater{}

class C\{\};

template \textless typename T0, typename T1=int\textgreater{}

class D

\{

public: void f(T1);

\};

template \textless{} typename T0, typename T1 \textgreater{}

void D\textless T0, T1\textgreater::f(T1 a)

\{

std::cout \textless\textless{} a \textless\textless{}
``\textbackslash n'';

\}

int main()

\{

D\textless C\textgreater{} obj1;

obj1.f(1.1);

D\textless C, double\textgreater{} obj2;

obj2.f(1.1);

\}\strut
\end{minipage}\tabularnewline
\bottomrule
\end{longtable}

When programming a class template, try it out for a specific type first.
After the code works, change to a generic type and include template.

WARNING: The following will give you an error.

\begin{longtable}[]{@{}l@{}}
\toprule
\endhead
\begin{minipage}[t]{0.97\columnwidth}\raggedright
template \textless typename T\textgreater{} class C\{...\};

template \textless typename T\textgreater{} class D\{...\};

int main()

\{

C\textless D\textless int\textgreater\textgreater{} c;

return 0;

\}\strut
\end{minipage}\tabularnewline
\bottomrule
\end{longtable}

Why? Because \emph{\textgreater\textgreater{}} is an operator and for
this code:

C\textless D\textless int\textgreater\textgreater{} c;

you compiler will think you are trying ot do this:

...int \textgreater\textgreater{} c...

as if you're doing some input!!!

So write

C\textless{} D\textless{} int \textgreater{} \textgreater{} c;

to prevent the above. In general when working with templates, always
have a space on the right of \textless{} and a space to the left of
\textgreater.

Function and struct templates

So far we've talked about class templates which are templates for
creating classes. But you can also have function templates and struct
templates. Struct templates are very similar to class templates. Here's
an example of a function template:

Run this and study it carefully:

\begin{longtable}[]{@{}l@{}}
\toprule
\endhead
\begin{minipage}[t]{0.97\columnwidth}\raggedright
\#include \textless iostream\textgreater{}

template \textless{} typename T \textgreater{}

T max(T x, T y)

\{

return (x \textgreater= y ? x : y);\\
\}

int main()

\{

std::cout \textless\textless{} max\textless{} int \textgreater(5, 2)
\textless\textless{} std::endl;

std::cout \textless\textless{} max\textless{} double \textgreater(1.2,
3.4) \textless\textless{} std::endl;

return 0;\\
\}\strut
\end{minipage}\tabularnewline
\bottomrule
\end{longtable}

For the case of function templates, usually the compiler can figure out
what types are intended for the type parameter. So you can run the above
test cases with this:

\begin{longtable}[]{@{}l@{}}
\toprule
\endhead
\begin{minipage}[t]{0.97\columnwidth}\raggedright
...

int main()

\{

std::cout \textless\textless{} max(5, 2) \textless\textless{} std::endl;

std::cout \textless\textless{} max(1.2, 3.4) \textless\textless{}
std::endl;

return 0;\\
\}\strut
\end{minipage}\tabularnewline
\bottomrule
\end{longtable}

i.e., without the \emph{\textless{} int \textgreater{}} and the
\emph{\textless{} double \textgreater{}}.

Like class templates, if you want to write a function template in a
separate file, then the whole function template must be a header file.

\textbf{Exercise.} Write a min function template. Test it.

Here's another example for you to study. Add test code in your
\emph{main()} function and run it. Study it carefully. Note that in this
case, the type parameter \emph{T} appears in the body of the function
template.

\begin{longtable}[]{@{}l@{}}
\toprule
\endhead
\begin{minipage}[t]{0.97\columnwidth}\raggedright
template \textless{} typename T \textgreater{}

T swap(T \& x, T \& y)

\{

\textbf{T} t = x;

x = y;

y = t;

\}\strut
\end{minipage}\tabularnewline
\bottomrule
\end{longtable}

\textbf{Exercise.} Write a print function template to print an array of
type \emph{T} values. The function should receive an array of \emph{T}
values and the number of things in the array to print starting at index
0:

\begin{longtable}[]{@{}l@{}}
\toprule
\endhead
\begin{minipage}[t]{0.97\columnwidth}\raggedright
template \textless{} typename T \textgreater{}

void print(T x{[}{]}, int n)

\{

...

\}\strut
\end{minipage}\tabularnewline
\bottomrule
\end{longtable}

\textbf{Exercise. }Write a linearsearch function template:

\begin{longtable}[]{@{}l@{}}
\toprule
\endhead
\begin{minipage}[t]{0.97\columnwidth}\raggedright
template \textless{} typename T \textgreater{}

int linearsearch(T x{[}{]}, int n, const T \& target)

\{

\ldots{}

\}\strut
\end{minipage}\tabularnewline
\bottomrule
\end{longtable}

Test it! Make sure you test it with an array of integer values, double
values, and char values.

\textbf{Exercise.} Write a bubblesort function template.

\begin{longtable}[]{@{}l@{}}
\toprule
\endhead
\begin{minipage}[t]{0.97\columnwidth}\raggedright
template \textless{} typename T \textgreater{}

void bubblesort(T a{[}{]}, int size);\strut
\end{minipage}\tabularnewline
\bottomrule
\end{longtable}

Test it by performing bubble sort on an array of integers and an array
of doubles. Is it possible to write this version?

\begin{longtable}[]{@{}l@{}}
\toprule
\endhead
\begin{minipage}[t]{0.97\columnwidth}\raggedright
template \textless{} typename T \textgreater{}

void bubblesort(T * start, T * end);\strut
\end{minipage}\tabularnewline
\bottomrule
\end{longtable}

\textbf{Exercise.} Write a binarysearch function template:

\begin{longtable}[]{@{}l@{}}
\toprule
\endhead
\begin{minipage}[t]{0.97\columnwidth}\raggedright
template \textless{} typename T \textgreater{}

int binarysearch(T a{[}{]}, int size, const T \& target);\strut
\end{minipage}\tabularnewline
\bottomrule
\end{longtable}

Test it! What about this:

\begin{longtable}[]{@{}l@{}}
\toprule
\endhead
\begin{minipage}[t]{0.97\columnwidth}\raggedright
template \textless{} typename T \textgreater{}

T * binarysearch(T * start, T * end, T * target);\strut
\end{minipage}\tabularnewline
\bottomrule
\end{longtable}

(I have a set of notes just on function templates and just on struct
templates. Study them both when you have time.)

Typename and class

Instead of

\begin{longtable}[]{@{}l@{}}
\toprule
\endhead
\begin{minipage}[t]{0.97\columnwidth}\raggedright
template \textless{} \textbf{typename} X \textgreater{}

class C\{\};\strut
\end{minipage}\tabularnewline
\bottomrule
\end{longtable}

You can say

\begin{longtable}[]{@{}l@{}}
\toprule
\endhead
\begin{minipage}[t]{0.97\columnwidth}\raggedright
template \textless{} \textbf{class} X \textgreater{}

class C\{\};\strut
\end{minipage}\tabularnewline
\bottomrule
\end{longtable}

They mean the same thing. However, \emph{typename} is better.

Template specialization

Suppose you have this \emph{\textbf{Array}} class template:

\begin{longtable}[]{@{}l@{}}
\toprule
\endhead
\begin{minipage}[t]{0.97\columnwidth}\raggedright
\#include \textless iostream\textgreater{}

template\textless{} typename T, int size \textgreater{}

class Array

\{

public:

Array(T x{[}{]})

\{

for(int i = 0; i \textless{} size; ++i)

x\_{[}i{]} = x{[}i{]};

\}

int sum() const

\{

int s = 0;

for (int i = 0; i \textless{} size; ++i)

s += x{[}i{]};

return s;

\}

private:

T x\_{[}size{]};

\};

int main()

\{

int x{[}{]} = \{2, 3, 5, 7, 11, 13, 17, 19\};

Array\textless{} int, 4 \textgreater{} a(x);

std::cout \textless\textless{} a.sum() \textless\textless{}
'\textbackslash n';

Array\textless{} int, 3 \textgreater{} b(x);

std::cout \textless\textless{} b.sum() \textless\textless{}
'\textbackslash n';\\
\}\strut
\end{minipage}\tabularnewline
\bottomrule
\end{longtable}

Run it.

Then you realize that in your software that you are building, one
extremely common usage of your \emph{\textbf{Array}} library is for the
case of modeling \emph{\textbf{int}}\textbf{ arrays of size 3}.

A loop over an array of size 3 to compute the sum is definitely slower
that unrolling the loop. The unrolled loop is going to be about twice as
fast.

Hmmm \ldots{} can we \textbf{speedup this case}?

Yes you can \ldots{} you can write a version of \emph{\textbf{Array}}
just for the case when T = \emph{\textbf{int}} and size = 3. This is
called a \textbf{template specialization}.

Run this:

\begin{longtable}[]{@{}l@{}}
\toprule
\endhead
\begin{minipage}[t]{0.97\columnwidth}\raggedright
\#include \textless iostream\textgreater{}

template\textless{} typename T, int size \textgreater{}

class Array

\{

// ... same as before ...

\};

template\textless\textgreater{}

class Array\textless{} int, 3 \textgreater{}

\{

public:

Array(int x{[}{]})

\{

std::cout \textless\textless{} "specialization
speedup!!!\textbackslash n";

x\_ = x{[}0{]}; y\_ = x{[}1{]}; z\_ = x{[}2{]};

\}

int sum() const

\{

std::cout \textless\textless{} "specialization
speedup!!!\textbackslash n";

return x\_ + y\_ + z\_;

\}

private:

int x\_, y\_, z\_;

\};

int main()

\{

// ... same as before ...\\
\}\strut
\end{minipage}\tabularnewline
\bottomrule
\end{longtable}

Template specialization is a very important idea and is used quite
frequently.

\textbf{Exercise.} Add \emph{\textbf{operator{[}{]}}} to both classes.
Test your code.

\textbf{Exercise.} Then you realize that \ldots{} wait a minute \ldots{}
this specialization idea is also helpful for a \emph{\textbf{double}}
array of size 3 and a \emph{\textbf{float}} array of size 3. Hmmm
\ldots{} can we partially specialize? In other words can we have
\emph{\textbf{size}} = 3 but retain \emph{\textbf{T}} as a template
parameter Try to figure it out \emph{\textbf{on your own}} before
turning the page for the answer!!! You have 20 minutes ...

\textbf{Exercise.} Write a \emph{\textbf{Matrix}} class template. A
matrix is just a 2D array. Here's a sample run:

\begin{longtable}[]{@{}l@{}}
\toprule
\endhead
\begin{minipage}[t]{0.97\columnwidth}\raggedright
\#include \textless iostream\textgreater{}

\#include "Matrix.h"

int main()

\{

int x{[}{]} = \{0, 1, 2, 3, 4, 5, 6, 7, 8, 9\};

Matrix\textless{} int, 2, 3 \textgreater{} m(x); // m is 2-by-3.

std::cout \textless\textless{} m \textless\textless{} '\textbackslash n'
// prints

// 0 1 2

// 3 4 5

Matrix\textless{} int, 2, 1 \textgreater{} n(x); // n is 2-by-1

std::cout \textless\textless{} n \textless\textless{} '\textbackslash n'
// prints

// 0

// 1

return 0;\\
\}\strut
\end{minipage}\tabularnewline
\bottomrule
\end{longtable}

That 2-by-1 matrix is sometimes called a ``2-dimensional column
vector''. Column vectors are used frequently. Furthermore processing a
regular 1-dimensional array of size 2 is faster than viewing it as a
2-by-1 2D array. So \ldots{} speedup your \emph{\textbf{Matrix}} library
by providing a specialization for the column vector case of a
\emph{\textbf{Matrix}} object where you use a 1-dimensional array in
this case The following should then work:

\begin{longtable}[]{@{}l@{}}
\toprule
\endhead
\begin{minipage}[t]{0.97\columnwidth}\raggedright
\#include \textless iostream\textgreater{}

\#include "Matrix.h"

int main()

\{

int x{[}{]} = \{0, 1, 2, 3, 4, 5, 6, 7, 8, 9\};

Matrix\textless{} int, 2, 3 \textgreater{} m(x); // m is 2-by-3.

std::cout \textless\textless{} m \textless\textless{} '\textbackslash n'
// prints

// 0 1 2

// 3 4 5

\textbf{Matrix\textless{} int, 2 \textgreater{}} n(x); // n is 2-by-1

std::cout \textless\textless{} n \textless\textless{} '\textbackslash n'
// prints

// 0

// 1

return 0;\\
\}\strut
\end{minipage}\tabularnewline
\bottomrule
\end{longtable}

Such a template library is very common in computer graphics and computer
vision. Such a library usually provide a typedef for
\emph{\textbf{Matrix\textless{} double, 4, 4 \textgreater{}}},
\emph{\textbf{Matrix\textless{} int, 2 \textgreater{}}}, etc.. For
instance there might be a typedef \emph{\textbf{mat4x4}} for
\emph{\textbf{Matrix\textless{} double, 4, 4 \textgreater{}}}\textbf{,
}\emph{\textbf{mat}\textbf{f}\textbf{4x4}} for
\emph{\textbf{Matrix\textless{} }\textbf{float}\textbf{, 4, 4
\textgreater{}}}, and \emph{\textbf{vec4f}} for
\emph{\textbf{Matrix\textless{} }\textbf{float}\textbf{, 4
\textgreater.}}

WARNING \ldots{} INCOMING SPOILER \ldots{}

Here's the (obvious?) answer to the question on the previous page:

\begin{longtable}[]{@{}l@{}}
\toprule
\endhead
\begin{minipage}[t]{0.97\columnwidth}\raggedright
template\textless{} typename T, int size \textgreater{}

class Array

\{

public:

Array(T x{[}{]})

\{

for(int i = 0; i \textless{} size; ++i)

x\_{[}i{]} = x{[}i{]};

\}

int sum() const

\{

int s = 0;

for (int i = 0; i \textless{} size; ++i)

s += x\_{[}i{]};

return s;

\}

private:

T x\_{[}size{]};

\};

template\textless{} typename T \textgreater{}

class Array\textless{} T, 3 \textgreater{}

\{

public:

Array(T x{[}{]})

\{

std::cout \textless\textless{} "specialization
speedup!!!\textbackslash n";

x\_ = x{[}0{]}; y\_ = x{[}1{]}; z\_ = x{[}2{]};

\}

int sum() const

\{

std::cout \textless\textless{} "specialization
speedup!!!\textbackslash n";

return x\_ + y\_ + z\_;

\}

private:

T x\_, y\_, z\_;

\};\strut
\end{minipage}\tabularnewline
\bottomrule
\end{longtable}

\end{document}
