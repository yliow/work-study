% Options for packages loaded elsewhere
\PassOptionsToPackage{unicode}{hyperref}
\PassOptionsToPackage{hyphens}{url}
%
\documentclass[
]{article}
\usepackage{lmodern}
\usepackage{amssymb,amsmath}
\usepackage{ifxetex,ifluatex}
\ifnum 0\ifxetex 1\fi\ifluatex 1\fi=0 % if pdftex
  \usepackage[T1]{fontenc}
  \usepackage[utf8]{inputenc}
  \usepackage{textcomp} % provide euro and other symbols
\else % if luatex or xetex
  \usepackage{unicode-math}
  \defaultfontfeatures{Scale=MatchLowercase}
  \defaultfontfeatures[\rmfamily]{Ligatures=TeX,Scale=1}
\fi
% Use upquote if available, for straight quotes in verbatim environments
\IfFileExists{upquote.sty}{\usepackage{upquote}}{}
\IfFileExists{microtype.sty}{% use microtype if available
  \usepackage[]{microtype}
  \UseMicrotypeSet[protrusion]{basicmath} % disable protrusion for tt fonts
}{}
\makeatletter
\@ifundefined{KOMAClassName}{% if non-KOMA class
  \IfFileExists{parskip.sty}{%
    \usepackage{parskip}
  }{% else
    \setlength{\parindent}{0pt}
    \setlength{\parskip}{6pt plus 2pt minus 1pt}}
}{% if KOMA class
  \KOMAoptions{parskip=half}}
\makeatother
\usepackage{xcolor}
\IfFileExists{xurl.sty}{\usepackage{xurl}}{} % add URL line breaks if available
\IfFileExists{bookmark.sty}{\usepackage{bookmark}}{\usepackage{hyperref}}
\hypersetup{
  hidelinks,
  pdfcreator={LaTeX via pandoc}}
\urlstyle{same} % disable monospaced font for URLs
\usepackage{longtable,booktabs}
% Correct order of tables after \paragraph or \subparagraph
\usepackage{etoolbox}
\makeatletter
\patchcmd\longtable{\par}{\if@noskipsec\mbox{}\fi\par}{}{}
\makeatother
% Allow footnotes in longtable head/foot
\IfFileExists{footnotehyper.sty}{\usepackage{footnotehyper}}{\usepackage{footnote}}
\makesavenoteenv{longtable}
\setlength{\emergencystretch}{3em} % prevent overfull lines
\providecommand{\tightlist}{%
  \setlength{\itemsep}{0pt}\setlength{\parskip}{0pt}}
\setcounter{secnumdepth}{-\maxdimen} % remove section numbering

\author{}
\date{}

\begin{document}

Objectives

\begin{itemize}
\tightlist
\item
  Pointer arithmetic
\item
  Understand the relationship between pointers and arrays
\item
  Allocate and deallocate memory for dynamic arrays in the heap
\item
  Understand array parameters in functions
\item
  Implement stacks and queues using dynamic arrays
\item
  Understand and implement directed graph using adjacency matrix
\item
  Model social media graphs
\end{itemize}

In this set of notes, we see how pointers can be used to create dynamic
arrays, i.e., pointers that point to arrays of with variable sizes.

Pointers are not arrays. But rather they have the address of the first
value in an array. However in terms of syntax and usage, pointers do
seem like arrays. I'll talk about the some of the similarities and
differences between arrays and pointers. But the point is, once p points
to an array, then the i-th index value of the array that p points to is
p{[}i{]}, which is exactly the array notation that you're already used
to.

An array is like a container of values. We store information in these
containers. (At this point you have only seen array of integers, of
doubles, etc. Later on you will learn about complex values, values such
as products in a company where each product contains product name,
price, UPC code, etc.) The problem is that the arrays you have seen up
to this point have fixed sizes. But no one can predict how large a
container should be. You might think an array of size 10000 is enough.
But what if after some time, you actually need 100000? Think for
instance of a game. At the beginning of the game, maybe there are only
10 enemy spaceships and at most 100 lasers are flying around in the
screen. But at a later level, maybe there are 1000 enemy spaceships and
1000000 lasers. Is that enough? Maybe not for an even later level.

Dynamic arrays allows you to handle the situations like the above and
they are therefore extremely important. There are actually many, many,
many types of containers: fixed size arrays, dynamic arrays, and then in
CISS350, I'll show you linked lists, trees, hashtables, etc.

Pointers also allow you to create complex 2D arrays which are not
rectangles. For instance if you want to create a 2D array where the
first row has 1000000 values, the second has 5000, the third has 20000,
and the fourth has 20, it can be done. Without pointers, you would have
to create a 2D array with 4 rows and each with 1000000 values.

Pointer arithmetic

Try this:

\begin{longtable}[]{@{}l@{}}
\toprule
\endhead
\begin{minipage}[t]{0.97\columnwidth}\raggedright
int x = 42;

int * p = \&x;

int * q = p + 1;

std::cout \textless\textless{} (unsigned int)(p) \textless\textless{} '
'

\textless\textless{} (unsigned int)(q) \textless\textless{}
'\textbackslash n'; \strut
\end{minipage}\tabularnewline
\bottomrule
\end{longtable}

Aha! You can add integers to pointers. Well this is not exactly shocking
since addresses are integers.

Note the address value stored in \emph{q} ... numerically speaking the
value of \emph{q }is actully the numerical value of \emph{p} plus 4.
when you add 1 to the address of an \emph{int} address -- IT'S NOT
ONE!!! Why 4? Because \ldots{} \textbf{4 comes from
}\emph{\textbf{sizeof(int)}}.

But why does it work this way? Because this allows you to think like
this:

Of course p + 2 has address value that is 8 away from 8. So if p has

address value 124500, then you should have this picture in your mind:

Now you might be wondering: ``Wait \ldots{} \emph{p} points to \emph{x}
and I do see that the value of \emph{x} is 42. But what is the 3467 that
\emph{p + 1} points to?''.

\emph{p + 1} simply points to an integer (because \emph{p} is a pointer
to \emph{int}), i.e., the four bytes starting at \emph{p + 1} as an
integer just happens to be 3467. It does not belong to any variable.
It's just part of your computer's memory. And of course it could be
garbage and the value would be different next time you run the same
program.

The above applies to any type and not just \emph{int}. In general
suppose \emph{T} is a type (\emph{int} or a \emph{char} or a
\emph{double} or a \emph{bool} or what-have-you), and you have the
following:

T * p = \&x;

and suppose a value of type \emph{T} uses k bytes (which can be obtained
using \emph{sizeof(T)}), then numerically the address value of \emph{p +
1} is always the address value of \emph{p} plus \emph{k}. Also, if we
have

T * p = \&x;

T * q = p - 1;

then \emph{q} is k smaller than \emph{p}.

\textbf{Remember ...}

\begin{itemize}
\tightlist
\item
  if \emph{p} is an \emph{int }pointer (and if you're using a 32-bit
  machine), then \emph{p + 1 }does \textbf{not} mean ``next address
  value after \emph{p}''. Rather, it means address of the next\textbf{
  integer} after the \textbf{integer} \emph{p} points to.
\item
  If \emph{q} is a pointer that points to a \emph{double}, then \emph{q
  + 1} is the address of the next\textbf{ double} after the
  \textbf{double} \emph{q} points to.
\item
  If \emph{r} is a pointer that points to a character, then \emph{r + 1}
  is address of the next\textbf{ character} after the \textbf{character}
  \emph{r} points to.
\end{itemize}

Get it?

By the way:

sizeof(double) is 8 (a double value takes up 8 bytes)

sizeof(char)is 1 (a char value takes up 1 byte)

However the \emph{sizeof(int)} can vary depending on your machine.

If you want to measure the difference between two pointers, you have to
type cast the address values not to unsigned int, but to int so that you
can get negative values. Here's an example:

\begin{longtable}[]{@{}l@{}}
\toprule
\endhead
\begin{minipage}[t]{0.97\columnwidth}\raggedright
int x = 42;

int * p = \&x;

int * q = p + 1;

int * r = p -- 1;

std::cout \textless\textless{} (int) q -- (int) p \textless\textless{} '
'

\textless\textless{} (int) r -- (int) p \textless\textless{}
'\textbackslash n';\strut
\end{minipage}\tabularnewline
\bottomrule
\end{longtable}

\textbf{Exercise.} Think about this: Is \emph{(int) (p + 1)} the same as
(\emph{(int) p) + 1}?

\textbf{Exercise.} There are four pointer values (i.e., address values)
in the following code segment. The first one is \emph{p}. As for the
rest:

\begin{itemize}
\tightlist
\item
  Is \emph{p + 2} larger or smaller than \emph{p}? How far apart are
  they?
\item
  Is \emph{p - 1} larger or smaller than \emph{p}? How far apart are
  they?
\item
  Is \emph{p - 3} larger or smaller than \emph{p}? How far apart are
  they?
\end{itemize}

\begin{longtable}[]{@{}l@{}}
\toprule
\endhead
\begin{minipage}[t]{0.97\columnwidth}\raggedright
int x = 42;

int * p = \&x;

std::cout \textless\textless{} (unsigned int)(p) \textless\textless{} '
'

\textless\textless{} (unsigned int)(p + 2) \textless\textless{} ' '

\textless\textless{} (unsigned int)(p - 1) \textless\textless{} ' '

\textless\textless{} (unsigned int)(p - 3) \textless\textless{}
'\textbackslash n'; \strut
\end{minipage}\tabularnewline
\bottomrule
\end{longtable}

Verify your answer by running the program.

\textbf{Exercise.} There are four pointer values (i.e., address values)
in the following code segment. The first one is \emph{p}. As for the
rest:

\begin{itemize}
\tightlist
\item
  Is \emph{p + 2} larger or smaller than \emph{p}? How far apart are
  they?
\item
  Is \emph{p - 1} larger or smaller than \emph{p}? How far apart are
  they?
\item
  Is \emph{p - 3} larger or smaller than \emph{p}? How far apart are
  they?
\end{itemize}

\begin{longtable}[]{@{}l@{}}
\toprule
\endhead
\begin{minipage}[t]{0.97\columnwidth}\raggedright
double d = 3.14;

double * p = \&d;

std::cout \textless\textless{} (unsigned int)(p) \textless\textless{} '
'

\textless\textless{} (unsigned int)(p + 5) \textless\textless{} ' ';

\textless\textless{} (unsigned int)(p - 2) \textless\textless{} ' ';

\textless\textless{} (unsigned int)(p - 4) \textless\textless{}
'\textbackslash n'; \strut
\end{minipage}\tabularnewline
\bottomrule
\end{longtable}

Verify your answer by running the program.

\textbf{Exercise.}

\begin{itemize}
\tightlist
\item
  Can we use pre- and post- increment and decrement operators on
  pointers?
\item
  What about augmented assignment operators?
\item
  What is the numeric value of the difference in address value of
  \emph{q}, \emph{r}, \emph{s} and \emph{p} in the following:
\end{itemize}

\begin{longtable}[]{@{}l@{}}
\toprule
\endhead
\begin{minipage}[t]{0.97\columnwidth}\raggedright
int x = 42;

int * p;

int * q = p;

int * r = p;

int * s = p;

q++;

r += 5;

s-\/-;

std::cout \textless\textless{} (int) q -- (int) p

\textless\textless{} ' '

\textless\textless{} (int) r -- (int) p

\textless\textless{} ' '

\textless\textless{} (int) s -- (int) p

\textless\textless{} '\textbackslash n';\strut
\end{minipage}\tabularnewline
\bottomrule
\end{longtable}

(Recall that if you have a newer laptop, you might need to use
\emph{\textbf{long long int}}.)

\textbf{Exercise.} What is the output:

\begin{longtable}[]{@{}l@{}}
\toprule
\endhead
\begin{minipage}[t]{0.97\columnwidth}\raggedright
double d = 3.14;

double * p = \&d;

double * q = p;

-\/-q;

double * r = q;

r += 4

double * s = r;

-\/-s;

std::cout \textless\textless{} (int) q -- (int) p

\textless\textless{} ' '

\textless\textless{} (int) r -- (int) p

\textless\textless{} ' '

\textless\textless{} (int) s -- (int) p \textless\textless{}
'\textbackslash n';\strut
\end{minipage}\tabularnewline
\bottomrule
\end{longtable}

\textbf{Exercise. }You have the following:

\begin{longtable}[]{@{}l@{}}
\toprule
\endhead
\begin{minipage}[t]{0.97\columnwidth}\raggedright
char c = 'a';

char * p = \&c;

\emph{char * q = p + 8;} \strut
\end{minipage}\tabularnewline
\bottomrule
\end{longtable}

How much larger is \emph{q} when compared with \emph{p}? Verify it
yourself. (If you get something different from the expected on your
computer, don't panic. Talk to me.)

Exercise. For a 32-bit machine, the integer values you can represent is
about -2 billion to 2 billion. More exactly, the range of values is from

-2\textsuperscript{31}, ...2\textsuperscript{31} - 1

(Why 31? It's 1 less than 32, the number of bits for a 32-bit machine.)
There is another integer type called the long integer. The formal name
is \emph{long} or \emph{long int}. Example:

\begin{longtable}[]{@{}l@{}}
\toprule
\endhead
\begin{minipage}[t]{0.97\columnwidth}\raggedright
long i = 0L;

long int j = 42; // OK ... C++ will type cast

// 42 to 42L for you\strut
\end{minipage}\tabularnewline
\bottomrule
\end{longtable}

(Most programmers type \emph{long} instead of \emph{long int}.)

\begin{enumerate}
\def\labelenumi{(\alph{enumi})}
\tightlist
\item
  How many bytes of memory is used by a \emph{long} value on your
  compiler? (Write a simple program.)
\item
  What is the output of the following code segment?
\end{enumerate}

\begin{longtable}[]{@{}l@{}}
\toprule
\endhead
\begin{minipage}[t]{0.97\columnwidth}\raggedright
long * p = new long;

std::cout \textless\textless{} (int) p -- (int) (p + 2)
\textless\textless{} '\textbackslash n';\strut
\end{minipage}\tabularnewline
\bottomrule
\end{longtable}

(First do this without your compiler. Then, check with your compiler.)

Exercise. There is yet another integer type called the long long
integer. The formal name is \emph{long} \emph{long} or \emph{long long
int}. Example:

\begin{longtable}[]{@{}l@{}}
\toprule
\endhead
\begin{minipage}[t]{0.97\columnwidth}\raggedright
long long i = 0LL;

long long int j = 42; // OK ... C++ will type

// cast 42 to 42LL for you\strut
\end{minipage}\tabularnewline
\bottomrule
\end{longtable}

(Most programmers just use \emph{long long} instead of \emph{long long
int}.)

\begin{enumerate}
\def\labelenumi{(\alph{enumi})}
\tightlist
\item
  How many bytes of memory is used by a \emph{long} \emph{long} value on
  your compiler? (Write a simple program.)
\item
  What is the output of the following code segment?
\end{enumerate}

\begin{longtable}[]{@{}l@{}}
\toprule
\endhead
\begin{minipage}[t]{0.97\columnwidth}\raggedright
long long * p = new long long;

std::cout \textless\textless{} (int) p -- (int) (p + 2)
\textless\textless{} '\textbackslash n';\strut
\end{minipage}\tabularnewline
\bottomrule
\end{longtable}

(First do this without your compiler. Then, check with your compiler.)

Memory layout of local variables

Suppose you have two integer variables declared next to each other:

\begin{longtable}[]{@{}l@{}}
\toprule
\endhead
\begin{minipage}[t]{0.97\columnwidth}\raggedright
int x = 42;

int y = 43;\strut
\end{minipage}\tabularnewline
\bottomrule
\end{longtable}

Then the address of y is \textbf{4 smaller} than the address of x --
assuming that \emph{\textbf{sizeof(int)}}\textbf{ is 4}. (If on your
machine, the \emph{sizeof(int)} is 8, then the address of \emph{y} is 8
smaller.)

\textbf{Caveat:} Note that that difference between \emph{\&x} and
\emph{\&y} can be more than 4 on your computer, but it cannot be less
than \emph{4}. The point is that an integer takes up \emph{sizeof(int)}
number of bytes. But your computer (and your compiler) might pad some
extra bytes between the memory used by two adjacent integer variables.
On my Windows machine, 8 extra bytes were padded. On my linux machine,
the padding is 0. In the notes below, I'm going to assume that the
amount of extra padding is 0. To check on the padding of your machine,
using the above code, print the address of x and the address of y and
subtract. If the difference is 12, then you know that the padding is 12
-- 4 = 8 since the \emph{sizeof(int)} is 4.

First run this:

\begin{longtable}[]{@{}l@{}}
\toprule
\endhead
\begin{minipage}[t]{0.97\columnwidth}\raggedright
int x = 42, y = 43, z = 44;

std::cout \textless\textless{} (unsigned int) \&x \textless\textless{} '
'

\textless\textless{} (unsigned int) \&y \textless\textless{} ' '

\textless\textless{} (unsigned int) \&z \textless\textless{}
'\textbackslash n';

int * p = \&x;

int * q = p - 1;

std::cout \textless\textless{} (unsigned int)(p) \textless\textless{} '
'

\textless\textless{} (unsigned int)(q) \textless\textless{}
'\textbackslash n'; \strut
\end{minipage}\tabularnewline
\bottomrule
\end{longtable}

(Recall: you might want to use \emph{unsigned long long}.) AHA \ldots{}
so \ldots{} q is indeed pointing to x! Try this:

\begin{longtable}[]{@{}l@{}}
\toprule
\endhead
\begin{minipage}[t]{0.97\columnwidth}\raggedright
int x = 42, y = 43, z = 44;

std::cout \textless\textless{} (unsigned int) \&x \textless\textless{} '
'

\textless\textless{} (unsigned int) \&y \textless\textless{} ' '

\textless\textless{} (unsigned int) \&z \textless\textless{}
'\textbackslash n';

int * p = \&x;

int * q = p - 1;

std::cout \textless\textless{} (unsigned int)(p) \textless\textless{} '
'

\textless\textless{} (unsigned int)(q) \textless\textless{}
'\textbackslash n';

*q = 0;

std::cout \textless\textless{} x \textless\textless{}
'\textbackslash n';\strut
\end{minipage}\tabularnewline
\bottomrule
\end{longtable}

Sure enough, \emph{*q = 0} changes \emph{x} to 0!!!

\textbf{Caveat:} In my case, because my linux machine does not pad extra
bytes between the memory of \emph{x} and the memory of \emph{y}, for
\emph{q} to point to \emph{y}, I only need \emph{q = p -- 1}. On my
Windows machine, there is a padding of 8 bytes, which means that for my
Windows machine, I would need to do \emph{q = p -- 1 - 2}, i.e., I have
to subtract 2 more because of the 8-byte padding.

So if the address of \emph{x} is 1000 (and assuming there is no extra
padding), then the address of \emph{y} is 996 and the address of
\emph{z} is 992.

(In some rare cases, your computer might arrange to have the addresses
going up instead of down. But that's rare.)

In general if you have two variables x and y of type T which are
declared next to each other (x first, then y), then

address of \emph{y} = address of \emph{x - sizeof(T)}

\textbf{Exercise.} Verify what I just said. For instance look at this:

\begin{longtable}[]{@{}l@{}}
\toprule
\endhead
\begin{minipage}[t]{0.97\columnwidth}\raggedright
double a = 3.14, b = 2.71, c = 1.41;

double * p = \&b;\strut
\end{minipage}\tabularnewline
\bottomrule
\end{longtable}

Check that p + 1 points to a and p -- 1 points to c. Since a double
takes up 8 bytes, the memory address value in p + 1 will be 8 larger
than the value in p and the memory address value in p - 1 will be 8
larger than the value in p

\textbf{Exercise. }What is the output?

\begin{longtable}[]{@{}l@{}}
\toprule
\endhead
\begin{minipage}[t]{0.97\columnwidth}\raggedright
int x = 42, y = 0, z = 1000;

int * p = \&x;

\emph{int * q = p - 1;}

\emph{*q = 5;}

q = p - 2;

\emph{*q = 99;}

std::cout \textless\textless{} x \textless\textless{} ' '
\textless\textless{} y \textless\textless{} ' ' \textless\textless{} z
\textless\textless{} std::endl;\strut
\end{minipage}\tabularnewline
\bottomrule
\end{longtable}

(Again I'm assuming no extra padding; otherwise you would need to adjust
the following).

\textbf{Exercise.} Look at this:

\begin{longtable}[]{@{}l@{}}
\toprule
\endhead
\begin{minipage}[t]{0.97\columnwidth}\raggedright
int a, b, c, d, e, f, g;

int * p = \&a;\strut
\end{minipage}\tabularnewline
\bottomrule
\end{longtable}

Using a for-loop, set \emph{a}, \emph{b}, \emph{c}, \emph{d}, \emph{e},
\emph{f}, \emph{g} to \emph{0}, \emph{1}, \emph{2}, \emph{3}, \emph{4},
\emph{5}, \emph{6}. Of course you have to verify. So ...

\begin{longtable}[]{@{}l@{}}
\toprule
\endhead
\begin{minipage}[t]{0.97\columnwidth}\raggedright
int a, b, c, d, e, f, g;

int * p = \&a;

// DO YOUR STUFF HERE USING A POINTER

std::cout \textless\textless{} a \textless\textless{} ' '
\textless\textless{} b \textless\textless{} ' ' \textless\textless{} c
\textless\textless{} ' '

\textless\textless{} d \textless\textless{} ' ' \textless\textless{} e
\textless\textless{} ' ' \textless\textless{} f \textless\textless{} ' '

\textless\textless{} g \textless\textless{} '\textbackslash n';\strut
\end{minipage}\tabularnewline
\bottomrule
\end{longtable}

Violà! Even though I have variables a, b, c, d, e, f, g (which does not
form an array), I can treat them as though they form an array if I use a
pointer to scan over them!!! (As another exercise, replace the print
statement with a for-loop that prints a,b,c,d,e,f,g.

\textbf{Exercise.} You are given the information that a \emph{double}
takes up 8 bytes. Suppose we have

\begin{longtable}[]{@{}l@{}}
\toprule
\endhead
\begin{minipage}[t]{0.97\columnwidth}\raggedright
double x = 3.14159;

double y = 2.718;

double z = 0.0;\strut
\end{minipage}\tabularnewline
\bottomrule
\end{longtable}

Suppose that in the memory layout of the above program, the value
for\emph{ y} is immediately after the value for \emph{x} and the value
for \emph{z} is immediately after \emph{y}. If the address of the value
of \emph{x} is 4800000, what is the address of the value for \emph{y}?
What is the address of \emph{z}?

Pointer arithmetic and arrays

Try this:

\begin{longtable}[]{@{}l@{}}
\toprule
\endhead
\begin{minipage}[t]{0.97\columnwidth}\raggedright
int x{[}3{]} = \{1, 2, 3\};

std::cout \textless\textless{} (unsigned int) \&x{[}0{]}
\textless\textless{} ' '

\textless\textless{} (unsigned int) \&x{[}1{]} \textless\textless{} ' '

\textless\textless{} (unsigned int) \&x{[}2{]} \textless\textless{}
'\textbackslash n';

int * p = \&x{[}0{]};

std::cout \textless\textless{} (unsigned int) p \textless\textless{} ' '

\textless\textless{} (unsigned int) (p + 1) \textless\textless{} ' '

\textless\textless{} (unsigned int) (p + 2) \textless\textless{}
'\textbackslash n';\strut
\end{minipage}\tabularnewline
\bottomrule
\end{longtable}

Note two things:

\begin{itemize}
\tightlist
\item
  the values in the array are laid out in \textbf{ascending} address
  values and
\item
  there are \textbf{no gaps} in the layout of the memory between values
  of the array (there's no memory padding -- see previous section)
\end{itemize}

Since \emph{sizeof(int)} is 4,

\textbf{ the address of }\emph{\textbf{x{[}i }\textbf{+
1}\textbf{{]}}}\textbf{ is the address }

\textbf{ of }\emph{\textbf{x{[}i{]}}}\textbf{ plus 4}

Note in particular the big difference between a bunch of \emph{int}
variables declared together and an array of integer values. In the
previous section, you saw that the addresses of a bunch of variables go
down. For an array the address of the values go up. For a bunch of
variables, there might be extra byte padding. But in the case of an
array, there's no extra byte padding.

In the same way, if we have

\begin{longtable}[]{@{}l@{}}
\toprule
\endhead
double y{[}4{]};\tabularnewline
\bottomrule
\end{longtable}

since \emph{sizeof(double)} is 8, the addresses of \emph{y{[}0{]}},
\emph{y{[}1{]}}, \emph{y{[}2{]}}, \emph{y{[}3{]}} go up by a step of 8.

In general if \emph{z} is an array of type \emph{T} values, then

the address of \emph{z{[}i + 1{]}} is the address

of \emph{z{[}i{]}} plus \emph{sizeof(T)}

This is great because

if \emph{p} points to \emph{z{[}i{]}}, then \emph{p + 1} points to

\emph{z{[}i + 1{]}}

and

if \emph{p} points to \emph{z{[}0{]}}, then \emph{p + i} points to

\emph{z{[}i{]}} and so \emph{*(p + i)} is the same as

\emph{z{[}i{]}}

You will see why this is important when we talk about dynamic memory
management of arrays in the memory heap.

Exercise. What is the output of this code segment:

\begin{longtable}[]{@{}l@{}}
\toprule
\endhead
\begin{minipage}[t]{0.97\columnwidth}\raggedright
int x{[}{]} = \{2, 3, 5, 7, 11, 13\};

int * p = \&(x{[}2{]});

std::cout \textless\textless{} *p \textless\textless{} std::endl;

++p;

std::cout \textless\textless{} *p \textless\textless{} std::endl;\strut
\end{minipage}\tabularnewline
\bottomrule
\end{longtable}

Exercise. What is the output of this code segment:

\begin{longtable}[]{@{}l@{}}
\toprule
\endhead
\begin{minipage}[t]{0.97\columnwidth}\raggedright
double x{[}{]} = \{0.0, 2.718, 3.141, 42.0\};

double * p = \&(x{[}1{]});

std::cout \textless\textless{} (*p + *(p + 1)) \textless\textless{}
std::endl;\strut
\end{minipage}\tabularnewline
\bottomrule
\end{longtable}

Exercise. What is the output of this code fragment:

\begin{longtable}[]{@{}l@{}}
\toprule
\endhead
\begin{minipage}[t]{0.97\columnwidth}\raggedright
int x{[}{]} = \{5, 4, 3, 2, 1\};

for (int * p = \&x{[}0{]}; p \textless= \&x{[}4{]}; ++p)

\{

std::cout \textless\textless{} *p \textless\textless{} ' ';

\}\strut
\end{minipage}\tabularnewline
\bottomrule
\end{longtable}

\textbf{Exercise.} Complete this code segment.

\begin{longtable}[]{@{}l@{}}
\toprule
\endhead
\begin{minipage}[t]{0.97\columnwidth}\raggedright
int x{[}{]} = \{1, 3, 5, 7, 9\};

int sum = 0;

// Complete the following to compute the sum

// of the array.

// You must not use an integer index variable but

// rather integer pointer variables.\strut
\end{minipage}\tabularnewline
\bottomrule
\end{longtable}

\textbf{Exercise.} Complete this code segment.

\begin{longtable}[]{@{}l@{}}
\toprule
\endhead
\begin{minipage}[t]{0.97\columnwidth}\raggedright
int x{[}{]} = \{1, 5, 3, 7, 9\};

int * p = \&x{[}0{]};

int * q = \&x{[}4{]};

int max = *p;

// Complete the following to compute the maximum

// value in the array.

// You must not use an integer index variable but

// rather integer pointer variables.\strut
\end{minipage}\tabularnewline
\bottomrule
\end{longtable}

Exercise. What is the output of this code segment?

\begin{longtable}[]{@{}l@{}}
\toprule
\endhead
\begin{minipage}[t]{0.97\columnwidth}\raggedright
char x{[}{]} = \{'h', 'e', 'l', 'l', 'o'\};

char * p = \&(x{[}0{]});

char * q = \&(x{[}4{]});

*p = 'j';

*(p + 2) = 'L';

while (p \textless= q)

\{

std::cout \textless\textless{} *p;

p++;\\
\}

std::cout \textless\textless{} std::endl;\strut
\end{minipage}\tabularnewline
\bottomrule
\end{longtable}

(First do this without your compiler. Next, check with your compiler.)

Exercise. What is the output of this code segment:

\begin{longtable}[]{@{}l@{}}
\toprule
\endhead
\begin{minipage}[t]{0.97\columnwidth}\raggedright
int x{[}{]} = \{2, 3, 5, 7, 11, 13\};

int * p = \&(x{[}2{]});

(*p)++;

std::cout \textless\textless{} *p \textless\textless{} std::endl;

*(p++);

std::cout \textless\textless{} *p \textless\textless{} std::endl;\strut
\end{minipage}\tabularnewline
\bottomrule
\end{longtable}

Can you explain what's happening? (By the way, do you need the
parentheses?)

Exercise. What is the output of this code segment:

\begin{longtable}[]{@{}l@{}}
\toprule
\endhead
\begin{minipage}[t]{0.97\columnwidth}\raggedright
char x{[}{]} = "c++ rules";

char * p = \&x{[}0{]};

while (*p)

\{\\
std::cout \textless\textless{} *p++;

\}\strut
\end{minipage}\tabularnewline
\bottomrule
\end{longtable}

Can you explain what's happening? (By the way, do you need the
parentheses?)

Static arrays

One of the problems with arrays is that they have fixed size.

\emph{int x{[}SIZE{]}}

The \emph{SIZE} must be constant.

(WARNING: Some C++ compilers allow you to declare arrays with variable
sizes. However this is not standard. Therefore do NOT use variable size
for arrays. You do want your C++ code to be as portable as possible,
right?)

The nice thing about an array is that it is a local variable, where the
memory of the values in the array is allocated within the scope of its
declaration and when the array goes out of the scope, the memory is
reclaimed. For instance:

\begin{longtable}[]{@{}l@{}}
\toprule
\endhead
\begin{minipage}[t]{0.97\columnwidth}\raggedright
if (a \textless{} 0)

\{

int x{[}SIZE{]};

// ... do something useful with x \ldots{}

\} // x goes out of scope, and array of values

// used by x is reclaim by the program.\strut
\end{minipage}\tabularnewline
\bottomrule
\end{longtable}

In other words, \emph{x} an automatic variable.

Such arrays are called \textbf{static arrays}.

The next 4 experiments are extremely important in your understanding of
the array, in the context of addresses. Make sure you run them all!

First experiment:

\begin{longtable}[]{@{}l@{}}
\toprule
\endhead
\begin{minipage}[t]{0.97\columnwidth}\raggedright
\#include \textless iostream\textgreater{}

int main()

\{

int x{[}5{]};

std::cout \textless\textless{} x \textless\textless{} std::endl;

return 0;\\
\}\strut
\end{minipage}\tabularnewline
\bottomrule
\end{longtable}

You will see an address!!! That tells you that the array \emph{x} is
somehow an address!!!

Second experiment:

\begin{longtable}[]{@{}l@{}}
\toprule
\endhead
\begin{minipage}[t]{0.97\columnwidth}\raggedright
\#include \textless iostream\textgreater{}

int main()

\{

int x{[}5{]};

std::cout \textless\textless{} x \textless\textless{} std::endl;

std::cout \textless\textless{} \&x{[}0{]} \textless\textless{}
std::endl;

return 0;\\
\}\strut
\end{minipage}\tabularnewline
\bottomrule
\end{longtable}

This tells you that if you think of \emph{x} as a pointer, then
\emph{\textbf{x}}\textbf{ is the address of the first value }of your
array of value.

So what's happening?!?

Whenever you use \emph{x} as though it's a value, the compiler will
actually replace \emph{x} with \emph{\&x{[}0{]}}, i.e., the address of
the first value of of the array:

\emph{\textbf{x}}\textbf{ is the same as }\emph{\textbf{\&x{[}0{]}}}

You also know that you can perform pointer arithmetic. Since \emph{x}
(besides being an array/collection of values) can be thought as a
pointer, you can think about the pointers:

\emph{x}, \emph{x + 1}, \emph{x + 2}, \emph{x + 3}, \emph{x + 4}

which are the same as

\emph{\&x{[}0{]}}, \emph{\&x{[}1{]}}, \emph{\&x{[}2{]}},
\emph{\&x{[}3{]}}, \emph{\&x{[}4{]}}

We'll verify this fact using the next experiment \ldots{}

Third experiment:

\begin{longtable}[]{@{}l@{}}
\toprule
\endhead
\begin{minipage}[t]{0.97\columnwidth}\raggedright
\#include \textless iostream\textgreater{}

int main()

\{

int x{[}5{]};

for (int i = 0; i \textless{} 5; ++i)

\{

std::cout \textless\textless{} x + i \textless\textless{} ' ';

\}

std::cout \textless\textless{} std::endl;

for (int i = 0; i \textless{} 5; ++i)

\{

std::cout \textless\textless{} \&x{[}i{]} \textless\textless{} ' ';

\}

std::cout \textless\textless{} std::endl;

return 0;\\
\}\strut
\end{minipage}\tabularnewline
\bottomrule
\end{longtable}

The above is about the addresses of the values in an array. What about
the values of the array? You know that you can access the values in the
array using

\emph{x{[}0{]}}, \emph{x{[}1{]}}, \emph{x{[}2{]}}, \emph{x{[}3{]}},
\emph{x{[}4{]}}

But since \emph{x}, \emph{x + 1}, \emph{x + 2}, \emph{x + 3}, \emph{x +
4} points to the above values, we conclude that you can also access the
values of the array using:

\emph{*x}, \emph{*(x + 1)}, \emph{*(x + 2)}, \emph{*(x + 3)}, \emph{*(x
+ 4)}

Let's check that \ldots{}

Fourth experiment ...

\begin{longtable}[]{@{}l@{}}
\toprule
\endhead
\begin{minipage}[t]{0.97\columnwidth}\raggedright
\#include \textless iostream\textgreater{}

int main()

\{

int x{[}5{]} = \{2, 3, 5, 7, 11\};

for (int i = 0; i \textless{} 5; ++i)

\{

std::cout \textless\textless{} *(x + i) \textless\textless{} ' ';

\}

std::cout \textless\textless{} std::endl;

for (int i = 0; i \textless{} 5; ++i)

\{

std::cout \textless\textless{} x{[}i{]} \textless\textless{} ' ';

\}

std::cout \textless\textless{} std::endl;

return 0;\\
\}\strut
\end{minipage}\tabularnewline
\bottomrule
\end{longtable}

You now arrive at the conclusion that

\emph{x{[}i{]}} is like \emph{*(x + i)}

In fact given \emph{\textbf{any}} pointer \emph{p} (not just an array)
and integer \emph{i}, in C/C++,

\emph{\textbf{p{[}i{]} }}\textbf{is the same as }\emph{\textbf{*(p +
i)}}

Remember that!!!

So let me summarize. When you declare this array variable:

int x{[}5{]};

\begin{itemize}
\item
  You can think of \emph{x} as a pointer that points to the first value
  in the array of 5 integers, i.e. \emph{x} has the address of
  \emph{x{[}0{]}}. In fact in your code, \emph{x}, is actually
  translated into \emph{\&x{[}0{]}}.
\item
  The values \emph{x{[}0{]}}, \emph{x{[}1{]}}, \emph{x{[}2{]}},
  \emph{x{[}3{]}}, \emph{x{[}4{]}} are laid out in memory in increasing
  address values with no gaps.
\item
  The following addresses are the same:
\item
  The following values are the same:
\end{itemize}

\textbf{Exercise.} You are given that the address of the first value of
the following array

\emph{int x{[}100{]};}

is 72008004. Assuming that each \emph{int} value takes up 4 bytes of
memory, what is the address of the value \emph{x{[}3{]}}? What is the
memory address of \emph{x{[}56{]}}?

\textbf{Exercise.} You have the following declarations:

int x = 0;

double y = 2.14;

int a{[}100{]};

double b{[}100{]};

Assuming that an \emph{int} takes up 4 bytes and a \emph{double} takes
up 8 byte, the address of \emph{x} is 120004000, what is the address of
\emph{a{[}3{]}} and \emph{b{[}20{]}}? (Assume that there is no gaps in
the memory layout of variables and address value grows.)

\textbf{Exercise.} What is the output of this code fragment:

\begin{longtable}[]{@{}l@{}}
\toprule
\endhead
\begin{minipage}[t]{0.97\columnwidth}\raggedright
int a{[}5{]} = \{2, 3, 5, 7, 11\};

int * b = a;

std::cout \textless\textless{} *(b + 2) \textless\textless{}
'\textbackslash n';\strut
\end{minipage}\tabularnewline
\bottomrule
\end{longtable}

Verify!

\textbf{Exercise.} What is the output of this code fragment:

\begin{longtable}[]{@{}l@{}}
\toprule
\endhead
\begin{minipage}[t]{0.97\columnwidth}\raggedright
int a{[}5{]} = \{2, 3, 5, 7, 11\};

int * b = (a + 2);

std::cout \textless\textless{} *(b + 1) \textless\textless{}
'\textbackslash n';\strut
\end{minipage}\tabularnewline
\bottomrule
\end{longtable}

Verify!

\textbf{Exercise.} What is the output of this code fragment:

\begin{longtable}[]{@{}l@{}}
\toprule
\endhead
\begin{minipage}[t]{0.97\columnwidth}\raggedright
int a{[}5{]} = \{2, 3, 5, 7, 11\};

int * b = (a + 3);

std::cout \textless\textless{} *(b - 2) + b{[}1{]} \textless\textless{}
'\textbackslash n';\strut
\end{minipage}\tabularnewline
\bottomrule
\end{longtable}

Verify!

\textbf{Exercise.} What is the output of this code fragment:

\begin{longtable}[]{@{}l@{}}
\toprule
\endhead
\begin{minipage}[t]{0.97\columnwidth}\raggedright
int a{[}5{]} = \{2, 3, 5, 7, 11\};

int * b = (a + 3);

std::cout \textless\textless{} *(b + 1) + b{[}-2{]} \textless\textless{}
'\textbackslash n';\strut
\end{minipage}\tabularnewline
\bottomrule
\end{longtable}

Verify!

\textbf{Exercise.} What is the output of this code fragment:

\begin{longtable}[]{@{}l@{}}
\toprule
\endhead
\begin{minipage}[t]{0.97\columnwidth}\raggedright
int a{[}5{]} = \{2, 3, 5, 7, 11\};

int b = 0;

for (int i = 0; i \textless{} 5; ++i)

\{

b += *(a + i);\\
\}

std::cout \textless\textless{} b \textless\textless{}
'\textbackslash n';\strut
\end{minipage}\tabularnewline
\bottomrule
\end{longtable}

Verify!

\textbf{Exercise.} What is the output of this code fragment:

\begin{longtable}[]{@{}l@{}}
\toprule
\endhead
\begin{minipage}[t]{0.97\columnwidth}\raggedright
int a{[}5{]} = \{2, 3, 5, 7, 11\};

int b = a + 3;

std::cout \textless\textless{} *(b -- 1) \textless\textless{}
'\textbackslash n';\strut
\end{minipage}\tabularnewline
\bottomrule
\end{longtable}

Verify!

\textbf{Exercise.} What is the output of this code fragment:

\begin{longtable}[]{@{}l@{}}
\toprule
\endhead
\begin{minipage}[t]{0.97\columnwidth}\raggedright
int a{[}5{]} = \{2, 3, 5, 7, 11\};

int * b = a + 4;

std::cout \textless\textless{} *(b -- 3) + *b \textless\textless{}
'\textbackslash n';\strut
\end{minipage}\tabularnewline
\bottomrule
\end{longtable}

Verify!

\textbf{Exercise.} What is the output of this code fragment:

\begin{longtable}[]{@{}l@{}}
\toprule
\endhead
\begin{minipage}[t]{0.97\columnwidth}\raggedright
int a{[}5{]} = \{2, 3, 5, 7, 11\};

int * b = a;

int * c = a + 1;

int s = 0;

for (int i = 0; i \textless{} 3; ++i)

\{

s += *c - *b;

++b;

++c;

\}

std::cout \textless\textless{} s \textless\textless{}
'\textbackslash n';\strut
\end{minipage}\tabularnewline
\bottomrule
\end{longtable}

Verify!

\textbf{Exercise.} Rewrite the following so that the array bracket
operator \emph{{[}{]}} is not used.

\begin{longtable}[]{@{}l@{}}
\toprule
\endhead
\begin{minipage}[t]{0.97\columnwidth}\raggedright
\#include \textless iostream\textgreater{}

\#include \textless ctime\textgreater{}

\#include \textless cstdlib\textgreater{}

int main()

\{

srand((unsigned int) time(NULL));

const int SIZE = 100;

const double MAX = double(RAND\_MAX);

double x{[}SIZE{]};

for (int i = 0; i \textless{} SIZE; i++)

\{

\emph{\textbf{x{[}i{]}}} = rand() / MAX * 2 - 1;\\
\}

double min = \emph{\textbf{x{[}0{]}}};

int index = 0;

for (int i = 1; i \textless{} SIZE; i++)

\{

if (\emph{\textbf{x{[}i{]}}} \textless{} m)

\{

min = \emph{\textbf{x{[}i{]}}};

index = i;

\}\\
\}

std::cout \textless\textless{} "min:" \textless\textless{} min

\textless\textless{} " at index:" \textless\textless{} index

\textless\textless{} '\textbackslash n';

return 0;\\
\}\strut
\end{minipage}\tabularnewline
\bottomrule
\end{longtable}

Dynamic arrays (in the heap)

With pointers you can allocate memory for an array (on the heap) with
\textbf{variable size}. Try this:

\begin{longtable}[]{@{}l@{}}
\toprule
\endhead
\begin{minipage}[t]{0.97\columnwidth}\raggedright
int s;

std::cin \textgreater\textgreater{} s;

int * x = \textbf{new int{[}s{]}}; // s can be a variable!!!

// Do something with array that x points to.

//

// (Technically speaking x points to the first value

// of the array in the heap.)

\textbf{delete {[}{]} x};\strut
\end{minipage}\tabularnewline
\bottomrule
\end{longtable}

Basically the command

new int{[}s{]}

tells the heap:

``hey heap \ldots{} gimme a chunk of an \emph{int} array of \emph{s
}values''

Note that \emph{s} can be a variable!

Mr Heap then goes through its memory and looks for an unused
\textbf{contiguous} chunk of memory for an array of \emph{s} \emph{int}
values, marks this chunk as ``in use'' (i.e. not free) and
\textbf{returns the address of the }\emph{\textbf{first}}\textbf{
integer value of this array}. Therefore, after calling

\begin{longtable}[]{@{}l@{}}
\toprule
\endhead
int * x = new int{[}s{]};\tabularnewline
\bottomrule
\end{longtable}

x points to (i.e., has the address of) the \textbf{first} int value of
an array of s int values. Again, this array of values is in the heap and
the number of int values allocated is exactly s. I will usually say x
points to the array although technically speaking x has the address of
the first int value of this array, If the value for s is 3, then the
memory model looks like this (I just put random integers into the
array):

\hfill\break

When you do not need the array that \emph{x} points to, you deallocate
the memory of this array by doing this:

\begin{longtable}[]{@{}l@{}}
\toprule
\endhead
delete {[}{]} x;\tabularnewline
\bottomrule
\end{longtable}

This will release the whole array of \emph{int} values at address
\emph{x} back to the heap so that the heap can reuse this memory for
another memory allocation.

A \textbf{big mistake} is to do this:

\begin{longtable}[]{@{}l@{}}
\toprule
\endhead
delete x;\tabularnewline
\bottomrule
\end{longtable}

which will \textbf{only release one value} (i.e. the single value that
\emph{x} points to), not the whole array, back to the heap. The problem
is that the compiler will not tell you that this is a mistake! This will
result in a \textbf{memory leak}. This means that if you do

\begin{longtable}[]{@{}l@{}}
\toprule
\endhead
\begin{minipage}[t]{0.97\columnwidth}\raggedright
int * x = new int{[}10{]};

delete x;\strut
\end{minipage}\tabularnewline
\bottomrule
\end{longtable}

you are going to have a memory leak of 9 \emph{int}s. For a 32-bit
machine, that's a loss of 9x4=36 bytes. The right thing to do is

\begin{longtable}[]{@{}l@{}}
\toprule
\endhead
\begin{minipage}[t]{0.97\columnwidth}\raggedright
int * x = new int{[}10{]};

delete {[}{]} x;\strut
\end{minipage}\tabularnewline
\bottomrule
\end{longtable}

So remember that in general,

\begin{itemize}
\tightlist
\item
  If you use \emph{new} for a single value, then you should do
  \emph{delete}.
\item
  If you do \emph{new} \emph{{[}{]}} for a whole array, you must do
  \emph{delete {[}{]}}.
\end{itemize}

Here's an example:

\begin{longtable}[]{@{}l@{}}
\toprule
\endhead
\begin{minipage}[t]{0.97\columnwidth}\raggedright
int * x;

x = new int; // ask for one int value ...

delete x; // ... give back one

x = new int{[}10{]}; // ask for 10 int values ...

delete {[}{]} x; // ... give back 10

x = new int{[}42{]}; // ask for 42 int values ...

delete {[}{]} x; // ... give back 42

x = new int; // ask for one int value ...

delete x; // ... give back one\strut
\end{minipage}\tabularnewline
\bottomrule
\end{longtable}

As in the case of dynamic memory management, it's also a good idea to
set a pointer to NULL after deallocation:

\begin{longtable}[]{@{}l@{}}
\toprule
\endhead
\begin{minipage}[t]{0.97\columnwidth}\raggedright
int * x;

x = new int; // ask for one int value ...

delete x; // ... give back one

x = NULL;

x = new int{[}10{]}; // ask for 10 int values ...

delete {[}{]} x; // ... give back 10

x = NULL;\strut
\end{minipage}\tabularnewline
\bottomrule
\end{longtable}

\textbf{Exercise.} Run this \ldots{} and see the program crash ...

\begin{longtable}[]{@{}l@{}}
\toprule
\endhead
\begin{minipage}[t]{0.97\columnwidth}\raggedright
int * x;

while (1)

\{

x = new int{[}5000{]};\\
\}\strut
\end{minipage}\tabularnewline
\bottomrule
\end{longtable}

\textbf{Exercise.} Run this \ldots{} and see the program crash ...

\begin{longtable}[]{@{}l@{}}
\toprule
\endhead
\begin{minipage}[t]{0.97\columnwidth}\raggedright
int * x;

while (1)

\{

x = new int{[}5000{]};

delete x;\\
\}\strut
\end{minipage}\tabularnewline
\bottomrule
\end{longtable}

\textbf{Exercise.} Run this \ldots{} and see what happens:

\begin{longtable}[]{@{}l@{}}
\toprule
\endhead
\begin{minipage}[t]{0.97\columnwidth}\raggedright
int * x;

while (1)

\{

x = new int{[}1000{]};

delete {[}{]} x;\\
\}\strut
\end{minipage}\tabularnewline
\bottomrule
\end{longtable}

Stop after a couple of minutes.

\textbf{Exercise.} Fixit time ...

\begin{longtable}[]{@{}l@{}}
\toprule
\endhead
\begin{minipage}[t]{0.97\columnwidth}\raggedright
int * w;

int * x = new int;

int * y = new int{[}100{]};

int * z;

delete w;

delete {[}{]} x;

delete y;

delete {[}{]} z;\strut
\end{minipage}\tabularnewline
\bottomrule
\end{longtable}

After

int * x = new int{[}s{]};

your x points to the first value in the array. How do you get to the
i-th value in this array? It's just

\emph{x{[}i{]}}

Remember? Recall that the address of the \emph{i}-th value in this array
is

\emph{x + i}

Therefore, dereferencing, the value at the address is

\emph{*(x + i)}

Recall that \emph{x{[}i{]}} is just a shorthand for \emph{*(x + i):}

\emph{x{[}i{]}}is the same as\emph{*(x + i)}

Most programmers use \emph{x{[}i{]}} since we want to think of \emph{x}
as an array (although technically it's a pointer.) The above explanation
was already presented in the previous section. Why am I repeating?
BECAUSE IT'S IMPORTANT!!!

Try this:

\begin{longtable}[]{@{}l@{}}
\toprule
\endhead
\begin{minipage}[t]{0.97\columnwidth}\raggedright
int size;

std::cin \textgreater\textgreater{} size;

double * x = new double{[}size{]};

for (int i = 0; i \textless{} size; i++)

\{

std::cin \textgreater\textgreater{} x{[}i{]};\\
\}

double product = 1.0;

for (int i = 0; i \textless{} size; i++)

\{

product *= x{[}i{]};

std::cout \textless\textless{} product \textless\textless{}
'\textbackslash n';\\
\}

delete {[}{]} x;\strut
\end{minipage}\tabularnewline
\bottomrule
\end{longtable}

\textbf{Exercise.} What's wrong with this:

\begin{longtable}[]{@{}l@{}}
\toprule
\endhead
\begin{minipage}[t]{0.97\columnwidth}\raggedright
int size;

std::cin \textgreater\textgreater{} size;

double * x = new double{[}size{]};

for (int i = 0; i \textless{} size; i++)

\{

std::cin \textgreater\textgreater{} x{[}i{]};\\
\}

double product = 1.0;

for (int i = 0; i \textless{} size; i++)

\{

product *= x{[}i{]};

std::cout \textless\textless{} product \textless\textless{}
'\textbackslash n';\\
\}

delete x;\strut
\end{minipage}\tabularnewline
\bottomrule
\end{longtable}

\textbf{Exercise.} Write a program that continually prompts the user for
a value for integer variable \emph{i}, allocate an array of integers of
size \emph{i} for pointer \emph{p}, put \emph{i} random integer values
into the array, sort the array using bubblesort, print all the values
\emph{p} points to, deallocate the memory. Terminate the program when
the user enters a 0 for \emph{i}.

\textbf{Exercise.} Write a program that

\begin{itemize}
\tightlist
\item
  Prompts the user for s.
\item
  Creates a dynamic array (on the heap) and declare pointer p to point
  to an array of s integer values.
\item
  Sets all values in the array to 1.
\item
  Sets all values at index 0, 2, 4, \ldots{} and toggle the value
  between 0 and 1, i.e., if the value is 1, set it to 0 and if it's 0,
  set it to 1.
\item
  Sets all values at index 0, 3, 6, \ldots{} and toggle the value
  between 0 and 1, i.e., if the value is 1, set it to 0 and if it's 0,
  set it to 1.
\item
  Sets all values at index 0, 4, 8, \ldots{} and toggle the value
  between 0 and 1, i.e., if the value is 1, set it to 0 and if it's 0,
  set it to 1.
\item
  Etc.
\item
  At the end of this process, prints the index values where the value is
  1.
\item
  Deallocates the array.
\end{itemize}

Comparing pointers and arrays

There are three concepts here:

\begin{itemize}
\tightlist
\item
  static arrays,
\item
  dynamic arrays in the heap, and
\item
  pointers.
\end{itemize}

differences. Most of the comparisons below are already presented
earlier. However, because of the similarities and differences, it's best
I put everything here so that you can study all the facts in one place.

A pointer can be used to access any value in your computer if you have
the address of that value (and if you have access rights). In
particular, if a pointer points to the first value of an array (static
or dynamic array), using pointer arithmetic, you can access value of the
array.

In terms of syntax, there are many similarities between a pointer and an
array (static of not). This is deliberate -- the designers of C++ made
it so.

\textbf{Static and dynamic arrays}

SIMILARITY \#1: Array

SIMILARITY \#3: Bracket operator.

The i-th value of a static array and the i-value of a dynamic array can
both be accessed using the bracket operator.

\begin{longtable}[]{@{}l@{}}
\toprule
\endhead
\begin{minipage}[t]{0.97\columnwidth}\raggedright
char x{[}10{]} = \{'a', 'b', 'c'\};

char * y = x;

int size = 10;

char * z = new char{[}size{]};

//

// index-2 value of x is x{[}2{]} or y{[}2{]}

// index-2 value of array z is pointing to is z{[}2{]}\strut
\end{minipage}\tabularnewline
\bottomrule
\end{longtable}

Remember that \emph{z{[}2{]}} is a shorthand for \emph{*(z + 2)}.

SIMILARITY \#4: Bracket operator and out of bound access

After you declare declare a static array or allocate memory for a
dynamic array (in the heap) with a size s, the array has s values
starting with index 0 and ending with index s -- 1. HOWEVER, it's
possible to access values \textbf{OUTSIDE} the array -- remember what I
said about pointers being able to access any value in your computer (as
long as you have the address and the access rights). For instance is x
is a static array of size 10 or a pointer to a dynamic array of size 10
in the heap, then \emph{x{[}i{]}} is just the value at location \emph{x
+ i}. You can have \emph{x{[}42{]}}, \emph{x{[}-5{]}}, etc.!!! You can
access the value at address \emph{x + 42} and \emph{x + (-5)} as long as
you have the access rights!!!

Now for the differences \ldots{}

\textbf{DIFFERENCE \#1:} \textbf{Memory location.}

The most obvious difference is the location of the values of the array.
In the case of

\begin{longtable}[]{@{}l@{}}
\toprule
\endhead
\begin{minipage}[t]{0.97\columnwidth}\raggedright
const int size = 10;

char x{[}size{]};\strut
\end{minipage}\tabularnewline
\bottomrule
\end{longtable}

the memory allocation is local (i.e. in the current execution block).
When you exit the block, the memory use for this array is reclaimed
automatically. Therefore static arrays are automatic variables. Dynamic
arrays created in the heap using \emph{new{[}{]}}, i.e.

\begin{longtable}[]{@{}l@{}}
\toprule
\endhead
\begin{minipage}[t]{0.97\columnwidth}\raggedright
int size = 10;

char * y = new char{[}size{]};\strut
\end{minipage}\tabularnewline
\bottomrule
\end{longtable}

lives in the heap, not in the local scope. (Of course the pointer
\emph{x} is local). The values of this array is not reclaimed
automatically; \emph{x} is local so the pointer value of \emph{x} is
reclaimed. You must deallocate the memory of this array yourself (when
you're done using the array) by doing this:

\begin{longtable}[]{@{}l@{}}
\toprule
\endhead
delete {[}{]} y;\tabularnewline
\bottomrule
\end{longtable}

\textbf{DIFFERENCE \#2:} \textbf{Size.}

The second difference is that the size of the array must be a constant
for static array:

\begin{longtable}[]{@{}l@{}}
\toprule
\endhead
\begin{minipage}[t]{0.97\columnwidth}\raggedright
const int size = 10;

char x{[}size{]}; // size must be constant\strut
\end{minipage}\tabularnewline
\bottomrule
\end{longtable}

However the size used in creating an array in the heap can be any
variable expression:

\begin{longtable}[]{@{}l@{}}
\toprule
\endhead
\begin{minipage}[t]{0.97\columnwidth}\raggedright
int size = 10;

char * y = new char{[}size{]};

...

delete {[}{]} y;\strut
\end{minipage}\tabularnewline
\bottomrule
\end{longtable}

\textbf{DIFFERENCE 3.} \textbf{Initialization.}

An array can be initialized. For instance

\begin{longtable}[]{@{}l@{}}
\toprule
\endhead
\begin{minipage}[t]{0.97\columnwidth}\raggedright
const int size = 10;

char x{[}size{]} = \{'a', 'b', 'c'\};\strut
\end{minipage}\tabularnewline
\bottomrule
\end{longtable}

There is no corresponding syntax for the case for a dynamic array on the
heap.

\begin{longtable}[]{@{}l@{}}
\toprule
\endhead
\begin{minipage}[t]{0.97\columnwidth}\raggedright
int size = 10;

char * y = new char{[}size{]} \{'a', 'b', 'c'\}; // WHAT?!?

delete {[}{]} y;\strut
\end{minipage}\tabularnewline
\bottomrule
\end{longtable}

You have to perform assignment:

\begin{longtable}[]{@{}l@{}}
\toprule
\endhead
\begin{minipage}[t]{0.97\columnwidth}\raggedright
int size = 10;

char * y = new char{[}size{]};

y{[}0{]} = 'a';

y{[}1{]} = 'b';

y{[}2{]} = 'c';

delete {[}{]} y;\strut
\end{minipage}\tabularnewline
\bottomrule
\end{longtable}

\textbf{Static arrays and pointers}

\textbf{SIMILARITY 1. Array name is like a pointer.}

You can do this:

\begin{longtable}[]{@{}l@{}}
\toprule
\endhead
\begin{minipage}[t]{0.97\columnwidth}\raggedright
const int SIZE = 10;

char x{[}SIZE{]};

char * y = x; // Remember that x is \&x{[}0{]}\strut
\end{minipage}\tabularnewline
\bottomrule
\end{longtable}

The reason (as stated above) is because in the above \emph{x} is the
same as \emph{\&x{[}0{]}}. So the above make \emph{x} like a pointer
(syntactically speaking).

DIFFERENCE 1: Array name is constant pointer, not just pointer.

However static arrays are not exactly pointers. If x and y are pointers
of the same type (with values), then you can assign between them:

\begin{longtable}[]{@{}l@{}}
\toprule
\endhead
x = y;\tabularnewline
\bottomrule
\end{longtable}

But try this:

\begin{longtable}[]{@{}l@{}}
\toprule
\endhead
\begin{minipage}[t]{0.97\columnwidth}\raggedright
char x{[}10{]};

char * y;

char * z = new char{[}10{]};

char w{[}10{]};

y = x; // OK

x = z; // NOPE!!!\strut
\end{minipage}\tabularnewline
\bottomrule
\end{longtable}

So what's \emph{x} if \emph{x} is an array since it's not a pointer? You
can view \emph{x} as a constant pointer. In other words, you can think
of \emph{x} as a pointer but the address of \emph{x} cannot be changed.
So for this code fragment:

\begin{longtable}[]{@{}l@{}}
\toprule
\endhead
\begin{minipage}[t]{0.97\columnwidth}\raggedright
char x{[}10{]};

char * y;

y = x; // OK\strut
\end{minipage}\tabularnewline
\bottomrule
\end{longtable}

If you want to make \emph{y} as close to \emph{x} as possible, you can
do this:

\begin{longtable}[]{@{}l@{}}
\toprule
\endhead
\begin{minipage}[t]{0.97\columnwidth}\raggedright
char x{[}10{]};

char * const y = x; // OK\strut
\end{minipage}\tabularnewline
\bottomrule
\end{longtable}

In that case the address value in \emph{y} cannot be changed.

\textbf{DIFFERENCE 2:}\emph{\textbf{ }\textbf{sizeof().}}

Another difference is in the behavior of the \emph{sizeof()} function.
Try this:

\begin{longtable}[]{@{}l@{}}
\toprule
\endhead
\begin{minipage}[t]{0.97\columnwidth}\raggedright
char x{[}10{]};

char * y = x;

std::cout \textless\textless{} sizeof(x) \textless\textless{}
'\textbackslash n';

std::cout \textless\textless{} sizeof(y) \textless\textless{}
'\textbackslash n';\strut
\end{minipage}\tabularnewline
\bottomrule
\end{longtable}

The \emph{sizeof()} on an array returns the size of memory used for the
array of values. The \emph{sizeof()} on a pointer returns the size of
memory used for the pointer itself, i.e., the memory used for the
address of the first value of the array, not the array.

Array and pointer parameters

I've already talked about pointers and arrays. Let's see what happens
when they are used as parameters of functions.

When you make a function call like this:

\begin{longtable}[]{@{}l@{}}
\toprule
\endhead
\begin{minipage}[t]{0.97\columnwidth}\raggedright
void f(int x)

\{\}

int main()

\{

int y = 42;

f(y); // give value of y to x of

// function f()

return 0;\\
\}\strut
\end{minipage}\tabularnewline
\bottomrule
\end{longtable}

you can pretty much understand the data passed from \emph{main()} to
\emph{f()} by running this code segment:

\begin{longtable}[]{@{}l@{}}
\toprule
\endhead
\begin{minipage}[t]{0.97\columnwidth}\raggedright
int y = 42;

int x = y; // give value of y to x\strut
\end{minipage}\tabularnewline
\bottomrule
\end{longtable}

(This is pass-by-value.)

However there is one case where this does not work. The following that
passes an array to a function does work:

\begin{longtable}[]{@{}l@{}}
\toprule
\endhead
\begin{minipage}[t]{0.97\columnwidth}\raggedright
void f(int x{[}{]})

\{\}

int main()

\{

int y{[}100{]} = \{1, 2, 3\};

f(y);

return 0;\\
\}\strut
\end{minipage}\tabularnewline
\bottomrule
\end{longtable}

You might be tempted to run this code segment:

\begin{longtable}[]{@{}l@{}}
\toprule
\endhead
\begin{minipage}[t]{0.97\columnwidth}\raggedright
int y{[}100{]} = \{1, 2, 3\};

int x{[}{]} = y;\strut
\end{minipage}\tabularnewline
\bottomrule
\end{longtable}

You \emph{\textbf{will}} get a error.

However \textbf{this works}:

\begin{longtable}[]{@{}l@{}}
\toprule
\endhead
\begin{minipage}[t]{0.97\columnwidth}\raggedright
int y{[}100{]} = \{1, 2, 3\};

int * x = y;\strut
\end{minipage}\tabularnewline
\bottomrule
\end{longtable}

So here's the thing you need to know:

An array parameter is like a pointer

Now the question is this: Is the pointer parameter a
\emph{\textbf{constant}} pointer? Let's try an experiment:

\begin{longtable}[]{@{}l@{}}
\toprule
\endhead
\begin{minipage}[t]{0.97\columnwidth}\raggedright
void f(int x{[}{]})

\{

int * y = new int{[}10{]};

x = y; // can parameter x be changed?

\}

int main()

\{

int y{[}100{]} = \{1, 2, 3\};

f(y);

return 0;\\
\} \strut
\end{minipage}\tabularnewline
\bottomrule
\end{longtable}

It turns out that the above does work. This means that the \textbf{x is
not a constant pointer}. This means that

\begin{longtable}[]{@{}l@{}}
\toprule
\endhead
\begin{minipage}[t]{0.97\columnwidth}\raggedright
void f(int x{[}{]})

\{\}\strut
\end{minipage}\tabularnewline
\bottomrule
\end{longtable}

is the same as

\begin{longtable}[]{@{}l@{}}
\toprule
\endhead
\begin{minipage}[t]{0.97\columnwidth}\raggedright
void f(int * x)

\{\}\strut
\end{minipage}\tabularnewline
\bottomrule
\end{longtable}

You can verify this piece of information using the following trick to
get your compiler to tell you the truth. Compile this program:

\begin{longtable}[]{@{}l@{}}
\toprule
\endhead
\begin{minipage}[t]{0.97\columnwidth}\raggedright
void f(int x{[}{]})

\{

abc; // a deliberate error

\}

int main()

\{

return 0;

\}\strut
\end{minipage}\tabularnewline
\bottomrule
\end{longtable}

Your compiler will give you an error. For g++, here's the error:

\begin{longtable}[]{@{}l@{}}
\toprule
\endhead
\begin{minipage}[t]{0.97\columnwidth}\raggedright
c.cpp: In function 'void f(int*)':

c.cpp:3:3: error: 'abc' was not declared in this scope\strut
\end{minipage}\tabularnewline
\bottomrule
\end{longtable}

AHA!!! The function header of \emph{f()} is really

\emph{void f(int *)}

and \emph{\textbf{not}}

\emph{void f(int {[}{]})}

So here's the extremely important principle to remember:

\textbf{An array parameter is actually a }

non-constant pointer

Default values

There's another thing that we should investigate: What about default
values? Let's try this:

\begin{longtable}[]{@{}l@{}}
\toprule
\endhead
\begin{minipage}[t]{0.97\columnwidth}\raggedright
void f(int x{[}{]} = \{1, 2, 3\})

\{\}

int main()

\{

f();

return 0;\\
\} \strut
\end{minipage}\tabularnewline
\bottomrule
\end{longtable}

You will get an error. Of course you will. You already know that
pointers cannot be initialized with an array. In other words the
following is invalid:

\begin{longtable}[]{@{}l@{}}
\toprule
\endhead
\begin{minipage}[t]{0.97\columnwidth}\raggedright
int main()

\{

int * x = \{1, 2, 3\};

return 0;

\}\strut
\end{minipage}\tabularnewline
\bottomrule
\end{longtable}

You can try other cases (with constantness):

\begin{longtable}[]{@{}l@{}}
\toprule
\endhead
\begin{minipage}[t]{0.97\columnwidth}\raggedright
const int * a = \{1, 2, 3\};

int * const b = \{1, 2, 3\};

const int * const c = \{1, 2, 3\};\strut
\end{minipage}\tabularnewline
\bottomrule
\end{longtable}

None of them will work.

In general the only default value you should use is NULL:

\begin{longtable}[]{@{}l@{}}
\toprule
\endhead
\begin{minipage}[t]{0.97\columnwidth}\raggedright
void f(int x{[}{]} = NULL)

\{\} \strut
\end{minipage}\tabularnewline
\bottomrule
\end{longtable}

There \emph{\textbf{is}} one case where a pointer can be initialized to
point to values and only for certain compilers. Therefore you shouldn't
use the following in real code:

const char * x = "hello world";

Again. Don't use this in real code. It's better to just be simple, like
this:

char x{[}{]} = "hello world";

Functions with array parameters: where to start

This is an old example:

\begin{longtable}[]{@{}l@{}}
\toprule
\endhead
\begin{minipage}[t]{0.97\columnwidth}\raggedright
int sum(int x{[}{]})

\{

int s = 0;

for (int i = 0; i \textless{} 10; ++i)

\{

s += x{[}i{]};\\
\}

return s;

\}\strut
\end{minipage}\tabularnewline
\bottomrule
\end{longtable}

This adds up the values x{[}0{]}, \ldots, x{[}9{]}. Great. But this one
is even better because it allows to specify how many values to add:

\begin{longtable}[]{@{}l@{}}
\toprule
\endhead
\begin{minipage}[t]{0.97\columnwidth}\raggedright
int sum(int x{[}{]}, int size)

\{

int s = 0;

for (int i = 0; i \textless{} size; ++i)

\{

s += x{[}i{]};\\
\}

return s;

\}\strut
\end{minipage}\tabularnewline
\bottomrule
\end{longtable}

You can use it like this:

\begin{longtable}[]{@{}l@{}}
\toprule
\endhead
\begin{minipage}[t]{0.97\columnwidth}\raggedright
int a{[}{]} = \{1, 3, 5, 7, 2, 4, 6, 8\};

std::cout \textless\textless{} sum(a, 8) \textless\textless{}
'\textbackslash n';\strut
\end{minipage}\tabularnewline
\bottomrule
\end{longtable}

which will add up all the 8 values in the array. Of course if I want to
sum up only the first 4 numbers, I would do this:

\begin{longtable}[]{@{}l@{}}
\toprule
\endhead
\begin{minipage}[t]{0.97\columnwidth}\raggedright
int a{[}{]} = \{1, 3, 5, 7, 2, 4, 6, 8\};

std::cout \textless\textless{} sum(a, 8) \textless\textless{}
'\textbackslash n';

std::cout \textless\textless{} sum(a, 4) \textless\textless{}
'\textbackslash n';\strut
\end{minipage}\tabularnewline
\bottomrule
\end{longtable}

But what if I want to compute \emph{x{[}4{]} + x{[}5{]} + x{[}6{]}}? In
other words, I want to specify the starting index of the summation. The
function starts with index position 0. Recall that in the notes on
functions and arrays, I have to rewrite the above function so that it
accepts a starting index value, like this:

\begin{longtable}[]{@{}l@{}}
\toprule
\endhead
\begin{minipage}[t]{0.97\columnwidth}\raggedright
int sum(int x{[}{]}, int start, int size)

\{

int s = 0;

for (int i = start; i \textless{} size; ++i)

\{

s += x{[}i{]};\\
\}

return s;

\}\strut
\end{minipage}\tabularnewline
\bottomrule
\end{longtable}

Actually it's \textbf{not necessary at all (if you know pointers)!}

You see I can do this:

\begin{longtable}[]{@{}l@{}}
\toprule
\endhead
\begin{minipage}[t]{0.97\columnwidth}\raggedright
\#include \textless iostream\textgreater{}

int sum(int x{[}{]}, int size)

\{

int s = 0;

for (int i = 0; i \textless{} size; i++)

\{

s += x{[}i{]};\\
\}

return s;

\}

int main()

\{

int a{[}{]} = \{1,3,5,7,2,4,6,8\};

std::cout \textless\textless{} sum(a, 8) \textless\textless{}
'\textbackslash n';

std::cout \textless\textless{} sum(a, 4) \textless\textless{}
'\textbackslash n';

std::cout \textless\textless{} \textbf{sum(a + 4, 3)}
\textless\textless{} '\textbackslash n';

return 0;

\}\strut
\end{minipage}\tabularnewline
\bottomrule
\end{longtable}

to compute \emph{x{[}4{]} + x{[}5{]} + x{[}6{]}}. In this case, the
value for size is the number of terms that is added.

Neat right? This means that when writing functions for arrays, you
\textbf{never have to specify an option for a starting index value}!!!
(For languages that do not support pointers and pointer arithmetic, the
alternative is almost always slower.)

Exercise. What is the output of this program:

\begin{longtable}[]{@{}l@{}}
\toprule
\endhead
\begin{minipage}[t]{0.97\columnwidth}\raggedright
\#include \textless iostream\textgreater{}

int sum(int x{[}{]}, int size)

\{

int s = 0;

for (int i = 0; i \textless{} size; i++)

\{

s += x{[}i{]};\\
\}

return s;

\}

double avg(int x{[}{]}, size)

\{

return double(sum(x, size)) / size;\\
\}

int main()

\{

int a{[}{]} = \{1, 3, 5, 7, 2, 4, 6, 8\};

std::cout \textless\textless{} avg(a + 4, 3) \textless\textless{}
'\textbackslash n';

return 0;

\}\strut
\end{minipage}\tabularnewline
\bottomrule
\end{longtable}

Exercise. Rewrite the following program so that the prototype of the max
function looks like this:

int max(int x{[}{]}, int size);

Modify the test code in \emph{main()} so that the same result is
produced.

\begin{longtable}[]{@{}l@{}}
\toprule
\endhead
\begin{minipage}[t]{0.97\columnwidth}\raggedright
\#include \textless iostream\textgreater{}

int max(int x{[}{]}, int start, int end)

\{

int m = x{[}start{]};

for (int i = start + 1; i \textless= end; i++)

\{

if (m \textless{} x{[}i{]}) m = x{[}i{]};\\
\}

return m;

\}

int main()

\{

int a{[}{]} = \{1, 3, 5, 7, 2, 4, 6, 8\};

std::cout \textless\textless{} max(a, 0, 2) \textless\textless{}
'\textbackslash n';

std::cout \textless\textless{} max(a, 4, 6) \textless\textless{}
'\textbackslash n';

std::cout \textless\textless{} max(a, 6, 6) \textless\textless{}
'\textbackslash n';

return 0;

\}\strut
\end{minipage}\tabularnewline
\bottomrule
\end{longtable}

Functions with array parameters: where to start and where to end

Let's continue with the example from the previous section:

\begin{longtable}[]{@{}l@{}}
\toprule
\endhead
\begin{minipage}[t]{0.97\columnwidth}\raggedright
int sum(int x{[}{]}, int size)

\{

int s = 0;

for (int i = 0; i \textless{} size; ++i)

\{

s += x{[}i{]};\\
\}

return s;

\}

int main()

\{

int a{[}{]} = \{1,3,5,7,2,4,6,8\};

std::cout \textless\textless{} sum(a, 8) \textless\textless{}
'\textbackslash n';

std::cout \textless\textless{} sum(a, 4) \textless\textless{}
'\textbackslash n';

std::cout \textless\textless{} sum(a + 4, 3) \textless\textless{}
'\textbackslash n';

return 0;

\} \strut
\end{minipage}\tabularnewline
\bottomrule
\end{longtable}

Again, in function sum, you want to think of x as a pointer to an array
of values. When you call

\begin{longtable}[]{@{}l@{}}
\toprule
\endhead
... sum(a + 4, 3) ...\tabularnewline
\bottomrule
\end{longtable}

You are basically telling the function to start at the value which is 4
integers away from the address a (i.e., the 5\textsuperscript{th}
integer in the array).

I can rewrite the \emph{sum} function like this so that it's really
clear that the array parameter is really a pointer parameter:

\begin{longtable}[]{@{}l@{}}
\toprule
\endhead
\begin{minipage}[t]{0.97\columnwidth}\raggedright
int sum(int * start\_pointer, int size)

\{

int s = 0;

for (int i = 0; i \textless{} size; ++i)

\{

s += start\_pointer{[}i{]};\\
\}

return s;

\}\strut
\end{minipage}\tabularnewline
\bottomrule
\end{longtable}

Since you're using a starting pointer address to specify where to find
the first value, why not specify the pointer address of the last value?
In that case your code becomes \textbf{more uniform} like this:

\begin{longtable}[]{@{}l@{}}
\toprule
\endhead
\begin{minipage}[t]{0.97\columnwidth}\raggedright
int sum(int * start\_pointer, int * end\_pointer)

\{

int s = 0;

for (int * p = start\_pointer;

p \textless= end\_pointer; ++p)

\{

s += *p;\\
\}

return s;

\}\strut
\end{minipage}\tabularnewline
\bottomrule
\end{longtable}

You do have to know for instance that ++p on a pointer will move the
pointer p to the next integer (and I've already talked about it.)

Now, it's actually more common to specifying the ending address value to
be the value that is just \textbf{one step outside} the array of values
to be processed:

\begin{longtable}[]{@{}l@{}}
\toprule
\endhead
\begin{minipage}[t]{0.97\columnwidth}\raggedright
int sum(int * start\_pointer, int * end\_pointer)

\{

int s = 0;

for (int * p = start\_pointer;

p \textbf{\textless{}} end\_pointer; ++p)

\{

s += *p;\\
\}

return s;

\}\strut
\end{minipage}\tabularnewline
\bottomrule
\end{longtable}

Simplifying the parameter names a little, I get \ldots{}

\begin{longtable}[]{@{}l@{}}
\toprule
\endhead
\begin{minipage}[t]{0.97\columnwidth}\raggedright
int sum(int * start, int * end)

\{

int s = 0;

for (int * p = start; p \textbf{\textless{}} end; ++p)

\{

s += *p;\\
\}

return s;

\}\strut
\end{minipage}\tabularnewline
\bottomrule
\end{longtable}

(Instead of start and end, it's also common to name the pointers begin
and end.)

In this form, \emph{main()} would have to look like this:

\begin{longtable}[]{@{}l@{}}
\toprule
\endhead
\begin{minipage}[t]{0.97\columnwidth}\raggedright
int main()

\{

int a{[}{]} = \{1, 3, 5, 7, 2, 4, 6, 8\};

std::cout \textless\textless{} sum(a, a + 8) \textless\textless{}
'\textbackslash n';

std::cout \textless\textless{} sum(a, a + 4) \textless\textless{}
'\textbackslash n';

std::cout \textless\textless{} sum(a + 4, a + 7) \textless\textless{}
'\textbackslash n';

return 0;

\}\strut
\end{minipage}\tabularnewline
\bottomrule
\end{longtable}

Let me repeat: the general practice is that \emph{end} points to one
value beyond what you want to process. So the values processed are

\emph{*start}, \emph{*(start + 1)}, \emph{*(start + 2)}, \ldots,
\emph{*(end - 1)}

Note that no values are process if start equals end.

\textbf{Exercise.} Complete and test this function that does the
obvious:

\begin{longtable}[]{@{}l@{}}
\toprule
\endhead
void println(int * start, int * end);\tabularnewline
\bottomrule
\end{longtable}

(It's clear when you should stop.) Here's the format of the output:

\begin{longtable}[]{@{}l@{}}
\toprule
\endhead
\{5, 1, 2, 4\}\tabularnewline
\bottomrule
\end{longtable}

A newline should be printed. When \emph{start} and \emph{end} have the
same value, the output looks like this:

\begin{longtable}[]{@{}l@{}}
\toprule
\endhead
\{\}\tabularnewline
\bottomrule
\end{longtable}

\textbf{Exercise.} Complete and test this function that does the
obvious:

\begin{longtable}[]{@{}l@{}}
\toprule
\endhead
int max(int * start, int * end);\tabularnewline
\bottomrule
\end{longtable}

(It's clear when you should stop.) You maybe assume at least one value
is processed, i.e., assume \emph{start} \textless= \emph{end -- 1}.

\textbf{Exercise.} Complete and test this function:

\begin{longtable}[]{@{}l@{}}
\toprule
\endhead
int * max(int * start, int * end);\tabularnewline
\bottomrule
\end{longtable}

that returns the address of the maximum value in \emph{*start}, \ldots,
\emph{*(end -- 1)} if \emph{start \textless= end -- 1}. Otherwise
\emph{NULL} is returned. You maybe assume at least one value is
processed, i.e., assume \emph{start} \textless= \emph{end -- 1}.

\textbf{Exercise.} Complete and test this function:

\begin{longtable}[]{@{}l@{}}
\toprule
\endhead
int * linearsearch(int * start, int * end, int target);\tabularnewline
\bottomrule
\end{longtable}

that returns the address of the first \emph{target} (left-to-right) that
occurs in the array of values \emph{*start}, \ldots, \emph{*(end -- 1)}.
If \emph{target} does not occur in the array, \emph{NULL} is returned.

\textbf{Exercise.} Complete and test this very important function:

\begin{longtable}[]{@{}l@{}}
\toprule
\endhead
void fill(int * start, int * end, int value);\tabularnewline
\bottomrule
\end{longtable}

that copies \emph{value} to the values at \emph{start}, \emph{start +
1}, etc.

\textbf{Exercise.} Complete and test this very important function:

\begin{longtable}[]{@{}l@{}}
\toprule
\endhead
void copy(int * start0, int * start1, int * end1);\tabularnewline
\bottomrule
\end{longtable}

that copies the value at \emph{start1} to the value at location
\emph{start0}, etc. (It's clear when you should stop.)

Exercise. Complete and test this very important function:

\begin{longtable}[]{@{}l@{}}
\toprule
\endhead
void resize(int ** start, int * size, int newsize);\tabularnewline
\bottomrule
\end{longtable}

Initially, \emph{*start} describes an array of \emph{size} integers in
the heap. When we return from the above function call, \emph{*start}
contains the address of an array of \emph{newsize} integers in the heap
and it contains (as many as possible) the values from the original
array. For instance if you have

\begin{longtable}[]{@{}l@{}}
\toprule
\endhead
int * p = NULL, size = 0;\tabularnewline
\bottomrule
\end{longtable}

after calling

\begin{longtable}[]{@{}l@{}}
\toprule
\endhead
resize(\&p, \&size, 5);\tabularnewline
\bottomrule
\end{longtable}

\emph{p} points to an array of 5 integers in the heap and \emph{size} is
changed to 5. Suppose we continue the above with

\begin{longtable}[]{@{}l@{}}
\toprule
\endhead
\begin{minipage}[t]{0.97\columnwidth}\raggedright
p{[}0{]} = 2;

p{[}1{]} = 3;

p{[}2{]} = 5;\strut
\end{minipage}\tabularnewline
\bottomrule
\end{longtable}

Here's the memory model:

Next if we call

\begin{longtable}[]{@{}l@{}}
\toprule
\endhead
resize(\&p, \&size, 7);\tabularnewline
\bottomrule
\end{longtable}

\emph{p} points to an array of 7 integers in the heap and \emph{size} is
changed to 7. Furthermore the first three values of the new array
\emph{p} points to are 2, 3, and 5. The memory model at this point looks
like this:

(Note: \emph{newsize} can be smaller than \emph{size}.)

Difference of pointers: binary search

Note that we have been using addition operator pointers:

Here's continue with the example from the previous section:

\begin{longtable}[]{@{}l@{}}
\toprule
\endhead
\begin{minipage}[t]{0.97\columnwidth}\raggedright
int a{[}{]} = \{1,3,5,7,2,4,6,8\};

int * p = \&a{[}1{]};

int * q = \&a{[}4{]}; \strut
\end{minipage}\tabularnewline
\bottomrule
\end{longtable}

That's addition of a pointer value and a constant integer.

Go ahead and try this:

\begin{longtable}[]{@{}l@{}}
\toprule
\endhead
\begin{minipage}[t]{0.97\columnwidth}\raggedright
int a{[}{]} = \{1,3,5,7,2,4,6,8\};

int * p = \&a{[}1{]};

int * q = p + 3;

std::cout \textless\textless{} \textbf{p + q} \textless\textless{}
'\textbackslash n';\strut
\end{minipage}\tabularnewline
\bottomrule
\end{longtable}

Read the error message carefully.

You \textbf{cannot add pointers}.

However try this:

\begin{longtable}[]{@{}l@{}}
\toprule
\endhead
\begin{minipage}[t]{0.97\columnwidth}\raggedright
int a{[}{]} = \{1,3,5,7,2,4,6,8\};

int * p = \&a{[}1{]};

int * q = \&a{[}4{]};

std::cout \textless\textless{} \textbf{p - q} \textless\textless{}
'\textbackslash n';\strut
\end{minipage}\tabularnewline
\bottomrule
\end{longtable}

It works. This tells you (of course) that the integer \emph{q} points to
is 3 integers away from the integer that \emph{p} points.

Why is this important?

Recall from the previous section, when it comes to functions that
computes on a section of an array, i.e. a subarray, you want to call the
function with two pointers. For instance:

\begin{longtable}[]{@{}l@{}}
\toprule
\endhead
\begin{minipage}[t]{0.97\columnwidth}\raggedright
int sum(int * start, int * end)

\{

// returns sum of *start, ..., *(end - 1)

...

\}\strut
\end{minipage}\tabularnewline
\bottomrule
\end{longtable}

Sometimes you want to know how many terms of the array is involved. This
is

end -- start

This is the size of the subarray that the function works with.

Another reason why the difference of pointers is important is because of
binary search. In the scenario when you are performing binary search on
a subarray where you specify the subarray using index values, the
function looks like this:

\begin{longtable}[]{@{}l@{}}
\toprule
\endhead
\begin{minipage}[t]{0.97\columnwidth}\raggedright
int binarysearch(int x{[}{]}, int start, int end,

int target)

\{

int left = start, right = end -- 1;

while (left \textless= right)

\{

int mid = (left + right) / 2;

...

\}

\}\strut
\end{minipage}\tabularnewline
\bottomrule
\end{longtable}

See chapter on 1D arrays of CISS240.

This version has a problem. Remember that an integer is usually (at
least for now) is made up of 32 bits. So the range of an integer value
is -2\^{}31, \ldots, 2\^{}31 -- 1. So when your \emph{left} and
\emph{right} are both large, for instance they are both very close to
2\^{}31 -- 1, then \emph{left + right} will overflow and become a
negative integer!!!

The way to fix this is to compute the \emph{mid} index like this:

\begin{longtable}[]{@{}l@{}}
\toprule
\endhead
\begin{minipage}[t]{0.97\columnwidth}\raggedright
int binarysearch(int x{[}{]}, int start, int end,

int target)

\{

int left = start, right = end -- 1;

while (left \textless= right)

\{

int mid = \textbf{left + (right - left) / 2;}

...

\}

\}\strut
\end{minipage}\tabularnewline
\bottomrule
\end{longtable}

This version will not cause an integer overflow. So from now on, you
must use this version of binary search.

In the above, \emph{right} is the largest index of the subarray the
current iteration of your binary search is working on.

You can also do this where \emph{right} is \textbf{one index beyond} the
subarray your binary search is working on:

\begin{longtable}[]{@{}l@{}}
\toprule
\endhead
\begin{minipage}[t]{0.97\columnwidth}\raggedright
int binarysearch(int x{[}{]}, start, int end,

int target)

\{

int left = start, right = \textbf{end};

while (left \textbf{\textless{}} right)

\{

int mid = left + (right - left) / 2;

...

\}

\}\strut
\end{minipage}\tabularnewline
\bottomrule
\end{longtable}

This is now the index value version of binary search that you must use:

\begin{longtable}[]{@{}l@{}}
\toprule
\endhead
\begin{minipage}[t]{0.97\columnwidth}\raggedright
int binarysearch(int x{[}{]}, int start, int end,

int target)

\{

int left = start, right = end;

while (left \textless{} right)

\{

int mid = (left + right) / 2;

if (x{[}mid{]} == target)

\{

return mid;

\}

else if (x{[}mid{]} \textgreater{} target)

\{

right = mid; // note: no "- 1"

\}

else

\{

left = mid + 1;

\}

\}

\}\strut
\end{minipage}\tabularnewline
\bottomrule
\end{longtable}

What about a pointer version of binary search? Using the CISS240 version
as a guide, when you try this:

\begin{longtable}[]{@{}l@{}}
\toprule
\endhead
\begin{minipage}[t]{0.97\columnwidth}\raggedright
int * binarysearch(int * start, int * end,

int target)

\{

int * left = start;

int * right = end;

while (left \textless{} right)

\{

int * mid = (left + right) / 2;

...

\}

\}\strut
\end{minipage}\tabularnewline
\bottomrule
\end{longtable}

you'll see that it does not even compile \ldots{} because you
\textbf{cannot add pointers}.

\begin{longtable}[]{@{}l@{}}
\toprule
\endhead
\begin{minipage}[t]{0.97\columnwidth}\raggedright
...

int * mid = (left \textbf{+} right) / 2;

...\strut
\end{minipage}\tabularnewline
\bottomrule
\end{longtable}

The solution is to compute the middle pointer in a different way:

\begin{longtable}[]{@{}l@{}}
\toprule
\endhead
\begin{minipage}[t]{0.97\columnwidth}\raggedright
int * binarysearch(int * start, int * end,

int target)

\{

int * left = start;

int * right = end;

while (left \textless{} right)

\{

int * mid = left + (right -- left) / 2;

...

\}

\}\strut
\end{minipage}\tabularnewline
\bottomrule
\end{longtable}

The quantity

(right -- left) / 2

is half the number of integer from left to just before right. It's
slightly less than half if the number of such values is odd. Add this to
left would then of course give you the address of the integer in the
middle of integers \textbf{*left,\ldots,*(right -- 1)}.

This middle pointer computation is just like the middle index
computation.

From now on you should use the midpoint calculation

mid = left + (right - left) / 2

for both middle index and middle pointer computation for binary search.

By the way, in general, there's no reason to compute the difference of
two addresses other than the address of two values in the same array.
One last thing: if you want to store the difference of two pointers, you
need to know that there's a type for pointer differences:

std::ptrdiff\_t d = p -- q;

You need to \#include \textless cstddef\textgreater{} if you want to use
std::ptrdiff\_t.

\textbf{Exercise.} Implement the pointer version of bubblesort and
binary search:

void bubblesort(int * start, int * end);

int * binarysearch(int * start, * end, int target);

In your \emph{main()}, get integers \emph{n}, \emph{start}, \emph{end},
\emph{target} from the user and then do the following:

\begin{itemize}
\tightlist
\item
  Allocate memory for a dynamic integer array of size \emph{n} and put
  random integers (from 0..9) into the array.
\item
  Print the array.
\item
  Bubblesort \emph{x{[}start{]}}, \ldots, \emph{x{[}end-1{]}}.
\item
  Print the array.
\item
  Perform binary search for \emph{target} in \emph{x{[}start{]}},
  \ldots, \emph{x{[}end-1{]} }and print the address and the value at
  that address.
\end{itemize}

\textbf{Exercise.} Let \emph{x} be an integer array of size \emph{n}.
Let \emph{y} be an array of integer pointers where \emph{y{[}0{]}}
points to \emph{x{[}0{]}}, \emph{y{[}1{]}} points to \emph{x{[}1{]}},
etc. Write a bubblesort on \emph{y} so that \emph{y{[}0{]}}, \ldots,
\emph{y{[}n - 1{]}} each points to a unique values in x and
*\emph{y{[}0{]} \textless= *y{[}1{]}, }*\emph{y{[}1{]} \textless=
*y{[}2{]},} etc. The values in \emph{x} are not moved. For instance say
\emph{x} is \emph{\{1, 5, 0\}} (when \emph{n} = 3). Initially
\emph{y{[}0{]}} points to \emph{x{[}0{]}}, \emph{y{[}1{]}} points to
\emph{x{[}1{]}} and \emph{y{[}2{]}} points to \emph{x{[}2{]}}. After
your bubblesort is done, \emph{y{[}0{]}} points to \emph{x{[}2{]}},
\emph{y{[}1{]}} points to \emph{x{[}0{]}} and \emph{y{[}2{]}} points to
\emph{x{[}1{]}}.

Stack using a dynamic array

A \textbf{stack} is like an array except that it supports the following
operations:

\begin{itemize}
\tightlist
\item
  You can put a value into the stack. This is called \textbf{push}. The
  value you just put into the stack is called the \textbf{top} of the
  stack.
\item
  You can remove a value from the stack. The value that is removed is
  always the last one that was put into the stack. This is called
  \textbf{pop}.
\item
  You have access to the number of values in the stack. I'm going to
  call this the \textbf{size} of the stack.
\item
  You can also look at the value that is to be popped. I'll call this
  \textbf{peek}. Peeking at a stack returns a copy of the value, it does
  not remove the top.
\end{itemize}

Think of a stack of plates at a buffet restaurant: When you add a plate
to the stack, that new plate goes on top. When you take a plate, you
take the one on top (well \ldots{} usually).

You can implement a stack using a dynamic array. The top of the stack is
the value at the last index of the array. If you push a value onto the
stack, you will need to enlarge the array by 1. If you pop the stack,
you will need to replace the array with another one that is 1 size
smaller; of course you need to copy the original values to the new stack
(except for the top.)

Pushing the stack is not new, you have actually seen this in an earlier
exercise on the \emph{array\_append() function.}

Stacks are very important and are used in many areas of CS including AI,
language processing, etc. A stack should be thought of as a memory
device. You can ask the stack ``Hey stack \ldots{} what was the
\textbf{last thing} I put into you?''

\textbf{Exercise.} Write a program that implements a stack of integers
using dynamic arrays. Test it thoroughly! You can test your stack
implementation with the following:

\begin{longtable}[]{@{}l@{}}
\toprule
\endhead
\begin{minipage}[t]{0.97\columnwidth}\raggedright
int main()

\{

int * stack = NULL;

int stack\_size = 0;

println(stack, stack\_size); // Get \{\}

push(\&stack, \&stack\_size, 5);

println(stack, stack\_size); // Get \{5\}

push(\&stack, \&stack\_size, 3);

println(stack, stack\_size); // Get \{5, 3\}

push(\&stack, \&stack\_size, 9);

println(stack, stack\_size); // Get \{5, 3, 9\}

int x;

x = peek(stack, stack\_size);

std::cout \textless\textless{} x \textless\textless{} std::endl; // Get
9

println(stack, stack\_size); // Get \{5, 3, 9\}

pop(\&stack, \&stack\_size);

println(stack, stack\_size); // Get \{5, 3\}

return 0;

\}\strut
\end{minipage}\tabularnewline
\bottomrule
\end{longtable}

There are many ways to implement a stack. Later you'll see that a more
efficient way of implementing a stack is to use a \textbf{singly linked
list}, something that uses pointers.

Queue using a dynamic array

A \textbf{queue} is like an array except that it supports the following
operations:

\begin{itemize}
\tightlist
\item
  You can put a value into a queue. This is called \textbf{enqueue}. The
  value you just put into the queue is called the \textbf{back} of the
  queue.
\item
  You can remove a value from the queue. The value that is removed is
  always the value in the queue that has been in the queue the longest.
  This is called the \textbf{front} of the queue. This operation is
  called \textbf{dequeue}.
\item
  You have access to the number of values in the queue. I'm going to
  call this the \textbf{size} of the queue.
\item
  You can also look at the value of the front and back of the queue
  without removing these values. These operations are sometimes called
  \textbf{front} and \textbf{back}.
\end{itemize}

You think of a queue as a line of people in front of a ticket booth:
When a person joins a queue, he/she joins it at the back. When the booth
is ready to serve the next customer, the person to leave the queue to go
to the booth is the one in front of the queue.

You can implement a queue using a dynamic array. The front of the queue
is the value at index 0 of the array. The back of the queue is the value
at the last index of the array. If you enqueue a value into the queue,
you will need to enlarge the array by 1. If you dequeue the stack, you
will need to replace the array with another one that is 1 size smaller.

Like stacks, queues are very important and are used in many areas of CS
including AI, language processing, etc. A queue should be thought of as
a memory device. You can ask the queue ``Hey queue \ldots{} what was the
\textbf{earliest thing} that you can think of?''

\textbf{Exercise.} Write a program that implements a queue of integers
using dynamic arrays. Test it thoroughly! You can test your queue
implementation with the following:

\begin{longtable}[]{@{}l@{}}
\toprule
\endhead
\begin{minipage}[t]{0.97\columnwidth}\raggedright
int main()

\{

int * queue = NULL;

int queue\_size = 0;

println(queue, queue\_size); // Get \{\}

enqueue(\&queue, \&queue\_size, 5);

println(queue, queue\_size); // Get \{5\}

enqueue(\&queue, \&queue\_size, 3);

println(queue, queue\_size); // Get \{5, 3\}

enqueue(\&queue, \&queue\_size, 9);

println(queue, queue\_size); // Get \{5, 3, 9\}

int x;

x = front(queue);

std::cout \textless\textless{} x \textless\textless{} std::endl; // Get
5

x = back(queue, queue\_size);

std::cout \textless\textless{} x \textless\textless{} std::endl; // Get
9

println(queue, queue\_size); // Get \{5, 3, 9\}

dequeue(\&queue, \&queue\_size);

println(queue, queue\_size); // Get \{3, 9\}

return 0;

\}\strut
\end{minipage}\tabularnewline
\bottomrule
\end{longtable}

There are many ways to implement a queue. Later you'll see that a more
efficient way of implementing a queue is to use a \textbf{doubly linked
list}, something that uses pointers (and is very similar to singly
linked list.)

Set using a dynamic array

A \textbf{set} is a container of values where you don't care about the
number of times a value occurs and you don't really care about the index
position of a value. It supports the following operations:

\begin{itemize}
\tightlist
\item
  You can \textbf{add} a value to a set. You can put it anywhere you
  like in your dynamic array. For instance you can put is at the last
  index position, like the \emph{array\_append()} function. Since you
  don't really care how many times a value occurs, if you try to add a
  value into the set and the set already contains that value, you simply
  do NOT add the value into the set, i.e., a set does not contain
  duplicates.
\item
  You can \textbf{remove} a value from a set. If you attempt to remove a
  value not in the set, then nothing happens -- the set is unchanged.
\item
  You can check if a value is \textbf{in} a set. I'll call this
  \emph{has\_member()} function.
\item
  You have access to the number of values in the set. I'll call this
  \textbf{size}.
\end{itemize}

You can think of a set also as a memory device except that you don't
care about when you started remembering a value.

\textbf{Exercise.} Write a program that implements a set of integers
using dynamic arrays. Test it thoroughly! You can test your set
implementation with the following:

\begin{longtable}[]{@{}l@{}}
\toprule
\endhead
\begin{minipage}[t]{0.97\columnwidth}\raggedright
int main()

\{

int * set = NULL;

int set\_size = 0;

println(set, set\_size); // Get \{\}

add(\&set, \&set\_size, 5);

println(set, set\_size); // Get \{5\}

add(\&stack, \&set\_size, 3);

println(set, set\_size); // Get \{5, 3\}

add(\&set, \&set\_size, 5);

println(set, set\_size); // Get \{5, 3\}

add(\&set, \&set\_size, 3);

println(set, set\_size); // Get \{5, 3\}

bool b;

b = has\_member(set, set\_size, 5);

std::cout \textless\textless{} b \textless\textless{} std::endl; // Get
1

println(set, set\_size); // Get \{5, 3\}

b = has\_member(set, set\_size, 3);

std::cout \textless\textless{} b \textless\textless{} std::endl; // Get
1

println(set, set\_size); // Get \{5, 3\}

b = has\_member(set, set\_size, 2);

std::cout \textless\textless{} b \textless\textless{} std::endl; // Get
0

println(set, set\_size); // Get \{5, 3\}

remove(\&set, \&set\_size, 3);

println(set, set\_size); // Get \{5\}

remove(\&set, \&set\_size, 0);

println(set, set\_size); // Get \{5\}

return 0;

\}\strut
\end{minipage}\tabularnewline
\bottomrule
\end{longtable}

There are many ways to implement a set. The above is actually not
efficient. Later you will see that better ways to implement a set
includes using a balanced tree and a hashtable.

Array of pointers and multi-dimensional dynamic arrays

I want you to think deeply about this

If I do ``int x{[}10{]};'' I have a fixed size 1D array x of size 10.

and

If I do ``int * x = new int{[}10{]};'' x points to a 1D array of size
10.

\textbf{Exercise.} Take a deep breath \ldots{} Draw the memory model of
the following code fragment. Study the code very carefully.

\begin{longtable}[]{@{}l@{}}
\toprule
\endhead
\begin{minipage}[t]{0.97\columnwidth}\raggedright
int ** p;

p = new int*{[}3{]};

// Draw a memory model here

for (int i = 0; i \textless{} 3; i++)

\{

p{[}i{]} = new int{[}5{]};

// Update memory model

\}

p{[}1{]}{[}1{]} = 42;

p{[}2{]}{[}3{]} = 24;

// Update memory model

for (int i = 0; i \textless{} 3; i++)

\{

delete {[}{]} p{[}i{]};

// Update memory model

\}

// Update memory model\strut
\end{minipage}\tabularnewline
\bottomrule
\end{longtable}

So what exactly is \emph{p} pointing to (so to speak?)

\textbf{Exercise.} Take a deep breath \ldots{} Draw the memory model of
the following code fragment. Read the code very carefully.

\begin{longtable}[]{@{}l@{}}
\toprule
\endhead
\begin{minipage}[t]{0.97\columnwidth}\raggedright
int ** p;

p = new int*{[}3{]};

// Draw a memory model here

for (int i = 0; i \textless{} 3; i++)

\{

p{[}i{]} = new int{[}i + 2{]};

// Draw a memory model

\}

p{[}1{]}{[}1{]} = 42;

p{[}2{]}{[}3{]} = 24;

// Update memory model

for (int i = 0; i \textless{} 3; i++)

\{

delete {[}{]} p{[}i{]};

// Update memory model

\}

// Update memory model\strut
\end{minipage}\tabularnewline
\bottomrule
\end{longtable}

See the power of pointers yet?

Graphs and social media

One of the most important structures in CS is the graph. A
\textbf{directed graph} is just a bunch of dots and arrows between the
dots. For instance here's a social media graph:

say this represents ``follows the twitters of''. For instance John
follows Sue's twitters. In general, nowadays social media type graphs
are extremely huge and complex -- and they keep growing. Trying to get
information out of social media graphs without computers is impossible.
(Obviously information gathered from social media graphs of all types
would be extremely valuable.)

One way to represent the above graph is to use a table or spreadsheet
like this:

\begin{longtable}[]{@{}llll@{}}
\toprule
\endhead
& john & sue & tom\tabularnewline
john & 0 & 1 & 1\tabularnewline
sue & 0 & 0 & 1\tabularnewline
tom & 1 & 0 & 0\tabularnewline
\bottomrule
\end{longtable}

It's pretty obvious how you should interpret the table. Like I said, in
the real world, graphs can be extremely huge. So we need to find a way
to put the table into a program and answer interesting questions such
as:

\begin{itemize}
\tightlist
\item
  Is there a sequence of arrows joining X to Y?
\end{itemize}

Clearly the table looks like a 2D array!

\begin{longtable}[]{@{}lll@{}}
\toprule
\endhead
0 & 1 & 1\tabularnewline
0 & 0 & 1\tabularnewline
1 & 0 & 0\tabularnewline
\bottomrule
\end{longtable}

This matrix (a matrix is just a 2D array of numbers) that described a
directed graph is called the \textbf{adjacency matrix} of the graph.

Of course we need to remember that john has a row and column index of 0,
sue has a row and column index of 1, and tom has a row and column index
of 2. So we can have an array of strings. Don't forget that strings are
character arrays (containing the null character `\textbackslash0'). So
the index-name associate can be described by a 2D array of characters:

\begin{longtable}[]{@{}llllllllll@{}}
\toprule
\endhead
`j' & `o' & `h' & `n' & `\textbackslash0' & & & & &\tabularnewline
`s' & `u' & `e' & `\textbackslash0' & & & & & &\tabularnewline
`t' & `o' & `m' & `\textbackslash0' & & & & & &\tabularnewline
\bottomrule
\end{longtable}

In the above, I'm assuming that there are 3 persons in the social media
graph and their names are at most 9 characters long (don't forget that
you need one character space for the null character.)

Therefore all in all, the space requirement is like this:

\begin{longtable}[]{@{}l@{}}
\toprule
\endhead
\begin{minipage}[t]{0.97\columnwidth}\raggedright
char name{[}3{]}{[}10{]};

int matrix{[}3{]}{[}3{]}; \strut
\end{minipage}\tabularnewline
\bottomrule
\end{longtable}

a total of 9 integers and 30 characters. In general, if there are k
people, then the space requirement is k x k integers and k x 10
characters (if names have a maximum string length of 9.)

\textbf{Exercise.} Write a program that will allow you to manage the
above social media graph, assuming that there will always be only 3
persons and their names are at most 9 characters long. Your program of
course should be able to display the names and the social media matrix.
You should allow the user to change the names and the matrix (but always
assuming that there are 3 persons in the social media graph.)

Exercise. Now improve on your above program by allowing the user to add
and delete person(s) from the social media graph.

Exercise. Now improve on your above program so that if a person's name
has 6 characters, then you use 7 characters for this person.
Furthermore, your program should allow names of length greater than 9.
This will save some memory. For instance for tom, you only need 4
characters for him.

Exercise. Now, how can we save memory for the social media matrix? You
see the problem is: If there are k people, then the matrix is made up of
k x k integers. Think about this: How many people do you think have a
facebook account? It's approximately 1.2 billion. If you use the above
2D array to represent the social media graph in FB, then the matrix
needs 1.2 billion x 1.2 billion integers!!! That's 1.44 x
10\textsuperscript{18} integers!!! {[}HINT: Most people are not friends
of everyone in the world. In other words, in a row of a social media
adjacency matrix, most of the values are actually 0s.{]}

\textbf{Exercise.} Path connectivity problem. Now add the following to
your program: The user to enter two names, X and Y, and the program
prints a sequence of names going X to Y, following the arrows. If
possible, print the sequence that is the shortest. {[}This is not easy!
You will probably need \textbf{everything} in the notes up to this
point!{]}

By the way, in the real world, many graphs are ``bidirectional'':

In that case, the directed graph diagram is usually drawn like this:

where each line represents the fact that the relationship between two
dots go both ways. A graph where the lines do not have arrowheads is
called an \textbf{undirected graph}.

Both directed and undirected graphs appear everywhere in CS and
engineering. As long as you have relationships between data, you will
have graphs. And they have been around for a very long time -- a family
tree is a directed graph. But it's also used in modern scenarios such as
social media graphs, a graph of nodes in a computer network, food web,

etc.

Solutions

\textbf{Exercise.} Implement the pointer version of bubblesort and
binary search:

int * bubblesort(int * start, * end)

int * binarysearch(int * start, * end, int target)

In your \emph{main()}, get integers \emph{n}, \emph{start}, \emph{end},
\emph{target} from the user and then do the following:

\begin{itemize}
\tightlist
\item
  Allocate memory for a dynamic integer array of size \emph{n}. and put
  random integers (from 0..9) into the array.
\item
  Print the array.
\item
  Bubblesort \emph{x{[}start{]}}, \ldots, \emph{x{[}end-1{]}}.
\item
  Print the array.
\item
  Perform binary search for \emph{target} in \emph{x{[}start{]}},
  \ldots, \emph{x{[}end-1{]} }and print the address and the value at
  that address.
\end{itemize}

Solution is below. Don't memorize the code. Understand the motivation
and intention of bubblesort and binarysearch, implement them using index
values, then convert to pointers.

\begin{longtable}[]{@{}l@{}}
\toprule
\endhead
\begin{minipage}[t]{0.97\columnwidth}\raggedright
\#include \textless iostream\textgreater{}

// print *start, .., *(end - 1)

void println(int * start, int * end) // or begin and end

\{

for (int * p = start; p \textless{} end; ++p)

\{

std::cout \textless\textless{} (*p) \textless\textless{} ' ';

\}

std::cout \textless\textless{} '\textbackslash n';

\}

// bubblesort *start, .., *(end - 1)

void bubblesort(int * start, int * end)

\{

for (int * q = end - 2; q \textgreater= start; -\/-q)

\{

for (int * p = start; p \textless= q; ++p)

\{

if (*p \textgreater{} *(p + 1))

\{

int t = *p; *p = *(p + 1); *(p + 1) = t;

\}

\}

\}

\}

// binarysearch for target in *start, .., *(end - 1)

int * binarysearch(int * start, int * end, int target)

\{

while (start \textless{} end)

\{

int * mid = start + (end - start) / 2;

if (*mid == target)

\{

return mid;

\}

else if (*mid \textgreater{} target)

\{

end = mid;

\}

else

\{

start = mid + 1;

\}

\}

return NULL;

\}

int main()

\{

int n, start, end, target;

std::cin \textgreater\textgreater{} n \textgreater\textgreater{} start
\textgreater\textgreater{} end \textgreater\textgreater{} target;

int * x = new int{[}n{]};

for (int i = 0; i \textless{} n; ++i)

\{

x{[}i{]} = rand() \% 10;

\}

print(\&x{[}0{]}, \&x{[}n{]});

bubblesort(\&x{[}start{]}, \&x{[}end{]});

print(\&x{[}0{]}, \&x{[}n{]});

int * p = binarysearch(\&x{[}start{]}, \&x{[}end{]}, target);

std::cout \textless\textless{} p \textless\textless{} ' '

\textless\textless{} (p != NULL ? *p : -1) \textless\textless{}
'\textbackslash n';

return 0;

\}\strut
\end{minipage}\tabularnewline
\bottomrule
\end{longtable}

\end{document}
