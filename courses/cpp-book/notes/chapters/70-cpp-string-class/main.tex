\newpage\TEXTTTASIZE{70. C++ \texttt{string} class}

\textsc{Objectives}

\begin{itemize}
\item
  Learn about C strings.
\item
  Learn about the \texttt{string} class from the C++ standard library.
\end{itemize}

\newpage\EMPHASIZE{Review: C-strings}

You already know that a string is an array of \texttt{char} terminated by '\0'. '\0' has numeric value 0. Here's an example:

\begin{console}
char x[] = "abc"; // same as {'a', 'b', 'c', '\0'}
\end{console}

These are called C-strings, i.e., C-strings are character arrays
containing the null character.

There are some useful functions to work with C strings of which some are
listed below (see CISS240):

\begin{console}
strcpy(s1, s2) // copy
strcmp(s1, s2) // compare
strcat(s1, s2) // concat
strlen(s1) // string length
\end{console}

Note that in order to work with the above functions, you must include
the header \texttt{cstring} like so:

\begin{console}
#include <cstring>
\end{console}

Note that the relational operators ==, !=, etc. don't work with the string content but rather with the pointer to the first value the array. In particular, note that \texttt{s1 == s2} compares the addresses of \texttt{s1} and \texttt{s2}. In other words, it does not compare the contents of the arrays. Likewise for $!=$, $<$, etc.

\newpage\EMPHASIZE{The \texttt{std::string} class in C++}

You can use C-strings just fine in C++ but C++ also provides a
\texttt{string} class as part of it's standard library
that can be easier to work with. The \texttt{string} class in C++ is
defined under the \texttt{std} namespace like the rest of the standard
library and provides the usual operations you would expect with strings
packaged together as a first class object.

Some major features of the C++ \texttt{string} class are outline below:
\begin{console}
string(const char *)    // constructor that lets you
                        // build a C++ string object
                        // from a C-string

=                       // assign a string to another
                        // works even with C strings
==     !=               // compare C++ string objects

>      <                // relate C++ string objects

+=     +                // concatenate string objects

[int]  at(int)          // access characters at a
                        // particular index

substr(int)             // Get substrings at particular
substr(int, int)        // locations in the string

length()                // Get the length of the string

empty()                 // Check if the string is empty

swap(string &)          // swap two string values\
\end{console}

Check C++ references online for more.

Here's a sample \texttt{main.cpp} that shows most of the
above in action. Ask yourself what the output should be then verify by
running it.

\begin{console}
#include <iostream>
#include <string>

int main()
{   
    std::string s1("Hello, World!"), s2("");
    std::string s3("Hello, World!");
    std::cout << "1:" << s1 << "\n";
    std::cout << "2:" << s1.length() << "\n";
    std::cout << "3:[" << s2 << "]\n";
    std::cout << "4:" << s2.length() << "\n";
    std::cout << "5:" << (s1 != s2) << "\n";
    std::cout << "6:" << (s1 == s2) << "\n";
    std::cout << "7:" << (s2.empty()) << "\n";
    std::cout << "8:" << (s2 == "") << "\n";
    s2 = "!!";
    std::cout << "9:" << s1 + s2 << "\n";
    std::cout << "10:" << s1 + "!!" << "\n";
    s1 += s3;
    std::cout << "11:" << s1 << "\n";
    s1 += "???";
    std::cout << "12:" << s1 << "\n";
    std::cout << "13:" << s1.substr(3) << "\n";
    std::cout << "14:" << s1.substr(3, 1) << "\n";
    std::cout << "15:" << s1[2] << "\n";
    s1[2] = 'X';
    std::cout << "16:" << s1 << "\n";
    return 0;
}
\end{console}

\newpage\EMPHASIZE{Searching for substrings}

You can use the methods \texttt{find()} and \texttt{rfind()} provided by the
string library to search for substrings within a \texttt{string} object.
Use \texttt{find()} to search from the left side of the string and
\texttt{rfind()} to search from the right side instead. You can use both
C++ \texttt{string} objects and C strings as the substring values to
search for.

Example.

\begin{console}
std::string s1 = "01234567890123456789";
std::string s2 = "23";
std::cout << "1." << s1.find(s2) << "\n";
std::cout << "2." << s1.find("23") << "\n";
std::cout << "3." << s1.find("$") << "\n";
std::cout << "4." << s1.rfind("23") << "\n"; \\
\end{console}

Both \texttt{find()} and \texttt{rfind()} return the index where the
substring first occurs. If the substring isn't found,
\texttt{std::string::npos} is returned which is the same as -1 but as an
\texttt{unsigned int}.

\newpage\EMPHASIZE{Searching for characters}

The standard \texttt{string} library provides the following methods to
search for characters in a string:

\begin{itemize}
\item
  \texttt{find_first_of()} which takes in a character, a C-string or a
  \texttt{string} object as its argument and finds the first index from
  the left side in the invoking \texttt{string} object where any of the
  characters in the argument occur.
\item
  \texttt{find_last_of()} which is similar to \texttt{find_first_of()}
  except it searches from the right side instead.
\item
  \texttt{find_first_not_of()} which also takes in a character, a C
  string or a \texttt{string} object as argument but finds the index of
  the first character in the invoking \texttt{string} object that
  isn't also in the argument provided to the method.
\item
  \texttt{find_last_not_of()} which is similar to
  \texttt{find_first_not_of()} except that it searches from the right
  side instead.
\end{itemize}

\textbf{Example.}

\begin{console}
std::string s = "01234567890123456789";
std::string t = "23";
std::cout << "1." << s.find_first_of(t) <<"\n";
std::cout << "2." << s.find_first_not_of("01") << "\n";
std::cout << "3." << s.find_last_of("012") << "\n";
std::cout << "4." << s.find_last_not_of("012") << "\n";
\end{console}

As with substring search, if nothing is found, \texttt{std::string::npos}
is returned which is the same as -1 but as an \texttt{unsigned int}.

\newpage\EMPHASIZE{Replacement}

You can use the \texttt{replace()} method to replace the contents of a
string with the content of another. You need to specify indices to
replace for the \texttt{string} object invoking the method but if you
want, you can also specify the portion of the string passed in as the
argument to use for the replacement.

\textbf{Example.}

\begin{console}
#include <iostream>
#include <string>

int main()
{   
    std::string s1("0123456789");
    std::string s2("abcdef");
    s1.replace(3, 2, s2);
    std::cout << s1 << "\n";
    s1 = "0123456789";
    s1.replace(3, 2, s2, 1, 3);
    std::cout << s1 << "\n";
    return 0;
} 
\end{console}

There are still other forms of \texttt{replace()} available which can take
in C strings or characters or even let you repeat a character a certain
number of times. Research using Google when you have free time.

\newpage\EMPHASIZE{Insertion}

You can use the \texttt{insert()} method to insert the contents of a
string into another. You need to specify the index of the invoking
\texttt{string} object to insert at but you can also specify the portion
of the string passed in as the argument to insert.

\textbf{Example.}

\begin{console}
#include <iostream>
#include <string>

int main()
{   
    std::string s1("0123456789");
    std::string s2("abcdef");
    s1.insert(3, s2);
    std::cout << s1 << "\n";
    s1 = "0123456789";
    s1.insert(3, s2, 2, 3);
    std::cout << s1 << "\n";
    return 0;
} 
\end{console}

Even more method signatures of \texttt{insert()} exist as well.

\newpage\EMPHASIZE{Type conversion}

Given a C++ string object, you can get a C string with the same content
using the \texttt{c_str()} method.

\textbf{Example.}

\begin{console}
#include <iostream>
int main()
{   
    std::string s1 = "Hi!";
    const char * s2;
    s2 = s1.c_str();
    std::cout << s2 << "\n";
    return 0;
}
\end{console}

You can use a regular constant character pointer to bind to the C-string returned but you \EMPHASIZE{shouldn't delete the pointer} if you do since the pointer will be pointing to the C-string content of the C++ \texttt{string} object.

\begin{console}
const char * s2;
s2 = s1.c_str();
std::cout << s2 << "\n";
delete [] s2; // BAD!!!
\end{console}

Since the C-string returned is a constant pointer to a character array,
you \EMPHASIZE{can't change its content} through simple
assignment.

\begin{console}
const char * s2;
s2 = s1.c_str();
s2[1] = '$'; // BAD!!! \\
\end{console}

\newpage\EMPHASIZE{External Resources}

There are many resources available to you to learn more if you are
interested.

Most textbooks contain some information on the \texttt{std::string} class.

The internet is full of websites containing a detailed reference to the
C++ standard library. The following are two good ones, just Google for
more:

\begin{itemize}
\item
  \href{http://www.cplusplus.com/}{http://www.cplusplus.com}
\item
  \href{http://en.cppreference.com/}{http://en.cppreference.com}
\end{itemize}

You should get used to researching content online. When you have
questions, you are more than likely running into an issue that other
people also have run into in the past. Leverage their experience. Find
and join a good C++ forum. At the very least, join the following
discussion websites meant for programmers:

\begin{itemize}
\item
  \url{http://stackoverflow.com/}
\item
  \url{http://programmers.stackexchange.com/}
\end{itemize}

Of course you shouldn't outright plagiarize content you find online. It won't really help you much to just copy-paste code in anyway.

\newpage\EMPHASIZE{Exercises}

\begin{ex}
Do an experiment to verify that \texttt{operator>>} (i.e., input) is overloaded to work with \texttt{std::string} objects. Note that a whitespace will terminate an input.
\end{ex}
\begin{ex}
Write a program that does the following: Declare a
\texttt{std::string} object and initialize it with a string. Create an
array of \texttt{std::string} objects from the words in the first string.
The separator of words include the space and the period. Print all the
words in the first string, each on a separate line. Print the total
number of words and the average number of characters per word.
\end{ex}
\begin{ex}
Write a program that prompts the user for a string
and prints ``integer'' if the string is made up of digit characters
where the first is not `0'; otherwise the program prints ``not
integer''.
\end{ex}
\begin{ex}
Write a function that accepts a \texttt{std::string}
object and returns the string with words capitalized. In other words, if
the object passed in is \texttt{std::string("hello world")}, then
\texttt{std::string"Hello World")} is returned; if the object passed in is
\texttt{std::string("ok. that's great!")}, then \texttt{std::string("Ok.
That's great!")} is returned. Assume that the only word separators are
the space, the period, and the exclamation mark.
\end{ex}
