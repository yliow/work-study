% Options for packages loaded elsewhere
\PassOptionsToPackage{unicode}{hyperref}
\PassOptionsToPackage{hyphens}{url}
%
\documentclass[
]{article}
\usepackage{amsmath,amssymb}
\usepackage{iftex}
\ifPDFTeX
  \usepackage[T1]{fontenc}
  \usepackage[utf8]{inputenc}
  \usepackage{textcomp} % provide euro and other symbols
\else % if luatex or xetex
  \usepackage{unicode-math} % this also loads fontspec
  \defaultfontfeatures{Scale=MatchLowercase}
  \defaultfontfeatures[\rmfamily]{Ligatures=TeX,Scale=1}
\fi
\usepackage{lmodern}
\ifPDFTeX\else
  % xetex/luatex font selection
\fi
% Use upquote if available, for straight quotes in verbatim environments
\IfFileExists{upquote.sty}{\usepackage{upquote}}{}
\IfFileExists{microtype.sty}{% use microtype if available
  \usepackage[]{microtype}
  \UseMicrotypeSet[protrusion]{basicmath} % disable protrusion for tt fonts
}{}
\makeatletter
\@ifundefined{KOMAClassName}{% if non-KOMA class
  \IfFileExists{parskip.sty}{%
    \usepackage{parskip}
  }{% else
    \setlength{\parindent}{0pt}
    \setlength{\parskip}{6pt plus 2pt minus 1pt}}
}{% if KOMA class
  \KOMAoptions{parskip=half}}
\makeatother
\usepackage{xcolor}
\usepackage{longtable,booktabs,array}
\usepackage{calc} % for calculating minipage widths
% Correct order of tables after \paragraph or \subparagraph
\usepackage{etoolbox}
\makeatletter
\patchcmd\longtable{\par}{\if@noskipsec\mbox{}\fi\par}{}{}
\makeatother
% Allow footnotes in longtable head/foot
\IfFileExists{footnotehyper.sty}{\usepackage{footnotehyper}}{\usepackage{footnote}}
\makesavenoteenv{longtable}
\setlength{\emergencystretch}{3em} % prevent overfull lines
\providecommand{\tightlist}{%
  \setlength{\itemsep}{0pt}\setlength{\parskip}{0pt}}
\setcounter{secnumdepth}{-\maxdimen} % remove section numbering
\ifLuaTeX
  \usepackage{selnolig}  % disable illegal ligatures
\fi
\IfFileExists{bookmark.sty}{\usepackage{bookmark}}{\usepackage{hyperref}}
\IfFileExists{xurl.sty}{\usepackage{xurl}}{} % add URL line breaks if available
\urlstyle{same}
\hypersetup{
  hidelinks,
  pdfcreator={LaTeX via pandoc}}

\author{}
\date{}

\begin{document}

82. Abstract classes

Objectives

\begin{itemize}
\tightlist
\item
  Understand the purpose of a pure virtual function
\item
  Write pure virtual function
\item
  Understand the purpose of abstract classes
\item
  Write and use abstract classes
\end{itemize}

Abstract base classes

\begin{quote}
\end{quote}

\begin{quote}
Abstract classes are abstract in the sense that they do not create
objects. The actual objects are created in a nonabstract subclass of the
abstract class. Members variables are declared and methods defined in
the subclass.
\end{quote}

\begin{quote}
\end{quote}

\begin{quote}
They are used for specifying an \textbf{interface}. An interface is the
collection of signatures of all methods of the class. The
\textbf{signature} of a function or method is made up of the name and
parameter types. The interface of a class tells you what an object of
the class is capable of doing.
\end{quote}

\begin{quote}
\end{quote}

\begin{quote}
Allows software designers to design an abstract interface without
exposing implementation details.
\end{quote}

\begin{quote}
\end{quote}

\begin{quote}
Here are some examples
\end{quote}

\begin{quote}
\end{quote}

\begin{quote}
Abstract base Subclasses
\end{quote}

\begin{quote}
Shape Circle, Square, \ldots{}
\end{quote}

\begin{quote}
HardDrive WesternDigital, Seagate, \ldots{}
\end{quote}

\begin{quote}
\end{quote}

\begin{quote}
You create an abstract class by defining a class with \textbf{at least
one} pure virtual function.
\end{quote}

\begin{quote}
\end{quote}

\begin{quote}
A \textbf{pure virtual function} is a method in a class that is declared
but not defined (i.e., has prototype but no body). To tell C++ that a
virtual method is pure, just attach \textbf{``= 0''} to the right of the
method.
\end{quote}

\begin{longtable}[]{@{}
  >{\raggedright\arraybackslash}p{(\columnwidth - 0\tabcolsep) * \real{1.0000}}@{}}
\toprule\noalign{}
 \\
\midrule\noalign{}
\endhead
\bottomrule\noalign{}
\endlastfoot
\begin{quote}
\#include \textless iostream\textgreater{}
\end{quote}

\begin{quote}
\end{quote}

\begin{quote}
class P
\end{quote}

\begin{quote}
\{
\end{quote}

\begin{quote}
public:
\end{quote}

\begin{quote}
\textbf{ virtual} void f() \textbf{= 0};
\end{quote}

\begin{quote}
private:
\end{quote}

\begin{quote}
\textbf{ }int x;
\end{quote}

\begin{quote}
\};
\end{quote}

\begin{quote}
\end{quote}

\begin{quote}
int main()
\end{quote}

\begin{quote}
\{
\end{quote}

\begin{quote}
P p;
\end{quote}

\begin{quote}
return 0;
\end{quote}

\begin{quote}
\}
\end{quote} \\
\end{longtable}

\begin{quote}
\end{quote}

\begin{quote}
A class is abstract if there is \textbf{at least one} pure virtual
function.
\end{quote}

\begin{quote}
Suppose C inherits P, and P is abstract. If C does not define (i.e. give
bodies to) all the pure virtual functions of P, then C is still
abstract.
\end{quote}

\begin{quote}
\end{quote}

\begin{quote}
Pure virtual functions are virtual.
\end{quote}

\begin{quote}
\end{quote}

\begin{quote}
If objects can be instantiated from a class, the class is said to be
\textbf{concrete} (i.e, not pure virtual functions).
\end{quote}

\begin{longtable}[]{@{}
  >{\raggedright\arraybackslash}p{(\columnwidth - 0\tabcolsep) * \real{1.0000}}@{}}
\toprule\noalign{}
 \\
\midrule\noalign{}
\endhead
\bottomrule\noalign{}
\endlastfoot
\begin{quote}
\#include \textless iostream\textgreater{}
\end{quote}

\begin{quote}
\end{quote}

\begin{quote}
class Shape2D
\end{quote}

\begin{quote}
\{
\end{quote}

\begin{quote}
public:
\end{quote}

\begin{quote}
virtual void draw() const = 0;
\end{quote}

\begin{quote}
virtual void rotate(int angle) = 0;
\end{quote}

\begin{quote}
\};
\end{quote}

\begin{quote}
\end{quote}

\begin{quote}
class Circle: public Shape2D
\end{quote}

\begin{quote}
\{
\end{quote}

\begin{quote}
public:
\end{quote}

\begin{quote}
Circle(int x0, int y0, int r0)
\end{quote}

\begin{quote}
: x(x0),y(y0),r(r0)
\end{quote}

\begin{quote}
\{\}
\end{quote}

\begin{quote}
void draw() const;
\end{quote}

\begin{quote}
void rotate(int angle);
\end{quote}

\begin{quote}
private:
\end{quote}

\begin{quote}
int x, y, r;
\end{quote}

\begin{quote}
\};
\end{quote}

\begin{quote}
\end{quote}

\begin{quote}
void Circle::draw() const
\end{quote}

\begin{quote}
\{
\end{quote}

\begin{quote}
std::cout \textless\textless{} "draw circle\textbackslash n";
\end{quote}

\begin{quote}
\}
\end{quote}

\begin{quote}
\end{quote}

\begin{quote}
void Circle::rotate(int angle)
\end{quote}

\begin{quote}
\{
\end{quote}

\begin{quote}
std::cout \textless\textless{} "rotate circle " \textless\textless{}
angle
\end{quote}

\begin{quote}
\textless\textless{} " deg\textbackslash n";
\end{quote}

\begin{quote}
\}
\end{quote}

\begin{quote}
\end{quote}

\begin{quote}
int main()
\end{quote}

\begin{quote}
\{\textbf{ }
\end{quote}

\begin{quote}
Shape2D * s = new Circle(0, 0, 10);
\end{quote}

\begin{quote}
\textbf{ }s-\textgreater draw();
\end{quote}

\begin{quote}
s-\textgreater rotate(10);
\end{quote}

\begin{quote}
return 0;
\end{quote}

\begin{quote}
\}
\end{quote} \\
\end{longtable}

Note that although you cannot create an object using an abstract class,
i.e., the following is an error:

\begin{longtable}[]{@{}
  >{\raggedright\arraybackslash}p{(\columnwidth - 0\tabcolsep) * \real{1.0000}}@{}}
\toprule\noalign{}
 \\
\midrule\noalign{}
\endhead
\bottomrule\noalign{}
\endlastfoot
\begin{quote}
class P
\end{quote}

\begin{quote}
\{
\end{quote}

\begin{quote}
public:
\end{quote}

\begin{quote}
P(int x)
\end{quote}

\begin{quote}
: x\_(x)
\end{quote}

\begin{quote}
\{\}
\end{quote}

\begin{quote}
\textbf{ }virtual void f() = 0;
\end{quote}

\begin{quote}
\};
\end{quote}

\begin{quote}
\end{quote}

\begin{quote}
int main()
\end{quote}

\begin{quote}
\{
\end{quote}

\begin{quote}
P p(42); // WRONG!!! ERROR!!! BAD!!!
\end{quote}

\begin{quote}
return 0;
\end{quote}

\begin{quote}
\}
\end{quote} \\
\end{longtable}

\begin{quote}
Nonetheless a subclass of P can call the constructor of P:
\end{quote}

\begin{longtable}[]{@{}
  >{\raggedright\arraybackslash}p{(\columnwidth - 0\tabcolsep) * \real{1.0000}}@{}}
\toprule\noalign{}
 \\
\midrule\noalign{}
\endhead
\bottomrule\noalign{}
\endlastfoot
\begin{quote}
class P
\end{quote}

\begin{quote}
\{
\end{quote}

\begin{quote}
public:
\end{quote}

\begin{quote}
P(int x)
\end{quote}

\begin{quote}
: x\_(x)
\end{quote}

\begin{quote}
\{\}
\end{quote}

\begin{quote}
virtual void f() = 0;
\end{quote}

\begin{quote}
\};
\end{quote}

\begin{quote}
\end{quote}

\begin{quote}
class C: public P
\end{quote}

\begin{quote}
\{
\end{quote}

\begin{quote}
public:
\end{quote}

\begin{quote}
C(int x)
\end{quote}

\begin{quote}
: P(x) // OK!!!
\end{quote}

\begin{quote}
\{\}
\end{quote}

\begin{quote}
void f()
\end{quote}

\begin{quote}
\{\}
\end{quote}

\begin{quote}
\};
\end{quote}

\begin{quote}
\end{quote}

\begin{quote}
int main()
\end{quote}

\begin{quote}
\{
\end{quote}

\begin{quote}
C c(42);
\end{quote}

\begin{quote}
return 0;
\end{quote}

\begin{quote}
\}
\end{quote} \\
\end{longtable}

\begin{quote}
\textbf{Design}
\end{quote}

\begin{quote}
\end{quote}

\begin{quote}
VERY IMPORTANT!!! LISTEN!!! Notice the way I wrote the program
\end{quote}

\begin{quote}
\end{quote}

\begin{quote}
The abstract class defines what objects of the subclasses \textbf{can}
do. But in \emph{main()}, I only call methods \textbf{in the base
class}.
\end{quote}

\begin{quote}
\end{quote}

\begin{quote}
I let \textbf{polymorphism} decide which \textbf{specific method} to
invoke during runtime.
\end{quote}

\begin{quote}
\end{quote}

\begin{quote}
In \emph{main()}, I try my best to work with only the methods declared
in the abstract base.
\end{quote}

\begin{quote}
Suppose I'm writing a game. There are many \emph{Shape2D} pointers to
concrete objects created. I keep the pointers in a list. I can use a
\emph{for}-loop to run through the pointers to get each to draw the
object its pointing to. If you want to add more new type of objects to
the game, just create a new concrete subclass from the abstract base.
\end{quote}

\begin{quote}
\end{quote}

\begin{quote}
My \emph{main()} code \textbf{does not change}. That's the point.
\end{quote}

\begin{quote}
So first design your program by deciding \textbf{abstractly} what all
objects in the concrete classes should ``do'' (their methods). Design
the abstract base (interface) based on the above information. This
becomes a contract for subclasses. Write your main code using only the
abstract interface. Create concrete subclasses one at a time and test as
you go along. Once the main code is done, you can ``plug'' in new
subclasses without too many changes to the \emph{main()} code.
\end{quote}

\textbf{Example.} Consider a game. Here's a possible piece of code.

\begin{longtable}[]{@{}
  >{\raggedright\arraybackslash}p{(\columnwidth - 0\tabcolsep) * \real{1.0000}}@{}}
\toprule\noalign{}
 \\
\midrule\noalign{}
\endhead
\bottomrule\noalign{}
\endlastfoot
\begin{quote}
while (1)
\end{quote}

\begin{quote}
\{
\end{quote}

\begin{quote}
\textbf{ }for (int i = 0; i \textless{} obj\_arr\_size; i++)
\end{quote}

\begin{quote}
\textbf{ }\{
\end{quote}

\begin{quote}
\textbf{ }obj = obj\_arr{[}i{]};
\end{quote}

\begin{quote}
\textbf{ }obj-\textgreater clear();
\end{quote}

\begin{quote}
\textbf{ }if (obj-\textgreater hit(missile\_arr, missile\_arr\_size))
\end{quote}

\begin{quote}
\textbf{ } obj-\textgreater explode();
\end{quote}

\begin{quote}
\textbf{ }else
\end{quote}

\begin{quote}
\textbf{ } obj-\textgreater move();
\end{quote}

\begin{quote}
\textbf{ }obj-\textgreater draw();
\end{quote}

\begin{quote}
\textbf{ }\}
\end{quote}

\begin{quote}
\}
\end{quote} \\
\end{longtable}

\begin{quote}
Notice that if I need to add another kind of \emph{obj} to the game I
just need to make sure that the new class supports
\end{quote}

\begin{itemize}
\item
  \begin{quote}
  clear
  \end{quote}
\end{itemize}

\begin{itemize}
\item
  \begin{quote}
  hit
  \end{quote}
\item
  \begin{quote}
  explode
  \end{quote}
\item
  \begin{quote}
  move
  \end{quote}
\item
  \begin{quote}
  draw
  \end{quote}

  \begin{quote}
  That's one of the powers of OO.
  \end{quote}
\end{itemize}

\begin{quote}
\end{quote}

\textbf{Exercise.} You know that in C/C++, an array is homogeneous,
i.e., an array can only hold values of one type. An \emph{int} array can
only hold integer values. A \emph{double} array can only hold double
values. We have a \emph{Rational} class. A \emph{Rational} array can
only hold \emph{Rational} values. Create an array that contains 3 values
of different types so that you can do the following:

\begin{longtable}[]{@{}
  >{\raggedright\arraybackslash}p{(\columnwidth - 0\tabcolsep) * \real{1.0000}}@{}}
\toprule\noalign{}
 \\
\midrule\noalign{}
\endhead
\bottomrule\noalign{}
\endlastfoot
\begin{quote}
std::vector\textless{} ??? \textgreater{} v;
\end{quote}

\begin{quote}
// put something into v{[}0{]}, v{[}1{]}, v{[}2{]}
\end{quote}

\begin{quote}
for (int i = 0; i \textless{} 3; ++i)
\end{quote}

\begin{quote}
\{
\end{quote}

\begin{quote}
std::cout \textless\textless{} {[}... something to do with v{[}i{]}
...{]}
\end{quote}

\begin{quote}
\textless\textless{} \textquotesingle{} \textquotesingle;
\end{quote}

\begin{quote}
\}
\end{quote}

\begin{quote}
// Assume v{[}0{]} models 42, v{[}1{]} models 3.14, v{[}2{]}
\end{quote}

\begin{quote}
// models 1/2.
\end{quote}

\begin{quote}
std::cout \textless\textless{}
\textquotesingle\textbackslash n\textquotesingle;
\end{quote} \\
\end{longtable}

\begin{quote}
so that you can this output:
\end{quote}

\begin{longtable}[]{@{}
  >{\raggedright\arraybackslash}p{(\columnwidth - 0\tabcolsep) * \real{1.0000}}@{}}
\toprule\noalign{}
 \\
\midrule\noalign{}
\endhead
\bottomrule\noalign{}
\endlastfoot
\begin{quote}
42 3.14 1/2
\end{quote} \\
\end{longtable}

\begin{quote}
\end{quote}

\begin{quote}
\end{quote}

\begin{quote}
\textbf{Exercise.} Continuing the above, write a function that
increments the values in the values in the above \emph{std::vector} by
1.
\end{quote}

\begin{longtable}[]{@{}
  >{\raggedright\arraybackslash}p{(\columnwidth - 0\tabcolsep) * \real{1.0000}}@{}}
\toprule\noalign{}
 \\
\midrule\noalign{}
\endhead
\bottomrule\noalign{}
\endlastfoot
\begin{quote}
std::vector\textless{} ??? \textgreater{} v;
\end{quote}

\begin{quote}
// put something into v{[}0{]}, v{[}1{]}, v{[}2{]}
\end{quote}

\begin{quote}
\end{quote}

\begin{quote}
for (int i = 0; i \textless{} 3; ++i)
\end{quote}

\begin{quote}
\{
\end{quote}

\begin{quote}
std::cout \textless\textless{} {[}... something to do with v{[}i{]}
...{]}
\end{quote}

\begin{quote}
\textless\textless{} \textquotesingle{} \textquotesingle;
\end{quote}

\begin{quote}
\}
\end{quote}

\begin{quote}
std::cout \textless\textless{}
\textquotesingle\textbackslash n\textquotesingle;
\end{quote}

\begin{quote}
increment(v);
\end{quote}

\begin{quote}
for (int i = 0; i \textless{} 3; ++i)
\end{quote}

\begin{quote}
\{
\end{quote}

\begin{quote}
std::cout \textless\textless{} {[}... something to do with v{[}i{]}
...{]}
\end{quote}

\begin{quote}
\textless\textless{} \textquotesingle{} \textquotesingle;
\end{quote}

\begin{quote}
\}
\end{quote}

\begin{quote}
std::cout \textless\textless{}
\textquotesingle\textbackslash n\textquotesingle;
\end{quote} \\
\end{longtable}

\begin{quote}
so that you get this output:
\end{quote}

\begin{longtable}[]{@{}
  >{\raggedright\arraybackslash}p{(\columnwidth - 0\tabcolsep) * \real{1.0000}}@{}}
\toprule\noalign{}
 \\
\midrule\noalign{}
\endhead
\bottomrule\noalign{}
\endlastfoot
\begin{quote}
42 3.14 1/2
\end{quote}

\begin{quote}
43 4.14 3/1
\end{quote} \\
\end{longtable}

\begin{quote}
\end{quote}

\begin{quote}
\end{quote}

\begin{quote}
The above exercise is very important. In your programs that you have
been writing (since CISS240), you work with integers and doubles more or
less separately, mixing them only when necessary. In fact integers are
collected together into the \emph{int} type. Doubles are collected
together into the \emph{double} type. This exercise says if you know
enough C/C++, you can treat then on equal footing and they can be mixed
together. When do you need to know this? Well \ldots{} when you write
more complex software such as a compiler, your integers and doubles
appear together in the same string as C/C++ program. And you compiler
(g++ or some other compiler) actually has to handle these values in a
``uniform way''. In other words your compiler has lots of algorithms
that does not differentiate between \emph{int} and \emph{double} values.
(Of course some do.)
\end{quote}

\begin{quote}
\end{quote}

\begin{quote}
Note that you can create such a \emph{std::vector\textless{} T
\textgreater{}} of different types \emph{T}. But
\emph{std::vector\textless{} int \textgreater{}},
\emph{std::vector\textless{} double \textgreater{}},
\emph{std::vector\textless{} Rational \textgreater{}} are all separate
things. And sure, you can write a function template to print values of
\emph{std::vector\textless{} T \textgreater{}}. But the function to
print \emph{std::vector\textless{} int \textgreater{}} that is
instantiated is different from the function to print
\emph{std::vector\textless{} double \textgreater{}}.
\end{quote}

\begin{quote}
\end{quote}

\begin{quote}
So the way in which template classes/functions/structs handle multiple
types is \textbf{very different} from the way in which abstract base
classes and polymorphism handle problems involving multiple types.
\end{quote}

\begin{quote}
\end{quote}

\begin{quote}
\end{quote}

\begin{quote}
\textbf{Exercise.} Continuing the above, let you increment function
allow you to specify the increment amount of \emph{int} or \emph{double}
or \emph{Rational} type:
\end{quote}

\begin{longtable}[]{@{}
  >{\raggedright\arraybackslash}p{(\columnwidth - 0\tabcolsep) * \real{1.0000}}@{}}
\toprule\noalign{}
 \\
\midrule\noalign{}
\endhead
\bottomrule\noalign{}
\endlastfoot
\begin{quote}
std::vector\textless{} ??? \textgreater{} v;
\end{quote}

\begin{quote}
// put something into v{[}0{]}, v{[}1{]}, v{[}2{]}
\end{quote}

\begin{quote}
for (int i = 0; i \textless{} 3; ++i)
\end{quote}

\begin{quote}
\{
\end{quote}

\begin{quote}
std::cout \textless\textless{} {[}... something to do with v{[}i{]}
...{]}
\end{quote}

\begin{quote}
\textless\textless{} \textquotesingle{} \textquotesingle;
\end{quote}

\begin{quote}
\}
\end{quote}

\begin{quote}
std::cout \textless\textless{}
\textquotesingle\textbackslash n\textquotesingle;
\end{quote}

\begin{quote}
increment(v, 1);
\end{quote}

\begin{quote}
increment(v, 0.1);
\end{quote}

\begin{quote}
increment(v, Rational(1, 3));
\end{quote} \\
\end{longtable}

\begin{quote}
(What should a \emph{double} be after you add 1/3 to it? I'm not talking
about the value. I'm talking about the type.) The above description is a
skeleton and might require quite a few changes here and there.
\end{quote}

\begin{quote}
\end{quote}

\begin{quote}
\end{quote}

\begin{quote}
\textbf{Exercise. }You can given three classes:
\end{quote}

\begin{longtable}[]{@{}
  >{\raggedright\arraybackslash}p{(\columnwidth - 0\tabcolsep) * \real{1.0000}}@{}}
\toprule\noalign{}
 \\
\midrule\noalign{}
\endhead
\bottomrule\noalign{}
\endlastfoot
\begin{quote}
Scissors
\end{quote}

\begin{quote}
\{\};
\end{quote}

\begin{quote}
Paper
\end{quote}

\begin{quote}
\{\};
\end{quote}

\begin{quote}
Stone
\end{quote}

\begin{quote}
\{\};
\end{quote} \\
\end{longtable}

\begin{quote}
When you print a \emph{Scissors} object, you see \emph{scissors. }When
you print a \emph{Paper} object, you see \emph{paper. }When you print a
\emph{Stone} scissors object, you see \emph{stone}. Create a
\emph{std::vector} of 3 values so that when you print them, you see
\end{quote}

\begin{longtable}[]{@{}
  >{\raggedright\arraybackslash}p{(\columnwidth - 0\tabcolsep) * \real{1.0000}}@{}}
\toprule\noalign{}
 \\
\midrule\noalign{}
\endhead
\bottomrule\noalign{}
\endlastfoot
\begin{quote}
paper scissors stone
\end{quote} \\
\end{longtable}

\begin{quote}
In a loop, let the first two values compete. paper scissors beats paper,
etc. Remove the loser from the vector. If the two values are the same,
that's a draw, and you keep both. Move on to the next two and let them
compete. For the above you get
\end{quote}

\begin{longtable}[]{@{}
  >{\raggedright\arraybackslash}p{(\columnwidth - 0\tabcolsep) * \real{1.0000}}@{}}
\toprule\noalign{}
 \\
\midrule\noalign{}
\endhead
\bottomrule\noalign{}
\endlastfoot
\begin{quote}
paper scissors stone
\end{quote}

\begin{quote}
scissors stone
\end{quote}

\begin{quote}
stone
\end{quote} \\
\end{longtable}

\begin{quote}
If you have the following initialization
\end{quote}

\begin{longtable}[]{@{}
  >{\raggedright\arraybackslash}p{(\columnwidth - 0\tabcolsep) * \real{1.0000}}@{}}
\toprule\noalign{}
 \\
\midrule\noalign{}
\endhead
\bottomrule\noalign{}
\endlastfoot
\begin{quote}
paper paper scissors stone paper stone stone
\end{quote} \\
\end{longtable}

\begin{quote}
you will get the following when they compete:
\end{quote}

\begin{longtable}[]{@{}
  >{\raggedright\arraybackslash}p{(\columnwidth - 0\tabcolsep) * \real{1.0000}}@{}}
\toprule\noalign{}
 \\
\midrule\noalign{}
\endhead
\bottomrule\noalign{}
\endlastfoot
\begin{quote}
paper paper scissors stone paper stone stone
\end{quote}

\begin{quote}
paper paper scissors stone paper stone stone
\end{quote}

\begin{quote}
paper scissors stone paper stone stone
\end{quote}

\begin{quote}
paper stone paper stone stone
\end{quote}

\begin{quote}
paper paper stone stone
\end{quote}

\begin{quote}
paper paper stone
\end{quote}

\begin{quote}
paper paper
\end{quote} \\
\end{longtable}

\begin{quote}
You may add methods to the classes. The above description is a skeleton
and might require quite a few changes here and there.
\end{quote}

\begin{quote}
\end{quote}

\begin{quote}
\end{quote}

\begin{quote}
Exercise. Write enough classes to achieve the following.
\end{quote}

\begin{longtable}[]{@{}
  >{\raggedright\arraybackslash}p{(\columnwidth - 0\tabcolsep) * \real{1.0000}}@{}}
\toprule\noalign{}
 \\
\midrule\noalign{}
\endhead
\bottomrule\noalign{}
\endlastfoot
\begin{quote}
SineFunction s; // s models sin(x)
\end{quote}

\begin{quote}
CosineFunction c; // c models cos(x)
\end{quote}

\begin{quote}
MonomialFunction x(1); // x models x
\end{quote}

\begin{quote}
MonomialFunction x2(2); // x2 models x\^{}2
\end{quote}

\begin{quote}
ConstFunction c(1); // c models constant function 1
\end{quote}

\begin{quote}
\end{quote}

\begin{quote}
std::cout \textless\textless{} s \textless\textless{} \textquotesingle{}
\textquotesingle{} \textless\textless{} c \textless\textless{}
\textquotesingle{} \textquotesingle{} \textless\textless{} x
\textquotesingle{} \textquotesingle{} \textless\textless{} x2
\end{quote}

\begin{quote}
\textless\textless{} \textquotesingle{} \textquotesingle{}
\textless\textless{} c \textquotesingle\textbackslash n\textquotesingle;
\end{quote}

\begin{quote}
// prints sin(x) cos(x) x x\^{}2 1
\end{quote}

\begin{quote}
\end{quote}

\begin{quote}
SumFunction f(x, s); // f models x + sin(x)
\end{quote}

\begin{quote}
std::cout \textless\textless{} f \textless\textless{}
\textquotesingle\textbackslash n\textquotesingle; // prints x + sin(x)
\end{quote}

\begin{quote}
ProdFunction g(x, s); // g models x * sin(x)
\end{quote}

\begin{quote}
std::cout \textless\textless{} g \textless\textless{}
\textquotesingle\textbackslash n\textquotesingle; // prints 1 * sin(x) +
x * cos(x)
\end{quote}

\begin{quote}
\end{quote}

\begin{quote}
std::cout \textless\textless{} Derivative(x) \textless\textless{}
\textquotesingle\textbackslash n\textquotesingle; // prints 1
\end{quote}

\begin{quote}
std::cout \textless\textless{} Derivative(x2) \textless\textless{}
\textquotesingle\textbackslash n\textquotesingle; // prints 2 * x
\end{quote}

\begin{quote}
std::cout \textless\textless{} Derivative(s) \textless\textless{}
\textquotesingle\textbackslash n\textquotesingle; // prints cos(x)
\end{quote}

\begin{quote}
std::cout \textless\textless{} Derivative(f) \textless\textless{}
\textquotesingle\textbackslash n\textquotesingle; // prints 1 + cos(x)
\end{quote} \\
\end{longtable}

\begin{quote}
The above is a skeleton and might require quite a few changes here and
there. (Can you do \emph{SumFunction f(x, SumFunction(s, c))}?)
\end{quote}

\begin{quote}
\end{quote}

\end{document}
