\tikzset{every picture/.append style={remember picture}}
\EMPHASIZE{07.\texttt{if} and \texttt{if-else}: Part 1}

\textbf{Objectives}
\begin{tightlist}
\li
  Write if statements (including blocks)
\li
  Write if-else statements (including blocks)
\li
  Write nested if-else statements
\end{tightlist}

We will now talk about writing statements that make decisions. In other
words, statements that decide whether to run other statements or not.

This is significant \dots we are now embarking on writing program
with intelligence \dots

ONWARD!!!

\newpage\EMPHASIZE{\texttt{if} statements}

So far you know that C++ can do arithmetic, print integer values, print
string values, accept integer values from the keyboards, create and
update variables.

You should expect more! (Otherwise C++ is pretty dumb.)

For instance you would expect programs to make decisions. Now
wouldn' t it be nice if C++ can do this:
\begin{align*}
&\texttt{if alien is killed:}\\
& \tab[3em]{\texttt{score = score + 10}}\\
\end{align*}
or
\begin{align*}
&\texttt{if my ship' s energy is 0:}\\
& \tab[3em]{\texttt{end the game}}\\
\end{align*}
or
\begin{align*}
&\texttt{if plane' s left engine has failed and }\\
&\texttt{plane' s right engine has failed:}\\
& \tab[3em]{\texttt{activate seat ejection!!!}}\\
\end{align*}
Note that for the first case, if the alien is not killed, you do
\textit{not} want to execute the statement \verb!score = score + 10!.
Otherwise everyone would keeping getting the same score! What kind of
dumb game is that?

In fact C++ \textit{does} understand the \verb!if! statement.

Run this program\\
\begin{console}
int x = 0;
std::cin >> x;
if (x > 0)
       std::cout << "spam!" << std::endl;
std::cout << "eggs" << std::endl;
\end{console}

Run the program several time, entering different values for x. Note that
the statement printing \verb@"spam!"@ is controlled by the \verb!if!.
However the statement that prints \EMPHASIZE{"eggs"} is not.

For readability, the if statement is written like this:\\
\begin{console}[commandchars=\~\@\$]
int x = 0;
std::cin >> x;
~textbf@if (x > 0)$
   ~textbf@std::cout << "spam!" << std::endl;$
std::cout << "eggs" << std::endl;
\end{console}
Big picture: The format of the \texttt{if} statement looks like this
\begin{center}
\EMPHASIZE{\textbf{if (}\textbf{[bool expr]}\textbf{)}}

\EMPHASIZE{[stmt]}
\end{center}
where [\verb!bool expr!] is a \EMPHASIZE{boolean expression} -- this
means that it's an expression that can be evaluated to a
boolean value -- and [\verb!stmt!] is a C++ statement (of course
there's a semi-colon at the end). Since we need boolean
expressions it's a good time now to recall the
following:
\begin{align*}
 & \texttt{!} & \text{"not"}\\
 & \texttt{\&\&} & \text{"and"}\\
 & \texttt{\textbar\textbar{}} & \text{"or"}\\
 & \texttt{==} & \text{"is equal to"}\\
 & \texttt{!=} & \text{"is not equal to"}\\
 & \texttt{\textless{}} & \text{"is less than"}\\
 & \texttt{\textless=} & \text{"is less than or equal to"}\\
 & \texttt{>} & \text{"is greater than"}\\
 & \texttt{>=} & \text{"is greater than or equal to"}\\
\end{align*}
An example of a boolean expression is

\EMPHASIZE{[variable] [bool op] [value]}

where [\verb!variable!] is a variable of int or double or float type, [\verb!bool op!] is one of the above boolean operators and [\verb!value!] is
an int or double or float. Don' t forget that if there is
a type mismatch, C++ will try to type promote one value to match the
type of the other. For instance if \verb!x! is an \verb!int! variable
and the boolean expression is

\tab[3em]{\verb!x < 1.23!}

then C++ will use the \verb!double! of \verb!x! (i.e., \verb!x! is
automatically promoted to a \verb!double!):

\tab[3em]{\verb!double(x) < 1.23!}

\textbf{Exercise.} Modify the above program so that \verb~"spam!"~ is printed
when the integer entered is 42. Test your program.

\textbf{Exercise.} Modify the above program so that the program prompts the user
for his/her favorite number and print \lq\lq you don' t like
42?!?'' if the integer entered is not 42. Here' s an execution:
\begin{console}[commandchars=\~\!\@]
what' s your favorite number? ~textbf!42@ 
\end{console}
Here' s another:
\begin{console}[commandchars=\~\@\$]
what' s your favorite number? ~textbf@1$
~textbf@you don't like 42?!?$  
\end{console}

\textbf{Exercise.} Write a program that prompts the user for an integer
and prints \lq\lq That' s even. Am I smart?'' if the integer
entered is even. Here' s an execution
\begin{console}[commandchars=\~\!\@]
~textbf!100@

That' s even. Am I smart?
\end{console}

Here' s another
\begin{console}
101 
\end{console}

\textbf{Exercise.} Write a program that prompts the user for his/her age
and prints \lq\lq You' re lying! Think you can fool me???'' if
the value entered is less than 18. Here' s an execution:
\begin{console}[commandchars=\~\@\$]
Enter your age: ~textbf@16$
You' re lying! Think you can fool me???
\end{console}
Here' s another:

\begin{console}[commandchars=\~\!\@]
Enter your age: textbf!100@
\end{console}
Of course this is another boolean expression:
\begin{center}
\EMPHASIZE{[var1] [bool op] [var2]}
\end{center}
In this case you' re comparing two (integer or double)
variables.

\textbf{Exercise.} Write a program that prompts the user for two integer
values and prints \lq\lq First is bigger'' if the first integer entered is
larger than the second. Here' s an execution of the
program:
\begin{console}[commandchars=\~\!\@]
~textbf!1 2@
\end{console}
Here' s another

\begin{console}[commandchars=\~\!\@]
~textbf!2 -1@ 
First is bigger
\end{console}

Here' s yet another boolean expression:

\begin{center}
\EMPHASIZE{[expr] [bool op] [value]}
\end{center}

where \verb![expr]! is an (integer or double) expression.

\textbf{Exercise.} Write a program that prompts the user for his/her
height (in ft) and weight (in lbs) and prints \lq\lq you will live more than
100 years!!!'' if the product of the height and weight is at least
200.45. (This is not supported by any form of research.)
Here' s an execution of the program:
\begin{console}[commandchars=\~\!\@]
~textbf!6.1 180.180@
\end{console}
\begin{console}
you will live more than 100 years!!!!
\end{console}
Here' s another
\begin{console}[commandchars=\~\@\$]
~textbf@3.4 23.4$
\end{console}

\textbf{Exercise.} Write a program that prompts the user for the year of
his/her date of birth and the current year and prints his/her
approximate age and then prints \lq\lq you must have watched lots of movies
by now'' if the difference of the two values is at least 20.
Here' s an execution:
\begin{console}[commandchars=\~\@\$]
~textbf@1900 2000$
you are about 100 years old
you must have watched lots of movies by now
\end{console}

Here' s another:
\begin{console}[commandchars=\~\!\@]
~textbf!2000 2006@
you are about 6 years old
\end{console}


One last one ... here' s one more boolean expression:
\begin{center}
\EMPHASIZE{[expr1] [bool op] [expr2]}
\end{center}

Now, we're comparing two expressions.

\textbf{Exercise.} Write a program that prompts the user for the month,
day of his/her date of birth and also his/her height (in ft) and weight
(in lbs). Print the approximate number of days from January 1 of the
year he/she was born and the sum of his/her height and weight. If the
first printed number is greater than the second, print \lq\lq According to
the theory of relativity, you are going to win the next powerball.''.
\begin{console}[commandchars=\~\!\@]
Enter the month, day of your DOB: ~textbf!1 1@
Enter your height and weight: ~textbf!6 100@
1 106
\end{console}
Here' s another:
\begin{console}[commandchars=\~\!\@]
Enter the month, day of your DOB: ~textbf!12 25@
Enter your height and weight: ~textbf!6 200@
355 206
According to the theory of relativity,
you are going to win the next powerball.
\end{console}
(You may assume the number of days in a month is 30.)

\textbf{Exercise.}Try this:\\
\begin{console}[commandchars=\~\!\@]
int numHeads = 0;
std::cout << "How many heads do you have? ";
std::cin >> numHeads;
if (numHeads == 2)
   std::cout << "Are you Zaphod?" << std::endl;
\end{console}

And then this:

\begin{console}
int numHeads = 0;
std::cout << "How many heads do you have? ";
std::cin >> numHeads;
if (numHeads != 2)
   std::cout << "You are not Zaphod" << std::endl;
\end{console}

Combine both into one program so that I get the following execution:\\
\begin{console}[commandchars=\~\!\@]
How many heads do you have?~textbf!1@
You are not Zaphod
\end{console}
And here's an execution of the same program:\\
\begin{console}[commandchars=\~\!\@]
How many heads do you have?~textbf!2@
Are you Zaphod?
\end{console}
Next modify the program so that, in addition to the above requirements,
if the number of heads entered is greater than 2, the program prints "No
kidding!". Finally, modify the program so that, in addition to the
above, if the number of heads is 1, the program prints "Let me introduce
you to our surgeons." Test your program by entering 0, 1, 2, 3.

(There' s a better way to do this exercise ...)

\newpage\EMPHASIZE{Mental picture: flow of execution for \texttt{if}}

The \verb!if! statement is difficult for some beginning programmers
because it alters the flow of execution of a program.

Before this set of notes, C++ executes one statement at a time from top
to bottom; it executes every statement.

The \verb!if! statement is different because it has a statement
\EMPHASIZE{inside} the \verb!if! statement.
%ISSUE this text alignment in T1 needs to be corrected, i also need to figure out how to mess with font types/sizing
\begin{python}
from latextool_basic import *
p = Plot()
M = r'''
int x = 0;
std::cin >> x;
                                                                          
if (x > 0)
    std::cout << "spam!" << std::endl;

std::cout << "eggs" << std::endl;
'''.strip()

B0 = Rect(0, -4.5, 13, -2.5, linestyle= 'ultra thick', linecolor= 'red')
T0 = Rect(6, -1, 10.5, 0.3,  linecolor= 'red', label = 'This is an if-statement', linewidth='0.2')
L0 = Line(x0=8, y0=-1, x1=6, y1=-2.5, endstyle='>', linewidth='0.15', linecolor='red', bend_right = 30)

B1 = Rect(1.3, -3.9, 12.5, -3.3, linestyle= 'ultra thick', linecolor= 'red')
T1 = Rect(12, -1, 19, 0.5, linewidth='0.2', linecolor= 'red', s = 'This is a statement inside the if-statement', font = r'\large', innersep = 0.02)
L1 = Line(x0=14.5, y0=-1, x1=12.5, y1=-3.6, endstyle='>', linewidth='0.15', linecolor='red', bend_left = 40)

p += B0
p += T0
p += L0
p += B1
p += T1
p += L1
code(p, M, x=0, y=0, height=0.8, width=0.33, border_linewidth=0.01, innersep=0.8)
print(p)
\end{python}


The statement in the \verb!if! statement is executed only when the
\verb!x > 0! is true. If the condition is not true, the
statement in the \verb!if! statement is not executed.

In the following diagram, you can follow the arrows to get a sense of
the flow of execution:
%ISSUE fix the coordinate plots for the true/false flowchart labels
\begin{python}
from latextool_basic import *
p = Plot()
M = r'''
int x = 0;

std::cin >> x;
                                                                          
if (x > 0)

   std::cout << "spam!" << std::endl;   

std::cout << "eggs" << std::endl;
'''.strip()

points = [(-1.5, 0.72), (-1.5, -0.72), (13.7, -0.72), (13.7, -2.1), (-1.5, -2.1), (-1.5, -3.72), (13.7, -3.72), (13.7, -5.1), (-1.5, -5.1), (-1.5, -6.3), (-0.8, -6.3), (13, -9.7), (13.7, -9.7), (13.7, -11.72), (-1.5, -11.72), (-1.5, -12.72), (-0.8, -12.72)]

p += Line(points = [(-1.5, 0.72), (-1.5, -0.72)], linewidth = 0.1, linecolor='red')
p += Line(points = [(-1.5, -0.72), (-0.8, -0.72)], linewidth = 0.1, endstyle='>', linecolor='red')
p += Line(points = [(13, -0.72), (13.7, -0.72)], linewidth = 0.1, linecolor='red')
p += Line(points = [(13.7, -0.72), (13.7, -2.1)], linewidth = 0.1, linecolor='red')
p += Line(points = [(13.7, -2.1), (-1.5, -2.1)], linewidth = 0.1, linecolor='red')
p += Line(points = [(-1.5, -2.1), (-1.5, -3.72)], linewidth = 0.1, linecolor='red')
p += Line(points = [(-1.5, -3.72), (-0.8, -3.72)], linewidth = 0.1, endstyle='>', linecolor='red')
p += Line(points = [(13, -3.72), (13.7, -3.72)], linewidth = 0.1, linecolor='red')
p += Line(points = [(13.7, -3.72), (13.7, -5.1)], linewidth = 0.1, linecolor='red')
p += Line(points = [(13.7, -5.1), (-1.5, -5.1)], linewidth = 0.1, linecolor='red')
p += Line(points = [(-1.5, -5.1), (-1.5, -6.3)], linewidth = 0.1, linecolor='red')
p += Line(points = [(-1.5, -6.3),(-0.8, -6.3)], linewidth = 0.1, endstyle='>', linecolor='red')
p += Line(points = [(13, -9), (13.7, -9)], linewidth = 0.1, linecolor='red')
p += Line(points = [(13.7, -9), (13.7, -11.72)], linewidth = 0.1, linecolor='red')
p += Line(points = [(13.7, -11.72), (-1.5, -11.72)], linewidth = 0.1, linecolor='red')
p += Line(points = [(-1.5, -11.72), (-1.5, -12.72)], linewidth = 0.1, linecolor='red')
p += Line(points = [(-1.5, -12.72), (-0.8, -12.72)], linewidth = 0.1, endstyle='>', linecolor='red')

p += Line(points = [(2.5, -7.22), (2.5, -8.22)], linewidth=0.1, linecolor='red')
p += Line(points = [(-0.1, -8.22), (13, -8.22)], endstyle='>', linewidth=0.1, linecolor='red')
p += Line(points = [(-0.1, -8.22), (-0.1, -9.6)], linewidth=0.1, linecolor='red')
p += Line(points = [(-0.1, -9.6), (0.45, -9.6)], endstyle='>', linewidth=0.1, linecolor='red')
p += Line(points = [(12.2, -9.6), (13, -9.6)], endstyle='>', linewidth=0.1, linecolor='red')

p += Rect(-0.8, -1.2, 13, -0.22, linecolor='red', linewidth=0.1, name='b1')
p += Rect(-0.8, -4.2, 13, -3.22, linecolor='red', linewidth=0.1, name='b2')
p += Rect(-0.8, -11.2, 13, -5.8, linecolor='red', linewidth=0.1, name='b3')
p += Rect(0.8, -7.22, 3.5, -6.22, linecolor='red', linewidth=0.1, name='b4')
p += Rect(0.45, -10.2, 12.2, -9.22, linecolor='red', linewidth=0.1, name='b5')
p += Rect(-0.8, -13.2, 13, -12.22, linecolor='red', linewidth=0.1, name='b6')
code(p, M, x=0, y=0, height=1.5, width=0.33, border_linewidth=0.01, innersep=0.8)

print(p)
\end{python}
%% TR = coordinate(1, -7.6, label='true', anchor='left')
%% FA = coordinate(3, -7.6, label='false', anchor='right')
%% p += TR
%% p += FA

Of course the program is still executing one statement at a time from
the top to the bottom if you view the if statement as a statement (i.e.
forget about the internals of the if statement).

Gotchas

\textbf{Exercise.} Try this:\\
\begin{console}
int x = 0;
std::cin >> x;
if x > 0
   std::cout << "spam!" << std::endl;
std::cout << "eggs" << std::endl;
\end{console}
Fix it.


\textbf{Exercise.} Try this:\\
\begin{console}
int x = 0;
std::cin >> x;
if (x > 0);
   std::cout << "spam!" << std::endl;
std::cout << "eggs" << std::endl;
\end{console}
What do you learn from this example?

\textbf{Exercise.} Here' s another wacky example ...
\begin{console}
int x = 0;
std::cin >> x;
if (x > 0);
std::cout << "eggs" << std::endl;
\end{console}
Multiplication game: Part 1

\textbf{Exercise.} Write a program that tests if the user can do two digit
multiplication. Right now we will just ask the one question: the product
of 97 and 94. The program congratulates the user if the right answer is
entered. Otherwise the program stops immediately. Here' s
an execution of the program:\\
\begin{console}[commandchars=\~\!\@]
What is the product of 97 and 94? ~textbf!1@
\end{console}
Here' s another execution of the program:\\
\begin{console}[commandchars=\~\@\$]
What is the product of 97 and 94? ~textbf@9118$
You smart dawg!
\end{console}

This is kind of a dumb program because it asks the same question again
and again.

Here' s the \EMPHASIZE{pseudocode} (pseudocode = informal
description of steps in a program):
\begin{align*}
& \texttt{Declare integer variable guess}\\
& \texttt{Print a prompt string to user}\\
& \texttt{Prompt user for integer value for guess}\\
& \texttt{If guess is 9118}\\
& \tab[3em]{\texttt{print congratulatory message}}\\
\end{align*}
Note that this is not a C++ program yet!!! You can' t run
it!!!

Complete the program using the above pseudocode. Test your program.

\newpage\EMPHASIZE{Blocks}

What if you want to execute \EMPHASIZE{\textit{several}} statements when a
boolean expression is true? Here's one way:\\
\begin{console}[commandchars=\~\@\#]
int x = 0;
std::cin >> x;
if (x > 0)
   std::cout << "spam!" << std::endl;
if (x > 0)
   std::cout << "ham!" << std::endl;
std::cout << "eggs" << std::endl;
\end{console}

It does work \dots but \dots that's the \EMPHASIZE{BAD}
way of executing two statements when a condition holds.

\EMPHASIZE{\textit{This}} is the right way:
\begin{console}[commandchars=\~\@\#]
int x = 0;
std::cin >> x;
if (x > 0)
~textbf@{#
        std::cout << "spam!" << std::endl;
        std::cout << "ham!" << std::endl;
~textbf@}#
std::cout << "eggs" << std::endl;
\end{console}

The \EMPHASIZE{\{...\}} is called a \EMPHASIZE{block} or \EMPHASIZE{block of
statements. }(You' ve actually seen this already. Look at
your \texttt{main() \dots}) You can have as many statements as you like
in a block, not just two.

\textbf{Exercise.} Here' s part of a game that
you're writing \dots complete it. If the missile hits the alien, increment the \verb!score! by 100, set \verb!play_explosion! to \verb`true`, increment your \verb!energy! by the alien's energy (i.e. \verb`alien_energy`), and set the alien's energy to 0. Test your code with different inputs.
\begin{console}
int score = 1234;
int energy = 6789;
int alien_energy = 42;
bool play_explosion = false;
bool missile_hit_alien;
std::cin >> missile_hit_alien;


std::cout << "score: " << score << '\n'
          << "energy: " << energy << '\n'
          << "alien_energy: " << alien_energy << '\n'
          << "play_explosion: " << score << '\n';
\end{console}
\newpage\EMPHASIZE{Random integers}

Generating a random integer occurs commonly in computer programs. For
instance if you want to write a strategy role-playing game, you might
have different rooms in a mansion and you want to put different things
in different rooms. Suppose you have a magic potion in a bottle. Now
suppose the rooms are numbered from 1 to 1000. You don' t
want your magic potion bottle to be in room 5 whenever you start the
game. You want each game to be different.

At a more complex level you might want to randomly make doors between
two rooms. This creates a random maze.

So you' d better learn how to generate random things.

Try this:
\begin{console}[commandchars=\~\@\$]
#include < iostream>

int main()
{
std::cout << "the potion is in room " << ~textbf@rand()$
          << std::endl;
std::cout << "the potion is in room " << ~textbf@rand()$
          << std::endl;
std::cout << "the potion is in room "<< ~textbf@rand()$
          << std::endl;
std::cout << "total number of rooms: "
          << ~textbf@RAND_MAX$ << std::endl;
          
return 0;
} 
\end{console}
(Note: You might need \verb!#include <cstdlib>!\\just after \verb!#include <iostream>!. This depends on your compiler.)

Looks like you' re getting random integers. But wait ...
run the program a second time ... a third time ... a fourth time. Notice
something?

OK. Let' s modify the program:
\begin{console}[commandchars=\~\@\$]
#include <iostream>
~textbf@#include <ctime>$
int main()
{
~textbf@srand((unsigned int) time(NULL));$   
std::cout << "the potion is in room " << rand()
          << std::endl;
std::cout << "the potion is in room " << rand()
          << std::endl;
std::cout << "the potion is in room " << rand()
          << std::endl;
std::cout << "total number of rooms: "
          << RAND_MAX << std::endl;
return 0;
\end{console}

Run this program several times. Notice that the room number is different
each time you run the program.

I will not go into details on the statement
\begin{center}
\texttt{srand((unsigned int) time(NULL));}
\end{center}
The only thing I will say is that it \lq\lq seeds'' the random generator so
as to make it \lq\lq more random''. The important thing to remember is that
you must execute this statement and furthermore you execute it
\textbf{\underline{once}} (usually) and \textbf{\underline{before}} you call the \texttt{rand()} function. The standard practice is to execute the seeding
in \texttt{main()} and at the beginning:
\begin{console}[commandchars=\~\@\$]
#include <iostream>
~textbf@#include <ctime>$
int main()
{
   ~textbf@srand((unsigned int) time(NULL));$
   . . .
   return 0;
}
\end{console}
Since \verb!rand()! gives you a random number from \verb!0! to \verb!RAND_MAX!,
\begin{center}
\texttt{rand() / RAND\_MAX}
\end{center}
will give you a random \verb!double! between \verb!0.0! and \verb!1.0!.
But this is not right because the / is an \textit{\textbf{integer}}
division. Therefore we do this:
\begin{center}
\texttt{double(rand()) / RAND\_MAX}
\end{center}
to get a random double from 0.0 to 1.0.

\textbf{Exercise.} Check that I was not lying by printing 10 random
doubles from 1.0 to 0.0. By the way, if the numbers are truly random,
you would expect their average to be close to 0.5. Check that too.

There are times when you need to generate a random
\textit{\textbf{integer}} between two bounds. For instance if you have a
maze game and the rooms are numbered 1 to 100, you want a random room
number for a magic potion. How many values are there from 1 to 100? You
have 100. Since \texttt{rand()} gives you a random integer from 0 to
\texttt{RAND\_MAX}, if you do

\begin{center}
        \texttt{rand() \% 100}
\end{center}
you get 100 numbers, from 0 to 99. You add 1 to it and you get a random
integer from 1 to 100. In other words
\begin{center}
        \texttt{1 + rand() \% 100}
\end{center}

does the trick.

\textbf{Exercise.} How do you generate random integers from 1,..., 6?
(For instance to simulate a dice roll.) Verify with a C++ program.

\textbf{Exercise.} How do you generate random integers from 1,..., 10?
Verify with a C++ program.

What about generating random integers from 5,..., 20? In this case the
the lower bound is 5 and not 1. You use the same idea. Note that if you
do something like
\begin{center}
\texttt{rand() \% 10}
\end{center}
or
\begin{center}
\texttt{rand() \% 18}
\end{center}
the range of numbers starts from 0. So if you want to generate random
numbers in the range of 5,..., 20, the first thing you do is to think
about the range
\begin{center}
0, \dots, 15
\end{center}
first. In other words move the given range 5, \ldots, 20 so that the
lower bound of the range becomes 0. That means subtracting 5 from the
range. To generate a number in the range of 0, \ldots, 15, you do
\begin{center}
\texttt{rand() \% 16}
\end{center}
This gives a number in the range 0, \ldots, 15. You then add 5 to it:
\begin{center}
\texttt{5 + rand() \% 10}
\end{center}
\textbf{Exercise.} Verify that the expression\\
\begin{center}
\texttt{5 + rand() \% 10}
\end{center}
does give you numbers in the range of 5, \dots, 20. .

\textbf{Exercise.} How do you generate random integers from 9,...142?
Verify with a C++ program.

\textbf{Exercise.} How do you generate random integers from -5,...,5?
Verify with a C++ program.

\textbf{Exercise.} How do you generate random \verb`double` from
0,...,2.0? Verify with a C++ program.

\textbf{Exercise.} How do you generate random double from -1.0,...,1.0?
Verify with a C++ program.

\textbf{Exercise.} How do you generate random double from 0,...,3.5?
Verify with a C++ program.

\textbf{Exercise.} How do you generate random double from -2.0,...,2.0?
Verify with a C++ program.

\newpage\EMPHASIZE{The important swap \lq\lq trick''}

Recall the following extremely important \lq\lq trick'' that you must know.
It allows you to swap the values in two variables:
\begin{console}
int a = 42, b = 24;
std::cout << a << " " << b << std::endl;
int t = a;
a = b;
b = t;
std::cout << a << " " << b << std::endl;
\end{console}

\textbf{Exercise.} Write a program that declares two doubles,
initializing them with 1.234 and 4.567. Print the values of the
variables. Swap their values. Print the values of the variables.

\newpage\EMPHASIZE{First sorting example}

Here's your first sorting problem:

\begin{tightlist}
\li
  Prompt the user for two integer values and assign them to x and y.
\li
  Sort the values so that the value of x is less than or equal to the
  value of y
\end{tightlist}

\textbf{Exercise.} Write the above program. Here are some test cases:

Test 1
\begin{console}[commandchars=\~\@\$]
~textbf@1 2$
1 2
\end{console}

Test 2
\begin{console}[commandchars=\~\@\$]
~textbf@2 1$
1 2
\end{console}

Test 3
\begin{console}[commandchars=\~\@\$]
~textbf@-1 5$
-1 5
\end{console}

Test 4
\begin{console}[commandchars=\~\@\$]
~textbf@5 -1$
-1 5
\end{console}

Test 5
\begin{console}[commandchars=\~\@\$]
~textbf@2 2$
2 2
\end{console}

\newpage\EMPHASIZE{More sorting examples: bubblesort}

Now suppose you have three integer variables and you want to sort the
values in the variables. What I mean is this. Suppose your variables are
x, y, and z. After some operations you want

\[x  <=  y\]  \&\&  \[y  <=  z\]

to be true (NOT \[x <= y <= z\] !!! Remember???)

Here' s the idea. First do this:\\
\begin{tightlist}
\li
  Swap the values of x and y so that x <= y.
\li
  Swap the values of y and z so that y <= z.
\end{tightlist}
This will guarantee that z is the largest.

\textbf{Exercise.} Take a piece of paper. And write down three integer
values for x, y, and z and follow the above steps. Do you see that the
largest will always be in variable z' s box?

\textbf{Exercise.} Does this means that the smallest will be in x?

The above two steps is said to be one \textit{\textbf{pass}} of this
sorting process. Now do this:
\begin{tightlist}
\li
  Swap (if necessary) the values of x and y so that x <= y.
\end{tightlist}

This will guarantee that the larger value between x and y will be in y.

Now everything is in ascending order, i.e.

\[x  <=  y\]  \&\&  \[y  <=  z\]

Altogether these are the steps:

\begin{quote}
Pass 1: To get largest among x, y, z into z do this:
\end{quote}
\begin{tightlist}
\li
  \begin{quote}
  Swap (if necessary) the values of x and y so that x <= y.
  \end{quote}
\li
  \begin{quote}
  Swap (if necessary) the values of y and z so that y <= z.
  \end{quote}
\end{tightlist}
\begin{quote}
Pass 2: To get largest among x, y into y
\end{quote}
\begin{tightlist}
\li
  \begin{quote}
  Swap (if necessary) the values of x and y so that x <= y.
  \end{quote}
\end{tightlist}

Such a \lq\lq recipe'' (a finite number of steps to achieve a goal) is called
an \EMPHASIZE{algorithm}. This sorting algorithm is called
\EMPHASIZE{bubblesort}. There are actually many different sorting
algorithms.

\textbf{Exercise.} Write a program that prompts the user for three
integers and prints the three integers in ascending order. Here are some
executions:
\begin{console}[commandchars=\~\@\$]
~textbf@1 2 3$
1 2 3
\end{console}
\begin{console}[commandchars=\~\@\$]
~textbf@1 3 2$
1 2 3
\end{console}
\begin{console}[commandchars=\~\@\$]
~textbf@2 1 3$
1 2 3
\end{console}
\begin{console}[commandchars=\~\@\$]
~textbf@2 3 1$
1 2 3
\end{console}
\begin{console}[commandchars=\~\@\$]
~textbf@3 1 2$
1 2 3
\end{console}
\begin{console}[commandchars=\~\@\$]
~textbf@3 2 1$
1 2 3
\end{console}

What if you have 4 values to sort (in ascending order)? Say the values
are stored in a, b, c, d. If you understood the above bubblesort for 3
values, the bubblesort for 4 values is similar:

\begin{quote}
Pass 1: To get largest among a, b, c, d into d do this:
\end{quote}
\begin{tightlist}
\li
  \begin{quote}
  Swap (if necessary) the values of a and b so that a <= b.
  \end{quote}
\li
  \begin{quote}
  Swap (if necessary) the values of b and c so that b <= c.
  \end{quote}
\li
  \begin{quote}
  Swap (if necessary) the values of c and d so that c <= d.
  \end{quote}
\end{tightlist}

\begin{quote}
Pass 2: To get largest among a, b, c into c
\end{quote}
\begin{tightlist}
\li
  \begin{quote}
  Swap (if necessary) the values of a and b so that a <= b.
  \end{quote}
\li
  \begin{quote}
  Swap (if necessary) the values of b and c so that b <= c.
  \end{quote}
\end{tightlist}

\begin{quote}
Pass 3: To get largest among a, b into b
\end{quote}
\begin{tightlist}
\li
  \begin{quote}
  Swap (if necessary) the values of a and b so that a <= b.
  \end{quote}
\end{tightlist}

\textbf{Exercise. }Test the above bubblesort for 4 values: Write a
program that prompts the user for 4 integers and prints the values in
ascending order.

\textbf{Exercise. }What if you have five variables a, b, c, d, e? How
would you take the values from a, b, c, d, e sort them into ascending
order, and put them into a, b, c, d, e?

\textbf{Exercise.} Write a program that prompts the user for 4
\texttt{\textbf{doubles}} and prints the \texttt{doubles} in ascending order.

\textbf{Exercise. }Write a program that prompts the user for 4 integers
and prints the integer values in \textbf{descending} order. (You need to
modify the above bubblesort algorithm.)

\textbf{Exercise.} Now, you have four variables a, b, c, d. You know
that one pass of the bubblesort will put the largest value of a, b, c, d
into d. What would you do if I ask you to write a program to prompt the
user for four values and then print the second largest. Now write the
program and test it. [Hint: You need \textbf{two} passes.]

\textbf{Exercise.} Again suppose you have four variables a, b, c, d.
Suppose during the second pass, you notice that no swap was performed.
Do you see that you do not need to perform any more passes?

You must memorize the bubblesort algorithm for any number of variables.
