% Options for packages loaded elsewhere
\PassOptionsToPackage{unicode}{hyperref}
\PassOptionsToPackage{hyphens}{url}
%
\documentclass[
]{article}
\usepackage{amsmath,amssymb}
\usepackage{iftex}
\ifPDFTeX
  \usepackage[T1]{fontenc}
  \usepackage[utf8]{inputenc}
  \usepackage{textcomp} % provide euro and other symbols
\else % if luatex or xetex
  \usepackage{unicode-math} % this also loads fontspec
  \defaultfontfeatures{Scale=MatchLowercase}
  \defaultfontfeatures[\rmfamily]{Ligatures=TeX,Scale=1}
\fi
\usepackage{lmodern}
\ifPDFTeX\else
  % xetex/luatex font selection
\fi
% Use upquote if available, for straight quotes in verbatim environments
\IfFileExists{upquote.sty}{\usepackage{upquote}}{}
\IfFileExists{microtype.sty}{% use microtype if available
  \usepackage[]{microtype}
  \UseMicrotypeSet[protrusion]{basicmath} % disable protrusion for tt fonts
}{}
\makeatletter
\@ifundefined{KOMAClassName}{% if non-KOMA class
  \IfFileExists{parskip.sty}{%
    \usepackage{parskip}
  }{% else
    \setlength{\parindent}{0pt}
    \setlength{\parskip}{6pt plus 2pt minus 1pt}}
}{% if KOMA class
  \KOMAoptions{parskip=half}}
\makeatother
\usepackage{xcolor}
\usepackage{longtable,booktabs,array}
\usepackage{calc} % for calculating minipage widths
% Correct order of tables after \paragraph or \subparagraph
\usepackage{etoolbox}
\makeatletter
\patchcmd\longtable{\par}{\if@noskipsec\mbox{}\fi\par}{}{}
\makeatother
% Allow footnotes in longtable head/foot
\IfFileExists{footnotehyper.sty}{\usepackage{footnotehyper}}{\usepackage{footnote}}
\makesavenoteenv{longtable}
\setlength{\emergencystretch}{3em} % prevent overfull lines
\providecommand{\tightlist}{%
  \setlength{\itemsep}{0pt}\setlength{\parskip}{0pt}}
\setcounter{secnumdepth}{-\maxdimen} % remove section numbering
\ifLuaTeX
  \usepackage{selnolig}  % disable illegal ligatures
\fi
\IfFileExists{bookmark.sty}{\usepackage{bookmark}}{\usepackage{hyperref}}
\IfFileExists{xurl.sty}{\usepackage{xurl}}{} % add URL line breaks if available
\urlstyle{same}
\hypersetup{
  hidelinks,
  pdfcreator={LaTeX via pandoc}}

\author{}
\date{}

\begin{document}

73. STL vector class

Objectives

\begin{itemize}
\tightlist
\item
  Use the \emph{std::vector} class
\end{itemize}

STL

STL = Standard Template Library .

STL is a collection of C++ class templates that come with most

C++ compilers . STL provides many useful classes. We will only scratch
the surface. More information on STL will appear in CISS350, CISS358,
CISS362, etc.

Vector class template

Recall you have an \emph{IntDynArr} class for modeling dynamic arrays.
There is a similar class called \emph{vector}. To use it, you must
include the \emph{vector} header (i.e. \emph{\#include
\textless vector\textgreater{}}). The vector class is in the namespace
\emph{std}. So

\emph{ std::vector\textless{} int \textgreater{} v;}

will declare object \emph{v} so that \emph{x} models a dynamic array of
integer values. In this case the size of \emph{v} is 0 -- it has no
values. You can also do this:

\emph{ std::vector\textless{} int \textgreater{} v(10);}

In this case the size of \emph{v} is 10. You can also do

\emph{ std::vector\textless{} int \textgreater{} v(10, 42);}

in which case \emph{v} is size 10 and all the values in \emph{v} are set
to 42.

\textbf{Exercise. }Create a vector of 10 characters, setting all values
to \emph{\textquotesingle{} \textquotesingle{}}.

Member Functions/Methods

\begin{longtable}[]{@{}
  >{\raggedright\arraybackslash}p{(\columnwidth - 0\tabcolsep) * \real{1.0000}}@{}}
\toprule\noalign{}
 \\
\midrule\noalign{}
\endhead
\bottomrule\noalign{}
\endlastfoot
assign() // Erases a vector and copies the

// specified elements to the empty

// vector.

at() // Returns a reference to the element at

// a specified location in the vector.

back() // Returns a reference to the last

// element of the vector.

capacity() // Returns the number of elements

// that the vector could contain

// without allocating more storage.

clear() // Erases all elements of the vector.

empty() // Tests if the vector container is

// empty.

erase() // Removes an element or a range

// of elements in a vector from

// specified positions.

front() // Returns a reference to the first

// element in a vector.

insert() // Inserts an element or a number of

// elements into the vector at a

// specified position.

max\_size() // Returns the maximum possible

// length of the vector.

pop\_back() // Deletes the element at the end of the

// vector.

push\_back() // Add an element to the end of the

// vector.

resize() // Specifies a new size for a vector.

reserve() // Reserves a minimum length of storage

// for a vector object.

size() // Returns the number of elements in the

// vector.

swap() // Exchanges the elements of two vectors.

vector() // Constructor. Constructs a vector of a

// specific size or with elements of

// specific value.

operator{[}{]} // Returns a reference to the vector

// element at a specified position. \\
\end{longtable}

Operators \emph{==}, \emph{!=}, \emph{=} do the obvious things.

Now for a quick experiment on some of the methods. Run the following
program and study the output carefully.

\begin{longtable}[]{@{}
  >{\raggedright\arraybackslash}p{(\columnwidth - 0\tabcolsep) * \real{1.0000}}@{}}
\toprule\noalign{}
 \\
\midrule\noalign{}
\endhead
\bottomrule\noalign{}
\endlastfoot
\#include \textless iostream\textgreater{}

\#include \textless vector\textgreater{}

std::ostream \& operator\textless\textless(std::ostream \& cout,

const std::vector\textless{} int \textgreater{} \& v)

\{

std::string sep = "";

cout \textless\textless{} \textquotesingle\{\textquotesingle;

for (int i = 0; i \textless{} v.size() - 1; ++i)

\{

cout \textless\textless{} sep \textless\textless{} v{[}i{]};

if (i == 0) sep = ", ";

\}

cout \textless\textless{} \textquotesingle\}\textquotesingle;

return cout;

\}

int main()

\{

int s = 3;

std::vector\textless{} int \textgreater{} v(s);

v{[}0{]} = 1; v{[}1{]} = 10; v{[}2{]} = 100;

std::cout \textless\textless{} "v: " \textless\textless{} v
\textless\textless{} "\textbackslash n";

v{[}0{]} = 42;

std::cout \textless\textless{} "v: " \textless\textless{} v
\textless\textless{} "\textbackslash n";

std::vector\textless{} int \textgreater{} u(v);

std::cout \textless\textless{} "u: " \textless\textless{} u
\textless\textless{} "\textbackslash n";

if (u == v)

\{

std::cout \textless\textless{} "same\textbackslash n";

\}

v.resize(10);

std::cout \textless\textless{} "v: " \textless\textless{} v
\textless\textless{} "\textbackslash n";

std::vector\textless{} int \textgreater{} w(4);

std::cout \textless\textless{} "w: " \textless\textless{} w
\textless\textless{} "\textbackslash n";

w = u;

std::cout \textless\textless{} "w: " \textless\textless{} w
\textless\textless{} "\textbackslash n";

return 0;

\} \\
\end{longtable}

Study the output carefully. Try all the methods on your own -- most of
them are very similar to the methods in the \emph{IntDynArr} class.

Iterators

Recall that we have written an \emph{IntPointer} class so that objects
of this class behaves like integer pointers. Your \emph{std::vector}
class also comes with a pointer class that can be used to point to
values in a \emph{std::vector} object. These are called
\textbf{iterator} classes.

Besides accessing values in the vector using an index integer value, you
can also use iterators. Iterators are object which pretend to be
pointers. This is similar to the previous program but using iterators.
Run it and study the code very carefully.

\begin{longtable}[]{@{}
  >{\raggedright\arraybackslash}p{(\columnwidth - 0\tabcolsep) * \real{1.0000}}@{}}
\toprule\noalign{}
 \\
\midrule\noalign{}
\endhead
\bottomrule\noalign{}
\endlastfoot
\#include \textless iostream\textgreater{}

\#include \textless vector\textgreater{}

\#include \textless string\textgreater{}

std::ostream \& operator\textless\textless(std::ostream \& cout,

const std::vector\textless{} int \textgreater{} \& v)

\{

cout \textless\textless{} \textquotesingle\{\textquotesingle;

std::string sep("");

for (std::vector\textless{} int \textgreater::const\_iterator p =

v.begin();

p != v.end(); ++p)

\{

cout \textless\textless{} sep \textless\textless{} (*p);

if (sep == "") sep = ", ";

\}

cout \textless\textless{} \textquotesingle\}\textquotesingle;

return cout;

\}

int main()

\{

int s = 3;

std::vector\textless{} int \textgreater{} v(s);

v{[}0{]} = 1; v{[}1{]} = 10; v{[}2{]} = 100;

std::cout \textless\textless{} "v: " \textless\textless{} v
\textless\textless{} "\textbackslash n";

std::vector\textless{} int \textgreater::iterator p = v.begin();

*p = 42;

++p;

*p = 43;

p = v.end();

-\/-p;

*p = 99;

std::cout \textless\textless{} "v: " \textless\textless{} v
\textless\textless{} "\textbackslash n";

p = v.begin();

p += 1;

v.erase(p);

std::cout \textless\textless{} "v: " \textless\textless{} v
\textless\textless{} "\textbackslash n";

return 0;

\} \\
\end{longtable}

\emph{std::vector\textless{} int \textgreater::const\_iterator} objects
are pointers that do not change the value they point to.
\emph{std::vector\textless{} int \textgreater::iterator} objects are
pointers that can change the value they point to. When you do

std::vector\textless{} int \textgreater::iterator p = v.begin();

the iterator p will point to the first value of v. In this case \emph{p}
can change the value that it points to. So you can change the first
value of v to 42 if you do this:

std::vector\textless{} int \textgreater::iterator p = v.begin();

*p = 42;

However if you do

std::vector\textless{} int \textgreater::const\_iterator p = v.begin();

then \emph{p} cannot change the value it points to. So for instance the
compiler will yell at you if you write this:

std::vector\textless{} int \textgreater::const\_iterator p = v.begin();

*p = 42;

By the way the reason for the second ``::'' in

std::vector\textless{} int \textgreater::iterator p;

is because the iterator class is defined \textbf{inside} the std::vector
class:

\begin{longtable}[]{@{}
  >{\raggedright\arraybackslash}p{(\columnwidth - 0\tabcolsep) * \real{1.0000}}@{}}
\toprule\noalign{}
 \\
\midrule\noalign{}
\endhead
\bottomrule\noalign{}
\endlastfoot
template \textless{} typename T \textgreater{}

class vector

\{

...

class iterator

\{

...

\};

class const\_iterator

\{

...

\};

...

\}; \\
\end{longtable}

Yes, that's right \ldots{} you can nest class definitions.

Let's go on. If you do

std::vector\textless{} int \textgreater::iterator p = v.end();

then p will point to the \textbf{very first value beyond} the last value
of v. Therefore a for-loop that runs an iterator through the value of v
would look like:

for (std::vector\textless{} int \textgreater::iterator p = v.begin();

p != v.end(); ++p)

\{

...

\}

In this case the iterator \emph{p} can change the values of \emph{v}. If
you do not want to change the value of \emph{v}, you can use a constant
iterator:

for (std::vector\textless{} int \textgreater::const\_iterator p =
v.begin();

p != v.end(); ++p)

\{

...

\}

Like a regular pointer, you can do pointer arithmetic on your iterators.

std::vector\textless{} int \textgreater::iterator p = v.begin();

*p = 42; // Index 0 value of v is change to 42

++p; // p now points to index 1 value of v

p += 3 // p not points to index 4 value of v

You can also do -\/-p and p -= 5.

While v.begin() gives you the address of the index 0 value of v, v.end()
will give you the address of the last value of v, i.e., the value at
index v.size() - 1.

If p is an iterator,

v.erase(p)

will erase the value that \emph{p} points to -- values in \emph{v} after
the value that is being erased will be move backward by one index step
and the size of \emph{v} is decremented by 1. You can also erase a whole
section of your vector. For instance if \emph{v} has 100 values, then

v.erase(v.begin() + 3, v.begin() + 10);

will erase the values from index 0 + 3 to 0 + 10; the size decrements by
8. Make sure you create a test case for this and verify it for yourself.

By the way, make sure you see

\begin{itemize}
\tightlist
\item
  the difference between \emph{v.front()} and \emph{v.begin()} and
\item
  the difference between \emph{v.back()} and \emph{v.end()}.
\end{itemize}

If you don't see the difference, start from the beginning of this
document and read it all over again.

Iterators are \textbf{extremely important} when working with containers
classes of C++ because all container classes (like std::string and
std::vector) have iterators. So iterators provide a \textbf{common way}
to access, modify, delete values in these containers.

\textbf{Exercise.} Create a vector v of 10 doubles. Using iterators, do
the following:

\begin{itemize}
\tightlist
\item
  Set the first value to 3.1
\item
  Set the second value to 3.14
\item
  Set the third value to 3.141
\item
  Using a for-loop, set the rest to -1.0
\item
  Using a for-loop, print all the values of v.
\item
  Erase all the values which are -1.0.
\item
  Print the size of v -- you should get 3.
\item
  Using a for-loop, print all the values of v -- you should see only 3
  values.
\end{itemize}

\textbf{Exercise. Implement} the following function prototypes (which
you learn from CISS240):

int count(const std::vector\textless{} int \textgreater{} \& v, int
target);

int linearsearch(const std::vector\textless{} int \textgreater{} \& v,
int target);

void bubblesort(std::vector\textless{} int \textgreater{} \& v);

int binarysearch(const std::vector\textless{} int \textgreater{} \& v,
int target);

2D arrays

Since \emph{std::vector} is a class template, I can create a vector
where each value in this vector \ldots{} \textbf{is also a vector!!!}

std::vector\textless{} std::vector\textless{} int \textgreater{}
\textgreater{} v;

v.resize(3);

At this point, v has 3 values, i.e., v has 3 vectors. Now I'll make each
of the 3 vectors in v into vectors of size 5:

for (int i = 0; i \textless{} 3; ++i)

\{

v{[}i{]}.resize(5);\\
\}

Viola \ldots{} v is now a 2D array of size 3-by-5. Get it?

\textbf{Exercise.} Write a function to print the values of a
std::vector\textless{} std::vector \textless{} int \textgreater{}
\textgreater{} object.

\textbf{Exercise.} Write a function to print the v above. Here's an
example of how the output should look like:

{[}{[}1 3 5 7{]}

{[}2 4 6 8{]}

{[}3 6 9 12{]}{]}

Your function should work for any sizes of v.

\textbf{Exercise.} Write a tic-tac-toe game where the board size is
n-by-n where n is specified by the user.

\textbf{\hfill\break
Exercise. }Create a 2D array of integer values using \emph{std::vector}
where even index rows have size 4 and odd index rows have size 5.

References

\begin{itemize}
\item
  C++ textbooks usually contain information on STL classes.
\item
  There are lots of C++ reference on the web. Example: cppreference.com.
\item
  Knowing the STL classes help the engineering of software.

  However knowledge of constructing your own classes is also important
  because (obviously) STL does not provide all the possibly useful
  classes! Sometimes you might need to write your

  own classes for performance reasons.
\item
  At other times, because of the nature of your problem, you might be
  able to write your own class which will produce objects with faster
  methods.
\item
  Therefore it\textquotesingle s extremely important to know how some of
  the standard classes in your C++ compilers are implemented.
\end{itemize}

\end{document}
