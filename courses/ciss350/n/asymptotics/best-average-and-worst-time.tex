%-*-latex-*-
\sectionthree{Best, average, and worst runtime}
\begin{python0}
from solutions import *; clear()
\end{python0}

Now let me consider an algorithm 
when the body of the for-loop 
contains an \verb!if!-statement.

The following computes the index in an (unsorted) 
array \verb!x! of size \verb!n!
where \verb!target! is 
first found, i.e., this is the linear search algorithm:

\begin{Verbatim}[frame=single, fontsize=\footnotesize]
index = -1
for i = 0, 1, 2, ..., n - 1:
    if x[i] is target:
        index = i
        break
\end{Verbatim}
Here's the above written in a way that makes timing calculation easier:
\begin{Verbatim}[frame=single, fontsize=\footnotesize]
                                 time   
         index = -1              t1 
         i = 0                   t2 
LOOP:    if i >= n:              t3  
             goto ENDLOOP        t4 
         if x[i] is not target:  t5
             goto ELSE           t6
         index = i               t7 
         goto ENDLOOP            t8
ELSE:    i = i + 1               t9 
         goto LOOP               t10 
ENDLOOP:
\end{Verbatim}

(For non-programmers:
when I say array \verb!x! is an array of size \verb!n! I mean that
you have \verb!x[0]!, \verb!x[1]!, \ldots, \verb!x[n-1]!
which is similar to the mathematical idea of 
a bunch of variables with scripts $x_0$, $x_1$, $\ldots$, $x_{n-1}$.
\verb!break! means to get out of the current loop.
The time to access the \verb!i!--th element \verb!x[i]! of \verb!x! is
constant.)

The amount of time needed of course depends on how fast we hit \verb!target!:
if \verb!target! happens to be at index 0, of course the algorithm ends
quickly. 
This is the best case scenario.
And if \verb!target! is at index $n-1$ or if it's not even in the array,
the algorithm would run longer since you would have to scan the whole array.
These are the worst case scenarios.

Note that in my first timing example that computes the sum from 1 to $n$,
I converted a for-loop into statements of a simplified
language that uses the goto and the
conditional branching statement.
For the above example, it's easy to see that if you have an algorithm
that has an if-else statement such as
\begin{Verbatim}[frame=single, fontsize=\footnotesize]
if x > 1:
    statement-1
    statement-2
else:
    statement-3    
    statement-4
\end{Verbatim}
you can rewrite that in the simplified language as
\begin{Verbatim}[frame=single, fontsize=\footnotesize]
         if x <= 1:
             goto ELSE
         statement-1
         statement-2
         goto ENDIF
ELSE:    statement-3
         statement-4
ENDIF:         
\end{Verbatim}
Note that the conditional branching leads to the else case, and therefore
the boolean condition is the opposite of the boolean condition of the
if-else statement.
Now let's go back to the timing calculations.

Here's the timing calculation for the best scenario:
\begin{Verbatim}[frame=single, fontsize=\footnotesize]
                                   time    number of times   
         index = -1                t1      1
         i = 0                     t2      1
LOOP:    if i >= n:                t3      1
             goto ENDLOOP          t4      0
         if x[i] is not target:    t5      1
             goto ELSE             t6      0
         index = i                 t7      1
         goto ENDLOOP              t8      1
ELSE:    i = i + 1                 t9      0
         goto LOOP                 t10     0
ENDLOOP:
\end{Verbatim}
The time taken is
\begin{align*}
\text{Time taken } 
&= A
\end{align*}
for some constant $A$.
In this case the constant function $f(n) = A$
is a constant multiple of the simpler function $g(n) = 1$.
Since we ignore multiples, I can say that for this 
\lq\lq optimistic'' case, the runtime is $O(1)$.
Mathemtically, I can write this:
\begin{align*}
\text{Time taken } 
&= A = O(1)
\end{align*}

For the worst case where the \verb!target! is the last element of the array,
i.e., at index \verb!n - 1!,
we have the following:
\begin{Verbatim}[frame=single, fontsize=\footnotesize]
                                   time     number of times   
         index = -1                t1       1
         i = 0                     t2       1
LOOP:    if i >= n:                t3       n
             goto ENDLOOP          t4       0
         if x[i] is not target:    t5       n 
             goto ELSE             t6       n - 1
         index = i                 t7       1
         goto ENDLOOP              t8       1
ELSE:    i = i + 1                 t9       n - 1
         goto LOOP                 t10      n - 1
ENDLOOP:
\end{Verbatim}
The time taken is
\begin{align*}
\text{Time taken } 
&= t_1 + t_2 + t_7 + t_8 + 
   (t_3 + t_5)n + 
   (t_6 + t_9 + t_{10})(n - 1) \\
&= (t_3 + t_5 + t_6 + t_9 + t_{10})n +
   (t_1 + t_2 - t_6 + t_7 
    + t_8 - t_9 - t_{10}) \\ 
&= An + B
\end{align*}
for constants $A$ and $B$.
In this case the runtime function is big-O of $n$, i.e., it is 
$O(n)$.
I write:
The time taken is
\[
\text{Time taken } 
= An + B = O(n)
\]

For the worst case where the \verb!target! is not found
we have the following:
\begin{Verbatim}[frame=single, fontsize=\footnotesize]
                                   time    number of times   
         index = -1                t1      1
         i = 0                     t2      1
LOOP:    if i >= n:                t3      n + 1
             goto ENDLOOP          t4      1
         if x[i] is not target:    t5      n 
             goto ELSE             t6      n
         index = i                 t7      0
         goto ENDLOOP              t8      0
ELSE:    i = i + 1                 t9      n
         goto LOOP                 t10     n
ENDLOOP:
\end{Verbatim}
The time taken is
\begin{align*}
\text{Time taken } 
  &= t_1 + t_2 + 
    t_4 + 
    n(t_3 + t_5 + t_6 + t_9 + t_{10}) +
    (n + 1) t_3 \\
  &= n(t_3 + t_5 + 
    t_6  + 
    t_9 +
    t_{10}
    )
    +
    t_1 + 
    t_2 + 
    t_3 + 
    t_4 \\
&= An + B \\
&= O(n)
\end{align*}
for constants $A$ and $B$.
In the second worst case scenario,
the runtime function is also big-O of $n$, 
i.e., it is $O(n)$.

In summary the best runtime and the two worst runtimes are
\begin{align*}
A_1 &= O(1) \\
A_2n + B_2 &= O(n) \\
A_3n + B_3 &= O(n) 
\end{align*}

Usually performance of algorithms are described in terms of
best, worst, and average times.
If you want to lump up all the runtimes (i.e., you don't really want to
be that specific), we want to say that runtime is $O(n)$, 
referring to the absolute worst scenario.

In other words you should think of the big-O notation $O(n)$
as some kind of \textit{upper} bound approximation, i.e.,
all the above times are bounded above by a large enough multiple of $n$:
\begin{align*}
A_1        &\leq C \cdot n \\
A_2n + B_2 &\leq C \cdot n \\
A_3n + B_3 &\leq C \cdot n
\end{align*}
where $C$ is some humongous fixed number and the above inequalities
are true for large values of $n$.

While talking about a specific algorithm, to distinguish between
the best, worst, and average runtime, I will write
$T_{b}(n)$,
$T_{w}(n)$,
and
$T_{a}(n)$.
(Technically speaking there are two different worst case scenarios
for the linear search above.
But they both yield the same big-O anyway.)

If I don't say which of the three cases, I always mean the 
worst case $T_w(n)$.

I have shown you above that for the linear search
\begin{align*}
T_b(n) &= O(1) \\
T_w(n) &= O(n)
\end{align*}

The average case is a little more complicated.
In the above linear search algorithm, we looked at
three cases (one best and two worst).
But in general, the \verb!target! can be anywhere and you would have to 
account for all of them, measure their times, 
say we call them $T_0(n)$, ..., $T_n(n)$
where the algorithm has $n + 1$ cases
($T_0(n)$ corresponding to the time where the \verb!target! is at index 0, ...
and $T_n(n)$ corresponding to the time where the \verb!target! is not
in the array at all)
and then take the average:
\[
\frac{T_0(n) + \cdots + T_n(n)}{n + 1}
\]
But that assumes something: 
That the cases corresponding to times $T_0(n)$, ..., $T_n(n)$ are
equally likely to occur.

Depending on specific scenario, 
there are cases that might occur more frequently.
For instance if in the above, the case where the index 0 occurs twice
as frequently as the rest, then the average would be
\[
\frac{2T_0(n) + \cdots + T_n(n)}{n + 2}
\]
To analyze the average runtime for
complicated cases requires a little more probability theory.
For now we will only handle very simplistic average cases.

In general, to compute \textit{an} (not \textit{the}) 
average runtime, you have to 
\begin{enumerate}[nosep]
\item[(1)] state what cases you're averaging over and 
\item[(2)] what is the likelihood of each case.
\end{enumerate}
When the cases are not stated, they are usually obvious.
Also, if the likelihood of each case is not stated, then it is assumed
that all cases are equally likely.

For many algorithms and many average scenarios, 
the average runtimes tend to be
the same as the worst runtime when you fudge the functions using big-O. 

Now that I've explained how to compute the average runtime, let's
compute the average runtime of the linear search assuming that we
are only averaging over the cases where
\verb!target! is at index $0, 1, 2, ...,  n - 1$.
Note that I'm not considering the case 
where \verb!target! is not in the array.
If you like you can think of this as the 
\lq\lq average runtime for a successful search''.
(In a later section, I'll include the case where the \verb!target!
is not in the array -- this is just slightly more complicated.)

Let me assume that \verb!target! is at index value $k$
where $0 \leq k \leq n - 1$.
Here's the number of times each statement will execute in this case:
\begin{Verbatim}[frame=single, fontsize=\footnotesize]
                                   time    number of times   
         index = -1                t1      1
         i = 0                     t2      1
LOOP:    if i >= n:                t3      k + 1
             goto ENDLOOP          t4      0
         if x[i] is not target:    t5      k + 1 
             goto ELSE             t6      k 
         index = i                 t7      1
         goto ENDLOOP              t8      1
ELSE:    i = i + 1                 t9      k
         goto LOOP                 t10     k
ENDLOOP:
\end{Verbatim}
Therefore time taken for this case is
\begin{align*}
T_k(n)
  &= (t_1 + t_2 + t_7 + t_8)
    + (t_3 + t_5)(k + 1)
    + (t_6 + t_9 + t_{10})k \\
  &= (t_3 + t_5 + t_6 + t_9 + t_{10}) k
    + (t_1 + t_2 + t_3 + t_5 + t_7 + t_8)
\end{align*}
for $k = 0, 1, 2, ..., n - 1$.
Let $T_n$ be the time corresponding to the case where
\verb!target! is \textit{not} in the array.
Earlier, I have already computed the runtime for the case where 
\verb!target! is not in the array:
\[
  T_n(n) =
  (t_3 + t_5 + 
    t_6  + 
    t_9 +
    t_{10}
    )n
    +
    t_1 + 
    t_2 + 
    t_3 + 
    t_4
\]
To simplify the constants 
(because later, we'll be computing by big-O anyway),
let
\begin{align*}
A &= t_3 + t_5 + t_6 + t_9 + t_{10} \\
B &= t_1 + t_2 + t_3 + t_5 + t_7 + t_8 \\
C &= t_3 + t_5 + 
    t_6  + 
    t_9 +
    t_{10} \\
D &= t_1 + 
    t_2 + 
    t_3 + 
    t_4
\end{align*}
Here's a summary:
\begin{align*}
T_k(n) &= Ak + B, \,\,\,\,\,(k = 0, 1, 2,..., n - 1) \\
T_n(n) &= Cn + D
\end{align*}

OK, now I'm going to compute the average runtime
assuming 
\begin{itemize}
\item the \verb!target! is in the array with equal likelihood
at all index positions.
\end{itemize}

Remember: This average runtime scenario
does \textit{not} include the case where the \verb!target!
is not in array \verb!x!.
So I need to average over $T_0(n), \ldots, T_{n-1}(n-1)$ ... 
do \textit{not} include $T_n(n)$.

This average runtime is
\begin{align*}
T_a(n) 
&= \frac{1}{n} 
   \left(
     T_0(n) + T_1(n) + \cdots + T_{n - 1}(n) 
   \right)
\\
&= \frac{1}{n}
   \left(
     (A\cdot 0 + B) + 
     (A\cdot 1 + B) + 
     \cdots
     (A\cdot (n - 1) + B)
   \right)
\\
&= \frac{1}{n}
   \left(
     A\cdot (0 + 1 + \cdots + (n - 1))
     + Bn 
   \right)
\\
&= \frac{1}{n}
   \left(
     A \cdot \frac{n(n-1)}{2} + Bn
   \right)
\\
&= \frac{1}{n}
   \left( 
     \frac{A}{2} n^2
     + \left( B - \frac{1}{2}A \right) n
   \right)
\\
&= \frac{A}{2} n
   + \left( B - \frac{1}{2}A \right)
\\
&= O(n)
\end{align*}


That's it!

\newpage%-*-latex-*-

\begin{ex} 
  \label{ex:prob-00}
  \tinysidebar{\debug{exercises/{disc-prob-28/question.tex}}}

  \solutionlink{sol:prob-00}
  \qed
\end{ex} 
\begin{python0}
from solutions import *
add(label="ex:prob-00",
    srcfilename='exercises/discrete-probability/prob-00/answer.tex') 
\end{python0}

\newpage%-*-latex-*-

\begin{ex} 
  \label{ex:prob-00}
  \tinysidebar{\debug{exercises/{disc-prob-28/question.tex}}}

  \solutionlink{sol:prob-00}
  \qed
\end{ex} 
\begin{python0}
from solutions import *
add(label="ex:prob-00",
    srcfilename='exercises/discrete-probability/prob-00/answer.tex') 
\end{python0}

\newpage%-*-latex-*-

\begin{ex} 
  \label{ex:prob-00}
  \tinysidebar{\debug{exercises/{disc-prob-28/question.tex}}}

  \solutionlink{sol:prob-00}
  \qed
\end{ex} 
\begin{python0}
from solutions import *
add(label="ex:prob-00",
    srcfilename='exercises/discrete-probability/prob-00/answer.tex') 
\end{python0}

\newpage%-*-latex-*-

\begin{ex} 
  \label{ex:prob-00}
  \tinysidebar{\debug{exercises/{disc-prob-28/question.tex}}}

  \solutionlink{sol:prob-00}
  \qed
\end{ex} 
\begin{python0}
from solutions import *
add(label="ex:prob-00",
    srcfilename='exercises/discrete-probability/prob-00/answer.tex') 
\end{python0}

\newpage%-*-latex-*-

\begin{ex} 
  \label{ex:prob-00}
  \tinysidebar{\debug{exercises/{disc-prob-28/question.tex}}}

  \solutionlink{sol:prob-00}
  \qed
\end{ex} 
\begin{python0}
from solutions import *
add(label="ex:prob-00",
    srcfilename='exercises/discrete-probability/prob-00/answer.tex') 
\end{python0}

