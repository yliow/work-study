%-*-latex-*-
\sectionthree{Floor and Ceiling}
\begin{python0}
from solutions import *; clear()
\end{python0}

No ... I'm not creating a diversion to building construction although
what I'm going to talk about is analogous to what you're standing on
and what's above your head.

Given a real number $x$.
The \defterm{floor} of $x$, 
written $\floor{x}$ is $x$ if $x$ is an integer or the
integer before $x$.
For instance
\[
\floor{3} = 3, \,\,\,\,\, 
\floor{1.7} = 1, \,\,\,\,\, 
\floor{-2.3} = -3
\]

The \defterm{ceiling} of $x$, written $\ceiling{x}$ is $x$ if
$x$ is an integer or the integer after $x$.
For instance
\[
\ceiling{3} = 3, \,\,\,\,\, 
\ceiling{1.7} = 2, \,\,\,\,\, 
\ceiling{-2.3} = -2
\]

Note that in your C/C++ programs when you do something like
\begin{Verbatim}[frame=single,fontsize=\footnotesize]
int y = 7;
int x = y / 2;
\end{Verbatim}
you know that the \lq\lq\verb!/!'' is integer division because
\verb!y! and \verb!2! are integers.
Therefore \verb!x! has a value of \verb!3!.
In mathematical notation, when you write $a/b$,
the \lq\lq / '' is \textit{real} division
regardless of whether $a$ and $b$ are real or integers.
So if I convert the above to mathematical notation
I would have to write
\begin{align*}
y &= 7 \\
x &= \floor{y/2}
\end{align*}

\textit{When} (not if) 
you want to make use of mathematical tools to study
algorithms, you have to translate your algorithms into mathematical notation
in order to move from the program world to the math world.
Since the meaning of \lq\lq \verb!/!'' in a program might be different from
the meaning of \lq\lq /'' in math, you have to be very careful.

The point is that when you convert a \verb!double! value
to \verb!int! value by type casting, 
the \textit{fractional part} of the \verb!double!
value is lost.
Therefore for the following C/C++ code:
\begin{Verbatim}[frame=single,fontsize=\footnotesize]
double y = 3.14159;
int x = y;
\end{Verbatim}
the value of \verb!x! is 3. 
(This is old CISS240 stuff, right?)
Therefore the integer cast is like the floor function in math.
But that's only true for \textit{positive} value because for the following:
\begin{Verbatim}[frame=single,fontsize=\footnotesize]
double y = -3.14159;
int x = y;
\end{Verbatim}
the value \verb!x! is \verb!-3! whereas, in math, 
the floor of $-3.14159$ is $-4$.
So remember the double-to-int cast (same for float-to-int of course)
is like the floor only for values $\geq 0$.

\begin{ex}\mbox{}
\begin{tightlist}
\item[(a)] Compute $\floor{3.5}$.
\item[(b)] Compute $\ceiling{3.5}$.
\item[(c)] Compute $\floor{3}$.
\item[(d)] Compute $\ceiling{3}$.
\end{tightlist}
\qed
\end{ex}

\begin{ex}
Let $x = 5$.
Compute
\[
\floor{\frac{x + 0}{4}} 
+ \floor{\frac{x + 1}{4}}
+ \floor{\frac{x + 2}{4}}
+ \floor{\frac{x + 3}{4}}
\]
and
\[
\ceiling{\frac{x + 0}{4}} 
+ \ceiling{\frac{x + 1}{4}}
+ \ceiling{\frac{x + 2}{4}}
+ \ceiling{\frac{x + 3}{4}}
\]

\qed
\end{ex}


\begin{ex}
Consider the following algorithm where \verb!x!, \verb!y!, and \verb!z!
are arrays. \verb!x! has size \verb!n! = 25.
\begin{Verbatim}[frame=single,fontsize=\footnotesize]
n0 = n / 3
for i = 0, 1, 2, ..., n0 - 1:
    y[i] = x[i]
for i = n0, n0 + 1, ..., n - 1:
    z[i - n0] = x[i]
\end{Verbatim}
\begin{tightlist}
\item[(a)] How many values were copied over to \verb!y!?
\item[(b)] How many values were copied over to \verb!z!?
\item[(c)] If \verb!n! has value 26, 
how many values were copied over to \verb!y!?
\item[(d)] If \verb!n! has value 26,
how many values were copied over to \verb!z!?
\item[(e)] If \verb!n! has value 27, 
how many values were copied over to \verb!y!?
\item[(f)] If \verb!n! has value 27,
how many values were copied over to \verb!z!?
\end{tightlist}
\end{ex}


\begin{ex}
Plot $y = \floor{x}$.
\qed
\end{ex}

\begin{ex}
Plot $y = \ceiling{x}$.
\qed
\end{ex}




\begin{ex}
Find all real numbers $x$ such that
\[
\floor{x} = 2
\]
\qed
\end{ex}




\begin{ex}
  \mbox{}
\begin{tightlist}
\item Write a function in C/C++ that accepts a double \verb!x! and returns the
ceiling of value of \verb!x!. Call your function \verb!ceiling!.
(Test your code ... I don't have to tell you that, right?)
\item Do the same for floor.
\end{tightlist}
\qed
\end{ex}


\begin{ex}
Plot $y = \frac{\ceiling{x} + \floor{x}}{2}$.
Solve
\[
\frac{\ceiling{x} + \floor{x}}{2} = x
\]
\qed
\end{ex}


\begin{ex}
Plot $y = 2 \floor{\frac{x}{2}}$.
Solve
\[
2 \floor{\frac{x}{2}} = x
\]
\qed
\end{ex}

\begin{ex}
Plot $y = \ceiling{x/2} + \floor{x/2}$.
Solve
\[
\ceiling{x/2} + \floor{x/2} = x
\]
\qed
\end{ex}


\begin{ex} Let $a \geq 0, b > 0, c > 0$ be integers.
Write a program to check
\[
\floor{\floor{a/b}/c} = \floor{a/(bc)}
\]
for 
$0 \leq a \leq 1000000$, 
$1 \leq b \leq 1000000$, 
and
$1 \leq b \leq 1000000$.
If the above is false, list a reasonably small counterexample.
Otherwise prove it.
\qed
\end{ex}

\begin{ex} Let $a \geq 0, b > 0, c > 0$ be integers.
Write a program to check
\[
\ceiling{\ceiling{a/b}/c} = \ceiling{a/(bc)}
\]
for 
$0 \leq a \leq 1000000$, 
$1 \leq b \leq 1000000$, 
and
$1 \leq b \leq 1000000$.
If the above is false, list a reasonably small counterexample.
Otherwise prove it.
\qed
\end{ex}

