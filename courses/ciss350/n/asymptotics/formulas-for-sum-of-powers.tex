%-*-latex-*-
\sectionthree{Sums of Powers}
\begin{python0}
from solutions import *; clear()
\end{python0}

I'm going to give you two formulas which are extremely 
useful in runtime computations.

The first formula
you have probably seen in quite a few math classes.
Here's the arithmetic sum formula:

\begin{prop}
\[
1 + 2 + \cdots + n = \frac{n(n+1)}{2}
\]
\qed
\end{prop}

I won't prove the above formula.

\begin{ex}
Check by hand that the arithmetic sum formula is correct
for $n = 1, 2, 3, 4$.
\qed
\end{ex}

Using the summation notation you can write the above as
\[
\sum_{i=1}^n i = \frac{n(n+1)}{2}
\]
Writing down the formula when the lower or upper limit of the sum
is slightly altered
is pretty easy.
For instance suppose I want to compute $2 + 3 + \cdots + n$
as a polynomial expression in decreasing powers of $n$.
Then from
\[
1 + 2 + \cdots + n = \frac{n(n+1)}{2}
\]
I just subtract 1 to get
\[
2 + \cdots + n = \frac{n(n+1)}{2} - 1 
\]
I then do some algebra to get this:
\[
\frac{n(n+1)}{2} - 1
= \frac{1}{2}n^2 + \frac{1}{2}n - 1
\]
Here's the solution to this problem:

\begin{eg}
Express $\sum_{i = 2}^n i$ as a polynomial in descending powers of $n$.
\end{eg}

\textit{Solution.}
\begin{align*}
\sum_{i=2}^n i
&= \sum_{i=1}^n i - 1 \\ 
&= \frac{n(n+1)}{2} - 1 \\ 
&= \frac{1}{2}n^2 + \frac{1}{2}n + \frac{1}{2} - 1 \\ 
&= \frac{1}{2}n^2 + \frac{1}{2}n - \frac{1}{2}
\end{align*} 
\qed

Now suppose I want 
\[
\sum_{i = 1}^{n - 1} i
\]
Note that the upper limit of this summation is $n - 1$ and not $n$.
In this case, I just replace the $n$ in the arithmetic sum
formula by $n - 1$ to get:
\[
\sum_{i = 1}^{n - 1} i
= \frac{(n - 1)((n - 1) + 1)}{2}
= \frac{(n - 1)n}{2}
\]

\begin{eg}
Write
\[
\sum_{i = 1}^{n - 1} i
\]
as a polynomial in descending powers of $n$.
\end{eg}

\textit{Solution.}
\begin{align*}
\sum_{i = 1}^{n - 1} i
&= \frac{(n - 1)((n - 1) + 1)}{2} \\
&= \frac{(n - 1)n}{2} \\
&= \frac{1}{2}n^2 - \frac{1}{2}n
\end{align*}
\qed



\newpage
\begin{ex}
Compute the sum $1 + 2 + 3 + \cdots + 1000$ by first rewriting
it as a summation and then using the arithmetic sum formula.
\qed
\end{ex}



\newpage
\begin{ex}
Compute 
\[
\sum_{i = 1}^{100} 2i
\]
using the arithmetic sum formula.
(Write down the first 3 terms of the summation and the last 3 just
to make sure you know what you're adding.)
\end{ex}


\newpage
\begin{ex}
Compute 
\[
\sum_{i = 1}^{100} (2i + 1)
\]
using the arithmetic sum formula.
(Write down the first 3 terms of the summation and the last 3 just
to make sure you know what you're adding.)
\end{ex}



\newpage
\begin{ex}
Compute the sum
\[
1 + 4 + 7 + 10 + \cdots + 100
\]
First write it as a summation. Next attempt to rewrite it so that you
can see $\sum_{i=1}^n i$ (for some $n$) so that you can use the 
arithmetic sum formula.
\end{ex}


\newpage
\begin{ex} 
Write the following as a polynomial in decreasing power of $n$:
\[
\sum_{i=2}^n i
\]
\qed
\end{ex}


\newpage
\begin{ex} 
Write the following as a polynomial in decreasing power of $n$:
\[
\sum_{i=2}^{n-1} i
\]
\qed
\end{ex}



\newpage
\begin{ex} 
Write the following as a polynomial in decreasing power of $n$:
\[
\sum_{i=0}^{n-2} i
\]
\qed
\end{ex}



\newpage
Besides the arithmetic sum formula,
there is also a sum of squares formula:

\begin{prop}
\[
1^2 + 2^2 + \cdots + n^2 = \sum_{i=1}^n i^2 = \frac{n(n+1)(2n+1)}{6}
\]
\end{prop}

I won't prove the above formula.


\newpage
\begin{ex}
Check by hand that 
the sum of squares formula is correct for $n = 1, 2, 3, 4$.
\end{ex}



\newpage
\begin{ex}
Compute the sum $1^2 + 2^2 + 3^2 + \cdots + 1000^2$.
\qed
\end{ex}



\newpage
\begin{ex}
Write the following as a polynomial in decreasing power of $n$:
\[
\sum_{i=1}^{n+1} i^2
\]
\qed
\end{ex}



\newpage
\begin{ex}
Write the following as a polynomial in decreasing power of $n$:
\[
\sum_{i=0}^{n-1} i^2
\]
\qed
\end{ex}


\newpage
Actually there are also formulas for 
\[
\sum_{i=1}^n i^3, \,\,\,\,\,
\sum_{i=1}^n i^4, \,\,\,\,\,
\sum_{i=1}^n i^5, ...
\]
Google for their formulas and their stories.
