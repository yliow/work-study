%-*-latex-*-
\section{$n + c = O(n)$ for $c < 0$}
Now let's look at a case where the constant term is negative.
Let's say that 
\[
f(n) = n + 1
\]
and 
\[
g(n) = n - 10
\]
Can we show that $f(n) = O(g(n))$?

We need to find $C$ and $N$ such that for $n \geq N$ we have
\[
|f(n)| \leq C|g(n)|
\]
i.e.,
\[
|n + 1| \leq C|n - 10|
\]

To make my life easier, I want to first remove the absolute values.
This means that I want $n + 1$ and $n - 10$ to be positive.
For $n + 1$ to be positive, I need $n$ to be at least $-1$.
For $n - 10$ to be positive, I need $n$ to be at least $10$.
Altogether if $n \geq 10$, the above 
\[
|n + 1| \leq C|n - 10|
\]
is the same as 
\[
n + 1 \leq C(n - 10)
\]
It's clear that I can't choose $C = 1$.
Let's try $C = 2$.
If $C = 2$, the above inequality becomes:
\[
n + 1 \leq 2(n - 10) = 2n - 20
\]
which is the same as
\[
1 \leq n - 20
\]
which is the same as
\[
21 \leq n
\]

Altogether, I'm choosing $C = 2$, $n$ to be at least 10 and 
$n$ to be at least $21$.
If $n$ is at least $21$, then of course $n$ is at least $10$.
So if I set $N = 21$, then $n \geq N$ will guarantee that 
$n$ is at least 21.

Now for the solution ...


\begin{eg}
Let
$f(n) = n + 1$
and 
$g(n) = n - 10$.
Then $f(n) = O(g(n))$.
\end{eg}

\textit{Solution.}
Let $C = 2$ and $N = 21$.
Let $n \geq N = 21$.
Then we have
\begin{align*}
21 &\leq n \\
\THEREFORE -1 &\leq n \\
\THEREFORE 0 &\leq n + 1 \\
\THEREFORE |n + 1| &= n + 1 \tag{1}
\end{align*}
and
\begin{align*}
21 &\leq n \\
\THEREFORE 10 &\leq n \\
\THEREFORE 0 &\leq n - 10 \\
\THEREFORE |n - 10| &= n - 10 \tag{2}
\end{align*}
Furthermore we also have
\begin{align*}
21 &\leq n \\
\THEREFORE 21 - 20 &\leq n - 20 \\
\THEREFORE 1 &\leq n - 20 \\
\THEREFORE n + 1 &\leq n + n - 20 \\
\THEREFORE n + 1 &\leq 2n - 20 \\
\THEREFORE n + 1 &\leq 2(n - 10) \\
\THEREFORE |n + 1| &\leq 2|n - 10| & & \text{by (1) and (2)}\\
\THEREFORE |f(n)| &\leq C|g(n)| \\
\end{align*}
Therefore $f(n) = O(g(n))$. 
\qed

\begin{ex}
Let
$f(n) = n + 1$
and 
$g(n) = n - 20$.
Show that $f(n) = O(g(n))$.
\qed
\end{ex}

\begin{ex}
Let
$f(n) = n - 20$
and 
$g(n) = n - 100$.
Show that $f(n) = O(g(n))$.
\qed
\end{ex}

\begin{ex}
Let $f(n) = n + a$ where $a$ is a real number and $g(n) = n$.
Show that $f(n) = O(g(n))$, i.e.,
\[
n + a = O(n)
\]
[NOTE: This follows from the previous execise.]
\qed
\end{ex}

\begin{ex}
Let
$f(n) = n + a$
and 
$g(n) = n + b$
where $a$ and $b$ are real numbers.
Show that $f(n) = O(g(n))$.
\qed
\end{ex}

