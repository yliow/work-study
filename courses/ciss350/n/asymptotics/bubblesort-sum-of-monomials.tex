%-*-latex-*-
\section{Bubblesort again: big-O of Sums}

The next theorem tells us two things:
you can compute the big-O of sums of $i^k$ and 
that change the limits of summations by a constant amount
does not change the big-O.
I'll do the first fact in this section.

\begin{thm} 
If $k \geq 0$ is an integer, then
\[ \sum_{i=1}^{n} i^k = O(n^{k+1}) \]
\end{thm}

You already know that
\[ \sum_{i=1}^{n} i = \frac{n(n+1)}{2} \]
therefore the big-O is $O(n^2)$.
So it's not too surprising that for higher powers:
\[ \sum_{i=1}^{n} i^k, \,\,\,\,\, k > 1 \]
there are actually exact formulas.
However we can get away without these exact formulas 
(which are very complicated) since we only care
about big-O.
Here's how you can prove the theorem.


\textit{Proof.}
For each $i$ in the sum, 
\[
1 \leq i \leq n
\]
Therefore
\[
i^k \leq n^k
\]
We conclude that
\begin{align*}
\sum_{i=1}^n i^k 
&\leq \sum_{i=1}^n n^k \\
&= n^k \sum_{i=1}^n 1 \\
&= n^kn \\
&= n^{k+1}
\end{align*}
Hence
\[
\sum_{i=1}^n i^k = O(n^{k+1})
\]
\qed


\begin{eg}
For instance the theorem tells us that 
\begin{align*}
\sum_{i = 1}^{n} i &= O(n^2) \\
\sum_{i = 1}^{n} i^{2} &= O(n^3) \\
\sum_{i = 1}^{n} i^{100} &= O(n^{101})
\end{align*}
\qed
\end{eg}


\begin{ex} Show that
$\sum_{i = 1}^n \ln i = O(n \ln n)$
\qed
\end{ex}

[MORE EXAMPLES + EXERCISES]
