%-*-latex-*-
\section{Quicksort: LRTP partition method}

The strategy of the above inplace quicksort is to to use the following
layout: pivot, \verb!left!, \verb!TODO!, \verb!right!:
\begin{python}
from latextool_basic import *
p = Plot()
s = chunkedarray(cellwidth=0.4, 
                   cellheight=0.4,
                   pipeheight=0.8,
                   arr=[8*[''], ['p'],3*[''],6*[''],4*[''],10*['']],
                   chunklabels=['', '', 'left', 'TODO', 'right'],
                   celllabels = [('start', 8, -2),
                                 ('end',   21, -2),
                                 ('thead', 12, -1),
                                 ('ttail', 17, -1),
                                ]
)
p.add(s)
print(p)
\end{python}

Here's a partition method that you will find in many textbooks
where the layout of the strategy is:
\verb!left!, \verb!right!, \verb!TODO!, pivot.
\begin{python}
from latextool_basic import *
p = Plot()
s = chunkedarray(cellwidth=0.4, 
                   cellheight=0.4,
                   pipeheight=0.8,
                   arr=[8*[''], 3*[''],6*[''],4*[''],['p'],10*['']],
                   chunklabels=['', 'left', 'right', 'TODO',('pivot',1.2)],
                   celllabels = [('start', 8, -1),
                                 ('end',   21, -1),],
)
p.add(s)
print(p)
\end{python}
To mark and keep track of the different sections from \verb!start! to
\verb!end!, you can use the following variables to mark the beginning
of the \verb!left!, \verb!right!, and \verb!TODO!:
\begin{python}
from latextool_basic import *
p = Plot()
from quicksort_init import *
s = chunkedarray(cellwidth=0.4, 
                   cellheight=0.4,
                   pipeheight=0.8,
                   arr=[8*[''], 3*[''],6*[''],4*[''],['p'],10*['']],
                   chunklabels=['', 'left', 'right', 'TODO',''],
                   celllabels = [('lhead', 8, -1),
                                 ('rhead', 11, -1),
                                 ('thead', 17, -1),
                                ],
)
p.add(s)
print(p)
\end{python}
Note that \verb!lhead! is redundant since it's the same as
\verb!start!, but it's easier if I use it to describe the partitioning
algorithm for this section.


\begin{ex}
At the beginning, you have to move the pivot.
What's the pseudocode? (This is easy.)
\qed
\end{ex}

Here's how to process the first (i.e. leftmost) value \verb!v!
in the \verb!TODO! list:
\begin{python}
from latextool_basic import *
p = Plot()
from quicksort_init import *
s = chunkedarray(cellwidth=0.4, 
                   cellheight=0.4,
                   arr=[8*[''], 3*[''],6*[''],['v','','',''],['p'],10*['']],
                   chunklabels=['', 'left', 'right', 'TODO',''],
)
p.add(s)
print(p)
\end{python}

If the value should go into the \verb!right! chunk
(i.e., \verb!v > p!), then it's already there.
I just have to increment the variable \verb!thead! that keeps track of the 
beginning of \verb!TODO!.
This sequence of pictures will tell you what to do:
\begin{python}
from latextool_basic import *
p = Plot()
def xyz(x, y, xs):
    return chunkedarray(x=x, y=y,
                        cellwidth=0.4, cellheight=0.4,
                        arr=xs,
                        chunklabels=['', 'left', 'right', 'TODO',''])

p.add(xyz(0, 0, [8*[''], 3*[''],6*[''],['v','','',''],['p'],10*['']]))
p.add(xyz(0, -2, [8*[''], 3*[''],6*['']+['v'],['','',''],['p'],10*['']]))
print(p)
\end{python}

If the value should go into the \verb!left! chunk (i.e., \verb!v <= p!), 
I do the following: I swap \verb!v!
with the leftmost value of the \verb!right! chunk,
and increment \verb!rhead! and \verb!thead!.
\begin{python}
from latextool_basic import *

def xyz(x, y, xs):
    return chunkedarray(x=x, y=y,
                        cellwidth=0.4, cellheight=0.4,
                        arr=xs,
                        chunklabels=['', 'left', 'right', 'TODO',''])

p = Plot()
p.add(xyz(0, 0, [8*[''], 3*[''],['?']+5*[''],['v','','',''],['p'],10*['']]))
p.add(xyz(0, -2, [8*[''], 3*[''],['v']+5*[''],['?','','',''],['p'],10*['']]))
p.add(xyz(0, -4, [8*[''], 3*['']+['v'],5*['']+['?'],['','',''],['p'],10*['']]))
print(p)
\end{python}
Note that in the first of the above three pictures,
the value \verb!?! was within the \verb!right! chunk
at the head.
In the third picture above, 
\verb!?! is still in the \verb!right! chunk
(which is what you should expect!) but it's at the tail.

\newpage
\begin{ex}
In the above, I assumed that \verb!right!
is not empty.
What happens if the \verb!right! is empty?
What's the right thing to do?
\qed
\end{ex}


\newpage
\begin{ex}
At the end, you have to swap the pivot value from index \verb!end! 
with a value at another index position.
Where is this other index position?
Are you sure you have considered \textit{all} cases?
\qed
\end{ex}


\newpage
\begin{ex}
\begin{tightlist}
\item Using the partitioning strategy in this section,
execute one single partitioning on the following array
from index \verb!start = 3! to index \verb!end = 8!
choosing the value at index 4 as pivot.

\begin{python}
from latextool_basic import *
p = Plot()
p += Array2d(0,0,[[3,1,5,2,7,8,4,6,9,0]],width=0.7,height=0.7)
print(p)
\end{python}

Show the state of your array after moving the pivot, at the end of
processing each value in the \verb!TODO! section of the array,
and after moving the pivot to the right place.
\item For each of the step from (a) (except for the first and last),
write down the value of \verb!lhead!, \verb!rhead!, and \verb!thead!.
\end{tightlist}
\end{ex}

\SOLUTION

1.
\begin{python}
from latextool_basic import *

def xyz(x, y, xs):
    return chunkedarray(x=x, y=y, cellwidth=1.0, cellheight=0.5, arr=xs)

p = Plot()
p.add(xyz(0,  0, [[3,1,5,2,7,8,4,6,9,0]]))
p.add(xyz(0, -1, [[3,1,5],[],[],[2,9,8,4,6],[7],[0]]))
p.add(xyz(0, -2, [[3,1,5],[2],[],[9,8,4,6],[7],[0]]))
p.add(xyz(0, -3, [[3,1,5],[2],[9],[8,4,6],[7],[0]]))
p.add(xyz(0, -4, [[3,1,5],[2],[9,8],[4,6],[7],[0]]))
p.add(xyz(0, -5, [[3,1,5],[2,4],[8,9],[6],[7],[0]]))
p.add(xyz(0, -6, [[3,1,5],[2,4,6],[9,8],[],[7],[0]]))
p.add(xyz(0, -7, [[3,1,5,2,4,6,7,8,9,0]]))
print(p)
\end{python}

2. DIY.
\qed


\newpage
\begin{ex}
\begin{tightlist}
\item Using the partitioning strategy in this section,
execute one single partitioning on the following array
from index \verb!start = 2! to index \verb!end = 8!
choosing the value at index 4 as pivot.

\begin{python}
from latextool_basic import *
p = Plot()
p += Array2d(0,0,[[0,1,2,3,4,5,6,7,8,9]],width=0.7,height=0.7)
print(p)
\end{python}

Show the state of your array after moving the pivot, at the end of
processing each value in the \verb!TODO! section of the array,
and after moving the pivot to the right place.
\item For each of the step from (a) (except for the first and last),
write down the value of \verb!lhead!, \verb!rhead!, and \verb!thead!.
\end{tightlist}
\qed
\end{ex}


\newpage
\begin{ex}
\begin{tightlist}
\item Using the partitioning strategy in this section,
execute one single partitioning on the following array
from index \verb!start = 2! to index \verb!end = 8!
choosing the value at index 4 as pivot.

\begin{python}
from latextool_basic import *
p = Plot()
p += Array2d(0,0,[[9,8,7,6,5,4,3,2,1,0]],width=0.7,height=0.7)
print(p)
\end{python}

Show the state of your array after moving the pivot, at the end of
processing each value in the \verb!TODO! section of the array,
and after moving the pivot to the right place.
\item For each of the step from (a) (except for the first and last),
write down the value of \verb!lhead!, \verb!rhead!, and \verb!thead!.
\end{tightlist}
\qed
\end{ex}

\newpage
\begin{ex}
\begin{tightlist}
\item Using the partitioning strategy in this section,
execute one single partitioning on the following array
from index \verb!start = 2! to index \verb!end = 8!
choosing the value at index 2 as pivot.

\begin{python}
from latextool_basic import *
p = Plot()
p += Array2d(0,0,[[7,8,9,6,0,3,4,2,1,5]],width=0.7,height=0.7)
print(p)
\end{python}

Show the state of your array after moving the pivot, at the end of
processing each value in the \verb!TODO! section of the array,
and after moving the pivot to the right place.
\item For each of the step from (a) (except for the first and last),
write down the value of \verb!lhead!, \verb!rhead!, and \verb!thead!.
\end{tightlist}
\qed
\end{ex}

\newpage
\begin{ex}
\begin{tightlist}
\item Using the partitioning strategy in this section,
execute one single partitioning on the following array
from index \verb!start = 2! to index \verb!end = 8!
choosing the value at index 4 as pivot.

\begin{python}
from latextool_basic import *
p = Plot()
p += Array2d(0,0,[[7,8,9,6,0,3,4,2,1,5]],width=0.7,height=0.7)
print(p)
\end{python}

Show the state of your array after moving the pivot, at the end of
processing each value in the \verb!TODO! section of the array,
and after moving the pivot to the right place.
\item For each of the step from (a) (except for the first and last),
write down the value of \verb!lhead!, \verb!rhead!, and \verb!thead!.
\end{tightlist}
\qed
\end{ex}


\newpage
\begin{ex}
\begin{tightlist}
\item Using the partitioning strategy in this section,
execute one single partitioning on the following array
from index \verb!start = 0! to index \verb!end = 2!
choosing the value at index 0 as pivot.

\begin{python}
from latextool_basic import *
p = Plot()
p += Array2d(0,0,[[0,9,9]],width=0.7,height=0.7)
print(p)
\end{python}

Show the state of your array after moving the pivot, at the end of
processing each value in the \verb!TODO! section of the array,
and after moving the pivot to the right place.
\item Is quicksort (using the partition strategy of this section)
a stable sorting algorithm?
\qed
\end{tightlist}
\end{ex}


\newpage
\begin{ex}
Describe arrays and pivot values in the arrays that will become
sorted when you execute one pass of the partition strategy of this section.
\qed
\end{ex}


\newpage
\begin{ex}
Describe sorted arrays and pivot values in the arrays that will become
unsorted when you execute one pass of the partition strategy of this section.
\qed
\end{ex}
