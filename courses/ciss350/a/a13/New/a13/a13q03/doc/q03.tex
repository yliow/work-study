Implementation of sets using hashtables

There are times when you might want a container to keep
keys instead of (key, value) pairs.
Think for instance of a set.
A set (just like a set in your math classes -- you first saw
sets in precalc) is just a container of value
where you want to:
\begin{tightlist}
  \item Check if a value in the set (or not).
  \item Add a value into the set.
  \item Remove a value from the set.
\end{tightlist}
(and possibly other basic operations such as the size of the set, etc.)
Note that you are only interested in whether a value is in the set --
you don't really care how many times a value occurs in the set.
Therefore sets do not have duplicates.
You also do not care about the ordering of values in the set --
you never ask what is the value before or after this value.


You can redevelop the hashtable or you can also do this:
For a set of \verb!std::string! objects, you are looking at a 
hashtable with \verb!std::string! for key type and your
value type be any any type and you can just put in some dummy data
for the value part.
(Obviously you want to put in something that's small -- like an
integer.
As a personal project, you can optimize by redoing Q1 without the value part.)

