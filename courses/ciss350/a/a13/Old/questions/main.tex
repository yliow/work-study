%-*-latex-*-
\input{myassignmentpreamble}
\input{yliow}
\input{ciss350}
\renewcommand\TITLE{Assignment 13}


\usepackage{listings}
\lstset{%
basicstyle=\ttfamily, frame=single
}


\begin{document}
\topmatter


\textsc{Objectives:}
\begin{enumerate}[topsep=0pt]
\item Implement as hashtable using separate chaining with
  \verb!std::vector! 
  and \verb!std::list!.
\item Implement a set using a hashtable.
\item Implement a sparse matrix using a hashtable.
\end{enumerate}

\textsc{Warning}:
As always skeleton code is meant to get you started.
Skeleton code needs to be corrected and completed.

SPRING 2019: OMIT Q2-Q5.

\textsc{Submission}.
Email your work to \verb!yliow@ccis.edu!.
Use your college email account.
The subject line of the email must be \verb!ciss350 a13!.

\newpage
Q1. Implementation of hashtable with separate chaining using
linked lists

Implement a hashtable where each record has a key of type 
\verb!std::string! and value of type \verb!double!.
Each bucket is a \verb!std::list! of (key, value) pairs.

The following is skeleton code that you should begin with (HT is hashtable for short
to save on writing):
{\small
\includesourcenonumbers{main.cpp}
}


\newpage
Q2. Implementation of template hashtable with separate chaining

The hashtable in Q1 accepts keys of \verb!std::string! type
and values of \verb!double! type.
Modify the hashtable from Q1 so that is a class template of the following
form:
\begin{console}
class < typename Key, typename Value >
class HT
{
    ...
};
\end{console}
so that
\begin{console}
HT< std::string, double >
\end{console}
becomes the class in Q1.

But wait!

Instead of having hash functions all over the place, it's better to tie
hash functions into the classes that we're using as \verb!Key!.

In fact, for practice, create a \verb!Hashable! abstract base class
and create a class \verb!HashableString! that subclasses
\verb!std::string! and \verb!Hashable!.
\verb!Hashable! has a pure virtual method \verb!hash!.
In that case
{\small
\begin{console}
HT< HashableString, double >
\end{console}
}
is what you have in Q1.
By doing it this way, you can have different hash functions
for different hashtables for even the same key type.
For instance suppoer you have a hashtable with keys 
made up of SSN and the SSNs you're putting into your hashtable
always begin with the same 3 characters -- you might want to ignore
the first 3 characters when computing the hash.

[Note: You should also modify your \verb!HTRow! class as well.
Note also that for different \verb!Key! types you need different
hash functions.]

SPOILERS ON NEXT PAGE ...

\newpage

{\small
\begin{console}
#include <string>

class Hashable
{
public:
    virtual unsigned int hash() = 0;
};


class HashableString: public std::string, Hashable
{
public:
    HashableString(const std::string & s)
        : std::string(s)
    {}
    
    unsigned int hash()
    {
        unsigned int h = 0;
        unsigned int power = 1;
        for (unsigned int i = 0; i < length(); i++)
        {
            h += operator[](i) * power; power *= 10;
        }
        return h;
    }
};
\end{console}
}


\newpage
Q3. Implementation of sets using hashtables

There are times when you might want a container to keep
keys instead of (key, value) pairs.
Think for instance of a set.
A set (just like a set in your math classes -- you first saw
sets in precalc) is just a container of value
where you want to:
\begin{tightlist}
  \item Check if a value in the set (or not).
  \item Add a value into the set.
  \item Remove a value from the set.
\end{tightlist}
(and possibly other basic operations such as the size of the set, etc.)
Note that you are only interested in whether a value is in the set --
you don't really care how many times a value occurs in the set.
Therefore sets do not have duplicates.
You also do not care about the ordering of values in the set --
you never ask what is the value before or after this value.


You can redevelop the hashtable or you can also do this:
For a set of \verb!std::string! objects, you are looking at a 
hashtable with \verb!std::string! for key type and your
value type be any any type and you can just put in some dummy data
for the value part.
(Obviously you want to put in something that's small -- like an
integer.
As a personal project, you can optimize by redoing Q1 without the value part.)

{\small
\begin{Verbatim}[frame=single]
class < typename T >
class Set
{
public:

    // Adds x to d is x is not in d.
    void insert(const T & x)
    {}

    // Removes x from d if x is in d. Otherwise KeyError is thrown.
    void remove(const T & x)
    {}

    // Returns true if x is in d
    bool has_element(const T & x) const
    {}

    // Returns true if each value in d is also in set.d
    bool issubset(const Set< T > & set) const
    {}

    // Same as issubset
    bool operator<=(const Set< T > & set) const
    {}

    // Returns true if each value in set.d is also in d
    bool issuperset(const Set< T > & set) const
    {}

    // Same as issuperset
    bool operator>=(const Set< T > & set) const
    {}

    // Returns true if *this and set contains the same values
    bool operator==(const Set< T > & set) const
    {}

private:
    HT< T, char > d;
};

// If set contains "a", "b", print as {a, b}
template < typename T >
std::ostream & operator<<(std::stream & cout, const Set< T > set)
{}
\end{Verbatim}
}

\textsc{Notes.}
\begin{tightlist}
  \item  C++ STL set (\verb!std::set!) is implemented as a tree and not
  hashtable.
  \item Python implements its
  \verb!set! class (and also \verb!frozenset! class)
  by using its dictionary (which is a hashtable)
  with the value part occupied by some dummy value.
\end{tightlist}


\newpage
Q4. Comparing hashtable set and BST set

Using the implementation from Q3, create a set of 100,000 random integers
and find the time taken to find 1,000 values in this set.
Next, use your BST and do the same.
Next repeat the above with
a set of 200,000 random integers.
Etc.
Compare the time and plot a graph.




\newpage
Q5. Implementation of sparse matrix using hashtable

Matrices are basically 2d arrays.
They appear everywhere in CS and math.
In real-life applications matrices are extremely huge, possibly
going up a size of to thousands by thousands.
Sometimes a matrix can be full of zeroes -- this is very common in
real life applications.
Such matrices are called sparse matrices.
There are many different ways to represent sparse matrices -- see
linked list notes for an implementation using doubly linked list.
Another way to represent a sparse matrix is to use hashtables.
In this case the key is the (row,column) of a value.
For instance if the matrix is a 2-by-2:
\[
\texttt{\{\{5,0\},\{0,7\}\}}
\]
Then at (row,column) = $(0,0)$, the value is $5$:
the key is $(0,0)$ and the
value is $5$.

Create a class \texttt{SparseMatrix} with doubles as values.
\begin{console}
SparseMatrix m(1000, 2000); // 1000-by-2000
m.set(10, 20, 3.1415);      // set value at row 10, column 20 to 3.1415
std::cout << m.get(10, 20); // prints 3.1415
std::cout << m.get(0, 0);   // prints 0 -- the default value
\end{console}
The implementation should use hashtables with separate chaining.
(Modify Q1.)
For the hash function, do the following:
say the (row, column) is (4, 5). First convert the key to the string
\[
\texttt{"4,5"}
\]
(note the comma) and then use the method from Q1.
Attempt to access a value with row or column outside the sizes
specified using constructor call will result in 
the exception object from class \texttt{IndexError} being thrown.

Note several things:
\begin{itemize}
  \item The runtime to access a value in the sparse matrix is
  (in the best case) $O(1)$ and the runtime to access the matrix as a
  2D array is also $O(1)$.
  \item 
  However if the matrix has a large rowsize and columnsize, then
  the 2d array method will use more memory.
  \item In fact if the rowsize and columnsize is too huge, the
  total size of the array (i.e., rowsize $\times$ columnsize)
  could be too big for the computer to handle -- there's a limit
  to how much memory is allowed for an array.
  In this case the hashtable can actually work.
\end{itemize}



\end{document}
