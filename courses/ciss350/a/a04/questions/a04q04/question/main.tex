%-*-latex-*-
The goal is to compute the big-$O$ of the best and worst runtime performance
of the following algorithm. 
Here's the algorithm:
\begin{Verbatim}[frame=single]
INPUT:  x - array of doubles
        n - size of x
        a - int
OUTPUT: None
ALGORITHM:

    a = x[0] / 2                           

    for i = 0, 1, 2, ..., 100000:

        for j = 0, 1, 2, ..., n-1:
            x[j] = x[j] + a                

        for j = 0, 1, 2, ..., (n - 1)/2:
            if x[j] < 5:                   
                x[j] = x[j] * x[0]         

\end{Verbatim}

\begin{Verbatim}[frame=single]
INPUT:  x - array of doubles
        n - size of x
        a - int
OUTPUT: None
ALGORITHM:

          a = x[0] / 2            time t1

          i = 0                   time t2           
LOOP1:    if i > 100000:          time t3
              goto ENDLOOP        time t4

          j = 0                   time t5
LOOP2:    if j > n - 1:           time t6
              goto ENDLOOP2       time t7
          x[j] = x[j] + a         time t8
          j = j + 1               time t9
          goto LOOP2              time t10
ENDLOOP2:
          j = 0                   time t11
LOOP3:    if j > (n - 1)/2:       time t12
              goto ENDLOOP3       time t13

          if x[j] >= 5:           time t14
              goto ELSE           time t15
          x[j] = x[j] * x[0]      time t16
ELSE:     j = j + 1               time t17
          goto LOOP3              time t18
ENDLOOP3:
          i = i + 1               time t19
          goto LOOP1              time t20
ENDLOOP1:
\end{Verbatim}


(a) Write down $T_w(n)$ in terms of $n$ and the $t_1$, $t_2$,....
You should write it as a polynomial of $n$ from the highest
degree term to the lowest.

(b) Write down the big-$O$ of $T_w(n)$ as $O(n^k)$
where $k$ is the smallest possible  positive integer.

(c) Write down the big-$O$ of $T_b(n)$ as
$O(n^k)$
where $k$ is the smallest possible  positive integer.
(You are strongly advised to compute the
 precise $T_b(n)$ on your own of course.
For grading purposes, you just have to write down the big-O.)


\newpage
\textsc{solution}\\
(a)
The timings with the number of times a statement is executed
is as follows (complete it like the solution given earlier):
\begin{Verbatim}[frame=single]
          a = x[0] / 2            time t1

          i = 0                   time t2           
LOOP1:    if i > 100000:          time t3
              goto ENDLOOP        time t4

          j = 0                   time t5
LOOP2:    if j > n - 1:           time t6
              goto ENDLOOP2       time t7
          x[j] = x[j] + a         time t8
          j = j + 1               time t9
          goto LOOP2              time t10
ENDLOOP2:
          j = 0                   time t11
LOOP3:    if j > (n - 1)/2:       time t12
              goto ENDLOOP3       time t13

          if x[j] >= 5:           time t14
              goto ELSE           time t15
          x[j] = x[j] * x[0]      time t16
ELSE:     j = j + 1               time t17
          goto LOOP3              time t18
ENDLOOP3:
          i = i + 1               time t19
          goto LOOP1              time t20
ENDLOOP1:
\end{Verbatim}

