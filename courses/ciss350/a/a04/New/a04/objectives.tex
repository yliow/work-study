\textsc{Objectives}
\begin{enumerate}[nosep]
\item Computation of big-$O$ of the worst case runtime performance
a given algorithm.
\item Computation of big-$O$ of the best case runtime performance
a given algorithm.
\item Computation of big-$O$ of the average case runtime performance
a given algorithm.
\end{enumerate}
\mbox{}

\textred{In your Fedora virtual machine, as \texttt{student} user, execute
  the following in your bash shell: \texttt{pip install scipy --user}.}

For this assignment, you will need to modify q02.tex, q03.tex, etc.
q01.tex has a complete solution for your reference.



Given an algorithm, 
$T_w(n)$ denotes the worst runtime, 
$T_a(n)$ the average runtime, and
$T_b(n)$ the best runtime.
$T(n)$ denotes the worst runtime
of the algorithm.

Recall that 
\[
An^2 + Bn + C = O(n^2)
\]
and
\[
Bn + C = O(n)
\]
and
\[
C = O(1)
\]
where $A, B$, and $C$ are constants.
In general
\[
a_dn^d + a_{d-1} n^{n-1} + \cdots + a_0 = O(n^d)
\]
where $a_i$'s are constants.
Of course it's also true that
\[
3n^2 - 5n + 10 = O(n^3)
\]
and
\[
3n^2 - 5n + 10 = O(n^{3.5})
\]
and in fact
\[
3n^2 - 5n + 10 = O(n^{k})
\]
for any real number $k \geq 2$.
But the practice is to provide the best (i.e. tightest) bound.

For grading purposes, you must follow the following instructions:
\begin{enumerate}[nosep]
\item When assigning times $t_i$, start with $t_1$ (not $t_0$)
and use $t_1, t_2, ...$ without skipping any index.
\item When writing down exact runtimes, list the term with the highest
growth rate. For instance, do not write
\[
T(n) = t_1 + (t_2 + t_6)n^{100} + t_3 n^{15.25}
\]
Instead, write this:
\[
T(n) = (t_2 + t_6)n^{100} + t_3 n^{15.25} + t_1 
\]
\item When writing the constants for each term in the runtime, arrange the
constants in ascending index values. For instance, do not write this:
\[
T(n) = (t_3 + t_1 + t_2) n^{100} + (t_7 + t_2 + t_1) n^{15.25}
\]
Instead, write this:
\[
T(n) = (t_1 + t_2 + t_3) n^{100} + (t_1 + t_2 + t_7) n^{15.25}
\]
\item Write legibly and clearly. If there's any ambiguity, then I
reserve the right to interpret what you are trying to convey.
And I am usually very good at picking the wrong interpretation.
\item Good writing means clear and unambiguous writing.
\lq\lq $x = 42$'' is clearer than \lq\lq $42$''.
It does not even take that much time (nor waste ink) to write the former.
If I see any ambiguity like the above, you will get a 0 for the whole question.

\item
There's a big difference between \lq\lq $x = 42$'' and \lq\lq $x \, \, 42$''.
No one would say
a\lq\lq John tall'' when it should be
``John is tall''.
If I see something similar to
\lq\lq $x \, \, 42$'', you will get a 0 for whole question.

\item In general, if I see any obviously bad/improper mathematical writing,
I will give you a zero.

\end{enumerate}
