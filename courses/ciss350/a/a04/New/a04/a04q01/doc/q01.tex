%-*-latex-*-

\textsc{Solution provided.}


This is a solved problem.
All the big-O computations in this assignment
are somewhat similar to this question.
Therefore study the solution very carefully.
It will help you present your solutions in \LaTeX.

The goal is to compute the big-$O$ of the runtime performance
of the following algorithm. 
Here's the algorithm:
\begin{Verbatim}[frame=single]
INPUT:  x - array of doubles
        n - size of x        
OUTPUT: result is stored in z
ALGORITHM:

    for i = 0, 1, 2, ..., n - 1:
        z = z + x[i] * x[i]
\end{Verbatim}
This is rewritten with timings as follows:
\begin{Verbatim}[frame=single]
INPUT:  x - array of doubles
        n - size of x
OUTPUT: result is stored in z
ALGORITHM:                      
         i = 0                   time t1     
LOOP:    if i >= n:              time t2      
             goto ENDLOOP        time t3      
         z = z + x[i] * x[i]     time t4      
         i = i + 1               time t5
         goto LOOP               time t6
ENDLOOP:
\end{Verbatim}
The only relevant timing of the statements are given.


(a) Write down $T(n)$ in terms of $n$ and the 
$t_1$, $t_2$, $t_3$, ....
You should write it as a polynomial of $n$ from the highest
degree term to the lowest.

(b) Write down the big-$O$ of $T(n)$ as $O(n^k)$
where $k$ is the smallest possible  positive integer.
\newpage
\textsc{Solution}
(a)
The timings with the number of times a statement is executed is as follows:
\begin{Verbatim}[frame=single]
         i = 0                   time t1    1
LOOP:    if i >= n:              time t2    n + 1
             goto ENDLOOP        time t3    1 
         z = z + x[i] * x[i]     time t4    n
         i = i + 1               time t5    n
         goto LOOP               time t6    n
ENDLOOP:
\end{Verbatim}
Therefore
\[
T(n) = (t_2 + t_4 + t_5 + t_6)n + (t_1 + t_2 + t_3)
\]

(b)
We have
\[
T(n) = O(n)
\]
\qed
