%-*-latex-*-
The goal is to compute the big-$O$ of the runtime performance
of the following algorithm. 
Here's the algorithm:
\begin{Verbatim}[frame=single]
INPUT:  x - array of doubles
        n - size of x
        a - double
        b - int
OUTPUT: result is stored in z
ALGORITHM:
    z = a + b * b                          

    for i = 0, 1, 2, ..., n-1:
        a = a + 1                          
        x[i] = a + z + b;                  
        z = z + 1                          

    for i = 0, 1, 2, ..., (n-1)/4:
        x[i] = x[i] * z                    

    z = z + n                              
\end{Verbatim}
This is rewritten with timings as follows:
\begin{Verbatim}[frame=single]
INPUT:  x - array of doubles
        n - size of x
        a - double
        b - int
OUTPUT: z

ALGORITHM:
          z = a + b * b                      time t1

          i = 0                              time t2
LOOP1:    if i >= n:                         time t3
              goto ENDLOOP1                  time t4 
          a = a + 1                          time t5
          x[i] = a + z + b;                  time t6
          z = z + 1                          time t7
          i = i + 1                          time t8
          goto LOOP1                         time t9
ENDLOOP1:

          i = 0                              time t10
LOOP2:    if i > (n - 1) / 4:                time t11
              goto ENDLOOP2                  time t12
          x[i] = x[i] * z                    time t13
          i = i + 1                          time t14
          goto LOOP2                         time t15
ENDLOOP2:
          z = z + n                          time t16   
\end{Verbatim}

(You can pretend that $n - 1$ is divisible by 4 so that
$(n - 1)/4$ is an integer.)

(a) Write down $T(n)$ in terms of $n$ and the 
$t_1$, $t_2$, $t_3$, ....
You should write it as a polynomial of $n$ from the highest
degree term to the lowest.

(b) Write down the big-$O$ of $T(n)$ as $O(n^k)$
where $k$ is the smallest possible  positive integer.
\newpage
\textsc{solution}\\
(a) The timings with the number of times a statement is executed
is as follows (complete it like the solution given earlier):
\begin{Verbatim}[frame=single]
          z = a + b * b                      time t1

          i = 0                              time t2
LOOP1:    if i >= n:                         time t3
              goto ENDLOOP1                  time t4
 
          a = a + 1                          time t5
          x[i] = a + z + b;                  time t6
          z = z + 1                          time t7
          i = i + 1                          time t8
          goto LOOP1                         time t9
ENDLOOP1:

          i = 0                              time t10
LOOP2:    if i > (n - 1) / 4:                time t11
              goto ENDLOOP2                  time t12
          x[i] = x[i] * z                    time t13
          i = i + 1                          time t14
          goto LOOP2                         time t15
ENDLOOP2:
          z = z + n                          time t16   
\end{Verbatim}


