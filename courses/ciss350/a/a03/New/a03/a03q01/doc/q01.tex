[Review of CISS245]

Write a matrix class of \texttt{double}s.
You should have addition, subtraction, multiplication,
scalar multiplication, etc.
You can also compare matrices: \verb!==! and \verb@!=@.
You can get the number of rows and number of columns using
\verb!m.nroww()!
and
\verb!m.ncols()!
If you are not familiar with matrix multiplication,
check online or see below.
Also, write the the determinant function.
\begin{Verbatim}[frame=single]
Mat m0(3, 4);    // 3-by-4 matrix
Mat m1(3, 4, 0); // 3-by-4 matrix of zeroes
m0(1, 2) = 5.0;  // Set row 1, column 2 to 5.0
Mat m2 = m0;     // The obvious copy constructor
m2 += m1;        // The obvious += operator
m2 = m2 + m1;
m2 -= m1;
m2 = m2 - m1;

Mat m3(4, 3, 2); // 4-by-3 matrix of 2s
Mat m4(3, 3);
m4 = m2 * m3;

m4 = 0.5 * m4;
m4 = m4 * 0.5;
m4 *= 0.5;

Mat m5 = Mat::identity(3); // m5 is 3-by-3 identity matrix

double x[] = {1.1, 2.2, 3.3, 4.4, 5.5, 6.6};
Mat m6(2, 3, x); // m6 is 2-by-3 filled with values from x
\end{Verbatim}
For output, if \verb!m! models this array
\[
\begin{bmatrix}
1 & 2 & 3.3 \\
4.56 & 5 & 6 
\end{bmatrix}
\]
the output should be
\begin{Verbatim}
[[   1 2 3.3]
 [4.56 5   6]]
\end{Verbatim}
(For the output, go online and check C++ string stream class.
You can cout the values of the matrix to a string stream object
which will then give you the lengths of the output of all entries.)

\newpage
Q1. Write a recursive function that implements a power function
recursively. You must use the following skeleton code which
contains tracing of your function.
The recursive power function below computes $a^n$ where $a$ has a 
\verb!double! value \verb!n! has an integer values that is $\geq 0$.
(You need not worry about the case when \verb!n! is $< 0$.)
\begin{Verbatim}[frame=single, fontsize=\small]
#include <iostream>

double pow_rec(double a, int n)
{
    if (n == -99999) // modify base condition
    {
        double ret = -99999; // set ret to correct value
        std::cout << "pow_rec(" << a << ", " << n << ") base case ... return "
                  << ret << '\n';
        return ret;
    }
    else
    {
        std::cout << "pow_rec(" << a << ", " << n << ") recursive case ...\n";
        double ret = -99999; // set ret to correct value
        std::cout << "pow_rec(" << a << ", " << n << ") recursive case ... "
                  << "return " << ret << '\n';
        return ret;        
    }
}

int main()
{
    double a;
    int n;
    std::cin >> a >> n;
    double p = pow_rec(a, n);
    std::cout << p << '\n';
    return 0;
}
\end{Verbatim}

Hint:
If $a$ is a real number say $3.1$ and $n$ is a positive say $5$.
You want to think of $3.1^5$ in terms of a smaller subproblem.
For instance is there a relationship between $3.1^5$ and $3.1^4$?

\textsc{Test 1}
\begin{Verbatim}[frame=single, fontsize=\small, commandchars=\\\{\}]
\underline{2 0}
pow_rec(2, 0) base case ... return 1
1
\end{Verbatim}

\textsc{Test 2}
\begin{Verbatim}[frame=single, fontsize=\small, commandchars=\\\{\}]
\underline{2 4}
pow_rec(2, 4) recursive case ...
pow_rec(2, 3) recursive case ...
pow_rec(2, 2) recursive case ...
pow_rec(2, 1) recursive case ...
pow_rec(2, 0) base case ... return 1
pow_rec(2, 1) recursive case ... return 2
pow_rec(2, 2) recursive case ... return 4
pow_rec(2, 3) recursive case ... return 8
pow_rec(2, 4) recursive case ... return 16
16
\end{Verbatim}

Make sure you create your own test cases as well.
