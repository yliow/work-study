Q4.
Write a recursive bubblesort function that implements
bubblesort recursively.
Specifically, \verb!bubblesort_rec(x, start, end)!
performs bubblesort on
\\
\verb!x[start]!,
\verb!x[start + 1]!,
\verb!x[start + 2]!, ...,
\verb!x[end - 1]!,

You must use the following skeleton code which
contains tracing of your function.

You must use a recursive relation between
\verb!bubblesort_rec(x, start, end)!
and
\\
\verb!bubblesort_rec(x, start, end - 1)!.

\begin{Verbatim}[frame=single, fontsize=\small]
#include <iostream>
#include <string>

void print(int x[], int end)
{
    std::string delim = "";
    std::cout << '{';
    for (int i = 0; i < end; ++i)
    {
        std::cout << delim << x[i];
        delim = ", ";
    }
    std::cout << '}';
}


void bubblesort_rec(int x[], int start, int end)
{
    if (start >= end - 1)
    {
        std::cout << "bubblesort_rec(";
        print(x, end);
        std::cout << ", " << start << ", " << end
                  << ") base case ... return\n";
        return;
    }
    else
    {
        std::cout << "bubblesort_rec(";
        print(x, end);
        std::cout << ", " << start << ", " << end
                  << ") recursive case ...\n";




        


        std::cout << "bubblesort_rec(";
        print(x, end);
        std::cout << ", " << start << ", " << end
                  << ") recursive case ... return\n";
        return;
    }
}

int main()
{
    int n;
    std::cin >> n;
    int * x = new int[n];
    for (int i = 0; i < n; ++i)
    {
        std::cin >> x[i];
    }
    bubblesort_rec(x, 0, n);
    print(x, n);
    std::cout << '\n';
    delete [] x;
    return 0;
}
\end{Verbatim}

\textsc{Test 1}
\begin{Verbatim}[commandchars=\~\!\@, fontsize=\small, frame=single]
~underline!5@
~underline!5 1 4 2 3@
bubblesort_rec({5, 1, 4, 2, 3}, 0, 5) recursive case ...
bubblesort_rec({1, 4, 2, 3}, 0, 4) recursive case ...
bubblesort_rec({1, 2, 3}, 0, 3) recursive case ...
bubblesort_rec({1, 2}, 0, 2) recursive case ...
bubblesort_rec({1}, 0, 1) base case ... return
bubblesort_rec({1, 2}, 0, 2) recursive case ... return
bubblesort_rec({1, 2, 3}, 0, 3) recursive case ... return
bubblesort_rec({1, 2, 3, 4}, 0, 4) recursive case ... return
bubblesort_rec({1, 2, 3, 4, 5}, 0, 5) recursive case ... return
{1, 2, 3, 4, 5}
\end{Verbatim}
(Clearly the inputs are the size of the array and the array.)

\textsc{Test 2}
\begin{Verbatim}[commandchars=\\\{\}, fontsize=\small, frame=single]
\underline{1}
\underline{5}
bubblesort_rec({5}, 0, 1) base case ... return
{5}
\end{Verbatim}

Make sure you create your own test cases as well.

Spoilers on next page.

\newpage
\textsc{Spoilers}

The base case is when there are either no values or only one value
to sort.
There is one value to sort if \verb!start! is \verb!end - 1!.
There are no values to sort if \verb!start! $>$ \verb!end - 1!.
Suppose recursively, \verb!bubblesort_rec(x, start, end)! calls
\\
\verb!bubblesort_rec(x, start, end - 1)!.
In that case, on return from
\\
\verb!bubblesort_rec(x, start, end - 1)!,
the values
\verb!x[start]!,
\verb!x[start + 1]!,
\verb!x[start + 2]!, ..., 
\verb!x[end - 2]!
are sorted.

For instance support \verb!start! is 0 and \verb!end! is 5.
\verb!bubblesort_rec(x, 0, 5)! is supposed to sort
\verb!x[0]!,
\verb!x[1]!,
\verb!x[2]!, 
\verb!x[3]!, 
\verb!x[4]!.
But since
\verb!bubblesort_rec(x, 0, 5)!
calls
\\
\verb!bubblesort_rec(x, 0, 4)!
that means that
\verb!bubblesort_rec(x, 0, 4)!
will sort
\verb!x[0]!,
\verb!x[1]!,
\verb!x[2]!,
\verb!x[3]!.
What must
\verb!bubblesort_rec(x, 0, 5)! do to 
\verb!x[0]!,
\verb!x[1]!,
\verb!x[2]!, ...
\verb!x[4]!.
before calling
\verb!bubblesort_rec(x, 0, 4)!?

\newpage
\textsc{Divide-and-conquer}

Make sure you review my notes on recursion (\#28).

In all the previous recursions, there's a common theme:
\begin{enumerate}[nosep]
\item Power computation: If I write \texttt{pow(x, n)} for $x$
to the power of $n$,
then the recursion is
\[
\text{\texttt{pow(x, n)}: uses \texttt{pow(x, n - 1)} (and \texttt{x})}
\]
\item Linear search: Let me write \texttt{x[0..n-1]}
to denote the array
of values \texttt{x[0],...,x[n-1]}
(and ignore implementation specifics
whether using array, \verb!std::vector!, pointer to array, etc.
If I write \verb!linearsearch(x[0..n-1], target)! for the
linear search of \verb!target!
in \verb!x[0..n-1]!, then the recursion is
{\small
\[
\text{\texttt{linearsearch(x[0..n-1], target)}: uses
\texttt{linearsearch(x[1..n-1], target)} (and \texttt{x[0]})}
\]
}
\item Bubblesort:  
If I write \verb!bubblesort(x[0..n-1])! for the bubblesort of
\verb!x[0..n-1]!, then the recursion is
\[
\text{\texttt{bubblesort of x[0..n-1]}: uses
\texttt{bubblesort(x[0..n-2])}
}
\]
\end{enumerate}
If $n$ denotes the size of a problem (in the case of the power computation
it's the exponent and in the case of array type problems it's the size of the
array), then in all the above the philosophy of recursion is this:
If $P(n)$ is a problem, then to solve $P(n)$ by recursion you might want to try
\[
\text{to solve $P(n)$: solve $P(n-1)$ (as well as some other computations)}
\]
For instance
\[
\text{to compute $x^n$: compute $x^{n-1}$ (and multiply it with $x$)}
\]
(Such recursions are called linear recursions of degree 1.)

A divide-and-conquer (DAC) recursion is \lq\lq usually" different.
The philosophy of DAC is like this:
\[
\text{to solve $P(n)$: solve one or two $P(n/2)$ problems (and do some computations)}
\]
The $n/2$ is not exactly $n/2$, but an integer value roughly $n/2$.

Note however that
some people consider \textit{both} methods,
\begin{tightlist}
\li $P(n)$ in terms of $P(n-1)$ and
\li $P(n)$ in terms of $P(n/2)$
\end{tightlist}
as DAC.
In other words for some people, as long as a problem is solved
in terms
of \lq\lq smaller" problem(s), it's a DAC solution,
whether it is smaller by 1 or smaller by a factor of $1/2$.

\end{document}
