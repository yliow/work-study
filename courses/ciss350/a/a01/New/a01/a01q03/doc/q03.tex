%-*-latex-*-
Write a program that accepts and prints the prime factorization
of the number. (Do not use array.)
[You definitely want to look at the review problem set as preparation
for this problem.]

\textsc{Test 1}
\begin{console}[commandchars=\\\{\}]
\userinput{1}
1
\end{console}

\textsc{Test 2}
\begin{console}[commandchars=\\\{\}]
\userinput{2}
2
\end{console}

\textsc{Test 3}
\begin{console}[commandchars=\\\{\}]
\userinput{3}
3
\end{console}

\textsc{Test 4}
\begin{console}[commandchars=\\\{\}]
\userinput{4}
2^2
\end{console}

\textsc{Test 5}
\begin{console}[commandchars=\\\{\}]
\userinput{5}
5
\end{console}

\textsc{Test 5}
\begin{console}[commandchars=\\\{\}]
\userinput{6}
2 * 3
\end{console}

\textsc{Test 6}
\begin{console}[commandchars=\\\{\}]
\userinput{8}
2^3
\end{console}

\textsc{Test 7}
\begin{console}[commandchars=\\\{\}]
\userinput{9}
3^2
\end{console}

\textsc{Test 8}
\begin{console}[commandchars=\\\{\}]
\userinput{10}
2 * 5
\end{console}

\textsc{Test 9}
\begin{console}[commandchars=\\\{\}]
\userinput{12}
2^2 * 3
\end{console}

\textsc{Test 10}
\begin{console}[commandchars=\\\{\}]
\userinput{100}
2^2 * 5^2
\end{console}

\textsc{Test 11}
\begin{console}[commandchars=\\\{\}]
\userinput{1265265}
3^2 * 5 * 31 * 907
\end{console}

WARNING: ... INCOMING SPOILERS ... Hints on next page. Use only if necessary.

\newpage
\textsc{Hints}


The following is meant to help you think through the problem-solving problem,
specifically to show you one possible path to breakdown the problem.
The pseudocode (or ideas) is pseudocode and is only meant to point you in certain directions.
It's not meant to be complete -- you have to add to it and modify it rather than to
use it blindly.
Also, after you have solved some smaller problems, sometimes when you glue two smaller
solutions together, you will realize that one of the smaller solutions might have to
be changed slightly.

OK, here we go ...

As mentioned before, you always try to find smaller problems
within your problem and solve those smaller problems.

Looking at this:
\begin{console}[commandchars=\\\{\}]
\userinput{1265265}
3^2 * 5 * 31 * 907
\end{console}

The above prints a bunch of prime-powers dividing the user input $n$.
In the above the first prime-power is $3^2$.
If you think about it, you should at least print the first prime-power.
(In other words, instead of printing 4 prime-powers, print the first
prime power. Why? Because if you can't print the first prime power,
you can't possibly print 4 of them.)

To print the first prime-power dividing $n$, you have to at least
print the smallest prime dividing $n$.
In other words, in the above example,
instead of computing $3^2$, you should try to compute $3$.

Based on the above breakdown of the problem, this is what I would do first:
Get a positive integer from the user and print the first
prime dividing the integer.

I will create about 5 test cases for this problem.
(Yes, it's perfectly OK to create test cases \textit{before} you
even design the pseudocode. This is called test-driven development.
This is what actually happens in real-life in situations like dating, buying a house,
etc.)

The first task looks like this:
{\small
\begin{console}
get n from user
print first prime dividing n
\end{console}
}
Getting $n$ from the user is easy.
For the second statement, of course you
can't tell C++ to give you the first prime.
So you have to translate that to something C++ understands.
The smallest prime is $2$ ($1$ is not a prime).
So you would have to do something like this:
{\small
\begin{console}
get n from user
let p = 2
if p divides n, prime p
\end{console}
}
But that only checks for the case of \verb!p! equals $2$.
You need to check the case when \verb!p! equals $3$, etc.
That's repetition: so there must be a loop.
The idea should look like this:
{\small
\begin{console}
get n from user
let p = 2
while p does not divide n:
    p = p + 1
print p
\end{console}
}
Wait ... \verb!p = p + 1! make \verb!p! goes to the next integer,
not the \textit{next prime} after \verb!p!.
So it should look like this:
{\small
\begin{console}
get n from user
let p = 2
while p does not divide n:
    p = next prime afer p
print p
\end{console}
}
The problem now is reduced to this idea of \textit{next prime after}.
To make the above pseudocode not overly messy, say we have a function:
{\small
\begin{Verbatim}[frame=single]
int get_next_prime_after(int p):
    ... TO BE COMPLETE ...

int main():
    get n from user
    let p = 2
    while p does not divide n:
        p = get_next_prime_after(p)
    print p
\end{Verbatim}
}
(Remember that the above is pseudocode, not property C++.)

Clearly the get next prime function requires you to know how to check if a number is
prime.
So the above becomes:
{\small
\begin{Verbatim}[frame=single]
int get_next_prime_after(int p):
    let i = p + 1
    while i is not prime:
        i = i + 1 (or ++i)

int main():
    get n from user
    let p = 2
    while p does not divide n:
        p = get_next_prime_after(p)
    print p
\end{Verbatim}
}
which means now you must have a function to check if a specific number is a
prime:
{\small
\begin{Verbatim}[frame=single]
bool is_prime(int i):
    ... TO BE COMPLETED ...
    
int get_next_prime_after(int p):
    let i = p + 1
    while !is_prime(i):
        i = i + 1 (or ++i)

int main():
    get n from user
    let p = 2
    while p does not divide n:
        p = get_next_prime_after(p)
    print p
\end{Verbatim}
}
The function call \verb!is_prime(i)! returns \verb!true!
if the value of \verb!i! is prime and it returns \verb!false!
if the value of \verb!i! is not a prime.
I will leave the logic of the function
\verb!bool is_prime(int)! to you since it's a pretty standard CISS240 type
problem and in fact is in my C++ notes.

Note that on completing the above, you have a problem that computes (and prints)
the first prime dividing \verb!n!.
After that you need to compute the power of that prime dividing \verb!n!.
How would you do that?
Once you've found the smallest prime, say $p$, divding $n$, you just continually
divide $n$ by $p$ until $n$ is \textit{not} divisible by $p$.
For instance if $n = 1000$ and you already know that $p = 2$ divides $n$,
then you would do this sequence of division:
\begin{align*}
  n &= 1000, \text{let } n = n / p, \text{then } n = 500 \\
  n &= 500, \text{let } n = n / p, \text{then }  n = 250 \\
  n &= 250, \text{let } n = n / p, \text{then }  n = 125 \\
\end{align*}
at which point $n$ is not divisible by $p = 2$ anymore.
Right?
That's no mystery here.
This is exactly what you did in elementary school.

Of course in the above you still need to keep a count of the power of 2 dividing
1000.
All you need is a counter variable.
So now you add the part that computes the prime power when a dividing prime is found:
{\small
\begin{Verbatim}[frame=single]
bool is_prime(int i):
    ... TO BE COMPLETE ...
    
int get_next_prime_after(int p):
    let i = p + 1
    while !is_prime(i):
        i = i + 1 (or ++i)

int main():
    get n from user
    let p = 2
    while p does not divide n:
        p = get_next_prime_after(p)
    print p

    count = 0
    while n is divisible by p:
        count = count + 1
        n = n / p
    print count
\end{Verbatim}
}

This prints the \textit{first} prime power dividing your $n$.
What about the second?
You just repeat the above process in a loop.
After all if the input is $n = 1000$, after you've printed out the $2^3$,
your $n$ becomes $125$ and you can execute the same logic on $n = 125$
to get $5^3$. Right?

There are other details, but the rest are minor.
For instance you have to print the \verb!*!
and also, note that if the power is 1, you do not print it.
(For the printing of \verb!*!, note that if you have 4 prime powers, you only print 3 \verb!*!'s.)

When you step back, you see that the highest level idea is something like this:
{\small
\begin{Verbatim}[frame=single]
get n from user

while n is not 1:
    set p to the the smallest prime dividing n 
    find k such that k is the largest p^k divides n
    set n to n / p^k
    print p and k
\end{Verbatim}
}
except that if you have already divided $n$ by the maximal power of 2, there's no
need to check for 2 again.
So we keep a variable \verb!p! to remember where we stopped:
{\small
\begin{Verbatim}[frame=single]
get n from user

p = 1
while n is not 1:
    set p to the next prime (after p) dividing n
    find k such that k is the largest p^k divides n
    set n to n / p^k
    print p and k
\end{Verbatim}
}
