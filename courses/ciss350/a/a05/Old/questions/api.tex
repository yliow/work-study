%-*-latex-*-
\textsc{Common interface}

All the sorting functions in this section have a common interface.
If \verb!x! is an array or
a \verb!std::vector< int >! object, then
\begin{console}[fontsize=\footnotesize]
bubblesort(&x[2], &x[6]);
\end{console}
will perform bubblesort on the values of \verb!x[2..5]!.
The sorting is done in ascending order.
You can also execute
\begin{console}[fontsize=\footnotesize]
bubblesort(&x[2], &x[6], less);
\end{console}
where \verb!less! is a comparator functor that specifies the order of the
sorting.
Specifically, \verb!less(a, b)! is true is either \verb!a! is the same as
\verb!b! or \verb!a! should appear before \verb!b! in the array
that is sorted.
For both prototypes above, there's an extra boolean parameter that specifies
if the function should display the progress of the sorting process.
This boolean parameter has a default value of \verb!false!.
So calling
\begin{console}[fontsize=\footnotesize]
bubblesort(&x[2], &x[6], true);
\end{console}
will result in displaying the bubblesorting process.
For instance you might see this:
\begin{console}[fontsize=\footnotesize]
{1, 9, 4, 2}
{1, 4, 2, 9}
{1, 2, 4, 9}
{1, 2, 4, 9}
\end{console}

Here's an example on sorting two integers where the sorting uses
a comparator functor for determining the order of sorting.
Study this carefully.
Modification to the bubblesort using a functor for comparison will be easy
using this example as a guide.
\VerbatimInput[fontsize=\footnotesize,frame=single]{sort2.cpp}

I already mentioned in CISS245, when writing a template
(class or function or struct),
it's usually a good idea to start off with a non-template version.



\newpage
\textsc{C\texttt{++} lambda expressions}

The above section is enough.
This section is optional, but you should probably study it.

In the previous section, the ordering of the sorting process
is determined by a comparator functor.
The functor is an object.
So you need to write a class for any comparator functor.
There's another way to achieve the same thing, but with
something called a lambda expression.
(Details about functional programming, lambda functions, etc. is covered
in CISS445.)

\VerbatimInput[fontsize=\footnotesize,frame=single]{sort2-lambda.cpp}

The
\[
\texttt{[](int x, int y)\{ return x < y; \}}
\]
and the
\[
\texttt{[](int x, int y)\{ return x > y; \}}
\]
are \cpp\ lambda expressions.
You can think of them as anonymous functions, i.e.,
functions without names.

Compare the above with the code from the previous section
and you should be able to figure out what the lambda expressions
achieve.
