%-*-latex-*-
[Sparse polynomial]

Using \verb!std::list!, implement an integer polynomial class where
zero coefficient terms are not stored.
(The case of a class template is easy once the integer version is done.)
\begin{console}[fontsize=\footnotesize]
// A monomial is a term in a polynomial. For instance 5 * x ^ 3 is a
// monomial. For this term, we store (3, 5), i.e., the degree and the
// coefficient.
class Monomial
{
public:
    int degree;
    int coefficient;
};

class Polynomial
{
public:
private:
    std::list< Monomial > p; // The term with smallest degree is at the head.
};
\end{console}
The following describes the expected usage of the class:
\begin{Verbatim}[frame=single,fontsize=\footnotesize]
Polynomial p("3*x^2 + 7*x^12 - 9*x^100 + -2*x^32 + x^5 + 0*x^6 + 6*x^2 + 32 + 1*x^0");
std::cout << p << '\n'; // prints -9x^100 - 2x^32 + 7x^12 + x^5 + 9x^2 + 33

Polynomial p0(42);      // p0 = 42 as a polynomial
Polynomial p1(p);       // p1 = p as a polynomial

std::cout << (p0 == p) << '\n';
std::cout << (p1 != p) << '\n';
std::cout << (5 == p) << '\n';
std::cout << (p != 5) << '\n';

std::cout << deg(p) << '\n'; // degree of p

Polynomial q("x^2 + x + 1");

p += q;
p -= q;
p *= q;

p += 42;
p -= 42;
p *= 42;

Polynomial r; // r = 0 as a polynomial

r = p - q;
r = p * q;

r = 42 + p;
r = p + 42;

r = 42 - p;
r = p - 42;

r = 42 * p;
r = p * 42;

r = 42 * p;
\end{Verbatim}
Make sure you examine the constructors and output carefully.
In particular, note that in the constructors, the input string need not
list monomials in any particular order by degree.
Also, in the printout of the polynomials, higher degree terms are
printed first.

\textsc{Optional DIY.}
\begin{enumerate}[nosep]
\li It's a good exercise to use your \verb!Rational! class to model
polynomials with fractions as coefficients.
Then you can also perform long division to get quotients
and remainders.
In fact your polynomial class should really be a class template.
\li Recall that in CISS240, there was an assignment question
that attempts to factorize polynomials.
With this class you can now attempt to factorize
an integer polynomial of any degree into two polynomials
of smaller degree.
For self-study and research,
you can google for algorithms for polynomial factorization.
\end{enumerate}
