%-*-latex-*-
[Infix to reverse polish notation]

Write a program that converts an integer expression written using the infix 
notation into RPN.

\textsc{Test 1}
\begin{console}[commandchars=\\\{\},fontsize=\footnotesize]
\underline{0 - 1 + 2 * 3 - 4 / 5 * 6}
0 1 - 2 3 * + 4 5 / 6 * - 
\end{console}

Try out more test cases of your own.

[Suppose the output of the above computation is an RPN expression.
For instance from \verb!1 + 2!, you want to compute \verb!1 2 +!.
Of course as a vector of tokens, the input is
\texttt{[IntTok(1), PlusTok, IntTok(2)]}
and the output is
\texttt{[IntTok(1), IntTok(2), PlusTok]}.
The algorithm to evaluate an infix expression
\texttt{[IntTok(1), PlusTok, IntTok(2)]}
basically computes \verb!IntTok(3)!.
But for this problem you won't want \verb!IntTok(3)!.
You want \texttt{[IntTok(1), IntTok(2), PlusTok]}.
All you need to do is to modify the algorithm to evaluation
an infix expression so that instead of
\verb!IntTok(3)!,
you compute \texttt{[IntTok(1), IntTok(2), PlusTok]}.
]
