%-*-latex-*-
\input{myassignmentpreamble}
\input{yliow}
\input{ciss350}
\renewcommand\TITLE{Assignment 10}

\usepackage{listings}
\lstset{%
basicstyle=\ttfamily, frame=single
}


\begin{document}
\topmatter


\textsc{Objectives:}
\begin{myenum}
\item Implement Binary Search Tree with supporting methods.
\end{myenum}

\newpage
Q1.

Implement a binary search tree of integers. 
(Once you have the tree working, it's easy to convert this to a 
template class -- I'll leave as a DIY exercise.)
Values contain in the tree are unique, i.e., there are no duplicate values.
As mentioned in class,
smaller values are stored in the left subtree and larger values are
stored in the right subtree.
(This implies that if you want to convert this to a template class,
the template type must support the obvious comparison operators.)

The following are some skeleton source files.
(Note: As always, skeleton
code means the code is incomplete and might contain errors.
The purpose is to give some initial guidance and structure of code.)
 
Note that the nodes have parent pointers and must be set correctly.
The only node with \verb!NULL! parent pointer is of course the root.
In the case of an empty tree, the \verb!root! is set to \verb!NULL!.
Note that the skeleton code is (obviously) incomplete and require
either addition or modification.


\VerbatimInput[fontsize=\scriptsize, frame=single]{bst/BinarySearchTreeNode.h}


\VerbatimInput[fontsize=\scriptsize, frame=single]{bst/BinarySearchTreeNode.cpp}

\VerbatimInput[fontsize=\scriptsize, frame=single]{bst/BinarySearchTree.h}

\VerbatimInput[fontsize=\scriptsize, frame=single]{bst/BinarySearchTree.cpp}

The following tests only some of the functions and methods. 
You should test all the functions/methods to be implemented.

\VerbatimInput[fontsize=\scriptsize, frame=single]{bst/test.cpp}

Note that the \verb!insert()! method of the 
\verb!BinarySearchTree! class uses the \verb!insert()! function in 
\verb!BinarySearchTreeNode.cpp!.
This should be the case for many of the methods in the \verb!BinarySearchTree!
class.
In software development, we would say that the \verb!BinarySearchTree!
class is a wrapper class, i.e., the class wraps up functionality already
provided somewhere else.

Note that in the \verb!BinarySearchTree! class, the \verb!insert()!
method calls the \verb!insert()! function in the 
\verb!BinarySearchTreeNode! (the cpp file). 
In order to achieve this, the function call to the \verb!insert()!
function must be referenced as
\begin{Verbatim}
::insert(root_, x)
\end{Verbatim}
The \verb!::! tells the compiler to look for \verb!insert! in the 
global scope and not within the class scope of \verb!BinarySearchTree!.

In the case of a failure to insert, the \verb!insert()! function
in \verb!BinarySearchTree!
returns a \verb!false!; otherwise it returns \verb!true!.
On the other hand, the \verb!insert()! method in \verb!BinarySearchTree!
throws an exception when an insert is not possible.

Your implementation for \verb!remove()! should be similar, i.e.,
there should be a \verb!remove()! function in \verb!BinarySearchTreeNode!
and a \verb!remove()! method in \verb!BinarySearchTree!.
On failure, 
the \verb!remove()! function \verb!BinarySearchTreeNode! 
should return \verb!false! whereas the \verb!remove()! method in 
\verb!BinarySearchTreeNode! should throw an exception.


\newpage
Q2.

With Q1 done, you can now make your BST objects self-balancing by including
balancing after an insert or a remove operation.
Make sure check the notes for left and right rotations and when/how
they are used after a BST insert and a BST remove.

The name of the class is \verb!AVL! and it should have all the
methods in the BST class from Q1.
Like the BST case, you only need to write AVL trees of integer values.
Note that almost all methods are the same as the methods in BST (Q1).

Make sure you test your AVL insert and remove thoroughly.

(After you're done with Q2, as a personal project, you should convert this into
a class template.)


\textsc{Note.}
Together with quadtrees, you now have two topics to investigate
further and can be used for an undergraduate research project/paper.
There are lots of information on quadtree (and their variations and other
spatial data structures)
and AVL trees (and their variations and other self-balancing trees).

\end{document}
