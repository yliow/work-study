%-*-latex-*-
\textsc{[Expression tree]}

You must use \verb!BinaryTreeNode! from the previous question.

Write a program that accepts an arithmetic expression in this form:
If the user wants to compute
\[
1 + (2 - 3)
\]
he enters
\begin{console}
1+(2-3)
\end{console}
(i.e., no spaces).
The arithmetic expression operates only on single-digit numbers 
from \verb!0! to \verb!9!
(it's pretty easy to include other multi-digit numbers but you don't
have to worry about these cases)
and the operators are \verb!+,-,*,/,%!.

The program then builds a tree
\begin{python}
from latextool_basic import *
print(r"""
\begin{center}
%s
\end{center}
""" % graph(yscale=0.9,layout='''
   A
  B C
   F G
''',
minimum_size='8mm',
edges='A-B,A-C,C-F,C-G',
A=r'label=\texttt{+}',
B=r'label=\texttt{1}',
C=r'label=\texttt{-}',
F=r'label=\texttt{2}',
G=r'label=\texttt{3}'))
\end{python}

Your program must generate the tree according to standard precedence
rules.
For instance multiplication goes before addition.
Therefore for the string \verb!1+2*3!, the tree is 
\begin{python}
from latextool_basic import *
print(r"""
\begin{center}
%s
\end{center}
""" % graph(yscale=0.9,layout='''
   A
  B C
   F G
''',
minimum_size='8mm',
edges='A-B,A-C,C-F,C-G',
A=r'label=\texttt{+}',
B=r'label=\texttt{1}',
C=r'label=\texttt{*}',
F=r'label=\texttt{2}',
G=r'label=\texttt{3}'))
\end{python}
(Note that some nodes hold operators and some hold integer values.)

Your program then generates a dot file for graphviz,
prints the arithmetic expression in RPN (in the console window), and
print the result of the expression (in the console window).
You then generate a \verb!ps! file to display the tree.
For the above expression \verb!1+2*3!, in RPN, it becomes
\verb!123*+!.

\textsc{Note.}
\begin{tightlist}
\item It's OK to use the textbook or the web to find a suitable algorithm to
create the tree from the arithmetic expression string.
You can also refer to my notes or the textbook.
But you must write the code yourself, of course.
\item Of course you also have unary operators (example: \lq\lq negative of'' and \lq\lq
positive of'')
and ternary operators (example: \texttt{(:?)}).
\end{tightlist}
