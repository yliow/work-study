%-*-latex-*-
\textsc{Graphviz}

Graphviz is a graph visualization software.
You can find out more about this research
project first initiated by AT\&T Labs.
It allows you to specify a graph including
the nodes 
and the edges between nodes.
The software produces image files drawing
the graph, attempting to place the nodes
in such a way that there is minimal edge crossings.
You can also specify shapes of the nodes (circle, square, etc.)
as well as color.
Besides putting into the nodes, you can also label edges.
This is probably among the most famous graph visualization software
and is used by many computer scientists

To install the program on your fedora virtual machine do this
(as root)
\begin{console}[fontsize=\footnotesize]
dnf -y install graphviz
\end{console}
This assumes that you're using a fedora machine.

Here's an example on how to create a directed graph.
Create a text file named \verb!graph.dot! (or any name you choose)
with the following contents:
\begin{console}[fontsize=\footnotesize]
digraph G
{
   a -> b;
   a -> c;
   a -> d;
   b -> e;
}
\end{console}
You are creating nodes $a, b, c, d, e$ with $a$ joined to $b$, etc.

Run the following command from your shell:
\begin{console}[fontsize=\footnotesize]
dot -Tps graph.dot -o graph.ps
\end{console}
and you will get an image file \verb!graph.ps! with the graph.
Of course you can choose any other filename for your image file. 
This generates the image file as a postscript file.
You can convert that to a JPG file doing this in linux:
\begin{console}
convert graph.ps graph.jpg
\end{console}
To draw an (undirected)  graph change your file to this:
\begin{console}[fontsize=\footnotesize]
graph G
{
   a -- b;
   a -- c;
   a -- d;
   b -- e;
}
\end{console}

You can change the label of a node and the shape like this:
\begin{console}[fontsize=\footnotesize]
digraph G
{
   a -> b;
   a -> c;
   a -> d;
   b -> e;

   a [shape=box];
   b [label="hello\nworld"];
}
\end{console}

You can find more about graphviz on the web.
