%-*-latex-*-
\textsc{[Prefix tree/trie]}

Suppose I want to write a spellchecker.
First I would have to collect list of words.
When the spellchecker runs, I will probably have to load the words
into the program and then for each word read from a document,
I will check against the collection of words loaded into the
program's memory.

Let's say that I want my program to recognize the following words
\verb!a!, \verb!cab!, \verb!cat!.
You can of course store the words into an array of strings.
Say you have a collection of 500,000 words and the 
average length of the words is 10 characters.
That means that I need about 5,000,000 bytes.
I'll probably sort the array so that I can search for 
words quickly.
The worst case is $\log_2(500000)$ word searches and then 
about 10 character comparisons for each word.

Let's think about it in a different way.
Notice that in the English language,
many words share common prefixes, i.e., left substrings.
For instance notice that 
\verb!cab! and \verb!cat!
have the same first two characters.

So first look at this picture:

\begin{python}
from latextool_basic import graph
print(r'''
\begin{center}
%s
\end{center}
''' % graph(yscale=1.5, layout='''
 C
D E
  F
H G I
  J
''',
minimum_size='8mm',
edges='C>D,C>E,E>F,F>G,F>H,F>I,G>J',
A=r"label=$$",
B=r'label=$$',
C=r'label=$$',
D=r'label=$\texttt{*}$',
E=r'label=$\texttt{}$',
F=r'label=$\texttt{}$',
G=r'label=$\texttt{*}$',
H=r'label=$\texttt{*}$',
I=r'label=$\texttt{*}$',
J=r'label=$\texttt{*}$',
edge_label={('A','B'):{'label':r'\texttt{c}'},
('C','D'):{'label':r'\texttt{a}', 'style':'pos=0.8,above,inner sep=3mm,circle'},
('C','E'):{'label':r'\texttt{c}'},
('E','F'):{'label':r'\texttt{a}'},
('F','G'):{'label':r'\texttt{n}'},
('F','H'):{'label':r'\texttt{b}', 'style':'pos=0.5,above,inner sep=1mm,circle'},
('F','I'):{'label':r'\texttt{t}', 'style':'pos=0.5,above,inner sep=1mm,circle'},
('G','J'):{'label':r'\texttt{e}'},
},
))
\end{python}

It should be clear what we're doing here.
Basically paths from the root in the tree form words.
However not all paths form valid words.
So to mark valid words with \texttt{*} in the sense that
when going from the root to a node marked \texttt{*},
the edges (or the letters corresponding to the edge) forms a valid word.
So in the above, you see word
\texttt{a},
\texttt{cab},
\texttt{cane}, and
\texttt{cat}.
How do we assume letters with the edges?
We can use the index positions of the pointers.
So for instance from the root, the 0--th pointer that points to child-0
represents character \verb!a!.
Likewise reading a character \verb!c! is the same as 
following the pointer to arrive at child-2.

Using the above idea, the above tree is constructed like this
\begin{console}[fontsize=\footnotesize]
int a = int('a' - 'a'); // i.e., 0
int b = int('b' - 'a'); // i.e., 1
int c = int('c' - 'a'); 
int t = int('t' - 'a');

TreeNodev< char > * p(' ');

p->insert(a, '*');
p->insert(c, ' ');

p->child(c)->insert(a, ' ');

p->child(c)->child(a)->insert(b, '*');
p->child(c)->child(a)->insert(t, '*');
// etc.
\end{console}

Basically the point is that at each node,
there can be 26 pointers, one pointer for each character.
(We only worry about lowercase.)
So to check if the tree contains the word \verb!dog!,
we go node to node,
following the appropriate pointer based on the character of the word.
Since \verb!d - a = 3!,
\verb!o - a = 14!,
\verb!g - a = 6!,
we check if 
\verb!p->child(3)->child(14)->child(6)->data()!
is \verb!'*'!.

You are given a word file \verb!word.txt!.
Write a program to do the following:

(a) Build the word tree using \verb!word.txt! according to the above scheme.

(b) Print the total size of the tree (i.e., number of nodes) and a newline.

(c) The program then reads the command-line argument for a string input and
  search the word tree for that string.
If the word is found, the program
print \verb!*! and then the word (and a newline).
For instance:
\begin{console}[commandchars=\\\{\},fontsize=\footnotesize]
g++ main.cpp
./a.out cab
1000
*cab
\end{console}
(Assuming that your tree contains 1000 nodes -- the actual number
is larger.)
If the word is not found, then there are two cases.
Suppose your tree can complete the word.
Then the output follows this format:
\begin{console}[commandchars=\\\{\},fontsize=\footnotesize]
g++ main.cpp
./a.out ca
1000
+ca:b,ble,t
\end{console}
if your program
finds the words \verb!cab!, \verb!cable!, \verb!cat!.
Note the order follows dictionary order.
If the string from command-line argument is \verb!zz!
which cannot be completed to a word, the program prints this:
\begin{console}[commandchars=\\\{\},fontsize=\footnotesize]
g++ main.cpp
./a.out zz
1000
?zz
\end{console}
