%-*-latex-*-
Implement a list-based double-ended queue, or deque.
As mentioned in class, a doubly linked list should be used.
Treat the front of a deque as the head of the underlying list
and the back of the deque as the tail of the underlying list.
The following is what you can do:
\begin{console}[fontsize=\footnotesize]
Deque< int > deque;
deque.push_front(5);      
deque.push_back(6);
int x = deque.front();     // x = 5
deque.front() = 4;         // front of deque is now 4
int y = deque.back();      // y = 6
deque.back() = 7;          // back of deque is 7
deque.pop_front();         // deque has now 1 value, i.e., 7
deque.pop_back();          // deque is now empty
int size = deque.size();   // size is zero
bool b = deque.is_empty(); // b is true
deque.push_front(1);
deque.push_front(1);
deque.push_front(1);
deque.clear();             // deque is now empty
\end{console} 
Here's the skeleton:
\begin{console}[fontsize=\footnotesize]
// Deque.h
class UnderflowError       // An UnderflowError object is thrown
{};                        // if you execute any operation to
                           // remove a value from a deque that is
                           // empty.
                           
template < typename T >
class Deque
{
public:

private:
    DLList< T > list_;
};
\end{console}
You can print a Deque object.
The output is similar to previous questions -- just 
make sure you print the front first.

Once the doubly linked list is done, the Deque class is extremely easy.
So you must finish the doubly linked list class first.
