%-*-latex-*-
\textsc{Objectives}
\begin{tightlist}
\item Implement iterators for a vector class.
\item Implement a singly-linked list with several supporting methods.
\item Implement a doubly-linked list with several supporting methods
\item Implement a list-based stack, list-based queue, and double-ended queue.
\end{tightlist}

The first objective of this assignment
is to implement an incomplete vector class with iterator classes.
You need not complete the vector class (although it's a very good
exercise and valuable experience and also a good personal open source
project).

The second objective is to implement linked list classes, both
the singly- and doubly-linked list.
For the singly-linked list, each object will have a pointer 
that points to a
head node or \verb!NULL!.
For the doubly-linked list, each object will have a 
head sentinel node and a tail sentinel node.

Once you have the singly- and doubly-linked list classes,
implementing a list-based stack,
list-based queue,
list-based double-ended queue is trivial.

Some skeleton code is provided.
Note that the skeleton requires modification and addition to the code;
it might contain errors.
The purpose is just to give you a starting point and  some ideas.

Reminders:
If an object contains member variables that consumes some resources
(example: memory from the heap)
that is allocated by your code during object construction, then
you must provide a copy constructor, the destructor, and
assignment operator (i.e., \verb!operatorname=!).
If you don't, something will go wrong.

\textsc{Aside.}
Note that, depending on what you want your 
list (singly- or doubly-linked) to achieve, usage of the list class
might not need to know anything about the node class:
\begin{console}[fontsize=\footnotesize]
SLList list;
list.insert_head(5);
list.insert_tail(3);
// Etc.
int x = list.remove_tail();
\end{console}
If that's the case, then you can put the node class inside the
list class so that it is a nested class within the list class:
\begin{console}[fontsize=\footnotesize]
class SLList
{
    ...

    class SLNode 
    {
    };
};
\end{console}
and to protect outsiders from using the nested class you can do this:
\begin{console}[fontsize=\footnotesize]
class SLList
{
    ...
private:
    class SLNode 
    {
    };
};
\end{console}
