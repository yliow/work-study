%-*-latex-*-
Complete the following class that implements a class similar to
the \cpp\ STL vector class.
The purpose is to understand the inner workings for the iterator
and constant iterator classes so that
\begin{enumerate}[nosep]
\item You can implement iterator classes in your own container classes.
\item You can work quickly with \cpp\ STL classes (since most uses of \cpp\ STL classes involve the use of builtin iterator classes).
\end{enumerate}
Therefore you need not complete all the methods for the vector
class. 
Remember that skeleton code is incomplete and might have errors.
The purpose of skeleton code is only to give you a rough
overview of the intention of the code.

Refer to the relevant iterator sections in chapter on containers.
You might also want to look for the \verb!IntPointer! class
in the CISS245 notes.

\VerbatimInput[fontsize=\footnotesize,frame=single]{vector.h}

When completed, you should be able to do the following:
\begin{Verbatim}[fontsize=\footnotesize, frame=single]
vector< int > v(10, 0); // v is 10 0s in array of capacity 20
typename vector< int >::iterator p = v.begin();
typename vector< int >::const_iterator q = v.begin();

std::cout << v[0] << '\n'; // prints 0
std::cout << (*p) << '\n'; // prints 0
std::cout << (*q) << '\n'; // prints 0

v[0] = 42;
std::cout << v[0] << '\n'; // prints 42
std::cout << (*p) << '\n'; // prints 42
std::cout << (*q) << '\n'; // prints 42

*p = -1;
std::cout << v[0] << '\n'; // prints -1
std::cout << (*p) << '\n'; // prints -1
std::cout << (*q) << '\n'; // prints -1

// *q = 0; // ERROR during compilation

v[1] = 1;
++p;
++q;
std::cout << v[1] << '\n'; // prints 1
std::cout << (*p) << '\n'; // prints 1
std::cout << (*q) << '\n'; // prints 1

v[10] = -9999; // Note: this is outside the valid index range of 0..9
p = v.end();
q = v.end();
std::cout << v[10] << '\n'; // prints -9999
std::cout << (*p) << '\n';  // prints -9999
std::cout << (*q) << '\n';  // prints -9999

v[9] = -9998;
--p;
--q;
std::cout << v[9] << '\n'; // prints -9998
std::cout << (*p) << '\n'; // prints -9998
std::cout << (*q) << '\n'; // prints -9998

*p = -9997;
std::cout << v[9] << '\n'; // prints -9997
std::cout << (*p) << '\n'; // prints -9997
std::cout << (*q) << '\n'; // prints -9997
\end{Verbatim}
Other features you can implement and test for your own understanding:
\begin{Verbatim}[fontsize=\footnotesize, frame=single]
p = v.begin();
std::cout << (*(p++)) << '\n';
std::cout << (*(++p)) << '\n';
p += 3; // NOTE: operator+= is not available for all \cpp\ STL classes
*p = 42;
std::cout << v[3] << '\n';
std::cout << *p << '\n';
p -= 2;
*p = 43;
std::cout << v[1] << '\n';
std::cout << (*p) << '\n';
\end{Verbatim}

You should be able to include
the following to your \verb!main.cpp!:
\begin{console}[fontsize=\footnotesize]
template < typename T >
std::ostream & operator<<(std::ostream & cout, const vector< T > & v)
{
    std::string delim = "";
    cout << '{';
    for (typename vector< T >::const_iterator p = v.begin();
         p != v.end();
         ++p)
    {
        cout << delim << (*p);
        delim = ", ";
    }
    cout << '}';
    return cout;
}
\end{console}
Note that in the above I'm using the \verb!const_iterator! because the
\verb!v! is a constant reference.
You cannot use \verb!iterator!.
If your object \verb!v! is from class \texttt{C< T >}
(which is not necessary from the \verb!vector! class)
has a \texttt{const\_iterator}, and methods
\verb!begin()! and \verb!end()!, you can do this:
\begin{Verbatim}[frame=single,fontsize=\footnotesize,commandchars=\~\#\@]
template < typename T >
std::ostream & operator<<(std::ostream & cout, const C< T > & v)
{
    std::string delim = "";
    cout << '{';
    for (typename C< T >::const_iterator p = v.begin();
         p != v.end();
         ++p)
    {
        cout << delim << (*p);
        delim = ", ";
    }
    cout << '}';
    return cout;
}
\end{Verbatim}


Your submission should include code for your vector class (\verb!vector.h!)
and a \verb!main.cpp! that includes the above tests.
(Remember: Template code should be complete in a header file.
There should be no \verb!vector.cpp!.)

For further practice on use of your own iterators, you can
implement other vector methods/functions (example:
the \verb!std::vector! class has an \verb!erase! method).

\textsc{Note}:
If you want to write a complete vector class, you can google for
g++ implementation details. There are some optimizations
that I did not mention in the CISS245 assignment the
integer dynamic array class assignment.
Also, once you have done several iterator classes for different containers,
you'll see that it's beneficial to have some iterator base classes.
