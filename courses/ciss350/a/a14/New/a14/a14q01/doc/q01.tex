[Max heap]

Implement the max heap functions (see below)
where the heap is implemented
using vectors of integers.
(Vectors as in \texttt{std::vector}.)
Write the max heap functions in
\texttt{intmaxheap.h}
and
\texttt{intmaxheap.cpp}.

Refer to the notes (and textbook) for details
on max heap operations.

The following describes all the max heap functions that
must be implemented.
Also, implement \verb!operator<<! so that you can
print your vector while testing your code (see below).
\begin{Verbatim}[frame=single, commandchars=\\\{\}]
int x;
std::vector< int > heap;

maxheap_insert(heap, 5);        // [5]
maxheap_insert(heap, 7);        // [7, 5]
maxheap_insert(heap, 9);        // [9, 5, 7]
std::cout << heap << std::endl; //  prints "[9, 5, 7]"

int a = maxheap_delete(heap);   // [7, 5]
                                // a = 9
x = maxheap_max(heap);          // x = \redtext{7}. Root is not deleted.

heap[0] = 1;                    // [1, 5]
maxheap_heapify_down(heap, 0);  // [5, 1]

heap[1] = 10;                   // [5, 10]
maxheap_heapify_up(heap, 1);    // [10, 5]

heap.resize(5);
heap[0] = 5;
heap[1] = 7;
heap[2] = 8;
heap[3] = 10;
heap[4] = 2;
maxheap_build(heap);            // [10, 7, 8, 5, 2]

heap.resize(5);
heap[0] = 2;
heap[1] = 6;
heap[2] = 8;
heap[3] = 10;
heap[4] = 5;
maxheap_heapsort(heap)          // [2, 5, 6, 8, 10]
std::cout << heap << std::endl; // prints "[2, 5, 6, 8, 10]"
\end{Verbatim}

Put the above in \texttt{main.cpp}:
\begin{console}
#include <iostream>
#include "intmaxheap.h"

int main()
{
    // test code
    return 0;
}
\end{console}

(As always skeleton code needs to be modified/edited/corrected/etc.)
