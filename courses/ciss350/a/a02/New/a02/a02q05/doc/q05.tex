[Review of classes of CISS245]

This is a review of writing a class (without pointers).

Write a \texttt{Fraction} class that should do the obvious.
No skeleton code is given.
The header file is named \texttt{Fraction.h}
and the implementation file is named \texttt{Fraction.cpp}.
You must design and implement the class yourself.
You should of course test your code, i.e., you should have a
\verb!main.cpp! that tests the features of your fraction library.
Try not to refer to your previous work if you have already implemented
such a class -- otherwise it defeats the purpose of review.
You can however refer to my notes on object-oriented
concepts/syntax of \cpp.
Only refer to your previous code briefly for the \verb!Fraction! if you are
totally lost.
You should be able to complete this library in less than 1 hour.

The following illustrates features of this class.

The basic constructor should work as expected:
\begin{Verbatim}[frame=single,fontsize=\footnotesize, commandchars=\~\!\@]
Fraction f(2, 3);
Fraction g(42);                           
Fraction h;                               // default constructor

std::cout << f << '\n';                   // prints "2/3" (without the double quotes)
std::cout << g << '\n';                   // prints "42"
std::cout << h << '\n';                   // prints "0" 

std::cout << Fraction(2, 4) << '\n';      // prints "1/2"
std::cout << Fraction(-2, 4) << '\n';     // prints "-1/2"
std::cout << Fraction(2, -4) << '\n';     // prints "-1/2"
std::cout << Fraction(-2, -4) << '\n';    // prints "1/2"

std::cout << Fraction(2, 2) << '\n';      // prints "1"
std::cout << Fraction(2, -2) << '\n';     // prints "-1"
std::cout << Fraction(-2, 2) << '\n';     // prints "-1"
std::cout << Fraction(-2, -2) << '\n';    // prints "1"

std::cout << Fraction(100, 20) << '\n';   // prints "5"
std::cout << Fraction(100, -20) << '\n';  // prints "-5"
std::cout << Fraction(-100, 20) << '\n';  // prints "-5"
std::cout << Fraction(-100, -20) << '\n'; // prints "5"

std::cout << Fraction(100) << '\n';       // prints "100"
std::cout << Fraction(-100, 1) << '\n';   // prints "-100"
std::cout << Fraction(100, -1) << '\n';   // prints "-100"
std::cout << Fraction(-100, -1) << '\n';  // prints "100"

std::cout << Fraction() << '\n';          // prints "0"
std::cout << Fraction(0, 100) << '\n';    // prints "0"
std::cout << Fraction(0, -100) << '\n';   // prints "0"


// Testing division by zero exception catching
try
{
    std::cout << Fraction(0, 0) << '\n';
}
catch (ZeroDivisionError & e)
{
    std::cout << "ZeroDivisionError exception caught\n";
}

try
{
    std::cout << Fraction(-10, 0) << '\n';
}
catch (ZeroDivisionError & e)
{
    std::cout << "ZeroDivisionError exception caught\n";
}
\end{Verbatim}
Note that the numerator and denominator are reduced before they are
saved in the \texttt{Fraction} object.

Copy constructor and assignment operator should work as expected:
\begin{Verbatim}[frame=single,fontsize=\footnotesize]
Fraction f0(1, 2);
Fraction f1(f0); // copy constructor
f1 = f0;         // assignment operator
\end{Verbatim}
(Do you need to overwrite the default copy constructor
and assignment operator for this class?)

Various unary and binary operators should work as expected:
\begin{Verbatim}[frame=single,fontsize=\footnotesize]
Fraction f0(1, 2), f1(3, 4), f2;
f2 = +f0;  
f2 = -f0;
f2 = (f0 += f1);
f2 = (f0 -= f1);
f2 = (f0 *= f1);
f2 = (f0 /= f1);
f2 = f0 + f1;  
f2 = f0 - f1;  
f2 = f0 * f1;  
f2 = f0 / f1;
\end{Verbatim}

The \verb!Fraction! class works seamlessly with integers:
\begin{Verbatim}[frame=single,fontsize=\footnotesize]
Fraction f0(1, 2), f1(3, 4), f2;
f2 = f0 + 42;
f2 = f0 - 42;
f2 = f0 * 42;
f2 = f0 / 42;

f2 = 42 + f0;
f2 = 42 - f0;
f2 = 42 * f0;
f2 = 42 / f0;

f2 = (f0 += 42);
f2 = (f0 -= 42);
f2 = (f0 *= 42);
f2 = (f0 /= 42);

int i = 42, j;
j = (i += f0);
j = (i -= f0);
j = (i *= f0);
j = (i /= f0);
\end{Verbatim}

Comparisons are possible:
\begin{Verbatim}[frame=single,fontsize=\footnotesize]
Fraction f0(-1, 2), f1(3, 4);
std::cout << (f0 == f1) << '\n';
std::cout << (f0 != f1) << '\n';
std::cout << (f0 < f1) << '\n';
std::cout << (f0 <= f1) << '\n';
std::cout << (f0 > f1) << '\n';
std::cout << (f0 >= f1) << '\n';
\end{Verbatim}

Finally it's possible to compute the absolute value
and integer power of a \texttt{Fraction}:
\begin{Verbatim}[frame=single,fontsize=\footnotesize]
Fraction f0(-1, 2);
std::cout << abs(f0) << '\n';
std::cout << pow(f0, 5) << '\n';
std::cout << pow(f0, -5) << '\n';
\end{Verbatim}


\newpage
\textsc{Review of pointers}

The following is a quick review of pointers.
Refer to my pointer notes for details.

\textsc{Declaration:}
Suppose \verb!T! is a type 
(example: \verb!int!, \verb!double!, \verb!char!, \verb!bool!, 
some struct, some class),
then you do the following to declare \verb!p! to be a pointer to a 
\verb!T! value:
\begin{console}[frame=single]
T * p;
\end{console}
At this point, \verb!p! does not point to a valid 
value (whether on the heap on in a scope/frame) yet.


\textsc{Allocating/using/deallocating a single value:}
To allocate a single \verb!T! value for \verb!p! to point to
(or we can also say \lq\lq to allocate a value for \verb!p!"),
we do
\begin{console}[frame=single]
p = new T;
\end{console}
After this you can use the value \verb!p! points to.
You refer to this value as \verb!*p!.

Note that the \textit{address} of this value is stored in \verb!p!.
The \textit{value} that \verb!p! points to is \verb!*p!.
Make sure you see the difference between \verb!p! and \verb!*p!.

When you're done using this value, you can release this memory, 
or we say deallocate the value that \verb!p! point to (or just say deallocate \verb!p!),
by doing
\begin{console}[frame=single,fontsize=\footnotesize]
delete p;
\end{console}
Here's a simple example:
\begin{console}[fontsize=\footnotesize]
#include <iostream>

int main()
{
    double * p = new double;
    std::cout << "enter a double: ";
    std::cin >> (*p);

    std::cout << "you entered " << (*p) << std::endl;
    delete p;

    return 0;
}
\end{console}

\textsc{Allocating/using/deallocating an array of values:}
Now suppose instead of a single value of type
\verb!T! you want to use an array of values
of type \verb!T!, say \verb!n! such values.
To allocate an array of \verb!n! 
\verb!T!--values for \verb!p!, you do this:
\begin{console}[frame=single,fontsize=\footnotesize]
p = new T[n];
\end{console}
After this, \verb!p! points to the first of \verb!n!
\verb!T!--values.
The \verb!i!--th \textit{value} of the 
array \verb!p! points to is just
\verb!p[i]!.
The \textit{address} of the
\verb!i!--th value is \verb!p+i!.
Therefore the \verb!i!--th value is also
known as \verb!*(p+i)!, i.e.,
\verb!p[i]!
and
\verb!*(p+i)!
mean the same thing.

After you're done with this array of \verb!T!
values, you can deallocate the memory that \verb!p! points to
(i.e. release the memory back to the heap)
if you do
\begin{console}[frame=single,fontsize=\footnotesize]
delete [] p;
\end{console}

Here's a simple example:
\begin{console}[frame=single,fontsize=\footnotesize]
#include <iostream>

int main()
{
    int n;
    std::cout << "how many double do you need? ";
    std::cin >> n;

    double * p = new double[n];
    for (int i = 0; i < n; i++)
    {
        std::cout << "enter double #" << i << ": ";
        std::cin >> p[i];
    }

    std::cout << "here are all your doubles: ";
    for (int i = 0; i < n; i++)
    {
        std::cout << p[i] << ' ';
    }
    std::cout << std::endl;

    delete [] p;

    return 0;
}
\end{console}
The above iterates over the value that \verb!p! points
to using index values.
You can iterate using a pointer like this:
\begin{console}[frame=single,fontsize=\footnotesize]
#include <iostream>

int main()
{
    int n;
    std::cout << "how many double do you need? ";
    std::cin >> n;

    double * p = new double[n];
    for (int i = 0; i < n; i++)
    {
        std::cout << "enter double #" << i << ": ";
        std::cin >> p[i];
    }

    std::cout << "here are all your doubles: ";
    for (double * q = p; q < p + n; ++q)
    {
        std::cout << (*q) << ' ';
    }
    std::cout << std::endl;

    delete [] p;

    return 0;
}
\end{console}
In this case, the pointer \verb!q! (in the second \verb!for!--loop)
iteratively points to the values
of the array that \verb!p! points to.

\textsc{Functions:}
Of course pointers, like any variable, can be passed to functions.
So of course you can rewrite:
\begin{console}[frame=single,fontsize=\footnotesize]
#include <iostream>

int main()
{
    int n;
    std::cout << "how many double do you need? ";
    std::cin >> n;

    double * p = new double[n];
    for (int i = 0; i < n; i++)
    {
        std::cout << "enter double #" << i << ": ";
        std::cin >> p[i];
    }

    std::cout << "here are all your doubles: ";
    for (int i = 0; i < n; i++)
    {
        std::cout << p[i] << ' ';
    }
    std::cout << std::endl;

    delete [] p;

    return 0;
}
\end{console}
as
\begin{console}[frame=single,fontsize=\footnotesize]
#include <iostream>

void fill_doubles(double * p, int n)
{
    for (int i = 0; i < n; ++i)
    {
        std::cout << "enter double #" << i << ": ";
        std::cin >> p[i];
    }
}

void print_doubles(double * start, double * end)
{
    std::cout << "here are all your doubles: ";
    for (double * q = start; q < end; ++q)
    {
        std::cout << (*q) << ' ';
    }
    std::cout << std::endl;
}

int main()
{
    int n;
    std::cout << "how many double do you need? ";
    std::cin >> n;

    double * p = new double[n];

    fill_doubles(p, n);
    print_doubles(p, p + n);

    delete [] p;

    return 0;
}
\end{console}

Recall that there's a memory debugging tool (google's asan -- address sanitizer)
that can check your code for memory problems.
See class website or my website.
