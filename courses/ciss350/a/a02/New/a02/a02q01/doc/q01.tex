[Review of CISS240]

Here's a very famous unsolved problem in math and computer science. 
Consider the function $f$
defined as follows:
\begin{enumerate}
\item If $n$ is even, then
\[
f(n) = n/2
\]
\item If $n$ is odd, then
\[
f(n) = 3n + 1
\]
\end{enumerate}
For instance
\[
f(7) = 22 \text{ and } f(22) = 11
\]
There is something very special about this function $f$. 
Note that if you start with $n=7$ and continually
apply $f$ you get this:
\begin{align*}
f(7)  &= 7 \times 3 + 1 = 22 \\
f(22) &= 22/2 = 11 \\
f(11) &= 11 \times 3 + 1 = 34 \\
f(34) &= 34/2 = 17 \\
f(17) &= 3 \times 17 + 1 = 52 \\
f(52) &= 52/2 = 26 \\
f(26) &= 26/2 = 13 \\
f(13) &= 13 \times 3 + 1 = 40 \\
f(40) &= 40/2 = 20 \\
f(20) &= 20/2 = 10 \\
f(10) &= 10/2 = 5 \\
f(5)  &= 3 \times 5 + 1 = 16 \\
f(16) &= 16/2 = 8 \\
f(8)  &= 8/2 = 4 \\
f(4)  &= 4/2 = 2 \\
f(2)  &= 2/2 = 1
\end{align*}
and you reach the value of 1. 
A very famous conjecture says that no matter what positive integer $n > 0$
you start with, you will always reach 1 if you apply enough times of $f$.
In the above case of $n = 7$, 
you need to apply $f$ sixteen times to reach 1, i.e.
\[
f(f(f(f(f(f(f(f(f(f(f(f(f(f(f(f(7))))))))))))))))
\]
Write a program that accepts a positive integer value for $n$ 
from the user and prints $n$, the values
obtained from continually applying function $f$, 
and the number of times function $f$ was applied.
Your code should include a (\cpp) function
\[
\verb!int f(int n);!
\]
which computes just like the above (mathematical) function $f$.


\textsc{Test 1}
\begin{console}[frame=single, fontsize=\footnotesize,commandchars=\\\{\}]
\userinput{1}
1 0
\end{console}
[\textsc{Note}: Underlined text refers to input. 
In the above test case, the user enters 1 and press the enter key.]

\textsc{Test 2}
\begin{console}[frame=single,fontsize=\footnotesize,commandchars=\\\{\}]
\userinput{7}
7 22 11 34 17 52 26 13 40 20 10 5 16 8 4 2 1 16
\end{console}


