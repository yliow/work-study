[Review of template class in CISS245 -- pointer class]

This is a review of writing a template class with a pointer that points
to a single value.

Write a template pointer class, \verb!ptr!.
(The previous problem is an array class with a pointer that points to an
array of values. For this problem, the pointer points to a single value.)

The header file is named \verb!ptr.h!.
Each object of \verb!ptr! contains a pointer that when allocated points to
\textit{one} value of type \verb!T! where \verb!T! is the template
type parameter of the \verb!ptr! class.
Call the pointer \verb!p_!.

Here are some usage examples illustrating what your class can do:
\begin{console}[fontsize=\footnotesize]
void f()
{
    ptr< int > p;    // p.p_ points to an int value in the heap
    ptr< double > q; // q.p_ points to a double value in the heap
    ptr< char > r;   // r.p_ points to a char value in the heap
  
    *p = 42;
    std::cout << (*p) << std::endl; // prints 42
    *p = 43;
    std::cout << (*p) << std::endl; // prints 43

    ptr< int > s(42); // here's another constructor. s.p_ will point to
                      // a 42 in the heap.
    
    // here ... where p,q,r,s go out of scope, their destructor will
    // deallocate memory used by their pointers.
}
\end{console}
Clearly you must overwrite the default destructor, copy constructor,
and assignment operator of \verb!ptr!.
\begin{console}[fontsize=\footnotesize]
void g()
{
    ptr< int > p(42); 
    ptr< int > q(p);  // q.p_ points to 42 in the heap which is different
                      // from the 42 that p.p_ points to
    ptr< int > r;
    r = p;            // the integer that r.p_ points to is changed to 42
}
\end{console}

Of course you have must ensure that there is no memory issues such as
memory leaks and double free errors.
You should definitely check your code with google asan.

\end{document}
