[Review of CISS245]

This is a very minor modification to Q3. 
The program now accepts a boolean value (besides the values described in Q3) 
whether or not to print the number of tries and the list of numbers obtained.

\textsc{Test 1}
\begin{console}[frame=single,fontsize=\footnotesize, commandchars=\\\{\}]
\userinput{3 1 2 1 9 1000 1}
3 1 2 1 1000 pass 0 1
3 1 2 2 1000 pass 1 2 1
3 1 2 3 1000 pass 7 3 10 5 16 8 4 2 1
3 1 2 4 1000 pass 2 4 2 1
3 1 2 5 1000 pass 5 5 16 8 4 2 1
3 1 2 6 1000 pass 8 6 3 10 5 16 8 4 2 1
3 1 2 7 1000 pass 16 7 22 11 34 17 52 26 13 40 20 10 5 16 8 4 2 1
3 1 2 8 1000 pass 3 8 4 2 1
3 1 2 9 1000 pass 19 9 28 14 7 22 11 34 17 52 26 13 40 20 10 5 16 8 4 2 1
\end{console}

In the case of Test 1, 
the last input value is 1 which indicates a request to print the values after 
the \lq\lq pass/fail'' column.


\textsc{Test 2}
\begin{console}[frame=single,fontsize=\footnotesize, commandchars=\\\{\}]
\userinput{3 1 2 1 9 1000 0}
3 1 2 1 1000 pass
3 1 2 2 1000 pass
3 1 2 3 1000 pass
3 1 2 4 1000 pass
3 1 2 5 1000 pass
3 1 2 6 1000 pass
3 1 2 7 1000 pass
3 1 2 8 1000 pass
3 1 2 9 1000 pass
\end{console}
In the case of Test 2, 
the last input value is 0 which indicates a request not to print the 
values after the \lq\lq pass/fail'' column.

\textsc{Test 3}
\begin{console}[frame=single,fontsize=\footnotesize, commandchars=\\\{\}]
\userinput{3 1 2 1 9 5 1}
3 1 2 1 5 pass 0 1
3 1 2 2 5 pass 1 2 1
3 1 2 3 5 fail 5 3 10 5 16 8 4
3 1 2 4 5 pass 2 4 2 1
3 1 2 5 5 pass 5 5 16 8 4 2 1
3 1 2 6 5 fail 5 6 3 10 5 16 8
3 1 2 7 5 fail 5 7 22 11 34 17 52
3 1 2 8 5 pass 3 8 4 2 1
3 1 2 9 5 fail 5 9 28 14 7 22 11
\end{console}

\textsc{Test 4}
\begin{console}[frame=single,fontsize=\footnotesize,commandchars=\\\{\}]
\userinput{3 1 2 1 9 5 0}
3 1 2 1 5 pass
3 1 2 2 5 pass
3 1 2 3 5 fail
3 1 2 4 5 pass
3 1 2 5 5 pass
3 1 2 6 5 fail
3 1 2 7 5 fail
3 1 2 8 5 pass
3 1 2 9 5 fail
\end{console}
