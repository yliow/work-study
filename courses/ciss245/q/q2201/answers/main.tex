%-*-latex-*-
\newcommand\COURSE{ciss350}
\newcommand\ASSESSMENT{a01}
\newcommand\ASSESSMENTTYPE{Assignment}
\newcommand\POINTS{	extwhite{xxx/xxx}}

\input{myquizpreamble}
\input{yliow}
\renewcommand\TITLE{\ASSESSMENTTYPE \ \ASSESSMENT}

\renewcommand\EMAIL{}
\input{\COURSE}
\textwidth=6in

 

\newcommand\BLANK{\sqcup}
% Used in DFA minimization
\newcommand\ind{\operatorname{index}}


\makeindex
\begin{document}
\topmatter


\renewcommand\AUTHOR{jdoe5@cougars.ccis.edu} % CHANGE TO YOURS

\begin{document}
\topmattertwo


In any of the following questions, write ERROR if there's
an error in the code fragment.

\nextq
The following program does not run.
Insert \textit{one} statement to correct the problem.
\\
\textsc{Answer:}\vspace{-2mm}
\begin{answercode}
#include <iostream>

int sum(int start, int end, int step);

int main()
{
    std::cout << sum(5, 10, 1) << std::endl;
    return 0;
}

int sum(int start, int end, int step)
{
    int s = 0;
    for (int i = start; i <= end; i += step)
    {
        s += i;
    }
    return s;
}
\end{answercode}

%------------------------------------------------------------------------------
\nextq
The function \verb!play_audio()! plays music file an audio file. The parameters are:
\begin{enumerate}[nosep]
\li \verb!filename!: a C-string that is the name of the audio file to be load
\li \verb!track_number!: an integer.
$0$ is the first track, $1$ is the second track, etc.
The default value is $-1$ which is \lq\lq play all tracks".
\li \verb!repeat!: a boolean. The default is \verb!false!.
\end{enumerate}
The function returns a $0$ if there are no errors,
a $-1$ if the file cannot be found using the \verb!filename!,
$-2$ if the \verb!track_number! is not $-1$ and the track number is not found
in the audio file.
Write down the prototype of this function.
\\
\textsc{Answer:}\vspace{-2mm}
\begin{answercode}
int play_audio(char filename[], int track_number, bool repeat = false);
\end{answercode}

%------------------------------------------------------------------------------
\nextq
The following function call
\begin{console}
push_back(x, x_len, 42);
\end{console}
sets \verb!x[x_len]! to \verb!42! and increments \verb!x_len! by 1.
Write down the function prototype of \verb!push_back!.
\\
\textsc{Answer:}\vspace{-2mm}
\begin{answercode}
void push_back(int x[], int & x_len, int value);
\end{answercode}

%------------------------------------------------------------------------------
\newpage
You are given the following (possibly incomplete files):
\begin{tightlist}
  \li \texttt{Rational.h}
  \li \texttt{Rational.cpp}
  \li \texttt{main.cpp} (the test code)
\end{tightlist}
\textsc{Important Warning:}
Again, the files are meant to be skeleton file and might not be
complete and might have deliberate missing details or even errors.

Create directory
\texttt{ciss245/a/a07/a07q01}.
Keep all your files in this directory.

If you're doing a copy-and-paste of the given code,
note that some character might be changed by PDF to other characters.
In particular the - character might actually not be the dash character.
Looking at the compiler error message will help you find these minor
annoying issues so that you can correct them.

Study the given test code.
Add tests if necessary to test all methods and functions.
Such low level function/method tests are called \defterm{unit tests}.

Observe the following very carefully:
\begin{tightlist}
\li All methods must be constant whenever possible. 
\li All parameters which are objects (or struct variables)
must be pass by reference or pass by constant reference as much as possible.
\li Reuse code as much as possible.
For instance \verb@operator!=()@ should use \verb!operator==()!. 
\end{tightlist}
Let me know ASAP if you see a typo.

\end{document}