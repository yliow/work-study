%-*-latex-*-
A group of 10 rebels were trapped in a cave. Very soon an army of invading soldiers will be upon them. Instead
of surrendering themselves, they decided to sit in a circle, and starting with someone and going around the
circle, every other person will take his own life. Here's the initial setup.
\begin{python}
from latextool_basic import *
p = Plot()
p += Circle(x=0,y=0,r=2,background='blue!10')
deg = 90
for i in range(10):
    p += r'\draw (%s: 2.3) node {$%s$};' % (deg, i)
    deg -= 360/10
print(p)
\end{python}
  
First the rebel at 0 will take his own life:
\begin{python}
from latextool_basic import *
p = Plot()
p += Circle(x=0,y=0,r=2,background='blue!10')
deg = 90
for i in range(10):
    if i not in [0]:
        p += r'\draw (%s: 2.3) node {$%s$};' % (deg, i)
    else:
        p += r'\draw (%s: 2.3) node {\textwhite{$%s$}};' % (deg, i)
    deg -= 360/10
print(p)
\end{python}

The next to take his life is the person at 2, i.e., going clockwise, it's
the 
second person alive after the person at 0 who just died:
\begin{python}
from latextool_basic import *
p = Plot()
p += Circle(x=0,y=0,r=2,background='blue!10')
deg = 90
for i in range(10):
    if i not in [0,2]:
        p += r'\draw (%s: 2.3) node {$%s$};' % (deg, i)
    else:
        p += r'\draw (%s: 2.3) node {\textwhite{$%s$}};' % (deg, i)
    deg -= 360/10
print(p)
\end{python}

This is followed by 4:
\begin{python}
from latextool_basic import *
p = Plot()
p += Circle(x=0,y=0,r=2,background='blue!10')
deg = 90
for i in range(10):
    if i not in [0,2,4]:
        p += r'\draw (%s: 2.3) node {$%s$};' % (deg, i)
    else:
        p += r'\draw (%s: 2.3) node {\textwhite{$%s$}};' % (deg, i)
    deg -= 360/10
print(p)
\end{python}
and then 6:
\begin{python}
from latextool_basic import *
p = Plot()
p += Circle(x=0,y=0,r=2,background='blue!10')
deg = 90
for i in range(10):
    if i not in [0,2,4,6]:
        p += r'\draw (%s: 2.3) node {$%s$};' % (deg, i)
    else:
        p += r'\draw (%s: 2.3) node {\textwhite{$%s$}};' % (deg, i)
    deg -= 360/10
print(p)
\end{python}
Next is 8:
\begin{python}
from latextool_basic import *
p = Plot()
p += Circle(x=0,y=0,r=2,background='blue!10')
deg = 90
for i in range(10):
    if i not in [0,2,4,6,8]:
        p += r'\draw (%s: 2.3) node {$%s$};' % (deg, i)
    else:
        p += r'\draw (%s: 2.3) node {\textwhite{$%s$}};' % (deg, i)
    deg -= 360/10
print(p)
\end{python}
Now note that the among the remaining rebels, 
the second to the left of where 8 was is the rebel at position 1.
Therefore the next to go is 1:
\begin{python}
from latextool_basic import *
p = Plot()
p += Circle(x=0,y=0,r=2,background='blue!10')
deg = 90
for i in range(10):
    if i not in [0,2,4,6,8,1]:
        p += r'\draw (%s: 2.3) node {$%s$};' % (deg, i)
    else:
        p += r'\draw (%s: 2.3) node {\textwhite{$%s$}};' % (deg, i)
    deg -= 360/10
print(p)
\end{python}
After 1, the next to be removed is 5:
\begin{python}
from latextool_basic import *
p = Plot()
p += Circle(x=0,y=0,r=2,background='blue!10')
deg = 90
for i in range(10):
    if i not in [0,2,4,6,8,1,5]:
        p += r'\draw (%s: 2.3) node {$%s$};' % (deg, i)
    else:
        p += r'\draw (%s: 2.3) node {\textwhite{$%s$}};' % (deg, i)
    deg -= 360/10
print(p)
\end{python}
then 9, followed by 7. Note that 3 is the last rebel.

In general given $n$ rebels labeled $0$ to
 $n-1$, with the same algorithm for removing the rebels, 
who is the last surviving rebel?
In the case of $n = 10$, the last surviving rebel is
rebel number $3$.

Write a function with prototype
\[
\texttt{int last\_alive(int n);}
\]
that returns the index of the last surviving rebel.
For instance, \texttt{last\_alive(10)} returns \verb!3!.
Your program must allow a maximum number of 1000 rebels.

\textsc{Test 1}
\begin{console}[commandchars=\\\{\}]
\userinput{10}
3
\end{console}

\textsc{Test 2}
\begin{console}[commandchars=\\\{\}]
\userinput{5}
1
\end{console}

WARNING: INCOMING SPOILER ... Hint is on the next page.
Use only if necessary.

\newpage
\textsc{Hint}

For Test 1, think about using an array of 10 boolean values.
Say the array is called \verb!alive!.
Set all 10 values of \verb!alive! to \verb!true!.
Go through the array just like going through the circle.
Say you use an index variable \verb!i!.
Once \verb!i! goes beyond \verb!9!, i.e.,
when \verb!i! reaches 10, you want to set it to \verb!0!.
In other words the values taken by \verb!i! is
\verb!0!,
\verb!1!,
\verb!2!,
\verb!3!,...
\verb!7!,
\verb!8!,
\verb!9!,
\verb!0!,
\verb!1!,
\verb!2!,
\verb!3!,...
\verb!7!,
\verb!8!,
\verb!9!,
\verb!0!,
\verb!1!,
\verb!2!,
etc.

Starting with \verb!i = 0!, set \verb!alive[i]! to \verb!false!.
You then advance \verb!i! to the next
\verb!alive[i]! that is
\verb!true!, and you keep going
until you reach the \textit{second} \verb!alive[i]! value that is \verb!true!,
you stop and set that \verb!alive[i]! to \verb!false!. Repeat.
You stop when the \textit{second}
\verb!alive[i]! that is \verb!true!
is where you started.
That tells you that there's only one alive.

The above only works for an array of 10 boolean values.
The problem states that it must work for different input.
For instance the user can enter \verb!150!.
The problem does say that you only need to consider a maximum input
of \verb!1000!.
So declare a boolean array of size \verb!1000!.
If the user enters, say, \verb!15!, you just use \verb!x[0]!, ...,
\verb!x[14]! and ignore the rest.
If the user enters \verb!42!, then just use
\verb!x[0]!, ...,
\verb!x[41]!
and ignore the rest.
