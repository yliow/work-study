%-*-latex-*-

This is a practice on pointers and using the free store (memory heap).

In the area of computer vision, it's very common to perform
certain \lq\lq weighted averaging" computation on 2D arrays of pixels.
For instance suppose an array has the following value
\[
\texttt{0 0 1 0 1 3 5 1}
\]
For the value at index 2, 
\[
\texttt{0 0 \underline{1} 0 1 3 5 1}
\]
if you compute the average of the value at index 1, 2, 3, you have
\verb!(0 + 1 + 0)/3! which will give you \verb!0! (I'm doing
integer division here):
\begin{align*}
\texttt{0 \underline{0 1 0} 1 3 5 1} & \\
\texttt{\textwhite{0 0} \underline{0} \textwhite{1 1 1 1 1}} &
\end{align*}
For the value at index 5, 
\[
\texttt{0 0 1 0 1 \underline{3} 5 1}
\]
<<<<<<< HEAD
if you compute the average of the value at index 5, 6, 7, you have
=======
if you compute the average of the value at index 4, 5, 6, you have
>>>>>>> fd78148a635584414d0f8a155afb8fb53d60ce54
\verb!(1 + 3 + 5)/3! which will give you \verb!3!:
\begin{align*}
\texttt{0 0 1 0 \underline{1 3 5} 1} & \\
\texttt{\textwhite{0 0 0 0 0 } 3 \textwhite{1 1}} &
\end{align*}

For this computation the average must use the values from the given array.
So you'll need to put the average values into another array.
Note that the first value and the last value do not have left and right
neighbors, so the computation is not performed on the first and last value
-- they are just copied to the target array.
I'll then have these two arrays where the top is the given array
and the bottom is the averaged array:
\begin{align*}
\texttt{0 0 1 0 1 3 5 1} & \\
\texttt{0 0 0 0 1 3 3 1} &
\end{align*}
This is an oversimplified example of what is called a linear
filter.
Linear filters are part of an area of study called signal processing.
Computer vision intersects signal processing.

You can generalize the above a little.
Let me do this in 2 steps.

First,
the average computation
\[
\texttt{(0 + 1 + 0)/3}
\]
can be generalized so that \lq\lq weights"
\verb!a!,
\verb!b!,
\verb!c!,
are added to the above computation:
\[
\texttt{(a * 0 + b * 1 + c * 0)/(a + b + c)}
\]
For the example above, the weights are
$\texttt{a}=1,\texttt{b}=1,\texttt{c}=1$.
If I use weights $1,2,1$, then at index 2,
\begin{align*}
\texttt{0 \underline{0 1 0} 1 3 5 1} & \\
\texttt{\textwhite{0 0} 0 \textwhite{1 1 3 5 1}} &
\end{align*}
the weighted average for \verb!0 1 0! is
\[
\texttt{(1 * 0 + 2 * 1 + 1 * 0) / (1 + 2 + 1) = 2 / 4}
\]
which is \verb!0! if I use integer division.

Second, you can further generalize the above by averaging over different number
of values.
In the above example, a value is averaged over three values.
Now let me average over 5 values.
For the value at index 2, I can average over 5 values with weights
$1,4,7,4,1$.
For the value at index 2:
\[
\texttt{\underline{0 0 1 0 1} 3 5 1}
\]
I then get
\[
\texttt{1 * 0 + 4 * 0 + 7 * 1 + 4 * 0 + 1 * 1 / (1 + 4 + 7 + 4 + 1)
= 9 / 17}
\]
which is 0 if I use integer division.
\begin{align*}
\texttt{\underline{0 0 1 0 1} 3 5 1} & \\
\texttt{\textwhite{0 0} \underline{0} \textwhite{1 1 3 5 1}} &
\end{align*}
In this case when there are five weights, the first two values
and the last two values are not averaged -- we just copy them to the
target array.

\begin{Verbatim}[commandchars=\~\!\@,frame=single]
#include <iostream>

void filter(int x[], int n)
{
    int * p;
    // Allocate an integer array of size n for p to point to.

    // Compute the weighted average values of x and store in the array
    // that p points to.

    // Copy the values in the array that p points to into
    // the values of array x.

    // Deallocate the memory used by p.
    
    return; 
}


int main()
{
    int x[1024];
    
    int n = 0;
    std::cin >> n;
    for (int i = 0; i < n; ++i)
    {
        std::cin >> x[i];
    }
    filter(x, n);
    for (int i = 0; i < n; ++i)
    {
        std::cout << x[i] << ' ';
    }
    std::cout << '\n';
    
    return 0;
}
\end{Verbatim}



\textsc{Test 1}
\begin{console}[commandchars=\\\{\}]
\underline{\texttt{8}}
\underline{\texttt{0 0 1 0 1 3 5 1}}
0 0 0 0 1 3 3 1 \\
\end{console}
