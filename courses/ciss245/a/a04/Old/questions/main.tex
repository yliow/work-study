%-*-latex-*-
\input{myassignmentpreamble}
\input{ciss245}
\input{yliow}
\renewcommand\TITLE{Assignment 4}

\begin{document}
\topmatter

\textsc{Objectives}
 \begin{enumerate}[nosep]
 \item Declare pointers
 \item Allocate memory for pointers
 \item Deallocate memory for pointers
 \item Use pointers for parameter passing
 \end{enumerate}


\textsc{Instructions}
\begin{enumerate}
\li Your program must be well-written. 
    You must follow the style in your notes as closely as possible. 
    Take note of the spaces and blank lines I used in my examples. 
    Badly written programs will very likely result in a poor grade for this 
    assignment. 
    Points will be taken off for sloppy work. 
\li It's important to remember this: In your printouts of your \cpp\ source
files, 
    there must be no wraparound. So each line of text in your C++ program
    should be at most 80 characters. (I usually limit to 79 characters.)
\li All outputs must match exactly the output shown. 
    That includes every single space and every blank line.

\li The format of your program must look like this
(replacing \lq\lq John Doe'' with your name of course!):
\end{enumerate}
\begin{Verbatim}[frame=single]
// File: a01q01.cpp
// Name: smaug

#include <iostream>

int main()
{
    *** YOUR WORK HERE ***

    return 0;
}
\end{Verbatim}

\begin{enumerate}
\item[] In particular:
\begin{enumerate}
\li You must have your name and the name of the file at the top of each 
    \cpp\ source file as shown above.
\li Your \cpp\ source file must end with a blank line.
\end{enumerate}

\end{enumerate}

Read the questions carefully before diving in!

%=================================================================================
\newpage Q1. [Pointer parameters and pass-by-reference]

%-*-latex-*-

\begin{ex} 
  \label{ex:prob-00}
  \tinysidebar{\debug{exercises/{disc-prob-28/question.tex}}}

  \solutionlink{sol:prob-00}
  \qed
\end{ex} 
\begin{python0}
from solutions import *
add(label="ex:prob-00",
    srcfilename='exercises/discrete-probability/prob-00/answer.tex') 
\end{python0}


%=================================================================================
\newpage Q2. [Pointer parameters and pass-by-reference]

%-*-latex-*-

\begin{ex} 
  \label{ex:prob-00}
  \tinysidebar{\debug{exercises/{disc-prob-28/question.tex}}}

  \solutionlink{sol:prob-00}
  \qed
\end{ex} 
\begin{python0}
from solutions import *
add(label="ex:prob-00",
    srcfilename='exercises/discrete-probability/prob-00/answer.tex') 
\end{python0}


%=================================================================================
\newpage Q3. [Pointer to array element and pointer arithmetic]

This is a practice on using pointers to access array elements.
Complete the following program by implementing the given functions.
You MUST use the given skeleton code.
The variables you need in the functions are already declared for you.
Do NOT declare any other extra variables.

\begin{Verbatim}[frame=single,commandchars=\~\!\@]
#include <iostream>


// Swaps the values *p, *q
void swap(int * p, int * q)
{
    int t;
}


// Performs bubblesort from *begin to one int before *end.
// For swapping, you should use the above swap function.
void bubblesort(int * begin, int * end)
{
    int * p;
    int * q;
}


// Prints array from *begin to one int before *end. The print format is
// {1, 4, -2, 42}.
void println(int * begin, int * end)
{
    int * p;
    // TO BE COMPLETED
    std::cout << '\n';
}


int main()
{
    int x[1024];
    int n;
    std::cin >> n;
    for (int i = 0; i < n; ++i)
    {
        std::cin >> x[i];
    }
    println(&x[0], &x[n]);
    
    bubblesort(&x[0], &x[n]);
    println(&x[0], &x[n]);
    
    return 0;
}
\end{Verbatim}

\textsc{Test 1}
\begin{console}[commandchars=\~\!\@]
~underline!~texttt!5@@
~underline!~texttt!15 13 11 12 14@@
{15, 13, 11, 12, 14}
{11, 12, 13, 14, 15}
\end{console}
(You are advised to perform more tests on your own.)

%=================================================================================
% factorial -- not good example
% linear filter?
% 1 1 0 1 0 0 0 1
% 1 1 1 0 0 0 0 1
\newpage Q4. [Pointers and memory management]
%%-*-latex-*-

\begin{ex} 
  \label{ex:prob-00}
  \tinysidebar{\debug{exercises/{disc-prob-28/question.tex}}}

  \solutionlink{sol:prob-00}
  \qed
\end{ex} 
\begin{python0}
from solutions import *
add(label="ex:prob-00",
    srcfilename='exercises/discrete-probability/prob-00/answer.tex') 
\end{python0}

%-*-latex-*-

\begin{ex} 
  \label{ex:prob-00}
  \tinysidebar{\debug{exercises/{disc-prob-28/question.tex}}}

  \solutionlink{sol:prob-00}
  \qed
\end{ex} 
\begin{python0}
from solutions import *
add(label="ex:prob-00",
    srcfilename='exercises/discrete-probability/prob-00/answer.tex') 
\end{python0}
 

%=================================================================================
\newpage Q5. [Pointers and memory management]

%-*-latex-*-

\begin{ex} 
  \label{ex:prob-00}
  \tinysidebar{\debug{exercises/{disc-prob-28/question.tex}}}

  \solutionlink{sol:prob-00}
  \qed
\end{ex} 
\begin{python0}
from solutions import *
add(label="ex:prob-00",
    srcfilename='exercises/discrete-probability/prob-00/answer.tex') 
\end{python0}



%\newpage Q3. %-*-latex-*-

\begin{ex} 
  \label{ex:prob-00}
  \tinysidebar{\debug{exercises/{disc-prob-28/question.tex}}}

  \solutionlink{sol:prob-00}
  \qed
\end{ex} 
\begin{python0}
from solutions import *
add(label="ex:prob-00",
    srcfilename='exercises/discrete-probability/prob-00/answer.tex') 
\end{python0}

%\newpage Q3. OMITTED

\newpage Q6. [Pointers and memory management]

%-*-latex-*-

\begin{ex} 
  \label{ex:prob-00}
  \tinysidebar{\debug{exercises/{disc-prob-28/question.tex}}}

  \solutionlink{sol:prob-00}
  \qed
\end{ex} 
\begin{python0}
from solutions import *
add(label="ex:prob-00",
    srcfilename='exercises/discrete-probability/prob-00/answer.tex') 
\end{python0}


\end{document}
