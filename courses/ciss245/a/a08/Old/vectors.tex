\textsc{Vectors}

Here's a quick note about vectors.
A 2-dimensional vector is sort of like a point.
For instance you know
(from MATH106 elementary algebra) that the following is a point:
\[
	(1.1, 3.5)
\]
Here's another point:
\[
	(0, 0)
\]
A 2-dimensional vector is similar but they are frequently written like this:
\[
	\langle 1.1, 3.5 \rangle
\]
The $1.1$ is called the $x$-component of the above vector and $3.5$ is
the $y$-component of the vector. You can add vectors. Here's an example:
\[
	\langle 1.1, \,\,\, 3.5 \rangle
        +
        \langle 2.2, \,\,\, 4.1 \rangle
        =
        \langle 1.1 + 2.2, \,\,\, 3.5 + 4.1 \rangle
        =
        \langle 3.3, \,\,\, 7.6 \rangle
\]
In other words you just add their corresponding components to get a new vector.
In general
\[
	\langle a, \,\,\, b \rangle
        +
        \langle c, \,\,\, d \rangle
        =
        \langle a + c, \,\,\, b + d \rangle
\]

You can multiply vector with a \verb!double!:
\[
	2.0 \cdot  \langle 1.1, \,\,\, 3.5 \rangle =
        \langle 2 \cdot 1.1, \,\,\, 2 \cdot 3.5 \rangle
        =
        \langle 2.2, 7.0 \rangle
\]
In general
\[
	c \cdot \langle a, \,\,\, b \rangle
        =
        \langle ca, \,\,\, cb \rangle
\]
Of course you can multiply the \verb!double! on the right of the vector:
\[
	\langle 1.1, 3.5 \rangle  \cdot  2.0 = 
        \langle 2.2, 7.0 \rangle
\]
This product is called the scalar product.
Not too surprising, you can divide vectors with a \texttt{double}:
\[
	\langle 1.1, 3.5 \rangle / 2.0 =
        \langle 1.1 / 2.0, 3.5 / 2.0 \rangle =  \langle 0.55, 1.75 \rangle
\]
In general
\[
	\cdot \langle a, \,\,\, b \rangle / c
        =
        \langle a/c, \,\,\, b/c \rangle
\]
There's another product for vectors called the dot product.
Mathematically, it is written like this:
\[
	\langle 1.1, \,\,\, 3.5 \rangle \cdot \langle 2.2, \,\,\, 4.1 \rangle
\]
This is defined to be the sum of the product of the components, i.e.,
\[
	\langle 1.1, \,\,\, 3.5 \rangle \cdot
        \langle 2.2, \,\,\, 4.1 \rangle
        =
        (1.1)(2.2) + (3.5)(4.1)
        = 2.42 + 14.35 = 16.77
\]
In general
\[
	\langle a, \,\,\, b \rangle \cdot
        \langle c, \,\,\, d \rangle
        =
        ac + bd
\]
Of course the dot product is a real number and not a vector.

The length of a vector $\langle x, y \rangle$ is defined by Pythagorus formula:
\[
\operatorname{len}( \langle x, \,\,\, y \rangle ) = \sqrt{x^2 + y^2}
\]
The normalization of a vector is the vector divided by its length.
Therefore the normalization of
$\langle 1.1, 3.5 \rangle$ is
\begin{align*}
\langle 1.1, \,\,\, 3.5 \rangle / \sqrt{1.1^2 + 3.5^2}
&= \langle 1.1, \,\,\, 3.5 \rangle / \sqrt{13.46} \\
&= \langle 1.1, \,\,\, 3.5 \rangle / 3.6687\cdots \\
&= \langle 1.1/3.6687\ldots, \,\,\, 3.5/3.6687\ldots \rangle \\
&= \langle 1.1/3.6687\ldots, \,\,\, 3.5/3.6687\ldots \rangle \\
&= \langle 0.2998\ldots, \,\,\, 0.9539\ldots \rangle 
\end{align*}

So what are vectors good for?
While points are just points in space, vectors represent motion. The vector
\[
	\langle 1.1, -3.5 \rangle
\]
represents the fact that something is moving in the $x$--axis direction by
$1.1$ and in the $y$--axis direction by $-3.5$.
Vectors are used extensive in modeling motion in the
real world.
Therefore it is extremely
important in physics and therefore extremely important in engineering
and computer
science. For instance, in scientific computation/simulation
and computer games,
frequently, you're modeling some physical entities in the real world. You need
vectors to describe their motion. To simulate water waves, you need vectors.

\begin{center}
  \includegraphics[width=4in]{vec2d.JPG}
\end{center}

(Yes, the above is really generated by a program – it's not a photo.)
Vectors are really everywhere.

The purpose of this assignment is to build a vector library.
This will only be for
2-dimensional vectors and therefore will be useful for 2D games.
But once you have
done 2-dimensional vectors, it's not too difficult to build a library for
3-dimentional vectors or even for n-dimensional vectors.
You do not need to know how to use
vectors to solve problems and you do not know about fluid
dynamics for simulating sea
waves.
You only need to know how to interpret basic numeric computations and the
language of \cpp\ object-oriented programming.





\newpage
\texttt{vec2d} \textsc{header file}

Note that the header file contains both methods and functions.
The prototypes for the
functions are given. All others must be methods in the class.

{\small
\VerbatimInput[frame=single]{a08q01/skel/vec2d.h}
}

\newpage
\texttt{vec2d} \textsc{Implementation File}


\textsc{Warning:}
Note that the functions in the header file are outside the class,
Therefore
they do not have direct access to the private instance variables
\verb!x_! and \verb!y_! of the
object.

{\small
\VerbatimInput[frame=single]{a08q01/skel/vec2d.cpp}
}

\newpage
\textsc{Test File}

The following is only a skeleton test code.
Complete it and test your library thoroughly.

{\small
\VerbatimInput[frame=single]{a08q01/skel/main.cpp}
}

Here are the test option numbering. First here are the input and output
operators
(i.e., \verb!operator<<! and
\verb!operator>>!) and class methods (i.e. member functions and
operators) in the class:
\begin{itemize}
  \item[ 1.] Constructor \verb!vec2d::vec2d(double, double)! and printing
  \item[ 2.] \textred{There's no case 2.}
  \item[ 3.] Constructor \verb!vec2d::vec2d()! (the default) and print
  \item[ 4.] Input (i.e. \verb!operator>>!)
  \item[ 5.] \verb!bool vec2d::operator==(const vec2d &) const!
  \item[ 6.] \texttt{bool vec2d::operator!=(const vec2d \&) const}
  \item[ 7.] \verb!double vec2d::get_x() const!
  \item[ 8.] \verb!double vec2d::get_y() const!
  \item[ 9.] \verb!void vec2d::set_x(double)!
  \item[10.] \verb!void vec2d::set_y(double)!
  \item[11.] \verb!double vec2d::operator[](int) const!
  \item[12.] \verb!double & vec2d::operator[](int)!
  \item[13.] \verb!vec2d vec2d::operator+() const!
  \item[14.] \verb!vec2d & vec2d::operator+=(const vec2d &)!
  \item[15.] \verb!vec2d vec2d::operator+(const vec2d &) const!
  \item[16.] \verb!vec2d vec2d::operator-() const!
  \item[17.] \verb!vec2d & vec2d::operator-=(const vec2d &)!
  \item[18.] \verb!vec2d vec2d::operator-(const vec2d &) const!
  \item[19.] \verb!vec2d & vec2d::operator*=(double)!
  \item[20.] \verb!vec2d vec2d::operator*(double) const!
  \item[21.] \verb!vec2d & vec2d::operator/=(double)!
  \item[22.] \verb!vec2d vec2d::operator/(double) const!
  \item[23.] \verb!double len() const!
\end{itemize}
And here are the test options for non-member functions:
\begin{itemize}
  \item[24.] \verb!vec2d operator*(double, const vec2d &)!
  \item[25.] \verb!double len(const vec2d &)!
  \item[26.] \verb!double dotprod(const vec2d &, const vec2d &)!
  \item[27.] \verb!vec2d norm(const vec2d &)!
\end{itemize}


\newpage
\textsc{Notes}

Be efficient when you code.
For instance \verb!operator!=! is the opposite of \verb!operator==!.
So get \verb!operator==! to work correctly and then code
\texttt{operator!=} using \verb!operator==!.

Likewise the binary
\verb!operator+! should be coded in terms of
\verb!operator+=!, etc.
It's really a very simple idea.
Here's a hint for you.
Suppose you look at integers.
How do you define \verb!+! in terms of \verb!+=!?
Suppose \verb!x!, \verb!y!, \verb!z! are integers and
you want to do the following:
{\small
\begin{console}
int x, y = 2, z = 3;
x = y + z;
\end{console}
}


Well, \verb!y + z! is really the value of \verb!y! incremented by \verb!z!.
Right? So the above statement can be written as:
{\small
\begin{console}
int x, y = 2, z = 3;
x = y;
x += z;
\end{console}
}

Notice now that \verb!+! does not appear in the two statements,
only \verb!=! and \verb!+=!.
On the other hand, it's easy to see that you can also write
\verb!+=! using \verb!+!.
However it's generally expected that \verb!+=! should be faster than
\verb!+!.
That means that \verb!+! should be implemented in terms of
\verb!+=! and not the other way round.

You know if an argument of a function is not changed,
you should declare it constant.
For instance void
\verb!f(const vec2d & a)! would imply that the body/implementation of the
function \verb!f! cannot change the value(s) of \verb!a!.
However for a class, the object invoking the method does not appear
in the parameter list:
{\small
\begin{console}
  class C
{
public:
    void f();
    ...
};
\end{console}
}

So what if you have
{\small
\begin{console}
int main()
{
    C obj;
    obj.f();
}
\end{console}
}
and what if you want to insist that
\verb!f()! does not change \verb!obj!?
You can make the object invoking the call constant within \verb!f()!.
Here's how you do it:
{\small
\begin{console}[commandchars=\~\!\@]
class C
{
public:
    void f() ~textbox!const@;
    ...
};
\end{console}
}
You've just made method \verb!f()! \defterm{constant}.
We say that f() is a \defterm{constant method}.

Note that so far the object invoking a method does not have a
\lq\lq name''
since it does not appear as a parameter in the method's prototype.
So what if we need to return the object itself?
For instance what if I want to write a \verb!clone()! method that returns a
copy of the object:
{\small
\begin{console}
// Robot.h
#ifndef ROBOT_H
#define

class Robot
{
public:
    ...
    Robot clone() const;
    ...
};

#endif
\end{console}
}

Actually there \textit{is} a parameter for the object invoking the
call to \verb!clone()!.
It's called \verb!this! in the implementation of the
\verb!clone()! method.
However it's not a \verb!Robot! object.
It's a pointer to the \verb!Robot! object invoking the call to \verb!clone()!.
{\small
\begin{console}
// Robot.cpp
...
Robot Robot::clone() const
{
    return *this; // need to dereference the "this" pointer
}
...
\end{console}
}

If \verb!u! and \verb!v! are \verb!vec2d! objects and you have a
method \verb!operator+=!:
{\small
\begin{console}
class vec2d
{
public:
    ...
    vec2d & operator+=(const vec2d &);
};
\end{console}
}
Then the statement
\[
	\texttt{u += v;}
\]
is \lq\lq translated'' to 
\[
	\texttt{u.operator+=(v);}
\]
Suppose \verb!u!, \verb!v!, \verb!w! are \verb!vec2d! objects. The statement:
\[
	\texttt{w = u + v} ;
\]
can be \lq\lq translated'' in two different ways:
\[
	\texttt{w = operator+(u, v);}
\]
or
\[
	\texttt{w = u.operator+(v);}
\]
which will call either the \textit{nonmember function}
\[
	\texttt{vec2d operator+(const vec2d \&, const vec2d \&);}
\]
or the \textit{method} (in the \verb!vec2d! class)
\[
	\texttt{vec2d vec2d::operator+(const  vec2d \&) const}
\]
This means that you cannot have both the method
\texttt{operator+(const vec2d \&) const} and the function
\texttt{operator+(const vec2d \&, const vec2d \&)}:
{\small
\begin{console}
class vec2d
{
public:
    ...
    vec2d operator+(const vec2d &) const;
    ...
};

...
vec2d operator+(const vec2d &, const vec2d &); // WRONG
...
\end{console}
}
This will cause an error during compiling.


This will result in an \textit{ambiguous declaration}.
For the \verb!vec2d! class, we want \verb!operator+! to be a method,
not a function.

As for the bracket operator, when you do
\[
	\texttt{std::cout << v[0];}
\]
you're really do this:
\[
	\texttt{std::cout << v.operator[](0);}
\]
This requires you to retrieve the value of \verb!v.x_!.
However if you do this:
\[
	\texttt{v[0] = 4.2;}
\]
then you're retrieve the value of \verb!v.x_!.
Instead you want to put \verb!4.2! at \verb!v.x_!.
That's why there are two \verb!operator[]! in your
\verb!vec2d! class.
(Check your notes on returning references to instance variables.)

\newpage
\textsc{Reminder: Giving access to instance variable by returning a reference
  to an instance variable}

If you want to set the value of an instance variable, you can do something
like this:
{\small
\begin{console}
#include <iostream>

class X
{
public:
    void set_i(int newi)
    {
        i = newi;
    }

    int i; // make this public for this experiment
};

int main()
{
    X x;
    x.set_i(5);
    std::cout << x.i << '\n'; // you should get 5
    return 0;
}
\end{console}
}
Another way to achieve the same thing is to give the client using your
class (in this
case client = main) access to the instance variable directly. You do that by
returning a reference to the instance variable:
{\small
\begin{console}[commandchars=\~\!\@]
#include <iostream>

class X
{
public:
    void set_i(int newi)
    {
        i = newi;
    }
    ~textbox!int &@ iref() // obviously this method cannot be const
    {
        return i;
    }

    int i; // make this public for this experiment
};

int main()
{
    X x;
    x.set_i(5);
    std::cout << x.i << '\n'; // you should get 5
    x.iref() = 6;             // lefthand side of = is a reference to x.i
    std::cout << x.i << '\n'; // you should get 6
    return 0;
}
\end{console}
}

Make sure you see the difference between \verb!X::set_i(int)! and
\verb!X::iref()!.
The former sets the value of \verb!i! for you.
The client has an indirect write access to \verb!i!.
For \verb!X::iref()!, the client has direct access to \verb!i!.

If an instance variable is an array say \verb!a!
and you want to give a client access \verb!a[0]!,
then you give the client a public method that returns a reference to
\verb!a[0]!, not the value of \verb!a[0]!.

Make sure you review notes on references and pointers.

The above is enough information for this assignment.
Read on if you want  to know a bit more ...

If you have the following in \cpp\ (actually for most languages):
\[
	\texttt{x = a + b + c;}
\]

Notice that there's a difference between the way you use the names
on the right (i.e., \verb!a!, \verb!b!, and \verb!c!)
and the name on the left (i.e., \verb!x!). Of course \verb!a!,
\verb!b!, \verb!c!, \verb!x! all refer to a \lq\lq boxes''
containing some value (more accurately, they refer to some part of your
computer's memorry).
However note that for \verb!a!, \verb!b!, \verb!c!,
you RETRIEVE (read access) the values at \verb!a!, \verb!b!, and \verb!c!.
For \verb!x!, you PUT A VALUE INTO (write access) the box corresponding to
\verb!x!. What actually happens is this: you read the values for
\verb!a!, \verb!b!, \verb!c! and say
\verb!a!,\verb!b!,\verb!c! have value \verb!1!,\verb!2!,\verb!3!,
then the expression
\verb!a+b+c! generates \verb!6!.
Now, \verb!a!, \verb!b!, \verb!c!
themselves have values which are of course at some fixed memory location.
However the value \verb!6! does not have a fixed location - it's temporary.
So we say that \verb!6! is an \defterm{rvalue}.
On the other hand the \verb!x! on the left of the assignment,
since it refers to the box of \verb!x!,
or more precisely to a chunk of memory, is an \defterm{lvalue}.

Informally, you can think of lvalue as \lq\lq something that can be
assigned a value and appears on the left of an assignment operator''
while an rvalue is \lq\lq something that cannot be assigned, is a value,
and appears on the right of the assignment operator''.
That's enough for right now. The real picture is slightly more complicated.

Knowing the above is important because some compilers will use the above terminology
(lvalue and rvalue) for error messages. For instance try to compile this program and
then read the error message:
{\small
\begin{console}
int main()
{
    2 = 5;
    return 0;
}
\end{console}
}
If you're using \gpp, it will give you an error message that basically says that the lefthand side of the assignment generates the value 2 which is not an lvalue: it cannot be assigned a value. 

\end{document}
