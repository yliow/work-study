
Follow the usual way of naming your folders.
For question Q1, create a folder \verb!a05q01! and put all 
\verb!*.cpp! and \verb!*.h! files into this folder.
Etc.
All folders should be kept in folder \verb!a05! and \verb!a05! should be
in \verb!a! in \verb!ciss245!.

Although there are many questions,
this is actually not a difficult assignment since this assignment is
mainly a practice on working with \texttt{struct}.

\newpage

We will write a simple simulation of several robots wandering about in a
finite 2-dimensional world.

Each robot has the following:
\begin{tightlist}
\item name
\item energylevel
\item x--coordinate
\item y--coordinate
\end{tightlist}
Here's our world with 3 robots
\begin{console}
time: 50
   0123456789
  XXXXXXXXXXXX
0 X          X
1 X  c    G  X
2 X     P    X
3 X P        X
4 X   XXXX   X
5 X   r      X
6 X   X      X
7 X   X   P  X
8 X d X      X
9 X   X      X
  XXXXXXXXXXXX
<Robot: name=c: x=2, y=1, energylevel=90>
<Robot: name=d: x=3, y=5, energylevel=45>
<Robot: name=r: x=1, y=8, energylevel=75>
<PowerStation: x=1, y=3, energylevel=50>
<PowerStation: x=5, y=2, energylevel=50>
<PowerStation: x=7, y=7, energylevel=50>
<Gold: x=7, y=1>
\end{console}
The world is 10-by-10.
Robots can only walk in the world.
There are two walls of length 4 in the world (look for the
\verb!X! in the above output).

When you run the program,
the two walls are erected randomly in the world (or \lq\lq room'' if you
wish to think of it that way.)
The walls are built randomly; they are either vertical or 
horizontal and of length 4.

Furthermore, there are power stations \verb!P! in the world.
There are 3 power stations and their positions are randomly selected.
Their positions are chosen on available spaces, i.e., they
are not in the walls and are not on top of each other and are not
on the same position as the pot of gold (see below).
Initially the power stations have a power level of 50.
Power stations do not move; they are stationary.

There is also a pot of gold, \verb!G!, placed randomly in the world.
Of course the pot of gold cannot be placed in the wall!
Also, the pot of gold cannot occupy the the same position as a power station.
\verb!G! is also stationary.

The robots are initially randomly placed in the world all starting
with energy level of 100.

When the simulation starts, the time is set to 100.
For each iteration of the simulation, the time decrements by 1.
The robots walk randomly and with each step, their energy level decrements by
1.
Of course the robot cannot walk into a wall nor can two robots stand
on the same position.
Likewise robots cannot stand on the same position as the \verb!P!'s or
\verb!G!.
Here's the first few iterations of a simulation:
{\small
\begin{console}[commandchars=\\\{\}]
\underline{42}
time: 100
   0123456789
  XXXXXXXXXXXX
0 X          X
1 X  c    G  X
2 X     P    X
3 X P        X
4 X   XXXX   X
5 X   r      X
6 X   X      X
7 X   X   P  X
8 X d X      X
9 X   X      X
  XXXXXXXXXXXX
<Robot: name:c, x=2, y=1, energylevel=100>
<Robot: name:d, x=1, y=8, energylevel=100>
<Robot: name:r: x=3, y=5, energylevel=100>
<PowerStation: x=1, y=3, energylevel=50>
<PowerStation: x=5, y=2, energylevel=50>
<PowerStation: x=7, y=7, energylevel=50>
<Gold: x=7, y=1>

time: 99
   0123456789
  XXXXXXXXXXXX
0 X          X
1 X   c   G  X
2 X     P    X
3 X P        X
4 X   XXXX   X
5 X  r       X
6 X   X      X
7 X d X   P  X
8 X   X      X
9 X   X      X
  XXXXXXXXXXXX
<Robot: name=c, x=3, y=1, energylevel=99>
<Robot: name=d, x=2, y=5, energylevel=99>
<Robot: name=r, x=1, y=7, energylevel=99>
<PowerStation: x=1, y=3, energylevel=50>
<PowerStation: x=5, y=2, energylevel=50>
<PowerStation: x=7, y=7, energylevel=50>
<Gold: x=7, y=1>
\end{console}
}
(The user entered \verb!42! as the seed for the random generator.
In the following, I will not display the user entry of the seed.)

When a robot moves to a position that is just next to a \verb!P!,
up to 10 points of energy is transferred from the power station to the robot.
Of course if the power station has a power level of 0, 
no power is transferred to the robot.
When I say \lq\lq next to'' I mean the robot is one square to the left
or right or top or bottom of \verb!P!.
Note that the energy is transferred once the robot moves to a square just
next to the power station.
Here's an example where Robot \verb!c! is just next to a power station
and will immediately get up to 10 points of energy:
{\small
\begin{console}
   0123456789
  XXXXXXXXXXXX
0 X          X
1 X       G  X
2 X    cP    X
3 X P        X
4 X   XXXX   X
5 X   r      X
6 X   X      X
7 X   X   P  X
8 X d X      X
9 X   X      X
  XXXXXXXXXXXX
\end{console}
}
However in the following, \verb!c! is not considered next to a power station:
{\small
\begin{console}
   0123456789
  XXXXXXXXXXXX
0 X          X
1 X    c  G  X
2 X     P    X
3 X P        X
4 X   XXXX   X
5 X   r      X
6 X   X      X
7 X   X   P  X
8 X d X      X
9 X   X      X
  XXXXXXXXXXXX
\end{console}
}
Note that in each iteration of the simulation, two robots
arrive at the power station at the same time.
For instance at time 25 we can have:
{\small
\begin{console}
   0123456789
  XXXXXXXXXXXX
0 X          X
1 X    c  G  X
2 X     P    X
3 X P    r   X
4 X   XXXX   X
5 X          X
6 X   X      X
7 X   X   P  X
8 X d X      X
9 X   X      X
  XXXXXXXXXXXX
\end{console}
}
and at time 24 we have this:
{\small
\begin{console}
   0123456789
  XXXXXXXXXXXX
0 X          X
1 X       G  X
2 X    cP    X
3 X P   r    X
4 X   XXXX   X
5 X          X
6 X   X      X
7 X   X   P  X
8 X d X      X
9 X   X      X
  XXXXXXXXXXXX
\end{console}
}
Both \verb!c! and \verb!r! are next to the same power station.
In this case \verb!c! and \verb!r! will get the same amount of energy points.
For instance if the power station have 9 energy points,
then they each get 4 (and then move away in the next iteration)
leaving the power station with 1 energy point.

When a robot's energy level goes to 0, the robot stops moving.

The simulation stops when the time reaches 0 or when one of the robots
found the gold, i.e., is next to \verb!G!.

(Note that since randomization is involved, you will 
need to use the \verb!rand()!
function.)

\newpage
\textsc{Skeleton code for Robot}

Th skelton code is provided only as a guide.
It's of course incomplete and might contain errors.

You will provide useful functionalities for
\verb!Robot! structure variables.

Here's the header file for \verb!Robot.h!:
{\small
\begin{Verbatim}[frame=single]
/*********************************************************************

File  : Robot.h
Author: smaug

void init(Robot &, char name, int x, int y, int energylevel);
Sets the name, x, y, and energylevel to the arguments passed into the 
function.

void print(const Robot & robot);
Prints the instance variables of the robot. If the name, x, y, 
energylevel of robot is 'c', 10, 20, 30, then the output is
<Robot: name=c, x=10, y=20, energylevel=30>
A newline is printed at the end.

void move_north(Robot & robot);
Decrement the y value of robot and decrements the energylevel by 1.

void move_south(Robot & robot);
Increments the y value of robot and decrements the energylevel by 1.

void move_east(Robot & robot);
Increments the x value of robot and decrements the energylevel by 1.

void move_west(Robot & robot);
Decrements the x value of robot and decrements the energylevel by 1.

void inc_energylevel(Robot & robot, int d);
Increments the energylevel of robot by d.

void dec_energylevel(Robot & robot, int d);
Decrements the energylevel of robot by d.

*********************************************************************/

#ifndef ROBOT_H
#define ROBOT_H

struct Robot
{
    char name;
    int x, y;
    int energylevel;
};

void init(Robot &, char name, int x, int y, int energylevel);
void print(const Robot & robot);
void move_north(Robot & robot);
void move_south(Robot & robot);
void move_east(Robot & robot);
void move_west(Robot & robot);
void inc_energylevel(Robot & robot, int d);
void dec_energylevel(Robot & robot, int d);

#endif

\end{Verbatim}
\begin{Verbatim}[frame=single]
/*********************************************************************

File  : Robot.cpp
Author: smaug

Implementation of functions prototypes in Robot.h.

*********************************************************************/
#include <iostream>
#include "Robot.h"

void init(Robot & r, char name, int x, int y, int energylevel)
{
    r.name = name;
    r.x = x;
    r.y = y;
    r.energylevel = energylevel;
}

\end{Verbatim}
}

