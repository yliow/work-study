The goal of this question is to implement the
\verb!init! and \verb!print!
of \verb!world.h!.
The implementation is of course placed in
\verb!world.cpp!.

The folder for this question
includes the files from the previous question.

The algorithm for \verb!init! is at the end of this question.
You must follow it.

{\small
\begin{Verbatim}[frame=single]
#include <iostream>
#include <cstdlib>
#include "Robot.h"
#include "world.h"

// Test functions from previous questions

void test_init_and_print_world()
{
    char world[10][10];
    init(world);
    print(world);
}


int main()
{
    int seed;
    std::cin >> seed;
    srand(seed);
    
    int option = 0;
    std::cin >> option;
    switch (option)
    {
        // code from Q1 - Q7.

        case 8:
            test_init_and_print_world();
            break;
    }
      
    return 0;
}
\end{Verbatim}
}

Test 1.
{\small
\begin{console}[commandchars=\\\{\}]
\underline{8}
   0123456789
  XXXXXXXXXXXX
0 X          X
1 X  XXXX    X
2 X          X
3 X          X
4 X      X   X
5 X      X   X
6 X      X   X
7 X      X   X
8 X          X
9 X          X
  XXXXXXXXXXXX
\end{console}
}
(The actual placement of the walls depend on the seed value of
\verb!srand!)

Test 2.
{\small
\begin{console}[commandchars=\\\{\}]
\underline{9}
  0123456789
  XXXXXXXXXXXX
0 X          X
1 X          X
2 X  X       X
3 X  X       X
4 X  XXXXX   X
5 X  X       X
6 X          X
7 X          X
8 X          X
9 X          X
  XXXXXXXXXXXX
\end{console}
}

Note that the walls do not overlap.
For instance the following is \textit{WRONG}:
\begin{Verbatim}[frame=single]
   0123456789
  XXXXXXXXXXXX
0 X          X
1 X  XXXX    X
2 X     X    X
3 X     X    X
4 X     X    X
5 X          X
6 X          X
7 X          X
8 X          X
9 X          X
  XXXXXXXXXXXX
\end{Verbatim}
In other words, within the world, there \textit{must} be exactly 8
\verb!X!'s.



\textsc{Algorithm for setting up the two random walls in the world}


First randomly pick an integer $x$ from 0 to 9 and do the same
to get a random $y$.
This gives you a point in the world which will serve
as a starting point for the
first wall.
Now randomly pick a random integer 0, 1, 2, 3
where 0 means north, 1 means south, 2 means east, and 3 means west.
If you picked 0, then starting the above $(x,y)$, you build a wall
in the north direction so that it has length 4.
If this is impossible (for instance the $(x,y)$ is too close to the
left of the world and the direction is west.), you start all over.
Similar for the other 3 directions.
If this wall is in the world, you continue on to the second wall.

The process for erecting the second wall is the same,
except that if it's impossible, you get rid of the first wall altogether
and start all over.
Note that the second wall has length 4, must stay in the world,
and must not overlap the first wall.


