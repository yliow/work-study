The next thing that the chat bot does is to be helpful to the user:
\begin{console}[fontsize=\small,frame=single,commandchars=\\\{\}]
Hi, what is your name?
\userinput{My name is John.}
Hi John. How are you?
\userinput{I'm sad.}
I'm sorry to hear that. Why are you sad?
\end{console}
Again, the chat bot must recognize the following variations:
\begin{tightlist}
  \li \verb!I'm sad.!
  \li \verb!i'm sad.!
  \li \verb!I am sad.!
  \li \verb!i am sad.!
  \li \verb!Sad.!
  \li \verb!sad!
\end{tightlist}
as well as without period and with extraneous spaces.
Your chat bot must also recognize \verb!depressed! and \verb!miserable! (you can add
other synonyms if you like).
So here's another chat session:
\begin{console}[fontsize=\small,frame=single,commandchars=\\\{\}]
Hi, what is your name?
\userinput{John}
Hi John. How are you?
\userinput{I'm depressed.}
I'm sorry to hear that. Why are you depressed?
\end{console}

The chat bot must also respond to the following emotions:
\verb!happy!, \verb!glad!.
(You can add others if you like.)
Here's another chat session:
\begin{console}[fontsize=\small,frame=single,commandchars=\\\{\}]
Hi, what is your name?
\userinput{My name is John.}
Hi John. How are you?
\userinput{I'm happy.}
I'm glad to hear that. Why are you happy?
\end{console}

Your chat bot must also recognize words like \verb!very!, \verb!extremely!, and \verb!really!.
\begin{console}[fontsize=\small,frame=single,commandchars=\\\{\}]
Hi, what is your name?
\userinput{My name is John.}
Hi John. How are you?
\userinput{I'm very sad.}
I'm sorry to hear that. Why are you so sad?
\end{console}
(Note the new word \verb!so! when the user adds \verb!very!.)

Once the user respond, the program does this:
\begin{console}[fontsize=\small,frame=single,commandchars=\\\{\}]
Hi, what is your name?
\userinput{My name is John.}
Hi John. How are you?
\userinput{I'm very sad.}
I'm sorry to hear that. Why are you \textred{so} sad?
\userinput{My dog died.}
I'm sorry. Tell me about your dog.
\end{console}
Here's another execution:
\begin{console}[fontsize=\small,frame=single,commandchars=\\\{\}]
Hi, what is your name?
\userinput{My name is John.}
Hi John. How are you?
\userinput{I'm very sad.}
I'm sorry to hear that. Why are you \textred{so} sad?
\userinput{My goldfish died.}
I'm sorry. Tell me about your goldfish.
\end{console}

Your program should respond to a general statement like:
\begin{console}[fontsize=\small,frame=single,commandchars=\\\{\}]
Hi, what is your name?
\userinput{My name is John.}
Hi John. How are you?
\userinput{I'm very sad.}
I'm sorry to hear that. Why are you \textred{red} sad?
\userinput{My [pet/friend/relative/etc.] died.}
I'm sorry. Tell me about your [pet/friend/relative/etc.].
\end{console}

On the happy side, your program should do this:
You program should respond to a general statement like:
\begin{console}[fontsize=\small,frame=single,commandchars=\\\{\}]
Hi, what is your name?
\userinput{My name is John.}
Hi John. How are you?
\userinput{I'm very happy.}
I'm glad to hear that. Why are you \textred{so} happy?
\userinput{I got a raise.}
I'm glad. Tell me about your raise.
\end{console}
or
\begin{console}[fontsize=\small,frame=single,commandchars=\\\{\}]
Hi, what is your name?
\userinput{My name is John.}
Hi John. How are you?
\userinput{I'm very happy.}
I'm glad to hear that. Why are you \textred{so} happy?
\userinput{I bought a car.}
I'm glad. Tell me about your car.
\end{console}

You only need to handle
\verb!got!,
\verb!bought!, and 
\verb!received!.
