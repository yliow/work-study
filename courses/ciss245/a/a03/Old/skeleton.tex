%-*-latex-*-
\textsc{Skeleton}

You will begin with the following skeleton files:
\begin{Verbatim}[frame=single,fontsize=\small]
// File  : mystring.h
// Author: smaug

#ifndef MYSTRING_H
#define MYSTRING_H

int str_len(char []);
void str_cat(char [], char []);

#endif

\end{Verbatim}

\begin{Verbatim}[frame=single,fontsize=\small]
// File  : mystring.cpp
// Author: smaug

#include "mystring.h"


int str_len(char x[])
{
    int len = 0;
    while (x[len] != '\0')
    {
        ++len;
    }

    return len;
}


void str_cat(char x[], char y[])
{
    int xlen = str_len(x);
    int ylen = str_len(y);

    for (int i = 0; i <= ylen; ++i)
    {
        x[xlen + i] = y[i];
    }

    return;
}
\end{Verbatim}

\begin{Verbatim}[frame=single,fontsize=\small]
// File  : main.cpp
// Author: smaug

int main()
{
    return 0;
}  
\end{Verbatim}

You will continually add code to the above for
the first few questions
as you develop your string library.
(I.e, start with the above for Q1, after you're done,
copy the files to Q2, etc.)
The string library (\verb!mystring.h! and \verb!mystring.cpp!)
is then used for the last question.
