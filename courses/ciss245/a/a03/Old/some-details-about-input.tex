%-*-latex-*-
\textsc{Some details about input}

What happens during input?
You should think of your \verb!std::cin! as
getting from a data stream, like an array of characters.
When you type something on your keyboard, the characters
are placed in the stream at the end.
\verb!std::cin! works by extracting characters
from the front of the stream.
(It's more like a queue.)

If you execute
\verb!std::cin.getline(s, 1024)!,
\verb!std::cin! will keep extracting
characters until it sees a newline character.
Character extracted from the input stream is 
placed in \verb!s!.
A \verb!'\0'! is then added to \verb!s!.
Note that the newline character is extracted from the
input stream but not placed in \verb!s!.
Note also that the \verb!1024! in
\verb!std::cin.getline(s, 1024)!
means that
\verb!std::cin!
will only write
at most 1023 characters into \verb!s!, reserving
one spot for \verb!'\0'!.
If \verb!std::cin! reads more than 1023 characters,
an error will occur.

A couple of things can go wrong ...

First, if
\verb!std::cin! reads more than 1023 characters without reading
a newline character, you'll get an error.

Second,
\verb!std::cin!
might read beyond the input stream, i.e.,
it has reached the
EOF.

Third,
something catastrophic might occur such as some
hardware failure or something went wrong with the OS.

Fourth, instead of using \verb!getline!,
when you use
\verb!std::cin! to read an input
for an integer,
but
what is read cannot be converted to an integer.
For instance if you try to \verb!std::cin >> i! where
\verb!i! is an \verb!int!, but you enter \verb!abc!.

\verb!std::cin! maintains information on
\lq\lq eof'', \lq\lq fail'', \lq\lq bad'' status
using 3 bits
(the eofbit, failbit, and badbit.)
\lq\lq eof'' means you've reached the end of data.
\lq\lq bad'' is the extreme case such as hardware or OS failure.
\lq\lq fail'' is the other types mentioned above.
So for instance if your program attempts to read data for an integer and
the input is \verb!"abc"!,
then the fail bit is turn on, i.e., to 1.
Once you've processed this error and you're ready to
get the next input (if that's what you want to do),
you need to set the fail bit to 0.

You use
\verb!std::cin.eof()!,
\verb!std::cin.fail()!, and
\verb!std::cin.bad()!
to check if the eofbit, failbit, badbit is turned on.
To set all these flags to 0, you do
\verb!std::cin.clear()!,

There are times when your want your
\verb!std::cin! to ignore some characters.
For instance say you have input lines where
each line has 3 data (say separated by spaces).
If an error occurs, you might want your
\verb!std::cin! to throw away all input characters
until the newline.
In that case you do
\verb!std::cin.ignore(100, '\n')!
which means ignore at most 100 characters or until the newline character.

\begin{Verbatim}[frame=single,fontsize=\small]
#include <iostream> 
#include <limits> 

const int MAX_BUF = 1024; 

int main() 
{ 
    while (1) 
    { 
        char s[MAX_BUF]; 
        std::cin.getline(s, MAX_BUF);
        if (std::cin.eof()) break;         
        if (std::cin.fail() || std::cin.bad()) 
        {
            // CASE: Error in input
            // Do whatever can be done with s
            std::cin.clear(); 
            std::cin.ignore(std::numeric_limits<std::streamsize>::max(), '\n');
        }
        else
        {
            // CASE: No error in input
            // Do something with s
        }
    }

    return 0;
} 
  \end{Verbatim}

The above information on how inputs work applies to almost
all input devices.
Read the above and use the code from the previous section.
Do NOT memorize the above information or the sample code from the previous
section.
The information is useful to know, but there's NO point in memorizing it.
It's NOT computer science.
In the future if you do work with low level I/O devices, just vaguely
remember there are various forms of input failures.
And if you need the \cpp\ code for input just look for it here or online.
