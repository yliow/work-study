%-*-latex-*-
\textsc{Stream I/O}

The input processing done like this:
\begin{console}[fontsize=\small]
int x;
double y;
char z[1024];

std::cin >> x >> y >> z;
\end{console}
works fine.
But note that for C-string input, whitespaces ends the input.
For instance if you run this:
\begin{console}[fontsize=\small]
char z[1024];

std::cin >> z;
std::cout << z << std::endl;
\end{console}
and you attempt to enter two words as input into \verb!z! you will see this:
\begin{console}[fontsize=\small,commandchars=\\\{\}]
\userinput{hello world}
hello
\end{console}
In other words \verb!z! contains only \verb!"hello"!.
\verb!world! is actually still in the \lq\lq input stream" -- it's not
transferred to your program.
You can of course get the second word \verb!world! like this:
\begin{console}[fontsize=\small]
char z[1024];
char w[1024];

std::cin >> z >> w;
std::cout << z << std::endl;
std::cout << w << std::endl;
\end{console}
Then you will get this when you execute the program:
\begin{console}[fontsize=\small, commandchars=\\\{\}]
\userinput{hello world}
hello
world
\end{console}

What if you want to have the whole line of input into a C-string
variable where the string contains whitespaces?

Run the following program:
\begin{console}[fontsize=\small]
#include <iostream> 
#include <limits> 

const int MAX_BUF = 1024; 

int main() 
{ 
    char s[MAX_BUF]; 
    std::cin.getline(s, MAX_BUF); 
    std::cout << '[' << s << "]\n"; 

    return 0;
} 
\end{console}
and enter a string with spaces.

If you want to writing a program that continually gets a string from the user,
try the following program:
\begin{console}[fontsize=\small]
#include <iostream> 
#include <limits> 

const int MAX_BUF = 1024; 

int main() 
{ 
    while (1) 
    { 
        char s[MAX_BUF]; 
        std::cin.getline(s, MAX_BUF); 
        if (std::cin.eof()) break; 
        if (std::cin.fail() || std::cin.bad()) 
        { 
            std::cin.clear(); 
            std::cin.ignore(std::numeric_limits<std::streamsize>::max(), '\n');
        }
        else
        {
            std::cout << '[' << s << "]\n";
        }
    }

    return 0;
} 
\end{console}
Note in particular that your program must terminate when an EOF
(i.e., end-of-file) is reached:
\begin{console}[fontsize=\small]
if (std::cin.eof()) break;
\end{console}
On the MS windows console window, if you do Ctrl-Z and then
press the enter key, you will get an EOF condition.
On the Linux/Unix environment, you would do Ctrl-D.

To make your program robust and platform independent,
you have to know that the string read from a stream might
have non-visible characters at the end of the string.
For instance on some systems,
when a user press the enter key, the character
\verb!'\r'! is placed in the stream.
The system might also placed \verb!'\n'! in the string stream as well.
This means that a string read from the stream (up to \verb!'\n'!)
might end with \verb!'\r'! (and possibly also contain \verb!'\n'!).
You are strongly advised to remove all such right-trailing characters from
your strings which are read from a stream.
In other words after a string input process, say with the input
going into character array variable \verb!s!,
you should look at the characters just before \verb!'\0'!
and see if there is \verb!\r! or \verb!\n! or both.
For instance after an input, say \verb!s! is
\verb!"hello world\r\n\0..."!
you should make it
\verb!"hello world\0..."!


To analyze all the characters in your C-string \verb!s!, you can
of course print the ASCII code of the characters
in \verb!s! and have an ASCII table in front of you:
{\small
\begin{Verbatim}[frame=single]
#include <iostream> 
#include <limits> 
#include "mystring.h"

const int MAX_BUF = 1024; 

int main() 
{ 
    while (1) 
    { 
        char s[MAX_BUF]; 
        std::cin.getline(s, MAX_BUF); 
        if (std::cin.eof()) break; 
        if (std::cin.fail() || std::cin.bad()) 
        { 
            std::cin.clear(); 
            std::cin.ignore(std::numeric_limits<std::streamsize>::max(), '\n');
        }
        else
        {
            std::cout << '[' << s << "]\n";
            for (int i = 0; i < str_len(s); ++i)
            {
                std::cout << i << ' ' << s[i] << ' ' << int(s[i]) << std::endl;
            }
        }
    }
    return 0;
} 
\end{Verbatim}
}
