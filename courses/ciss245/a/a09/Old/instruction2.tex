
Here are the test option numbering.
\begin{tightlist}
    \item   Print and \verb!mystring()!
    \item   Print and \verb!mystring(char)!
    \item   Print and \verb!mystring(char [])!
    \item   Print and copy constructor
    \item   Input
    \item   \verb!operator==!
    \item   \verb!operator!=!
    \item   \verb!operator=!
    \item   \verb!empty!
    \item   \verb!char operator[](int) const!
    \item   \verb!char & operator[](int)!
    \item   \verb!mystring & operator+=(const mystring &)!
    \item   \verb!mystring & operator+=(char)!
    \item   \verb!mystring operator+(const mystring &)!
    \item   \verb!mystring operator+(char)!
    \item   \verb!void resize(int i)!
    \item   \verb!void resize(int i, char c)!
    \item   \verb!int find(char c, start = 0)!
    \item   \verb!int find(const mystring &)!
    \item   \verb!mystring substr(int i, int len)!
\end{tightlist}
There are no tests for nonmember functions other than input and output
which are tested in 1--5.





\newpage
\textsc{Reminder: Giving access to instance varaiable by returning a reference to an instance varaiable}

If you want to set the value of an instance variable, you can do something like this:
\begin{Verbatim}[frame=single]
#include <iostream>

class X
{
public:
    void set_i(int newi)
    {
        i = newi;
    }

    int i;  // make this public for this experiment
};

int main()
{
    X x;
    x.set_i(5);
    std::cout << x.i << '\n'; // you should get 5
    return 0;
}
\end{Verbatim}
Another way to achieve the same this is to give the client using your class (in this case client = main)
access to the instance variable directly. You do that by returning a reference to the instance variable.
\begin{Verbatim}[frame=single]
#include <iostream>

class X
{
public:
    void set_i(int newi)
    {
        i = newi;
    }
    int & iref()  // obviously this method cannot be constant
    {
        return i;
    }

    int i; // make this public for this experiment
};

int main()
{
    X x;
    x.set_i(5);
    std::cout << x.i << '\n';  // you should get 5
    x.iref() = 6;              // lefthand side of = is a reference to x.i
    std::cout << x.i << '\n';  // you should get 6
    return 0;
}
\end{Verbatim}
Make sure you see the difference between \verb!X::set_i(int)! and \verb!X::iref()!. The former sets the value of \verb!i! for
you. The client has an indirect write access to \verb!i!. For \verb!X::iref()!, the client has direct access to \verb!i!.

If an instance variable is an array, say \verb!a!, and you want to give a client access to \verb!a[0]!, then you give the
client a public method that returns a reference to \verb!a[0]!, not the value of \verb!a[0]!.

Make sure you review the notes on references and pointers.

The above is enough information for this assignment. Read on if want to know a bit more ...

If you have the following in \cpp\ (actually for most languages):
\begin{Verbatim}
            x = a + b + c;
\end{Verbatim}
Notice that there's a difference between the way you use the names on the right (i.e., \verb!a!, \verb!b!, and \verb!c!) and
the name on the left (i.e., \verb!x!). Of course \verb!a!, \verb!b!, \verb!c!, \verb!x! all refer to ``boxes'' containing some 
value (more accurately, they refer to some part of your computer's memory). Howerver note that for \verb!a!, \verb!b!, \verb!c!
, you RETRIEVE (read access) the values at \verb!a!, \verb!b!, and \verb!c!. For \verb!x!, you PUT A VALUE INTO (write access)
the box corresponding to \verb!x!. What actually happens is this: you read the values for \verb!a!, \verb!b!, \verb!c! and say
\verb!a!, \verb!b!, \verb!c! have values 1, 2, 3, then the exression \verb!a+b+c! generates 6. Now \verb!a!, \verb!b!, 
\verb!c! themselves have values which are of course at some fixed memory locations. However the value 6 does not have a fixed 
loaction - it's temporary, so we say that 6 is an \textbf{r-value}. On the other hand \verb!x! on the left of the assignment,
 since it refers to the box of \verb!x!, or more preciesely to a chunck of memory, is an \textbf{l-value}.

Informally, you can think of a \verb!r-value! as ``something that can be assigned a value and appear on the left
of an assignment operator'' while a \verb!l-value! is ``something that cannot be assigned, is a value, and
appears on the right of the assignment operator''. That's enough for right now. The real picture is slightly more complicated.

Knowing the above is important because some compilers will use the above terminology (\verb!l-value! and \verb!r-value!) for 
error messages. For instance try to compile this program and then read the error message:
\begin{Verbatim}[frame=single]
int main()
{
    2 = 5;
    return 0;
}
\end{Verbatim}
If you're using \gpp, it will give you an error message that basically says that the lefthand side of the assignment generates
the value 2 which is not a \verb!l-value!, i.e., it cannot be assigned a value.
\end{document}
