The goal is to implement a string class.
Recall from CISS240, we have the concept of C-strings.
There are lots of functions provided by the compiler to operate on C-strings.
(Either see CISS240 notes on C-strings or look at the C-string review section below.)
The goal is is to write a string class, called \verb!mystring!, so that \verb!mystring! objects
model C-strings and \verb!mystring! objects will be easier to work with than C-string.
For instance if you have two C-strings \verb!s! and \verb!t! and you want to concatenate them (i.e., join them) into \verb!u!,
you have to do
\begin{console}
u[0] = '\0';
strcat(u, s);
strcat(u, t);
\end{console}
Using our \verb!mystring! class, we can do this:
\begin{console}
u = s + t;
\end{console}

You are given the following (possibly incomplete files and possible incorrect files -- these are meant to help you get started):
\begin{tightlist}
    \li    The test codes is given. It must be used in your project.
    \li    The header file is included.
    \li    The implementation file contains the implementation of \verb!operator<<!.
\end{tightlist}

Make sure you read the whole document before you dive into the code.

\textsc{Important warning}: Again, the files are meant to be skeleton files and might not be complete and
might have deliberate missing details or even errors.
The goal of the skeleton files it to give you something to work with.

If you're doing a copy-and-paste of the given code,
note that some characters might be changed by
PDF to other characters. In particular the \verb!'-'!,
character might actually not be the dash character.
(There are two ASCII characters that look like \verb!-!.)
Looking
at the compiler error message will help you find these minor annoying issues so that you can correct
them.

All methods must be constant whenever possible.
All parameters which are objects (or struct
variables) must be pass by reference or pass by constant reference
as much as possible.
(I already mentioned these two points in my notes and in class.)

Let me know ASAP if you see a typo.

