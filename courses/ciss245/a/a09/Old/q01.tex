Strings

Review your notes on C-strings right away. The following is a quick review and
is not complete.

A C-string (also called a null-terminated string is a character array
containing the character \verb!'\0'!. The 
character \verb!'\0'! has an ASCII integer value of 0. Try this:

{\small
\begin{Verbatim}[frame=single, commandchars = \~\@\!]
char s[10] = {'a', 'b', 'c', '\0', 'd', 'e'};
std::cout << s[0] << ' ' << int(s[0]) << '\n'; // prints 'a' and ASCII
                                               // value of 'a'
std::cout << s[3] << ' ' << int(s[3]) << '\n'; // prints '\0' and the
                                               // ASCII value of '\0'
\end{Verbatim}
}
(Google \lq\lq ascii table".)

Note that \verb!s! is a character array of size 10.
The first 6 characters are initialized. The last 4 are therefore
initialized to the character with integer value (ASCII value) of 0, i.e.,
the remaining 4 are initialzed to
\verb!'\0'!. The character \verb!'\0'! is also called the null character
since the ASCII value is 0.

When we call \verb!s! a string, what we really meant is the \verb!"abc"!,
i.e., 4 characters with the last being character \verb!'\0'!.
The point of the \verb!'\0'! character is simply to
tell the computer when the string data (characters) end.
In other words, you can (and should) think \verb!'\0'! as
a sentinel value in the array.
The \verb!'\0'! is therefore used to indicate
\lq\lq when the string ends within the character array".
This is used for instance in printing \verb!s!:
{\small
\begin{Verbatim}[frame=single, commandchars = \~\@\!]
char s[10] = {'a', 'b', 'c', '\0', 'd', 'e'};
std::cout << s << std::endl;
\end{Verbatim}
}
You see that the printing of the characters stop once the
printing operation sees the character \verb!'\0'!, i.e.,
the \verb!'\0'! terminates the print operation. You can therefore tell
that the \verb!operator<<! probably looks
\lq\lq something like this":
{\small
\begin{Verbatim}[frame=single, commandchars = \~\@\$]
std::ostream & operator<<(std::ostream & cout, const char s[])
{
    int i = 0;
    while (s[i] != '\0')
    {
        cout << s[i];
        i++;
    }
    return cout;
}
\end{Verbatim}
}
As you can see in the code above: \verb!'\0'! is used to terminate
(i.e., as a sentinel value) for the print operation.

Recall that the double-quotes string notation
{\small
\begin{Verbatim}[frame=single, commandchars = \~\@\!]
char s[10] = "abc";
\end{Verbatim}
}
is really the same as (a shorthand for):
{\small
\begin{Verbatim}[frame=single, commandchars = \~\@\!]
char s[10] = {'a', 'b', 'c', '\0'};
\end{Verbatim}
}
Make sure you see the difference between \verb!s! as an array and s as a string:
\begin{tightlist}
 \li as an array \verb!s! has 10 characters - the array size of \verb!s! is 10
 \li as a string \verb!s! has 3 characters - the string length of \verb!s! is 3
\end{tightlist}

There are several C-string functions that come with all \cpp\ compilers.
Here are some (again refer to
my old notes on strings):
{\small
\begin{Verbatim}[frame=single, commandchars=\~\@\!]
#include <iostream>
#include <cstring> // include prototypes for C-string functions

int main()
{
    char s[1024] = "hello world";
    char t[50] = "ABCDEF";
    char u[100] = "ABCDEF\0GHIJKL";

    std::cout << strlen(s) << '\n'   // string length function:
                                     // length of the string, i.e.,
                                     // number of characters up to but
                                     // not including '\0'

    strcat(s, t);             // string concatenation function:
    std::cout << s << '\n';   // string concatenation, i.e., join the string
                              // t to s.
    int i = strcmp(s, t);
    std::cout << i << '\n';              // compare s and t. The return
    std::cout << strcmp(t, u) << '\n';   // value is 0 if the strings
                                         // compared are the same. Otherwise
                                         // it's nonzero.
    return 0;
}
\end{Verbatim}
}
Make sure you study and run the above code.

There are many other C-string functions. Google "c string functions" or
refer to my notes or refer to any
C/\cpp\ textbook.

Note that \verb!s! is a character array of size 1000, then at most you
can store a string of length 999 since
one character is used to indicate end-of-string.

There's another way to describe a string: that's with a character array and
a \lq\lq length" integer variable.
For instance if I have this:
{\small
\begin{Verbatim}[frame=single, commandchars= \~\@\!]
char s[1024] = "hello world ... how are you ... ?"
int len_s = 11;
\end{Verbatim}
}
Then the variable \verb!len_s! tells me that only 11 characters in the
character array are considered part of the
string that \verb!s! is trying to represent. Note: \verb!s! does have a
capacity to hold 1024. Only the first 11
characters are considered the string that \verb!s! is representing.

This style of representing strings is called a length-based representation.
(We've talked about this before
in CISS240.)

What are the benefits of using a length-based representation for strings?
Well think about it. Suppose
you want to concatenate two C-strings
{\small
\begin{Verbatim}[frame=single, commandchars = \~\@\!]
char s[1024] = "hello world ... how are you ?";
char t[1024] = " I'm fine.";
strcat(s, t);
\end{Verbatim}
}

Do you realize that the characters \verb!' '!, \verb!'I'!, \verb!'\''!,
\verb!'m'!, etc, from \verb!t! must be copied to the end of \verb!s!,
overwriting
the \verb!'\0'! of \verb!s!? That means that I have to know where is the
\verb!'\0'! in \verb!s!. That means that before I begin
concatenating \verb!t! to \verb!s!, I have to find \verb!'\0'!.
If \verb!s! is a very long string, that will waste some time.
In particular this will get slower and slower:
{\small
\begin{Verbatim}[frame=single, commandchars = \~\@\!]
char s[1024] = "hello world ... how are you?";
char t[1024] = " I'm fine.";
for (int i = 0; i < 100; ++i)
{
    strcat(s,t);
}
\end{Verbatim}
}
Of course computers are so fast nowadays that you won't notice the
difference if this is the only code
in your program. If your program concats millions of strings (which
is very common in a
buisness application or an AI program on natural language processing),
the effect will be noticeable.

For this question, we're going to build a string class that uses a
length-based approach. The class will
hold a character array of size 1024 and it will also hold a length variable.
There's another variable that
tell us that the maximum capacity is 1024 -- this will be a constant since
(for this assignment
question), the size of the character array does not change.
(Later you will work on the concept of a dynamic array.)

Strings are extremely important. Look at this document.
Clearly it's made up of strings. Strings (or more
accurate, languages) are used to communicate ideas and thoughts and emotions.
Since human ideas are expressed
strings, the ability to compute with strings is important when it comes to
analyzing the thoughts of a
human. NLP (natural language processing) is an area of study in computer
science with the goal of
understanding the structure and meaning behind strings (written or oral)
prodcued by human beings.

If you want to try to play with a fake psycotherapist, go to this website and
talk to Eliza and explain to
\lq\lq her" (it?) your problems:
\underline{http://www.manifestation.com/neruotoys/eliza.php3}.
It's not particularly smart
but it does show you that Eliza can at least understand basic English
structure and therefore can
pretend to be intelligent to some extent.

As an example of an object of our class, suppose I execute this:
{\small
\begin{Verbatim}
       mystring y("abc");
\end{Verbatim}
}
then \verb!y! is a \verb!mystring! object with three instance variables:
{\small
\begin{Verbatim}
       y.s_         a character array of size 1024
       y.length_    an int variable
       y.capacity_  an int constant (set to 1024)
\end{Verbatim}
}
The values placed in \verb!y! by this constructor are:
{\small
\begin{Verbatim}
       y.s_         array containing 'a', 'b', 'c', ...
       y.length_    3
       y.capacity_  1024
\end{Verbatim}
}
Since \verb!y.length_! is 3 in this case, only the first 3 characters of
\verb!y.s_! are relevant. The constructor in
this case is
{\small
\begin{Verbatim}
       mystring::mystring(const char x[]);
\end{Verbatim}
}
since this one accepts a character array.

The \verb!capacity_! instance variable can be set in the initializer of
the constructor since we know that it's
just 1024. A C-string is passed into the constructor as an argument.
Note that in this case we know that
\verb!y.length_! is 3, but the constructor needs to scan the character array
\verb!'a'!, \verb!'b'!, \verb!'c'!, \verb!'\0'!, ... to
know that there are three characters before the first \verb!'\0'!.
Therefore \verb!y.length_! cannot be directly
initialized. So \verb!y.length_! is set in the body of the constructor.
The character array \verb!y.s_! is also set in
the body of the constructor. Therefore the constructor that accepts a
character array looks like this (in
pseudocode):
{\small
\begin{Verbatim}[commandchars = ~\@\!]
       mystring::mystring(const char x[]);
          Initialize capacity_ to 1024
          Loop over x[i] (i = 0, 1, 2, 3, 4, 5, ...) and put x[i] into s_[i]; stop
          when x[i] is '\0':
\end{Verbatim}
}
\newcommand\ddd{$\downarrow$}
{\small
\begin{Verbatim}[commandchars = \\\{\}]
                  x[0]  x[1]  x[2] ...
                    \ddd{}     \ddd{}     \ddd{}
                  s_[0] s_[1] s_[2] ...
\end{Verbatim}
}
{\small
\begin{Verbatim}[commandchars = ~\@\!]
          Also, set length_ to the string length of x_, i.e., length_ is set to the
          number of characters in x up to but not including the first '\0'.
\end{Verbatim}
}
Note that in the class definition of \verb!mystring! in the header file,
there are three constructors: one
that accepts a character array, one that is the default constructor,
and there's the copy constructor. Note that in
the test code, there's another that accepts a character.
This is also mentioned in the documentation.
Therefore you will need to write a constructor that accepts a character.

Here are more examples. If I do:
{\small
\begin{Verbatim}
      mystring a("hello");  // using constructor that accepts a character array
\end{Verbatim}
}
then
{\small
\begin{Verbatim}
      a.s_         has array values 'h','e','l','l','o', ...
                       (we don't care about the other values)
      a.length_    has value 5
      a.capacity_  has value 1024
\end{Verbatim}
}
If I do
{\small
\begin{Verbatim}
      mystring b;  //using the default constructor
\end{Verbatim}
}
then
{\small
\begin{Verbatim}
      b.s_         we don't care about the values
      b.length_    has value 0
      b.capacity_  has value 1024
\end{Verbatim}
}
If I do
{\small
\begin{Verbatim}
      mystring c('$');   // using the constructor that accepts a character
\end{Verbatim}
}
then
{\small
\begin{Verbatim}
      c.s_         has values '$', ... (we don't care about the rest)
      c.length_    has value 1
      c.capacity_  has value 1024
\end{Verbatim}
}

The copy constructor does the obvious. For instance if object \verb!a!
is the above, then if I do this:
{\small
\begin{Verbatim}
      mystring d(a);
\end{Verbatim}
}
then
{\small
\begin{Verbatim}
      d.s_         has array values 'h','e','l','l','o', ... 
                      (we don't care about the other values)
      d.length_    has value 5
      d.capacity_  1024
\end{Verbatim}
}
