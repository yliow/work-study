The objective is to write a class, \verb!IntDynArr!, that 
models a dynamic array of integers. In particular, the class to 
be implemented uses system resources (i.e. memory) and hence several
default operators (example: the destructor, copy constructor and
\verb!operator=!) 
must be overwritten. 

You will need to write three files:
\begin{tightlist}
\li \verb!testIntDynArr.cpp!: This contains a program to test the \verb!IntDynArr! class 
and it's supporting functions and operators.
\li \verb!IntDynArr.h!: This is the header function containing the \verb!IntDynArr! class.
\li \verb!IntDynArr.cpp!: This file contains the definitions for the methods in 
\verb!IntDynArr.h!.
\end{tightlist}

As before all methods/functions implemented should be tested carefully in the 
test code. 

Read the following carefully before you dive straight into coding. As before, 
you should minimize code duplication, provide reasonable access control 
(i.e., \verb!const! a reference parameter whenever possible), code with proper 
indentation, etc. 

Such a class is extremely important because in the real world, we want to 
work with arrays with different sizes during runtime. In fact it's so important 
that C++ actually has a library call the Standard Template Library (STL) that 
includes a class similar to the class in this assignment. 

With this class written, you can do the following easily:
\begin{Verbatim}[frame=single]
// Work with x which simulates an array of 5 integers
IntDynArr x(5);
x.resize(5);
for (int i = 0; i < 5; i++)
{
    x[i] = 2 * i + 1;
}
std::cout << x << std::endl; // prints [1, 3, 5, 7, 9]

x.insert(0, -100); // put -100 at index 0
std::cout << x << std::endl; // prints [-100, 1, 3, 5, 7, 9]
// Note that now x simulates an array of size 6, not 5.
// Therefore your array class actually allows x to change its
// size auto-magically!

x.remove(3); // remove value at index 3
std::cout << x << std::endl; // prints [-100, 1, 3, 7, 9]
// Note that now x simulates an array of size 5.

x.remove(3); // remove value at index 3
std::cout << x << std::endl; // prints [-100, 1, 3, 9]
// Note that now x simulates an array of size 4.
\end{Verbatim}
Other methods and operators are described more fully in 
the next section. 

Once this class is completed, you can easily modify it for 
different values such as a dynamic array of strings, of 
doubles, of alien spaceships, etc.

