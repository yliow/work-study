%-*-latex-*-
\input{myassignmentpreamble}
\input{ciss245}
\input{yliow}

\renewcommand\TITLE{Assignment 10}

\begin{document}
\topmatter

\textsc{Objectives}
    \begin{enumerate}[nosep]
    \item Declare objects
    \item Access member variables of an object
    \item Write member functions/operators
    \item Work with classes containing pointer members
    \item Overwrite default methods (constructors, destructors)
    \item Overload operators
    \end{enumerate}

\newpage

Q1. The objective is to write a class, \verb!IntDynArr!, that 
models a dynamic array of integers. In particular, the class to 
be implemented uses system resources (i.e. memory) and hence several
default operators (example: the destructor, copy constructor and
\verb!operator=!) 
must be overwritten. 

You will need to write three files:
\begin{tightlist}
\li \verb!testIntDynArr.cpp!: This contains a program to test the \verb!IntDynArr! class 
and it's supporting functions and operators.
\li \verb!IntDynArr.h!: This is the header function containing the \verb!IntDynArr! class.
\li \verb!IntDynArr.cpp!: This file contains the definitions for the methods in 
\verb!IntDynArr.h!.
\end{tightlist}

As before all methods/functions implemented should be tested carefully in the 
test code. 

Read the following carefully before you dive straight into coding. As before, 
you should minimize code duplication, provide reasonable access control 
(i.e., \verb!const! a reference parameter whenever possible), code with proper 
indentation, etc. 

Such a class is extremely important because in the real world, we want to 
work with arrays with different sizes during runtime. In fact it's so important 
that C++ actually has a library call the Standard Template Library (STL) that 
includes a class similar to the class in this assignment. 

With this class written, you can do the following easily:
\begin{Verbatim}[frame=single]
// Work with x which simulates an array of 5 integers
IntDynArr x(5);
x.resize(5);
for (int i = 0; i < 5; i++)
{
    x[i] = 2 * i + 1;
}
std::cout << x << std::endl; // prints [1, 3, 5, 7, 9]

x.insert(0, -100); // put -100 at index 0
std::cout << x << std::endl; // prints [-100, 1, 3, 5, 7, 9]
// Note that now x simulates an array of size 6, not 5.
// Therefore your array class actually allows x to change its
// size auto-magically!

x.remove(3); // remove value at index 3
std::cout << x << std::endl; // prints [-100, 1, 3, 7, 9]
// Note that now x simulates an array of size 5.

x.remove(3); // remove value at index 3
std::cout << x << std::endl; // prints [-100, 1, 3, 9]
// Note that now x simulates an array of size 4.
\end{Verbatim}
Other methods and operators are described more fully in 
the next section. 

Once this class is completed, you can easily modify it for 
different values such as a dynamic array of strings, of 
doubles, of alien spaceships, etc.


\newpage
\textsc{Requirements}

You know that C/C++ arrays have constant sizes if the 
arrays are in your local blocks scope. One way to 
\lq\lq make" an array dynamic is to use pointers to point 
to arrays in the heap. Hence it makes sense to create a 
class that models a \lq\lq dynamic" array and hide all 
the complexities in this class. 

The private members of \verb!IntDynArr! includes 
\begin{tightlist}
\li \verb!size! (an \verb!int! -- you can name it \verb!size_! if you wish), 
\li \verb!capacity! (an \verb!int! -- you can name it \verb!capacity_! if you wish), and
\li \verb!x! (an \verb!int *! -- you can name it \verb!x_! if you wish)
\end{tightlist}

What is the purpose of the \verb!capacity! instance variable? 
(I've talked about this in CISS240. Also, refer to the notes 
in CISS245.)

Suppose you construct an \verb!IntDynArr! object that models 
an array with \verb!size! 100. At a later point in time, you might 
want the object to model an array of \verb!size! 90 instead. You 
can for instance allocate new memory of 90 integers for this 
object to refer, copy the contents of the old array to the 
new, and deallocate the original 100 integers. But the 
problem is, suppose you later want to model an array of 100 
integers again. You have to do the memory deallocation and 
allocation again. A better scheme in this case is to allocate 
memory for 100 integers and set the \verb!size! to 100. When you need 
the array size to be 90, just set the \verb!size! to 90; you still have 
100 integers available, it's just that 10 of them should not be 
accessed. When you need 100 integers again, just set the \verb!size! to 
100. See notes below for details on \verb!size! and \verb!capacity!.

Recall that since each \verb!IntDynArr! object has a pointer member that 
points to an array of integers (i.e. it uses system resource that is not 
automatically released), you need to overwrite the 
\begin{tightlist}
\li Default destructor \verb!~IntDynArr()! to release the memory x is 
pointing to
\li Default copy constructor \verb!IntDynArr(const IntDynArr &)!
\li Default \verb!IntDynArr & operator=(const IntDynArr &)!
\end{tightlist}
The reason for overwriting the above default methods are already 
mentioned in class.

Here are the methods/features of the class:
\begin{enumerate}
\item \verb!IntDynArr(int capacity0 = 5)!:
  Constructs an object so that 
  \verb!size! is set to \verb!0!,
  \verb!capacity! is set to \verb!capacity0!,
  and \verb!x! points to an 
  integer array in the heap with \verb!capacity0! integers.

\item \verb!IntDynArr(int size0, int a[])!:
  \textred{Constructs an object so that 
  \texttt{size}
  and \texttt{capacity}
  are both set to \texttt{size0},
  and \texttt{x} points to an 
  integer array of size \texttt{size0}.
  The values \texttt{a[0]}, ..., \texttt{a[size0 - 1]} are copied to the array \texttt{x}
  points to.}
  
\item \verb!IntDynArr(const IntDynArr &)!: This is the obvious copy 
constructor. 

\item \verb!~IntDynArr()!: The destructor must release all resources held by 
an object.

\item \verb!int get_size() const!: returns the value of the \verb!size!.
\item \verb!int get_capacity() const!: returns the value of the 
\verb!capacity!.

\item \verb!bool operator==(const IntDynArr & arr) const!: Returns \verb!true! if 
the array object \verb!*this! and \verb!arr! are the same in the sense that 
their sizes and \lq\lq contents" are the same. Here, \lq\lq contents" mean the 
values in the array from index 0 to \verb!size - 1!. 

\item \verb@bool operator!=(const IntDynArr & arr) const@: The opposite of 
\verb!operator==!.

\item \verb!IntDynArr & operator=(const IntDynArr & arr)!: After assignment, 
\verb!*this==arr! would return \verb!true!. See notes below.

\item \verb!std::ostream & operator<<(std::ostream &, const IntDynArr & a)!: 
Prints the contents of the array \verb!a.x! is pointing to. Integers are 
separated by spaces. For instance if \verb!a.x! points to the array 
4, 5, 8, 9,2 and \verb!a.size! is 3, then \lq\lq [4, 5, 8]” is printed 
(without the quotes of course). Note that this is a function outside the 
class, i.e., it's a nonmember function.

\item \verb!std::istream & operator>>(std::istream &, IntDynArr & a)!: 
Read in a list of integers from the keyboard until you see a -1 and 
fill in the dynamic array with the integers you just read. You must clear 
the existing values in the array and adjust its size and capacity as 
appropriate. As you should know by now, this is also a nonmember function. 

\item \verb!void resize(const int size0)!: This method \lq\lq resizes" the 
array \verb!x! is pointing to. This is what it does: Suppose \verb!a! is an 
\verb!IntDynArr! object that models an array of \verb!size! 10. After 
calling \verb!a.resize(15)!, the \verb!a! models an array of size 15. The 
contents of the original array of 10 integers are copied over to the 
new array of size 15 starting at index 0. This means that 5 elements of 
the new array can contain arbitrary values. Note that there should not be 
any memory leaks after the call. For the above example, the array of 10 
integers \verb!a.x! was pointing to must be released back to the heap. See notes 
below for details.

\item \verb!int operator[](int i) const! and \verb!int & operator[](int i)!:
Note that there are two operators here. They returns \verb!x[i]! either as an 
integer value or an integer reference. The first allows us to do this:

\begin{Verbatim}[frame=single]
IntDynArr arr(5);
std::cout << arr[0] << std::endl;  // arr[0] is arr.operator[](0)
\end{Verbatim}

Now you might wonder why are we need a version \verb!operator[]! that 
returns a reference (\verb!int &!) instead of just an integer. If 
\verb!operator[]! returns an integer, then the return value is an 
integer value and in that case

\begin{Verbatim}[frame=single]
std::cout << arr[0] << std::endl;  // arr[0] is arr.operator[](0)
\end{Verbatim}

works. But

\begin{Verbatim}[frame=single]
arr[0] = 3;
\end{Verbatim}

does not work. If \verb!arr.x[0]! is 5 and we returned an \verb!int! 
(instead of an integer reference), then \lq\lq \verb!arr[0] = 3!" would 
be the same as saying \lq\lq \verb!5 = 3;!" which is of course nonsense. 
However if \verb!operator[]! returns a reference, in that case what is 
returned in not an integer but a reference to \verb!arr.x[0]!, an integer 
“variable”. (Refer to your lecture notes on references.) Therefore for the 
statement

\begin{Verbatim}[frame=single]
arr[0] = 3;
\end{Verbatim}

the second \verb!operator[]! (the one that returns a reference) is called. 
This was already mentioned in class several times when I talked about 
returning references.

\item \verb!IntDynArr & operator+=(const IntDynArr & a)!: This operator will 
\lq\lq append" (or concatenate) the contents of \verb!a! to the object invoking this 
operator. For instance if object \verb!x! models the array 1, 2, 3, and object 
\verb!a! models the array 7, 8, 9, then \verb!x += a! will modify \verb!x! so that it 
models the array 1, 2, 3, 7, 8, 9.

\item \verb!IntDynArr operator+(const IntDynArr & a) const!: This operator will 
return a copy of the object which is the concatenation of the contents of the 
object invoking the call with the contents of \verb!a!. For instance of object 
\verb!x! models the array 1, 2, 3, and object \verb!a! models the array 7, 8, 9, then 
\verb!x + a! will return an object that models the array 1, 2, 3, 7, 8, 9.

\item \verb!IntDynArr & insert(int index, int val)!: This will modify the 
object invoking this method by inserting value val at index \verb!index!. 
For instance if the object \verb!x! models the array 1, 2, 3, after the 
method call \texttt{x.insert(\textred{1, 9})}, \verb!x! will model 
the array 1, 9, 2, 3.

\item \verb!IntDynArr & remove(int index)!: This will modify the object 
invoking this method by removing the value at index \verb!index!. For instance 
if the object \verb!x! models the array 1, 2, 3, after the method call 
\verb!x.remove(1)!, \verb!x! will model the array 1, 3.

\item \verb!IntDynArr subarray(int index, int length = -1) const!: This 
will return an \\ \verb!IntDynArr! object which is \lq\lq part of” the object 
invoking the method. For instance if the object \verb!x! models the 
array 90, 91, 92, 93, 94, 95, 96, then \verb!x.subarray(3, 2)! will 
return an \verb!IntDynArr! object that models the array 93, 94. If the 
length argument is not specified, then all the values from the specified 
index to the end of object are used in creating the new object. For 
instance, with the above object \verb!x!, calling  \verb!x.subarray(3)! 
will return an \verb!IntDynArr! object that models the array 93, 94, 95, 9\textred{6}.

\item \verb!void print() const!: This is the 
\texttt{std::ostream \& operator<<(std::ostream \&, IntDynArr \& a)} except that 
it prints the \verb!size! and \verb!capacity! as well. It also prints a newline. 
This can be used for debugging purposes. For instance if \verb!a.x! points to 
the array 4, 5, 8, 9, 2 and \verb!a.size! is 3, then
\lq\lq \texttt{[4,\ 5,\ 8],\ size:3,\ capacity:5$\backslash$n!}" is printed (without the quotes of 
course).

\end{enumerate}

Note that you will see many actions taken by methods are similar. For instance 
arrays are copied in both the copy constructor and \verb!operator=!. You must 
avoid code duplication. You can create \lq\lq helper" functions used by methods in 
your class. These can be either functions or methods. For example:
\begin{Verbatim}[frame=single]
// C.h
void helper(...);
class C
{
    ...
};

// C.h
class C
{
    ...
    void helper(...);
    ...
};
\end{Verbatim}

Note that if declared as a method, you can \lq\lq hide” the helper method 
from outsiders by placing it under the \verb!private! section of the class:

\begin{Verbatim}[frame=single]
// C.h
class C
{
    ...
private:
    ...
    void helper(...);
    ...
};
\end{Verbatim}

\newpage
\textsc{Error on Allocating Memory}

It is a good practice (and you must follow it) to check your pointer after a 
memory allocation.  Failure of memory allocation can be checked by comparing 
the pointer's value. If memory is not allocated, the pointer is set to 
\verb!NULL!. The following is an example:

\begin{Verbatim}[frame=single]
int * p = new int;
if (p == NULL)
{
    std::cout << "ERROR!!! You ran out of member!!!"
}
else // memory is allocated
{
    ...
}
\end{Verbatim}

If a pointer is not allocated memory after a new operator call, your program 
must print the string \lq\lq ERROR: memory allocation returns NULL".

This is the only action you have to take when there is a memory allocation 
error. In a real software, it's likely that the program will print a message 
and then completely halt the program. 

(Actually there are two possible ways to check for memory allocation errors 
and both should be used if possible. We won't be able to talk about the other 
method because it involves the concept of “exceptions” which we will be talk 
about later.)


\newpage 
\textsc{\textnormal{\texttt{operator=}} and copying}

[This section is included for your benefit. We already talked about the 
problems of direct member-wise copying for pointer members in class.]

The following notes involves some form of copying between objects in the 
context where pointers are involved. Suppose we first look at a very simple 
situation:

\begin{Verbatim}[frame=single]
class Dummy
{
private:
    int x;
    double y;
    char z;
};
\end{Verbatim}

If you execute the following code:

\begin{Verbatim}[frame=single]
Dummy a, b;
b = a;  // b invokes operator= with a as argument
Dummy c = a;    // c invokes copy constructor with a as argument
\end{Verbatim}

In both cases, the values of instance variables of \verb!a! are copied to 
corresponding instance variables of \verb!b! and \verb!c!. In other words, 
\verb!a.x! is copied over to \verb!b.x!, etc.

This is not always appropriate. For instance in the case where an instance 
variable points to an array. The following discussion on copying the contents 
from an object to another there a pointer to an array is involved is relevant 
to any copying behavior in an object, including \verb!operator=! and the copy 
constructor. 

Suppose \verb!a! is an object where the \verb!x! member is a pointer to an 
array of 4 integers in the heap:
\begin{python}
s = open('diagram1.py', 'r').read()
from latextool_basic import *
execute(s)
\end{python}

(The following has nothing to do with the values of \verb!size! and \verb!capacity!
so I left them out.)
Now suppose 
you have another object from the same class:
\begin{python}
s = open('diagram2.py', 'r').read()
from latextool_basic import *
execute(s)
\end{python}


After the statement \lq\lq \verb!b = a;!", you want this:

\begin{python}
s = open('diagram3.py', 'r').read()
from latextool_basic import *
execute(s)
\end{python}

This is obvious different from just executing the 
statement \lq\lq \verb!b.x = a.x;!" which will give this 
picture instead:

\begin{python}
s = open('diagram4.py', 'r').read()
from latextool_basic import *
execute(s)
\end{python}

where \verb!b.x! will point to what \verb!a.x! is pointing 
to.
As mentioned in class, when \verb!a! calls the destructor, the
memory \verb!a.x! is pointing will be released
back to the heap and \verb!b.x! 
will be pointing to
memory in the heap that has not been allocated:
\begin{python}
s = open('diagram5.py', 'r').read()
from latextool_basic import *
execute(s)
\end{python}

so that when \verb!b! called the destructor an error will occur 
when we try to do a \verb!delete []! on \verb!x!
causing a double free error.
(We already 
talked about this in class.) 

The correct sequence of events for \lq\lq\verb!b = a;!"
is depicted below.
We start with this:

\begin{python}
s = open('diagram6.py', 'r').read()
from latextool_basic import *
execute(s)
\end{python}

Step 1. First release the memory \verb!b.x! is pointing to:
\begin{python}
s = open('diagram7.py', 'r').read()
from latextool_basic import *
execute(s)
\end{python}
(We executed \lq\lq\verb!delete [] b.x!".
\verb!b.x! still have the same address as before
but that array's memory is not allocated to any pointer
and belongs to the heap.)

Step 2. Allocate enough memory for \verb!b.x! to point to  
\begin{python}
s = open('diagram8.py', 'r').read()
from latextool_basic import *
execute(s)
\end{python}
(I'm writing \verb!?! to denote garage integer values.
Of course the array \verb!b.x! points to is an
integer array and do hold integer values.)

Step 3.
Copy values \verb!a.x! is pointing to array
\verb!b.x! is pointing to:
\begin{python}
s = open('diagram9.py', 'r').read()
from latextool_basic import *
execute(s)
\end{python}

TADA! That's it. But there's a problem.
This is a very subtle problem. 
Suppose the above is the sequence of events in \verb!operator=! when 
we execute \lq\lq \verb!a = b;!".
If \verb!a! and \verb!b! are distinct 
objects, then the above operations are in fact correct. 
But what if \verb!a! and \verb!b!
are actually the same object?? This can happen when 
for instance we have this:

\begin{Verbatim}[frame=single]
IntDynArr a(5);
IntDynArr & b = a; // b is a reference to a
\end{Verbatim}

Now let's track the above sequence of events for this
scenario using our 3-step algorithm above.
We start off with this:
\begin{python}
s = open('diagram10.py', 'r').read()
from latextool_basic import *
execute(s)
\end{python}

Step 1.
First release the memory \verb!b.x! is pointing to
(same as releasing \verb!a.x! since \verb!b!
is a reference to \verb!a!):
\begin{python}
s = open('diagram11.py', 'r').read()
from latextool_basic import *
execute(s)
\end{python}

Step 2.
Allocate enough memory for \verb!b.x! to point to (same as allocating 
memory for \verb!a.x!) 
\begin{python}
s = open('diagram12.py', 'r').read()
from latextool_basic import *
execute(s)
\end{python}

Step 3.
Copy values \verb!a.x! is pointing to into the array \verb!b.x! is pointing to 
(same as copying \verb!a.x! to \verb!a.x!!!!) 
\begin{python}
s = open('diagram12.py', 'r').read()
from latextool_basic import *
execute(s)
\end{python}

Of course this means that you have lost all
the original values of \verb!a!.
Now what?? Well, to correct the above,
if \verb!a! and \verb!b! are the same objects
(for instance if
\verb!b! is a reference to \verb!a! or \verb!a!
is a reference to \verb!b!), 
then we don't even want to perform anything at all since
... well ... they are 
the same already.
This means that before you perform the above
\lq\lq\verb!delete []!", 
\lq\lq\verb!new []!", copy values operations,
you need to check if the objects are the same. 
And you can check if the objects are the same
by looking at their memory 
addresses.
Here's then the corrected \verb!operator=! (the same idea 
applies to the copy constructor):

\begin{Verbatim}[frame=single]
IntDynArr & IntDynArr::operator=(const IntDynArr & b) 
{
    if (this != &b)
    {
        ... // do something only if *this and b are different
    }
    return (*this);
}
\end{Verbatim}

This is the general template for \verb!operator=! for all
classes.
(In most cases, the return type should be
\verb!const IntDynArr &!, but it wouldn't be wrong
to return \verb!IntDynArr &!.
It depends on what you want to do after
\verb!operator=!.)

This is a very subtle error so make sure you remember it.

Since the copy constructor and \verb!operator=! are very similar, parts 
of their code will be the same. Do not duplicate code. You can create a 
function that both the copy constructor and
\verb!operotor=! can both use.



\newpage
\textsc{Size and Capacity}

The size of the array the \verb!x! instance variable is pointing 
to is \verb!capacity!. The instance variable \verb!size! actually 
indicates the number of elements in the array that are valid. When 
you call the constructor

\begin{Verbatim}[frame=single]
IntDynArr a(5);
a.resize(5);
for (int i = 0; i < 5; i++) a[i] = i;
\end{Verbatim}
\verb!a.x! points to an array of 5 integers.
Both \verb!size! and \verb!capacity! 
are set to 5.
\begin{python}
s = open('diagram13.py', 'r').read()
from latextool_basic import *
execute(s)
\end{python}

Now when we call
\begin{Verbatim}[frame=single]
a.resize(3);
\end{Verbatim}
we get the following picture:
\begin{python}
s = open('diagram14.py', 'r').read()
from latextool_basic import *
execute(s)
\end{python}

Although \verb!a.x! still points to an array of 5 integers, only 
the first three make up the array \verb!a! is modeling. \verb!a.x[3]!
and \verb!a.x[4]! are are not considered part of the array. Accessing 
them should be considered an incorrect operation. We are not enforcing 
the check now but later I will show you how to generate an error 
(called an \lq\lq exception" object which can cause a program to halt). 

Now suppose we call
\begin{Verbatim}[frame=single]
a.resize(4);
\end{Verbatim}
we get this picture:
\begin{python}
s = open('diagram15.py', 'r').read()
from latextool_basic import *
execute(s)
\end{python}
and when we call
\begin{Verbatim}[frame=single]
a.resize(5);
\end{Verbatim}
we get the original picture:
\begin{python}
s = open('diagram13.py', 'r').read()
from latextool_basic import *
execute(s)
\end{python}

But what if we request for more memory such as:
\begin{Verbatim}[frame=single]
a.resize(6);
\end{Verbatim}
the end result of this call can be described by this picture:
\begin{python}
s = open('diagram16.py', 'r').read()
from latextool_basic import *
execute(s)
\end{python}
(where ? denotes garbage value).

Note that the \verb!capacity! is 12,
i.e., twice of \verb!size!.
First of all 
let me tell you how to achieve this.
Next I will explain why you try to grab 
so much memory, i.e., 12 and not just 6.

The above \verb!resize!
operation can be achieved by doing this.
We start off with this:
\begin{python}
s = open('diagram13.py', 'r').read()
from latextool_basic import *
execute(s)
\end{python}


Step 1.
Declare a local \verb!int! pointer, say \verb!p!,
to point to an array with
2*6 integers and set 
\verb!capacity! to 2*6: 
\begin{python}
s = open('diagram17.py', 'r').read()
from latextool_basic import *
execute(s)
\end{python}

Step 2.
Copy old values
from the array \verb!a.x! points to
into the array \verb!p! points to: 
\begin{python}
s = open('diagram18.py', 'r').read()
from latextool_basic import *
execute(s)
\end{python}

Step 3.
Deallocate memory \verb!a.x! is pointing to (in order to avoid memory leak): 
\begin{python}
s = open('diagram19.py', 'r').read()
from latextool_basic import *
execute(s)
\end{python}

Step 4.
Point \verb!a.x! to memory \verb!p! is pointing to (i.e., \verb!x! and \verb!p! 
have the same address): 
\begin{python}
s = open('diagram20.py', 'r').read()
from latextool_basic import *
execute(s)
\end{python}

Step 5.
Set the size to 6:
\begin{python}
s = open('diagram21.py', 'r').read()
from latextool_basic import *
execute(s)
\end{python}

When you return from the resize method \verb!p! is destroyed (since
it's a local variable declared in the scope of \verb!resize!):
\begin{python}
s = open('diagram22.py', 'r').read()
from latextool_basic import *
execute(s)
\end{python}

Note that when you \verb!resize!, values in the array are preserved whenever 
possible.

The above refers to \verb!resize! in the context of a new larger size which 
is larger than the \verb!capacity!.
In this case, you always allocate memory for \textit{twice} the new size.

Now why do you want to do something like that? Why not just allocate memory for the 
exact new size? Suppose you are in a situation where you \verb!resize! by 1 more each time. 
For instance you want to execute \verb!a.resize(6), a.resize(7), a.resize(8),! etc. 
Then the memory deallocation and allocation will be too costly. The above method of 
grabbing more memory than needed is to save on CPU time for memory allocation and deallocation. 
After the resize operation, if you need more memory for the array, you must have twice 
as much as you need.

What happens when you \verb!resize! to a \textit{smaller} size? If 
the new \verb!size! is smaller than 1/3 of the \verb!capacity!, you have to deallocate 
so that the new \verb!capacity! is again twice the actual size needed. Let's do an example.

\begin{Verbatim}[frame=single]
IntDynArr a(12);
a.resize(12);  
for (int i = 0; i < 12; i++) a[i] = i;
\end{Verbatim}

The picture at this point is
\begin{python}
s = open('diagram23.py', 'r').read()
from latextool_basic import *
execute(s)
\end{python}

Now suppose we execute:
\begin{Verbatim}[frame=single]
a.resize(3);
\end{Verbatim}
Note that 3 is less than 1/3 of 12 (which is 4).
The \verb!resize! method should do this:
\begin{python}
s = open('diagram24.py', 'r').read()
from latextool_basic import *
execute(s)
\end{python}
The array that \verb!a.x! points to
is now smaller and the capacity is again
twice the new size.

If instead of the above we execute this:
\begin{Verbatim}[frame=single]
a.resize(5);
\end{Verbatim}
we get the following picture instead:
\begin{python}
s = open('diagram25.py', 'r').read()
from latextool_basic import *
execute(s)
\end{python}

In summary, if you \verb!resize! and the new size is
larger than the available capacity, you need to
deallocate--and--allocate where the new array
has a capacity that is twice the new size.
If the \verb!resize! is requested for a new size that is
less than 1/3 of original capacity,
you also have to deallocate--and--allocate
so that the new array (again) has a capacity
that is twice the new size.
Deallocate-and-allocate is sometimes called reallocation.

[NOTE:
Some dynamic array classes
reallocate with a new capacity that is
1.5 times of the new size instead of 2 times.]


The methods in your class must follow the following rules on size and capacity:
\begin{tightlist}
  \li The amount of memory (i.e. the capacity) allocated by constructors
  is the capacity passed in and the size is set to 0.
\li Except for \verb!operator[]!, all other relevant methods and operators 
must change the capacity of the object according to the above rule. In 
other words, there is no change in the memory allocation when the 
new size is between capacity/3 and capacity; but when the size is \textless
\, capacity/3 or \textgreater \, capacity, memory is re-allocated so that the new capacity 
is twice the new size. This applies to \verb!resize()!, \verb!operator+=!, 
\verb!operator=!, \verb!insert()!, and \verb!remove()!.
\end{tightlist}

\newpage
\textsc{A Common Error}

A very common error I see is that students simply change the value of 
\verb!capacity! assuming that the the point x will point to an array of 
different size. If your object has point x pointing to an array of 100 
integers (and capacity is set to 100), obviously simply by changing 
\verb!capacity! 1000 does not mean that x has valid access to 1000 
integers in the heap!!!

That's like saying if you have a bathtub labeled 50 gallons and you 
simply took black paint and change the label to 500 gallons,
then magically
your 
bathtub will hold 500 gallons!!! That obviously is not true!!! You have 
to change the bathtub to actually achieve a new capacity!!!

Likewise if you want your \verb!x! (in the object) to point to an array of 
with 1000 integers instead of 100, you have to request for 1000 from 
the heap and point your \verb!x! to that new array in the heap (and of course 
making sure that the memory for the original 100 was deallocated properly.)


\newpage
\textsc{\lq\lq Default" Value}

Take note of the following cases:
\begin{tightlist}
\li There is no check on array bound when \verb!operator[]()! is called. 
(In other words in such cases, the program can cause an error.) You can 
however choose your own default value for testing purposes.
\li If an array is resized, your class need not assign values to the extra 
elements, i.e., their values can be random. You can however choose your own 
default value for testing purposes.
\end{tightlist}


\newpage
\textsc{Problems with Heap Usage}

There are several tools for debugging programs with incorrect memory (or heap) 
usage.
For instance you might want to google “c++ debug memory leak” or more 
specifically “c++ g++ debug memory leak” or “c++ .net studio debug memory leak”.
(I've already talked about asan, the address sanitizer tool from google.
I've also talked about gdb.)

But the most important thing for you is to analyze carefully what your code 
should do and write your code carefully rather than dive in without careful 
thought and then hope that memory debugging tools will save you. You will find 
that memory debugging a badly written piece of code can take a lot more time than 
just simply writing better and correct code to begin with.

If you have to debug (and you will) your best friend when it comes to debugging 
and tracing code is the print statement. (For more complex systems, debugging tools 
are frequently more useful than printing.)

To trace where a statement crashes your program, all you need to do is to insert 
print statements. 
\begin{Verbatim}[frame=single,fontsize=\small]
... some code ...

std::cout << "here ... 0" << std::endl;

... some code ...

std::cout << "here ... 1" << std::endl;

... some code ...
\end{Verbatim}

If you don't see the second print statement when your program crashes, then of 
course the bug is between the two print statements. You then insert another print 
statement in the middle:
\begin{Verbatim}[frame=single,fontsize=\small]
... some code ...

std::cout << "here ... 0" << std::endl;

... some code ...

std::cout << "here ... 0.5" << std::endl;

... some code ...

std::cout << "here ... 1" << std::endl;

... some code ...
\end{Verbatim}

This will allow you to \lq\lq squeeze" the bug to find out it's exact location 
very quickly using the above \lq\lq binary search" method.

One very important thing to note is that your program might crash before the 
print content printed arrived on your console window or terminal shell. To 
ensure that the printed content arrives before your program shuts down, you 
must print \verb!std::endl!:
\begin{Verbatim}[frame=single,fontsize=\small]
std::cout << "here ... 0" << std::endl; // GOOD FOR DEBUGGING!!!
\end{Verbatim}

and not this:
\begin{Verbatim}[frame=single,fontsize=\small]
std::cout << "here ... 0" << '\n'; // BAD FOR DEBUGGING!!!
\end{Verbatim}


\newpage
\textsc{Test File}

The following is only a skeleton test code. Complete it and test your library thorougly.

{\small
\begin{Verbatim}[frame=single,fontsize=\small]
/************************************************************************************
 * File  : testIntDynArr.cpp
 * Author:
 * Date  :
 *
 * Description: This is the test program for the IntDynArr class.
 ***********************************************************************************/

#include <iostream>
#include "IntDynArr.h"

#define SIZE(x) (sizeof(x)/sizeof(int))


void test_IntDynArr()
{
    IntDynArr a;
    std::cin >> a;
    std::cout << a << std::endl;
}


void test_IntDynArr_size()
{
    int size;
    std::cin >> size;
    IntDynArr a(size);
    std::cout << a << std::endl;
}


void test_IntDynArr_array()
{
    int x[] = {1, 2, -4, 0};
    IntDynArr a(SIZE(x), x);
    std::cout << a << std::endl;
}


void test_print()
{
    int x[] = {1, 2, -4, 0};
    IntDynArr a(SIZE(x), x);
    std::cout << a << std::endl;
}


void test_size()
{
    IntDynArr a;
    std::cin >> a;
    std::cout << a.get_size() << std::endl;
}


void test_get_capacity()
{
    IntDynArr a;
    std::cin >> a;
    std::cout << a.get_capacity() << std::endl;
}

void test_eq()
{
    IntDynArr a, b;
    std::cin >> a >> b;
    std::cout << (a == b) << std::endl;
}

void test_plus_eq()
{
    IntDynArr a, b;
    std::cin >> a >> b;
    std::cout << (a += b) << ' ';
    std::cout << a << ' ';
    std::cout << ((a += b) += b) << std::endl;
}

void test_remove()
{
    IntDynArr a;
    int i;
    std::cin >> a >> i;
    std::cout << a << ' ' << a.remove(i) << std::endl;
}

int main()
{
    int option;
    std::cin >> option;
    switch (option)
    {
        case 1:
            test_IntDynArr();
            break;
        case 2:
            test_IntDynArr_size();
            break;
        case 3:
            test_IntDynArr_array();
            break;
        case 7:
            test_eq();
            break;
        case 17:
            test_remove();
            break;
        case 19:
            test_print();
            break;
    }

    return 0;            
}
\end{Verbatim}
}


The test option numbering follows the same numbering as in the explanation earlier.

\end{document}
