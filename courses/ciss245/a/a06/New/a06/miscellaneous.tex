\textsc{Coda and miscellaneous notes}

You're done with the assignment.
Here are some comments.
(You do not need to implement any of the features mentioned below.)

The above implies that you can, for instance,
given a polynomial with fractional coefficients,
you
now find the exact fractional roots in a range and
up to an accuracy of, say 1/100.
For instance to find fractional roots in [0, 100]
of  
\[
\frac{1}{3} x^3 - \frac{5}{3} x + \frac{7}{42} = 0
\]
you do
\begin{Verbatim}[frame=single,fontsize=\small,commandchars=\~\!\@]
Fraction start = get_Fraction(0, 1);
Fraction end = get_Fraction(100, 1);
Fraction step = get_Fraction(1, 100);

Fraction a = get_Fraction(1, 3);
Fraction b = get_Fraction(3, 3);
Fraction c = get_Fraction(7, 42);
Fraction zero = get_Fraction(0, 1);

for (Fraction x = start; x <= end; x += step)
{
    reduce(~textred!x@);
    if (a * x * x * x - b * x + c == zero)
    {
        std::cout << x << std::endl;
    }
}
\end{Verbatim}

Note that there are several convenient features which are
still missing in your \verb!Fraction! library.

You do not have mixed mode operations.
For instance you cannot perform addition of a fraction and an int:
\begin{console}
Fraction f = get_Fraction(1, 3);
f = f + 2; // NOT POSSIBLE
\end{console}
Right now, we add two \verb!Fraction! values.
We have not taught our library
how to add a \verb!Fraction! with an \verb!int! or an \verb!int! with a
\verb!Fraction!.
To do this, 
you need to implement the following:
\begin{console}
Fraction operator+(const Fraction &, int);
\end{console}
This also appears in the above:
\begin{console}
...
    if (a * x * x * x - b * x + c == zero)
...
\end{console}
You do not have \verb!operator==! that compares a \verb!Fraction! value with an
int.
If you do have
\begin{console}
bool operator==(const Fraction &, int);
\end{console}
then the above boolean expression simplies to
\begin{console}
...
    if (a * x * x * x - b * x + c == 0)
...
\end{console}

Etc.

The issue of converting a \verb!double! to a \verb!Fraction! also requires more
thought. For instance, the \verb!double! $0.333333333$ is probably meant to
be the fraction $\frac{1}{3}$. However the type conversion function from
\verb!double! to \verb!Fraction! might be written so that the fraction is
$\frac{333333333}{11000000000}$.
