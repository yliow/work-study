\texttt{\bf operator-}

Once you've understood \verb!operator+!, this part is easy.

Define \verb!operator-! so that when \verb!operator-! is called with two
\verb!Fraction! variables, the return value is a \verb!Fraction! value that
models the difference between the fractions in the \lq\lq real" world. So
what's the appropriate prototype?

Mathematically this is how you subtract two fraction numbers:
\[
\frac{a}{b} -
\frac{c}{d} =
\frac{ad - bc}{bd}
\]

You obviously need the following prototype:

\begin{Verbatim}[frame=single]
Fraction operator-(const Fraction &, const Fraction &);
\end{Verbatim}

Add it to your \verb!Fraction.h! and the following to your \verb!main()!:

{\small
\begin{Verbatim}[frame=single]
...
void test_subtract()
{
    int n0, d0, n1, d1;
    std::cin >> n0 >> d0 >> n1 >> d1;  
    Fraction r0 = {n0, d0};
    Fraction r1 = {n1, d1};
    std::cout << (r0 - r1) << std::endl;
}


int main()
{
    int option;
    std::cin >> option;
    
    switch (option)
    {
        // other cases
        case 3:
            test_subtract();
            break;
    }

    return 0; 
}
\end{Verbatim}
}

Here is a test case for you:

\textbf{Test 1:}
\begin{console}[commandchars=~!@]
~underline!3 2 -3 -2 5@
-4/15
\end{console}

Add more test cases to your \verb!stdin.txt!.
