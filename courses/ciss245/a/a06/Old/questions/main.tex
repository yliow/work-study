%-*-latex-*-
%-*-latex-*-
\input{mybookpreamble.tex}
\input{yliow}
\textwidth=5.5in

%-*-latex-*-
%\usepackage{makeidx}
%\usetikzlibrary{shapes.geometric}
%\usetikzlibrary{arrows}
\usetikzlibrary{fit}
%\usetikzlibrary{positioning}

\renewcommand\debug[1]{#1}
%\renewcommand\debug[1]{}

\newcommand\starprob{ * }
\newcommand\bangprob{ ! }


\makeindex
\begin{document}
\topmatter


\begin{document}
\topmatter

\textsc{Objectives:}
The purpose of this assignment is to build a simple library for a struct.
\begin{enumerate}[nosep]
\item Declare a struct variable.
\item Access member variables of a struct variable.
\item Write function/operator with struct parameters.
\item Write function/operator with struct return value.
\end{enumerate}

For CISS245, I also emphasize rigorous software testing. Much of the testing
strategies that you will learn in this class (although in a small scale) is
actually being practiced in the real software engineering world. Specifically,
for this and future assignments I will emphasize the testing of the smallest
units of a software such as functions. These are called \textbf{unit tests}.


\newpage
\textsc{Structure Variables}

A structure variable is just a variable that contains variables (refer to your
notes). Run this:
\begin{Verbatim}[frame=single,fontsize=\small]
#include <iostream>
#include <iomanip>

struct Time
{
    int hour;
    int min;
    int sec;
}

int main()
{
    Time t0;
    t0.hour = 5;
    t0.min = 18;
    t0.sec = 0;

    std::cout << std::setw(2) << std::setfill('0') << t0.hour
              << ':'
              << std::setw(2) << std::setfill('0') << t0.min
              << ':'
              << std::setw(2) << std::setfill('0') << t0.sec
              << std::endl;

    return 0;
}
\end{Verbatim}

In this case, I want to work with time which is represented by hour, minute,
and second. However, I frequently want to view hour, minute, and second not as
three things but one. So I create this new type:
\begin{console}[fontsize=\small]
struct Time
{
    int hour;
    int min;
    int sec;
}
\end{console}

I can then create a \verb!Time! (structure) variable like this:
\begin{console}[fontsize=\small]
Time t0;
\end{console}

And now \verb!t0! contains three variables. \verb!t0! has:
\begin{tightlist}
  \li \verb!t0.hour! which is an \verb!int! variable.
  \li \verb!t0.min! which is an \verb!int! variable.
  \li \verb!t0.sec! which is an \verb!int! variable.
\end{tightlist}

The variables inside a structure variable are usually called
\textbf{member variables}.

\textsc{Memory model}

Here's the memory model of \verb!main()! for the variable \verb!t0! in the
above example:
\begin{python}
from latextool_basic import *
p = Plot()
t0_label = Rect(x0=-0.5, y0=0.5, x1=0.5, y1=1.5, linewidth=0.0, label=r"\texttt{t0}")
t0 = array(x0=0.5, y0=0.5, width=2, height=1, xs=[5,18,0])
main_label = Rect(x0=-0.5, y0=3.0, x1=0.5, y1=3.5, linewidth=0.0, label=r"\texttt{main}")
main = Rect(x0=-0.5, y0=-1.0, x1=7.0, y1=3.0, linewidth=0.05)
p.add(main_label)
p.add(main)
p.add(t0_label)
p.add(t0)

print(p)
\end{python}

(Like I said, \verb!t0! is just a variable containing variables.)

Here's another example of a structure type:
\begin{console}
struct Student
{
    int id;
    double gpa;
}
\end{console}

In this case, a \verb!Student! variable contains $2$ variables. For instance,
after this declaration:
\begin{console}
Student jdoe;
\end{console}
the variable \verb!jdoe! contains:
\begin{tightlist}
  \li \verb!jdoe.id!, an \verb!int! variable and
  \li \verb!jdoe.gpa!, a \verb!double! variable.
\end{tightlist}

Note that the variables in a structure can have different types. In the
\verb!Student! case, there is one \verb!int! variable and one \verb!double!
variable.

Note also that to go into a variable inside a structure variable, you use the
dot. For instance, to go to the hour of \verb!t0!, you use:
\begin{console}
t0.hour
\end{console}

The variables inside a structure variable are just like any regular variable so
you do know how to work with them (input, output, operators etc.) For instance,
back in the first \verb!Time! example, we assign values to the variable
\verb!t0!:
\begin{console}
t0.hour = 5;
t0.min = 18;
t0.sec = 0;
\end{console}

We also read the values in \verb!t0! and print them:
\begin{console}
std::cout << std::setw(2) << std::setfill('0') << t0.hour
          << ':'
          << std::setw(2) << std::setfill('0') << t0.min
          << ':'
          << std::setw(2) << std::setfill('0') << t0.sec
          << std::endl;
\end{console}

(By the way, although a structure variable is sort of like an array, you see
from the above that there are differences. For instance, an array can only
contain values of the same type. You can't have an array of $2$ integers and
$3$ doubles. An array is an \verb!int! array or a \verb!double! array. You
can't have it both ways! To go into an array, you have the bracket \verb![]!
with an index value. You get into a structure using the dot and a name for the
variable inside the structure that you're trying to access.)

\textsc{Initialization}

Instead of doing this:
\begin{console}
Time t0;
t0.hour = 5;
t0.min = 18;
t0.sec = 0;
\end{console}
to initialize structure variables, you can use the same notation as in array
initialization:
\begin{console}
Time t0 = {5, 18, 0};
\end{console}




\textsc{Functions: struct parameters}

Just like variables of basic types, it's not surprising that you can pass
structure variables into a function. Run this:
\begin{console}[commandchars=\\\@\$]
#include <iostream>
#include <iomanip>

struct Time
{
    int hour;
    int min;
    int sec;
};

\textbf@void print(Time t)
\textbf@{
    \textbf@std::cout << std::setw(2) << std::setfill('0') << t.hour$
              \textbf@<< ':'$
              \textbf@<< std::setw(2) << std::setfill('0') << t.min$
              \textbf@<< ':'$
              \textbf@<< std::setw(2) << std::setfill('0') << t.sec$
              \textbf@<< std::endl;$
\textbf@}$

int main()
{
    Time t0 = {5, 18, 0};
    \textbf@print(t0);$

    return 0;
}
\end{console}

Structure variables in function calls are by default pass--by--value. This
means that the variable inside the function cannot change the variable in the
calling function. Try this:
\begin{console}[commandchars=\\\@\$]
...

\textbf@void inc(Time t)$
\textbf@{$
    \textbf@t.sec++;$
\textbf@}$

...

int main()
{
    ...
    \textbf@inc(t0);$
    \textbf@print(t0); // t0.sec is the same!!!$

    return 0;
}
\end{console}

The variable \verb!t0! is not changed when you return from the function call to
\verb!inc()!. So, \textbf{if you do want to change \texttt{t0}}, you have to force a
pass--by--reference like this:
\begin{console}[commandchars=\\\@\$]
...

void inc(Time \underline@\textbf@&$$ t)
{
    t.sec++;
}

...

int main()
{
    ...
    inc(t0);
    print(t0); // t0.sec has changed!!!

    return 0;
}
\end{console}
So, in terms of parameter--passing, structure variables are like \verb!int! and
\verb!double! parameters which are also pass--by--value.




\textsc{Function: Reference--to--struct parameters}

For performance reasons (i.e., speed), we force all \textbf{structure variables
to be passed by reference}. Make this change and run your program again:
\begin{console}[commandchars=\\\@\$]
void print(Time \textbf@\underline@&$$ t)
{
    std::cout << std::setw(2) << std::setfill('0') << t.hour
              << ':'
              << std::setw(2) << std::setfill('0') << t.min
              << ':'
              << std::setw(2) << std::setfill('0') << t.sec
              << std::endl;
}
\end{console}
This avoids creating memory for \verb!t!; \verb!t! simply references the
variable in the calling function.

If a \verb!struct! parameter is a reference, the function can change the
variable in the calling function:
\begin{console}[commandchars=\\\@\$]
void print(Time \textbf@\underline@&$$ t)
{
    std::cout << std::setw(2) << std::setfill('0') << t.hour
              << ':'
              << std::setw(2) << std::setfill('0') << t.min
              << ':'
              << std::setw(2) << std::setfill('0') << t.sec
              << std::endl;
    \textbf@// This will change the Time variable that t references!!!$
    \textbf@t.sec = 0;$
}
\end{console}

This means that the \verb!print()! function now has the ability to change the
variable in the calling function. To avoid accidentally changing this variable,
we force the parameter \verb!t! to be constant:
\begin{Verbatim}[frame=single, commandchars=\\\@\$]
void print(\textbf@\underline@const$$ Time \textbf@\underline@&$$ t)
{
    std::cout << std::setw(2) << std::setfill('0') << t.hour
              << ':'
              << std::setw(2) << std::setfill('0') << t.min
              << ':'
              << std::setw(2) << std::setfill('0') << t.sec
              << std::endl;
}
\end{Verbatim}

So, let me repeat myself:

All stucture variables must be passed by reference (it's also possible to pass
the address of the \verb!struct! variable to a pointer receiving the address --
see later.) If a function wants to change the variable in the calling function,
there's nothing else to do. If a function should not change the variable in the
calling function, make the parameter \verb!const! as well.




\textsc{Functions: struct return value}

You can return a structure value:
\begin{console}[commandchars=~!@]
...

~textbf!Time addOneHour(const Time & t)@
~textbf!{@
~textbf!    Time newtime = t;@
~textbf!    newtime.hour++;@

~textbf!    return newtime;@
~textbf!}@

int main()
{
    ...

    ~textbf!Time t1 = addOneHour(t0);@
    ~textbf!print(t1);@

    return 0;
}
\end{console}

(So, structure variables are also different from arrays because you can return
them. Recall that you cannot return arrays.)




\textsc{Functions: pointer--to--struct parameters}

As mentioned earlier, it's also possible to pass the address of a
\verb!struct! variable. Try this version:
\begin{console}[commandchars=~!@]
void print(~textbf!~underline!const@@ Time * t)
{
    std::cout << std::setw(2) << std::setfill('0') << t->hour 
              << ':'
              << std::setw(2) << std::setfill('0') << t->min 
              << ':'
              << std::setw(2) << std::setfill('0') << t->sec
              << std::endl;
}

... 

int main()
{
    Time t0;
    t0.hour = 5;
    t0.min = 18;
    t0.sec = 0;
    print(~textbf!~underline!&@@t0);

    return 0;
}
\end{console}
where \verb!t->hour! is really just \verb!(*t).hour!.

The corresponding \verb!addOneHour()! function looks like this:
\begin{Verbatim}[frame=single, commandchars=~!@]
Time addOneHour(const Time * t)
{
    Time newtime = *t;
    newtime.hour++;

    return newtime;
}

int main()
{
    ...
    Time t1 = addOneHour(~textbf!~underline!&@@t0);

    return 0;
}
\end{Verbatim}

Let me summarize:
\begin{itemize}
  \li If you want a function to work with a \verb!struct! value, instead of
pass--by--value, you should use either pass--by--reference or pass the address,
i.e., the receiving parameter in the function must be a reference or a pointer.
  \li If the parameter in the function should not change the \verb!struct!
value from the calling function, then the parameter in the function must be a
constant reference or a pointer to a \verb!const! value.
\end{itemize}




\newpage
\textsc{Now for the assignment}

The goal is to build a simple library for computing fractions and test it
rigorously.

To make it easy for you, I have written it in parts.

The questions build upon each other. This means that subsequent questions are
going to rely significantly on previous questions.
Once you are done with a question, you make a copy of that folder
for the next question and continue work on the next folder.
For instance if you are done with question 3, you make a copy of the
folder for question 3 and rename it appropriately for question 4 and
continue work on the folder for question 4. Etc.



\newpage
\textsc{The Given Code Base}

Note that C\texttt{++}
does not natively support fractions. (Doubles are not fractions.)
In this assignment we will be developing useful functions and operators to
support the use of fractions. I'll give you the structure definition:
\begin{console}
struct Fraction
{
    int n; // numerator
    int d; // denominator
};
\end{console}

This means of course that if you want to model the mathematical fraction
\[x = \frac{3}{4}\]
you would do this in your program:
\begin{console}
Fraction x;
x.n = 3;
x.d = 4;
\end{console}
or better:
\begin{console}
Fraction x = {3, 4};
\end{console}

You will need to write three files:
\begin{itemize}
  \li \verb!test_fraction.cpp!: This contains a program to test the
      \verb!Fraction struct! and its supporting functions and operators.
  \li \verb!Fraction.h!: This is the header function containing the
      \verb!Fraction struct! and its prototypes.
  \li \verb!Fraction.cpp!: This file contains the definition of the prototypes
      in \verb!Fraction.h!.
\end{itemize}

The following are the skeleton files (remember that skeleton files
are incomplete and might contain errors):
{\small
\begin{Verbatim}[frame=single]
// Author: 
// Date  : 
// File  : test_fraction.cpp

#include <iostream>
#include "Fraction.h"


void test_print()
{
    int n = 0, d = 0;
    std::cin >> n >> d;
    Fraction f = {n, d};
    std::cout << f << std::endl;
}


int main()
{
    int option;
    std::cin >> option;
 
    switch (option)
    {
        case 1:
            test_print();
            break;
    }

    return 0; 
}
\end{Verbatim}
}

(Note: the tester uses option $1$ to test the print feature.)

\begin{Verbatim}[frame=single]
// Author: 
// Date  : 
// File  : Fraction.h

#ifndef FRACTION_H
#define FRACTION_H

#include <iostream>

struct Fraction
{
    int n; // numerator
    int d; // denominator
};


std::ostream & operator<<(std::ostream &, const Fraction &);

#endif
\end{Verbatim}

\begin{Verbatim}[frame=single]
// Author: 
// Date  : 
// File  : Fraction.cpp

#include <iostream>
#include "Fraction.h"


std::ostream & operator<<(std::ostream & cout, const Fraction & r)
{
    cout << r.n << '/' << r.d;
    return cout;
}
\end{Verbatim}

You should compile (correcting any errors if necessary)
and run the program 
to make sure that it works before
continuing. Note that in \verb!Fraction.cpp! the \verb!operator<<! is defined.
Right now, the only thing you need to know is that in the code for
\verb!operator<<!, if you want to print \verb!r.n! (say), call
\begin{console}
cout << r.n
\end{console}
instead of
\begin{console}
std::cout << r.n
\end{console}
i.e.:
\begin{Verbatim}[frame=single, commandchars=~!@]
// Author: 
// Date  : 
// File  : Fraction.cpp

#include <iostream>
#include "Fraction.h"


std::ostream & operator<<(std::ostream & cout, const Fraction & r)
{
    ~textbf!~underline!cout@@ << r.n << '/' << r.d;
    return cout;
}
\end{Verbatim}

This is how you should print inside this \lq\lq function" (technically speaking
this is an operator, not a function). Also, do not modify the prototype of this
\lq\lq function" nor remove the last statement in the body:
\begin{Verbatim}[frame=single, commandchars=~!@]
std::ostream & operator<<(std::ostream & cout, const Fraction & r)
{
    ...
    ~textbf!~underline!return cout;@@
}
\end{Verbatim}

In more detail, in \verb!main()!, suppose you have \verb!Time! variable called
\verb!r!, if you execute
\begin{console}
std::cout << r;
\end{console}
C\texttt{++} actually executes
\begin{console}
operator<< (std::cout, r);
\end{console}
i.e., it calls the \lq\lq function" (or rather the operator) \verb!operator<<!
so that in this \lq\lq function" the parameter \verb!cout! references
\verb!std::cout! and the reference variable \verb!r! references the \verb!r! in
\verb!main()!.

\newpage
\textsc{Compiling and Automating Test Inputs}

The following instructions are for those using Fedora.

Be neat! Have a directory just for this program and do not have your source
files cluttered with other things! To compile all your source files you do
\begin{console}
g++ *.cpp -o prog
\end{console}

This will take all your cpp files in the current directory and compile an
executable named \verb!prog!. (If you don't like \verb!prog! you can
choose any other name.)

You can automate the testing by entering test data into a file, say you called
it \verb!stdin.txt! (you can use any name you like):
\begin{console}
0 1 2
0 -1 2
0 4 7
0 15 10
\end{console}

To send the input from \verb!stdin.txt! to your program (instead of wasting
time typing the test data again and again), do this:
\begin{console}
./prog < stdin.txt
\end{console}

You can also send the output not to your terminal window but to a file like
this:
\begin{console}
./prog < stdin.txt > stdout.txt
\end{console}

You can now open \verb!stdout.txt! to see the output.

Yet another level of software testing automation is to figure out the correct
output and type the correct output to another file, say \verb!correct.txt!. You
can then get Fedora to test if \verb!stdout.txt! is the same as
\verb!correct.txt! using this command:
\begin{console}
diff correct.txt stdout.txt
\end{console}
If both files are the same, nothing is printed to your terminal window.

Hence your code--test cycle involves the following linux commands:
\begin{console}
g++ *.cpp -o prog
./prog < stdin.txt > stdout.txt
diff correct.txt stdout.txt
\end{console}

(Of course you have to continually add test inputs to \verb!stdin.txt! and the
correct output to \verb!correct.txt!.)




\newpage
\textsc{Operators}

Operators are easy to understand. They are just functions except that you call
them in a different way. Let me give you an example. First run the following
example:
\begin{Verbatim}[frame=single]
#include <iostream>


struct Blah
{
    int x;
    int y;
};


double funnyOperator(const Blah & i, const Blah & j)
{
    double d;
    d = (double)i.x / i.y + (double)j.x / j.y;
    return d;
}

  
int main()
{
    Blah b = {2, 3};
    Blah c = {5, 9};
    double d = funnyOperator(b, c);
    std::cout << d << std::endl;
    return 0;
}
\end{Verbatim}

(The actual body of the function is not important. Focus on how the function
is called.) There are no surprises here. Now try this:
\begin{Verbatim}[frame=single, commandchars=~!@]
#include <iostream>


struct Blah
{
    int x;
    int y;
};


double ~textbf!~underline!operator+@@(const Blah & i, const Blah & j)
{
    double d;
    d = (double) i.x / i.y + (double) j.x / j.y;
    return d;
}


int main()
{
    Blah b = {2, 3};
    Blah c = {5, 9};
    double d = ~textbf!~underline!operator+@@(b, c);
    std::cout << d << std::endl;
    return 0;
}
\end{Verbatim}
The name of the function, \verb!funnyOperator! is changed to \verb!operator+!.

And finally run this program:
\begin{Verbatim}[frame=single, commandchars=~!@]
#include <iostream>


struct Blah
{
    int x;
    int y;
};


double operator+(const Blah & i, const Blah & j)
{
    double d;
    d = (double) i.x / i.y + (double) j.x / j.y;
    return d;
}

  
int main()
{
    Blah b = {2, 3};
    Blah c = {5, 9};
    double d = ~textbf!~underline!b + c@@;
    std::cout << d << std::endl;
    return 0;
}
\end{Verbatim}

As you can see, in the above program:
\begin{console}
b + c
\end{console}
is really a call to the \lq\lq function" \verb!operator+!:
\begin{console}
 operator+(b, c)
\end{console}
except that the \lq\lq function" \verb!operator+!, in this context, is really
called an \textbf{\large operator}. Once again, the following are actually the
same:
\begin{Verbatim}
                       b + c            operator+(b, c)
\end{Verbatim}

You can define all kinds of operators in C\texttt{++}, including
\begin{console}
 >>      <<      +      -      *      /
\end{console}
etc.

A simple \verb!operator<<! is already defined for you which is why you can
print a \verb!Fraction! variable \verb!r! with
\begin{console}
std::cout << r;
\end{console}


\newpage Q1. \subimport*{../a04q01/question/}{main.tex}
\newpage Q2. \subimport*{../a04q02/question/}{main.tex}
\newpage Q3. \subimport*{../a04q03/question/}{main.tex}
\newpage Q4. \subimport*{../a04q04/question/}{main.tex}
\newpage Q5. \subimport*{../a04q05/question/}{main.tex}
\newpage Q6. \subimport*{../a04q06/question/}{main.tex}
\newpage Q7. \subimport*{../a04q07/question/}{main.tex}
\newpage Q8. \subimport*{../a04q08/question/}{main.tex}


\newpage
\textsc{Initializer, Get and Set functions}

An important principle in Software Engineering is \lq\lq \textbf{information hiding}".

The concept of numerator and denominator is now contained in the concept of
\verb!Fraction!. As much as possible, we want to prevent users from directly
accessing the \verb!n! and \verb!d! member variables. Why? If users of our
library minimize their access to member variables, then we can change the way
we design the concept. For instance, suppose that instead of using two integers
as member variables, suppose I know that the numerators and denominators of all
the fractions I care about are positive and less than $1000$. In that case,
instead of modeling the fraction $\frac{2}{3}$ with
\begin{console}
n = 2, d = 3
\end{console}
I can model it with \textbf{\underline{ONE}} single integer
\begin{console}
nd = 0002003
\end{console}
i.e., I use the lower--order $3$ digits of a single integer to model the
denominator and the higher--order $3$ digits to model the numerator. The
highest--order digit models the sign. This saves me $50\%$ of memory for every
\verb!Fraction! variable. In this case, my \verb!Fraction struct! would look like
\begin{console}
struct Fraction
{
    int nd;
};
\end{console}
and I have to change the body of the operators and functions accordingly.

If users do NOT touch member variables, their code need NOT change a single bit
if I made the change to this new \verb!struct!. This technique therefore allows
software engineers to work independently without stepping on each other's toes.
It's a technique for containing software change.

But for me to have this freedom, I do NOT want users of my
\verb!Fraction struct! to touch the member variables. Therefore, I must provide
ways to read and write/modify the numerator and denominator of \verb!Fraction!
variables without accessing them directly. The following functions are
therefore needed:
\begin{Verbatim}[frame=single]
Fraction get_Fraction(int n, int d); // Used to create a Fraction
                                     // with the given n and d
int get_n(const Fraction & f);       // Return copy of the numerator
int get_d(const Fraction & f);       // Return copy of the denominator
void set_n(Fraction & r, int new_n); // Set r.n to new_n
void set_d(Fraction & r, int new_d); // Set r.d to new_d
void set(Fraction & r, int new_n, int new_d); // Sets both n and d of r
                                              // with the new values
\end{Verbatim}

(Why are some \lq\lq pass by reference" while others \lq\lq pass by constant
reference"?)

To make these functions concrete to you, here are examples of how these
functions can be used:
\begin{Verbatim}[frame=single]
Fraction f = get_Fraction(7, 5); // f.n is 7 and f.d is 5

std::cout << get_n(f) << '\n'; // 7 is printed
std::cout << get_d(f) << '\n'; // 5 is printed

set_n(f, 100);                  // f.n becomes 100
std::cout << get_n(f) << '\n'; // 100 is printed
std::cout << get_d(f) << '\n'; // 5 is printed

set_d(f, -23);                  // f.d becomes -23
std::cout << get_n(f) << '\n'; // 100 is printed
std::cout << get_d(f) << '\n'; // -23 is printed

set(f, 22, 7);                  // f.n becomes 22 and f.d becomes 7
std::cout << get_n(f) << '\n'; // 22 is printed
std::cout << get_d(f) << '\n'; // 7 is printed
\end{Verbatim}

Once this is done, you should look at your test code and make sure it minimizes
access to member variables. For instance, this line in
\verb!test_fraction.cpp!:
\begin{Verbatim}[frame=single]
...
           Fraction r = {n, d};
...
\end{Verbatim}
can be changed to
\begin{Verbatim}[frame=single]
...
           Fraction r = get_Fraction(n, d);
...
\end{Verbatim}

This will allow your \verb!get_Fraction()! to set the numbers in whatever way
you like, instead of exposing the fact that it's a struct with two values.

By the way, functions for initializing, getting, and setting values in a
\verb!struct! variable are usually developed first.

Note that there is nothing to prevent the users from accessing the \verb!n!
and \verb!d! member variables. You can tell them to use your initialier,
getter, and setter functions, but they can still go behind your back. Later,
when we talk about classes and objects, we'll see how to strictly enforce
information hiding to protect the internal representation of objects.

\underline{\textit{\textbf{Don't simplify}}}
the fraction for any of these functions. If you get a
$\frac{2}{-0}$, store it as a \verb!Fraction! with \verb!n = 2! and
\verb!d = -0!.

\newpage Q9. \subimport*{../a04q09/question/}{main.tex}
\newpage Q10. \subimport*{../a04q10/question/}{main.tex}
\newpage Q11. \subimport*{../a04q11/question/}{main.tex}
\newpage Q12. \subimport*{../a04q12/question/}{main.tex}
\newpage Q13. \subimport*{../a04q13/question/}{main.tex}
\newpage Q14. \subimport*{../a04q14/question/}{main.tex}

\newpage
\textsc{Type conversion functions}

Frequently types don't work alone. You usually want to convert data from one
form to another. The most obvious features we want for our \verb!Fraction!
library is to work with integers and doubles.

Here are some of the ways we want our functions to work:
\begin{Verbatim}[frame=single]
Fraction f = get_Fraction(1, 2);

double d = get_double(f);             // d has value 0.5
int i = get_int(get_Fraction(12, 5)); // i has value 2, essentially the 
                                      // int value of 12.0/5
\end{Verbatim}

What are the appropriate prototypes for the \verb!get_double()! and
\verb!get_int()! functions?

\newpage Q15. \subimport*{../a04q15/question/}{main.tex}
\newpage Q16. \subimport*{../a04q16/question/}{main.tex}
\newpage Q17. \subimport*{../a04q17/question/}{main.tex}
\newpage Q18. \subimport*{../a04q18/question/}{main.tex}
\newpage Q19. \subimport*{../a04q19/question/}{main.tex}
\newpage Q20. \subimport*{../a04q20/question/}{main.tex}


\newpage
\textsc{Coda and miscellaneous notes}

You're done with the assignment.
Here are some comments.
(You do not need to implement any of the features mentioned below.)

The above implies that you can, for instance,
given a polynomial with fractional coefficients,
you
now find the exact fractional roots in a range and
up to an accuracy of, say 1/100.
For instance to find fractional roots in [0, 100]
of  
\[
\frac{1}{3} x^3 - \frac{5}{3} x + \frac{7}{42} = 0
\]
you do
\begin{Verbatim}[frame=single,fontsize=\small,commandchars=\~\!\@]
Fraction start = get_Fraction(0, 1);
Fraction end = get_Fraction(100, 1);
Fraction step = get_Fraction(1, 100);

Fraction a = get_Fraction(1, 3);
Fraction b = get_Fraction(3, 3);
Fraction c = get_Fraction(7, 42);
Fraction zero = get_Fraction(0, 1);

for (Fraction x = start; x <= end; x += step)
{
    reduce(~textred!x@);
    if (a * x * x * x - b * x + c == zero)
    {
        std::cout << x << std::endl;
    }
}
\end{Verbatim}

Note that there are several convenient features which are
still missing in your \verb!Fraction! library.

You do not have mixed mode operations.
For instance you cannot perform addition of a fraction and an int:
\begin{console}
Fraction f = get_Fraction(1, 3);
f = f + 2; // NOT POSSIBLE
\end{console}
Right now, we add two \verb!Fraction! values.
We have not taught our library
how to add a \verb!Fraction! with an \verb!int! or an \verb!int! with a
\verb!Fraction!.
To do this, 
you need to implement the following:
\begin{console}
Fraction operator+(const Fraction &, int);
\end{console}
This also appears in the above:
\begin{console}
...
    if (a * x * x * x - b * x + c == zero)
...
\end{console}
You do not have \verb!operator==! that compares a \verb!Fraction! value with an
int.
If you do have
\begin{console}
bool operator==(const Fraction &, int);
\end{console}
then the above boolean expression simplies to
\begin{console}
...
    if (a * x * x * x - b * x + c == 0)
...
\end{console}

Etc.

The issue of converting a \verb!double! to a \verb!Fraction! also requires more
thought. For instance, the \verb!double! $0.333333333$ is probably meant to
be the fraction $\frac{1}{3}$. However the type conversion function from
\verb!double! to \verb!Fraction! might be written so that the fraction is
$\frac{333333333}{11000000000}$.

\end{document}
