%-*-latex-*-
\verb!operator+!

Define \verb!operator+! so that when \verb!operator+! is called with two
\verb!Fraction! variables, the return value is a \verb!Fraction! value
that models the addition of two fractions in the \lq\lq real" world.
Mathematically, this is how you add two fractions:
\[
  \frac{a}{b} + \frac{c}{d} = \frac{ad + bc}{bd}
\]

Note that \verb!operator+! accepts two \verb!Fraction! values and returns a
\verb!Fraction! value. Therefore, you have to add this to the \verb!Fraction!
header file:
\begin{Verbatim}[frame=single]
// Author: 
// Date  : 
// File  : Fraction.h

#ifndef FRACTION_H
#define FRACTION_H

struct Fraction
{
    int n; // numerator
    int d; // denominator
};

std::ostream & operator<<(std::ostream &, const Fraction &);
Fraction operator+(const Fraction &, const Fraction &);

#endif
\end{Verbatim}

And the following test code to your \verb!main()!:
{\small
\begin{Verbatim}[frame=single]
// Author: 
// Date  : 
// File  : test_fraction.cpp

#include <iostream>
#include "Fraction.h"


void test_print()
{
    int n = 0, d = 0;
    std::cin >> n >> d;
    Fraction f = {n, d};
    std::cout << f << std::endl;
}


void test_add()
{
    int n0, d0, n1, d1;
    std::cin >> n0 >> d0 >> n1 >> d1;  
    Fraction r0 = {n0, d0};
    Fraction r1 = {n1, d1};
    std::cout << (r0 + r1) << std::endl;
}


int main()
{
    int option;
    std::cin >> option;
    
    switch (option)
    {
        case 1:
            test_print();
            break;
        case 2:
            test_add();
            break;
    }

    return 0; 
}
\end{Verbatim}
}

Here are a couple of test cases for you:

\resett
\nextt
\begin{Verbatim}[frame=single, commandchars=\\\{\}]
\userinput{2 1 3 2 4}
5/6
\end{Verbatim}

\nextt
\begin{Verbatim}[frame=single, commandchars=\\\{\}]
\userinput{2 5 7 -2 13}
51/91
\end{Verbatim}

Add more test cases to your \verb!stdin.txt!.

\newpage
\textsc{Hint: Spoilers ahead ... read only if you really need it ...}

Mathematically this is how we add fractions:
\[
  \frac{a}{b} + \frac{c}{d} = \frac{ad + bc}{bd}
\]

If \verb!x! and \verb!y! are the parameters in the body of \verb!operator+!,
then conceptually in terms of the members of \verb!x! and \verb!y!, we have:

\verb!                  x.n   y.n    (x.n) * (y.d) + (x.d) * (y.n)!\\
\verb!         x + y = ! ${\hspace{0.3in}}\over{\hspace{0.3in}}$ \verb!+!
${\hspace{0.3in}}\over{\hspace{0.3in}}$\verb! = !${\hspace{2.5in}}\over{\hspace{2.5in}}$\\
\verb!                  x.d   y.d           (x.d) * (y.d)!

The operator looks like this:
\begin{console}
Fraction operator+(const Fraction & x, const Fraction & y)
{
    Fraction sum;
    return sum;
}
\end{console}

Of course you need to set the numerator and denominator of the value to return:
\begin{console}
Fraction operator+(const Fraction & x, const Fraction & y)
{
    Fraction sum = {???, ???};
    return sum;
}
\end{console}
