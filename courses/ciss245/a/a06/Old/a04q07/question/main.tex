\texttt{\bf operator!=}

Now add \verb~operator!=~ to your code. Note that if \verb!x! and \verb!y! are
\verb!Fraction! values, then
\begin{console}
x != y
\end{console}
is the same as
\begin{console}
operator!=(x, y)
\end{console}

What is the prototype of this operator?

It's very important to note the following: Instead of implementing
\verb~operator!=~ from scratch, your \verb~operator!=~ MUST use
\verb!operator==!, i.e., \verb~operator!=~ only needs to call
\verb!operator==!.

Add appropriate code to your \verb!Fraction.h! and \verb!main()! as needed.
This is test option 7.

\textbf{Test 1:}
\begin{Verbatim}[frame=single, commandchars=\\\{\}]
\underline{7 1 3 2 6}
0
\end{Verbatim}

Add more test cases to your \verb!stdin.txt!.

[HINT: \verb~operator!=~ is just the \lq\lq opposite" of \verb!operator==!.]
