
\textsc{Initializer, Get and Set functions}

An important principle in Software Engineering is \lq\lq \textbf{information hiding}".

The concept of numerator and denominator is now contained in the concept of
\verb!Fraction!. As much as possible, we want to prevent users from directly
accessing the \verb!n! and \verb!d! member variables. Why? If users of our
library minimize their access to member variables, then we can change the way
we design the concept. For instance, suppose that instead of using two integers
as member variables, suppose I know that the numerators and denominators of all
the fractions I care about are positive and less than $1000$. In that case,
instead of modeling the fraction $\frac{2}{3}$ with
\begin{console}
n = 2, d = 3
\end{console}
I can model it with \textbf{\underline{ONE}} single integer
\begin{console}
nd = 0002003
\end{console}
i.e., I use the lower--order $3$ digits of a single integer to model the
denominator and the higher--order $3$ digits to model the numerator. The
highest--order digit models the sign. This saves me $50\%$ of memory for every
\verb!Fraction! variable. In this case, my \verb!Fraction struct! would look like
\begin{console}
struct Fraction
{
    int nd;
};
\end{console}
and I have to change the body of the operators and functions accordingly.

If users do NOT touch member variables, their code need NOT change a single bit
if I made the change to this new \verb!struct!. This technique therefore allows
software engineers to work independently without stepping on each other's toes.
It's a technique for containing software change.

But for me to have this freedom, I do NOT want users of my
\verb!Fraction struct! to touch the member variables. Therefore, I must provide
ways to read and write/modify the numerator and denominator of \verb!Fraction!
variables without accessing them directly. The following functions are
therefore needed:
\begin{Verbatim}[frame=single]
Fraction get_Fraction(int n, int d); // Used to create a Fraction
                                     // with the given n and d
int get_n(const Fraction & f);       // Return copy of the numerator
int get_d(const Fraction & f);       // Return copy of the denominator
void set_n(Fraction & r, int new_n); // Set r.n to new_n
void set_d(Fraction & r, int new_d); // Set r.d to new_d
void set(Fraction & r, int new_n, int new_d); // Sets both n and d of r
                                              // with the new values
\end{Verbatim}

(Why are some \lq\lq pass by reference" while others \lq\lq pass by constant
reference"?)

To make these functions concrete to you, here are examples of how these
functions can be used:
\begin{Verbatim}[frame=single]
Fraction f = get_Fraction(7, 5); // f.n is 7 and f.d is 5

std::cout << get_n(f) << '\n'; // 7 is printed
std::cout << get_d(f) << '\n'; // 5 is printed

set_n(f, 100);                  // f.n becomes 100
std::cout << get_n(f) << '\n'; // 100 is printed
std::cout << get_d(f) << '\n'; // 5 is printed

set_d(f, -23);                  // f.d becomes -23
std::cout << get_n(f) << '\n'; // 100 is printed
std::cout << get_d(f) << '\n'; // -23 is printed

set(f, 22, 7);                  // f.n becomes 22 and f.d becomes 7
std::cout << get_n(f) << '\n'; // 22 is printed
std::cout << get_d(f) << '\n'; // 7 is printed
\end{Verbatim}

Once this is done, you should look at your test code and make sure it minimizes
access to member variables. For instance, this line in
\verb!test_fraction.cpp!:
\begin{Verbatim}[frame=single]
...
           Fraction r = {n, d};
...
\end{Verbatim}
can be changed to
\begin{Verbatim}[frame=single]
...
           Fraction r = get_Fraction(n, d);
...
\end{Verbatim}

This will allow your \verb!get_Fraction()! to set the numbers in whatever way
you like, instead of exposing the fact that it's a struct with two values.

By the way, functions for initializing, getting, and setting values in a
\verb!struct! variable are usually developed first.

Note that there is nothing to prevent the users from accessing the \verb!n!
and \verb!d! member variables. You can tell them to use your initialier,
getter, and setter functions, but they can still go behind your back. Later,
when we talk about classes and objects, we'll see how to strictly enforce
information hiding to protect the internal representation of objects.

\underline{\textit{\textbf{Don't simplify}}}
the fraction for any of these functions. If you get a
$\frac{2}{-0}$, store it as a \verb!Fraction! with \verb!n = 2! and
\verb!d = -0!.
