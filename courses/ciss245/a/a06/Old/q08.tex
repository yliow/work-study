%-*-latex-*-
\verb!reduce()! function

Note that all the above \lq\lq functions" are operators. For this part, you
need to write a function. This function reduces the \verb!Fraction! to its
lowest terms. Mathematically, the fraction
\[
\frac{18}{60}
\]
is not a reduced (or simplified) fraction. That's because $2$ is a factor of
$18$ and $60$:
\[
\frac{18}{60} = \frac{2 \cdot 9}{2 \cdot 30}
\]
If $2$ is removed from the numerator and denominator, we get
\[
\frac{18}{60}
= \frac{2 \cdot 9}{2 \cdot 30}
= \frac{9}{30}
\]
This is still not reduced since $3$ is a factor of $9$ and $30$. Removing $3$
we get
\[
\frac{18}{60}
= \frac{2 \cdot 9}{2 \cdot 30}
= \frac{9}{30}
= \frac{3 \cdot 3}{3 \cdot 10}
= \frac{3}{10}
\]
This fraction, $\frac{3}{10}$ is now reduced: you can't find any integer factor
of both $3$ and $10$ other than $1$ and $-1$.

For this part, you need to implement:
\begin{console}
void reduce(Fraction &);
\end{console}
that reduces the reference parameter so that the fraction is in its reduced
form. For instance,
\begin{console}
Fraction r = {18, 60};
reduce(r);
// at this point, r.n = 3 and r.d = 10
\end{console}

It's not enough to just remove a single or two common divisor: you must remove
\textbf{\underline{all}} common divisors which are greater than $1$. There are
a couple of other points:
\begin{itemize}
  \li The \verb!reduce()! function must also ensure that the denomiator of the
      parameter is positive:
\begin{Verbatim}[frame=single]
Fraction r = {1, -2};
reduce(r);
// at this point, r.n = -1 and r.d = 2

Fraction s = {-1, -2};
reduce(s);
// at this point, s.n = 1 and s.d = 2
\end{Verbatim}
  \li If the numerator is $0$ and the denominator is not $0$, then the
      denominator is set to $1$:
      \begin{Verbatim}[frame=single]
Fraction r = {0, -1500};
reduce(r);
// at this point, r.n = 0 and r.d = 1
      \end{Verbatim}
  \li If the denominator is $0$ (the case where the fraction is not defined),
      the numerator is set to $1$:
      \begin{Verbatim}[frame=single]
Fraction r = {-42,  0};
reduce(r);
// at this point, r.n = 1 and r.d = 0
      \end{Verbatim}
\end{itemize}

Add test code and cases to the different files as necessary.
Here's the test code:
\begin{console}
void test_reduce()
{
    int n, d;
    std::cin >> n >> d;
    Fraction f = {n, d};
    reduce(f);
    std::cout << f << std::endl;
    return;
}
\end{console}
This is test option 8.

\textsc{Test 1}\vspace{-6pt}
\begin{Verbatim}[frame=single, commandchars=\\\{\}]
\userinput{8 18 60}
3/10
\end{Verbatim}

[Hint: If \verb!d! divides \verb!b!, then,
For instance, $2$ divides $18$ and $60$. Therefore,
\[
\frac{a}{b}
= \frac{a/d}{b/d}
\]
For instance, $2$ divides $18$ and $60$. Therefore,
\[
\frac{18}{60}
=
\frac{18/2}{60/2}
=
\frac{18/2}{60/2}
=
\frac{9}{30}
\]

Your code should find the greatest common divisors \verb!g! of the numerator
and the denominator of a fraction, and then divide the numerator and
denominator of this fraction by \verb!g!. For this assignment, you need not
worry about efficiency.]
