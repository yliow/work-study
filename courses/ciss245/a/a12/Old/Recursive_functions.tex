\textsc{CISS240 Review: Recursive Functions}

The Fibonacci sequence is the sequence of integers 1,1,2,3,5,8,13,...
It starts with 1,1
and from there on, a term in the sequence is the sum of the previous two terms.
Mathematically you can think of this as a function \verb!fib()!:

\begin{tightlist}
  \li \verb!fib(0)! = 1 \verb!    ! (this is by definition)
  \li \verb!fib(1)! = 1 \verb!    ! (this is by definition)
  \li \verb!fib(2)! = 2 \verb!    ! (i.e. \verb!fib(1)! + \verb!fib(0)!)
  \li \verb!fib(3)! = 3 \verb!    ! (i.e. \verb!fib(2)! + \verb!fib(1)!)
  \li \verb!fib(4)! = 5 \verb!    ! (i.e. \verb!fib(3)! + \verb!fib(2)!)
  \li \verb!fib(5)! = 8 \verb!    ! (i.e. \verb!fib(4)! + \verb!fib(3)!)
  \li \verb!fib(6)! = 13 \verb!   ! (i.e. \verb!fib(5)! + \verb!fib(4)!)
\end{tightlist}

\defterm {Self-Exercise:}
Complete the following table. You will need the table for your test code:

\begin{python}
from latextool_basic import table
print table([('$0\ $', '$1\ \ \ \ \ \ $'),
             ('$1\ $', '$1\ \ \ \ \ \ $'),
             ('$2\ $', '$2\ \ \ \ \ \ $'),
             ('$3\ $', '$3\ \ \ \ \ \ $'),
             ('$4\ $', '$5\ \ \ \ \ \ $'),
             ('$5\ $', '$8\ \ \ \ \ \ $'),
             ('$6\ $', '$13\ \ \ \ \ $'),
             ('$7\ $', '$21\ \ \ \ \ $'),
             ('$8\ $', '$34\ \ \ \ \ $'),
             ('$9\ $', '$55\ \ \ \ \ $'),
             (10, '$89\ \ \ \ \ $'),
             (11, ''),
             (12, ''),
             (13, ''),
             (14, ''),
             (15, ''),
             (16, ''),
             (17, ''),
             (18, ''),
             (19, ''),
             (20, '')
            ],
            col_headings = ['$n$', '$\ \ \ fib(n)\ \ \ $'])
\end{python}




\newpage
\begin{python}
from latextool_basic import table
print(table([(21, ''),
             (22, ''),
             (23, ''),
             (24, ''),
             (25, ''),
             (26, ''),
             (27, ''),
             (28, ''),
             (29, ''),
             (30, ''),
             (31, ''),
             (32, ''),
             (33, ''),
             (34, ''),
             (35, ''),
             (36, ''),
             (37, ''),
             (38, ''),
             (39, '')
            ],
            col_headings = ['$n$', '$\ \ \ fib(n)\ \ \ $']))
\end{python}

The following function \verb!fib(n)! will return the n-th value of the Fibonacci sequence:


\begin{Verbatim}[frame=single]
int fib(int n)
{
    if (n <= 1)
        return 1;
    else
        return fib(n - 1) + fib(n - 2);
}
\end{Verbatim}

(For simplicity, we will return 1 whenever the argument is negative.)

Test this function by printing out \verb!fib(n)! for a few values of n.
Compare against the table.

