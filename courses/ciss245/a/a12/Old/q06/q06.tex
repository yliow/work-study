\textsc{The Math header and cpp File}

The next step in this assignment is written in two parts to make it easier for your to accomplish the
task.

We now have a class \texttt{Fib5} and there's only one object in it called \texttt{Fib5::fib5}.

In the same manner, suppose you have another mathematical function that you want to write, say the
factorial function. Does that mean we need to write \texttt{Factorial::factorial} if we use the same
technique of a table lookup?

That's cumbersome and a little artificial. Why do I need to say \texttt{Fib5} \textbf{and} \verb!fib5!? Or \texttt{Factorial} \textbf{and}
\verb!factorial!? How silly can this get ... ?!? ...

\begin{Verbatim}[frame=single]
std::cout << Fib5::fib5(5) << "\n";
Fib5::fib5.resize(100);
std::cout << Factorial::factorial(6) << "\n";
std::cout << Log::log(7.4) << "\n";
\end{Verbatim}

Wouldn't it be more natural to say

\begin{Verbatim}[frame=single]
std::cout << fib(5) << "\n";
fib.resize(100);
std::cout << factorial(6) << "\n";
std::cout << log(7.4) << "\n";
\end{Verbatim}

Do the following:
\begin{tightlist}
  \li In your \texttt{Fib5.cpp}, declare a \texttt{Fib5} reference (i.e. \texttt{Fib5 \&}) called \verb!fib! and initialize it to
  \texttt{Fib5::fib5}.
  \li To access \verb!fib! from \texttt{Fib5}, in your cpp containing
  \texttt{main()} you add this statement:
  \texttt{extern Fib5 \& fib;}.
  This tells your compiler that the identifier \verb!fib! is in some cpp file.
  Of course \verb!fib! is really the
  same as \texttt{Fib5::fib5}. At this point your should test that you
  can execute this in your
  \texttt{main()}:
      \verb!std::cout << fib(7) << std::endl;!
      \li To make this even easier for users of your Fibonacci computations as
      well as other
      mathematical functions, you now move your external declarations into
      a header file. Create a
  header file called \texttt{Math.h}. Move your external declaration of \verb!fib! to the file \texttt{Math.h}.
\end{tightlist}

(Now ... keep this header file carefully. Build upon it by including all the useful things you might need.
For instance you can declare a constant \verb!PI! as the double 3.14159365, etc. in your \texttt{Math.cpp} and
create an external declaration for \verb!PI!, etc. in your \texttt{Math.h}.)
