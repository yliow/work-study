\textsc{The Fib5 class: Static Object in Fib5 Class and Private Constructors}

It's annoying that we always have to create a Fibonacci object before using it. Furthermore ... do we
really need to be able to create two Fibonacci objects?

Create a static object called \verb!fib5! in \texttt{Fib5}. Initialize the lookup table with a size of 20. Next, since we
want users to use the static \texttt{Fib5::fib5} object and not create their own objects, ... \mybold{for the first}
\mybold{time} ... make the constructors \textbf{private} (both the default and the copy constructor). In other words
make the following code impossible to compile:

\begin{Verbatim}[frame=single]
Fib5 fib5;                // won't compile ... yeah!
Fib5 fib5(Fib5::fib5);    // STILL won't compile ... :)
\end{Verbatim}

Include a static method to resize the size of the table. Call it \texttt{resize()} (duh). So \texttt{Fib5::fib5(5)}
computes the 5 th Fibonacci number while \texttt{Fib5::resize(100)} resizes the table lookup to a size of
100.

(Note: Although you don't have to do this, you can use your \texttt{IntDynArr} class instead of \verb!int*! for
\verb!table!. You also realize that earlier we made \verb!table! static in order to prevent objects from having
their own \verb!table!. Now that the constructors are private, and there is only one public static \texttt{Fib5}
object, this is not an issue anymore.)


