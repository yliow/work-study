\textsc{The Fib2 Class: Fibonacci Computations with Table Lookup}

In this question you will develop another class for Fibonacci computation. The name of the class is
\texttt{Fib2}.

If a function is used frequently, it makes sense to keep previously computed values. In your next
Fibonacci computation class \texttt{Fib2}, each object should have an array,
called \verb!table!, of 20 integers to
keep computed Fibonacci numbers. You should initialize all the integers in the array to -1, except for

\verb!                 table[0] = 1, table[1] = 1!

Suppose you call \verb!fib2(0)! where \verb!fib2! is a \texttt{Fib2} object.
Then \verb!table! should be used.
(You can also initialize \verb!table! with \verb!0! since the
fibonacci numbers are all at least \verb!1!.)

However suppose you call \verb!fib2(2)!. You check that \verb!table[2]! is -1 which means that this is the
first time \verb!fib2(2)! is being computed. So you use the formula \verb!fib2(1) + fib2(2)!. This will call
\verb!fib2(1)! which will return 1 and \verb!fib2(0)! which will return 1. So you get 2 for \verb!fib2(1) +!
\verb!fib2(0)!. Before you return this value, you should set \verb!table[2]! to 2.

In general each time \verb!fib2(n)! is executed, the table should be used as a lookup table for previously
computed values. Once \verb!fib2(n)! is known, it should be kept in the object at \verb!table[n]!, if \verb!table[n]!
is still -1.

So the pseudocode looks like this:

\begin{Verbatim}[frame=single]
if n is negative return 1
if table[n] is -1, set table[n] to fib(n-1) + fib(n-2)
return table[n]
\end{Verbatim}

But wait. Note that the \verb!table!
has maximum size of 20 (so the maximum lookup is \verb!table[19]!, i.e.,
\verb!fib(19)!). So for the computation of \verb!fib(25)! for instance you should not refer to the table when you
need \verb!fib(24)! and \verb!fib(23)!. You should compute using the above formula, i.e. compute \verb!fib(24)!
\verb!+ fib(23)!. The pseudocode looks like this:

\begin{Verbatim}[frame=single]
if n is negative return 1
else if n < 20
    // use the table
    if table[n] is -1
        set table[n] to fib(n - 1) + fib(n - 2)
    return table[n]
else
    // do not use the table
    return fib(n - 1) + fib(n - 2)
\end{Verbatim}

For testing purposes, you must also include \texttt{operator[]} so that if \verb!fib2! is a \texttt{Fib2} object, then \verb!fib2[4]!
returns the value of \verb!fib2.table[4]!.

Test your new and improved \texttt{Fib2}:

\begin{Verbatim}[frame=single]
#include <iostream>
#include "Fib1.h"
#include "Fib2.h"

int main()
{
    int correct[] = {1, 1, 2, 3, 5, 8, 13, 21, 34, 55};

    ... test code for Fib1 ...

    std::cout << "Testing Fib2:" << std::endl;
    Fib2 fib2;
    std::cout << "Test 1 " << (fib2[2] == -1 ? "pass" : "FAIL")
              << std::endl;

    for (int i = 0; i < 10; ++i)
    {
        std::cout << "Test " << i + 2 << ' '
                  << (fib2(i) == correct[i] ? "pass" : "FAIL")
                  << std::endl;
    }

    std::cout << "Test 12 " << (fib2[2] == 2 ? "pass" : "FAIL")
              << std::endl;

    return 0;
}
\end{Verbatim}

To understand the improvement made in \texttt{Fib2}, perform the following experiment. Print \verb!fib1(i)! for i
= 0, ..., 39 and then print \verb!fib2(i)! for i = 0, ..., 39. The second object should compute a lot faster.


\end{Verbatim}


