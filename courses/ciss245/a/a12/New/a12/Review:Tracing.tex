\textsc{CISS240 Review: Tracing a Recursive Function}

How does our C/C++ function work? Let's do an example. Suppose you call \verb!fib(3)!:

\verb!    std::cout << fib(3);!

The execution of \verb!fib(3)! will call \verb!fib(3 - 1)! i.e. \verb!fib(2)! because of

\begin{Verbatim}[frame=single, commandchars=\\\{\}]
    running fib(3):
        ...
        returning \mybold{fib(n - 1)} + fib(n - 2);
\end{Verbatim}

Now the execution of \verb!fib(2)! will call \verb!fib(2 - 1)! i.e. \verb!fib(1)! because of

\begin{Verbatim}[frame=single, commandchars=\\\{\}]
    running fib(2):
        ...
        returning \mybold{fib(n - 1)} + fib(n - 2);
\end{Verbatim}

The execution of \verb!fib(1)! will execute this:

\begin{Verbatim}[frame=single, commandchars=\\\{\}]
    running fib(1):
        ...
        \mybold{return 1};
\end{Verbatim}

On returning to \verb!fib(2)! we have the value of \verb!fib(n - 1)!, i.e. 1, and we still
have to call \verb!fib(n - 2)! i.e. \verb!fib(0)!

\begin{Verbatim}[frame=single, commandchars=\\\{\}]
    running fib(2):
        ...
        returning fib(n - 1) + \mybold{fib(n - 2)};
\end{Verbatim}

This call will execute this:

\begin{Verbatim}[frame=single, commandchars=\\\{\}]
    running fib(0):
        ...
        \mybold{return 1};
\end{Verbatim}



\newpage
Returning to the execution of \verb!fib(2)! with \verb!fib(n - 1)! = 1 and \verb!fib(n - 2)! = 1
we can now execute the addition in


\begin{Verbatim}[frame=single, commandchars=\\\{\}]
    running fib(2):
        ... 
        returning \mybold{fib(n - 1) + fib(n - 2)};
      \end{Verbatim}
      
which will return 2 for the call for \verb!fib(n - 1)! during the execution of \verb!fib(3)! (did
you even remember where that was??)

\begin{Verbatim}[frame=single, commandchars=\\\{\}]
     running fib(3):
        ...
        returning fib(n - 1) + \mybold{fib(n - 2)};
\end{Verbatim}

which will result in the call to

\begin{Verbatim}[frame=single, commandchars=\\\{\}]
    running fib(1):
        ...
        \mybold{return 1};
\end{Verbatim}

which returns 1 to \verb!fib(n - 2)! in the execution of \verb!fib(3)!. Now that all calls have
returned (\verb!fib(n - 1)! returned 2 and \verb!fib(n - 2)! returned 1), we can execute the addition in

\begin{Verbatim}[frame=single, commandchars=\\\{\}]
    running fib(3):
        ...
        return \mybold{fib(n - 1) + fib(n - 2)};
\end{Verbatim}

which will return 3.


\textbf{Phew!!} Basically you can see the calls if you use mathematical equations:
{\small
\begin{Verbatim}
fib(3):
  
= fib(2) + fib(1)           fib(3) calls fib(2) and fib(1)
= fib(1) + fib(0) + fib(1)  fib(2) calls fib(1) and fib(0)
= 1 + 1 + fib(1)            fib(1) returns 1 to fib(2), fib(0) returns 1 to fib(2)
= 2 + fib(1)                fib(2) returns 2 to fib(3), fib(3) calls fib(1)
= 2 + 1                     fib(1) returns 1 to fib(2)
= 3                         fib(3) returns 3
      \end{Verbatim}
      }



\newpage
Of course if I'm not interested in keeping track of the number of function
calls and I'm only interested in
the result I can write

\begin{Verbatim}
fib(3)    = fib(2) + fib(1)
          = fib(1) + fib(0) + 1
          = 1 + 1 + 1
          = 3
\end{Verbatim}
        
There ... I've explained it twice. Now I want to do it \textbf{again} with a different picture. The arrows denote
function calls. The number above the arrows show you the value returned:

\begin{Verbatim}
             2                1
fib(3) ----+---> fib(2) ----+---> fib(1)
           |                | 1
           | 1              +---> fib(0)
           +---> fib(1)
\end{Verbatim}
         
There you go ... three explanations. Note that there were two (repeated) function calls of \verb!fib(1)!:

\begin{Verbatim}
             3                2                1
fib(4) ----+---> fib(3) ----+---> fib(2) ----+---> fib(1)
           |                |                | 1
           |                | 1              +---> fib(0)
           |                +---> fib(1)
           |
           | 2                1
           +---> fib(2) ----+---> fib(1)
                            | 1
                            +---> fib(0)
\end{Verbatim}

Obviously \verb!fib(4)! returns 3 + 2 = 5. Now in this case note that \verb!fib(4)! makes the following calls
(directly or indirectly):

\begin{tightlist}
    \li \verb!fib(3)! -- 1 call
    \li \verb!fib(2)! -- 2 calls
    \li \verb!fib(1)! -- 3 calls
    \li \verb!fib(0)! -- 2 calls
\end{tightlist}



\newpage
Hey this is fun! (For me ...). Let's do \verb!fib(5)!. I know that
\verb!fib(5)! will call

\begin{Verbatim}
             3                2                1
fib(4) ----+---> fib(3) ----+---> fib(2) ----+---> fib(1)
           |                |                | 1
           |                | 1              +---> fib(0)
           |                +---> fib(1)
           |
           | 2                1
           +---> fib(2) ----+---> fib(1)
                            | 1
                            +---> fib(0)
\end{Verbatim}           
and
\begin{Verbatim}
             2                1
fib(3) ----+---> fib(2) ----+---> fib(1)
           |                | 1
           | 1              +---> fib(0)
           +---> fib(1)
\end{Verbatim}

So putting them together I get:

\begin{Verbatim}
         4               3                2              1
fib(5)-+--->fib(4) ----+---> fib(3) ----+---> fib(2)---+---> fib(1)
       |               |                |              | 1
       |               |                | 1            +---> fib(0)
       |               |                +---> fib(1)
       |               |
       |               | 2                1
       |               +---> fib(2) ----+---> fib(1)
       |                                | 1
       |                                +---> fib(0)
       |   
       | 3               2                1
       +--->fib(3) ----+---> fib(2) ----+---> fib(1)
                       |                | 1
                       | 1              +---> fib(0)
                       +---> fib(1)
\end{Verbatim}
(Such a diagram is called a function call tree.)


\newpage
The calls from \verb!fib(5)! are:
\begin{tightlist}
  \li \verb!fib(4)! -- 1 call
  \li \verb!fib(3)! -- 2 calls
  \li \verb!fib(2)! -- 3 calls
  \li \verb!fib(1)! -- 5 calls
  \li \verb!fib(0)! -- 3 calls
\end{tightlist}

Obviously there is a lot of repeated calls and it will grow as you increase
with larger \verb!n! for \verb!fib(n)!. These
are small numbers. But how fast do they grow?? Well you can do the math, but
let's get the program
to tell us:

\begin{Verbatim}[frame=single]
int fib(int n)
{
    std::cout << "fib(" << n << ") ...\n";
    if (n <= 1)
        return 1;
    else
        return fib(n - 1) + fib(n - 2);
}
\end{Verbatim}      

Now call \verb!fib(10)!. Since I'm too lazy (or too smart) to count, I'll get
the program to count for me:

\begin{Verbatim}[frame=single]
int count[10]; // global variable, accessible to all functions in this file

void reset()
{
    for (int i = 0; i < 10; ++i)
    {
        count[i] = 0;
    }  
}
  
void print()
{
    for (int i = 0; i < 10; ++i)
    {
        std::cout << count[i] << " ";
    }
}
  
int sum()
{
    int s = 0;
    for (int i = 0; i < 10; ++i)
    {
        s += count[i];
    }
}

int fib(int n)
{
    // std::cout << "fib(" << n << ") ...\n";
    count[n]++;
    if (n <= 1)
    {
        return 1;
    }
    else
    {
        return fib(n-1) + fib(n-2);
    }
}
\end{Verbatim}

with \verb!main()! as:

\begin{Verbatim}[frame=single]
for (int i = 0; i < 10; ++i)
{
    reset();
    fib(i);
    print();
    std::cout << "sum:" << sum() << std::endl;
}
\end{Verbatim}

You see that the number of calls grow very quickly and many of the calls are
actually repeated calls.
How does \verb!fib! perform
\lq\lq in the long run"? Try this for your \verb!main()!:

\begin{Verbatim}[frame=single]
for (int i = 0; i < 100; ++i)
{
    std::cout << fib(i) << "\n";
}
\end{Verbatim}

Are you convinced that the implementation of \verb!fib! is really bad?


