
\newcommand\COURSE{ciss350}
\newcommand\ASSESSMENT{a01}
\newcommand\ASSESSMENTTYPE{Assignment}
\newcommand\POINTS{	extwhite{xxx/xxx}}

\input{myquizpreamble}
\input{yliow}
\renewcommand\TITLE{\ASSESSMENTTYPE \ \ASSESSMENT}

\renewcommand\EMAIL{}
\input{\COURSE}
\textwidth=6in

 

\newcommand\BLANK{\sqcup}
% Used in DFA minimization
\newcommand\ind{\operatorname{index}}


\makeindex
\begin{document}
\topmatter

\newcommand\myincludetex[1]{\textbox{{\scriptsize \texttt{#1}}}

    \input{#1}
}

\newcommand\myincludesrc[1]{\textbox{{\scriptsize \texttt{#1}}}
    
    \VerbatimInput[fontsize=\footnotesize,frame=single]{#1}
}

\textsc{Objectives:}
\begin{enumerate}[topsep=0pt]
\item View problems recursively
\item Write recursive functions
\end{enumerate}


The skeleton code given for each question must be used.
The trace printing in the code cannot be removed.
(Just as in CISS245, the skeleton code is of course not complete
and might have errors. The point is to give you something to begin your
work.)

\newpage \textsc{Function objects and function call operator}

A \textbf{function object} is simply an object that looks like a function, i.e.,
if \verb!a! is a function object, then \verb!a! is an object
(so it's constructed from a class) and you can execute for instance
\verb!a(42)!, i.e.,
you can execute \verb!a.operator()(42)!, i.e., \verb!a! has a method
of the form \verb!operator()(int)!.

You can also choose to make the \verb!operator()! accept two integers, i.e.,
\verb!operator()(int, int)!.
In that case you can execute \verb!a(42, 0)! which will result in
the invocation of \verb!a.operator()(42, 0)!.

In general you can 
\texttt{a.operator()(\textnormal[some prototype])}.
The operator \verb!operator()! is sometimes called the
\textbf{function call operator}.

Of course if your \verb!a! is simply a function (as in a regular
CISS240 function),
then there's no reason to make it an object.
You want to create \verb!a! to be a function object only when \verb!a!
behaves like a function ... and more.

For instance what if you want \verb!a! to return the last value it computed?
Suppose the last function call operator executed was \verb!a(42)! and the
value returned was \verb!999!, and you want to know what was the last computed
value. Then you need to store that \verb!999! in object \verb!a!,
and maybe have a method called \verb!last_result!, so that
you can call \verb!a.last_result()! to return \verb!999!.
Maybe you also want \verb!a.last_input()! which will return \verb!42!.

And what if you want to compute the most frequently returned result?
Then you probably want to keep as many values computed by \verb!a!
as you can, and allow the execution of \verb!a.most_common_result()!.

One common use of function objects is to store computations
to \textit{avoid re-computations}.
This is a very common theme in computer science, especially
when certain computations are expensive (time-wise or memory-wise)
and can result in dramatic computational performance improvement.
The storing of expensive computations to avoid re-computation
is called \textbf{memoization} -- i.e., create a 
\textit{memorandum} or \textit{memo} of
results.
The memorandum is called the \textbf{memo table}
or informally the \textbf{lookup table}.

\newpage \textsc{CISS240 Review: Recursive Functions}

The Fibonacci sequence is the sequence of integers 1,1,2,3,5,8,13,...
It starts with 1,1
and from there on, a term in the sequence is the sum of the previous two terms.
Mathematically you can think of this as a function \verb!fib()!:

\begin{tightlist}
  \li \verb!fib(0)! = 1 \verb!    ! (this is by definition)
  \li \verb!fib(1)! = 1 \verb!    ! (this is by definition)
  \li \verb!fib(2)! = 2 \verb!    ! (i.e. \verb!fib(1)! + \verb!fib(0)!)
  \li \verb!fib(3)! = 3 \verb!    ! (i.e. \verb!fib(2)! + \verb!fib(1)!)
  \li \verb!fib(4)! = 5 \verb!    ! (i.e. \verb!fib(3)! + \verb!fib(2)!)
  \li \verb!fib(5)! = 8 \verb!    ! (i.e. \verb!fib(4)! + \verb!fib(3)!)
  \li \verb!fib(6)! = 13 \verb!   ! (i.e. \verb!fib(5)! + \verb!fib(4)!)
\end{tightlist}

\defterm {Self-Exercise:}
Complete the following table. You will need the table for your test code:

\begin{python}
from latextool_basic import table
print table([('$0\ $', '$1\ \ \ \ \ \ $'),
             ('$1\ $', '$1\ \ \ \ \ \ $'),
             ('$2\ $', '$2\ \ \ \ \ \ $'),
             ('$3\ $', '$3\ \ \ \ \ \ $'),
             ('$4\ $', '$5\ \ \ \ \ \ $'),
             ('$5\ $', '$8\ \ \ \ \ \ $'),
             ('$6\ $', '$13\ \ \ \ \ $'),
             ('$7\ $', '$21\ \ \ \ \ $'),
             ('$8\ $', '$34\ \ \ \ \ $'),
             ('$9\ $', '$55\ \ \ \ \ $'),
             (10, '$89\ \ \ \ \ $'),
             (11, ''),
             (12, ''),
             (13, ''),
             (14, ''),
             (15, ''),
             (16, ''),
             (17, ''),
             (18, ''),
             (19, ''),
             (20, '')
            ],
            col_headings = ['$n$', '$\ \ \ fib(n)\ \ \ $'])
\end{python}




\newpage
\begin{python}
from latextool_basic import table
print(table([(21, ''),
             (22, ''),
             (23, ''),
             (24, ''),
             (25, ''),
             (26, ''),
             (27, ''),
             (28, ''),
             (29, ''),
             (30, ''),
             (31, ''),
             (32, ''),
             (33, ''),
             (34, ''),
             (35, ''),
             (36, ''),
             (37, ''),
             (38, ''),
             (39, '')
            ],
            col_headings = ['$n$', '$\ \ \ fib(n)\ \ \ $']))
\end{python}

The following function \verb!fib(n)! will return the n-th value of the Fibonacci sequence:


\begin{Verbatim}[frame=single]
int fib(int n)
{
    if (n <= 1)
        return 1;
    else
        return fib(n - 1) + fib(n - 2);
}
\end{Verbatim}

(For simplicity, we will return 1 whenever the argument is negative.)

Test this function by printing out \verb!fib(n)! for a few values of n.
Compare against the table.


\newpage \textsc{CISS240 Review: Tracing a Recursive Function}

How does our C/C++ function work? Let's do an example. Suppose you call \verb!fib(3)!:

\verb!    std::cout << fib(3);!

The execution of \verb!fib(3)! will call \verb!fib(3 - 1)! i.e. \verb!fib(2)! because of

\begin{Verbatim}[frame=single, commandchars=\\\{\}]
    running fib(3):
        ...
        returning \mybold{fib(n - 1)} + fib(n - 2);
\end{Verbatim}

Now the execution of \verb!fib(2)! will call \verb!fib(2 - 1)! i.e. \verb!fib(1)! because of

\begin{Verbatim}[frame=single, commandchars=\\\{\}]
    running fib(2):
        ...
        returning \mybold{fib(n - 1)} + fib(n - 2);
\end{Verbatim}

The execution of \verb!fib(1)! will execute this:

\begin{Verbatim}[frame=single, commandchars=\\\{\}]
    running fib(1):
        ...
        \mybold{return 1};
\end{Verbatim}

On returning to \verb!fib(2)! we have the value of \verb!fib(n - 1)!, i.e. 1, and we still
have to call \verb!fib(n - 2)! i.e. \verb!fib(0)!

\begin{Verbatim}[frame=single, commandchars=\\\{\}]
    running fib(2):
        ...
        returning fib(n - 1) + \mybold{fib(n - 2)};
\end{Verbatim}

This call will execute this:

\begin{Verbatim}[frame=single, commandchars=\\\{\}]
    running fib(0):
        ...
        \mybold{return 1};
\end{Verbatim}



\newpage
Returning to the execution of \verb!fib(2)! with \verb!fib(n - 1)! = 1 and \verb!fib(n - 2)! = 1
we can now execute the addition in


\begin{Verbatim}[frame=single, commandchars=\\\{\}]
    running fib(2):
        ... 
        returning \mybold{fib(n - 1) + fib(n - 2)};
      \end{Verbatim}
      
which will return 2 for the call for \verb!fib(n - 1)! during the execution of \verb!fib(3)! (did
you even remember where that was??)

\begin{Verbatim}[frame=single, commandchars=\\\{\}]
     running fib(3):
        ...
        returning fib(n - 1) + \mybold{fib(n - 2)};
\end{Verbatim}

which will result in the call to

\begin{Verbatim}[frame=single, commandchars=\\\{\}]
    running fib(1):
        ...
        \mybold{return 1};
\end{Verbatim}

which returns 1 to \verb!fib(n - 2)! in the execution of \verb!fib(3)!. Now that all calls have
returned (\verb!fib(n - 1)! returned 2 and \verb!fib(n - 2)! returned 1), we can execute the addition in

\begin{Verbatim}[frame=single, commandchars=\\\{\}]
    running fib(3):
        ...
        return \mybold{fib(n - 1) + fib(n - 2)};
\end{Verbatim}

which will return 3.


\textbf{Phew!!} Basically you can see the calls if you use mathematical equations:
{\small
\begin{Verbatim}
fib(3):
  
= fib(2) + fib(1)           fib(3) calls fib(2) and fib(1)
= fib(1) + fib(0) + fib(1)  fib(2) calls fib(1) and fib(0)
= 1 + 1 + fib(1)            fib(1) returns 1 to fib(2), fib(0) returns 1 to fib(2)
= 2 + fib(1)                fib(2) returns 2 to fib(3), fib(3) calls fib(1)
= 2 + 1                     fib(1) returns 1 to fib(2)
= 3                         fib(3) returns 3
      \end{Verbatim}
      }



\newpage
Of course if I'm not interested in keeping track of the number of function
calls and I'm only interested in
the result I can write

\begin{Verbatim}
fib(3)    = fib(2) + fib(1)
          = fib(1) + fib(0) + 1
          = 1 + 1 + 1
          = 3
\end{Verbatim}
        
There ... I've explained it twice. Now I want to do it \textbf{again} with a different picture. The arrows denote
function calls. The number above the arrows show you the value returned:

\begin{Verbatim}
             2                1
fib(3) ----+---> fib(2) ----+---> fib(1)
           |                | 1
           | 1              +---> fib(0)
           +---> fib(1)
\end{Verbatim}
         
There you go ... three explanations. Note that there were two (repeated) function calls of \verb!fib(1)!:

\begin{Verbatim}
             3                2                1
fib(4) ----+---> fib(3) ----+---> fib(2) ----+---> fib(1)
           |                |                | 1
           |                | 1              +---> fib(0)
           |                +---> fib(1)
           |
           | 2                1
           +---> fib(2) ----+---> fib(1)
                            | 1
                            +---> fib(0)
\end{Verbatim}

Obviously \verb!fib(4)! returns 3 + 2 = 5. Now in this case note that \verb!fib(4)! makes the following calls
(directly or indirectly):

\begin{tightlist}
    \li \verb!fib(3)! -- 1 call
    \li \verb!fib(2)! -- 2 calls
    \li \verb!fib(1)! -- 3 calls
    \li \verb!fib(0)! -- 2 calls
\end{tightlist}



\newpage
Hey this is fun! (For me ...). Let's do \verb!fib(5)!. I know that
\verb!fib(5)! will call

\begin{Verbatim}
             3                2                1
fib(4) ----+---> fib(3) ----+---> fib(2) ----+---> fib(1)
           |                |                | 1
           |                | 1              +---> fib(0)
           |                +---> fib(1)
           |
           | 2                1
           +---> fib(2) ----+---> fib(1)
                            | 1
                            +---> fib(0)
\end{Verbatim}           
and
\begin{Verbatim}
             2                1
fib(3) ----+---> fib(2) ----+---> fib(1)
           |                | 1
           | 1              +---> fib(0)
           +---> fib(1)
\end{Verbatim}

So putting them together I get:

\begin{Verbatim}
         4               3                2              1
fib(5)-+--->fib(4) ----+---> fib(3) ----+---> fib(2)---+---> fib(1)
       |               |                |              | 1
       |               |                | 1            +---> fib(0)
       |               |                +---> fib(1)
       |               |
       |               | 2                1
       |               +---> fib(2) ----+---> fib(1)
       |                                | 1
       |                                +---> fib(0)
       |   
       | 3               2                1
       +--->fib(3) ----+---> fib(2) ----+---> fib(1)
                       |                | 1
                       | 1              +---> fib(0)
                       +---> fib(1)
\end{Verbatim}
(Such a diagram is called a function call tree.)


\newpage
The calls from \verb!fib(5)! are:
\begin{tightlist}
  \li \verb!fib(4)! -- 1 call
  \li \verb!fib(3)! -- 2 calls
  \li \verb!fib(2)! -- 3 calls
  \li \verb!fib(1)! -- 5 calls
  \li \verb!fib(0)! -- 3 calls
\end{tightlist}

Obviously there is a lot of repeated calls and it will grow as you increase
with larger \verb!n! for \verb!fib(n)!. These
are small numbers. But how fast do they grow?? Well you can do the math, but
let's get the program
to tell us:

\begin{Verbatim}[frame=single]
int fib(int n)
{
    std::cout << "fib(" << n << ") ...\n";
    if (n <= 1)
        return 1;
    else
        return fib(n - 1) + fib(n - 2);
}
\end{Verbatim}      

Now call \verb!fib(10)!. Since I'm too lazy (or too smart) to count, I'll get
the program to count for me:

\begin{Verbatim}[frame=single]
int count[10]; // global variable, accessible to all functions in this file

void reset()
{
    for (int i = 0; i < 10; ++i)
    {
        count[i] = 0;
    }  
}
  
void print()
{
    for (int i = 0; i < 10; ++i)
    {
        std::cout << count[i] << " ";
    }
}
  
int sum()
{
    int s = 0;
    for (int i = 0; i < 10; ++i)
    {
        s += count[i];
    }
}

int fib(int n)
{
    // std::cout << "fib(" << n << ") ...\n";
    count[n]++;
    if (n <= 1)
    {
        return 1;
    }
    else
    {
        return fib(n-1) + fib(n-2);
    }
}
\end{Verbatim}

with \verb!main()! as:

\begin{Verbatim}[frame=single]
for (int i = 0; i < 10; ++i)
{
    reset();
    fib(i);
    print();
    std::cout << "sum:" << sum() << std::endl;
}
\end{Verbatim}

You see that the number of calls grow very quickly and many of the calls are
actually repeated calls.
How does \verb!fib! perform
\lq\lq in the long run"? Try this for your \verb!main()!:

\begin{Verbatim}[frame=single]
for (int i = 0; i < 100; ++i)
{
    std::cout << fib(i) << "\n";
}
\end{Verbatim}

Are you convinced that the implementation of \verb!fib! is really bad?



\newpage Q1. \subimport*{a12q01/doc/}{q01.tex}
\newpage \newpage Q2. \subimport*{a12q02/doc/}{q02.tex}
\myincludesrc{a12q02/skel/main.cpp}
To understand the improvement made in \texttt{Fib2}, perform the following experiment. Print \verb!fib1(i)! for i
= 0, ..., 39 and then print \verb!fib2(i)! for i = 0, ..., 39. The second object should compute a lot faster.
\newpage
 Q3. \subimport*{a12q03/doc/}{q03.tex}
\myincludesrc{a12q03/skel/main.cpp}
\newpage Q4. \subimport*{a12q04/doc/}{q04.tex}
\myincludesrc{a12q04/skel/main.cpp}
Q5. \subimport*{a12q05/doc/}{q05.tex}
\newpage \myincludesrc{a12q05/skel/main.cpp}
Q6. \subimport*{a12q06/doc/}{q06.tex}
\myincludesrc{a12q06/skel/main.cpp}
Note that written in this fashion, the user does not need to know anything
about any Fibonacci
computation class. All of that is taken care of by \texttt{Math.h}.
 
\printindex
\end{document}
    