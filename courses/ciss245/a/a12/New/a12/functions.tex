\textsc{Function objects and function call operator}

A \textbf{function object} is simply an object that looks like a function, i.e.,
if \verb!a! is a function object, then \verb!a! is an object
(so it's constructed from a class) and you can execute for instance
\verb!a(42)!, i.e.,
you can execute \verb!a.operator()(42)!, i.e., \verb!a! has a method
of the form \verb!operator()(int)!.

You can also choose to make the \verb!operator()! accept two integers, i.e.,
\verb!operator()(int, int)!.
In that case you can execute \verb!a(42, 0)! which will result in
the invocation of \verb!a.operator()(42, 0)!.

In general you can 
\texttt{a.operator()(\textnormal[some prototype])}.
The operator \verb!operator()! is sometimes called the
\textbf{function call operator}.

Of course if your \verb!a! is simply a function (as in a regular
CISS240 function),
then there's no reason to make it an object.
You want to create \verb!a! to be a function object only when \verb!a!
behaves like a function ... and more.

For instance what if you want \verb!a! to return the last value it computed?
Suppose the last function call operator executed was \verb!a(42)! and the
value returned was \verb!999!, and you want to know what was the last computed
value. Then you need to store that \verb!999! in object \verb!a!,
and maybe have a method called \verb!last_result!, so that
you can call \verb!a.last_result()! to return \verb!999!.
Maybe you also want \verb!a.last_input()! which will return \verb!42!.

And what if you want to compute the most frequently returned result?
Then you probably want to keep as many values computed by \verb!a!
as you can, and allow the execution of \verb!a.most_common_result()!.

One common use of function objects is to store computations
to \textit{avoid re-computations}.
This is a very common theme in computer science, especially
when certain computations are expensive (time-wise or memory-wise)
and can result in dramatic computational performance improvement.
The storing of expensive computations to avoid re-computation
is called \textbf{memoization} -- i.e., create a 
\textit{memorandum} or \textit{memo} of
results.
The memorandum is called the \textbf{memo table}
or informally the \textbf{lookup table}.
