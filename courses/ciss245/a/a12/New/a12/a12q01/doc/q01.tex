\textsc{The Fib1 Class: \textnormal{\texttt{operator()}} and Fibonacci
Computations}

The aim is to create a class whose objects behave like functions. In particular when we run the
following program:

\begin{Verbatim}[frame=single]
#include <iostream>
#include "Fib1.h"

int main()
{
    Fib1 fib1;
    for (int i = 0; i < 6; ++i)
    {
        std::cout << fib1(i) << std::endl;
    }

    return 0;
}
\end{Verbatim}
we have the same output as before.

Check your notes and textbook on \verb!operator()!. If \verb! obj ! is an
object, then

\verb!            obj(x)      ! is the same as \verb!    obj.operator()(x)!
\\
\verb!            obj(x,y)    ! is the same as \verb!    obj.operator()(x,y)!
\\
\verb!            obj(x,y,z)  ! is the same as \verb!    obj.operator()(x,y,z)!


The following test code must be included:

\myincludesrc{a12q01/skel/main.cpp}

Test the speed of your \texttt{Fib1} class by printing out the Fibonacci numbers for i = 49. Is it fast???
(Yeah ... ) How much time did it take? (No, don't answer). By the way, since an int is probably 32-bit,
you will have an overflow and get a negative integer. Don't worry about that.


