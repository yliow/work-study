\textsc{The STL Vector Class}

Although we already have an \verb!IntDynArr! class that can be used 
to model our \verb!LongInt! class, we will use the STL (standard 
template library) \verb!vector! class. Note that although C++ 
compilers comes with the \verb!vector! class which you can use right 
away, it is still important to implement your version of the 
\verb!vector! class (which we call \verb!IntDynArr!). Only then will 
you understand the performance differences between “container” classes  
for storing values -- the \verb!vector! class is not the only class used 
for storing values. One class might be faster in a certain scenario while 
another might be better in a different scenario. Also in the real world, 
there are situations where you have to modify the classes that come with 
your C++ compiler. This means that you cannot restrict yourself to being 
a user of classes supplied by your C++ compiler. You have to know the 
inner guts of these classes.

Many of the methods you have implemented in the \verb!IntDynArr! class 
actually appear in this STL \verb!vector! class. This \verb!vector! class 
comes with most C++ compilers. Make sure you try the following examples 
to understand some methods available in this class.

\begin{Verbatim}[frame=single, commandchars=\^\@\~, fontsize=\small]
#include <iostream>
#include <vector>

int main()
{
    std::vector< int > v; // default constructor
    std::cout << "size:" << v.size() << '\n';
    
    v.push_back(5); // expand v by inserting 5 to the end of v
    std::cout << "size:" << v.size() << "    array:";
    for (int i = 0; i < v.size(); i++)
        std::cout << v[i] << ' ';
    std::cout << '\n';

    v.push_back(-54);
    std::cout << "size:" << v.size() << "    array:";
    for (int i = 0; i < v.size(); i++)
        std::cout << v[i] << ' ';
    std::cout << "\n";

    v.push_back(13542);
    std::cout << "size:" << v.size() << "    array:";
    for (int i = 0; i < v.size(); i++)
        std::cout << v[i] << ' ';
    std::cout << "\n";

    v.resize(5); // change the size to 5 so that we also have v[3], v[4]
    v[3] = 3;
    v[4] = 4;
    std::cout << "size:" << v.size() << "    array:";
    for (int i = 0; i < v.size(); i++)
        std::cout << v[i] << ' ';
    std::cout << "\n";
    
    v.resize(0);
    std::cout << "size:" << v.size() << "    array:";
    for (int i = 0; i < v.size(); i++)
        std::cout << v[i] << ' ';
    std::cout << "\n";

    v.resize(5);
    for (int i = 0; i < 5; i++) v[i] = i;
    std::vector< int > u = v; // invoke copy constructor 
    std::cout << "size:" << u.size() << "    array:";
    for (int i = 0; i < u.size(); i++)
        std::cout << u[i] << ' ';
    std::cout << "\n";

    u[0] = -13579;
    u[1] = -24680;
    for (int i = 2; i < u.size(); i++) u[i] = 0; 
    v = u; // Invoking the assignment operator
    std::cout << "size:" << v.size() << "    array:";
    for (int i = 0; i < v.size(); i++)
        std::cout << v[i] << ' ';
    std::cout << "\n";

    v.resize(2);
    std::cout << "size:" << v.size() << "    array:";
    for (int i = 0; i < v.size(); i++)
        std::cout << v[i] << ' ';
    std::cout << "\n";
    
    return 0;
}
\end{Verbatim}

\newpage

You can also find information on the vector class on the web. 
Here are some references:
\begin{tightlist}
\li \url{http://www.cppreference.com/wiki/stl/vector/start}
\li \url{http://www.cplusplus.com/reference/stl/vector/}
\end{tightlist}

It's important for you (at this point in your CS career) to be 
comfortable using resources on the web – without plagiarism of 
course – for references and to learn new things. You can learn 
almost anything you want as long as you have internet access. 
But this is especially the case for CS since CS people are heavy 
users of the internet. After all we did invent the internet. 
Therefore you will find lots of CS resources on the web. Nonetheless, 
it's still good to have a copy of well-written CS books. Sometimes 
the problem with the web is that there's TOO much information and 
you spend time differentiating between good and up-to-date 
information from mis-information.

The vector class is usually mentioned in most C++ textbooks.

Note that you should only use online resources only when you
are allowed to do so.
Otherwise you are plagiarizing.
Plagiarism is a serious academic misconduct -- think academic crime.
