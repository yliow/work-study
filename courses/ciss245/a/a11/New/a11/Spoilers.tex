\textsc{SPOILERS: Advice and recursion}

The following are some suggestions/advice. You need not follow 
them. In fact, treat them as spoilers. Look at them only when 
you're stuck. Also, I'll try not to give you too much help below 
so that you can learn from your struggles. But if you really need 
help, read the following and if you need more help, obviously you 
can talk to me or discuss with fellow students and share ideas. 
And remember that plagiarism – as in copying code – is not allowed. 

Again, implement one small feature at a time, testing it completely 
before moving on to the next.

Clearly you must implement constructors first. The next thing to 
implement is printing. After all, if you can't print, you can't debug!

After you have read the requirements above (carefully), 
you might want to think about how you want to split the logic in your methods
into cases before writing methods for all cases at once.


For \verb!operator+! and \verb!operator+=!, remember that you can easily 
implement \verb!operator+! once you have \verb!operator+=! since for 
almost any imaginable \verb!class C!, once you have the copy constructor 
and \verb!operator+=! working correctly, \verb!operator+! will always 
look like this:

\begin{Verbatim}[frame=single,fontsize=\footnotesize]
C operator+(const C & c) const
{
    return C(*this) += c;
}
\end{Verbatim}

So the focus is on \verb!operator+=!. I suggest you first handle the case 
of adding positive \verb!LongInt! objects. In other words test 
\begin{console}[fontsize=\footnotesize]
LongInt(123) + LongInt(456)
\end{console}
and
\begin{console}[fontsize=\footnotesize]
LongInt i(123);
i += LongInt(456); 
\end{console}
and not 
\begin{console}[fontsize=\footnotesize]
LongInt i(123);
i += (LongInt(-456)
\end{console}
or
\begin{console}[fontsize=\footnotesize]
LongInt(123) + (LongInt(-456)
\end{console}

This means that you might (I'm not 
saying you must) want your \verb!operator+=! to look like this:

\begin{Verbatim}[frame=single,fontsize=\footnotesize]
LongInt & operator+=(const LongInt & I)
{
    if (*this is positive and I is positive)
    {
    }
    // leave the other cases out first
}
\end{Verbatim}
and if so, you might want to have an “is positive method” in your class, 
i.e., have your \verb!operator>=! done first. This means that you might 
as well finish all the boolean operators first:
\begin{Verbatim}[frame=single,fontsize=\footnotesize]
operator==
operator!=
operator<
operator<=
operator>
operator>=
\end{Verbatim}
(After all these are much simpler.)
Once you're done with \verb!operator+=! for positive \verb!LongInt! objects, 
you should realize after some thought that adding two negative numbers is 
actually very easy. Why? If you try to add -123 and -456, isn't that the same 
as adding 123 and 456 and then sticking in a negative sign? Therefore the addition 
of two negative numbers actually depend on the addition of two position numbers. 
I'll let you think about that.
The pseudocode is then:
\begin{Verbatim}[frame=single,fontsize=\footnotesize]
LongInt & operator+=(const LongInt & I)
{
    if (*this is positive and I is positive)
    {
    }
    else (*this is negative and I is negative)
    {
        let J be the same as *this but with sign = 1
        let K be the same as I but with sign = 1
        then both J and K are positive
        perform J += K (recursion)
        copy J to *this, but set the sign of *this to -1
    }
    // other cases
}
\end{Verbatim}

But what about the addition of a positive and a 
negative? For instance 123 and -456. You can actually think of that as the subtraction 
$123 - 456$. So in this case, it depends on \verb!operator-=! where the two number are 
positive. Etc.

Altogether there are 4 cases for \verb!operator+=! and also several cases for \verb!operator-=!.
Figure out the simple cases, write and test them, and then
get the other cases to call these simple cases (recursively) and it will
save you a lot of work.


\newpage
\textsc{SPOILERS: Multiplication}

You should use the standard column multiplication algorithm which you should be very 
familiar with. Here's an example. Suppose you want to multiply $A = 123$ with $B = 142$, you would do this:

\begin{Verbatim}
                      1 2 3
                x     1 4 2
                ------------
                      2 4 6
                    4 9 2
                + 1 2 3
                ------------
                  1 7 4 6 6
                ------------
\end{Verbatim}

Of course this is just

\begin{Verbatim}
                      1 2 3
                x     1 4 2
                ------------
                      2 4 6
                    4 9 2 0
                + 1 2 3 0 0
                ------------
                  1 7 4 6 6
                ------------
\end{Verbatim}

(I've added zeroes.) The rows that you're adding are:

\begin{Verbatim}
                      2 4 6
                    4 9 2 0
                  1 2 3 0 0
\end{Verbatim}

And where do they come from?

\begin{Verbatim}
                      2 4 6 = 123 x 1
                    4 9 2 0 = 123 x 4 x 10
                  1 2 3 0 0 = 123 x 2 x 100
\end{Verbatim}

or

\begin{Verbatim}
                      2 4 6 = 123 x 1 x 10^0
                    4 9 2 0 = 123 x 4 x 10^1
                  1 2 3 0 0 = 123 x 2 x 10^2
\end{Verbatim}

(To make things more regular looking.) If I replace the 123 by \verb!A!, 
the rows become

\begin{Verbatim}
                      2 4 6 = A x 1 x 10^0
                    4 9 2 0 = A x 4 x 10^1
                  1 2 3 0 0 = A x 2 x 10^2
\end{Verbatim}

Of course the 1 and 4 and 2 are the digits of B.

\begin{Verbatim}
                      2 4 6 = A x B[0] x 10^0
                    4 9 2 0 = A x B[1] x 10^1
                  1 2 3 0 0 = A x B[2] x 10^2
\end{Verbatim}

This means that as long as you have a way to multiply your 
\verb!LongInt! objects by a digit and you have a way to multply 
your \verb!LongInt! object by 10 powers, you can generate the 
rows in your column multiplication. Suppose we call these 
functions (or methods – it's up to you) \verb!MULT_BY_DIGIT! 
and \verb!MULT_BY_TENPOWER!, the above is

\begin{Verbatim}
                  2 4 6 = MULT_BY_TENPOWER(MULT_BY_DIGIT(A, B[0]), 0)
                4 9 2 0 = MULT_BY_TENPOWER(MULT_BY_DIGIT(A, B[1]), 1)
              1 2 3 0 0 = MULT_BY_TENPOWER(MULT_BY_DIGIT(A, B[2]), 2)
\end{Verbatim}

Clearly this is a loop over the digits of B. Once you have your program 
printing out the row correctly, you can start to add them. The 
pseudocode is of course this:

\begin{Verbatim}
                sum = 0
                for i = 0, 1, 2, ...:
                    term = MULT_BY_TENPOWER(MULT_BY_DIGIT(A, B[i]), i)
                    sum += term
\end{Verbatim}

Of course it's clear that you MUST have all the addition operations 
done before doing multiplication. And of course you have to make the 
above work for all \verb!LongInt! objects including objects with 
different sizes. You also have to make sure that you can multiply 
integers which are negative. This is easy. For instance if you're 
told in school to multiply 123 and -142, you would just multiply 123 
and 142 and then add a negative sign to the product.

(The high school multiplication method is actually not the only 
way to perform multiplication. Its been thought for a very long time 
that it's the most efficient way. It was only very recent – a couple 
of decades ago – that a more efficient method was discovered by a 
Russian around the 60s. This is called the Karatsuba algorithm.
See CISS358.)


\newpage 
\textsc{Division and mod \%}

I'll leave division and mod (i.e., \texttt{\%}) to you. You only need to think 
about \verb!LongInt! division for positive integers.

I have already given you the algorithm in CISS240:
\[
\frac{21}{4} = 5 \frac{1}{4}
\]
where $5$ is the integer division of 21 by 4 (also called the quotient)
and $1$ is the remainder, i.e., $1$ is $21 \% 4$.
And how do you calculate the 5 and the 1?

Pretend you have 21 dollars and you want to give that out evenly to 4 friends.
Continually give out 4 dollars to each friend.
At some point in time you'll be left with $1$. That's the remainder.
And each friend has 5 dollars.
Therefore the quotient and remainder can be calculated
by repeated subtraction.

There's a fastest way based on \lq\lq long division" method from middle
school.
For instance if you want to compute quotient and remainder of
$3001$ by $2$ using the repeated subtraction, you would have to perform
$1500$ subtractions of $2$ from $3001$ before you stop.
A faster way is to realize that since $2$ is so small, you can
subtract 1000 of $2$'s.
That will bring $3001$ down to $1001$. Right?
That's the idea behind long division from middle school.
At this point, your running-quotient is $1000$.
You can still subtract $2$ from $1001$.
But instead of subtracting a $2$ at a time, you can subtract a larger
number of $2$'s.
For instance you can subtract $500$ $2$'s from $1001$
to get it down to $1$.
At this point your running quotient is $1000 + 500 = 1500$.
Since you are now down to $1$, you cannot subtract $2$ from $1$ and stay
positive.
So the quotient is $1500$ and the remainder is $1$.
Get it?
If you don't you should go online and review long division.
Andif you have not been paying attention in your K-12 math classes, then you
have been missing out on lots of algorithms discovered for the past 4000 years.
The general idea is this:
\begin{console}[fontsize=\footnotesize]
INPUT: n and divisor
OUTPUT: quotient and remainder
1. Let quotient = 0.
2. Find some m such that n - m * divisor >= 0. If you cannot, go to 5.
3. Compute n = n - m * divisor and quotient = quotient + m.
4. Go to 2.
5. Return quotient and n.
\end{console}
The repeated subtraction is when $m$ is always chosen to be $1$.

I leave it to you to decide if you want to implement division and mod using
repeated subtraction or long division.

