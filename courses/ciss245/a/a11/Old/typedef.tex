\textsc{Typedef}

A typedef is just a shorthand for a type. For instance support you
are too lazy to type \verb!bool!. You can do this:

\begin{Verbatim}[frame=single,fontsize=\footnotesize]
typedef bool B; // B is the same as bool
B someflag = true; 
\end{Verbatim}

It's important to know that \verb!B! is the same as \verb!bool!. This 
means for instance that you cannot have the following:
\begin{Verbatim}[frame=single,fontsize=\footnotesize]
void f(bool);
void f(B);
\end{Verbatim}
since the two functions have the same prototype. When compared with 
structures:
\begin{Verbatim}[frame=single,fontsize=\footnotesize]
struct W 
{
    bool flag;
};

struct X
{
    bool flag;
};
\end{Verbatim}
the above \verb!W! and \verb!X! are actually different event though 
they have the same content. Likewise, the following classes are 
considered different:
\begin{Verbatim}[frame=single,fontsize=\footnotesize]
class Y
{
private:
    bool flag;
};

class Z
{
private:
    bool flag;
};
\end{Verbatim}

Therefore it's OK to have the following:
\begin{Verbatim}[frame=single,fontsize=\footnotesize]
void f(W);
void f(X);
void f(Y);
void f(Z);
\end{Verbatim}

You can have typedefs for arrays, pointers, and references. 
This is how you do it:
\begin{Verbatim}[frame=single,fontsize=\footnotesize]
typedef int G [100];
typedef int * H;
typedef int & J;
G g;           // g is an array of 100 integers
H h = new int; // h is a pointer to an int value
J j = *h;      // j is a reference to an int value
\end{Verbatim}

Typedefs are frequently used for really long types or classes. 
For instance you can do this:

\begin{Verbatim}[frame=single,fontsize=\footnotesize]
typedef LongInt Z;
Z i; // i is a LongInt object
\end{Verbatim}

