The goal is to implement a class and functions for integer computations, not at
the level of the C/C++ \verb!int!, but rather integers with an extremely huge
number of digits. We will call this the \verb!LongInt! class.  Such
computations occur frequently in cryptography, physics, etc. This is,
conceptually speaking, probably one of the first problem we have to solve.
After all, if we can't operate with integers of any size, how can we handle
real numbers, complex numbers, functions, etc.?

With this class, you will be able to compute, for instance, the factorial of
1000 (i.e. 1000!) and more.  (Check the web for the factorial of 1000.) Also,
you can modify your \verb!Rational! class to use \verb!LongInt! instead of
\verb!int!  for numerator and denominator. With that, you can compute with
fractions that look like this:

\verb!  12312312415412312365476798 / 123598347693429857485202845!

and you can check if

\verb!  12312312415412312365476791!

is prime.  This is helpful in many area of math and CS. For instance modern
cryptography (example RSA cryptography) uses extremely huge prime numbers of
several hundred digits in length (at this point in time). RSA cryptography
depends on the difficulty of factoring numbers. In particular, it involves
numbers such as

\begin{Verbatim}
2260138526203405784941654048610197513508038915719776718321197768109445641817
 9666766085931213065825772506315628866769704480700018111497118630021124879281
 99487482066070131066586646083327982803560379205391980139946496955261
\end{Verbatim}

This is a product of two primes. The goal is to find these two primes. Clearly
you can't even begin if your computer cannot work with numbers of this size. If
you can factorize the above number into two primes, then ciphertext (i.e.,
encrypted text) that is encrypted using RSA and using the above number is
completely broken and you can gain access to the encrypted data.

The fact that C/C++ natively can only handle integers with 10 digits is not a
drawback of the language. No one expects a programming language to supply all
the possible types and classes in the world. It's more important for a language
to be extensible, i.e., that it provides features to create new types of
values.  That's why object-oriented programming is important.  Now back to our
“long integer” class.

Note that the integers in C/C++ are limited in size. Specifically an \verb!int!
value is usually about -2 billion to 2 billion. The goal of this assignment is
to write a class to compute integers outside of this range.  This can be
achieved using arrays. For instance for the integer 1352 (the numeric concept
of “one thousand, three hundred and fifty-two”), can be stored in an array,
\verb!size!, and \verb!sign! variables:

\begin{Verbatim}
            x[0] = 2 x[1] = 5 x[2] = 3 x[3] = 1 size = 4 sign = 1 // sign
 equals 1 means the number being modeled // is “positive”
\end{Verbatim}

and the integer -237 (the numeric concept of “negative two hundred and
thirty-seven”) can be modeled using:

\begin{Verbatim}
            x[0] = 7 x[1] = 3 x[2] = 2 size = 3 sign = -1 // sign equals -1
 means the number being modeled // is “negative”
\end{Verbatim}

Note that the \lq\lq ones" digit go into \verb!x[0]!, the \lq\lq 10s" digit go
into \verb!x[1]!, etc. Note that 0 (the concept of \lq\lq zero") can be modeled
using

\begin{Verbatim}
            x[0] = 0 size = 1 sign = 1
\end{Verbatim}

Design a class for such objects. The name of the class is \verb!LongInt!.  For
the integer array, you should use the STL \verb!vector! class – see section
below for information on the \verb!vector! class. This is similar to our
\verb!IntDynArr! class. (The point of working on \verb!IntDynArr!  is to gain
an understanding of the internal workings of \verb!std::vector!.)  Note that
each \verb!vector! object already maintains a \verb!size! instance
variable. Therefore for this class, you only need to add \verb!sign! to the
class. Hence the class should look like this:

\begin{Verbatim}
            class LongInt { private: std::vector< int > x; int sign; };
\end{Verbatim}

Note that we are using object composition. By the way, do not confuse the C++
STL \verb!vector! with vectors from math or our \verb!vec2d! class.  In
computer science, a dynamic array of values is also called a vector.

Note the curious syntax of the class we're using for the dynamic array of
digits:

\verb!  std::vector< int >!

This basically tells C++ that you want a vector of \verb!int! values.  (For
details refer to notes on templates.)

Now let me describe what features we want for this class ...

With your class, you should be able to do the following:

\begin{Verbatim}[frame=single,fontsize=\footnotesize]
LongInt i; // i models 0 with default constructor std::cout << i << '\n';; //
 Prints 0 LongInt j(123); // Constructor call with int value std::cout << j <<
 '\n'; // Prints 123 LongInt k("9876543210"); // Constructor call with C-string
 std::cout << k << '\n'; // Prints 9876543210 LongInt l("-9876543210");
 std::cout << l << '\n'; // Prints -9876543210 LongInt m = l; // Copy
 constructor: m models the same // integer that l models std::cout << m <<
 '\n'; // Prints -9876543210
\end{Verbatim}

The above describes the various constructors for this class. Of course you need
to implement all the arithmetic operators:

\begin{Verbatim}
            LongInt i("1000000000000"); LongInt j(3); i += j; std::cout << i <<
 '\n'; // prints 1000000000003 i -= j; std::cout << i << '\n'; // prints
 1000000000000
\end{Verbatim}

[\textsc{Important debugging suggestion}: It's a good idea while testing your
  code to temporarily print your \verb!LongInt! objects in a slightly different
  way:

\begin{Verbatim}
            LongInt i("123456789"); LongInt j("-123456789"); std::cout << i <<
 '\n'; // prints < + 1 2 3 4 5 6 7 8 9 > std::cout << j << '\n'; // prints < -
 1 2 3 4 5 6 7 8 9 >
\end{Verbatim}

Why? Because if the output is for instance -123 it could be that the values in
the \verb!vector! is 3, -12 and the \verb!sign! is 1 or it could be 3, 2, -1
and the \verb!sign! is 1. Once you know your class works correctly, you change
the output back to what it should be.]

Read the following very carefully ...

Note that if your class \verb!LongInt! class has a pointer to a dynamic array
of integers in the heap for the digits (like the \verb!IntDynArr!  class) then
you would need to define your own destructor, and copy constructor, and
\verb!operator=!. However we are using the STL \verb!vector! class (which
contains a pointer for the same purpose). The \verb!vector!  class already has
a destructor to deallocate the memory used by the pointer in \verb!vector!
objects. Therefore when your \verb!LongInt!  default destructor is called, C++
will call the destructor of \verb!vector! for you, doing the right
thing. Likewise when you call \verb!operator=! for your \verb!LongInt! object
like this
\begin{Verbatim}[frame=single,fontsize=\footnotesize]
i = j;
\end{Verbatim}
where \verb!i! and \verb!j! are \verb!LongInt! objects, C++ will call
\verb!operator=! on the STL \verb!vector! object inside the \verb!LongInt!
objects, again doing the right thing. So you do not have to implement
\verb!operator=! for the \verb!LongInt! class. (But just because I say you
don't have to, it does not mean you don't have to know why.)

Other augmented operators (besides the \verb!+=! and \verb!-=! from above) you
must have are the \verb!*=, /=, %=!. Of course there are the binary operators
\verb!+, -, *, /, %.!

Your \verb!LongInt! class must be able to inter-operate with \verb!int!  values
too. For instance
\begin{Verbatim}[fontsize=\footnotesize,frame=single]
LongInt i(2); i += 3; std::cout << i << std::endl; // prints 5 std::cout << (i
 + 1) << std::endl; // prints 6 std::cout << (1 + i) << std::endl; // prints 7
 int j = 2; j += i; std::cout << j << std::endl; std::cout << (1 + j) <<
 std::endl; std::cout << (1 + j) << std::endl;
\end{Verbatim}

The same applies to \verb!-=, *=, %=, -, *, /.! (For this, you should refer to
your \verb!Rational! class for review and hints.)

All these operators behave in the usual mathematical way. For instance you have
to be careful about carries:
\begin{Verbatim}[fontsize=\footnotesize,frame=single]
LongInt i("10099999"); i = i + 2; std::cout << i << std::endl; // prints
 10100001
\end{Verbatim}
and you have to be careful with borrows during subtraction:
\begin{Verbatim}[fontsize=\footnotesize,frame=single]
LongInt i("3000000000000"); i = i - 1; std::cout << i << std::endl; // prints
 2999999999999
\end{Verbatim}

To allow easy assignment for extremely long integers you must overload
operator= to be able to do this:
\begin{Verbatim}[fontsize=\footnotesize,frame=single]
LongInt i; i = "9876543210"; std::cout << i << std::endl; // prints 9876543210
 i = "-9876543210"; std::cout << i << std::endl; // prints -9876543210
\end{Verbatim}

Your class must also allow comparisons among objects and with integers using
\verb@==@, \verb@!=@, \verb!<!, \verb!<=!, \verb!>!, \verb!>=!.  For instance
\begin{Verbatim}[fontsize=\footnotesize,frame=single]
LongInt i(123), j(74); bool b0 = (i == j); // false bool b1 = (i < 42); //
 false bool b1 = (42 <= j); // true bool b2 = (-1 == j); // false
\end{Verbatim}

Etc.

Other operators include \verb!++! and \verb!--!. Note that there are two types
of \verb!++! and \verb!--!. C++ differentiates between the pre- and
post-increment operators in the following way:
\begin{align*}
\texttt{++i} \,\,\,\, &\text{ same as } \,\,\,\, \texttt{i.operator++()}
\\ \texttt{i++} \,\,\,\, &\text{ same as } \,\,\,\, \texttt{i.operator++(0)}
\end{align*}
(Refer to the notes on operator overloading.) The difference between the pre-
and post-increment operator is that the pre- version all return a reference to
the object after some form of operator. Here's a simple experiment on ++ for
int type:

\begin{Verbatim}[fontsize=\footnotesize,frame=single]
int i = 0; int & j = (++i); // i becomes 1 and j references i
\end{Verbatim}

The post version actually returns a clone of the old value of \verb!i!

\begin{Verbatim}[fontsize=\footnotesize,frame=single]
int i = 0; int j = (i++); // i becomes 1 but j is set to 0, the original //
 value of i
\end{Verbatim}

Therefore in term of implementation in a class say C, the pre- and
post-increment operators would look like this:

\begin{Verbatim}[fontsize=\footnotesize,frame=single]
class C { public: C & operator++() // pre-increment. Note return type.  { //
 some code to change object *this return *this; } C operator++(int i) //
 post-increment. Note return type.  { C old_object = (*this); // some code to
 change object *this return old_object; } };
\end{Verbatim}

You should also implement the “negative of” operator:

\begin{Verbatim}[fontsize=\footnotesize,frame=single]
LongInt i(123); LongInt j = -i; // j is the same as LongInt(-123)
\end{Verbatim}

Note that

\verb!  -i same as i.operator-();!

This is the unary “negative” operator (which is not the same as the subtraction
binary operator) which is not the same as \verb!i - j!  which is
\verb!i.operator-(j)!, i.e. the two operators have different signatures.

Of course there's also the unary positive operator:

\verb!  +i same as i.operator+();!

(Refer to the \verb!Rational! class for a quick review.)

You should also implement an \verb!abs()! method that returns the absolute
value of the LongInt object:

\begin{Verbatim}[fontsize=\footnotesize,frame=single]
LongInt i(-123); LongInt j = i.abs(); // j is the same as LongInt(123)
\end{Verbatim}

For convenience, you should also have a non-member function \verb!abs()!:

\begin{Verbatim}[fontsize=\footnotesize,frame=single]
LongInt i(-123); LongInt j = abs(i); // j is the same as LongInt(123)
\end{Verbatim}

With all the above methods you can more or less treat \verb!LongInt! the same
as \verb!int!. For instance you can print the factorial of 1000:

\begin{Verbatim}[fontsize=\footnotesize,frame=single]
LongInt product(1); for (int i = 1; i <= 1000; i++) { product *= i; std::cout
 << i << ' ' << product << std::endl; }
\end{Verbatim}

The factorial of 1000 (i.e. 1000!) is an extremely long integer. You can check
the correctness of your output by searching for it on the web.  Of course you
can even declare i to be a \verb!LongInt! object:

\begin{Verbatim}[fontsize=\footnotesize,frame=single]
LongInt product(1); for (LongInt i = 1; i <= 1000; i++) { product *= i;
 std::cout << i << ' ' << product << std::endl; }
\end{Verbatim}

but of course you would expect \verb!int! variables to perform computations
faster than \verb!LongInt! objects.

\textsc{Other Requirements}

As always:
\begin{tightlist}
    \li Methods must be constant wherever possible \li Object parameters passed
    by reference whenever possible; they must also be made constant wherever
    possible \li Code duplication must be kept to a minimal. In general the
    augmented assignment operators should be implemented first, with the
    non-augmented ones depending on the augmented ones.  Similarly,
    \texttt{operator!=} is just the opposite of \verb!operator==!.  \li You
    must have a header file \verb!LongInt.h! and a class implementation file
    \verb!LongInt.cpp!.  \li You must include a test file
    \verb!testLongInt.cpp! that tests all the functions and methods of your
    \verb!LongInt! library.
\end{tightlist}
