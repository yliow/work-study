\textsc{Unsigned Integers}

In the short tutorial on STL vector class above, you would notice 
that the compiler might give you a warning here:

\begin{Verbatim}[frame=single, commandchars=\^\@\~, fontsize=\footnotesize]
#include <iostream>
#include <vector>

int main()
{
    std::vector< int > v; // default constructor
    std::cout << "size:" << v.size() << '\n';
    
    v.push_back(5); // expand v by inserting 5 to the end of v
    std::cout << "size:" << v.size() << "    array:";
    ^underline@^textbf@for (int i = 0; i < v.size(); i++)~~
        std::cout << v[i] << ' ';
    std::cout << '\n';


    return 0;
}
\end{Verbatim}

saying that you're comparing \verb!unsigned int! with \verb!int!.

There are many integer types in C++. (Refer to your CISS240 notes). 
The \verb!int! is the most common. 
\begin{Verbatim}[frame=single,fontsize=\footnotesize]
int x = 0;
\end{Verbatim}
In this case, assuming a 32-bit machine and compiler, your \verb!x! 
can hold an integer value from $-2^{31}$ to $2^{31} - 1$, i.e., roughly 
-2 billion to +2 billion. An \verb!unsigned int! is declared like this:
\begin{Verbatim}[frame=single,fontsize=\footnotesize]
unsigned int x = 0;
\end{Verbatim}
or 
\begin{Verbatim}[frame=single,fontsize=\footnotesize]
unsigned x = 0;
\end{Verbatim}
In this case \verb!x! can hold an integer value from 0 to $2^{32} - 1$, 
i.e., from 0 to roughly 4 billion. That's all there is to 
\verb!unsigned int!.

If \verb!v! is a STL vector object, then \verb!v.size()! is an 
\verb!unsigned int!. In the code above \verb!i! is declared to 
be an \verb!int!. That's why there was a warning. It's not 
necessarily an error. This is not an issue when \verb!i! and 
\verb!v.size()! have values in their common ranges, i.e., 
0 to $2^{31} - 1$. To be absolutely safe and to avoid having compiler 
warning messages, you do this:
\begin{Verbatim}[frame=single,fontsize=\footnotesize]
...

int main()
{
    ...

    for (unsigned int i = 0; i < v.size(); i++)
        std::cout << v[i] << ' ';
    
    ...
}
\end{Verbatim}

