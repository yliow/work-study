
WARNING: I've already mentioned this. You must keep all the code for
templates in header files. 

This illustrates a very common approach to developing a class template.
You usually first pick a special case of the template and develop it as a
concrete class before you generalize it to a class template.
In many cases once the specific concrete case works correctly,
the generic template case works with very little extra work.
On the other hand, if you begin with the class template directly,
you usually end up getting lots of convoluted error messages.
In general error message from template classes requires a lot more
carefully reading. Hence it's more time consuming to work directly on
a class template.
 
NOTE: You can assume the user will \textit{not} perform
mixed type computations such as 
\texttt{vec2< int >(1, 2) + vec2< double >(1.1, 2.2)} which involves
\verb!int! and \verb!double! type values.
The
normalization of a vector with \texttt{double}s is a vector of \texttt{double}s
and the
normalization of a vector with \texttt{float}s is a vector of \texttt{float}s.
