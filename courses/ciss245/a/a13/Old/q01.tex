
The goal is to convert a (regular) class to a template class. The class chosen is our
\verb!vec2d! class.

We already have a \verb!vec2d! class.
Objects of this class models 2-dimensional vectors with \verb!double!
coordinates.
Frequently in games (or scientific simulation),
\verb!float!s are sufficient.
Note that \verb!double!s can represent more real numbers than \verb!float!s.
Furthermore in many \lq\lq simple" games such as 2-d games from the 80s,
integer coordinates are enough and integer operation are a lot faster than
\verb!double! or \verb!float! operations. 

The goal of this question is to build a template 2-dimensional vector class,
\verb!vec2!.
You should use the \verb!vec2d! class from our previous assignment.
With this class we
can create 2-dimensional vectors of different numeric types
(\verb!int!, \verb!float!, \verb!double!) like this:
\begin{Verbatim}[frame=single]
vec2< int > u(1, 2);         // u is a 2-d vector with integer coordinates
vec2< float > v(1.2f, 3.4f); // v is a 2-d vector with float coordinates
vec2< double > w(1.2, 3.4);  // w is a 2-d vector with double coordinates
\end{Verbatim}
  
In your \verb!vec2! header file you should include three typedefs:
\begin{tightlist}
\li \verb!vec2i! which is an alias for \verb!vec2< int >!
\li \verb!vec2f! which is an alias for \verb!vec2< float >!
\li \verb!vec2d! which is an alias for \verb!vec2< double >!
\end{tightlist}
Of course this depends on \verb!vec2!,
so these typedefs should be after the \verb!vec2! class.

With these typedefs the above examples become
\begin{Verbatim}[frame=single]
vec2i u(1, 2); // u is a 2-d vector with integer coordinates
vec2f v(1.2f, 3.4f); // v is a 2-d vector with float coordinates
vec2d w(1.2, 3.4); // w is a 2-d vector with double coordinates
\end{Verbatim}

Note that the length function,
\verb!len()!,
must return a \verb!double! regardless of the template type parameter.

Also, note that the argument for \verb!operator[]! is either 0 or 1.
If a value other than 0 or 1 is given, then you must
throw a \verb!ValueError! exception. This class should be included
at the the top of your \verb!vec2.h! file, i.e.,
\begin{console}
#ifndef VEC2_H
#define VEC2_H

class ValueError
{};

...
#endif
\end{console}
and the following will catch a \verb!ValueError! object:
\begin{console}
vec2f v(1, 2);
try
{
    std::cout << v[42] << '\n';
}
catch (ValueError & e)
{
    std::cout << "caught ValueError object\n";
}
\end{console}

Test your code thoroughly.
The test file must be named \verb!testvec2.cpp!.
(The test cases you should include should be very similar to the test code
for \verb!vec2d!.)
