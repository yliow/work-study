%-*-latex-*-
\newcommand\COURSE{ciss350}
\newcommand\ASSESSMENT{a01}
\newcommand\ASSESSMENTTYPE{Assignment}
\newcommand\POINTS{	extwhite{xxx/xxx}}

\input{myquizpreamble}
\input{yliow}
\renewcommand\TITLE{\ASSESSMENTTYPE \ \ASSESSMENT}

\renewcommand\EMAIL{}
\input{\COURSE}
\textwidth=6in

 

\newcommand\BLANK{\sqcup}
% Used in DFA minimization
\newcommand\ind{\operatorname{index}}


\makeindex
\begin{document}
\topmatter


%\usepackage[T1]{fontenc}
%\usepackage{upquote}

\usepackage[T1]{fontenc}  % access \textquotedbl
\usepackage{textcomp}     % access \textquotesingle
\usepackage{mathptmx}     % load "Times New Roman" text font 
                          % (note: the "times" package is obsolete!)
                          
\renewcommand\AUTHOR{} % CHANGE TO YOURS

\begin{document}
\topmattertwo

This is a closed-book, no compiler, 2 minute quiz.

First trace the following program and write down the
The console window output of the following program
\begin{console}[fontsize=\small]
#include <iostream>

int main()
{
    std::cout << 'I' << 't' << "was a" << '\n' << "dark\n" 
              << std::endl << std::endl << "and  \"stormy\"" << ' ' << " " 
              << "night\n";
    
    return 0;
}
\end{console} 
is (use one square for each printed character):

\begin{python}
import string
#def ph(c):
#    s = string.ascii_letters + string.digits + string.punctuation
#    return r'{\vphantom{%s}\texttt{%s}}' % (s, c)
        
from latextool_basic import *
m = [['I','t','w','a','s','','a'] + ['' for i in range(21-7)],
     ['d','a','r','k'] + ['' for i in range(21-4)],
     ['' for i in range(21)],
     ['' for i in range(21)],
     list('and  "stormy"  night '),
     ['' for _ in range(21)],
]

# replace with spaces
m = [[' ' for c in row] for row in m]
m[0][0] = 'I'

p = Plot()
C = table2(p, m, rowlabel='x', collabel='y',
           do_not_plot=True,rownames=[], colnames=[])

import string
def f(r, c, ch, background='blue!20', vphantom=string.printable):
    x0,y0 = C[r][c].bottomleft()
    x1,y1 = C[r][c].topright()
    return Rect(x0=x0, y0=y0, x1=x1, y1=y1,
                background=background,
                label=ph(ch),
                linewidth=0)

x0,y0 = C[0][2].bottomleft()
x1,y1 = C[0][2].topright()
p += Rect(x0=x0, y0=y0, x1=x1, y1=y1,
          background='blue!20',label=r'\texttt{ }', linewidth=0)

#x0,y0 = C[0][6].bottomleft()
#x1,y1 = C[0][6].topright()
#p += Rect(x0=x0, y0=y0, x1=x1, y1=y1,
#          background='blue!20',label=r'\texttt{ }', linewidth=0)

x0,y0 = C[1][0].bottomleft()
x1,y1 = C[1][0].topright()
p += Rect(x0=x0, y0=y0, x1=x1, y1=y1,
          background='blue!20',label=r'\texttt{ }', linewidth=0)

x0,y0 = C[3][1].bottomleft()
x1,y1 = C[3][1].topright()
p += Rect(x0=x0, y0=y0, x1=x1, y1=y1,
          background='blue!20',label='', linewidth=0)

x0,y0 = C[4][5].bottomleft()
x1,y1 = C[4][5].topright()
p += Rect(x0=x0, y0=y0, x1=x1, y1=y1,
          background='blue!20', label=r'\texttt{ }', linewidth=0)

x0,y0 = C[4][15].bottomleft()
x1,y1 = C[4][15].topright()
p += Rect(x0=x0, y0=y0, x1=x1, y1=y1,
          background='blue!20',label=r'\texttt{ }', linewidth=0)

# Draw table one more time to get border of C[1][4] correct
table2(p, m, rowlabel=None, collabel=None,
rownames=[r'\texttt{%s}' % r for r in range(0, 6)],
colnames=[r'\texttt{%s}' % r for r in range(0, 21)])
print(p)
\end{python}
The first output character \verb!'I'! has already been filled for you.

Characters graded are in the shaded cells.
Now write down the character (remember your single quotes!)
printed at the row number and column number.
The row and column numbering starts with 0.

\nextq Character at row 0, column 2: \answerbox{ }\\
\nextq Character at row 1, column 0: \answerbox{ }\\
\nextq Character at row 3, column 1: \answerbox{ }\\
\nextq Character at row 4, column 5: \answerbox{ }\\
\nextq Character at row 4, column 15: \answerbox{ }\\

\begin{python}
from latextool_basic import *
print consolegrid(numrows=7)
\end{python}

%------------------------------------------------------------------------------
\newpage
You are given the following (possibly incomplete files):
\begin{tightlist}
  \li \texttt{Rational.h}
  \li \texttt{Rational.cpp}
  \li \texttt{main.cpp} (the test code)
\end{tightlist}
\textsc{Important Warning:}
Again, the files are meant to be skeleton file and might not be
complete and might have deliberate missing details or even errors.

Create directory
\texttt{ciss245/a/a07/a07q01}.
Keep all your files in this directory.

If you're doing a copy-and-paste of the given code,
note that some character might be changed by PDF to other characters.
In particular the - character might actually not be the dash character.
Looking at the compiler error message will help you find these minor
annoying issues so that you can correct them.

Study the given test code.
Add tests if necessary to test all methods and functions.
Such low level function/method tests are called \defterm{unit tests}.

Observe the following very carefully:
\begin{tightlist}
\li All methods must be constant whenever possible. 
\li All parameters which are objects (or struct variables)
must be pass by reference or pass by constant reference as much as possible.
\li Reuse code as much as possible.
For instance \verb@operator!=()@ should use \verb!operator==()!. 
\end{tightlist}
Let me know ASAP if you see a typo.

\end{document}
