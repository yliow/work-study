\newpage\section{Statement}

Here's the first jargon. Look at our program again:
\begin{console}[commandchars=\~\%\@]
#include <iostream>

int main()
{
    ~textbf%std::cout << "Hello, world!\n";@

    return 0;
}
\end{console}

The line in bold is a 
\EMPHASIZE{statement}. 
In C++, statements must terminate with a 
\EMPHASIZE{semi-colon}. 
Note that the semi-colon is part of the statement. 

At this point you should think of a statement as something that will cause 
your computer to perform some operation(s) when you run the program.

If you like, you can think of a statement as a sentence and the semicolon as a
period. When you're told to write a C++ statement, don't forget the 
semicolon!!! 
That would be like writing a sentence without a period (or question mark or 
exclamation mark) in an english essay.
%period-exercise

See what I mean by this advice:

\begin{itemize}
\item[]
\textit{A quick advice to ease the pain of learning your first programming 
language: Type the program 
\EMPHASIZE{exactly} as given. 
Even the spaces and blank lines must match my programs.}
\end{itemize}

Computers are dumb (and picky about details.) 
We are the smart ones. So … when you communicate with your computer, 
you have to be exact and explicit in your programs.

%-*-latex-*-

\begin{ex} 
  \label{ex:prob-00}
  \tinysidebar{\debug{exercises/{disc-prob-28/question.tex}}}

  \solutionlink{sol:prob-00}
  \qed
\end{ex} 
\begin{python0}
from solutions import *
add(label="ex:prob-00",
    srcfilename='exercises/discrete-probability/prob-00/answer.tex') 
\end{python0}

%-*-latex-*-

\begin{ex} 
  \label{ex:prob-00}
  \tinysidebar{\debug{exercises/{disc-prob-28/question.tex}}}

  \solutionlink{sol:prob-00}
  \qed
\end{ex} 
\begin{python0}
from solutions import *
add(label="ex:prob-00",
    srcfilename='exercises/discrete-probability/prob-00/answer.tex') 
\end{python0}

