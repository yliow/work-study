%-*-latex-*-
\sectionthree{Hello world}
\begin{python0}
from solutions import *; clear()
\end{python0}


Go ahead and run your first program:
\begin{console}
#include <iostream>

int main()
{
    std::cout << "Hello, world!\n";

    return 0;
}
\end{console}

\textit{ ... commercial break ... go to notes on software tool(s)
  for writing and running a program ...}
\sidebox{a}{
       Practice doing a hello
         world program until you
         can do it in $<$ 2 minutes
         and without your notes.
}

Now let's go back to the C\texttt{++} code.
Right now, you should think of the stuff in bold as the program:
\begin{console}[commandchars=\~\!\@]
#include <iostream>

int main()
{
    ~textbf!*** YOUR "PROGRAM" GOES HERE ***@

    return 0;
}
\end{console}

Therefore you can treat this as a template:
\begin{consolethree}[escapeinside=||]
#include <iostream>

int main()
{
                |\tikzmarknode{b}{}\sidebox[-0.6cm]{a}{\textnormal{Enter a single program here.}}\DrawArrow{a}{b}|

    return 0;
}
\end{consolethree}

I will not explain things like 
\lq\lq \verb!#include <iostream>!" or 
\lq\lq \verb!int main()!" until later. 
(Technically, they are also part of  the program.)

Try this
\begin{console}
#include <iostream>

int main()
{
    std::cout << "Hello ... world! ... mom!\n";

    return 0;
}
\end{console}

%-*-latex-*-

\begin{ex} 
  \label{ex:prob-00}
  \tinysidebar{\debug{exercises/{disc-prob-28/question.tex}}}

  \solutionlink{sol:prob-00}
  \qed
\end{ex} 
\begin{python0}
from solutions import *
add(label="ex:prob-00",
    srcfilename='exercises/discrete-probability/prob-00/answer.tex') 
\end{python0}

%-*-latex-*-

\begin{ex} 
  \label{ex:prob-00}
  \tinysidebar{\debug{exercises/{disc-prob-28/question.tex}}}

  \solutionlink{sol:prob-00}
  \qed
\end{ex} 
\begin{python0}
from solutions import *
add(label="ex:prob-00",
    srcfilename='exercises/discrete-probability/prob-00/answer.tex') 
\end{python0}

%-*-latex-*-

\begin{ex} 
  \label{ex:prob-00}
  \tinysidebar{\debug{exercises/{disc-prob-28/question.tex}}}

  \solutionlink{sol:prob-00}
  \qed
\end{ex} 
\begin{python0}
from solutions import *
add(label="ex:prob-00",
    srcfilename='exercises/discrete-probability/prob-00/answer.tex') 
\end{python0}



