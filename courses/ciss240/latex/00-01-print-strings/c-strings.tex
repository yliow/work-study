%-*-latex-*-
\sectionthree{C-strings}
\begin{python0}
from solutions import *; clear()
\end{python0}

The stuff in quotes is called a \EMPHASIZE{C-string} or just a \EMPHASIZE{string}:
\begin{consolethree}[escapeinside=||]
#include <iostream>

int main()
{
    std::cout << |\tikzmarknode[tikzmarknode thickred]{b}{"hello world\bstt n"}\sidebox[-0.5cm]{a}{\textnormal{C-string}}\DrawArrow{a}{b}|;

    return 0;
}
\end{consolethree}

You can think of a string as textual data. 
(Soon I'll talk about numeric data for numeric computations. 
Got to have that for computer games, right?)

Double quotes are used to mark the beginning and ending of the string. 
So technically they are \textit{\textbf{\underline{not}}} part of the string. 
That's why when you run the above program, you do not see double-quotes.

%-*-latex-*-

\begin{ex} 
  \label{ex:prob-00}
  \tinysidebar{\debug{exercises/{disc-prob-28/question.tex}}}

  \solutionlink{sol:prob-00}
  \qed
\end{ex} 
\begin{python0}
from solutions import *
add(label="ex:prob-00",
    srcfilename='exercises/discrete-probability/prob-00/answer.tex') 
\end{python0}


In the string 
\verb@"Hello, world!\n"@, 
\verb!H! is a \EMPHASIZE{character}. 
The next character is \verb!e!. Etc. 
When we talk about characters we enclose them with single-quotes. 
So I should say character \verb!'H'! rather than 
\verb!H! or \verb!"H"!. 
You can think of the character as the smallest unit of data in a string. 

The string \verb@"Hello, world!\n"@ contains characters 
\verb!'H'! and \verb!'e'! and \verb!'l'! and \verb!'l'! and 
... 
However a character such as 
\verb!'H'! can contain one and only one unit of textual data. 
So you \textit{\textbf{\underline{cannot}}} say that \verb!'Hello'! is a character. 

Note that a string can contain as many as characters as you like: 
10, 20, 50, 100, etc. 
There's actually a limit, but we won't be playing around with a string with 
1,000,000 characters anyway for now --
that would be a pain to type!!! 
Note that a string can contain no characters at all: 
\verb!""!. 
Of course a string can contain exactly one character. 
For instance, here's a string with one character: \verb!"H"!. 

Note that \verb!"H"! is a string with one character whereas 
\verb!'H'! is a character. 
You just have to look at what quotes are used to tell if the thingy is a 
string or a character. That's all.

%-*-latex-*-

\begin{ex} 
  \label{ex:prob-00}
  \tinysidebar{\debug{exercises/{disc-prob-28/question.tex}}}

  \solutionlink{sol:prob-00}
  \qed
\end{ex} 
\begin{python0}
from solutions import *
add(label="ex:prob-00",
    srcfilename='exercises/discrete-probability/prob-00/answer.tex') 
\end{python0}

%-*-latex-*-

\begin{ex} 
  \label{ex:prob-00}
  \tinysidebar{\debug{exercises/{disc-prob-28/question.tex}}}

  \solutionlink{sol:prob-00}
  \qed
\end{ex} 
\begin{python0}
from solutions import *
add(label="ex:prob-00",
    srcfilename='exercises/discrete-probability/prob-00/answer.tex') 
\end{python0}

%-*-latex-*-

\begin{ex} 
  \label{ex:prob-00}
  \tinysidebar{\debug{exercises/{disc-prob-28/question.tex}}}

  \solutionlink{sol:prob-00}
  \qed
\end{ex} 
\begin{python0}
from solutions import *
add(label="ex:prob-00",
    srcfilename='exercises/discrete-probability/prob-00/answer.tex') 
\end{python0}

