%-*-latex-*-
\sectionthree{Whitespaces}
\begin{python0}
from solutions import *; clear()
\end{python0}

A whitespace is ... well ... a white space.

Spaces, tabs, and newlines are whitespaces.


\begin{ex} 
  \label{ex:some-decision1}
  \tinysidebar{\debug{exercises/{empty0/question.tex}}}
  \solutionlink{sol:some-decision1}
  \qed
\end{ex} 
\begin{python0}
from solutions import *
add(label="ex:some-decision1",
    srcfilename='exercises/some-decision1/answer.tex') 
\end{python0}


\begin{ex} 
  \label{ex:some-decision1}
  \tinysidebar{\debug{exercises/{empty0/question.tex}}}
  \solutionlink{sol:some-decision1}
  \qed
\end{ex} 
\begin{python0}
from solutions import *
add(label="ex:some-decision1",
    srcfilename='exercises/some-decision1/answer.tex') 
\end{python0}


In general you can insert whitespaces between 
\lq\lq basic words'' understood by C++. 
These \lq\lq basic words'' are called \textbf{tokens}. 
(Oooooo another big word.)

If you think of statements as sentences, 
semi-colons as periods, then you can think of tokens as words.


For instance the following are some tokens from the above program: 
\texttt{int}, \texttt{return} and even \texttt{\{}.

We say that C++ \textbf{ignores whitespace}. 


\begin{ex} 
  \label{ex:some-decision1}
  \tinysidebar{\debug{exercises/{empty0/question.tex}}}
  \solutionlink{sol:some-decision1}
  \qed
\end{ex} 
\begin{python0}
from solutions import *
add(label="ex:some-decision1",
    srcfilename='exercises/some-decision1/answer.tex') 
\end{python0}



This shows you that \textbf{::} is a token -- you cannot break it down.


\begin{ex} 
  \label{ex:some-decision1}
  \tinysidebar{\debug{exercises/{empty0/question.tex}}}
  \solutionlink{sol:some-decision1}
  \qed
\end{ex} 
\begin{python0}
from solutions import *
add(label="ex:some-decision1",
    srcfilename='exercises/some-decision1/answer.tex') 
\end{python0}


Try this:
\begin{console}
#include <iostream>

int main()
{
    std::cout << "Hello,          world\n";

    return 0;
}
\end{console}

The whitespaces between tokens are removed when your computer runs your C++ program. So, to the computer, it doesn't matter how much whitespace you insert between tokens -- it still works.

However spaces in a \EMPHASIZE{string} 
are \EMPHASIZE{not} removed before the program runs. 
The space characters \verb!' '!
(there are 10 in the above string) are actually characters within the string.

Note that although you can insert whitespaces, 
in general, good spacing of a program makes it easier to read. 
Therefore you must follow the style of spacing shown in my notes. 
In particular, I 
\EMPHASIZE{don't} want to see monstrous programs like this 
(although the program does work):
\begin{console}
#include <iostream>

int                        main(){std
::
                    cout<<

"Hello, world!\n"; return 0; }
\end{console}
or this: 
\begin{console}
#include <iostream>
int main(){std::cout<<"Hello, world!\n";return 0;}
\end{console}
And don't try to be cute this like ...
\begin{console}
#include <iostream>
int
 main
  ()
   {
    std
     ::
      cout
       <<
        "Hello, world!\n"; 
         return 0; 
          }
\end{console}
The above examples are excellent candidates for the F grade. 
The correct coding style produces this:
\begin{console}
#include <iostream>

int main()
{
    std::cout << "Hello, world!\n";

    return 0;
}
\end{console}

The left trailing spaces of a statement is called the 
\EMPHASIZE{indentation} 
of the statement:
\begin{consolethree}[escapeinside=||]
#include <iostream>

int main()
{
|\tikzmarknode{a}{}|   |\tikzmarknode{b}{}|   |\tikzmarknode{c}{}|std::cout << "Hello, world!\n"; 
      return 0;
}
\end{consolethree}
\sidebox[-1.5cm]{d}{\textnormal{Indentation}}\DrawArrow[red,line width=0.05mm]{b}{a}\DrawArrow[red,line width=0.05mm]{b}{c}%
It's a common programming style to use 4 spaces for indentation. 

