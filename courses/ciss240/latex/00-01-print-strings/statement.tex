%-*-latex-*-
\sectionthree{Statement}
\begin{python0}
from solutions import *; clear()
\end{python0}

Here's the first jargon. Look at our program again:
\begin{console}[commandchars=\~\%\@]
#include <iostream>

int main()
{
    ~textbf%std::cout << "Hello, world!\n";@

    return 0;
}
\end{console}

The line in bold is a 
\EMPHASIZE{statement}. 
In C\texttt{++}, statements must terminate with a 
\EMPHASIZE{semi-colon}. 
Note that the semi-colon is part of the statement. 

At this point you should think of a statement as something that will cause 
your computer to perform some operation(s) when you run the program.

If you like, you can think of a statement as a sentence and the semicolon as a
period. When you're told to write a C\texttt{++} statement, don't forget the 
semicolon!!! 
That would be like writing a sentence without a period (or question mark or 
exclamation mark) in an english essay.


\begin{ex} 
  \label{ex:some-decision1}
  \tinysidebar{\debug{exercises/{empty0/question.tex}}}
  \solutionlink{sol:some-decision1}
  \qed
\end{ex} 
\begin{python0}
from solutions import *
add(label="ex:some-decision1",
    srcfilename='exercises/some-decision1/answer.tex') 
\end{python0}


\begin{ex} 
  \label{ex:some-decision1}
  \tinysidebar{\debug{exercises/{empty0/question.tex}}}
  \solutionlink{sol:some-decision1}
  \qed
\end{ex} 
\begin{python0}
from solutions import *
add(label="ex:some-decision1",
    srcfilename='exercises/some-decision1/answer.tex') 
\end{python0}


See what I mean by this advice:

\begin{itemize}
\item[]
\textit{A quick advice to ease the pain of learning your first programming 
language: Type the program 
\EMPHASIZE{exactly} as given. 
Even the spaces and blank lines must match my programs.}
\end{itemize}

Computers are dumb (and picky about details.) 
We are the smart ones. So ... when you communicate with your computer, 
you have to be exact and explicit in your programs.



\begin{ex} 
  \label{ex:some-decision1}
  \tinysidebar{\debug{exercises/{empty0/question.tex}}}
  \solutionlink{sol:some-decision1}
  \qed
\end{ex} 
\begin{python0}
from solutions import *
add(label="ex:some-decision1",
    srcfilename='exercises/some-decision1/answer.tex') 
\end{python0}

