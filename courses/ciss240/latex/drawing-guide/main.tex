%-*-latex-*-
%
% https://tex.stackexchange.com/questions/216245/tikz-node-as-character-in-text

\documentclass[a4paper,
               openany,oneside,chapterprefix,
               12pt,
               parskip=full]
              {scrbook}              
\setlength{\textwidth}{5in}
\setlength{\hoffset}{-0.25in}
\setlength{\marginparwidth}{2in}

\newcommand\helvetica{\renewcommand{\familydefault}{phv}}
\helvetica
\usepackage{showframe} % added to make it easier to see the layout

\usepackage{fancyvrb}
\usepackage{listings}

\usepackage{marginnote}
\usepackage{tikz}
\usetikzlibrary{calc}
\usetikzlibrary{fit}
\usetikzlibrary{tikzmark}
\usetikzlibrary{calc}
\usetikzlibrary{decorations.pathreplacing}
\usetikzlibrary{calc}
\usepackage{tikzpagenodes}

\DefineVerbatimEnvironment%
 {console}{Verbatim}
 {frame=single,fontsize=\footnotesize}

\tikzset{%
  tikzmarknode plain/.style={inner sep=0pt,rounded corners=0.0pt, thick},
  tikzmarknode red/.style={inner sep=2pt,thick, draw=red},
  tikzmarknode thickred/.style={inner sep=2pt,line width=0.1cm, draw=red},  
  tikzmarknode roundedthickred/.style={inner sep=2pt,rounded corners=2.0pt, line width=0.1cm, draw=red},
  every tikzmarknode/.append style={tikzmarknode plain},
}

\newcommand\bstt{\texttt{\char`\\}}

\newcommand\sidebox[3][0cm]{
  \renewcommand*{\marginfont}{\normalsize}
  \marginpar{\centering%
      \vspace{#1} % <<< ADDED FOR VERTICAL SHIFT
      \begin{tikzpicture}[overlay,remember picture, baseline=(#2.base)]
        \node[rectangle, draw=red, text width=4cm, minimum width=2cm, line width=0.1cm, inner sep=0.2cm, name=#2] (#2) 
             {
               \begin{minipage}[t]{4.0cm}
                 {#3}
               \end{minipage}
             };
      \end{tikzpicture}
    \renewcommand*{\marginfont}{\tiny}
}}

\lstnewenvironment{consolethree}[1][]
{
  \lstset{
    basicstyle=\ttfamily\footnotesize,
    breaklines=true,
    columns=fullflexible,
    keepspaces=true,
    frame=single,
    xleftmargin=3.4pt,
    xrightmargin=3.4pt,
    #1
  }%
}{}%

\newcommand*{\DrawArrow}[3][red]{%
    % #1 = draw options
    % #2 = left node
    % #3 = right node
  \begin{tikzpicture}[overlay,remember picture]
    \draw [-latex, #1, line width=0.1cm] (#2) to (#3);
  \end{tikzpicture}%
}
\newcommand*{\DrawArrowH}[3][red]{%
    % #1 = draw options
    % #2 = left node
    % #3 = right node
  \begin{tikzpicture}[overlay,remember picture]
    \draw [-latex, #1, line width=0.1cm] (#2.west |- #3) -- (#3);
  \end{tikzpicture}%
}
\newcommand*{\DrawArrowVH}[3][red]{%
    % #1 = draw options
    % #2 = left node
    % #3 = right node
  \begin{tikzpicture}[overlay,remember picture]
    \draw [-latex, #1, line width=0.1cm] (#2) |- (#3.east);
  \end{tikzpicture}%
}
\newcommand*{\DrawArrowHV}[3][red]{%
  \begin{tikzpicture}[overlay,remember picture]
    \draw [-latex, #1, line width=0.1cm] (#2.west) -| (#3);
  \end{tikzpicture}%
}
\newcommand*{\DrawArrowPoints}[4][red]{%
  \begin{tikzpicture}[overlay,remember picture]
    \draw [-latex, #1, line width=0.1cm] (#2) #4 to (#3);
  \end{tikzpicture}%
}



\begin{document}

\section{Creating a node in the main text body}

Here's a node: \tikzmarknode{end2}{DEFGHI}.
\\
I used \verb!\tikzmarknode{a}{DEFGHI}!.
The name of the node is \verb!a!.

Here's a node: \tikzmarknode[tikzmarknode red]{a}{DEFGHI}.
\\
I used \verb!\tikzmarknode[tikzmarknode red]{a}{DEFGHI}!.

Here's a node: \tikzmarknode[tikzmarknode thickred]{a}{DEFGHI}.
\\
I used \verb!\tikzmarknode[tikzmarknode thickred]{a}{DEFGHI}!.

Here's a node: \tikzmarknode[tikzmarknode roundedthickred]{a}{DEFGHI}.
\\
I used \verb!\tikzmarknode[tikzmarknode roundedthickred]{a}{DEFGHI}!.
 
Here's a node: \tikzmarknode{a}{{\textcolor{red}{\textbf{\Huge DEFGHI}}}}.
\\
I used \verb!\tikzmarknode{a}{{\textcolor{red}{\textbf{\Huge DEFGHI}}}}!.


\newpage
\section{Create a node in the margin with boxed text}

I'm going to create a tikz node in the margin using with a boxed text.\sidebox{b}{ABCDEF}
\\
I used \verb!\sidebox{b}{ABCDEF}!.
This is a tikz node named \verb!b!.

Here's how to create paragraphs.\sidebox{b}{ABCDEF \\ \\ GHIJKL}
\\
I used \verb!\sidebox{b}{ABCDEF \\ \\ GHIJKL}!.

\mbox{}\vspace{1in}
TEST TEST TEST:

I'm going to create a tikz node in the margin using with a boxed text.\sidebox{b}{ABCDEF}

\mbox{}\vspace{1in}
TEST TEST TEST:
I'm going to create a tikz node in the margin using with a boxed text.\sidebox[-1cm]{b}{ABCDEF}

\newpage
\section{Draw arrow}

Here's a node: \tikzmarknode{a}{ABCDEF}.\sidebox{b}{Point to ABCDEF}
\DrawArrow{b}{a}
\\
There's a sidebox pointing to the above node.
The \verb!\DrawArrow{b}{a}! draws an arrow from node \verb!b! to node \verb!a!.
\\
\LaTeX:\vspace{-0.3cm}
\begin{console}
Here's a node: \tikzmarknode{a}{ABCDEF}.\sidebox{b}{Point to ABCDEF}
\DrawArrow{b}{a} 
\end{console}

\newpage
\textbf{Example.} Sidebox is lower.
\\
\tikzmarknode{a}{ABCDEF}
dummy text dummy text dummy text dummy text dummy text dummy text 
dummy text dummy text dummy text dummy text dummy text dummy text 
dummy text dummy text dummy text dummy text dummy text dummy text 
\sidebox{b}{Point to ABCDEF}\DrawArrow{b}{a}
\\
\LaTeX:\vspace{-0.3cm}
\begin{console}
\tikzmarknode{a}{ABCDEF}
dummy text dummy text dummy text dummy text dummy text dummy text 
dummy text dummy text dummy text dummy text dummy text dummy text 
dummy text dummy text dummy text dummy text dummy text dummy text 
\sidebox{b}{Point to ABCDEF}\DrawArrow{b}{a}
\end{console}

\newpage
\textbf{Example.} Sidebox is higher.
\\
\sidebox{b}{Point to ABCDEF}
dummy text dummy text dummy text dummy text dummy text dummy text 
dummy text dummy text dummy text dummy text dummy text dummy text 
dummy text dummy text dummy text dummy text dummy text dummy text 
\tikzmarknode{a}{ABCDEF}\DrawArrow{b}{a}
\\
\LaTeX:\vspace{-0.3cm}
\begin{console}
\tikzmarknode{a}{ABCDEF}
dummy text dummy text dummy text dummy text dummy text dummy text 
dummy text dummy text dummy text dummy text dummy text dummy text 
dummy text dummy text dummy text dummy text dummy text dummy text 
\sidebox{b}{Point to ABCDEF}\DrawArrow{b}{a}
\end{console}

\newpage
\textbf{Example.} Sidebox is tall.
\\
\tikzmarknode{a}{ABCDEF}\sidebox{b}{Point to ABCDEF. dummy text dummy text dummy text}
\DrawArrow{b}{a}

dummy text dummy text

You can force the arrow to be horizontal using \verb!DrawArrowH!

\tikzmarknode{a}{ABCDEF}\sidebox{b}{Point to ABCDEF. dummy text dummy text dummy text}
\DrawArrowH{b}{a}
\\
\LaTeX:\vspace{-0.3cm}
\begin{console}
\tikzmarknode{a}{ABCDEF}\sidebox{b}{Point to ABCDEF. dummy text dummy text dummy text}
\DrawArrowH{b}{a}
\end{console}

\newpage
\textbf{Example.} You can draw an arrow vertical-then-horizontal.

\sidebox{a}{Point to ABCDEF}

dummy text dummy text dummy text dummy text
dummy text dummy text dummy text dummy text
dummy text dummy text dummy text dummy text

Here's a node: \tikzmarknode{b}{ABCDEF}.
\DrawArrowVH{a}{b}
\\
\LaTeX:\vspace{-0.3cm}
\begin{console}
\sidebox{a}{Point to ABCDEF}

dummy text dummy text dummy text dummy text
dummy text dummy text dummy text dummy text
dummy text dummy text dummy text dummy text

Here's a node: \tikzmarknode{b}{ABCDEF}.
\DrawArrowVH{a}{b}
\end{console}

\newpage
\textbf{Example.} You can draw an arrow horizontal-then-vertical.

\sidebox{a}{Point to ABCDEF}

dummy text dummy text dummy text dummy text
dummy text dummy text dummy text dummy text
dummy text dummy text dummy text dummy text

Here's a node: \tikzmarknode{b}{ABCDEF}.
\DrawArrowHV{a}{b}
\\
\LaTeX:\vspace{-0.3cm}
\begin{console}
\sidebox{a}{Point to ABCDEF}

dummy text dummy text dummy text dummy text
dummy text dummy text dummy text dummy text
dummy text dummy text dummy text dummy text

Here's a node: \tikzmarknode{b}{ABCDEF}.
\DrawArrowHV{a}{b}
\end{console}


\newpage
\textbf{Example.}
You can specify displacement points for the arrow to follow using \verb!DrawArrowPoints!.
Here's an example:

Here's a node: \tikzmarknode{cccc}{CCCC}.\sidebox{dddd}{DDDD}
\DrawArrowPoints{dddd}{cccc}{-- ++(1,2.5) -- ++(-1.5,-1)}

\LaTeX:\vspace{-0.3cm}
\begin{console}
Here's a node: \tikzmarknode{cccc}{CCCC}.\sidebox{dddd}{DDDD}
\DrawArrowPoints{dddd}{cccc}{-- ++(1,2.5) -- ++(-1.5,-1)}
\end{console}

The \verb!--! means \lq\lq line".
The \verb!++! means \lq\lq displace by".
The \verb!-- ++(1,2.5)! means \lq\lq draw a line up to a point that is $(1,2.5)$ from
the previous point".

Instead of displacements, the points can be nodes.
Here's an example:

Here's a node: \tikzmarknode{cccc}{CCCC}.

Here's another node .................... \tikzmarknode{eeee}{EEEE}.

And another node \tikzmarknode{ffff}{FFFF}.

\sidebox{dddd}{DDDD}
\DrawArrowPoints{dddd}{cccc}{-- ++(1,2.5) -- (eeee) -- (ffff)}

\LaTeX:\vspace{-0.3cm}
\begin{console}
Here's a node: \tikzmarknode{cccc}{CCCC}.

Here's another node .................... \tikzmarknode{eeee}{EEEE}.

And another node \tikzmarknode{ffff}{FFFF}.\sidebox{dddd}{DDDD}
\DrawArrowPoints{dddd}{cccc}{-- ++(1,2.5) -- (eeee) -- (ffff)}
\end{console}

\newpage
\textbf{Example.}
You can have multiple sideboxes pointing to one node.

Here's a node: \tikzmarknode{a}{A}.\sidebox{b}{B}

dummy text

\sidebox{c}{C}

dummy text

\sidebox{d}{D}

\DrawArrow{b}{a}
\DrawArrow{c}{a}
\DrawArrowHV{d}{a}
\\
\LaTeX:\vspace{-0.3cm}
\begin{console}
Here's a node: \tikzmarknode{a}{A}.\sidebox{b}{B}

dummy text

\sidebox{c}{C}

dummy text

\sidebox{d}{D}

\DrawArrow{b}{a}
\DrawArrow{c}{a}
\DrawArrowHV{d}{a}
\end{console}

\newpage
\textbf{Example.}
You can have a sidebox pointing to multiple nodes.

Here are some nodes:
\tikzmarknode{a}{A}

\tikzmarknode{b}{B}\sidebox{c}{C}

\tikzmarknode{d}{D}
\DrawArrow{c}{a}
\DrawArrow{c}{b}
\DrawArrow{c}{d}
\\
\LaTeX:\vspace{-0.3cm}
\begin{console}
Here are some nodes:
\tikzmarknode{a}{A}

\tikzmarknode{b}{B}\sidebox{c}{B}

\tikzmarknode{c}{C}.
\DrawArrow{c}{b}
\DrawArrow{c}{d}
\end{console}




\newpage
\section{Tikz node inside code}
Here's an example where the tikz node is inside code:
\sidebox{b}{CCC}
\begin{consolethree}[escapeinside=||]
#include <iostream>

int main()  
{
    |\tikzmarknode{a}{return}| 0; // a is a node containing return
}
\end{consolethree}
\DrawArrow{b}{a}%
\LaTeX:\vspace{-0.3cm}
\begin{console}
\sidebox{b}{CCC}
\begin{consolethree}[escapeinside=||]
#include <iostream>

int main()  
{
    |\tikzmarknode{a}{return}| 0; // a is a node containing return
}
\end{consolethree}
\DrawArrow{b}{a}
\end{consolethree}
\DrawArrow[red]{start12}{end12}
\end{console}

NOTE: \verb!console! and \verb!Verbatim! cannot be used.
\verb!consolethree! uses the \verb!listings! package.
The \verb!escapeinside||! basically means whatever is within \verb!|...|!
is treated as latex command.
If the C\texttt{++} or python code contains \verb!|!, use another
pair of characters.

\newpage
\textbf{Example:}
Here's an example with tikz node inside code and the node
contains a backslash character:
\sidebox{start13}{Point to a string}
\begin{consolethree}[escapeinside=||]
#include <iostream>

int main()  
{
    std::cout << |\tikzmarknode[tikzmarknode thickred]{end13}{"hello world\char`\\n"}|;
    return 0;
}
\end{consolethree}
\DrawArrow[red]{start13}{end13}
\LaTeX:\vspace{-0.3cm}
\begin{console}
\sidebox{a}{Point to a string}
\begin{consolethree}[escapeinside=||]
#include <iostream>

int main()  
{
    std::cout << |\tikzmarknode[tikzmarknode thickred]{b}{"hello world\char`\\n"}|;
    return 0;
}
\end{consolethree}
\DrawArrow{a}{b}
\end{console}

Note the proper way of inserting the backslash character for the newline character.

To make it easier, I have a macro for the backslash using typewriter font: \verb!\bstt!
(memory aid: bstt = backslash typewriter text).

\textbf{Example.} ABC\bstt DEF is done using \verb!ABC\bstt DEF!.

\newpage
\textbf{Example:}
Here's the earlier example using the \verb!\bstt! macro:

\sidebox{a}{Point to a string}
\begin{consolethree}[escapeinside=||]
#include <iostream>

int main()  
{
    std::cout << |\tikzmarknode[tikzmarknode thickred]{b}{"hello world\bstt n"}|;
    return 0;
}
\end{consolethree}
\DrawArrow{a}{b}
\LaTeX:\vspace{-0.3cm}
\begin{console}
\sidebox{a}{Point to a string}
\begin{consolethree}[escapeinside=||]
#include <iostream>

int main()  
{
    std::cout << |\tikzmarknode[tikzmarknode thickred]{b}{"hello world\bstt n"}|;
    return 0;
}
\end{consolethree}
\DrawArrow[{a}{b}
\end{console}


\newpage
\textbf{Example:}
You might want to align a sidebox with a specific line of code.
The best is to put the sidebox next to that line
(and maybe put the draw arrow there as well).

\begin{consolethree}[escapeinside=||]
#include <iostream>

int main()  
{
    std::cout << |\tikzmarknode[tikzmarknode thickred]{b}{"hello world\bstt n"};||\sidebox{a}{Point to a string}|
    return 0;
}
\end{consolethree}
\DrawArrow{a}{b}
\\
\LaTeX:\vspace{-0.3cm}
\begin{console}[fontsize=\tiny]
\begin{consolethree}[escapeinside=||]
#include <iostream>

int main()  
{
    std::cout << |\tikzmarknode[tikzmarknode thickred]{b}{"hello world\bstt n"};||\sidebox{a}{Point to a string}|
    return 0;
}
\end{consolethree}
\DrawArrow{a}{b}
\end{console}
It might not be completely aligned. You can then shift the sidebox vertically -- see next section.

Also, note that since the sidebox is drawn inside \verb!consolethree!, the font used is typewriter font.
In the sidebox, you switch to normal font using \verb!\textnormal!.
\begin{consolethree}[escapeinside=||]
#include <iostream>

int main()  
{
    std::cout << |\tikzmarknode[tikzmarknode thickred]{b}{"hello world\bstt n"}\sidebox{a}{\textnormal{Point to a string}}|;
    return 0;
}
\end{consolethree}
\DrawArrow{a}{b}
\\
\LaTeX:\vspace{-0.3cm}
\begin{console}[fontsize=\tiny]
\begin{consolethree}[escapeinside=||]
#include <iostream>

int main()  
{
    std::cout << |\tikzmarknode[tikzmarknode thickred]{b}{"hello world\bstt n"}\sidebox{a}{\textnormal{Point to a string}}|;
    return 0;
}
\end{consolethree}
\DrawArrow{a}{b}
\end{console}


\newpage
\section{Shift sidebox vertically}
You can move the sidebox vertically.
Here's an example where the sidebox is not shifted:

Here's a node \tikzmarknode{b}{AAAA}.\sidebox{a}{BBBB}
\DrawArrow{a}{b}

dummy text dummy text dummy text dummy text
dummy text dummy text dummy text dummy text
dummy text dummy text dummy text dummy text

Here's the same example where the sidebox moved up by 1.5cm.

Here's a node \tikzmarknode{b}{AAAA}.\sidebox[-1.5cm]{a}{BBBB}
\DrawArrow{a}{b}

\LaTeX:\vspace{-0.3cm}
\begin{console}
Here's a node \tikzmarknode{b}{AAAA}\sidebox[-1.5cm]{a}{BBBB}.
\DrawArrow{a}{b}
\end{console}
\DrawArrow{a}{b}
Note the \verb![-1.5cm]!.

\newpage
\textbf{Example:}
\begin{consolethree}[escapeinside=||]
#include <iostream>

int main()  
{
    std::cout << |\tikzmarknode[tikzmarknode thickred]{b}{"hello world\bstt n"}|;|\sidebox[-0.2cm]{a}{\textnormal{Point to a string}}|
    return 0;
}
\end{consolethree}
\DrawArrow{a}{b}
\\
\LaTeX:\vspace{-0.3cm}
\begin{console}[fontsize=\tiny]
\begin{consolethree}[escapeinside=||]
#include <iostream>

int main()  
{
    std::cout << |\tikzmarknode[tikzmarknode thickred]{b}{"hello world\bstt n"}\sidebox[-0.5cm]{a}{\textnormal{Point to a string}}|;
    return 0;
}
\end{consolethree}
\DrawArrow{a}{b}
\end{console}

\newpage
\section{Reuse tikz node names}

You can reuse tikz node names.


\newpage
\section{Warning on disappearing tikz diagrams}

Watch out: if a tikz diagram is too small and too close to the bottom of a page,
the diagram might be accidentally truncated/removed by latex.


\newpage
\section{Minor variations on draw arrow}


\tikzmarknode{b}{B}\sidebox{a}{A}\DrawArrow[dashed]{a}{b}
\\
\LaTeX:\vspace{-0.3cm}
\begin{console}
\tikzmarknode{b}{B}\sidebox{a}{B}\DrawArrow[dashed]{a}{b}
\end{console}

\tikzmarknode{b}{B}\sidebox{a}{A}\DrawArrow[dashed, red]{a}{b}
\\
\LaTeX:\vspace{-0.3cm}
\begin{console}
\tikzmarknode{b}{B}\sidebox{a}{A}\DrawArrow[dashed, red]{a}{b}
\end{console}

\tikzmarknode{b}{A}\sidebox{a}{A}\DrawArrow[blue]{a}{b}
\\
\LaTeX:\vspace{-0.3cm}
\begin{console}
\tikzmarknode{b}{A}\sidebox{a}{A}\DrawArrow[blue]{a}{b}
\end{console}


\end{document}

