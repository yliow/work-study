%-*-latex-*-
Uh-oh \dots another ASCII art program.

\resett
\nextt
\begin{console}[frame=single, commandchars=\\\{\}]
\underline{1}
  *
 ***
*****
\end{console}

\nextt
\begin{console}[frame=single, commandchars=\\\{\}]
\underline{2}
     *
    ***
   *****
  *     *
 ***   *** 
***** *****
\end{console}

\nextt
\begin{console}[frame=single, commandchars=\\\{\}]
\underline{3}
        *
       ***
      *****
     *     *
    ***   *** 
   ***** *****
  *     *     *
 ***   ***   ***
***** ***** *****
\end{console}

\nextt
\begin{console}[frame=single, commandchars=\\\{\}]
\underline{4}
           *
          ***
         *****
        *     *
       ***   *** 
      ***** *****
     *     *     *
    ***   ***   ***
   ***** ***** *****
  *     *     *     *
 ***   ***   ***   ***
***** ***** ***** *****
\end{console}


\newpage
\textsc{Spoiler Warning \dots Incoming Hints \dots}

Look at the case of \textsc{Test 4}:

\textsc{Test 4}
\begin{console}[frame=single, commandchars=\\\{\}]
\underline{4}
           *
          ***
         *****
        *     *
       ***   *** 
      ***** *****
     *     *     *
    ***   ***   ***
   ***** ***** *****
  *     *     *     *
 ***   ***   ***   ***
***** ***** ***** *****
\end{console} 

There are two possible outermost for-loop: 

Either
\begin{verbatim}
        for i = 1, 2, ..., 12
            [some code to draw 1 line]
\end{verbatim}

or
\begin{verbatim}
        for i = 1, 2, 3, 4:
            [some code to draw 3 lines]
\end{verbatim}

In general, you can design your pseudocode as either
\begin{verbatim}
        get value for n from user
        for i = 1, 2, ..., 3 * n:
            [some code to draw 1 line]
\end{verbatim}
or
\begin{verbatim}
        get value for n from user
        for i = 1, 2, ..., n:
            [some code to draw 3 lines]
\end{verbatim}

It doesn't really matter which for--loop you use. You can solve this program
using either form. The solution I will give uses the second form. For each $i$,
you print a layer of triangles. For instance, in the case of $n = 4$, when
$i = 3$, your program should print the third layer of triangles:
\begin{console}
...
     *     *     *
    ***   ***   ***
   ***** ***** *****
...
\end{console}
(layers $1$, $2$ and $4$ not shown.) 

The pseudocode should look something like this:

\begin{Verbatim}[commandchars=\\\{\}]
        get a value for n from user

        // this is the number of spaces to the left before
        // anything is printed on each line
        num_spaces = ???

        for i in 1, 2, 3, 4, …, n:
            print num_spaces of ' '
            num_spaces--
            print first line of the i-th layer of triangles

            print num_spaces of ' '
            num_spaces--
            print second line of the i-th layer of triangles

            print num_spaces of ' '
            num_spaces--
            print third line of the i-th layer of triangles
\end{Verbatim}
		
[Think about this: how many triangles are there at layer \verb!i!?]
