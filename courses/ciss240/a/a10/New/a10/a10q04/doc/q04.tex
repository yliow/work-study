%-*-latex-*-
A perfect number is a positive integer which is the sum of its divisors
strictly less than itself. For instance, the positive divisors of $6$ are
exactly $1$, $2$, $3$, $6$. The positive divisors which are strictly less than
$6$ are $1$, $2$, $3$. Therefore, the sum of the positive divisors which are
strictly less than $6$ is $1 + 2 + 3$  which is $6$. Here are some examples:

\begin{longtable}{|l|l|l|}
\hline
$n$ & divisors of $n$ strictly less than $n$ & sum of divisors strictly less
than $n$ \\
\hline 
1 &   & 0 \\
2 & 1 & 1 \\
3 & 1 & 1 \\
4 & 1,2 & 3 \\
5 & 1 & 1 \\
6 & 1,2,3 & 6 \\
7 & 1 & 1 \\
8 & 1,2,4 & 7 \\
9 & 1,3 & 4 \\
10 & 1,2,5 & 8 \\
11 & 1 & 1 \\
12 & 1,2,3,4,6 & 16 \\
\hline
\end{longtable}

Note that the sum of divisors of $n$ which are strictly less than $n$ can be
less than $n$, equal to $n$, or greater than $n$. The goal of this program is
to prompt the user for $a$, $b$ and for each integer $n$ from $a$ to $b$
(inclusive), print the sum of divisors which are strictly less than $n$, and
\lq\lq perfect\rq\rq if $n$ is a perfect number, and print the total number of
perfect numbers found.

Note that you must not manually enter known perfect numbers in your code. Your
code must search for them. For instance, from the above you know that $6$ is a
perfect number. You code must not contain something similar to this pseudocode:
\lq\lq If $n$ is $6$, print that it is a perfect number without checking its
divisors.\rq\rq

You will see from your tests that perfect numbers are very rare. Although the
ancient greek mathematicians and philosophers were interested in perfect
numbers, before Euler, they did not know of too many: only $4$. That was quite
an amazing feat without computers. It probably took them hundreds of years to
find the $4$--th perfect number. Yet, with the right programming skills, we
can discover all the first $4$ perfect numbers in a second.

Do you see something common among all the perfect numbers in the above test
cases? With the help of your program, can you discover the $5$--th perfect
number? Despite the simplicity of the definition of perfect numbers, there are
still problems about perfect numbers which are not solved. For instance, it is
not known if there are odd perfect numbers. [You can find out more about
perfect numbers on \href{http://www.en.wikipedia.org}{Wikipedia}.]

\resett
\nextt
\begin{console}[frame=single, commandchars=\\\{\}]
\userinput{1 12}
perfect number(s): 6
number of perfect numbers found: 1
\end{console}

\nextt
\begin{console}[frame=single, commandchars=\\\{\}]
\userinput{10 20}
number of perfect numbers found: 0
\end{console}

\nextt
\begin{console}[frame=single, commandchars=\\\{\}]
\userinput{1 30}
perfect number(s): 6 28
number of perfect numbers found: 2
\end{console}

\nextt
\begin{console}[frame=single, commandchars=\\\{\}]
\userinput{1 100}
perfect number(s): 6 28
number of perfect numbers found: 2
\end{console}

\nextt
\begin{console}[frame=single, commandchars=\\\{\}]
\userinput{1 1000}
perfect number(s): 6 28 496
number of perfect numbers found: 3
\end{console}

\nextt
\begin{console}[frame=single, commandchars=\\\{\}]
\userinput{1 10000}
perfect number(s): 6 28 496 8128
number of perfect numbers found: 4
\end{console}
