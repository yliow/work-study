%-*-latex-*-
The following is a method for generating primes up to a given upper bound.
For our first example, suppose we want to compute the primes up to 99.
First, we write down the integers from $0$ to $99$:

\begin{python}
from latextool_basic import *
p = Plot()

c = RectContainer(x=1, y=1, align='bottom', direction='top-to-bottom')
for row in range(10):
    c0 = RectContainer(x=1, y=1, align='bottom', direction='left-to-right')
    for col in range(10):
        c0 += Rect2(x0=0, y0=0, x1 = 1.6, y1 = 0.6, linewidth=0.02,
                    label = r'$%s$' % (10 * row + col))
    c += c0
    c0.x = c0.x0; c0.y = c0.y0; c0.layout()

p += c
print(p)
\end{python}

First of all $0$ and $1$ are not primes: By definition a prime is at least
$2$. So first cross out $0$ and $1$:

\begin{python}
from latextool_basic import *
p = Plot()

c = RectContainer(x=1, y=1, align='bottom', direction='top-to-bottom')
for row in range(10):
    c0 = RectContainer(x=1, y=1, align='bottom', direction='left-to-right')
    for col in range(10):
        if row == 0 and col < 2:
            c0 += Rect2(x0=0, y0=0, x1 = 1.6, y1 = 0.6, linewidth=0.02,
                        background='black!20!white', 
                        label = r'$%s$' % (10 * row + col))
        else:
            c0 += Rect2(x0=0, y0=0, x1 = 1.6, y1 = 0.6, linewidth=0.02, 
                        label = r'$%s$' % (10 * row + col))
    c += c0
    c0.x = c0.x0; c0.y = c0.y0; c0.layout()

p += c
print(p)
\end{python}

Now we begin the actual algorithm. The first number that is not crossed
out is $2$. Cross out all the multiples of $2$ in the above table except
for $2$ itself. That means crossing out $4, 6, 8, 10, \dots$ in the list.
Think of $2$ as a prime that kills off numbers which are not primes because
they are divisible by $2$; $2$ is a killer prime.

\begin{python}
from latextool_basic import *
p = Plot()

c = RectContainer(x=1, y=1, align='bottom', direction='top-to-bottom')
for row in range(10):
    c0 = RectContainer(x=1, y=1, align='bottom', direction='left-to-right')
    for col in range(10):
        val = 10 * row + col
        if (row == 0 and col < 2) or (val % 2 == 0 and val != 2):
            c0 += Rect2(x0=0, y0=0, x1 = 1.6, y1 = 0.6, linewidth=0.02,
                        background='black!20!white', 
                        label = r'$%s$' % val)
        else:
            c0 += Rect2(x0=0, y0=0, x1 = 1.6, y1 = 0.6, linewidth=0.02, 
                        label = r'$%s$' % val)
    c += c0
    c0.x = c0.x0; c0.y = c0.y0; c0.layout()

p += c
print(p)
\end{python}

Now we move from $2$ to the next number that is not crossed out. From the
table you see that it is $3$: $3$ is the next killer prime. We now cross
out numbers which are multiples of $3$ except for $3$ itself. Therefore we
cross out $6, 9, 12, 15, \dots$ . (Note that $6$ is crossed out twice but
that's fine. This basically means that $6$ is not prime because both $2$
and $3$ divide $6$.) We get the following:

\begin{python}
from latextool_basic import *
p = Plot()

c = RectContainer(x=1, y=1, align='bottom', direction='top-to-bottom')
for row in range(10):
    c0 = RectContainer(x=1, y=1, align='bottom', direction='left-to-right')
    for col in range(10):
        val = 10 * row + col
        if (row == 0 and col < 2) or (val % 2 == 0 and val != 2) or \
           (val % 3 == 0 and val != 3):
            c0 += Rect2(x0=0, y0=0, x1 = 1.6, y1 = 0.6, linewidth=0.02,
                        background='black!20!white', 
                        label = r'$%s$' % val)
        else:
            c0 += Rect2(x0=0, y0=0, x1 = 1.6, y1 = 0.6, linewidth=0.02, 
                        label = r'$%s$' % val)
    c += c0
    c0.x = c0.x0; c0.y = c0.y0; c0.layout()

p += c
print(p)
\end{python}

We now move from $3$ to the next number that is not crossed out. Note that
in this case $4$ is already crossed out. The number after that is $5$: $5$
is the next killer prime. We now cross out all the multiples of $5$ except
for $5$ itself, i.e., we cross out $10, 15, 20, \dots$ and we get this:

\begin{python}
from latextool_basic import *
p = Plot()

c = RectContainer(x=1, y=1, align='bottom', direction='top-to-bottom')
for row in range(10):
    c0 = RectContainer(x=1, y=1, align='bottom', direction='left-to-right')
    for col in range(10):
        val = 10 * row + col
        if (row == 0 and col < 2) or (val % 2 == 0 and val != 2) or \
           (val % 3 == 0 and val != 3) or (val % 5 == 0 and val != 5):
            c0 += Rect2(x0=0, y0=0, x1 = 1.6, y1 = 0.6, linewidth=0.02,
                        background='black!20!white', 
                        label = r'$%s$' % val)
        else:
            c0 += Rect2(x0=0, y0=0, x1 = 1.6, y1 = 0.6, linewidth=0.02, 
                        label = r'$%s$' % val)
    c += c0
    c0.x = c0.x0; c0.y = c0.y0; c0.layout()

p += c
print(p)
\end{python}

We move on to the first number after $5$ that is not crossed out. In this
case the next killer prime is $7$. We cross out all the multiples of $7$
except for $7$ itself to obtain:

\begin{python}
from latextool_basic import *
p = Plot()

c = RectContainer(x=1, y=1, align='bottom', direction='top-to-bottom')
for row in range(10):
    c0 = RectContainer(x=1, y=1, align='bottom', direction='left-to-right')
    for col in range(10):
        val = 10 * row + col
        if (row == 0 and col < 2) or (val % 2 == 0 and val != 2) or \
           (val % 3 == 0 and val != 3) or (val % 5 == 0 and val != 5) or \
           (val % 7 == 0 and val != 7):
            c0 += Rect2(x0=0, y0=0, x1 = 1.6, y1 = 0.6, linewidth=0.02,
                        background='black!20!white', 
                        label = r'$%s$' % val)
        else:
            c0 += Rect2(x0=0, y0=0, x1 = 1.6, y1 = 0.6, linewidth=0.02, 
                        label = r'$%s$' % val)
    c += c0
    c0.x = c0.x0; c0.y = c0.y0; c0.layout()

p += c
print(p)
\end{python}

The first number after $7$ that is not crossed out is $11$. Note that this
algorithm terminates when you reach a killer prime that is greater than the
square root of the upper bound which in this case is $99$. The square root
of $99$ is approximately $9.9489\dots$ . At this point, the number we're
looking at is $11$ which is greater than $9.9489\dots$ . Hence the algorithm
stops. Why? If you look at killer prime $11$, you notice that the multiples
it crosses out are $22, 33, 44, 55, \dots, 99$. They are already crossed out
by primes less than $11$. The fact that in this algorithm the primes larger
than the square root of $99$ will only cross out numbers which are already
crossed out can be proven mathematically (although I won't do it here.)

The numbers in the table that are not crossed out must be primes. From the
table we see that these numbers are:
\[
2, 3, 5, 7, 11, 13, 17, 19, 23, 29, 31, 37, 41, 43, 47, 53, 59, 61, 67, 71,
73, 79, 83, 89, 97
\]
There are $25$ primes less than $100$.

This method works for any upper bound $n$. For instance, to compute all the
primes upto $n = 250$,  you write down a table of integers from $0$ to $250$,
cross out $0$ and $1$, cross out all the multiples of $2$ except for $2$,
cross out the multiples of $3$ except for $3, \dots$ . In general, after each
crossing out, you move on to the next number that is not crossed out and
repeat the process. This algorithm terminates when the killer prime is larger
than the square root of the upper bound, i.e., $250$.

Write a function
\begin{Verbatim}[frame=single,fontsize=\footnotesize]
void es(int n, int prime[], int & len);
\end{Verbatim}
that prints the list of remaining integers at the end of each crossing out
stage. For instance, on calling \verb!es(99, prime, len)!, the function goes
through the above algorithm, and prints the following:

\texttt{2 3 4 5 6 7 8 9 10 11 12 13 14 15 16 17 18 19 20 21 22 23 24 25 26 27 28 29 30 31 32 33 34 35 36 37 38 39 40 41 42 43 44 45 46 47 48 49 50 51 52 53 54 55 56 57 58 59 60 61 62 63 64 65 66 67 68 69 70 71 72 73 74 75 76 77 78 79 80 81 82 83 84 85 86 87 88 89 90 91 92 93 94 95 96 97 98 99} \\
\texttt{2:\,\,2 3 5 7 9 11 13 15 17 19 21 23 25 27 29 31 33 35 37 39 41 43 45 47 49 51 53 55 57 59 61 63 65 67 69 71 73 75 77 79 81 83 85 87 89 91 93 95 97 99} \\
\texttt{3:\,\,2 3 5 7 11 13 17 19 23 25 29 31 35 37 41 43 47 49 53 55 59 65 67 71 73 77 79 83 85 89 91 95 97} \\
\texttt{5:\,\,2 3 5 7 11 13 17 19 23 29 31 37 41 43 47 49 53 59 67 71 73 77 79 83 85 89 91 97} \\
\texttt{7:\,\,2 3 5 7 11 13 17 19 23 29 31 37 41 43 47 53 59 61 67 71 73 79 83 89 97}


Note that you must first print all the number from $2$ to the upper bound,
then for each \lq\lq killer prime", print the killer prime and the resulting
array after the killer prime has removed the relevant integers, and, at the
end of the algorithm, the array \verb!prime! is set to contain the following
values:

\verb!    2 3 5 7 11 13 17 19 23 29 31 37 41 43 47 53 59 61 67 71 73 79 83 89 97!

and \verb!len! is set to $25$. The \verb!main()! that tests the function
\verb!es()! is

\begin{Verbatim}[frame=single,commandchars=\~\!\@]
int main()
{
    int n, len;
    int prime[1000];
    std::cin >> n;
    
    es(n, prime, len);
    std::cout << "\nprimes from 2 to " << n << ~redtext!"@:\n~redtext!"@; 
    for (int i = 0; i < len; i++)
    {
        std::cout << prime[i] << ' ';
    }
    std::cout << "len: " << len << std::endl;

    return 0;    
}
\end{Verbatim}


\resett
\nextt
\begin{console}[frame=single, commandchars=\\\{\}]
\userinput{10}
2 3 4 5 6 7 8 9 10
2: 2 3 5 7 9
3: 2 3 5 7

primes from 2 to 10:
2 3 5 7
len: 4
\end{console}

\nextt
\begin{console}[frame=single, commandchars=\\\{\}]
\userinput{20}
2 3 4 5 6 7 8 9 10 11 12 13 14 15 16 17 18 19 20
2: 2 3 5 7 9 11 13 15 17 19
3: 2 3 5 7 11 13 17 19

primes from 2 to 20:
2 3 5 7 11 13 17 19
len: 8
\end{console}

\nextt
\begin{console}[frame=single, commandchars=\\\{\}]
\userinput{50}
2 3 4 5 6 7 8 9 10 11 12 13 14 15 16 17 18 19 20 21 22 23 24 25 26 27 28 29
30 31 32 33 34 35 36 37 38 39 40 41 42 43 44 45 46 47 48 49 50
2: 2 3 5 7 9 11 13 15 17 19 21 23 25 27 29 31 33 35 37 39 41 43 45 47 49
3: 2 3 5 7 11 13 17 19 23 25 29 31 35 37 41 43 47 49
5: 2 3 5 7 11 13 17 19 23 29 31 37 41 43 47 49
7: 2 3 5 7 11 13 17 19 23 29 31 37 41 43 47

primes from 2 to 50:
2 3 5 7 11 13 17 19 23 29 31 37 41 43 47
len: 15
\end{console}

\nextt
\begin{console}[frame=single, commandchars=\\\{\}]
\userinput{99}
2 3 4 5 6 7 8 9 10 11 12 13 14 15 16 17 18 19 20 21 22 23 24 25 26 27 28 29
30 31 32 33 34 35 36 37 38 39 40 41 42 43 44 45 46 47 48 49 50 51 52 53 54
55 56 57 58 59 60 61 62 63 64 65 66 67 68 69 70 71 72 73 74 75 76 77 78 79
80 81 82 83 84 85 86 87 88 89 90 91 92 93 94 95 96 97 98 99
2: 2 3 5 7 9 11 13 15 17 19 21 23 25 27 29 31 33 35 37 39 41 43 45 47 49 51
53 55 57 59 61 63 65 67 69 71 73 75 77 79 81 83 85 87 89 91 93 95 97 99
3: 2 3 5 7 11 13 17 19 23 25 29 31 35 37 41 43 47 49 53 55 59 65 67 71 73
77 79 83 85 89 91 95 97
5: 2 3 5 7 11 13 17 19 23 29 31 37 41 43 47 49 53 59 67 71 73 77 79 83 85
89 91 97
7: 2 3 5 7 11 13 17 19 23 29 31 37 41 43 47 53 59 61 67 71 73 79 83 89 97

primes from 2 to 99:
2 3 5 7 11 13 17 19 23 29 31 37 41 43 47 53 59 61 67 71 73 79 83 89 97
len: 25
\end{console}

Note: The important thing about this function is that it computes and
stores primes in an array. The printing of the work done is not that
important. The only reason why I'm requiring the printing is to verify
that you are coding the algorithm correctly.

\newpage

\textsc{Spoiler Warning \dots Hints Ahead!!! Watchout!!!}

\textbf{HERE IT COMES \dots}

HINT:
 
Use an array \verb!x! of integers. Set all the values of \verb!x! to $0$.
Crossing out the number \verb!5! corresponds to the action of setting
\verb!x[5]! to $1$. In this case I'm using $0$ to mean
\lq\lq not crossed out" and $1$ to mean \lq\lq crossed out". You can switch
the meaning of $0$ and $1$. The only reason why I'm using $0$ to mean not
crossed out is because it's easier to initialize all values to $0$ rather
than $1$. (An array of boolean values can also be used since we only need to
store $2$ possible values in the array.)
