%-*-latex-*-
\input{myassignmentpreamble}
\input{ciss240}
\input{yliow}
\renewcommand\TITLE{Assignment 1}
\usepackage{import}

\begin{document}
\topmatter


\textsc{Objectives}
 \begin{myenum}
 \li Output string of characters including escape characters
 \end{myenum}
\mbox{}\\

\textsc{Instructions}
\begin{enumerate}[topsep=0pt,nosep]
\li Your program must be well-written. 
    You must follow the style in your notes as closely as possible. 
    Take note of the spaces and blank lines I used in my examples. 
    Badly written programs will very likely result in a poor grade for this 
    assignment. 
    Points will be taken off for sloppy work. 
\li It's important to remember this: In your printouts for all assignments, 
    there must be no wraparound.
\li All outputs must match exactly the output shown. 
    That includes every single space and every blank line.

\li The format of your program must look like this
(replacing \lq\lq smaug'' with your name of course!):
\begin{Verbatim}[frame=single,fontsize=\small]
// File: a01q01.cpp
// Name: smaug

#include <iostream>

int main()
{
    *** YOUR WORK HERE ***

    return 0;
}
\end{Verbatim}
In particular:
\begin{myenum}
\li You must have your name and the name of the file at the top of each 
    C++ source file as shown above.
\li Your C++ source file must end with a blank line.
\end{myenum}

\li Instructions on uploading your work will be provided in class.

\end{enumerate}


Read the questions carefully before diving in!

Note that you should create a new project for each question. 
For easy maintenance of your assignments, 
I suggest you have a folder \verb!ciss240! somewhere in your 
\verb!Documents!, and in that you have a folder \verb!a!, 
and in folder \verb!a! you have a folder \verb!a01!, 
and you have solutions folders \verb!a01q01!, \verb!a01q02!, etc. in the 
folder \verb!a01!:

\begin{Verbatim}
    .
    .
    .
    ciss240
    |
    +--- a
         |
         +--- a01
              |
              +--- a01q01
              |
              +--- a01q02
\end{Verbatim}

Note that the name for the C++ source file for question 1 
(i.e. the cpp file in 
project \verb!a01q01!) must be \verb!a01q01.cpp!, etc.


\newpage{Q1. }\subimport*{../a01q01/question/}{main.tex}
\newpage{Q2. }\subimport*{../a01q02/question/}{main.tex}
\newpage{Q3. }\subimport*{../a01q03/question/}{main.tex}
\newpage{Q4. }\subimport*{../a01q04/question/}{main.tex}
\newpage{Q5. }\subimport*{../a01q05/question/}{main.tex}

\end{document}
