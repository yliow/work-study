
The purpose of this assignment is to write functions. You will also see how
problems are broken down into subproblems (or subgoals). You have seen this
before in the context of breaking down a prograrm into different goals within
the code in \verb!main()!. The difference now is that we break down the
program into pieces and then create functions for the pieces instead of
keeping these pieces inside \verb!main()!.

The first problem (Q1 and Q2) involves writing a simple tool to analyze
the representation of primes using polynomials. A basic function is developed
in Q1 so that it can be used in Q2. The function checks if an integer is
prime.

Questions Q3--Q5 involve the rewrite of our calendar month printing
program. Q3 involves writing a function that tells us if a year is a leap
year. The second function (in Q4) tells us how many days there are in a
given month and year. Q4 obviously must use Q3 since the days in a month
for the case of February depends on whether the year is a leap year or not.
Finally, Q5 uses the functions in Q3--Q4 to print the calendar itself.

Q6 involves writing several functions to analyze the famous unsolved
\lq\lq3x+1" problem.

There are several short functions to be implemented in Q7. The functions are
then used to compute an approximation to the value of $\pi$ (which we know is
roughly $3.14159\ldots$) using a technique called Monte--Carlo simulation.
