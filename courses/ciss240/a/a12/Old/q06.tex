%-*-latex-*-
The following problem is called the \lq\lq 3x+1" problem. Consider the
following function $T$:
\begin{align*}
&\text{if $n$ is odd, $T(n) = 3n + 1$} \\
&\text{if $n$ is even, $T(n) = n / 2$}
\end{align*}

Another way to write this (mathematically) is:
\[
T(n) =
\begin{cases}
3n + 1, & \text{ if $n$ is odd} \\
n/2,    & \text{ if $n$ is even}
\end{cases}
\]

Note that there $T(n)$ is made up of two formulas; you need to choose the
right one. Here are some examples:
\begin{align*}
& T(42) = 42/2 = 21 & \hspace{2 cm}
& \text{(you choose the $n / 2$ formula since $n = 42$ is even)} \\
& T(5) = 3(5) + 1 = 16 & \hspace{2 cm}
& \text{(you choose the $3n + 1$ formula since $n = 5$ is odd)}
\end{align*}

Note that you can use the \lq\lq output" of $T(42)$ which is $21$ and
apply $T$ to it again: $T(21)$. 
\begin{align*}
T(42) &= 21 \\
T(21) &= 3(21) + 1 = 64
\end{align*}
and yet again:
\[T(64) = 64/2 = 32\]

Of course you can do this as many times as you like, using outputs as inputs:
\begin{align*}
T(42) &= 42/2 = 21 & &\text{because $42$ is even} \\
T(21) &= 3(21) + 1 = 64	& &\text{because $21$ is odd} \\
T(64) &= 64/2 = 32 & &\text{because $64$ is even} \\
T(32) &= 32/2 = 16 & &\text{because $32$ is even} \\
T(16) &= 8  & &\text{because $16$ is even} \\
T(8) &= 4 & &\text{because $8$ is even} \\
T(4) &= 2 & &\text{because $4$ is even} \\
T(2) &= 1 & &\text{because $2$ is even}
\end{align*}

OK \dots that's enough \dots let's stop at $1$. Altogether we applied $T$
eight times and in doing that we got a sequence of numbers:
\[42, 21, 64, 32, 16, 8, 4, 2, 1\]

The \lq\lq 3x+1" problem/conjecture states that if you apply enough $T$ to a
given strictly positive integer $n$, you will always get $1$.

Well, this is definitely the case for $42$. But is this true in general for
all strictly positive $n$?

The \lq\lq 3x+1" problem is still an open problem: No one has proven it yet.
(By the way, there is a \pounds$1000$ reward for proving it \dots in case you
need some money.) This problem is an example of a dynamical system.

Write a program that prompts the user for two integers and verifies the
conjecture for all integers in the range described by the two integers by
printing the sequences for each integer until it reaches $1$. (See explanation
in \textsc{Test 1}.)

The following skeleton must be used. 
\begin{Verbatim}[frame=single]
#include <iostream>

//------------------------------------------------------------------------
// The function T returns T(n) where:
//   if n is odd, T(n) = 3 * n + 1
//   if n is even, T(n) = n / 2 (integer division)
// (This function has no output.)
//------------------------------------------------------------------------
int T(int n)
{
    ... your code here ...
}


//------------------------------------------------------------------------
// The function print_3x_plus_1(n) prints the sequence of integers on
// applying T until 1 is reached.
// 
// Example: On calling print_3x_plus_1(3),
// 3 10 5 16 8 4 2 1
// is printed. 
//------------------------------------------------------------------------
void print_3x_plus_1(int n)
{
    ... your code here ...
}


int main()
{
    int a, b;
    std::cin >> a >> b;
    
    for (int n = a; n <= b; ++n)
    {
        print_3x_plus_1(n); 
    }

    return 0;
}
\end{Verbatim}


\resett
\nextt
\begin{console}[frame=single, commandchars=\\\{\}]
\userinput{1 10}
1
2 1
3 10 5 16 8 4 2 1
4 2 1
5 16 8 4 2 1
6 3 10 5 16 8 4 2 1
7 22 11 34 17 52 26 13 40 20 10 5 16 8 4 2 1
8 4 2 1
9 28 14 7 22 11 34 17 52 26 13 40 20 10 5 16 8 4 2 1
10 5 16 8 4 2 1
\end{console}
As you can see, the user enters $1$ and $10$ and the program runs $n$ from $1$
to $10$ and for each $n$, continually applies $T$ until $1$ is reached. For
instance, when $n$ is $6$, continually applying $T$ gives the sequence
$6, 3, 10, 5, 16, 8, 4, 2, 1$. In every case, the sequence does ultimately
reach $1$.

\nextt
\begin{console}[frame=single, commandchars=\\\{\}]
\userinput{11 20}
11 34 17 52 26 13 40 20 10 5 16 8 4 2 1
12 6 3 10 5 16 8 4 2 1
13 40 20 10 5 16 8 4 2 1
14 7 22 11 34 17 52 26 13 40 20 10 5 16 8 4 2 1
15 46 23 70 35 106 53 160 80 40 20 10 5 16 8 4 2 1
16 8 4 2 1
17 52 26 13 40 20 10 5 16 8 4 2 1
18 9 28 14 7 22 11 34 17 52 26 13 40 20 10 5 16 8 4 2 1
19 58 29 88 44 22 11 34 17 52 26 13 40 20 10 5 16 8 4 2 1
20 10 5 16 8 4 2 1
\end{console}
