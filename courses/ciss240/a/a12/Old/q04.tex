%-*-latex-*-
Write a function \verb!days_in_month()! such that
\verb!days_in_month(month, year)! returns the number of days in the given
\verb!month! (an \verb!int!) of the given \verb!year! (an \verb!int!).
(See $a06$.)

You must use the skeleton code given below. Note that the test cases are
included in the code.

You will need the function from Q3. Therefore you should copy-and-paste the
function from Q3 to the code for this question.
\begin{console}
#include <iostream>


//------------------------------------------------------------------------
// This function returns true if year is a leap year. Otherwise false is
// returned.
// (This function has no output.)
//------------------------------------------------------------------------
bool is_leap_year(int year)
{
    ... your code here ...
}


//------------------------------------------------------------------------
// This function returns the number of days in given month and year.
// (This function has no output.)
//------------------------------------------------------------------------
int days_in_month(int month, int year)
{
    ... your code here ...
}


int main()
{
    int m, y;
    std::cin >> m >> y;
    std::cout << days_in_month(m, y) << std::endl;

    return 0;
}
\end{console}

For your reference, here's a relevant program from a previous assignment:
\begin{Verbatim}[frame=single]
#include <iostream>

int main()
{
    int yyyymmdd = 0;
    std::cin >> yyyymmdd;
    
    int yyyy = yyyymmdd / 10000;    // year
    int mm = yyyymmdd / 100 % 100;  // month
    int dd = yyyymmdd % 100;        // day-of-month
        
    bool valid = false; // true exactly when yyyymmdd is a valid date
    
    if (1 <= mm && mm <= 12)
    {
        if (mm == 2)
        {
            //------------------------------------------------------------
            // CASE: February. 
            // There are 29 or 28 days depending on whether the year
            // is a leap year or not. 
            //------------------------------------------------------------
            if (yyyy % 4 == 0 
                && ((yyyy % 100 != 0) || (yyyy % 100 == 0) 
                    && (yyyy % 400 == 0)))
            {
                valid = (1 <= dd && dd <= 29);
            }
            else
            {
                valid = (1 <= dd && dd <= 28);
            }
        }
        else if ((mm % 2 == 1 && mm <= 7) || (mm % 2 == 0 && mm >= 8))
        {
            //------------------------------------------------------------
            // CASE: 31 days in months 1,3,5,7,8,10,12
            //------------------------------------------------------------
            valid = (1 <= dd && dd <= 31);
        }
        else
        {
            //------------------------------------------------------------
            // CASE: 30 days in months 4,6,9,11
            //------------------------------------------------------------
            valid = (1 <= dd && dd <= 30);
        }
    }
    
    if (valid) 
    {
        std::cout << "correct" << std::endl;
    }
    else
    {
        std::cout << "incorrect" << std::endl;
    }

    return 0;
}
\end{Verbatim}

Make sure you test your program thoroughly.

\resett
\nextt
\begin{console}[frame=single, commandchars=\\\{\}]
\userinput{2 2004}
29
\end{console}

\nextt
\begin{console}[frame=single, commandchars=\\\{\}]
\userinput{2 2003}
28
\end{console}

\nextt
\begin{console}[frame=single, commandchars=\\\{\}]
\userinput{5 2003}
31
\end{console}

\nextt
\begin{console}[frame=single, commandchars=\\\{\}]
\userinput{2 1900}
28
\end{console}

\nextt
\begin{console}[frame=single, commandchars=\\\{\}]
\userinput{9 2008}
30
\end{console}

\nextt
\begin{console}[frame=single, commandchars=\\\{\}]
\userinput{2 1200}
29
\end{console}
