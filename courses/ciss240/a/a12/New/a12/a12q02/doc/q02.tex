%-*-latex-*-
For this question, you should use the \verb!isprime()! function from $Q1$.

This question involves polynomials that generate primes. Consider the
following polynomial:
\[p(x) = x^2 + 2x + 3\]
Note that
\begin{align*}
	p(0) &= 0 * 0 + 2 * 0 + 3 = 3, \text{which is a prime} \\
	p(1) &= 1 * 1 + 2 * 1 + 3 = 6, \text{which is not a prime} \\
	p(2) &= 2 * 2 + 2 * 2 + 3 = 11, \text{which is a prime}
\end{align*}

Therefore this $p(x)$, when $x$ ranges from $0$ to $2$ (inclusive), generates
$2$ primes, i.e., $3$ and $11$.

Write a program that accepts $A, B, C, N$ and then analyzes all the polynomials
\[p(x) = a\,x^2 + b\,x + c\]
for $a$ ranging from $1$ to $A$ (inclusive), $b$ ranging from $1$ to $B$
(inclusive), $c$ ranging from $1$ to $C$ (inclusive) and reporting the number
of $x$ ranging from $0$ to $N$ (inclusive) such that $p(x)$ is prime. 

Finally, print the polynomial that represents the most primes. If there's a 
tie, report the one that is smallest according to \lq\lq dictionary order",
i.e., the one that appears first.

Note that all integers are printed with \verb!std::setw(5)!.

[An experiment for fun: Is there a polynomial $p(x)$ that represents primes for
\underline{\bf all} $x = 0, 1, 2, 3, \dots$? In general, suppose that the prime
length of a polynomial $p(x)$ is the number $n$ such that $p(0), p(1), p(2),
\ldots, p(n-1)$ are all primes. What is the largest prime length you can find
and what is the polynomial?]

\textsc{Test 1}
\begin{console}[frame=single, commandchars=\\\{\}]
\userinput{1 1 1 5}
    1x^2 +     1x +     1:     4
largest
    1x^2 +     1x +     1:     4
\end{console}
In this case $A = 1$, $B = 1$, $C = 1$, $N = 5$.

Therefore, we're only analyzing a single polynomial, i.e.,
$x^2 + x + 1$ for $x = 0, 1, 2, 3, 4, 5$.

Here is the analysis by hand:
\begin{longtable}{lll}	
$\underline{x}$ & \hspace{2cm}$\underline{x^2+x+1}$ & \hspace{2cm} \\
$0$ & \hspace{2cm}$0^2+0+1 = 1$ & \hspace{2cm}not prime \\
$1$ & \hspace{2cm}$1^2+1+1 = 3$ & \hspace{2cm}prime \\
$2$ & \hspace{2cm}$2^2+2+1 = 7$ & \hspace{2cm}prime \\
$3$ & \hspace{2cm}$3^2+3+1 = 13$ & \hspace{2cm}prime \\
$4$ & \hspace{2cm}$4^2+4+1 = 21$ & \hspace{2cm}not prime \\
$5$ & \hspace{2cm}$5^2+5+1 = 31$ & \hspace{2cm}prime \\
\end{longtable}
Hence, there are $4$ primes represented by this polynomial for $x$ running
from $0$ to $5$.

\resett
\nextt
\begin{console}[frame=single, commandchars=\\\{\}]
\userinput{2 2 2 10}
    1x^2 +     1x +     1:     6
    1x^2 +     1x +     2:     1
    1x^2 +     2x +     1:     0
    1x^2 +     2x +     2:     5
    2x^2 +     1x +     1:     5
    2x^2 +     1x +     2:     5
    2x^2 +     2x +     1:     6
    2x^2 +     2x +     2:     1
largest
    1x^2 +     1x +     1:     6
\end{console}
In this case, note that there is a tie: there are two polynomials that achieve
$6$ primes for $x$ running from $0$ to $10$. The earlier one is reported as the
\lq\lq winner".

\nextt
\begin{console}[frame=single, commandchars=\\\{\}]
\userinput{10 10 10 100}
    1x^2 +     1x +     1:    32
    1x^2 +     1x +     2:     1
    1x^2 +     1x +     3:    15
    1x^2 +     1x +     4:     0
    1x^2 +     1x +     5:    30
    1x^2 +     1x +     6:     0
    1x^2 +     1x +     7:    32
    1x^2 +     1x +     8:     0
    1x^2 +     1x +     9:    13
    1x^2 +     1x +    10:     0
    1x^2 +     2x +     1:     0
    1x^2 +     2x +     2:    19
    1x^2 +     2x +     3:    11
    1x^2 +     2x +     4:    17
    1x^2 +     2x +     5:    21
    1x^2 +     2x +     6:     7
\emph{... output not shown ...}
   10x^2 +    10x +     5:     1
   10x^2 +    10x +     6:     0
   10x^2 +    10x +     7:    27
   10x^2 +    10x +     8:     0
   10x^2 +    10x +     9:    11
   10x^2 +    10x +    10:     0
largest
    8x^2 +    10x +     1:    53
\end{console}
The output of this test case is too long to be shown in its entirety. Part of
it is snipped. Note that, in this case, the maximum density of primes
represented is $53/101$, about $52.4\%$. Can we find a better polynomial to
represent primes for $x = 0, \dots, 100$ if we expand our search?

\nextt
\begin{console}[frame=single, commandchars=\\\{\}]
\userinput{20 20 20 100}
\emph{... output not shown ...}
largest
    7x^2 +     7x +    17:    70
\end{console}
Now we have achieved a density of $70/101$, almost $70\%$.

\nextt
\begin{console}[frame=single, commandchars=\\\{\}]
\userinput{30 30 30 100}
\emph{... output not shown ...}
largest
    1x^2 +     23x +    23:    75
\end{console}
The maximum density is about $75\%$.

\nextt
\begin{console}[frame=single, commandchars=\\\{\}]
\userinput{50 50 50 100}
\emph{... output not shown ...}
largest
    1x^2 +     1x +    41:    87
\end{console}
Still improving!

\nextt
\begin{console}[frame=single, commandchars=\\\{\}]
\userinput{60 60 60 100}
\emph{... output not shown ...}
largest
    1x^2 +     1x +    41:    87
\end{console}
Oops \dots we still have the same polynomial.

\nextt
\begin{console}[frame=single, commandchars=\\\{\}]
\userinput{70 70 70 100}
\emph{... output not shown ...}
largest
    1x^2 +     1x +    41:    87
\end{console}
Wait \dots what's happening??? Expanding our brute force search doesn't come
up with a better polynomial \dots hmmm \dots

\nextt
\begin{console}[frame=single, commandchars=\\\{\}]
\userinput{80 80 80 100}
\emph{... output not shown ...}
largest
    1x^2 +     1x +    41:    87
\end{console}
Is this a coincidence???

\nextt
\begin{console}[frame=single, commandchars=\\\{\}]
\userinput{100 100 100 100}
\emph{... output not shown ...}
largest
    1x^2 +     1x +    41:    87
\end{console}
What!!! Is there something special about $x^2 + x + 41$?!?
