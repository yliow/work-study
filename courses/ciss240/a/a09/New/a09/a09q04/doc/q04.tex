%-*-latex-*-
Brute force search for integer solutions

Here's something from
\href{http://en.wikipedia.org/wiki/1729_\%28number\%29}{Wikipedia}:

\fbox{\begin{minipage}{0.97\textwidth}
1729 is known as the Hardy-Ramanujan number, after a famous anecdote of
the British mathematician
\href{http://en.wikipedia.org/wiki/G._H._Hardy}{G. H. Hardy} regarding a
hospital visit to the Indian mathematician
\href{http://en.wikipedia.org/wiki/Srinivasa_Ramanujan}{Srinivasa Ramanujan}.
In Hardy's words: \\

\lq\lq I remember once going to see him when he was ill at
\href{http://en.wikipedia.org/wiki/Putney}{Putney}. I had ridden in taxi cab
number $1729$ and remarked that the number seemed to me rather a dull one, and
that I hoped it was not an unfavorable \href{http://en.wikipedia.org/wiki/Omen}
{omen}. \lq\lq No,\rq\rq he replied,\lq\lq
it is a very interesting number; it is the smallest number expressible as the
sum of two cubes in two different ways.\rq\rq \rq\rq \\

Numbers such as
\[1729 = 1^3 + 12^3 = 9^3 + 10^3\]
that are the smallest number that can be expressed as the sum of two cubes in
$n$ distinct ways have been dubbed
\href{http://en.wikipedia.org/wiki/Taxicab_number}{taxicab numbers}. $1729$ is
the second taxicab number (the first is $2 = 1^3 + 1^3$). The number was also
found in one of Ramanujan's notebooks dated years before the incident.
\end{minipage}}

Write a program that prompts the user for an integer $z$ and finds all positive
(i.e., \textred{at least 1}) integer solutions to the equation
\[x^3 + y^3 = z\]
Do not list repeats. For instance when $z$ is $1729$
\begin{center}
$1^3 + 12^3 = 1729$ \\
$12^3 + 1^3 = 1729$ \\
\end{center}
But only $1, 12$ should be listed (i.e., do not list $12, 1$.)

[Hint: Write a double for-loop.]

\resett
\nextt
\begin{console}[commandchars=\\\{\}]
\userinput{1}
\end{console}

\nextt
\begin{console}[commandchars=\\\{\}]
\userinput{2}
1,1
\end{console}

\nextt
\begin{console}[commandchars=\\\{\}]
\userinput{1729}
1,12 9,10
\end{console}

\nextt
\begin{console}[commandchars=\\\{\}]
\userinput{2000}
10,10
\end{console}
