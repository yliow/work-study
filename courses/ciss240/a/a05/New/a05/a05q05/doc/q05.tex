%-*-latex-*-
The following is not a program. You need to write your work to a text file.
You can use Notepad (on Windows 7 or 8) or some other text editor
(such as emacs/xemacs in Linux) and save the file with the name
\verb!a05q05.txt!. (Talk to me or any of the senior CS tutors if you have never
used a text editor before.)

There is a \textit{boolean} expression below.
You need to evaluate it one operator at a
time according to the standard precedence and associative rules
until you obtain a boolean value.
Note that
type casting is considered a step. There is also a rule below on removing
parentheses. Here's how you should present your work in the text file. For the
expression \verb!1 + 2 < 3 * 4!, if you believe the \verb!+! goes first,
followed by the \verb!<!, and then the \verb!*!,  you write your answer like
this in the file:
\begin{console}
// Name: smaug
// File: a05q05.txt

1 + 2 < 3 * 4
= 3 < 3 * 4                                       by 1 + 2 = 3
= false * 4                                       by 3 < 3 = false
= 0 * 4                                           by int(false) = 0
= 0                                               by 0 * 4 = 0
= false                                           by bool(0) = false
\end{console}
Note that, in the above, there are \underline{two} type casts.

Of course the above is incorrect. The point is just to show you how to present
your work.

You need not follow the exactly column placement of the reasoning
that starts with \lq\lq by \dots\rq\rq. For instance, see another example
below where the placement is different.

Now I will show you how to remove parentheses. If you believe for the following
that the parenthesized expression should be evaluated first, followed by the
\verb!&&!, and finally by the \verb!||!, this is how you must present your
work:
\begin{console}
// Name: smaug
// File: a05q05.txt

true && (2 < 3) || false
= true && (true) || false                 by 2 < 3 = true
= true && true || false                   by (true) = true
= true || false                           by true && true = true
= true                                    by true || false = true
\end{console}
Note that the removal of the parentheses is a step of its own; look
at the above
computation very carefully.

Now for the question:

Evaluate the following expression according to the above
requirements:
\begin{Verbatim}[frame=single]
true || (2 + 3 < 5 - 1) && !(3 / 4 >= (5 + (2 - 1) * 2)) || (2 < 1)
\end{Verbatim}

\textsc{Grading.}
The grading of this question is very strict: make sure you
follow the above rules.
Furthermore, grading will stop once an error occurs.
For instance if you have typed 20 lines in your file and
the first time the grader discovers an error is at the 18--th line, then
your score is equivalent to 17 out of 20.
Check your work very carefully.
Extraneous spaces are removed.
For instance the grader will automatically change this file
\begin{console}
// Name: smaug
// File: a05q05.txt

1 + 2 < 3 * 4
= 3 < 3 * 4                                       by 1 + 2 = 3
= false * 4                                       by 3 < 3 = false
= 0 * 4                                           by int(false) = 0
= 0                                               by 0 * 4 = 0
= false                                           by bool(0) = false
\end{console}
to this:
\begin{console}
// Name: smaug
// File: a05q05.txt

1 + 2 < 3 * 4 
= 3 < 3 * 4 by 1 + 2 = 3
= false * 4 by 3 < 3 = false
= 0 * 4 by int(false) = 0
= 0 by 0 * 4 = 0
= false by bool(0) = false
\end{console}
You should still present your work with the word \lq\lq \verb!by!'' aligned
so that it's easier for you to check your work.
