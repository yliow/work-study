%-*-latex-*-
Q4. In many scientific and business applications, one is interested in solving a 
system of linear
equations. 
The following is an example of a linear system of 2 equations with 2 unknowns:
\begin{align*}
2x + 3y &= 8 \\
5x - y &= 3
\end{align*}
Check for yourself that $x = 1$, $y = 2$ is a solution. 
There are cases where such a system has no
solutions or more than one solution (in fact infinitely many). There are also situations where one is
interested in inequalities instead of equations:
\begin{align*}
2x + 3y &\leq 8 \\
5x -  y &\leq 3
\end{align*}
(In this case solving the system of inequalities is tied into optimizing 
another function. 
The inequalities
might represent a business or system constraints such as manpower and cost 
and the function to
optimize might be operational efficiency or profit. Such problems are called linear programming
problems.)

For many applications, doing computations by hand is in fact impossible. 
For instance an airline
reservation system, 5000 or more variables is typical. 
This exercise implements a system of two
equations with 2 unknowns for the case where there is exactly one solution. 
(We will not consider the
case where there are no solution or more than one solution.)

Given a system of 2 linear equations in 2 unknowns $x$ and $y$:
\begin{align*}
ax + by &= A \\
cx + dy &= B
\end{align*}
($a$, $b$, $c$, $d$, $A$, $B$ are given numbers), 
if it has a unique solution, then the solution is given by
\begin{align*}
x &= (Ad - bB) / \operatorname{det} \\
y &= (aB - cA) / \operatorname{det}
\end{align*}
where
\[
\operatorname{det} = ad - bc
\]
Write a program that prompts the user for \texttt{double}s for 
$a$, $b$, $A$, $c$, $d$, $B$, and $n$ and prints the solution. The
values for $a$, $b$, $A$, $c$, $d$, $B$,
determines the equations
\begin{align*}
ax + by &= A \\
cx + dy &= B
\end{align*}
The \texttt{double}s
printed must be to $n$ decimal places in fixed point format 
where the value of $n$ is the last
user-input value.

\resett
\nextt
\begin{console}[commandchars=\\\{\}]
\userinput{2 3 8 5 -1 3 5}
2.00000x + 3.00000y = 8.00000
5.00000x + -1.00000y = 3.00000
solution: x = 1.00000, y = 2.00000
\end{console}

\nextt
\begin{console}[commandchars=\\\{\}]
\userinput{1 2 3 4 5 6 1}
1.0x + 2.0y = 3.0
4.0x + 5.0y = 6.0
solution: x = -1.0, y = 2.0
\end{console}

\nextt
\begin{console}[commandchars=\\\{\}]
\userinput{6.2 2.5 3.4 -1.3 2.9 0.1 4}
6.2000x + 2.5000y = 3.4000
-1.3000x + 2.9000y = 0.1000
solution: x = 0.4527, y = 0.2374
\end{console}
