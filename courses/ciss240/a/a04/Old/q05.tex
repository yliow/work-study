%-*-latex-*-
According to Dr. Sigmond Fried, the IQ of a person is given by the following 
formula:
\[
iq = PI \times skullRadius^2 +
\frac{weight}{height + thumbLength}
\times classes +
\frac{5}{numFingers}
\]
where the values in the formula are the person's height, weight, length of thumb, radius of skull, the
number college classes that he/she has taken (and passed!), 
and the number of fingers. 
[Note: The $\times$ in the above formula is multiplication and not a variable.]

Write a program that does the following. It prompts for the user's height, 
weight, length of his/her
thumb, radius of his/her skull, the number of 
college classes that he/she has taken, and number of fingers,
and displays the
user's IQ using the above formula.
Your program should use the following for variable names.
\begin{tightlist}
\li \verb!height!
\li \verb!weight!
\li \verb!thumbLength!
\li \verb!skullRadius!
\li \verb!classes!
\li \verb!numFingers!
\li \verb!iq!
\end{tightlist}
PI is the constant $3.14159$  -- approximately, of course! 
Your code must of course contain the definition of PI 
(as a double of
course!) 
You must use this variable in your code.
However instead of something like this:
\begin{console}
double PI = 3.14159;
\end{console}
since the mathematical constant  $\pi$ does not change,
we force it to be a constant 
to prevent it from being changed:
\begin{console}
const double PI = 3.14159;
\end{console}


Your program must get the 5 inputs from the user 
(without printing and prompting string) and then print
the result in fixed point format with 2 decimal places. 
For instance here's an example of the
input/output session while running the program:
\begin{console}[commandchars=\\\{\}]
\userinput{6.1 180.2 2.4 3.7 5 10}
123.45
\end{console}
(Input is in this order: height, weight, thumbLength, skullRadius, classes, 
numFingers.) 
Note that the above is given to fix the expected interaction between the user 
and the program. 
The above output is actually incorrect; it is just to show you the
expected input/output format.

For this question, \textit{you} have to come up with your own test case.
You are strongly advised to write down at least 4 test cases, 
compute the results (i.e. \verb!iq!) with a
calculator, and then compare your calculator results against 
the outputs of your C\texttt{++} program.

You must declare variables of the appropriate type. 
Do not use a \verb!double! when an \verb!int! is enough and
do not use an \verb!int! when you do need the precision of a \verb!double!.
