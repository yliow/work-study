This is a continuation on branching, i.e. on if and if-else statements. 
The programs are now longer and the logic has a lot more
intricate layers.
This is to increase your ability to read, analyze, and write longer programs.
In some of the questions, you will see repetition of logic, hinting
at the need to have some kind of repetition structure for 
efficient programming.

Note that Q2-Q5 are actually related and solves one problem.
This is deliberate: it shows you that usually problem--solving
usually involves breaking down a problem into smaller subproblems
and solving the subproblems, growing the collection of sub-solutions
until you have solved the whole problem.
In Q2-Q5, you will see that  
a problem that is considered
very simple to human beings requires 
a lot of thought and 
programming logic. 
Q4 and Q5 will be very \lq\lq dense''.
I suggest you try Q1, Q2, Q3 and then jump to Q6 if you find
that you need time to think about Q4 and Q5.
