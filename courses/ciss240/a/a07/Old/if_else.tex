\textsc{Preventing indentation in a stack of if-else}

Note that if you have code like this where 
in the \verb!else! block you only have an 
\verb!if! statement or an 
\verb!if-else! statement:
{\footnotesize
\begin{console}
if (bool1)
{
   ...
}
else
{
    if (bool2)
    {
        ...
    }
    else
    {
        if (bool3)
        {
            ...
        }
        else
        {
            ...
        }
    }
}
\end{console}
}
then, it's the same as the following 
(the \verb!else! has only one \verb!if-else! statement
so the block is redundant -- check your notes):
{\footnotesize
\begin{console}
if (bool1)
{
   ...
}
else
    if (bool2)
    {
        ...
    }
    else
        if (bool3)
        {
            ...
        }
        else
        {
            ...
        }
\end{console}
}
which is the same as the following (because
 C++ ignores whitespaces -- check your notes):
{\footnotesize
\begin{console}
if (bool1)
{
   ...
}
else if (bool2)
{
    ...
}
else if (bool3)
{
    ...
}
else
{
    ...
}
\end{console}
}
In other words, we join an \verb!if! with the preceding \verb!else!.

Writing it this way will prevent too much indentation toward the 
right and make your program look like a table.
This is one of the very few cases where we ignore the standard
indentation rules in order not to over--indent.

Note that this won't work if the \verb!else! contains more than
just an \verb!if-else!, for instance
{\footnotesize
\begin{console}[commandchars=\~\!\@]
if (bool1)
{
   ...
}
else
{
    ~underline!x = 1;@
    if (bool2)
    {
        ...
    }
    else
    {
        ...
    }
}
\end{console}
}
In this case, since the outermost \verb!else! has two statements,
you cannot get rid of the block symbols \verb!{}! to join up the 
inner \verb!if! with the outer \verb!else!.
