%-*-latex-*-
This is a continuation of Q3.

Make sure that your program prints a zero whenever a polynomial has zeroes for all its coefficients. 
The following is a test case:
\begin{console}[commandchars=\\\{\}]
\userinput{0 1 1 0 0 0}
(x + 1)(0) = 0
\end{console}

Make sure your code passes all tests in the previous question.

\resett
\nextt
\begin{console}[commandchars=\\\{\}]
\userinput{0 1 1 0 0 0}
(x + 1)(0) = 0
\end{console}

\nextt
\begin{console}[commandchars=\\\{\}]
\userinput{0 0 0 1 2 3}
(0)(x^2 + 2x + 3) = 0
\end{console}

\nextt
\begin{console}[commandchars=\\\{\}]
\userinput{3 0 0 1 2 0}
(3x^2)(x^2 + 2x) = 3x^4 + 6x^3
\end{console}

\nextt
\begin{console}[commandchars=\\\{\}]
\userinput{1 2 3 4 5 6}
(x^2 + 2x + 3)(4x^2 + 5x + 6) = 4x^4 + 13x^3 + 28x^2 + 27x + 18
\end{console}
