%-*-latex-*-
Questions 2-5 are essentially the same problem. 
The goal is to show you how to break a problem down into smaller 
subproblems, attacking them one at a time. 
You might want to read all of Q2-Q5 before beginning work on Q2.

The following is the product of two polynomials of degree 2:
\[
(ax^2 + bx + c)(dx^2 + ex + f) = 
(ad)x^4 + (ae + bd)x^3 + (af + be + cd)x^2 + (bf + ce)x + (cf)
\]
The following code segment:
\begin{console}
int a = 0, b = 0, c = 0, d = 0, e = 0, f = 0;
 
std::cin >> a >> b >> c >> d >> e >> f;
 
std::cout << '('
          << a << "x^2 + "
          << b << "x + "
          << c << ")("
          << d << "x^2 + "
          << e << "x + "
          << f << ") = "
          << a * d << "x^4 + "
          << (a * e + b * d) << "x^3 + "
          << (a * f + b * e + c * d) << "x^2 + "
          << (b * f + c * e) << "x + "
          << (c * f) << std::endl;
\end{console}
computes the product of two polynomial with integer coefficients. 
For instance this is an execution of the program:
\begin{console}[commandchars=\\\{\}]
\userinput{1 2 3 4 5 6}
(1x^2 + 2x + 3)(4x^2 + 5x + 6) = 4x^4 + 13x^3 + 28x^2 + 27x + 18
\end{console}

Modify the program so that zero terms are not printed. 
Here's a test case:
\begin{console}[commandchars=\\\{\}]
\userinput{0 1 1 0 1 1}
(1x + 1)(1x + 1) = 1x^2 + 2x + 1
\end{console}
Note that none of the \verb!0x^2! terms are printed. 

\resett

\nextt
\begin{console}[commandchars=\\\{\}]
\userinput{0 1 1 0 1 1}
(1x + 1)(1x + 1) = 1x^2 + 2x + 1
\end{console}

\nextt
\begin{console}[commandchars=\\\{\}]
\userinput{0 1 0 1 1 1}
(1x)(1x^2 + 1x + 1) = 1x^3 + 1x^2 + 1x
\end{console}

\nextt
\begin{console}[commandchars=\\\{\}]
\userinput{0 1 1 0 1 -1}
(1x + 1)(1x + -1) = 1x^2 + -1
\end{console}

\nextt
\begin{console}[commandchars=\\\{\}]
\userinput{2 0 1 0 1 -1}
(2x^2 + 1)(1x + -1) = 2x^3 + -2x^2 + 1x + -1
\end{console}

\nextt
\begin{console}[commandchars=\\\{\}]
\userinput{1 0 2 2 0 1}
(1x^2 + 2)(2x^2 + 1) = 2x^4 + 5x^2 + 2
\end{console}

\nextt
\begin{console}[commandchars=\\\{\}]
\userinput{2 -3 0 3 2 0}
(2x^2 + -3x)(3x^2 + 2x) = 6x^4 + -5x^3 + -6x^2
\end{console}

\nextt
\begin{console}[commandchars=\\\{\}]
\userinput{3 4 0 3 -4 0}
(3x^2 + 4x)(3x^2 + -4x) = 9x^4 + -16x^2
\end{console}

\nextt
\begin{console}[commandchars=\\\{\}]
\userinput{2 0 3 -2 0 3}
(2x^2 + 3)(-2x^2 + 3) = -4x^4 + 9
\end{console}
