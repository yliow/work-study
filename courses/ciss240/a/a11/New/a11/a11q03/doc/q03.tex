%-*-latex-*-
You are given the following skeleton code which declares an array of $10$
integers and prompts the user for $10$ integers to be placed in the array. The
program then prompts the user for an integer value for variable \verb!j!.

\begin{Verbatim}[frame=single]
// Name:
// File:

#include <iostream>

int main()
{
    const int SIZE = 10;
    int a[SIZE];

    for (int i = 0; i < SIZE; i++)
    {
        std::cin >> a[i];
    }

    int j = 0;
    std::cin >> j;

    // YOUR CODE HERE

    return 0;
}
\end{Verbatim}

Your program must read the value in the array at index \verb!j! and then print
the number of times this value appears \underline{\bf consecutively} in the
array \underline{\bf starting at} index \verb!j! going to the
\underline{\bf right}. Read the test cases carefully before writing your
program.

\resett
\nextt
\begin{console}[frame=single, commandchars=\\\{\}]
\userinput{0 0 5 5 5 5 5 5 9 5 2}
6
\end{console}

In this case the value in the array at index $2$ is the value $5$. Starting at
index $2$ and going to the right, there are six consecutive $5$s.

\nextt
\begin{console}[frame=single, commandchars=\\\{\}]
\userinput{5 5 5 5 5 5 5 5 9 5 2}
6
\end{console}

The value printed is still $6$ because although there are two $5$s to the left
of the $5$ at index $3$, they do not count.

\nextt
\begin{console}[frame=single, commandchars=\\\{\}]
\userinput{4 4 4 4 4 7 4 4 7 7 3}
2
\end{console}

\nextt
\begin{console}[frame=single, commandchars=\\\{\}]
\userinput{4 4 4 4 4 7 4 4 7 7 9}
1
\end{console}
