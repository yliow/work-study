%-*-latex-*-
Recall that we already know how to check if an integer $n$ is prime. The
algorithm is basically the following: 
\begin{tightlist}
  \li Test all integers $d$ from $2$ to $n - 1$.
  \li If $d$ divides $n$, then $n$ is not prime.
  \li Otherwise, $n$ is not divisible by any integer other than $1$ and
itself and therefore must be prime.
\end{tightlist}

This method does work but in fact, some checks are redundant. For instance, if
$n = 97$, there's no point in checking if $93$ divides $97$ : $93$ is way too
big to divide $97$. In fact, instead of checking all $d$ from $2$ to $n - 1$,
it's enough to check for $d$ from $2$ to the square root of $n$.

Write a program that computes and stores the first $100,000$ primes in an
array. Print the first $5$ and the last $5$ primes. Continuously get a user
input \verb!x! and an option \verb!i!:

\begin{itemize}
  \li If \verb!i! is $0$ (the number zero): perform a binary search for
      \verb!x! in the array, and print report to the user denoting whether
      \verb!x! is prime or not and notifying the user of it's position in the
      array.
  \li If \verb!i! is $1$: perform a prime factorization on \verb!x! using your
      prime array.
\end{itemize}

The program ends when the user enters $-1$ for \verb!x!. (You may assume all
numbers will use only primes in the array. For instance, the user will not ask
about a prime $\textgreater 1,000,000$ or ask the program to factor a number
that contains a prime $\textgreater 1,000,000$)

Note:
As stated above, you MUST use binary search for option $0$.
For option $1$, you should not scan the prime array more than once (i.e., there
should only be a single for--loop that runs through the prime table array once
during the prime factorization.) 

\resett
\nextt
\begin{console}[fontsize=\small,frame=single, commandchars=\\\{\}]
building prime table of 100,000 primes ...
primes: 2, 3, 5, 7, 11, ..., 1299647, 1299653, 1299673, 1299689, 1299709
\userinput{2 0}
2 is prime # 1
\userinput{3 0}
3 is prime # 2
\userinput{4 0}
4 is not a prime
\userinput{5 0}
5 is prime # 3
\userinput{6 0}
6 is not a prime
\userinput{7 0}
7 is prime # 4
\userinput{631 0}
631 is prime # 115
\userinput{-1}
\end{console}
(Note that $2$ is prime \# $1$ and not prime \# $0$).

\nextt
\begin{console}[fontsize=\small,frame=single, commandchars=\\\{\}]
building prime table of 100,000 primes ...
primes: 2, 3, 5, 7, 11, ..., 1299647, 1299653, 1299673, 1299689, 1299709
\userinput{1 1}
1 = 1
{2 1}
2 = 2
\userinput{3 1}
3 = 3
\userinput{4 1}
4 = 2^2
\userinput{5 1}
5 = 5
\userinput{6 1}
6 = 2 * 3
\userinput{7 1}
7 = 7
\userinput{8 1}
8 = 2^3
\userinput{9 1}
9 = 3^2
\userinput{10 1}
10 = 2 * 5
\userinput{11 1}
11 = 11
\userinput{12 1}
12 = 2^2 * 3
\userinput{100 1}
100 = 2^2 * 5^2
\userinput{-1}
\end{console}

You're advised to include more test cases.


\newpage
\textsc{Spoilers Aheads!!! \dots}

Hint for option $1$: Test if prime $2$ divides \verb!x!; if it does, keep
dividing it by $2$ to compute the largest power of $2$ dividing \verb!x!. Once
you're done, move on the the NEXT prime $3$. If $3$ does not divide \verb!x!,
move on to the next prime $5$. If $5$ divides \verb!x!, continuously divide
\verb!x! by $5$ to compute the highest power of $5$ dividing \verb!x!. Etc.
Therefore, the prime array is scanned only once.
