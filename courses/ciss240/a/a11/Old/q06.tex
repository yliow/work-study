%-*-latex-*-
Write a program that prompts the user for two arrays of $1000$ integers, say
\verb!x! and \verb!y!, and prints the index in \verb!x! where \verb!y! occurs
as a sequence for the first time. For instance, note that if

\begin{Verbatim}
            x: 3, 1, 6, 2, 8, 9, 6, 2, 4
            y: 6, 2
\end{Verbatim}

we see that \verb!y! occurs in \verb!x! at index position $2$ and index
position $6$:

\begin{Verbatim}[commandchars=\\\{\}]
            x: 3, 1, \userinput{6, 2}, 8, 9, \userinput{6, 2}, 4
            y: \userinput{6, 2}
\end{Verbatim}

The program should print $2$ in this case. On the other hand if

\begin{Verbatim}
            x: 3, 1, 6, 2, 8, 9, 6, 2, 4
            y: 6, 8
\end{Verbatim}

the program should print $-1$ to indicate that \verb!y! does not occur as a
sequence within \verb!x!. Make sure you read the test cases below carefully
before diving into your code.

[When the arrays are arrays of characters, this problem is called the
\lq\lq substring problem\rq\rq and is used extensively in areas such as
bioinformatics and computational genetics. The first array is usually a DNA
sequence and the second is a section of DNA that, for instance, might be a DNA
sequence that indicates a dangerous mutation.]

The user enters the values for \verb!x! and \verb!y! by entering integers with
the integer value $-9999$ as a sentinel to terminate data entry. Your program
must allow a maximum size of $1000$ for \verb!x! and \verb!y!.


\resett
\nextt
\begin{console}[frame=single, commandchars=\\\{\}]
\userinput{3 1 6 2 8 9 6 2 4 -9999}
\userinput{6 2 -9999}
2
\end{console}

\nextt
\begin{console}[frame=single, commandchars=\\\{\}]
\userinput{3 1 6 2 8 9 6 2 4 -9999}
\userinput{6 8 -9999}
-1
\end{console}

\nextt
\begin{console}[frame=single, commandchars=\\\{\}]
\userinput{3 1 6 2 8 9 6 2 4 -9999}
\userinput{3 1 6 2 8 9 6 2 4 -9999}
0
\end{console}

\nextt
\begin{console}[frame=single, commandchars=\\\{\}]
\userinput{3 1 6 2 8 9 6 2 4 8 9 6 2 1 8 9 6 2 5 -9999}
\userinput{8 9 6 2 5 -9999}
14
\end{console}

\nextt
\begin{console}[frame=single, commandchars=\\\{\}]
\userinput{3 1 2 4 6 2 4 6 8 9 -9999}
\userinput{2 4 6 8 -9999}
5
\end{console}

\nextt
\begin{console}[frame=single, commandchars=\\\{\}]
\userinput{3 1 2 4 6 2 4 6 8 9 -9999}
\userinput{2 4 6 8 9 9 -9999}
-1
\end{console}

\nextt
\begin{console}[frame=single, commandchars=\\\{\}]
\userinput{1 2 3 4 5 6 7 8 -9999}
\userinput{1 3 5 7 -9999}
-1
\end{console}
