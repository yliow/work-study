%-*-latex-*-
Recall that a sequence is simply an ordered list.

A sequence of numbers is in ascending order if the integers are arranged from
the smallest to the largest. In other words, they are increasing as you go
through the sequence. For instance, $2, 9, 11$ is a sequence in ascending
order.

Consecutive integers are integers that follow right after each other in an
uninterrupted succession, i.e., they always differ by $1$. For instance, 
$4, 5$ and $6$ are consecutive. Note that they also happen to be ascending.

Based on the discussion above, it should be obvious that a sequence of 
consecutive integers in ascending order is thus a list of integers that are
arranged from the smallest to the largest such that each term is $1$ larger
than the preceding term. $4, 5, 6$ is an example of such a sequence. $7, 8, 9$
and $10, 11, 12$ are couple of other examples that fit the bill.

Write a program that accepts a strictly positive ($>0$) integer $n$ from the
user and then, finds and prints the longest sequence of consecutive digits in
ascending order in the integer $n$ going from left to right.

For instance, the longest sequence of consecutive ascending digits in the
integer $123475634$ is $1234$, $456$ is the longest consecutive ascending
sequence in $1294563$ and the longest sequence of consecutive digits in
the ascending order in the integer $123956780$ is $5678$.

If there is a tie, i.e., there are two or more sequences with the same number
of consecutive ascending digits which are longer than all other such sequences,
print the one that occurs first. Therefore, if the user enters
\verb!126784563!, the program should print \verb!678!.

You must discard leading zeroes in the input but zeroes inside the integer can
lead the sequence. For instance, if the user enters \verb!01234!, your program
should print \verb!1234! but if the user enters \verb!56701234!, your program
should print \verb!01234!.

Try all the test cases and devise some more of your own. Note that the largest
value that an \verb!int! variable can hold is $2,147,483,647$ so you should
only test your code with integers that are at most $9$ digits long or you will
make the world explode!

\resett
\nextt
\begin{console}[commandchars=\\\{\}]
\userinput{126784563}
678
\end{console}

\nextt
\begin{console}[commandchars=\\\{\}]
\userinput{123475634}
1234
\end{console}

\nextt
\begin{console}[commandchars=\\\{\}]
\userinput{1294563}
456
\end{console}

\nextt
\begin{console}[commandchars=\\\{\}]
\userinput{123956780}
5678
\end{console}

\nextt
\begin{console}[commandchars=\\\{\}]
\userinput{34040123}
0123
\end{console}

\nextt
\begin{console}[commandchars=\\\{\}]
\userinput{0123456}
123456
\end{console}

\nextt
\begin{console}[commandchars=\\\{\}]
\userinput{56701234}
01234
\end{console}

\nextt
\begin{console}[commandchars=\\\{\}]
\userinput{1203424}
12
\end{console}

\nextt
\begin{console}[commandchars=\\\{\}]
\userinput{78945678}
45678
\end{console}

\nextt
\begin{console}[commandchars=\\\{\}]
\userinput{10437586}
1
\end{console}

\nextt
\begin{console}[commandchars=\\\{\}]
\userinput{11234566}
123456
\end{console}

\nextt
\begin{console}[commandchars=\\\{\}]
\userinput{3456789}
3456789
\end{console}

You can see that every integer is itself a sequence of digits. Therefore, you
are actually finding the longest consecutive sequence of digits from a longer
sequence that isn't necessarily consecutive. In other words, you essentially
are searching for a particular subsequence in a larger sequence. This has very
important applications that you will see in the future.
For instance a huge part of computational biology (bioinformatics)
is concerned with finding patterns in DNAs.
