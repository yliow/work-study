%-*-latex-*-
Write a program that accepts an integer value $n$ from the user and then
poses $n$ multiplication problems to the the user. Each question involves the
product of two random numbers between $0$ and $9$. If the user answers
correctly, he/she gets a point. After asking $n$ questions, the
program prints the score.

To make testing easier, instead of doing this:
\begin{console}
...

int main()
{
    srand((unsigned int) time(NULL));

    ... YOUR CODE ...

    return 0;
}
\end{console}
you MUST do the following (i.e., fix the seeding of the random generator):
\begin{console}
...

int main()
{
    srand((unsigned int) 0);

    ... YOUR CODE ...

    return 0;
}
\end{console}

\resett
\nextt
\begin{console}[commandchars=\\\{\}]
\userinput{1}
8 * 9 = \userinput{72}
1
\end{console}

\nextt
\begin{console}[commandchars=\\\{\}]
\userinput{1}
8 * 9 = \userinput{0}
0
\end{console}

\nextt
\begin{console}[commandchars=\\\{\}]
\userinput{3}
8 * 9 = \userinput{72}
8 * 7 = \userinput{56}
5 * 7 = \userinput{35}
3
\end{console}

\nextt
\begin{console}[commandchars=\\\{\}]
\userinput{3}
8 * 9 = \userinput{0}
8 * 7 = \userinput{0}
5 * 7 = \userinput{35}
1
\end{console}

\nextt
\begin{console}[commandchars=\\\{\}]
\userinput{3}
8 * 9 = \userinput{72}
8 * 7 = \userinput{0}
5 * 7 = \userinput{35}
2
\end{console}

\nextt
\begin{console}[commandchars=\\\{\}]
\userinput{3}
8 * 9 = \userinput{2}
8 * 7 = \userinput{2}
5 * 7 = \userinput{2}
0
\end{console}

\nextt
\begin{console}[commandchars=\\\{\}]
\userinput{10}
8 * 9 = \userinput{1}
8 * 7 = \userinput{1}
5 * 7 = \userinput{1}
5 * 5 = \userinput{1}
0 * 2 = \userinput{1}
3 * 0 = \userinput{1}
2 * 1 = \userinput{1}
7 * 1 = \userinput{1}
5 * 5 = \userinput{1}
7 * 0 = \userinput{1}
0
\end{console}
