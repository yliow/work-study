%-*-latex-*-
Write a program that shows the work for adding two $2$--digit numbers using
column addition. See the output of the test cases for the format of the output.
Note in particular all the spaces and also the carryover (see Test $2$ and
Test $4$).

Advice: Don't try to think of all the possible cases. Look at Test $1$ and try
to make your program work for Test $1$. Then \lq\lq grow\rq\rq your solution to
handle other cases.

\resett
\nextt
\begin{console}[commandchars=\\\{\}]
\userinput{12 34}
  1 2
+ 3 4
-----
  4 6
-----
\end{console}

\nextt
\begin{console}[commandchars=\\\{\}]
\userinput{17 38}
  1
  1 7
+ 3 8
-----
  5 5
-----
\end{console}

\nextt
\begin{console}[commandchars=\\\{\}]
\userinput{70 81}
  7 0
+ 8 1
-----
1 5 1
-----
\end{console}

\nextt
\begin{console}[commandchars=\\\{\}]
\userinput{79 87}
  1
  7 9
+ 8 7
-----
1 6 6
-----
\end{console}

After finishing this question, you should realize that you have just 
\lq\lq taught\rq\rq the computer how to do column addition -- at least for two
columns.




%\begin{comment}
{\bf SPOILER ALERT ... HINTS ON THE NEXT PAGE ... USE THEM ONLY IF YOU NEED IT ...}

\newpage

\textsc{Spoiler Hints}

{\it Read this only if you need to.}

Remember: solve a problem in small steps. This looks like a complicated 
problem. So what \lq\lq smaller\rq\rq problem can you handle first? One
possible \lq\lq smaller\rq\rq problem is the case where the column addition
does not have any carryovers.

Using Test $1$ as an example, first write a program that does this:
\begin{console}[commandchars=\\\{\}]
\userinput{12 34}
  1 2
+ 3 4
-----
   
-----
\end{console}

Next make it do this:
\begin{console}[commandchars=\\\{\}]
\userinput{12 34}
  1 2
+ 3 4
-----
    6
-----
\end{console}

The next step is to make it do this:
\begin{console}[commandchars=\\\{\}]
\userinput{12 34}
  1 2
+ 3 4
-----
  4 6
-----
\end{console}

The next step is to handle the carryover from the first column; see Test $2$.
The last step is to handle the carry from the second column.

If you look at all the numbers present in the output you will notice that there
is a maximum of eight digits; see Test $4$:
\begin{console}[commandchars=\\\{\}]
\userinput{79 87}
  1
  7 9
+ 8 7
-----
1 6 6
-----
\end{console}

You can do this problem if you know how to compute the values for these eight 
slots (use $8$ variables):
\begin{console}[commandchars=\\\{\}]
\userinput{79 87}
  X
  X X
+ X X
-----
X X X
-----
\end{console}
%\end{comment}
