\section{More decision problems for TM}

The acceptance problem and halting problem earlier are examples
of what we call \defterm{decision problems for Turing machines}.
You can of course rephrase them for NFAs,
context-free grammars (or PDAs)
and DFAs.

Here are more decision problems for TMs.

\newpage
\newcommand\SIZEONE{\mathsc{Size-One}}
\begin{ex}
Is the following language decidable:
\[
\SIZEONE_\TM = \{ \langle M \rangle \mid |L(M)| = 1 \}
\]
\end{ex}

Do not peek ... the solution is on the next page.


\newpage
\SOLUTION

We will reduce to this the acceptance problem, i.e.,
\[
\ACCEPT_\TM \leq \SIZEONE_\TM
\]
Suppose $\SIZEONE_\TM$ is decidable, say it's decided by $S$.
Then I claim that I can construct a decider for $\ACCEPT_\TM$.

Construct $A$ as follows.
Suppose $\langle M \rangle \# \langle w \rangle$ is given
as an input to $A$ where
$M$ is a TM and $w$ is an input for $M$.
(If the input is not of the form $\langle M \rangle \# \langle w \rangle$,
$A$ rejects immediately.)
$A$ will first construct $M_w$, a Turing machine
that works exactly like $M$ if the input is $w$
but if the input is not $w$, $M$ will reject after checking that the
input is not $w$.
In other words $L(M_w) = \{\}$ or $\{w\}$.
$A$ then does the following:
\begin{tightlist}
  \li $A$ will first simulate
  $S$ with $M_w$.
  Note that $S$ is a decider and therefore must halt.
  \li If while $A$ is simulating
  $S$, $A$ sees that $S$ halts in the accept state, then
  $A$ will enter its accept state.
  [Note that $S$ accepts iff $M_w$
  accept $w$ since by definition $M_w$ can only accept $w$
  and nothing else.]
  \li
  If while $A$ is simulating $S$, $A$ sees that $S$
  halts in the reject state, then $A$ will enter its reject state.
  [Note that $S$ rejects iff $M_w$ rejects
  which means that $M$ does not accept $w$.]
  \li Again, note that $S$ is a decider and therefore must enter
  the accept or reject state at some point -- i.e., $S$
  cannot run forever.
\end{tightlist}
The above means that $A$ accepts $\langle M \rangle \# \langle w \rangle$
iff $M$ accepts $w$. Furthermore $A$ is a decider.
We have shown that
But this means that $\ACCEPT_\TM$ is decidable (by the decider $A$)
which is a contradiction.



\newpage
\begin{ex}
Let $k \geq 0$ be an integer.
Is the following language decidable:
\[
\{ \langle M \rangle \mid |L(M)| = k \}
\]
\end{ex}


\newpage
\begin{ex}
Is the following language decidable:
\[
\{ \langle M \rangle \mid |L(M)| \text{ is finite} \}
\]
\end{ex}




\newpage
\newcommand\INFINITE{\mathsc{Infinite}}
\begin{ex}
Is the following language decidable:
\[
\INFINITE_\TM = \{ \langle M \rangle \mid L(M) \text{ is infinite } \}
\]
\end{ex}


\newpage
\newcommand\Null{\mathsc{Null}}
\begin{ex}
Is the following language decidable:
\[
\Null_\TM = \{ \langle M \rangle \mid \ep \in L(M) \}
\]
\end{ex}


\newpage
\SOLUTION

Assume that $\Null_\TM$ is decidable, say it is decided by $D$.

Construct a TM $A$ that does this:
Let $\langle M \rangle \# \langle w \rangle$ be an input for $A$.
\begin{tightlist}
  \item $A$ construct a TM $M'$ (which depends on $M$ and $w$)
  which does the following:
  When $M'$ executes, it replaced the input with $w$,
  and then proceeds to
  work exactly like $M$.
  Note that if we run $M'$ on $\ep$, then it would be the
  same as running $M$ on $w$.
  Of course when I say $A$ \lq\lq constructs'' $M'$, I  mean $A$ write
  the encoding, $\langle M'\rangle$ somewhere on its tape.
  \item
  $A$ then simulates $D$ on $\langle M' \rangle$.
  Note that $D$ is decidable therefore $D$ must either accept or reject
  ($D$ cannot run forever).
  This means that as $A$ simulates $D$ one step at a time
  $A$ will at some point in time see $D$ entering the
  accept or the reject state.
  \item If while $A$ is simulating $D$ running on $\langle M'\rangle$,
  $A$ sees that $D$ accepts $\langle M' \rangle$ (i.e., 
  $M'$ accepts $\ep$, which means that $M$ accepts $w$), then
  $A$ enters its accept state.
  \item If while $A$ is simulating $D$ running on $\langle M'\rangle$,
  $A$ sees that $D$ rejects $\langle M' \rangle$ (i.e., 
  $M'$ does not accepts $\ep$, which means that $M$ does not
  accepts $w$), then
  $A$ enters its reject state.
  \end{tightlist}
We have shown that
\[
\ACCEPT_\TM \leq \Null_\TM
\]
This means that $A$ is a decider for $\ACCEPT_\TM$ which is a contradiction.
Therefore $\Null_\TM$ cannot be a decidable language.
\qed

As exercise, try to write a proof that shows
\[
\mathsc{HALT}_{\TM} \leq \Null_\TM
\]

\newpage
\newcommand\NULL{\mathsc{Null}}
\begin{ex}
Is the following language decidable:
\[
\{ \langle M \rangle \mid M \text{ has 42 states } \}
\]
\end{ex}


\newpage
\SOLUTION

Yes.
Depending on the way TMs are encoded,
design a TM, $D$, such that when $D$ is given $\langle M \rangle$,
$D$ moves the
read-write head to
the section of $\langle M \rangle$ that encodes the set of states
and begin counting the number of states.
If the counting stops before 42, $D$ rejects.
If the counting goes up to 43, $D$ rejects.
Otherwise $D$ accepts.


\newpage
\newcommand\ALL{\mathsc{All}}
\begin{ex}
Is the following language decidable:
\[
\ALL_\TM = \{ \langle M \rangle  \,|\, L(M) = \Sigma^*
\text{ where is the input alphabet of $M$} \}
\]
\end{ex}


\newpage
\begin{ex}
Is the following language decidable:
\[
\mathsc{Equal}_\TM =
\{
\langle M_1 \rangle
\#
\langle M_2 \rangle
\,|\, L(M_1) =  L(M_2)
\}
\]
\end{ex}

