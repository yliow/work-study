\section{Nondeterministic Turing Machines}

From our definition of TM and also from your experience from DFA,
you would expect a nondeterministic version of a TM. In our
definition of a (deterministic) TM, for each state $q$ and $a \in
\Gamma$, there is a corresponding new state $q'$ and a symbol $a'$
you overwrite $a$ with and also there is a direction $D$ taken by
the read/write head ($D$ is either $L$,$R$ or $S$):
\[
 \delta(q,a) = (q',a',D)
\]
Recall as stated above sometime we do allow $\delta(q,a)$ to be
undefined. In this case, the machine crashes. This is equivalent
to saying that $\delta(q,a) = q_{\reject}$. As mentioned earlier TM
are so huge that usually authors omit such description of $\delta$
with the understanding that the transition leads to a rejecting
state. Therefore with this convention, $\delta(q,a)$ is either
defined or not.

The definition of a nondeterministic TM (NTM) is easy. You allow
multiple options for $\delta(q,a)$. In other words $\delta(q,a)$
is a subset of $Q \times \Gamma \times \{L,R,S\}$, i.e.,
\[
\delta: Q \times \Gamma
\rightarrow
P(Q \times \Gamma \times \{L, R, S\})
\]
Of course if an
NTM has a $\delta$ satisfying $|\delta(q,a)| = 1$, then it is
really just a deterministic TM.

Given a NTM $N$ and a string $x$, we say that $N$ accepts $x$ is
there is a sequence of IDs that leads to an accepting ID. Note
that just like the case of the NFA, there might be several choices
at some time. To accept a string, you only need to have
\textit{one}
accepting computation; the others can fail.

And $\ldots$ (surprise, surprise) $\ldots$:

\begin{thm} Deterministic TM is just as powerful as
nondeterministic TM. In other words:
\begin{itemize}
\item[(a)] If $M$ is a TM, then there is an NTM $N$ such that
$L(M) =
 L(N)$.
\item[(b)] If $N$ is an NTM, then there is a $TM$ $M$ such that
$L(N) = L(M)$.
\end{itemize}
\end{thm}

Here's a sketch of the proof. (a) is easy and is already hinted
upon above. What about (b)? Suppose you're given $N$. You need to
design a TM $M$ that can try out all possible computational
sequence of $N$. How would you do that? Suppose
\[
r = \max_{q,a} | \delta(q,a) |
\]
where $q,a$ runs over all the states in $Q$ and all symbols in
$\Gamma$ of $N$. Suppose you want to run $N$ on string $x$. The
initialize configuration is $q_0x$. There can be at most $r$
possibilities. Basically the new $M$ runs track of all possible
computations of the 
\lq\lq clones'' on a single tape with the
clone separated by an appropriate separator character.
The computations are of course recorded using IDs.
The TM will need to divide time equally among all the clones!
In other words, if there are 10 ongoing computations, 
the TM must keep track of which computation is currently active,
and execute one step for each clone and move on to the next.
The reason why this is necessary is because one particular clone might
run on forever.
So you can't just put all your time into one clone!
This is important.


TODO:
\begin{itemize}
  \li 
  Mention given a TM $M$, it's possible to design another TM $M'$ that
  simulates $M$ executing an ID for one step.
  The input is the $\langle ID \rangle$
  and after executing this in one step,
  $M'$ halts.
  \li It's possible to design $U$ such that
  given $\langle M, ID \rangle$, it runs $M$ on the ID by one step
  so that the input tapes becomes
  $\langle M, ID' \rangle$
\end{itemize}
