\section{Languages which are not Turing--Recognizable again}
Now I'm going to prove that the 
set of languages over $\Sigma$ (say $\Sigma = \{0, 1\}$)
is not countable.
I'll use $\Sigma = \{0, 1\}$ but the argument works in general.

First of all, recall that $\Sigma^*$ is countable
(see above ... remember???)
Say you list the words of $\Sigma^*$ as
\[
w_1, w_2, w_3, w_4, w_5, \ldots
\]
For instance in the case of $\Sigma = \{0, 1\}$,
you can like the $w_i$ like this:
\begin{align*}
w_1 &= \ep \\
w_2 &= 0 \\
w_3 &= 1 \\
w_4 &= 00 \\
w_5 &= 01 \\
w_6 &= 10 \\
w_7 &= 11 \\
...
\end{align*}


Now let's count $\LANG_\Sigma$, the collection of languages over $\Sigma$.
Given a language $L$,
$L$ contains words in $\Sigma^*$.
For instance suppose
\[
L = \{w_3, w_7, w_9\}
\]
In that case I can construct a function $\N \rightarrow \{0,1\}$
associated with $L$ is a natural way:
\begin{align*}
f(3) &= 1 \\
f(7) &= 1 \\
f(9) &= 1 \\
\end{align*}
and $f(n) = 0$ for $n \neq 3, 7, 9$.
In other words 
$f(n) = 1$ exactly when $w_n \in L$.
It's clear that if I have a function $\N \rightarrow \{0, 1\}$
you can construct a language which is a subset of $\Sigma^*$.
This associated between functions of the form $\N \rightarrow \{0, 1\}$
and languages of $\LANG_\Sigma$ is clearly
a 1--1 and onto function.
Therefore 
there must be as many languages in $\LANG_\Sigma$
as there are functions $\N\rightarrow \{0, 1\}$.
I have already shown you that the collection of functions
of the form $\N \rightarrow \{0,1\}$ is not countable.
Therefore $\LANG_\Sigma$ must also be uncountable.

\begin{thm}
$\LANG_\Sigma$ 
is uncountable.
\end{thm}

\begin{ex}
Let $X$ be a finite set with $|X| > 1$.
Is $P(X)$ countable? ($P(X) = $ powerset of $X$.)
\end{ex}

The consequence is of course ...

\begin{thm}
There is a language that is not Turing--recognizable
(i.e., not recursively enumerable.)
\end{thm}

Too bad!!!
This means that there is a language that is not \lq\lq computable''.

The above theorem is not constructive in the sense that
it does not tell you what a non--Turing--recognizable language looks like.

Let me give you an example.
Although I \textit{am} going to describe the language,
I'll just warn you right away that it's not a language that is easily
described using \lq\lq patterns''
such as 
\[
a^n b^n, n \geq 0
\]
(like our first non-regular language) or something like
\[
a^n b^n c^n, n \geq 0
\]
(like our first non-context--free language).

First of all recall from above that $\Sigma^*$ is countable,
say the words are
\[
w_1, w_2, w_3, \ldots
\]
Also, note that the collection of Turing machines (over $\Sigma$) is also 
countable, say they are
\[
M_1, M_2, M_3, \ldots
\]

Notice that just like Cantor's diagonal trick, we again have
a grid which is (infinitely) countable in both directions:
\begin{python}
from latextool_basic import table
print(table([('', '', '', ''),
('', '', '', ''),
('', '', '', ''),
('', '', '', '')
],
col_headings = ['$w_1$', '$w_2$', '$w_3$', '$w_4$', r'$\ldots$'],
row_headings = ['$M_1$', '$M_2$', '$M_3$', '$M_4$', r'$\ldots$']))
\end{python}

In the cell for $M_i$ and $w_j$, we put a \lq\lq accept'' 
if $M_i$ accepts $w_j$.
and \lq\lq not accept'' (not \lq\lq reject''!) if $M_i$ 
does not accept $w_j$.
(Why can't you put \lq\lq reject''?)
To make the above easier to read, you can think of using 
$1$ for \lq\lq accept'' and $0$ for \lq\lq not accept.
For instance if the table looks like this:
\begin{python}
from latextool_basic import table
print(table([('0', '0', '1', '0', '...'),
('1', '0', '1', '1', '...'),
('0', '1', '1', '1', '...'),
('1', '0', '1', '1', '...'),
],
col_headings = ['$w_1$', '$w_2$', '$w_3$', '$w_4$', r'$\ldots$'],
row_headings = ['$M_1$', '$M_2$', '$M_3$', '$M_4$', r'$\ldots$']))
\end{python}

then $M_1$ does not accept $w_1$ but accept $w_3$.

Following Cantor's technique,
I want $L$ not to be accepting by $M_1$,
not accepted by $M_2$, etc.
For $L$ \textit{not} be the language
accepted by $M_1$, I will want the following to be true:
\[
w_1 \in L \iff M_1 \text{ does not accept }  w_1
\]
In other words, if $M_1$ accepts $w_1$, then I will not put $w_1$ into $L$.
However if $w_1$ is not accepted by $M_1$, I will put $w_i$ into $L$.
Since $L$ has the exact 
opposite behavior as $L(M_1)$ in terms of accepting $w_1$,
$L$ cannot be $L(M_1)$.
I do the same for $w_2$, $w_3$, ...

In general for $i \geq 1$, I want
\[
w_i \in L \iff M_i \text{ does not accept }  w_i
\]
This ensures that $L \neq L(M_i)$ for all $i \geq 1$.
If you want to see $L$ in set notation, here you go:
\[
L = \{ w_i \mid i \geq 1 \text{ and } M_i \text{ does not accept } w_i \}
\]

AHA!
This means that $L$ cannot be accepted by any of the TMs
$M_1$, $M_2$, $M_3$, ...
Since this is a complete list of all TMs,
$L$ cannot be Turing--recognizable!!!

This $L$ is sometimes called the diagonal language.
I will write
\[
\DIAGONAL_\Sigma
\]
for this language.
Instead of using the collection of all TMs over $\Sigma$, $\TM_{\Sigma}$
to form a
diagonal language, you can also use the collection of all
DFAs, $\DFA_{\Sigma}$. So if I need to be more specific, I will write
$\DIAGONAL_{\TM_{\Sigma}}$,
$\DIAGONAL_{\DFA_{\Sigma}}$, etc.

Note that this is a subset of $\Sigma^*$, i.e., 
it is a language in $\LANG_\Sigma$.
(WARNING: Technically, this 
language depends on the ordering of both the TMs and 
words in $\Sigma^*$.)


\begin{comment}
(Note:
The above simply gives a pairing between $\TM$ and $\Sigma^*$:
\begin{align*}
\TM \times \Sigma^* &\rightarrow \{0,1\} \\
\langle M, w \rangle &=
\begin{cases}
1 & \text{ if $M$ accepts $w$} \\ 
0 & \text{ otherwise} \\ 
\end{cases}
\end{align*}
Composing with $\N$ (using any encoding of TMs and $\Sigma^*$, we get
\begin{align*}
\N \rightarrow \TM \times \Sigma^* &\rightarrow \{0,1\} \\
i \mapsto \langle M_i, w_i \rangle &=
\begin{cases}
1 & \text{ if $M$ accepts $w$} \\ 
0 & \text{ otherwise} \\ 
\end{cases}
\end{align*}
\end{comment}
