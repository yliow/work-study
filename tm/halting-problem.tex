\section{The Halting Problem}

Consider this problem:

Is it possible to write a C++ program $M$ that when presented with 
a C++ program, say $P$, with input $w$, $M$ can tell us if $P$
(when compiled and executable) will run forever?

This is definitely helpful since we can use $M$ to prevent us from 
executing programs (with inputs) that contain infinite loops.

It turns out that this is impossible!
You may write a C++ program that analyzes many common bad 
programming patterns
that give rise to infinite loops.
However your program can never catch all possible cases.

This can be proven using what we know about Turing machines at this point.
First let me formulate the problem but in terms of Turing machines:
\[
\mathsc{HaltingProblem}
: \text{Determine if Turing machine $M$ with input $w$ will halt}
\]
Here's the corresponding language 
\[
\mathsc{HaltingLanguage} = 
\{
\langle M \rangle \# \langle w \rangle \mid M \text{ halts on input } 
w\}
\]
For convenience, I'll rewrite it as:
\[
\HALT = \{\langle M \rangle \# \langle w \rangle 
\mid 
\text{$w$ is an input for $M \in \TM$ and }
M(w) = 0, 1\}
\]
Remember our convention:
$M(w) = 1$ means that when $M$ executes with $w$ as input, $M$
will enter $q_\accept$;
$M(w) = 0$ means that $M$ will enter $q_\reject$.
Finally $M(w) = \infty$ means that $M$ will not enter $q_\accept$
or $q_\reject$ but will run forever.

Now when I say I want to solve the halting problem, I mean that 
I want an algorithm (or program that does stop at some point)
that will solve this problem.
This is the same as saying that I want a Turing decider to solve
this problem.
Now let me assume that $\HALT$ is decidable.
Remember that this means that there is some Turing machine,
say I call it $H$,
such that $L(H) = \HALT$ and $H$ will 
enter $q_\accept$ or $q_\reject$ for any input.

I will use $H$ to build a Turing machine $A$ that is a decider ...
and more importantly ... $A$ accepts $\ACCEPT_\TM$
which gives us a contradiction since in the previous section,
I have shown you that $\ACCEPT_\TM$ is not decidable.

This is how $A$ works with input $\langle M \rangle \# \langle w \rangle$.
First $M$ will simulate $H$ on 
$\langle M \rangle \# \langle w \rangle$.
Note that I have assumed that $H$ is a decider.
So $H$ must halt:
\[
H(\langle M \rangle \# \langle w \rangle) = 0 \text{ or } 1
\]
(i.e., not $\infty$.)

\begin{enumerate}

\item Suppose $H(\langle M \rangle \# \langle w \rangle) = 1$.
In this case, 
$A$ proceeds to simulate $M$ with input $w$.
Note that $H$ already tells us this simulation will halt since
\[
H(\langle M \rangle \# \langle w \rangle) = 0, 1
\]
If the simulation of $M$ with input $w$ ends with an accept state, 
$A$ will also accept.
If the simulation of $M$ with input $w$ ends with a reject state, 
$A$ will also reject.
In both cases, you see that
\[
A(\langle M \rangle \# \langle w \rangle) = M(w)
\]
Note that $A$ does not run on forever.

\item Suppose $H(\langle M \rangle \# \langle w \rangle) = 0$.
In this case, $H$ tells us that 
if $M$ executes with input $w$, $M$ will not halt, i.e.,
\[
M(w) = \infty
\]
In this case, $A$ will not bother with simulating $M$ with input $w$:
$A$ simply enters its reject state and halt.
In other words, in this case
\[
A(\langle M \rangle \# \langle w \rangle) = 0
\]
and 
\[
M(w) = \infty
\]
Note that in this case, $A$ does not run on forever.
\end{enumerate}

Note that $A$ halts in both cases: $A$ is a decider.
Now I want to show
\[
L(A) = \ACCEPT_\TM
\]
If 
\[
\langle M \rangle \# \langle w \rangle \in L(A)
\]
then by definition
\[
A(\langle M \rangle \# \langle w \rangle) = 1
\]
This means that I'm not in case 2 above.
So I must be in case 1.
This means that 
\[
M(w) = A(\langle M \rangle \# \langle w \rangle) = 1
\]
Therefore $M$ accepts $w$, i.e.,
\[
\langle M \rangle \# \langle w \rangle \in \ACCEPT_\TM
\]
by definition of $\ACCEPT_\TM$.
This proves that
\[
L(A) \subseteq \ACCEPT_\TM
\]

Now I will prove 
$\ACCEPT_\TM \subseteq L(A)$.
Let 
$\langle M \rangle \# \langle w \rangle \in \ACCEPT_\TM$.
This means that $M$ accepts $w$.
Again, this means that I'm not in case 2.
This implies that I'm in case 1.
Therefore
\[
A(\langle M \rangle \# \langle w \rangle) = M(w) = 1
\]
This means that
\[
\langle M \rangle \# \langle w \rangle \in L(A)
\]
Hence $\ACCEPT_\TM \subseteq L(A)$.

Where are we?
At this point, I have shown that
\[
\ACCEPT_\TM = L(A)
\]
But don't forget: $A$ is a Turing decider!!!
That contradicts the fact that $\ACCEPT_\TM$ is not Turing decidable
which was proven in the previous section.
Therefore my assumption that $\HALT$ is decidable is incorrect.

We conclude that $\HALT$ is \textit{not} decidable.

\begin{thm}
$\HALT$ is not decidable, i.e., the halting problem is not decidable.
\end{thm}

Notice that the technique above reduces the decidability of 
the halting problem (or the decidability of the $\HALT$ language)
to the Turing--recognizability of the acceptance problem (or the
$\ACCEPT_\TM$ language).
This is a very standard technique in proving
the non--Turing--decidability of a 
language $L$.
You assume your language is decidable, say it's decided by a
Turing decider $M$ and you follow up by designing another Turing machine
(or decider) that contradicts some known result.
In the above case, a decider for $\HALT$ would result in a 
decider for $\ACCEPT_\TM$.

This technique is called \lq\lq reduction''.
The idea of reducibility therefore relates languages.
We say that $\HALT$ is reducible to $\ACCEPT_\TM$ and we write
\[
\ACCEPT_\TM \leq \HALT
\]
Our proof technique above is this:
The assumption that $\HALT$ is decidable forces $\ACCEPT_\TM$ to be
decidable as well.

In general, you have the following facts:

\begin{thm}
Let $L$ and $L'$ be two languages. Then
\begin{tightlist}
\item $L \leq L'$, $L'$ decidable $\implies$ $L$ decidable.
\item $L \leq L'$, $L$ not decidable $\implies$ $L$ not decidable.
\end{tightlist}
\end{thm}

You can define $\leq$ more formally.
You can think of $\leq$ as a way to relate
languages in the space of $\LANG_\TM$ through their complexity.
This is somewhat analogous to $\leq$ on real numbers $\R$. 
