\sectionthree{Back to Basics ... Counting (a la Cantor)}
\begin{python0}
from solutions import *; clear()
\end{python0}

Let's begin by reviewing what we mean by counting.
When I give you the following:
\[
\text{A, B, C, D, E}
\]
you should tell me that you are seeing 5 things.
Why is that? 
Because since you started to learn, you were told to associate
1, 2, 3, ... to the things you see.
So for instance in the above, you would do put your
finger on A and say \lq\lq one'',
move your finger to B and say \lq\lq two'', etc.
That's the same as doing this:
\begin{longtable}{ccccc}
A & B & C & D & E \\
$\uparrow$ & $\uparrow$ & 
$\uparrow$ & 
$\uparrow$ & 
$\uparrow$  \\
1 & 2 & 3 & 4 & 5
\end{longtable}
i.e., associating the number 1 to A (instead of associating your
verbal \lq\lq one'' to A), etc.
This association (think \lq\lq function'') is 1--1 and onto
(if you were taught to count correctly!)
If it's not 1--1, it might be something like this:
you might have for instance 1 and 2 pointing to A,
you would be counting A twice.
If the association is not onto, you would be missing some letter.

Another way to think of counting the above is there you're finding a
1--1 and onto function from the set 
\[
\{1, 2, 3, 4, 5\}
\]
to the set
\[
\{\text{A, B, C, D, E}\}
\]
Therefore counting
\[
\text{A, B, C, D, E}
\]
is the same as trying to a 1--1 and onto function from 
\[
\{1\}
\]
to $\{\text{A, B, C, D, E}\}$, or from 
\[
\{1, 2\}
\]
to $\{A, B, C, D, E\}$, or from 
\[
\{1, 2, 3\}
\] 
etc. until you succeed.
In our case we succeed when we use 
\[
\{1, 2, 3, 4, 5\}
\]
You can think of 
\[
\{1\}, \hspace{0.25cm}
\{1,2\}, \hspace{0.25cm}
\{1,2,3\}, \hspace{0.25cm}
\{1,2,3,4\}, \hspace{0.25cm}
\{1,2,3,4,5\}, \hspace{0.25cm}
\{1,2,3,4,5,6\}, \hspace{0.25cm}
\{1,2,3,4,5,6,7\}, ... 
\]
as our standard measurements for counting
sort of like standard weights for weighing things.
But why should we use 1--1 and onto functions and these sets as
a way to define counting?!?
Counting is so basic and such a primitive concept, why
bother trying to view counting as something involving sets, functions,
and 1--1 and onto functions?!?
Shouldn't \lq\lq counting'' be simpler than \lq\lq 1--1, onto functions''???

Well, the reason is because this definition of counting
uses sets ... and \textit{sets can be infinite}.
For instance
I can use the whole set of natural numbers to count
the number of things in the set $X$:
\[
\{0, 1, 2, 3, ...\} \rightarrow X
\] 
and the standard measuring set $\{0, 1, 2, 3, ...\}$ is not finite!!!
(Remember: depending on who you talk to the set $\N$ might or might not
contain 0.)

Therefore if you want to say that two sets $X$ and $Y$ are different,
one way would be to say that $X$ has as many things as 
$\N = \{0, 1, 2, 3, ...\}$
while $Y$ has as many things as $\R$ ... that is if you know that 
$\R$ is \lq\lq bigger'' than $\N$ which I will show you in a minute.

First of all let me say that two sets $X$ and $Y$ are equinumerous
(or informally, they have the same size) if there is a 1--1 and onto function
from $X$ to $Y$.
In that case I will write $|X| = |Y|$.

Next, because \lq\lq finite or same size as $\N$'' is so commonly used,
I will create a definition for it.
I will say that a set $X$ is \defone{countable} if one of the following
is true:
\begin{tightlist}
\item $X$ is finite, or
\item $|X| = |\N|$.
\end{tightlist}
If you do want to distinguish between the two, the first is said to be 
\defone{countably finite} and the second is said to be 
\defone{countably infinite}.

Now for some facts about countability ... be ready for this because
when you enter infinity ... things can be pretty weird.

First of all I claim that $\N$ has the same number of things
as $\N \cup \{\pi\}$, i.e.,
\[
|\N| = |\N \cup \{\pi\}|
\]
This seems to be obviously wrong.
But remember that counting means finding a 1--1 and onto function.
In this case to say that there are as many things in $\N$ as there
are in $\N \cup \{\pi\}$, I need to show you a 1--1 and onto function
between the two sets.
Here you go:
\begin{align*}
\N &\rightarrow \N \cup \{\pi\} \\
1 &\mapsto \pi \\
2 &\mapsto 1 \\
3 &\mapsto 2 \\
4 &\mapsto 3 \\
5 &\mapsto 4 \\
\vdots &\hspace{0.7cm} \vdots
\end{align*}
It should be clear that this function is 1--1 and onto.
Therefore
\[
|\N| = |\N \cup \{\pi\}|
\]

The above is closely related to a very interesting example created
by the famous German
mathematician David Hilbert called the Grand Hotel ...

You arrive at a hotel, the Grand Hotel (sometimes people call this
the Hotel Infinity).
The rooms are numbers 1, 2, 3, ... and there are as many rooms
as $\N$.
Unfortunately the hotel is full.
However the hotel owner, because he's really smart (probably
because he took automata theory),
simply told you to go to room 1 and tell the occupant in room 1
to go to room 2 and tell the occupant in room 2 to move to room 3 and
tell the occupant to do likewise, etc.
In general, the occupant in room $n$ moves to room $n + 1$
and the previous occupant in $n + 1$ moves to room $n + 2$.

As you can see the Grand Hotel is able to accomodate you even though 
it's already full 
Neat right?
(Unfortunately, all occupants are annoyed by having to move ...)

Now obviously, this scheme also works if you arrive with
a friend, i.e., there are \textit{two} new hotel guests.
No problem!
The two new guests occupy room 1 and room 2 and ... you know what to do.

But what if the number of new guests is not finite?
What if the number of new guests is in fact as numerous as $\N$ itself!!!
In terms out that 
\[
|\N \cup \N| = |\N|
\] 
Unbelievable!!!
Here's how you assign rooms 1, 2, 3, to $\N \cup \N$:
\[
pic
\]

What a happy owner of Grand Hotel!

But what if you have $\N \cup \N \cup \N$ guests?
No problemo amigo! (Sorry ...)
\[
pic
\]

Not only that ... in fact $| \N \cup \N \cup \N \cup \cdots | = |\N|$
where the union on the left is a union of countably infinite copies of 
$\N$. Wow!
To be more precise, I would write
\[
\left|
\bigcup_{n=1}^\infty \N
\right|
=
|\N|
\]
A union involving countable many sets is called a countable union.
The \lq\lq countable'' in \lq\lq countable union'' refers to the
number of sets you're trying to \lq\lq union-up'', not the sets themselves.
 
\begin{thm}
The countable union of countable sets is countable.
\end{thm}

Instead of cobbling together more and more copies of $\N$, 
let's take a break and look at other sets.
For instance there's $\Q$, the set of fractions.
Not surely there are more things in $\Q$ than in $\N$ because
$\Q$ contains $\N$ and surely the fraction $1/2$ is not in $\N$.
Right?

WRONG!

That's because you can associating values of $\N$ with values in $\Q$
in a wrong way.
You can think of $\Q$ as (more or less) $\N \times \N$.
For instance $3/7$ corresponds to the point $(3, 7)$ in $\N^2$
But wait a minute ... look at the picture of $\N \times \N$.
Doesn't it remind you of $\N \cup \N \cup \N \cup \cdots$?
To be more concrete, let me show you how to associate $\N$ with
the points in $\N \times \N$:
\[
pic
\]

There are some details you have to be careful about for instance
What about negative fractions?
The fractions also does not include something like $5/0$
(which is not defined!)
However in $\N \times \N$ we do have $(5, 0)$.
What about $(3, 6)$ that would correspond to the fraction $6/3$.
But this is also the same as $2/1$.
All these are not too difficult to overcome.

Note the technique very carefully.
If you try to label the points \textit{horizontally} like this:
\[
\]
you will get into trouble because you're ever going to return to 
start labeling the next row!!!

Your mind hurts, right?

%-*-latex-*-

\begin{ex} 
  \label{ex:prob-00}
  \tinysidebar{\debug{exercises/{disc-prob-28/question.tex}}}

  \solutionlink{sol:prob-00}
  \qed
\end{ex} 
\begin{python0}
from solutions import *
add(label="ex:prob-00",
    srcfilename='exercises/discrete-probability/prob-00/answer.tex') 
\end{python0}


Now at this point, you might give up and say that $\R$ is probably
as big as $\N$.
Actually ... $\R$ is strictly bigger than $\N$ in size!!!
Phew!!! For a moment ... we thought the whole world is nothing more than
$\N$.

The proof that $\R$ is in fact bigger than $\N$ is important 
because the technique itself will be used later in a totally different
context ... in the world of languages and automata.

This is so important that I need a new section!!! ...
