\tinysidebar{\debug{exercises/{universal-turing-machine5/question.tex}}}
  There are possibly many variations to the above description of $U$.
  For instance I've \lq\lq hardcoded'' the directions of the read-write head
  as
  \begin{align*}
    \langle L \rangle &= 1^1 \\
    \langle R \rangle &= 1^2 \\
    \langle S \rangle &= 1^3 \\
  \end{align*}
  You can also make your more TM \lq\lq flexible'' and tell the
  user to specify the encodings of $L$, $R$, $S$ by telling them where to
  put that information.
  For instance the encoding of a TM to be simulated can be of the form
  \[
  \langle M \rangle
  =
  \langle \Sigma \rangle 00
  \langle \Gamma \rangle 00
  \langle Q \rangle 00
  (\langle L \rangle 0
  \langle R \rangle 0
  \langle S \rangle) 00
  \langle \delta \rangle  
  \]
  Of course I have also hardcoded the fact that
  the first three states encoded within $\langle Q \rangle$
  are the initial state, the accept state, and the reject state.
  Again, you can make the specification of $\langle M \rangle$
  more flexible by allowing a user to choose the encoding for
  these states: you would have to include three places in your
  $\langle M \rangle$ to specify these states. For instance
  \[
  \langle M \rangle
  =
  \langle \Sigma \rangle 00
  \langle \Gamma \rangle 00
  \langle Q \rangle 00
  \langle \text{initial state} \rangle 00
  \langle q_\accept \rangle 00
  \langle q_\reject \rangle 00  
  \langle L \rangle 0
  \langle R \rangle 0
  \langle S \rangle 00
  \langle \delta \rangle  
  \]
  Note that although you're making the encoding of $M$ more flexible,
  you still have to somehow hardcode some things: you have to
  hardcode the \textit{position} in the encoding of $M$ where
  the above are placed.

  \textred{
  Using the TM software given in this class,
  build a universal TM that according to the \lq\lq more flexible''
  specification of the universal TM above.}
