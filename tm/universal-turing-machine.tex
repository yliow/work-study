\section{Universal Turing machine}

The universal Turing machine, frequently denoted by
$M_U$ or $U$, is a Turing machine that
will accept the description of a Turing machine $M$ and an input $w$
for $M$ and produce the same output as $M$ running on $w$.
Whenever a TM $M_1$ runs another TM $M_2$, I will say that $M_1$
simulates $M_2$.

First of all, I need to tell you what I mean that
the description of a Turing machine.

\subsection{Encoding a Turing machine}

Suppose you're given a Turing machine $M = (\Sigma, \Gamma, Q, q_0, q_\accept, q_\reject,
\delta)$.
First of all $\Gamma$ is finite, say (for simplicity),
\[
\Gamma = \{ \$, \sqcup, a, b\}
\]
Let me encode $ \$ $, $\sqcup$, $a$, $b$ as
\begin{align*}
  \langle \$ \rangle  &= 1^1 \\
  \langle \sqcup \rangle  &= 1^2 \\
  \langle a \rangle  &= 1^3 \\
  \langle b \rangle  &= 1^4 
\end{align*}
and then encode $\Gamma$ as
\[
\langle \Gamma \rangle
= 1^1 0 1^2 0 1^3 0 1^4
\]
(It should be clear what you need to do if $\Gamma$ has more symbols.)
Also, say
\[
\Sigma = \{ a, b \}
\]
then
\[
\langle \Sigma \rangle = 1^3 0 1^4
\]
Next say
\[
Q = \{q_0, q_1 = q_\accept, q_2 = q_\reject, \cdots, q_{10}\}
\]
Note that I'm assuming that the first state must be the initial state,
the second state must be the accept state, and the third state
must be the reject state.
I encode the states as follows:
\begin{align*}
  \langle q_0 \rangle  &= 1^1 \,\,\,\,\, \text{(initial state)} \\
  \langle q_1 \rangle  &= 1^2 \,\,\,\,\, \text{(accept state)} \\
  \langle q_2 \rangle  &= 1^3 \,\,\,\,\, \text{(reject state)} \\
  \langle q_1 \rangle  &= 1^4 \\
  \vdots &= \vdots \\
  \langle q_{10} \rangle  &= 1^{11} \\
\end{align*}
Note that I have \lq\lq hard-coded'' the initial state, accept state,
and reject state.
I then encode $Q$ as
\[
\langle Q \rangle =
1^1 0 1^2 0 1^3 0 \cdots 0 1^{11}
\]
For the read-write head directions, $D = \{ L, R, S \}$,
I can choose the following encoding:
\begin{align*}
  \langle L \rangle &= 1^1 \\
  \langle R \rangle &= 1^2 \\
  \langle S \rangle &= 1^3 \\
\end{align*}
Each $\delta(q, x) = (q', y, d)$ can be easily encoded.
For instance say
\[
\biggl\langle \delta(q, x) = (q', y, D) \biggr\rangle = 
\langle q \rangle 0
\langle x \rangle 0
\langle q' \rangle 0
\langle y \rangle 0
\langle d \rangle 0
\]
Clearly we can encode $\delta$ as a 
sequence of $\langle \delta(q, x) = (q', y, d) \rangle$
separated by $000$.
Then whole Turing machine $M$ can then be encoded as
\[
\langle \Sigma \rangle 00
\langle \Gamma \rangle 00
\langle Q \rangle 00
\langle \delta \rangle
\]
The $00$ acts as a separator between the components of $M$
and $000$ acts as a separator between the transition rules.

For instance consider the Turing machine $M$
that replaces $a$ and $b$ by $\sqcup$ and then halt.
\begin{align*}
  \delta(q_0, \$) &= (q_0, \$, R) \\
  \delta(q_0, a)  &= (q_0, \sqcup, R) \\
  \delta(q_0, b)  &= (q_0, \sqcup, R) \\
  \delta(q_0, \sqcup)  &= (q_\accept, \sqcup, S) \\
  \delta(q_\accept, \$) &= (q_\accept, \$, S) \\
  \delta(q_\accept, a)  &= (q_\accept, \sqcup, S) \\
  \delta(q_\accept, b)  &= (q_\accept, \sqcup, S) \\
  \delta(q_\accept, \sqcup)  &= (q_\accept, \sqcup, S) \\
  \delta(q_\reject, \$) &= (q_\reject, \sqcup, S) \\
  \delta(q_\reject, a) &= (q_\reject, \sqcup, S) \\
  \delta(q_\reject, b) &= (q_\reject, \sqcup, S) \\
  \delta(q_\reject, \sqcup) &= (q_\reject, \sqcup, S) \\
\end{align*}
I choose the encoding
\begin{align*}
  \langle \$ \rangle  &= 1^1 \\
  \langle \sqcup \rangle  &= 1^2 \\
  \langle a \rangle  &= 1^3 \\
  \langle b \rangle  &= 1^4 \\
\end{align*}
Then $M$ is described by
\[
\underbrace{(1^3 0 1^4)}_{\textstyle\langle \Sigma \rangle} 00
\underbrace{(1^1 0 1^2 0 1^3 0 1^4)}_{\textstyle\langle \Gamma \rangle} 00
\underbrace{(1^1 0 1^2 0 1^3)}_{\textstyle \langle Q \rangle} 00
\underbrace{(1^1 0 1^1 0 1^1 0 1^1 0 1^2) 000 \cdots}_{\textstyle \langle \delta \rangle}
\]
(I've added parentheses to make the description readable.)
Here
\begin{align*}  
\langle \Sigma \rangle &= 1^3 0 1^4 \\
\langle \Gamma \rangle &= 1^1 0 1^2 0 1^3 0 1^4 \\
\langle Q \rangle &= 1^1 0 1^2 0 1^3 \\
\langle \delta \rangle &= (1^1 0 1^1 0 1^1 0 1^1 0 1^2) 000 \cdots \\
\end{align*}
where
\begin{align*}
\biggl\langle \delta(q_0, \$) = (q_0, \$, R) \biggr\rangle 
&=
\langle q_0 \rangle 0
\langle \$ \rangle 0
\langle q_0 \rangle 0
\langle \$ \rangle 0
\langle R \rangle \\
&= 1^1 0 1^1 0 1^1 0 1^1 0 1^2
\end{align*}

\subsection{Hardcoding some encodings}

Note that you can choose any encoding for $\Gamma$.
But there are some encodings that must be fixed.
For instance, the encodings for the initial state of $M$,
the accept state and reject state must be fixed because
the universal Turing machine will be using the encodings
to determine how to initialize the machine to be simulated,
and how to halt the machine.

\subsection{Encoding inputs for a Turing machine}

Suppose that $\Sigma = \{a, b, \}$ just like the previous section
where
\begin{align*}
\langle a \rangle = 1^3 \\
\langle b \rangle = 1^4 
\end{align*}
If $w \in \Sigma^*$ is a string which is an input for
a Turing machine $M$, say
\[
w = aaba
\]
I want to talk about the encoding of $w$, which I will
denote by $\langle w \rangle$.
Note that
\[
\langle w \rangle =
\langle aaba \rangle =
\langle a \rangle 
\langle a \rangle 
\langle b \rangle 
\langle a \rangle 
\]
is not going to work since in that case
\[
\langle w \rangle =
1^3 1^3 1^4 1^3
\]
So I will use 0 as separators.
In other words
\[
\langle w \rangle =
\langle aaba \rangle =
\langle a \rangle 0
\langle a \rangle 0
\langle b \rangle 0
\langle a \rangle
=
1^3 0 1^3 0 1^4 0 1^3
\]

\subsection{How to setup $U$ for execution}

Note that to use the universal Turing machine $U$, a user must put
\[
\langle M \rangle \#
\langle w \rangle 
\]
on the input tape of $U$.
Note that the universal Turing machine recognize the
special character $\#$.
This character $\#$ is just to separate the description
of the Turing machine $M$
and the input $w$.
(Technically speaking, instead of $\#$, 
we could have used $0000$ or even $\$ $ instead of $\#$.
Anything that can uniquely tell us \lq\lq end of $M$'' would do.
But it has to be fixed.)


\subsection{How does $U$ do its work: the big picture}

During execution, $U$'s tape would look like
\[
\$ \langle M \rangle \# \langle w \rangle \$ x
\]
where $x$ is (basically) the instantaneous description (ID)
of $M$ when $M$ runs on input
$w$.
Note that the description of $M$ is never changed.
Likewise the encoding of $w$ is also never changed.
In a sense the tape of $U$ is made up of three sections.
In fact in most books, $U$ is described as a multi-tape Turing machine.
Also, some books use extra tapes to keep for instance the initial state
encoding, etc.
Actually we don't really need to keep $w$, but I'm just going to keep it
there anyway.

\subsection{How does $U$ do its work: details and encoding of ID of $M$}

Note that the simulated ID contains the encoded characters of $M$
and a state.
For instance this might be an ID while $M$ is running:
\[
a a q_3 b a
\]
i.e., $M$ is at state $q_3$, the tape has $aaba$,  and
$M$ is about to read the third character.
Of course in $M$, the symbols $a$, $b$, ... and $q_0, q_1$, ... are distinct
symbols.
But (in $U$) because we're encoding all the above
into $1$s
the above ID
\[
a a q_3 b a
\]
on the tape for $U$ now is just a bunch of $1$s!!!
\[
1^3 1^3 1^4 1^4 1^3
\]
YIKES!!!
That's not going to work.

First we'll encode the above ID by add $0$s to separate the
encoding of the symbols.
This gives us
\[
\langle a \rangle 0
\langle a \rangle 0
\langle q_3 \rangle 0
\langle b \rangle 0
\langle a \rangle
\]
i.e.,
\[
1^3 0
1^3 0 
1^4 0
1^4 0
1^3
\]
Ahhh ... \textit{now} we can see that there are 5 things in the ID.

\textit{But wait} ... I still have to differentiate between the encoding
of states and read-write symbols.
In the encoding scheme of $M$,
$\langle q_3 \rangle = \langle b \rangle = 1^4$!!!
They are encoded the same as parts of the encoded ID!!!
(The reason why there's no confusion in the encoding of $M$
is because the states and read-write symbols appear in fixed places
in $\langle M \rangle$.)

What I'll do is this:
since $@$ will never appear in the encoding of
an ID of $M$, we can use it to tell us that what follows is a state.
In other words, the above ID is encoded as
\[
1^3 0 1^3 @ 1^4 0 1^4 0 1^3 
\]
i.e., in the encoding of an ID,
\[
@ 1^4
\]
means the fourth state in $Q$ of $M$ while
\[
0 1^4
\]
means the fourth character in $\Gamma$ of $M$.

Get it?

\subsection{How does $U$ simulate $M$ on $w$.}

So say you have a TM $M$ and you want to run $M$ with input $w$.
First you choose an encoding for $M$ -- remember that the
encoding for the initial state, accept state, and reject state of $M$
is fixed.
With that you get the encoding of $M$ and $w$.
You then enter
\[
\langle M \rangle
\#
\langle w \rangle
\]
into the tape of $U$.
(Don't forget that $U$ understands $0$, $1$ and $\#$.)
The input tape looks like
\[
\langle M \rangle \# \langle w \rangle 
\]
When you run $U$, it first modifies the tape to become this:
\[
\$ \langle M \rangle
\#
\langle w \rangle
\$
@
\langle q_0 \rangle
0\langle w \text{ with 0 separating symbols}\rangle
\]
(Technically speaking, I don't really need to retain
$\langle w \rangle$ in the second section of the tape.)
In other words, $U$ adds the encoding of the initial ID of $M$ as the
third section of the $U$'s tape.

Suppose $w = aaba$ and I have chosen
$\langle a \rangle = 1^3$ and
$\langle b \rangle = 1^4$.
Then the tape of $U$ above is
\[
\$ \langle M \rangle \# \langle w \rangle
\$ @ 1^1 0 1^3 0 1^3 0 1^4 0 1^3
\]
From this part of the string:
\[
\$ \langle M \rangle \# \langle w \rangle
\$ \underline{@ 1^1 0 1^3} 0 1^3 0 1^4 0 1^3
\]
my $U$ can see that in the simulation of $M$ running on $w$,
at this point, the encoded state is $1^1 = \langle q_0 \rangle$,
i.e., the state of $M$ is $q_0$.
Furthermore $M$'s read-write head is
pointing to $\langle a \rangle = 1^3$, i.e., to $a$.
Therefore $M$ must execute according to
the value of
\[
\delta(q_0, a)
\]
Suppose
\[
\delta(q_0, a) = (q_4, b, R)
\]
Note that
\[
\biggl\langle
\delta(q_0, a) = (q_4, b, R)
\biggr\rangle
=
1^1 0 1^3 0 1^5 0 1^4 0 1^2
\]
Therefore this must be in $\langle M \rangle$.
\[
\$
\underbrace{...
  00
  \overbrace{1^1 0 1^3 0 1^5 0 1^4 0 1^2}
  ^
  {\textstyle \langle \delta(q_0, a) = (q_4, b, R) \rangle}
  00
  ...}_{\textstyle \langle M \rangle}
\#
1^3 0 1^3 0 1^4 0 1^3
\$
\underline{@ 1^1 0 1^3} 0 1^3 0 1^4 0 1^3
\]
$U$ will need to look for $ @ $ in the ID, then match
what follows (the state and the character)
with the first part of the relevant transition in $\langle M \rangle$:
\[
\$
...
  00
  \underline{1^1 0 1^3} 0 1^5 0 1^4 0 1^2
  00
  ...
\#
1^3 0 1^3 0 1^4 0 1^3
\$
@ \underline{1^1 0 1^3} 0 1^3 0 1^4 0 1^3
\]
Clearly it's easy to design $U$ so that the above
match is found.
Note that $U$ has to check each transition one at a time until
the right transition is found.
I would need to mark the transition rule that I need to
try to match.
I can for instance use $ @ $ in $\langle M \rangle$ to
mark the transition rule $U$ is checking.
If the marked transition rule does now what what $U$
needs to rewrite the encoded ID (to simulate the execution of
one computation step of $M$), then if replaces the @ with 0 and
move onto the next transition rule.
At some point the correct transition rule is found:
\[
\$
...
  0@
  \underline{1^1 0 1^3} 0 1^5 0 1^4 0 1^2
  00
  ...
\#
1^3 0 1^3 0 1^4 0 1^3
\$
@ \underline{1^1 0 1^3} 0 1^3 0 1^4 0 1^3
\]



Once a match is found, based on the rest of the
string describing the transition
\[
\$
...
  0@
  \underline{1^1 0 1^3 0 1^5 0 1^4 0 1^2}
  00
  ...
\#
1^3 0 1^3 0 1^4 0 1^3
\$
@ \underline{1^1 0 1^3} 0 1^3 0 1^4 0 1^3
\]
$U$ modifies the encoded ID accordingly.
Note that $U$ would need to move back and force
between the two $ @ $ in order to do that.

After each simulation (i.e., ID rewriting),
$U$ examines the ID and checks if the state is
an accept or reject state of $M$.
If $U$ see an accept state, $U$ cleans up
the third section of the tape by removing the state
(so the the third section of the tape of $U$ imitates
the tape of $M$) and then $U$ enters its accept state.
Likewise if $U$ sees a reject state, except that $U$
enters its reject state.
Otherwise $U$ continues simulating $M$.

(If I want to, I can even modify $U$ so that before $U$
enters the accept or reject state,
$U$ replaces its tape with the
encoding of the
tape of $M$ when $M$ halts.)

\begin{thm}
There is a Turing machine $M_U$ (or $U$)
such that for every TM $M$ with input $w$,
$U$ accepts $\langle M \rangle \# \langle w \rangle$
iff $M$ accepts $w$
and 
$U$ rejects $\langle M \rangle \# \langle w \rangle$
iff $M$ rejects $w$.
\textred{Specifically:
  If $M$ rejects $w$ by entering its reject state, then
  $U$ will also enter its reject state;
  if $M$ rejects $w$ by never halting, then $U$ also
  never halt while executing on the input
  $\langle M \rangle \# \langle w \rangle$.}
\qed
\end{thm}

There's something that's really really really important (but obvious),
but if you're
not careful you might not realize ...

Do you see that although $U$ has finitely many states, \textit{but}
the TM that $U$ simulates can have an arbitrary number of states.
What I mean is that, say our $U$ has 100 states.
$U$ can simulate a TM with 1000 states,
and it can simulate a TM with $1,000,000$ states,
and it can simulate a TM with $1,000,000,000,000$ states.

As an aside, note that for $U$, the substring of $0$s have
special meanings.
Although I'm using \$ as a symbol used by $U$ to mark beginning of
tape, note that if $0^5$ is not in used, then $0^5$ can also play the role
of \$.
We can also use $0^5$ intead of $\#$ too if we like.

\subsection{Input checking}

One other thing.
Before $U$ simulates
$M$ running $w$,
it's a good idea for $U$ to do the following:
\begin{tightlist}
  \item Check that the input is of the form
  \[
  x \# y
  \]
  where $x, y \in \{0,1\}^*$.
  \item Check the first part of input, i.e.,
the $x$ in
\[
x \# y
\]
represents a valid TM $M$. Clearly
a random string of 0s and 1s will not necessarily be a TM according to
our TM encoding scheme.
\item Check that $y$ in
\[
x \# y
\]
represents a string
that the TM $M$ can run with.
\end{tightlist}


\newpage
\begin{ex} (String search)
  Construct a TM such that when given a string of the form
  \[
  \# x_1 \# x_2 \# x_3 \# x_4 \# \cdots \# x_n \#\# y
  \]
  with $x_i, y \in \{0, 1\}^*$, it will find the first $x_i$
  that is equal to $y$ and mark it by changing the \# in front of
  $x_i$ from \# to @ and then accepts.
  If no such $x_i$ is found, the machine rejects the input.
  For instance when given this:
  \[
  \# 0010 \# 11000 \# 10101010 \# 11100  \#\# 10101010
  \]
  The machine accepts with this on the tape:
  \[
  \$\# 0010 \# 11000 @ 10101010 \# 11100  \#\# 10101010
  \]
  If you can do the above, then you believe that the part
  of $U$ that finds the relevant transition rule is doable.
  \qed
\end{ex}

\newpage
\begin{ex} (String insert)
  Construct a TM such then given a string of the form
  \[
  \# x_1 \# x_2 \# x_3 \# \textred{\cdots} @ x_i \# x_{\textred{i}+1} \# \cdots \# x_n \#\# y@z
  \]
  if \lq\lq inserts a copy of $x_i$ after \textred{@''}.
  For instance when given this input:
  \[
  \# 000111 \# 01 @ 111 \# 0 \# 1010 \#\# 1010@0101
  \]
  the TM will halt with this on the tape:
  \[
  \$\# 000111 \# 01 @ 111111 \# 0 \# 1010 \#\# 1010@1111110101
  \]
  If you do the above, then you believe that the part of $U$
  that changes one state encoding to another is doable.
  \qed
\end{ex}


\newpage
\begin{ex} (Moving a substring)
  Construct a TM that such given a string of the form
  \[
  \# c \# x @y00 z 
  \]
  where $c$ is 1 or 11 or 111, $x,z in \{1\}^*$, and $y \in \{0, 1\}^*$
  that does not contain 00,
  then if $c = 1$, the string $@y00$ will move to the left by 1,
  if $c = 11$, the string $@y00$ will move to the right by 1,
  and if $c = 111$, the input does not change.
  For instance if the input is
  \[
  \# 1 \# 11111 @11100 11111 
  \]
  then the TM will change the input to this:
  \[
  \# 1 \# 1111 @11100 111111 
  \]
  and accept.
  If the input is
  \[
  \# 11 \# 11111 @11100 11111 
  \]
  then the TM will change the input to this:
  \[
  \# 11 \# 111111 @11100 1111 
  \]
  and accept.
  You get the idea.
  If the input is not of the form above, the TM rejects.
  
  If you can do the above, then you will believe that it's possible
  to move the encoding of the state in the encoding of the ID
  to the left or to the right or make it stay.
  \qed
\end{ex}
  
\newpage
\begin{ex}
  Construct a universal TM - mathematically.
  \qed
\end{ex}


\newpage
\begin{ex}
  Using the TM software provided in this class,
  construct a universal TM.
  Test it.
  \qed
\end{ex}


\newpage
\begin{ex}
  There are possibly many variations to the above description of $U$.
  For instance I've \lq\lq hardcoded'' the directions of the read-write head
  as
  \begin{align*}
    \langle L \rangle &= 1^1 \\
    \langle R \rangle &= 1^2 \\
    \langle S \rangle &= 1^3 \\
  \end{align*}
  You can also make your more TM \lq\lq flexible'' and tell the
  user to specify the encodings of $L$, $R$, $S$ by telling them where to
  put that information.
  For instance the encoding of a TM to be simulated can be of the form
  \[
  \langle M \rangle
  =
  \langle \Sigma \rangle 00
  \langle \Gamma \rangle 00
  \langle Q \rangle 00
  (\langle L \rangle 0
  \langle R \rangle 0
  \langle S \rangle) 00
  \langle \delta \rangle  
  \]
  Of course I have also hardcoded the fact that
  the first three states encoded within $\langle Q \rangle$
  are the initial state, the accept state, and the reject state.
  Again, you can make the specification of $\langle M \rangle$
  more flexible by allowing a user to choose the encoding for
  these states: you would have to include three places in your
  $\langle M \rangle$ to specify these states. For instance
  \[
  \langle M \rangle
  =
  \langle \Sigma \rangle 00
  \langle \Gamma \rangle 00
  \langle Q \rangle 00
  \langle \text{initial state} \rangle 00
  \langle q_\accept \rangle 00
  \langle q_\reject \rangle 00  
  \langle L \rangle 0
  \langle R \rangle 0
  \langle S \rangle 00
  \langle \delta \rangle  
  \]
  Note that although you're making the encoding of $M$ more flexible,
  you still have to somehow hardcode some things: you have to
  hardcode the \textit{position} in the encoding of $M$ where
  the above are placed.

  \textred{
  Using the TM software given in this class,
  build a universal TM that according to the \lq\lq more flexible''
  specification of the universal TM above.}
  \qed
\end{ex}

\newpage
\begin{ex}
  A TM can potentially run forever.
  Describe a universal TM $U'$
  that allows you to specify how long to
  simulate $M$ on input $w$.
  Note that this $U'$ will always halt:
  if $M$ accepts/rejects $w$ in $n$ steps, then $U'$ halts in the
  accept/reject state.
  If $M$ does not enter its accept/reject state in $n$ steps,
  then $U'$ enters its reject state and halt.
  [HINT: For the input, you need the encoding of the
  TM, encoding of the input, and encoding of \lq\lq time''.]
  \qed
\end{ex}

\newpage
\begin{ex}
  Design a universal TM $U''$ that simulates two machines,
  running the pair in a fair manner, i.e., run one transition
  on one of the TM and then run one transition on the other
  and then repeat.
  If any of the two accepts, $U'$ accepts.
  If both rejects, $U'$ rejects.  
  %If anyone of the two halts, just continuing simulating the remaining.
  %Of course if both halts, $U''$ halts too.
  \qed
\end{ex}

\newpage
\begin{ex}
  Design a universal TM $U'''$ that simulates a TM $M$ running on $w$
  and keeps a history of its computation, i.e., it record of all the IDs.
  If $U'''$ sees a two completed IDs which are the same,
  $U'''$ halts in the reject state.
  Otherwise it halts in the same halt state of $M$ running on $w$.
  \qed
\end{ex}
