\sectionthree{Generators}
\begin{python0}
from solutions import *; clear()
\end{python0}

A TM $M$ is a generator if it has several ``work'' tapes and a
special ``output'' tape. This is how the TM is used. All tapes are
initially blank. You run the TM without any input. While the
machine is running, you observe the output tape. At any point in
time, if you see the string $\#w\#$ where $\#$ is a special
delimiter and $w$ is a string 
(the second $\#$ basically is an end--of--string marker to tell us the 
output for the current string is done), then you say that the $M$ has just
generated $w$. We will write $G(M)$ for the set of strings
generated by $M$. Note there is no particular order in the strings
being generated. Also, some strings can be generated more than
once.

Note that this is not more powerful than
the regular TM.

\begin{thm}
\begin{tightlist}
\item If $M$ is a TM, then there is some generators $M'$ such that
$G(M') = L(M)$.
\item If $M'$ is a generator, then there is some TM $M$ such that
$L(M) = G(M')$.
\end{tightlist}
\end{thm}
