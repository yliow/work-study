\sectionthree{RAM - random access machine}
\begin{python0}
from solutions import *; clear()
\end{python0}

Our TM moves on its input tape relatively: it moves to the left or right or
stay with respect to its current position.
What if we allow the TM to go to the memory cell at any index
position?
(This is like a modern computer accessing a byte in its RAM at
any address location.)
Specifically, this TM has another tape -- I'll call it the register tape.
It contains the index that the read-write head should go to.
This means the following:
Suppose the random access machine has its read-write head at index
position $n$ and it's at state $q$.
Furthermore the character at index $n$ of the input tape is $c$.
If the transition is
\[
\delta(q, c) = (q', c')
\]
this means that the character $c$ at index $n$ is ovewritten by character
$c'$, the machine enters state $q'$ and the read-write goes to
cell at index position $n$.
Initially, the read-write head is at index 0.

Let $\TM_{\operatorname{RAM}}$ denote the class of such machines.

%-*-latex-*-

\begin{ex} 
  \label{ex:prob-00}
  \tinysidebar{\debug{exercises/{disc-prob-28/question.tex}}}

  \solutionlink{sol:prob-00}
  \qed
\end{ex} 
\begin{python0}
from solutions import *
add(label="ex:prob-00",
    srcfilename='exercises/discrete-probability/prob-00/answer.tex') 
\end{python0}


More generally, a random access machine can have more than one, but a
finite number of, register tapes.
In that case the transition would be $delta(q, c) = (q', c', i)$
where the position the read-write head should move to is on register
tape number $i$.

The above is my simplification of a Turing random access machine.
The full blown random access machine is more like a modern computer.

