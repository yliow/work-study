\section{Turing Decidable $\neq$ 
Turing Recognizable:
The Acceptance Problem}

I have just shown you that there is language that is not
Turing recognizable, i.e., the 
collection of Turing recognizable language is not the 
set of all languages.

In this section, I want to show you
that the collection of Turing decidable language is not the 
collection of Turing recognizable languages.

I have already shown you that $\Sigma^*$ is countable.
Therefore the words in $\Sigma^*$ can be listed 
as $w_0, w_1, w_2, ...$.

Remember that each Turing machine can already be encoded as a
sequence of $0$'s and $1$'s.
Likewise we can also encode $w_i$ which I will also write
as $\langle w_i \rangle$.

%We already know that
%$\ACCEPT_\TM$
%is Turing--recognizable.
%Now if 
%$\ACCEPT_\TM$
%is Turing--recognizable, then
%by the above theorem,
%$\ACCEPT_\TM$
%is also Turing decidable.

Define
\[
\ACCEPT_\TM
= \{ \langle M \rangle \# \langle w \rangle \mid M \text{ accepts } w \}
\]
Note that this $\Sigma_1$ set has
another character $\#$ and the purpose is just to 
separate the TM encoding from the word encoding, i.e.,
$\Sigma_1 = \{0, 1, \#\}$.


This language 
\[
\ACCEPT_\TM
\]
has some interesting properties.

First of all,
$\ACCEPT_\TM$ 
is Turing--recognizable.
Why?
Because the universal TM $M_U$
can be used.
Build a TM $\cal M$
like this:
If an input is not of the form
$\langle M \rangle \# \langle w\rangle$,
this TM $\cal M$
just rejects.
If it is, $\cal M$
will use $M_U$ to simulate
$M$ running with input $w$.
If $M_U$ accepts, the TM $\cal M$ will also accept.
Likewise for the reject case.
If the simulation of $M_U$ of $M$ running on $w$
does not halt, then $M_U$ does not halt
and our TM $\cal M$ also does not halt,
therefore 
$\langle M \rangle \# \langle w\rangle$
is not accepted.

Let's record this ...

\begin{thm}
$\ACCEPT_\TM$
is Turing--recognizable.
\end{thm}

Now, I am going to show you that $\ACCEPT_\TM$ is not 
decidable.
The proof is pretty cunning.

I will prove this by contradiction.
Suppose
$\ACCEPT_\TM$
is Turing--decidable, say it is accept by $A$
where $A$ is a TM that always halts, i.e., it will always enter the
$q_\accept$ or the $q_\reject$ state for any input.
Furthermore since $A$ accepts $\ACCEPT_\TM$, we know that 
$A$ accepts $\langle M  \rangle \#\langle w \rangle$
when $M$ accepts $w$, and rejects otherwise.
For simplicity, if $M$ is a Turing machine and $w$ is a word,
I will write
\[
M(w) = 1
\]
if $M$ enters $q_\accept$ at some point when given an input of $w$
and 
\[
M(w) = 0
\]
if $M$ enters $q_\reject$ at some point when the input is $w$.
Don't forget that it's possible for $M$ to run forever
for input $w$.
In this case I will write
\[
M(w) = \infty
\]
This third case cannot happen if $M$ is a Turing decider.

By our definition of $A$, 
we have
\[
A(\langle M \rangle \# \langle w\rangle) = 1
\iff M \text{ accepts } w
\]
and 
\[
A(\langle M \rangle \# \langle w\rangle) = 0
\iff M \text{ does not accepts } w
\]
Note that since $A$ is a Turing decider, we will never have
\[
A(\langle M \rangle \# \langle w\rangle) = \infty
\]

I will now build a Turing machine $A'$ as follows.
It will take as input 
$\langle M \rangle$.
(If the input is not the encoding of a TM, $A'$ just rejects.)
$A'$ then simulates $A$ running with input $\langle M \rangle \# \langle M \rangle$.
While it simulates $A$, if $A$ enters $q_\accept$, 
$A'$ will do the opposite: 
$A'$ will enter its $q_\reject$ state.
Likewise if 
if $A$ enters $q_\reject$, 
$A'$ will do the opposite: $A'$ 
will enter its $q_\accept$ state.
Note that since $A$ is a decider, $A$ will always halt in the
accept or reject state -- it cannot run forever.
Therefore $A'$ will also never run forever -- $A'$ must
enter the accept or reject state.
Now what?


First of all as I have just said, $A'$ cannot run forever (for any input).
$A'$ must halt (i.e., enter its accept or reject state).
Therefore $A'$ is a decider.
Therefore
\[
A'(\langle M \rangle) = 0 \text{ or } 1
\]
i.e., cannot be $\infty$.
(If the input is not the encoding of the TM, $A'$ just reject.)
Let's look at these two cases carefully.

Suppose $A'(\langle M \rangle) = 1$.
Then the simulation of $A$ with input 
$\langle M \rangle \# \langle M \rangle$ must give 0 by definition.
In the same way, if
$A'(\langle M \rangle) = 0$
then the simulation of $A$ on 
$\langle M \rangle \# \langle M \rangle$ must give 1.
In other words,
\[
A'(\langle M \rangle) = 1
\iff
A(\langle M \rangle \# \langle M \rangle) = 0
\tag{1}
\]
But don't forget that $A$ is the decider for $\ACCEPT_\TM$.
So we have
\begin{align*}
A(\langle M \rangle \# \langle M \rangle) = 1
&\iff M(\langle M \rangle) = 1 \\
A(\langle M \rangle \# \langle M \rangle) = 0
&\iff M(\langle M \rangle) = 0, \infty
\end{align*}
In the case when $M$ is a decider, $M$ will never run forever.
So when $M$ is a decider, the above becomes
\begin{align*}
A(\langle M \rangle \# \langle M \rangle) = 1
&\iff M(\langle M \rangle) = 1 \\
A(\langle M \rangle \# \langle M \rangle) = 0
&\iff M(\langle M \rangle) = 0
\end{align*}
i.e.,
\[
A(\langle M \rangle \# \langle M \rangle)
= M(\langle M \rangle)
\tag{2}
\]

Putting (1) and (2) together, when $M$ is a decider,
we get
\[
A'(\langle M \rangle) = 1
\iff
A(\langle M \rangle \# \langle M \rangle) = 0 = M(\langle M \rangle)
\]

Now what? ... \textit{drumroll} ... 

We run $A'$ with $\langle A' \rangle$!!!
Why? ... Because the above becomes
\[
A'(\langle A' \rangle) = 1
\iff
A(\langle A' \rangle \# \langle A' \rangle) = 0
= A'(\langle A' \rangle)
\]
Checkmate!!! ... because the above
gives this bizarre contradiction:
\[
A'(\langle A' \rangle) = 1
\iff
A'(\langle A' \rangle) = 0
\]


Let me write down what we have at this point ...

\begin{thm}
$\ACCEPT_\TM$ is Turing--recognizable but 
 not Turing--decidable.
\qed
\end{thm}

Recall that I have already shown you that
$\DIAGONAL$ is not Turing--recognizable.
Now I give you another:

\begin{thm}
$\overline{\ACCEPT_\TM}$ is not Turing--recognizable.
\end{thm}

Why?
Recall this result from the section on closure rules:
\[
L, \overline L \text{ Turing recognizable } \iff L \text{ decidable} 
\]

This implies that

\begin{thm}
If $L$ is Turing--recognizable and not Turing--decidable, then
$\overline L$ not Turing--recognizable.
\end{thm}

Why?
Otherwise, if $L$ and $\overline L$ are both Turing--recognizable,
then $L$ is decidable but $L$ is not decidable.

This and the theorem just proven immediately tells us that
$\overline{\ACCEPT_\TM}$ is not Turing--recognizable.
So now you have two languages which are not Turing--recognizable:
\begin{tightlist}
\item $\DIAGONAL$
\item $\overline{\ACCEPT_\TM}$
\end{tightlist}


