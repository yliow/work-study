%-*-latex-*-
\section{The Famous P = NP Problem}


Now I'm going to described a very famous and unsolved problem
in Computer Science and Mathematics:
the $\POLY = \NP$ problem.
This requires some definitions first.

First I have to define $\POLY$.

The class
\[
\TIME(n^d)
\]
is defined to be the collection of problems that can be solved by 
an algorithm with a runtime of $O(n^d)$.
For instance
here's a problem:
\[
\SORTARRAY:
\text{Sort an array in ascending order}
\]
For simplicity, 
I will assume that the arrays always have distinct integer values.

We know that an array can be sorted in $O(n \ln n)$ times
(for instance using quicksort).
Since $O(n \ln n) \subset O(n^2)$,
we have
\[
\SORTARRAY \in \TIME(n^2)
\]

It's important to see the difference between 
$\TIME(n^2)$ and 
$O(n^2)$.
$\TIME(n^2)$ is a collection of {\it problems}
whereas $O(n^2)$ is a collection of {\it functions}.

The $\POLY$ class is defined to the 
collection of all problems that can be solved by 
a polynomial time algorithm, i.e.,
\[
\POLY = \bigcup_{d = 0}^\infty \TIME(n^d)
\]

Now I'm going to define another collection of problems.
The collection
\[
\NTIME(n^d)
\]
is the collection of problems where there is an algorithm
with runtime of $O(n^d)$ that can check if a given solution to the 
problem is actually a solution or not.

To make sure you understand $\NTIME(n^d)$ let me give you an example.
Let's look at our problem again:
\[
\SORTARRAY:
\text{Sort an array in ascending order}
\]
Suppose I give you a problem in $\SORTARRAY$:
\[
\text{Sort $[6, 3, 2, 9, 8]$ in ascending order}
\]
Now, I'm not asking you to sort $[6,3,2,9,8]$.
Rather, I'm going to give you another problem.
I'm going to give you another an array 
$[2,3,6,8,9]$ and you have to tell me
this second array is the sorted version of the first
array $[6,3,2,9,8]$.
Here's an instance of this new problem:
\[
\text{VERIFY-SORT-ARRAY: 
After $[6, 3, 2, 9, 8]$ is sorted, you get $[2,3,6,8,9]$}
\]
Of course your algorithm must say \lq\lq YES'' in this case.
However if I give you {\it this} instance:
\[
\text{VERIFY-SORT-ARRAY: 
After $[0, 3, 2, 4, 8]$ is sorted, you get $[8,2,3,4,0]$}
\]
your algorithm must say \lq\lq NO''.
Your algorithm must also say \lq\lq NO'' to this instance:
\[
\text{VERIFY-SORT-ARRAY: 
After $[0, 3, 2, 4, 8]$ is sorted, you get $[1,1,1]$}
\]

\begin{ex}
Design an algorithm for VERIFY-SORT-ARRAY that runs in 
polynomial (in fact quadratic) time.
\qed
\end{ex}

Let me repeat that because that's lot to swallow.
Suppose $P$ is a problem.
Suppose that when given a specific problem instance $p$
and a potential solution $s$,
there is a verifying algorithm $V$ that takes $(p, s)$ and outputs 
\lq\lq YES'' if $s$ is a solution of $p$
and outputs \lq\lq NO'' if $s$ is not a solution of $p$,
and furthermore if this verifying algorithm $V$ runs in polynomial time,
then this problem $P$ is in $\NP$.


Finally we define
\[
\NP = \bigcup_{d = 0}^\infty \NTIME(n^d)
\]
Informally, $\NP$ is the collection of problems that can be
verified with solutions in polynomial time.

The famous
\[
\POLY = \NP \text{ problem}
\]
or 
\[
\POLY \neq \NP \text{ problem}
\]
asks if the two collections of problems $\POLY$ and $\NP$ are the same.

This is an extremely important question.
Why?

First of all, the problems in $\NP$ are known to have exponential runtime.
Why is that?
For instance in the case of $\SORTARRAY$ problem, we have a 
verifying algorithm $V$ that runs in polynomial time.
Now how do we solve $\SORTARRAY$ in exponential time?
For each problem instance $p$, we continually generate all arrays
$s_0, s_1, s_2, \ldots$ and feed $(p, s_i)$ into $V$.
When $V(p, s_i)$ says \lq\lq YES'', then of course $s_i$ is the
sorted array of the problem $p$.
Why does this algorithm run in exponential time?
Because it might take exponential time to reach the correct solution 
$s_i$.
Pictorially, suppose I want to sort $[10,3,5,2,4]$ using $V$,
I would see this algorithm doing something like this:
\begin{Verbatim}[frame=single,fontsize=\footnotesize]
EXPONENTIAL-TIME-SORTING-ALGORITHM([10,3,5,2,4,6]):
  V([10,3,5,2,4], [])           = NO  
  V([10,3,5,2,4], [2])          = NO  
  V([10,3,5,2,4], [5,7])        = NO  
  V([10,3,5,2,4], [0,2,6,42])   = NO 
  V([10,3,5,2,4], [9,17,2])     = NO 
  ...
  V([10,3,5,2,4], [2,3,4,5,10]) = YES 
                        ---> OUTPUT [2,3,4,5,10]} 
\end{Verbatim}
As you can see, you might have to wait for a very long time for the 
right array to appear!

It's easy to see that 
\[
\POLY \subseteq \NP
\]

If 
\[
\POLY = \NP
\]
is true, this implies that problems with extremely bad exponential runtime
algorithms actually have a polynomial runtime algorithm.
It seems that most researchers working in areas related to 
the $\POLY = \NP$ problem tend to believe
\[
\POLY \neq \NP
\]
The $\POLY = \NP$ or $\POLY \neq \NP$ problems are still unsolved.
In fact there is a \$1M reward for solving either problem.
(Google for \lq\lq millenium problems clay institute''.) 

The above is only a very brief overview of the problem.
To get a better understanding of this problem and other insanely fun
stuff, you would have to 
know topics like universal Turing machines, etc.
(See CISS362, Theory of automata.)
