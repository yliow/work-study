%-*-latex-*-
\section{$n + c = O(n)$, $c \geq 0$} 
Suppose 
\[
g(n) = n + 1
\] 
and 
\[ 
f(n) = n
\] 
I claim that $f(n) \in O(g)$. 
Well, to say that I have to show you that
\[
\text{there exist $C$ and $N$ such that $|f(n)| \leq C|g(n)|$ 
for $n \geq N$}
\]
Putting in the definition of $g(n)$ and $f(n)$, I need to show you
\[
\text{there exist $C$ and $N$ such that $|n| \leq C|n+1|$ 
for $n \geq N$}
\]
In other words I need to find some numbers $C$ and $N$ satisfying
\[
\text{$|n| \leq C|n+1|$ for $n \geq N$}
\]
Well if I choose $C = 5$ and $N = -3$, then I want to know if 
the following is true:
\[
\text{$|n| \leq 5|n+1|$ for $n \geq -3$}
\]
That's a terrible choice! Because I could have chosen a positive $N$ and then for $n > N$, the condition
\[
\text{$|n| \leq C|n+1|$ for $n \geq N$}
\]
becomes
\[
\text{$n \leq C(n+1)$ for $n \geq N$}
\]
since $n \geq N$ and $N$ is positive implies that $n$ (and therefore $n+1$ as 
well) must be positive. In other words I don't really need the absolute values at all.

OK. 
So first we insist that we choose a positive $N$ 
(it can be $N = 0$ or $N = 42$, etc., we will be specific later.) 
So we need to find a $C$ and a positive $N$ such that
\[
\text{$n \leq C(n+1)$ for $n \geq N$}
\]

Duh! Wait a minute ... this statement
\[
\text{$n \leq n+1$}
\]
is true for any positive $n$! So let's choose $C = 1$ and $N = 0$. Then
\[
|n| \leq 1 \cdot |n+1| \text{ for $n > 0$ } 
\]
Hence $f(n) \in O(g)$.

Now I'm done with the experimenting and designing the proof.
I'm going to write everything down neatly so that I can check that there are
no gaps in my proof.

But before that, let me list some basic facts about inequalities
from your previous math classes ...

