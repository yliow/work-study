%-*-latex-*-
\section{Difference method}

Suppose you have this recurrence
\[
a_n = f(n) + a_{n-1}
\]
Then you can do this:
\[
a_n - a_{n-1} = f(n) 
\]
Therefore
\begin{align*}
a_n - a_{n-1} &= f(n) \\ 
a_{n-1} - a_{n-2} &= f(n-1) \\
\vdots & \vdots \\
a_1 - a_{0} &= f(1) 
\end{align*}
The above assumes the recurrence is for $n > 1$.
Adding the equations, we get
\[
a_n - a_0 = f(n) + \cdots + f(1)
\]
and hence
\[
a_n = f(n) + \cdots + f(1) + a_0
\]

You will frequently see the following notation if $a_n$ is a function of
$n$:
\[
\Delta a_n = a_{n+1} - a_n
\]
If you compare against the definition of the derivative of a function $f(x)$:
\[
f'(x) = \lim_{h \rightarrow 0} \frac{f(x+h) - f(x)}{h}
\]
you'll see where $\Delta a_n$ comes from.
The expression in the limit is
\[
\frac{f(x+h) - f(x)}{h}
\]
This is sometimes called the \defone{difference quotient}.
If you look at the difference quotient for $h = 1$,
you get
\[
\frac{f(x+1) - f(x)}{1} = f(x+1) - f(x)
\]

\newpage
\begin{ex}
  Using the above method, rewrite $a_n$ in terms of $n$ if
  \[
  a_n = a_{n-1} + 3
  \]
  for $n > 0$.
  Find a closed form for $a_n$ (this should depend on $a_0$).
  \qed
\end{ex}


\newpage
\begin{ex}
  Using the above method, rewrite $a_n$ in terms of $n$ if
  \[
  a_n = a_{n-1} + n
  \]
  for $n > 0$.
  Find a closed form for $a_n$ (this should depend on $a_0$).
  \qed
\end{ex}


\newpage
\begin{ex}
  Using the above method, rewrite $a_n$ in terms of $n$ if
  \[
  a_n = a_{n-1} + 5n + 2
  \]
  for $n > 1$.
  Find a closed form for $a_n$ (this should depend on $a_0, a_1$.)
  \qed
\end{ex}

\newpage
\begin{ex}
  What if the recurrence is
\[
na_n = f(n) + (n-1)a_{n-1}
\]
for $n > 0$?  
\end{ex}

\newpage
\begin{ex}
  What if the recurrence is
\[
(n-1)a_n = f(n) + n a_{n-1}
\]
for $n > 0$?  
\end{ex}

\newpage
Now if you look at \textit{this} recurrence,
\[
a_n = 2 a_{n-1} + 1
\]
which holds, say, for $n > 0$, then
the same method above will give you a problem.
Why?
Well ... think about this ...
Consider the above technique, say, for this example:
\[
a_n = a_{n-1} + 2, \,\,\,\,\, n > 0
\]
The basic idea is that if we rewrite like this:
\[
a_n - a_{n-1} = 2
\]
and then write two of the above like this:
\begin{align*}
a_n - a_{n-1}     &= 2 \\
a_{n-1} - a_{n-1} &= 2 
\end{align*}
and add the two equations, we get this:
\[
a_n - a_{n-2} = 2 + 2 
\]
Of course the point is that the second index, i.e., $n-2$,
drop from $n-1$ by 1.
If you keep doing that, the second index will hit $0$:
\[
a_n - a_0 = 2 + 2 + \cdots + 2 
\]
So the BIG picture is that there's a useful cancellation when you add
the above two questions:
\begin{align*}
a_n - \mathso{a_{n-1}}     &= 2 \\
\mathso{a_{n-1}} - a_{n-1} &= 2 
\end{align*}
Or in (better) math notation:
\[
(a_n - \mathso{a_{n-1}}) + (\mathso{a_{n-1}} - a_{n-2}) = 2 + 2
\]
Think about the above comments very carefully ... and solve
the following using the above difference method (with a slight adjustment) ...


\newpage
\begin{ex}
  \begin{tightlist}
    \item Given he following recurrence relation for $a_n$
    \[
    a_n = 2 a_{n-1} + 1, \,\,\,\,\, n > 0
    \]
    rewrite $a_n$ as a summation that does not involve $a_i$.
    \item Find a closed form for $a_n$.
  \end{tightlist}
\qed
\end{ex}

\newpage


\begin{ex}
Suppose you have this recurrence
\[
a_n = f(n) \cdot a_{n-1}, \,\,\,\,\, n > 0
\]
Write $a_n$ as an expression involving $f(i)$ ($i = 1, 2, ..., n$)
and $a_0$.
\end{ex}

\SOLUTION

From the recurrence, we have
\[
\frac{a_n}{a_{n-1}} = f(n) 
\]
Therefore
\begin{align*}
\frac{a_n}{a_{n-1}} &= f(n) \\ 
\frac{a_{n-1}}{a_{n-2}} &= f(n-1) \\
 &\vdots \\
\frac{a_1}{a_{0}} &= f(1) 
\end{align*}
The above assumes the recurrence is for $n \geq 1$.
Multiplying the equations, we get
\begin{align*}
  \frac{a_n}{a_{n-1}}
  \frac{a_{n-1}}{a_{n-2}}
  \cdots 
  \frac{a_{2}}{a_{1}}
  \frac{a_1}{a_0} &= f(n) \cdot \cdots \cdot f(1) \\
  \THEREFORE
  \frac{a_n}{a_0}
  &=  \prod_{i=1}^n f(i) \\
  \THEREFORE
  a_n &= a_0 \prod_{i=1}^n f(i)
\end{align*}




\newpage
\section{Difference equation [SECTION BY ITSELF]}

This is an in depth study of the difference method.
What you saw above was only a quick-and-dirty introduction.

Recall that given $a_n$, we can define
\[
\Delta (a_n) = a_{n+1} - a_n
\]
You can (and should) think of $\Delta a_n$ as some sort of a
derivative.
Write $D_x$ for $\frac{d}{dx}$, then for a function $f(x)$
defined for real $x$ (on an open interval for instance):
\[
D_x (f(x)) = f'(x) = \lim_{ h \rightarrow 0 } \frac{f(x+h) - f(x)}{h}
\]

Now I'm going to define the second difference of $a_n$:
\[
\Delta^2 (a_n) = \Delta (a_{n+1}) - \Delta (a_n)
\]
I can go on (of course) so that
\[
\Delta^{k+1} (a_n)
=
\Delta^k (a_{n+1})
-
\Delta^k (a_{n})
\]
These are analogous to higher derivative of a real valued
function (if the derivatives exist), i.e.,
$\Delta^k (a_n)$ if analogous to $f^{(k)}(x)$, the $k$--th
derivative of $f(x)$.

Here's an example: Let
\[
a_n = 42
\]
Then
\[
\Delta (a_n) = a_{n+1} - a_n = 42 - 42 = 0
\]
Hmmm ... interesting ... because if
\[
a_n = c
\]
where $c$ is a constant, just like the above,
it's also very easy to see that
\[
\Delta (a_n) = 0
\]
and that's exactly what happens in calculus: if
\[
f(x) = c
\]
then
\[
f'(x) = 0
\]

Now's you turn:

\begin{ex}
  \begin{tightlist}
    \item Let $a_n = cn$. Compute $\Delta (a_n)$. Do you get $c$?
    \item Let $a_n = cn + d$. Compute $\Delta (a_n)$. Do you get $c$?
  \end{tightlist}
\qed
\end{ex}

\begin{ex}
  \begin{tightlist}
    \item Let $a_n = n^2$. Compute $\Delta (a_n)$. Do you get $2n$?
    \item Let $a_n = n^3$. Compute $\Delta (a_n)$. Do you get $3n^2$?
  \end{tightlist}
\qed
\end{ex}

Now I'm going to define the \defterm{falling factorials}:
\begin{align*}
n^{\underline 0} &= 1 \\
n^{\underline 1} &= n \\ 
n^{\underline 2} &= n(n-1) \\ 
n^{\underline 3} &= n(n-1)(n-2) \\ 
n^{\underline 4} &= n(n-1)(n-2)(n-3) \\
n^{\underline 5} &= n(n-1)(n-2)(n-3)(n-4) 
\end{align*}
In general
\[
n^{\underline {k+1}}
=
n^{\underline {k}} (n - k)
\]
for $k > 0$.
The above are falling factorial expressions in $n$.
But you can also talk about falling factorials of integers.
For instance
\[
5^{\underline{3}} = 5 \cdot 4 \cdot 3
\]
Now this is not something totally alien because you have seen this
before ... note that
\[
\binom{5}{3} = \frac{5!}{3!2!} = \frac{5 \cdot 4 \cdot 3}{3!} = \frac{5^{\underline{3}}}{3!}
\]
In general
\[
\binom{n}{k} = \frac{n^{\underline{k}}}{k!}
\]

\begin{ex}
  Write the following as polynomials, arranged so that highest order
  terms appears early (i.e. on the left).
  I'm doing the first three (the easy ones of course).
  \begin{tightlist}
    \item $n^{\underline{0}} = 1$
    \item $n^{\underline{1}} = n$
    \item $n^{\underline{2}} = n^2 - n$
    \item $n^{\underline{3}}$
    \item $n^{\underline{4}}$
    \item $n^{\underline{5}}$
    \item $n^{\underline{6}}$
    \item What is the highest degree term of $n^{\underline{k}}$?
    \item What is the degree of the second highest degree term of
    $n^{\underline{k}}$? What is the coefficient?
    \item What is the degree of the third highest degree term of
    $n^{\underline{k}}$? What is the coefficient?
    \item What is coefficient of the constant term of
    $n^{\underline{k}}$?
  \end{tightlist}
  \qed
\end{ex}

\begin{ex}
  Compute the following,
  writing the resulting expressions in
  linear combination of
  $n^{\underline{0}}$,
  $n^{\underline{1}}$,
  $n^{\underline{2}}$, ...
  instead of
  $n^{{0}}$,
  $n^{{1}}$,
  $n^{{2}}$, ...
  \begin{tightlist}
    \item Compute $\Delta (n^{\underline{0}})$
    \item Compute $\Delta (n^{\underline{1}})$
    \item Compute $\Delta (n^{\underline{2}})$
    \item Compute $\Delta (n^{\underline{3}})$
    \item Compute $\Delta (n^{\underline{4}})$
    \item Compute $\Delta (n^{\underline{5}})$
    \item Compute $\Delta (n^{\underline{6}})$
  \end{tightlist}
\end{ex}

You should be somewhat convinced that
although
\[
\Delta (n^k) \neq k n^{k-1}
\]
there's evidence that the above is true if we replace
the \textit{exponents} by \textit{falling factorial exponents} ...


\begin{thm} For $k > 0$. 
\[
\Delta ( n^{\underline{k}} ) = k n^{ \underline{k-1} }
\]
\end{thm}

\begin{ex} Let $a_n$ and $b_n$ be functions $\N \rightarrow \R$,
  $c,d$ be constants, and $k$ be a positive integer.
  Are the following true:
  \begin{tightlist}
    \item $\Delta (c a_n) = c \cdot \Delta (a_n)$.
    \item $\Delta (a_n + b_n) =  \Delta (a_n) + \Delta (b_n)$.
  \end{tightlist}
\qed
\end{ex}

\begin{ex}
  \begin{tightlist}
    
    \item Suppose 
    \[
    \Delta(a_n) = \Delta(b_n)
    \]
    Then
    \[
    a_n = b_n + c
    \]
    
    \item
    A solution to 
    \[
    \Delta (y_n) = n^{\underline{k}}
    \]
    is
    \[
    y_n = \frac{1}{k+1} n^{\underline{k+1}}
    \]
    
  \end{tightlist}
\end{ex}
  
So what's the point of all the above?
Suppose you have a recurrence that looks like this:
\[
a_n = a_{n-1} + n^2 + 3
\]
Then
\[
a_n - a_{n-1} = n^2 + 3
\]
In other words the recurrence gives rise to a difference equation
\[
\Delta (a_n) = n^2 + 3
\]
Think of $a_n$ as a variable.
Now I hope to since a function of $n$, $a_n$ that satisfies the
above difference equation
\[
\Delta (y_n) = n^2 + 3
\]
Now if I can find some $y_n$ then I think that the general solution
must be $y_n + c$ where $c$ is a constant.
$\Delta$ works nicely with falling factorials.
So I need to change $n^2 + 3$ to an expression involving
combinations of falling factorials.
I have
\begin{align*}
  n^{\underline{0}} &= 1 \\
  n^{\underline{1}} &= n \\
  n^{\underline{2}} &= n(n-1) = n^2 - n \\
\end{align*}
Therefore
\begin{align*}
  n^2 + 3
  &= (n^2 - n) + n + 3 \\
  &= n^{\underline{2}} + n + 3 \\
  &= n^{\underline{2}} + n^{\underline{1}} + 3 \\
  &= n^{\underline{2}} + n^{\underline{1}} + 3 \\
\end{align*}
Therefore
\begin{align*}
a^n
&= \frac{1}{3} n^{\underline{3}} + \frac{1}{2} n^{\underline{2}}
+ 3 n^{\underline{1}} + c \\
&= \frac{1}{3} n(n-1)(n-2) + \frac{1}{2} n(n-1)
+ 3 n + c \\
&= ...
\end{align*}
and you can finish the rest.

Get the point?

\begin{thm}
  If $a_n$ satisfies
  \[
  \Delta ( a_n ) = a_n
  \]
  then
  \[
  a_n = c 2^n
  \]
  for some constant $c$.
  \qed
\end{thm}

\begin{ex}
  Solve
  \[
  \Delta( a_n ) = c a_n
  \]
  where $c$ is a constant. (The case of $c = 1$ is above.)
  Did you look at the case when $c = -1$?
  \qed
\end{ex}
  
Suppose
\[
a_n = 5n^2 + 3n + 1
\]
Then
\begin{align*}
\Delta (a_n)
&= a_{n+1} - a_n \\
&= (5(n+1)^2 + 3(n+1) + 1) - (5n^2 + 3n + 1) \\
\end{align*}

What about higher order differences?
\begin{align*}
\Delta^2(a_n)
&= \Delta(a_{n+1}) - \Delta(a_{n}) \\
&= (a_{n+2} - a_{n+1}) +  (a_{n+1} - a_{n})\\
&= (a_{n+2} - a_{n})\\
\end{align*}

Note that
\[
\Delta( \Delta (a_n) )
= \Delta( a_{n+1} - a_n )
= (a_{n+2} - a_{n+1}) - (a_{n+1} - a_n)
= a_{n+2} - a_n
\]
Therefore this is an easy exercise:
\begin{thm} Let $k \geq 0$.
  \begin{tightlist}
    \item $\Delta^k(a_n) = a_{n+k} - a_n$
    \item $\Delta^k(c a_n) = c \cdot \Delta^k( a_n )$
    \item $\Delta^k(a_n + b_n) = \Delta^k( a_n ) + \Delta^k( b_n )$
  \end{tightlist}
  \qed
\end{thm}

Also,
\[
\Delta^2( n^{\underline{k}} )
= \Delta( \Delta( n^{\underline{k}} ) )
= \Delta( k n^{\underline{k - 1}} )
= k \Delta( n^{\underline{k - 1}} )
= k (k-1) n^{\underline{k - 2}}
\]
when $k \geq 2$.
And you see quickly that

\begin{thm} Let $k \geq \ell \geq 0$. Then
\[
\Delta^{\ell} ( n^{\underline{k}} )
= k (k-1) (k-2) (k - \ell + 1) n^{ \underline{k - \ell} }
\]
\end{thm}

\begin{ex}
  For each of the following,
  find a closed form for $a_n$ given following information about
  $a_n$.
  \begin{tightlist}

    \item 
    $a_n = a_{n-2} + cn + d$, $n \geq 2$

    \item 
    $a_n = a_{n-3} + bn^2 + cn + d$, $n \geq 3$

    \item
    $\Delta^2 (a_n) = a_n$

    \item When you're done with the previous problem, try this:
    $\Delta^k (a_n) = a_n$ for $k \geq 1$.
    
  \end{tightlist}
\end{ex}

\SOLUTION

3. Recall that for the case of $\Delta (a_n) = a_n$,
\begin{align*}
  a_{n+1} - a_n &= a_n \\
  \THEREFORE a_{n+1} - 2 a_n &= 0 \\
  \THEREFORE \frac{1}{2^{n+}} a_{n+1} - \frac{1}{2^n} a_n &= 0 \\
  \THEREFORE \Delta \left( \frac{1}{2^n} a_n \right) &= 0\\
  \THEREFORE \frac{1}{2^n} a_n &= c \text{ (where $c$ is a constant)} \\
  \THEREFORE a_n &= c 2^n
\end{align*}
What about the above case $\Delta^2 ( a_n ) = a_n$?
\begin{align*}
  a_{n+2} - a_n &= a_n \\
  \THEREFORE a_{n+2} - 2 a_n &= 0 \\
  \THEREFORE \frac{1}{2^{n+1}} a_{n+2} - \frac{1}{2^n} a_n &= 0 \\
\end{align*}

