%-*-latex-*-
\section{Recursion with Floors}

Recall the runtime of the mergesort is of the form
\[
T(n) 
= 
\begin{cases}
A                        & \text{if $n = 0, 1$} \\
T(\floor{n/2}) + T(\ceiling{n/2}) + Bn + C & \text{otherwise}
\end{cases}
\]
Recall that in order to compute the runtime, I simplified the recurrence
to
\[
T(n) = 2T(n/2) + Bn + C
\]
which is the same as the original recurrence \textit{if} $n$ is a power of $2$.
Using $n = 2^m$, I showed that 
\[
T(n) = An + Bn \log_2(n) + C(n-1)
\]
and therefore $T(n) = O(n\log_2 n)$ (assuming $B \neq 0)$.

As I mentioned earlier, the above argument is not quite right
since I assumed $n = 2^m$.
In this section I want to analyze the above recurrence
relation for \textit{all} positive integers, not just powers of $2$.

Now, in order to make the computations digestible, 
I'm just going to focus on floors.
(I'll show you how to handle ceilings later.)
So for this section, I will show that the above big-O is in fact true but
I'm going to use this function:
\[
T(n) 
= 
\begin{cases}
A                        & \text{if $n = 0, 1$} \\
2T(\floor{n/2}) + Bn + C & \text{otherwise}
\end{cases}
\tag{1}
\]
For convenience, let me define
\[
g(n) = An + Bn\log_2(n) + C(n - 1)
\]

Let me plot $T(n)$ and $g(n)$ for the case of
\[
A = B = C = 1
\]

Here's the plot of $T(n)$ (blue) and $g(n)$ (red):
\begin{python}
from latextool_basic import *
N = 33
STEP = 1
table = {}
def t(n):
    n = int(n)
    if n <= 1: return 1
    else:
        if not table.has_key(n):
            table[n] = 2 * t(n/2) + n + 1
        return table[n]

p = FunctionPlot(domain="0:%s" % N, vars=['x'])
points = [(n, t(n)) for n in range(0, N + 1, STEP)]
p.add(points, style='step', color='blue', python=1)
p.add("x + x * log(x, 2) + (x - 1)", color='red', python=1, pin='above left',
pin_message='n + n \log_2 n + (n - 1)')
print(p)
\end{python}

By the way, you notice that $T(n)$ has some sharp jumps.
For instance there is a sharp jump at $n = 16$
for $T(n)$.
I'll let you think about that.

Here's a zoom in around $n = 16$:
\begin{python}
from latextool_basic import *
N = 18
STEP = 1
table = {}
def t(n):
    n = int(n)
    if n <= 1: return 1
    else:
        if not table.has_key(n):
            table[n] = 2 * t(n/2) + n + 1
        return table[n]

p = FunctionPlot(domain="14:%s" % N, vars=['x'])
points = [(n, t(n)) for n in [14, 15, 16, 17, 18]]
p.add(points, style='step', color='blue', python=1)
p.add("x + x * log(x, 2) + (x - 1)", color='red', python=1)
print(p)
\end{python}

\begin{ex}
Well ... did you think about it?
Compute by hand $T(15), T(16), T(17)$.
Do you see why there must be a sharp jump at $16$?
If not, compute $T(31), T(32), T(33)$ and think about it.
\qed
\end{ex}

We see that 
\[
T(n) \leq g(n)
\]

Let's get back to our definition of $T(n)$:
\[
T(n) 
= 
\begin{cases}
A                        & \text{if $n = 0, 1$} \\
2T(\floor{n/2}) + Bn + C & \text{otherwise}
\end{cases}
\tag{1}
\]

Now given a value of $n$, say $n = 42$, how would you
compute $T(42)$?
Of course you keep using the recurrence relation
so that $T(42)$ is in terms of $T(21)$ (and other stuff),
and you replace $T(21)$ by $T(10)$ (and other stuff), etc.
until you reach $T(1)$.

\begin{ex}
Compute $T(42)$ by hand.
There's always this rhythm: 
\lq\lq use recurrence--tidy up'', 
\lq\lq use recurrence--tidy up'', 
\lq\lq use recurrence--tidy up'', 
...
\qed
\end{ex}

For the general case, of course you can't work with a specific value of $n$.
You have to work with a general integer $n$.
But you pretty much to do the same thing, 
using recurrence until you reach $T(1)$.
If I want to get rid of the floor, I might have do this:
\begin{align*}
T(2k) &= 2T(k) + B(2k) + C \\
T(2k + 1) &= 2T(k) + B(2k + 1) + C 
\end{align*}
i.e., consider the case when $n$ is even and of the form $2k$
and of the form $n = 2k$
and when $n$ is odd and of the form $n = 2k + 1$.
Hmmm ... I would have to work with two cases. 
Then I would have to worry about two cases for $k$, one for the
case when $k$ is even and another for the case when $k$ is odd.
Too much work ... because this would make the argument way too 
untidy.

I'm going to use binary numbers to simplify the computations.
You'll see that in action in a bit.
Later, I'll show you that in fact using binary numbers 
when there are floors of the type $\floor{n/2}$ is redundant.


If $n$ is written as a binary number:
\[
n = (c_d c_{d-1} \cdots c_1 c_0)_2 = c_d2^d + \cdots + c_02^0
\] 
The lowest order bit $c_0$ is 0 when $n$ is even,
and when $n$ is $2k + 1$ is the lowest
order bit $c_0$ is $1$.
And when you compute the floor of either $2k$ or $2k + 1$ to $k$,
in terms of binary representation of $n$, this just mean
removing the lowest order bit, i.e.,
\[
\floor{(c_d \cdots c_0)_2/2} 
= \floor{c_d2^{d-1} + \cdots + c_1 + \frac{c_0}{2}}
= (c_d \cdots c_1)_2
\]

\begin{ex} The following verifies that $\floor{n/2}$ is the same as the 
binary representation of $n$ with the lowest order bit removed.
\begin{itemize}
\li Compute $\floor{42/2}$. Compute the binary representation of $42$ and then the integer with the lowest order bit removed.
\li Compute $\floor{21/2}$. Compute the binary representation of $21$ and then the integer with the lowest order bit removed.
\li Compute $\floor{10/2}$. Compute the binary representation of $10$ and then the integer with the lowest order bit removed.
\li Compute $\floor{5/2}$. Compute the binary representation of $5$ and then the integer with the lowest order bit removed.
\li Compute $\floor{2/2}$. Compute the binary representation of $2$ and then the integer with the lowest order bit removed.
\qed
\end{itemize}

\end{ex}

That's a more uniform way of describing the recurrence.
To simplify, I'll just write 
$c_d \cdots c_0$ 
instead of 
$(c_d \cdots c_0)_2$. 
Let's try that out and see if it works ...

Writing $n$ as a binary number
\[
n = c_d\cdots c_0 = c_d2^d + \cdots + c_0d^0
\]
we have
\[
T(c_d \cdots c_0) 
= 2T(c_d \cdots c_1) + (c_d \cdots c_0)B + C \\
\]
Note that in this form, I don't have to worry about two cases where
$n$ is even or $n$ is odd and the floor goes away.
In other words I have a uniform description of my recurrence relation
and without floors.
Therefore I can say the following for \textit{any} $n = c_d \cdots c_0$
(of course $c_d = 1$):
\begin{align*}
&T(c_d \cdots c_0) \\
&= 2T(c_d \cdots c_1) + (c_d \cdots c_0)B + C \\
&= 2[2T(c_d \cdots c_2) + (c_d \cdots c_1)B + C] + (c_d \cdots c_0)B + C \\
&= 2^2T(c_d \cdots c_2) + [2(c_d \cdots c_1) + (c_d \cdots c_0)]B + (2+1)C \\
&= 2^2[2T(c_d \cdots c_3) + (c_d \cdots c_3)B + C] + [2(c_d \cdots c_1) + (c_d \cdots c_0)]B + (2+1)C \\
&= 2^3T(c_d \cdots c_3) 
   + [2^2 (c_d \cdots c_2) + 2(c_d \cdots c_1) + (c_d \cdots c_0)]B 
   + (2^2 + 2 + 1)C \\
&=... \\
&= 2^dT(c_d) 
   + [2^{d-1}(c_d c_{d-1}) + \cdots + 2(c_d \cdots c_1) + (c_d \cdots c_0)]B 
   + (2^{d-1} + \cdots + 2 + 1)C \\
&= 2^dT(1) 
   + [2^{d-1}(c_d c_{d-1}) + \cdots + 2(c_d \cdots c_1) + (c_d \cdots c_0)]B 
   + (2^d - 1)C 
\\
&= 2^dA
   + [2^{d-1}(c_d c_{d-1}) + \cdots + 2(c_d \cdots c_1) + (c_d \cdots c_0)]B 
   + (2^d - 1)C 
\tag{$*$}\\
\end{align*}
Now let's try to prove that 
\[
T(n) \leq nA + (n \log_2 n)B + (n - 1)C
\]
Of course this means reverting back to $n$ instead of the 
binary bits of $n$.
The right-hand side of $(*)$ contains information derived from $n$,
i.e., $d$, $c_0$, ..., $c_d$.
The relationship between $n$ and all these stuff is
\[
n = c_d \cdots c_0 = c_d 2^d + \cdots + c_0 2^0
\]
The \lq\lq $2^d$'' thingies (that appear twice) in $(*)$ seems simpler ... so let's do that first.

Now clearly we have, since $0 \leq c_i \leq 1$,
\begin{align*}
n = c_d2^d + \cdots + c_02^0 
\geq c_d2^d = 2^d
\end{align*}
I put
\[
2^d \leq n
\]
into ($*$) to get
\begin{align*}
T(n) 
&= 2^dA
   + [2^{d-1}(c_d c_{d-1}) + \cdots + 2(c_d \cdots c_1) + (c_d \cdots c_0)]B 
   + (2^d - 1)C
\\
&\leq nA 
   + [2^{d-1}(c_d c_{d-1}) + \cdots 
   + 2^1(c_d \cdots c_1) + 2^0(c_d \cdots c_0)]B 
   + (n - 1)C 
\end{align*}
Now I suspect that the middle part of the above:
\[
2^{d-1}(c_d c_{d-1}) + \cdots + 2^1(c_d \cdots c_1) + 2^0(c_d \cdots c_0) 
\]
is somehow related to $n \log_2(n)$.
In particular, we hope that
\[
2^{d-1}(c_d c_{d-1}) + \cdots + 2^1(c_d \cdots c_1) + 2^0(c_d \cdots c_0) 
\leq n \log_2 (n)
\]

I hope you notice that
\[
2^{d-1}(c_d c_{d-1}) 
= (c_d c_{d-1} 0 \cdots 0)_2 
\leq (c_dc_{d-1} \cdots c_0)_2 
= n
\]
Likewise all the other terms on the left are actually $n = (c_d \cdots c_0)$ 
except that 
some bits are set to 0 and therefore each of these terms is $\leq n$.
There are $d$ such terms.
Hence 
\[
2^{d-1}(c_d c_{d-1}) + \cdots + 2^1(c_d \cdots c_1) + 2^0(c_d \cdots c_0) 
\leq nd
\]
OK ... now I need to get rid of that pesky $d$.
And I suspect that $d \leq \log_2(n)$.
Don't forget that $d$ is defined by 
\[
n = c_d 2^d + \cdots + c_0 2^0
\]
Also, remember that $c_d = 1$.
Therefore
\[
n = c_d 2^d + \cdots + c_0 2^0 \geq c_d 2^d = 2^d
\]
Therefore
\[
\log_2 (n) \geq d
\]
Hence
\[
2^{d-1}(c_d c_{d-1}) + \cdots + 2^1(c_d \cdots c_1) + 2^0(c_d \cdots c_0) 
\leq nd \leq n \log_2(n)
\]

Altogether we have shown that
\begin{align*}
T(n) 
&\leq nA 
   + [2^{d-1}(c_d c_{d-1}) + \cdots 
   + 2^1(c_d \cdots c_1) + 2^0(c_d \cdots c_0)]B 
   + (n - 1)C \\
&\leq An 
   + Bn \log_2(n) 
   + C(n - 1)
\end{align*}

Phew!!!

Oh, by the way, I hope you notice that besides the fact that
\[
(c_d c_{d-1} \cdots c_1) = \floor{\frac{n}{2}}
\]
(remove the lowest order bit of $n$ is the same as taking
floor after dividing by $2$)
we also have
\[
(c_d c_{d-1}\cdots c_2) = \floor{\frac{n}{2^2}}
\]
Etc.
Therefore the above expression
\[
2^{d-1}(c_d c_{d-1}) + \cdots + 2^1(c_d \cdots c_1) + 2^0(c_d \cdots c_0)
\]
is the same as 
\[
2^{d-1}\floor{ \frac{n}{2^{d-1}} } 
+ 2^{d-2}\floor{ \frac{n}{2^{d-2}} } 
+ \cdots
+ 2^1\floor{ \frac{n}{2^1} } 
+ 2^0\floor{ \frac{n}{2^0} } 
\]
Therefore the relation I derived earlier:
\[
T(c_d \cdots c_0) 
= An 
   + \left(
     2^{d-1}(c_d c_{d-1}) + \cdots + 2(c_d \cdots c_1
     \right) 
   + (c_d \cdots c_0)]B 
   + C(n-1) 
\]
can also be written as
\begin{align*}
T(n) 
&= 2^dA \\
&\hspace{0.5cm}
+ \left(
2^{d-1}\floor{ \frac{n}{2^{d-1}} } 
+ 2^{d-2}\floor{ \frac{n}{2^{d-2}} } 
+ \cdots
+ 2^1\floor{ \frac{n}{2^1} } 
+ 2^0\floor{ \frac{n}{2^0} } 
\right)B \\ 
&\hspace{0.5cm}+ (2^d - 1)C 
\end{align*}
where $d$ is defined by $n = c_d 2^d + \cdots + c_02^0$ with $c_d = 1$, i.e.,
$d + 1$ is the length of $n$ when written as a binary number.
If you like, you can also define $d$ without binary numbers:
$d$ is the positive integer such that 
\[
\floor{
\frac{n}{2^d}
} = 1
\]
Also,
\[
2^d \leq n \leq 2^d + \cdots 2^0 = 2^{d+1} - 1
\]
we have
\[
d \leq \log_2(n) \leq \log_2(2^{d+1} - 1) < d+1
\]
And therefore
\[
d = \floor{\log_2(n)} 
\]
Let me collect that up for you:

\begin{thm}
If $n = (c_d \cdots c_0)_2 \geq 1$, i.e., $n$ has $d + 1$ binary digits,
then $d$ can be defined as follows:
\begin{itemize} 
\item[(a)] $d$ can be defined as the positive integer such that
\[
\floor{\frac{n}{2^d}} = 1
\]
\item[(b)] $d = \floor{\log_2(n)}$
\end{itemize}
\end{thm}

Anyway,
\begin{align*}
T(n) 
&= 2^dA \\
&\hspace{0.5cm}
+ \left(
2^{d-1}\floor{ \frac{n}{2^{d-1}} } 
+ 2^{d-2}\floor{ \frac{n}{2^{d-2}} } 
+ \cdots
+ 2^1\floor{ \frac{n}{2^1} } 
+ 2^0\floor{ \frac{n}{2^0} } 
\right)B \\ 
&\hspace{0.5cm}+ (2^d - 1)C  
\\
&\leq An + Bn \log_2(n) + C(n - 1) 
\end{align*}
A more compact way of writing the above equality computation is
\begin{align*}
T(n) 
&= A 2^{\floor{\log_2(n)}}
 + B \sum_{i=0}^{\floor{\log_2(n)} - 1} 2^i\floor{ \frac{n}{2^i} } 
 + C \left(2^{\floor{\log_2(n)}} - 1\right)
\end{align*}

\begin{thm}
Let $T(n)$ be defined by
\[
T(n) =
\begin{cases}
A                        &\text{if $n = 0, 1$} \\
2T(\floor{n/2}) + Bn + C &\text{otherwise}
\end{cases}
\]
(actually I will only use $T(1)$).
Then
\[
T(n) 
= A 2^{\floor{\log_2(n)}}
 + B \sum_{i=0}^{\floor{\log_2(n)} - 1} 2^i\floor{ \frac{n}{2^i} } 
 + C \left(2^{\floor{\log_2(n)}} - 1\right)
\]
\end{thm}

The above are \textit{exact equalities} of course.
For comparison we have the following:

\begin{thm} Let $d + 1$ be binary length of $n > 0$.
\begin{itemize}
\item[(a)] $d \leq \log_2(n)$
\item[(b)] $2^i\floor{ \frac{n}{2^i} } \leq n$ for $i = 0, 1, 2, \ldots, d$
\item[(c)] $\sum_{i=0}^{\floor{\log_2(n)} - 1} 2^i\floor{ \frac{n}{2^i} } \leq n \log_2(n)$
\end{itemize}
\end{thm}

By the way, if $B \neq 0$, of course the big-O of $T(n)$ 
above is $n \log_2(n)$.
However if $B = 0$, then
the determining factor is $2^d$ and $2^d - 1$.
In that case, I might want to be more accurate than just to approximate
\[
2^d
\]
with
\[
2^d \leq n
\]
I can do the following:
From
\[
d = \floor{\log_2(n)}
\]
In that case (i.e., if $B = 0$), 
\[
T(n) = A2^{\floor{\log_2(n)}} + C(2^{\floor{\log_2(n)}} - 1)
= (A + C) 2^{\floor{\log_2(n)}} - C
\]

An important idea used above is that if you have something like this:
\[
T(n) = f(n) T(\floor{n/2}) + g(n)
\]
then writing $n = c_d2^d + \cdots c_02^0 = (c_d \cdots c_0)$, we have
\[
T(c_d\cdots c_0) = f(c_d \cdots c_0) T(c_d \cdots c_1) + g(c_d \cdots c_0)
\]


Now I want to rederive the closed form
of $T(n)$:
\begin{align*}
T(n) 
&= 2^dA \\
&\hspace{0.5cm}
+ \left(
2^{d-1}\floor{ \frac{n}{2^{d-1}} } 
+ 2^{d-2}\floor{ \frac{n}{2^{d-2}} } 
+ \cdots
+ 2^1\floor{ \frac{n}{2^1} } 
+ 2^0\floor{ \frac{n}{2^0} } 
\right)B \\ 
&\hspace{0.5cm}+ (2^d - 1)C  
\\
&\leq An + Bn \log_2(n) + C(n - 1) 
\end{align*}
without using binary numbers.
This means working directly with floors without the binary numbers.
The two methods are the same.
\begin{align*}
T(n) 
&= 2T(\floor{n/2}) + Bn + C \\
&= 2(2T(\floor{\floor{n/2}/2}) + B\floor{n/2} + C) + Bn + C \\
&= 2^2 T(\floor{\floor{n/2}/2}) + (2\floor{n/2} + n)B + (2 + 1)C
\end{align*}
Now I use the crucial formula
\[
\floor{\floor{a/b}/c} = \floor{a/(bc)}
\]
to get
\[
T(n) 
= 2^2 T \left(
  \floor{n/2^2} \right) + 
  \left( 2\floor{n/2} + n \right)B + (2 + 1)C
\]
Now I continue with another use of the recurrence relation:
\begin{align*}
T(n) 
&= 2^2 \left(
         2T \left( 
              \floor{ 
                \floor{n/{2^2}} 
                / 2
              } 
           \right) 
         + 
        B \floor{n/2^2} 
         + C
       \right) 
   + \left(2 
         \floor{
          n/2
         } 
         + n\right)B 
   + (2 + 1)C \\
&= 2^2 \left(
         2T \left( 
                \floor{
                  n/2^3 
              } 
           \right) 
         + 
        B \floor{
             n/2^2
           } 
         + C
       \right) 
   + \left(2 
         \floor{
           n/2
         } 
         + n\right)B 
   + (2 + 1)C \\
&= 2^3 
      T \left( 
              \floor{
               n/2^3 
              } 
        \right)  
   + \left(
             2^2\floor{n/2^2}
             + 2 
         \floor{
          n/2
         } 
         + n
      \right)B 
   + (2^2 + 2 + 1)C \\
\end{align*}
And I again apply the recurrence on  
\[
      T \left( 
              \floor{
               n/2^3 
              } 
        \right)  
\]
to get something with
\[
      T \left( 
              \floor{
               n/2^4 
              } 
        \right)  
\]
And when do I stop?
When I reach $T(1)$.
Now I define $d$ to be the positive integer such that 
\[
\floor{\frac{n}{2^d}} = 1
\]
Basically $d$ is the number of times I use the recurrence relation
to move toward $T(1) = A$.
\begin{align*}
T(n) 
&= 2^3 
      T \left( 
              \floor{
               \frac{n}{2^3} 
              } 
        \right)  
   + \left(
             2^2\floor{\frac{n}{2^2}}
             + 2 
         \floor{
          \frac{n}{2}
         } 
         + n
      \right)B 
   + (2^2 + 2 + 1)C \\
&= ... \\
&= 2^d
      T \left( 
              \floor{
               \frac{n}{2^d} 
              } 
        \right) \\ 
&\hspace{0.5cm}   + \left(
             2^{d - 1} \floor{\frac{n}{2^{d-1}}}
 + \cdots + 
             2^2\floor{\frac{n}{2^2}}
             + 2 
         \floor{
          \frac{n}{2}
         } 
         + n
      \right)B \\
&\hspace{0.5cm}  
   + (2^{d-1} + \cdots + 2^2 + 2 + 1)C \\
&= 2^d A \\ 
&\hspace{0.5cm}   + \left(
             2^{d - 1} \floor{\frac{n}{2^{d-1}}}
 + \cdots + 
             2^2\floor{\frac{n}{2^2}}
             + 2 
         \floor{
          \frac{n}{2}
         } 
         + n
      \right)B \\
&\hspace{0.5cm}  
   + (2^d - 1)C \\
\end{align*}
That's it.
And as you can see, you don't really need binary numbers.
However it's usually easier to see the above computation
using binary numbers first.

\begin{ex}
What would you do if you're given this recurrence instead:
\[
T(n) = 4T(\floor{n/2}) + 3n + 2
\]
[Follow the above method either using binary numbers or using
floors -- it's up to you -- and then make similar
approximations at the end.]
\qed
\end{ex}

\begin{ex}
What would you do if you're given this recurrence instead:
\[
T(n) = 3T(\floor{n/3}) + Bn + C
\]
[Use the same idea as above but write $n$ in base 3, but 2. 
After you're done
try to convert whatever you have to floors.]
\qed
\end{ex}

\begin{ex}
What would you do if you're given this recurrence instead:
\[
T(n) = 3T(\floor{n/2}) + Bn + C
\]
\qed
\end{ex}

\begin{ex}
What would you do if you're given this recurrence instead:
\[
T(n) = 2T(\floor{n/3}) + Bn + C
\]
\qed
\end{ex}

\begin{ex}
What would you do if you're given this recurrence instead:
\[
T(n) = 2T(\floor{n/3}) + Bn^2 + C
\]
\qed
\end{ex}

\begin{ex}
What would you do if you're given this recurrence instead:
\[
T(n) = 2T(\floor{n/3}) + B\sqrt{n} + C
\]
\qed
\end{ex}

\begin{ex}
What would you do if you're given this recurrence instead:
\[
T(n) = 2T(\floor{n/3}) + B\log_2(n) + C
\]
\qed
\end{ex}

\begin{ex}
What would you do if you're given this recurrence instead:
\[
T(n) = 2T(\floor{n/3}) + Bn\log_2(n) + C
\]
\qed
\end{ex}

\begin{ex}\mbox{}
\begin{itemize}
\li Let $T(n)$ satisfy
\[
T(n) =
\begin{cases}
A                          & \text{if $n = 0, 1$} \\
f(n) T(\floor{n/2}) + g(n) & \text{otherwise} 
\end{cases}
\]
Derive an exact expression for $T(n)$.
\li Note that the case of $f(n) = 2$ and $g(n) = Bn + C$,
derive a closed form for $T(n)$.
You should get the same expression I derived in this section.
\end{itemize}
\end{ex}

\textit{Solution.}
\begin{align*}
T(n) 
&= f(n) T(\floor{n/2}) + g(n) \\
&= f(n) (f(\floor{n/2}) T(\floor{n/2^2}) + g(\floor{n/2})) + g(n) \\
&= f(n)f(\floor{n/2}) T(\floor{n/2^2}) + f(n)g(\floor{n/2}) + g(n) \\
&= f(n)f(\floor{n/2}) (f(\floor{n/2^2})T(\floor{n/2^3}) + g(\floor{n/2^2}))+ f(n)g(\floor{n/2}) + g(n) \\
&= f(n)f(\floor{n/2})f(\floor{n/2^2}T(\floor{n/2^3}) + f(n)f(\floor{n/2})g(\floor{n/2^2}))+ f(n)g(\floor{n/2}) + g(n) \\
\end{align*}
Hence, if $\floor{n/2^d} = 1$, i.e., $d = \floor{\log_2(n)}$, then
\begin{align*}
T(n) 
&=
   \left( 
       f(n)f(\floor{n/2})\cdots f(\floor{n/2^{d-1}} 
   \right)
   T(\floor{n/2^d}) \\
&\hspace{0.5cm} + 
   \left(
       f(n)f(\floor{n/2}) \cdots f(\floor{n/2^{d-2}}) g(\floor{n/2^{d-1}}))
         + \cdots 
         + f(n)g(\floor{n/2}) + g(n)
    \right)
\\
&= 
   \left( 
       \prod_{i=0}^{d-1} f(\floor{n/2^{i}} 
   \right)
   A \\
&\hspace{0.5cm} + 
   \left(
       \prod_{i=0}^{d-2} f(\floor{n/2^{i}})  
   \right)
   g(\floor{n/2^{d-1}})
   + \cdots + 
   \left(\prod_{i=0}^{0} f(\floor{n/2^i}) \right) g(\floor{n/2^1}) \\
&\hspace{0.5cm} + g(n)
\end{align*}
For simplicity, let me define
\[
F(j) = \prod_{i=0}^{j} f(\floor{n/2^{i}})
\]
The above becomes
\begin{align*}
T(n) 
&= F(d-1) A + 
   F(d-2) g(\floor{n/2^{d-1}})
   + \cdots + 
   F(0) g(\floor{n/2^1}) + g(n) \\
&= F(d-1) A +
   \sum_{j = 1}^{d-1} F(j-1)g(\floor{n/2^{j}})
   + g(n) 
\end{align*}
For the case when $f(n) = 2$
\[
F(j) = \prod_{i=0}^j 2 = 2^{j+1}
\]
the above becomes
\begin{align*}
T(n) 
&= 2^{d-1 + 1} A +
   \sum_{j = 1}^{d-1} 2^{j-1 + 1}g(\floor{n/2^{j}})
   + g(n) 
\\
&= 2^d A +
   \sum_{j=1}^{d-1} 2^{j}g(\floor{n/2^{j}})
   + g(n) 
\end{align*}
And, in addition, when $g(n) = Bn + C$, the above becomes:
\begin{align*}
T(n) 
&= 2^{d} A +
   \sum_{j=1}^{d-1} 2^{j}( B \floor{n/2^{j}} + C)
   + (Bn + C)
\\
&= 2^{d} A +
   B\sum_{j=1}^{d-1} 2^{j}\floor{n/2^{j}} + C\sum_{j=1}^{d-1} 2^j
   + (Bn + C)
\\
&= 2^{d} A +
   B\sum_{j = 1}^{d-1} 2^{j}\floor{n/2^{j}} + C\sum_{j=1}^{d-1} 2^j
   + (B 2^0 \floor{n/2^0} + C2^0)
\\
&= 2^{d} A +
   B\sum_{j = 0}^{d-1} 2^{j}\floor{n/2^{j}} + C\sum_{j=0}^{d-1} 2^j
\\
&= 2^{d} A +
   B\sum_{j = 0}^{d-1} 2^{j}\floor{n/2^{j}} + C(2^{d} - 1)
\end{align*}
\qed

\begin{ex}
(a) Let
\[
T(n) =
\begin{cases}
A                          & \text{if $n = 0, 1$} \\
f(n) T(\floor{n/2}) + g(n) & \text{otherwise} 
\end{cases}
\]
Prove that
\begin{align*}
T(n) 
&= F(d-1) A +
   \sum_{j = 1}^{d-1} F(j-1)g(\floor{n/k^{j}})
   + g(n) 
\end{align*}
where
\[
d = \floor{\log_k (n)}
\]
and 
\[
F(j) = \prod_{i=0}^{j} f(\floor{n/k^{i}})
\]

(b) Let $f(n) = B$ (constant).
Simplify $T(n)$.
\end{ex}

\begin{align*}
F(j) 
= \prod_{i=0}^{j} f(\floor{n/k^{i}}) = \prod_{i=0}^j B
= B^{j + 1}
\end{align*}
Then
\begin{align*}
T(n) 
&= F(d-1) A +
   \sum_{j = 1}^{d-1} F(j-1)g(\floor{n/k^{j}})
   + g(n) 
\\
&= B^d A +
   \sum_{j = 1}^{d-1} B^j g(\floor{n/k^{j}})
   + g(n) 
\\
&= B^{\floor{\log_k(n)}} A +
   \sum_{j = 1}^{\floor{\log_k(n)}-1} B^j g(\floor{n/k^{j}})
   + g(n) 
\\
&= B^{\floor{\log_k(n)}} A +
   \sum_{j = 0}^{\floor{\log_k(n)}-1} B^j g(\floor{n/k^{j}})
\\
&= B^{\floor{\log_k(n)}} A +
   \sum_{j = 0}^{\floor{\log_k(n)}-1} 
   \left( 
     \frac{B}{k}
   \right)^j 
   \cdot k^j g(\floor{n/k^{j}})
\end{align*}

