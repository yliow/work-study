\tinysidebar{\debug{exercises/{factorial/question.tex}}}
(TODO: Move to CISS358.)
Let's compare $n!$ and $n^n$ further.
From the above
\[
f(n) = O(n!) \implies f(n) = O(n^n) 
\]
Is it possible that
\[
f(n) = O(n!) \impliedby f(n) = O(n^n)
\]
(Hint:
Stirling's formula says that $n!$ is \lq\lq very close" to
$\sqrt{2\pi n} (n/e)^n$ in the sense that
\[
n! \sim \sqrt{2\pi n} \left( \frac{n}{e} \right)^n
\]
i.e.,
\[
\lim_{n \rightarrow \infty}
\frac{n!}{ \sqrt{2\pi n} \left( \frac{n}{e} \right)^n} = 1
\]
Two function $f(n)$ and $g(n)$ are \defone{asymptotically equivalent} if
\[
\lim_{n \rightarrow \infty} \frac{f(n)}{g(n)} = 1
\]
Of course this requires Calculus.
For more details about $\sim$, see CISS358.)
