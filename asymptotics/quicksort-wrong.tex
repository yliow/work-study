%-*-latex-*-
\section{Quicksort: A
Partition Strategy that is Reasonble, Short,
and Completely Wrong}

\textbf{PARTITION.}
There are three issues to resolve.
The first is to figure a way to not create new arrays all the time.
The other two are related to the partition procedure/strategy.
\begin{itemize}
\li How to choose the pivot. This is easy. As I've mentioned to you,
there are two standard way of choosing the pivot.
The basic quicksort picks the first value of the array.
The randomized quicksort picks a random element from the array.
Done.
\li The next is to decide where exactly to put values in the
\verb!left! of the \verb!right! chunk.
In the above long example, I form the \verb!left! chunk by 
simply scanning the given array left-to-right and put a value into
\verb!left! to the end of \verb!left!.
Same for \verb!right!.
Because I'm going to use the same array, the above
does not work.
I'll show you two ways of forming the \verb!left! and \verb!right!
pile of values using the same array.
\end{itemize}




Let's try something just to get into the swing of things.
Here's a naive way of getting an array partitioned.
Suppose we have this array:
\begin{python}
from quicksort_init import *
p = Plot()
s = chunkedarray(cellwidth=cellwidth, 
                 cellheight=cellheight,
                 arr=[[3,1,6,2,4]],
                 pipeheight=pipeheight,
)
p.add(s)
print(p)
\end{python}
and suppose the pivot I choose is \verb!2!  at index 4.
I now scan the array and put the values relative to the pivot ...
and don't forget that I want to do this inplace, i.e., work on the 
array directly and not use a separate temporary array.
Scanning left-to-right the only reasonble first value to
process is \verb!3!.
Clearly \verb!3! must be on the right of the pivot \verb!2!.
A reasonable thing to do is to swap \verb!3! and \verb!2!:
\begin{python}
from quicksort_init import *
p = Plot()
s = chunkedarray(x=0, y=0, cellwidth=cellwidth, 
                   cellheight=cellheight,
                   arr=[[3,1,6,2,4]],
                   pipeheight=pipeheight,
                   swaps=[(0,3)],
)
p.add(s)
s = chunkedarray(x=0, y=-2, cellwidth=cellwidth, 
                   cellheight=cellheight,
                   arr=[[2,1,6,3,4]],
                   pipeheight=pipeheight,
)
p.add(s)
print(p)
\end{python}




Now the index of my pivot is 0.
Next I swap \verb!1! to get to the left of the pivot:
\begin{python}
from quicksort_init import *
p = Plot()
s = chunkedarray(x=0, y=0, cellwidth=cellwidth, 
                   cellheight=cellheight,
                   arr=[[2,1,6,3,4]],
                   pipeheight=pipeheight,
                   swaps=[(0,1)],
)
p.add(s)
s = chunkedarray(x=0, y=-2, cellwidth=cellwidth, 
                   cellheight=cellheight,
                   arr=[[1,2,6,3,4]],
                   pipeheight=pipeheight,
)
p.add(s)
print(p)
\end{python}




At this point, \verb!1! and \verb!3! are on the correct side of the 
pivot \verb!2!.
The next index position to process has value \verb!6! which is already
on the correct side of the pivot so we don't do anything.
The next value to process has value \verb!3! which
was actually already processed, but if we loop over the values
left-to-right without knowing which one is processed, we would have to 
process this value anyway.
Since \verb!3! is on the correct side we don't do anything.
Same thing for \verb!4!.
We're done.




Hmmm ... it looks like the strategy works.
The end result is:
\begin{python}
from quicksort_init import *
p = Plot()
s = chunkedarray(cellwidth=cellwidth, 
                   cellheight=cellheight,
                   arr=[[1],[2],[6,3,4]],
                   pipeheight=pipeheight,
                   chunklabels=[('left'),('pivot',1.2), ('right')],
)
p.add(s)
print(p)
\end{python}




The algorithm should look like this:
\begin{console}
FUNCTION partition(x, pivot_index):
    pivot = x[pivot_index]
    for i = 0, 1, 2, ..., len(x) - 1:
        if i is not pivot_index:
            if x[i] <= pivot and i > pivot_index:
                swap x[i] and x[pivot_index]
                pivot_index = i
            elif x[i] > pivot and i < pivot_index:
                swap x[i] and x[pivot_index]
                pivot_index = i
\end{console}



\newpage
\begin{ex}
Execute the above algorithm on this array
with \verb!2! as the pivot:
\begin{python}
from quicksort_init import *
p = Plot()
s = chunkedarray(cellwidth=cellwidth, 
                   cellheight=cellheight,
                   arr=[[3,1,0,2,6,4]],
                   pipeheight=pipeheight,
)
p.add(s)
print(p)
\end{python}
Does the partition procedure partitioned the above correctly?
\qed
\end{ex}



\newpage
\begin{ex}
Execute the above algorithm and this array
with \verb!2! as the pivot:
\begin{python}
from quicksort_init import *
p = Plot()
s = chunkedarray(cellwidth=cellwidth, 
                   cellheight=cellheight,
                   arr=[[2,0,1,3,6,4]],
                   pipeheight=pipeheight,
)
p.add(s)
print(p)
\end{python}
Does the partition procedure partitioned the above correctly?
\qed
\end{ex}



\newpage
\begin{ex}
Execute the above algorithm and this array
with \verb!2! as the pivot:
\begin{python}
from quicksort_init import *
p = Plot()
s = chunkedarray(cellwidth=cellwidth, 
                   cellheight=cellheight,
                   arr=[[3,1,6,0,2,4]],
                   pipeheight=pipeheight,
)
p.add(s)
print(p)
\end{python}
Does the partition procedure really partitioned the above.
\qed
\end{ex}


\newpage
Now you see that our greedy and short-sighted approach 
fails.
In the next few sections, I'll show you different strategies that
actually work.
The key idea is to \textit{temporarily move the pivot out of the way}.
