%-*-latex-*-
\section{Average Time Using Probability Theory}

Now I'm going to rephrase our computations of the various
average times in terms of probability theory.
Of course you'll get the same results.
So why do we bother?
Because by doing so we can bring in facts/formulas/techniques/big guns
from probability theory without having to re-invent the wheel.

Recall that for out linear search, we looked at this average time scenario:
\begin{itemize}
\li \verb!target! is at index 0, 1, 2, ..., n - 1 in array \verb!x!
\li \verb!target! is not in array \verb!x!
\end{itemize}
When phrased in terms of probability theory the above are called
events.
For instance \lq\lq \verb!target! is at index 2'' is an event.
However I want to simplify this and just call it \lq\lq 2''.
I'm going to call the event \lq\lq \verb!target! is not in the array''
the event \lq\lq $n$''.
So here's the set of all possible events
\[
S = \{0, 1, 2, ..., n\}
\]
The set of all events of a particular \textit{experiment} is called
the sample space of the experiment.
Of course in this case the \textit{experiment} is the act of finding
\verb!target! in \verb!x!.

Now let me add our \lq\lq frequency'' assumption:
\begin{itemize}
\li \verb!target! is at index 0, 1, 2, ..., n - 1 in array \verb!x!
\li \verb!target! is not in array \verb!x!
\li All the above cases are equally likely.
\end{itemize}

In terms of probability theory, this is the same as assignment a measure
of likelyhood to all events such that they add up to 1 (i.e. to 100\%).
I'm going to using a function for this assignment of measure of likelihood:
\[
p: S \rightarrow \R
\]
and of course since the events in $S$ are equally likely and there are 
$n + 1$
events, the function $p$ is given by
\[
p(x) = \frac{1}{n + 1}
\]

Next I will attach a quantity to each event that I want to count or measure.
In our case, I attach to each event the runtime.
For instance for the event \lq\lq \verb!target! is at index 2''
I attach the quantity $T_2$.
This assignment is called a random variable.
I'm going to use $X$ for this assignment.
So here's the assignment:
\[
X: S \rightarrow \{T_0, T_1, ..., T_n\}
\]
where
\[
X(i) = T_i
\]

Now everything is set up.

So what's our average time? Or in probability theory we would say
what is the \lq\lq expected'' time or 
the \lq\lq expectation of $X$''?
It's written $E[X]$ and the formula is what I have already shown you:
\[
E[X] = \frac{1}{n + 1} (T_0 + T_1 + \cdots T_n)
\]
However I want to rewrite it in a different form:
\begin{align*}
E[X] 
&= \frac{1}{n + 1} (T_0 + T_1 + \cdots + T_n) \\
&= \frac{1}{n + 1} T_0 + \frac{1}{n+1}T_1 + \cdots + \frac{1}{n + 1}T_n \\
&= p(0) X(0) + p(1) X(1) + \cdots + p(n) X(n) \\
&= \sum_{x \in S} p(x) X(x)
\end{align*}

Here's a summary of how to setup quantities in order to use probabily
theory to compute average values:
\begin{itemize}
\li State your experiment. In our case: \lq\lq
Using the linear search algorithm, 
find the index where \verb!target! occurs in an array \verb!x!
of size \verb!n!. If it's not found set index to -1.''
\li State the sample space, i.e., all the events.
In our case: 
\lq\lq
The index is 0, 1, 2, ..., \verb!n! - 1 if \verb!target!
is in \verb!x! or the index is -1 when \verb!target! is not in \verb!x!.
To simplify notation we call these events 0, 1, 2, $n-1$, $n$.
Let $S = \{0, 1, 2, ..., n\}$.
\li State the probability function.
In our case (because we assume the events are equally likely):
\[
p: S \rightarrow \R
\]
where 
\[
p(x) = \frac{1}{n + 1}
\]
\li State the random variable.
In our case: \lq\lq
\[
X : S \rightarrow \R
\]
where
\[
X(i) = A + Bn, \,\,\,\,\, i = 0, 1, 2, ..., n - 1
\]
and 
\[
X(n) = C + Dn
\]
\li Compute the average or the expectation of $X$ using
\[
E[X] = \sum_{x \in S} p(x) X(x)
\]
\end{itemize}
