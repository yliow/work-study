%-*-latex-*-
\section{Another Way to Look at big-O}

Recall that if I say \lq\lq $f = O(g)$'' I mean
you can find $C$ and $N$ such that
for $n \geq N$,
\[
|f(n)| \leq C \cdot |g(n)|
\]

In the above, $O(g)$ is purely symbolic.
Like I said, when you see \lq\lq $f = O(g)$'',
you just think of
\lq\lq there exist $C$ and $N$ such that
for $n \geq N$,
$|f(n)| \leq C \cdot |g(n)|$''.

Now I'm going to define $O(g)$ as a \textit{set} like this:
\[
O(g) = 
\biggl\{
f \,\biggl|\,
\text{there is some $N$ and $C$ such that $|f(n)| \leq C\cdot |g(n)|$} 
\biggr\}
\]

With this definition, I can say
\[
f \in O(g)
\]
This would be the say as saying
\[
f = O(g)
\]
in our previous definition.
Whether you write \lq\lq $f = O(g)$ or
\lq\lq $f \in O(g)$", they mean the same thing, i.e.,
\[
\text{there are $C$ and $N$ such that for $n \geq N$,
$|f(n)| \leq C \cdot |g(n)|$}
\]

Besides seeing statements like
\[
f(n) \in O(g(n))
\]
or 
\[
f(n) = O(g(n))
\]
you might see something like this:
\[
   f(n) = 3n^2 + 4n + 5 + 6 \ln n = 3n^2 + O(n)
\]
or
\[
   f(n) = 3n^2 + 4n + 5 + 6 \ln n \in 3n^2 + O(n)
\]
What does this mean:
\[
3n^2 + O(n)
\]
Seems like the author is trying to add a
function $3n^2$ with a set $O(n)$!!!

Basically the author is trying
to say that he cares only about $3n^2$ and the \textit{smaller terms} are
bunched up as \textit{some} function in $O(n)$, i.e.,
\[
   f(n) = 3n^2 + 4n + 5 + 6 \ln n = 3n^2 + f_1(n) 
   \text{ for some $f_1 \in O(n)$}
\]

If he wants to be more
accurate he can also say
\[
  f(n) = 3n^2 + 4n + 5 + 6 \ln n = 3n^2 + 4n + O(\ln n)
\]
he meant that $f(n)$ is $3n^2 + 4n$ together with a smaller function
$f_1(n)$ which is in $O(\ln n)$:
\[
  f(n) = 3n^2 + 4n + 5 + 6 \ln n = 3n^2 + 4n + f_1(n) 
  \text{ for some $f_1(n) \in O(\ln n)$}
\]

[PUT STUFF BELOW IN ANOTHER SECTION]

In general let's look at an expression of the form
\[
  f_1(n) + O(g_1(n)) = f_2(n) + O(g_2(n))
\]
First of all let's define $f_1(n) + O(g_1(n))$. This is the sum of
a function and a set. Basically you just add $f_1$ to all the
functions in $O(g_1(n))$. More generally if you have a function
$f$ and a set of functions $S$, then we define
\[
 f + S = \{ f + g \,|\, g \in S \}
\]

With this definition,
\[
 f_1(n) + O(g_1(n)) = f_2(n) + O(g_2(n))
\]
is the same as saying
\[
 \biggl\{ f_1 + g \,\,\biggl|\,\, g \in O(g_1(n)) \biggr\}
 \subseteq
 \biggl\{ f_2 + g \,\,\biggl|\,\, g \in O(g_2(n)) \biggr\}
\]
Remember that when it comes to $O$ and $\Theta$ equations, $=$
always means $\subseteq$!!


If you stop to think about it, not only can you define $f + S$
where $f$ is a function and $S$ is a set of functions, you can
also define $S + f$ as $\{ g + f \,|\, g \in S \}$ too. Of course
since $f+g = g + f$, we have $f + S = S + f$. Right? Furthermore,
you can also define $fS$ and $Sf$.


\begin{ex}
Let $S, T, U$ be sets of functions and $f$ be a function. Equality
here is really equality and not $\in$ or $\subseteq$.
\begin{enumerate}
 \item Define $S + T$ and prove that $S + T = T + S$ (duh) and
 $(S+T)+U = S+(T+U)$ (duh).
 Therefore we can write $S+T+U$ without fear of ambiguity or
 lawsuits or flamemail. Of course with your definition we can just
 define $f+S$ as ${f} + S$.
 \item Define $fS$ and $Sf$. Prove that $fS = fS$ (duh).
 \item Define $ST$. Prove that $ST = TS$ and $(ST)U = S(TU)$
 (two duhs). So again we can write $STU$.
\end{enumerate}
\end{ex}


\begin{ex}
Try this out yourself: What does this means
\[
 f + O(g) + O(h) = a + O(b) + O(c)
\]
where $f,g,h,a,b,c,$ are functions. Don't forget that in this case
$=$ means $\subseteq$.\end{ex}

