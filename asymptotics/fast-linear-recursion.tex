%-*-latex-*-
\sectionthree{Speeding up linear recursion}
\begin{python0}
from solutions import *; clear()
\end{python0}

Recall that if you have a runtime that is like this:
\[
T(n) 
= 
\begin{cases}
A                   & \text{if $n = 0, 1$} \\
T(n-1) + T(n-2) + B & \text{otherwise}
\end{cases}
\]
which is the case for the Fibonacci function:
\begin{Verbatim}[frame=single,fontsize=\footnotesize]
def fib(n):
    if n < 2:
        return 1
    else:
        return fib(n - 1) + fib(n - 2)
\end{Verbatim}
then the runtime is exponential:
\[
T(n) = O(2^n)
\]

When you look at the computation of the degree 2 linear recurrence
computation-by-hand in a previous section, 
you see that the problem is the repeated computations of 
\verb!fib(i)!
(\verb!i = n - 2, ...!) when you compute \verb!fib(n)!.

One way to prevent re-computation is to ... save your work!!!
So let's say we have an array \verb!table! of size 1000
to store \verb!fib(0)!, \verb!fib(1)!, ..., \verb!fib(999)!.
Initially, the table is empty.
To indicate that \verb!table[i]! is \lq\lq empty'',
I'll set it to $-1$ since the Fibonacci numbers are positive.

Then, whenever I compute \verb!fib(n)!, I first check if it's already in the 
\verb!table!.
If it is, I'll use the number.
If it's not, I'll compute \verb!fib(n)!, save it in the table,
then return it.
Now the function looks like this:
 
\begin{Verbatim}[frame=single,fontsize=\footnotesize]
for i = 0, ..., 999:
    table[i] = -1

def fib(n):
    if n < 2:
        return 1
    else:
        if table[n] is -1:
            table[n] = fib(n - 1) + fib(n - 2)
        return table[n]
\end{Verbatim}

This is not new.
You have already seen this technique in CISS245.
The concept of storing computations in a table to avoid
re-computation is extremely important and appears
in a huge number of algorithms (not just fibonacci computations).

\newpage
\begin{ex}
  Assuming your \verb!table! is large enough to store
  all computation of
  \verb!fib(0)!,
  \verb!fib(1)!,
  \verb!fib(2)!,
  \verb!fib(3)!,
  ...,
  \verb!fib(n)!,
  compute the runtime 
  of the above implementation.
\qed
\end{ex}

Of course it's possible that you want $f(10000000)$.
You can either increase the size of your \verb!table!
or simply use the \verb!table! only when \verb!n! is less than 1000.

\newpage
\begin{ex}
Modify the above problem to use the \verb!table! only when 
\verb!n! is less than 1000.
\end{ex}


\begin{comment}
\newpage
\begin{ex}
Note that we use \verb!table[5]! to store $f(5)$, i.e.,
\verb!table[0, ..., 999]! corresponds to 
$f(0), ..., f(999)$.
What if you want to store 
$f(10)$, ..., $f(1009)$ in \verb!table[0,...,999]! instead?
\end{ex}
\end{comment}

\newpage
Notice that the above method requires storage to save computations
in order to prevent re-computation.
There's another way to compute the $n$--th Fibonacci without too much 
storage space.
Here's the idea:
Instead of compute \lq\lq top-to-bottom''
(i.e., going from $n$ to $1$ and $0$)
you start at $0$ and $1$ and compute up to $n$.
Here's an execution of $f(6)$.
First of all, the Fibonacci numbers up to the $6$--th is this:
\begin{Verbatim}[frame=single, fontsize=\footnotesize]
1, 1, 2, 3, 5, 8, 13
\end{Verbatim}
You need only 3 variables.
I'll call them $a, b, c$.
First you set $a,b,c$ to these values:
\begin{Verbatim}[frame=single, fontsize=\footnotesize]
1, 1, 2, 3, 5, 8, 13
a  b  c
\end{Verbatim}
In other words $c = a + b$ after initializing $a$ and $b$ to 1.
Next you want to perform some computation to get this:
\begin{Verbatim}[frame=single, fontsize=\footnotesize]
1, 1, 2, 3, 5, 8, 13
   a  b  c
\end{Verbatim}
Next you want to perform some computation to get this:
\begin{Verbatim}[frame=single, fontsize=\footnotesize]
1, 1, 2, 3, 5, 8, 13
      a  b  c
\end{Verbatim}
and then this:
\begin{Verbatim}[frame=single, fontsize=\footnotesize]
1, 1, 2, 3, 5, 8, 13
         a  b  c
\end{Verbatim}
and finally this:
\begin{Verbatim}[frame=single, fontsize=\footnotesize]
1, 1, 2, 3, 5, 8, 13
            a  b  c
\end{Verbatim}
Basically \verb!a! plays the role of \verb!f(i - 2)!, 
\verb!b! plays the role of \verb!f(i - 1)!, 
and \verb!c! plays the role of \verb!f(i)!.
You run your \verb!i! from \verb!i = 2! to \verb!i = n!.

I'm going to call this the \defone{bottom-up} technique. 
You can use a for-loop or recursion to implement a
bottom-up algorithm.


\newpage
\begin{ex}
Implement this new Fibonacci function (a) using a for-loop and 
(b) using recursion.
Test them.
Compute the runtime.
\qed
\end{ex}


\newpage
\begin{ex}
The following is a linear recursion:
\begin{Verbatim}[frame=single, fontsize=\footnotesize]
def f(n):
    if n == 0:
        return 1
    elif n == 1:
        return 2
    else:
        return f(n - 1) + 3 * f(n - 2) + 1
\end{Verbatim}
Run this and compute \verb!f(0)!, \verb!f(1)!, ..., \verb!f(50)!.
Notice that it slows down tremendously.
Compute the runtime of this implementation.
Re-implement this using the bottom-to-top technique
(first using a for-loop and next using recursion.)
\qed
\end{ex}


\newpage
\begin{ex}
The following is a linear recursion:
\begin{Verbatim}[frame=single, fontsize=\footnotesize]
def f(n):
    if n == 0:
        return 0
    elif n == 1:
        return 1
    elif n == 2:
        return 2
    else:
        return f(n - 1) + 2*f(n - 2) + 3*f(n - 3) 
\end{Verbatim}
Run this and compute $f(0)$, $f(1)$, ..., $f(50)$.
Notice that it slows down tremendously.
Compute the runtime of this implementation.
Re-implement this using the bottom-to-top technique
(first using a for-loop and next using recursion.)
\qed
\end{ex}

%\newpage
%\subsection*{Solutions}
%
\newpage
\section*{Solutions}
Solution to Exercise \ref{ex:power-series-11}\labeltext{}{sol:power-series-11}.

\debug{\tinysidebar{exercises/{power-series-11/answer.tex}}}
 
(a) From
\begin{align*}
\sum_{n = 0}^\infty \frac{1}{2^n} x^n \cdot \sum_{n = 0}^\infty \frac{1}{2^n} x^n
&=
\left(
1 + \frac{1}{2}x + \frac{1}{4}x^2 + \frac{1}{8}x^3 + \cdots
\right)
\left(
1 + \frac{1}{2}x + \frac{1}{4}x^2 + \frac{1}{8}x^3 + \cdots
\right)
\end{align*}
the coefficient of $x^3$ is
\[
1 \cdot \frac{1}{8}
+ \frac{1}{2} \cdot \frac{1}{4}
+ \frac{1}{4} \cdot \frac{1}{2}
+ \frac{1}{8} \cdot 1
= 4 \cdot \frac{1}{8} = \frac{1}{2}
\]
The coefficient of $x^n$ is
\[
\sum_{k=0}^n \frac{1}{2^k} \cdot \frac{1}{2^{n-k}}
= \sum_{k=0}^n \frac{1}{2^k \cdot 2^{n-k}}
= \sum_{k=0}^n \frac{1}{2^n}
= \frac{n + 1}{2^n}
\]

(b)
First let's derive the coefficient of $x^n$ in general. 
The coefficient of $x^n$ is
\begin{align*}
\sum_{k=0}^n \frac{1}{2^k} \cdot \frac{1}{3^{n-k}}
&= \sum_{k=0}^n \frac{1}{2^k} \cdot \frac{3^k}{3^n} 
= \frac{1}{3^n} \sum_{k=0}^n \left(\frac{3}{2}\right)^k \\
&= \frac{1}{3^n} \cdot \frac{1 - (3/2)^{n+1}}{1 - 3/2} \\
&= \frac{1}{3^n} \cdot \frac{1 - (3/2)^{n+1}}{-1/2} \\
&= \frac{1}{3^n} \cdot \frac{(3/2)^{n+1} - 1}{1/2} \\
&= \frac{2}{3^n} \cdot \left( \frac{3^{n+1}}{2^{n+1}} - 1 \right) \\
&= 2 \cdot \left( \frac{3^{n+1} - 2^{n+1}}{2^{n+1}3^n} \right) \\
&= \frac{3^{n+1} - 2^{n+1}}{6^n}
\end{align*}
The coefficient of $x^3$ is
\[
\frac{3^4 - 2^{4}}{6^4} = \frac{65}{216}
\]



\newpage

Solution to Exercise \ref{ex:power-series-15}\labeltext{}{sol:power-series-15}.

\debug{\tinysidebar{exercises/{power-series-15/answer.tex}}}
We have
\begin{align*}
\left( \sum_{n=0}^\infty x^n \right)^{100} 
&= \left( \frac{1}{1 - x} \right)^{100} \\
&= \sum_{n=0}^\infty \binom{100 + n - 1}{n} x^n \\
&= \sum_{n=0}^\infty \binom{n + 99}{n} x^n \\
&= \sum_{n=0}^\infty \binom{n + 99}{99} x^n
\end{align*}
Hence the coefficient of $x^n$ is $\binom{n + 99}{99} x^n$ for $n \geq 0$.


\newpage

Solution to Exercise \ref{ex:power-series-16}\labeltext{}{sol:power-series-16}.

\debug{\tinysidebar{exercises/{power-series-16/answer.tex}}}

We have
\begin{align*}
&\left( 
2 + 5x + \frac{7}{1 - x}
\right)
\left( \sum_{n=0}^\infty x^n \right)^{100}
\\
&= \left( 
2 + 5x + \frac{7}{1 - x}
\right)
\left( \frac{1}{1 - x} \right)^{100}
\\
&=
2\left( \frac{1}{1 - x} \right)^{100}
+ 5x \left( \frac{1}{1 - x} \right)^{100}
+ \frac{7}{1 - x} \left( \frac{1}{1 - x} \right)^{100}
\\
&=
2 \sum_{n=0}^\infty \binom{100 + n - 1}{n} x^n
+ 5x \sum_{n=0}^\infty \binom{100 + n - 1}{n} x^n
+ 7 \left( \frac{1}{1 - x} \right)^{101} 
\\
&=
2 \sum_{n=0}^\infty \binom{n + 99}{n} x^n
+ 5x \sum_{n=0}^\infty \binom{n + 99}{n} x^n
+ 7 \sum_{n=0}^\infty \binom{101 + n - 1}{n}
\\
&=
\sum_{n=0}^\infty 2 \binom{n + 99}{99} x^n
+ \sum_{n=0}^\infty 5 \binom{n + 99}{99} x^{n+1}
+ \sum_{n=0}^\infty 7 \binom{n + 100}{n} x^n
\\
&=
\sum_{n=0}^\infty 2 \binom{n + 99}{99} x^n
+ \sum_{p=1}^\infty 5 \binom{p + 98}{99} x^{p}
+ \sum_{n=0}^\infty 7          \binom{n + 100}{n} x^n  \,\,\, \text{(let $p = n + 1$)}
\\
&=
\sum_{n=0}^\infty 2\binom{n + 99}{99} x^n
+ \sum_{n=1}^\infty 5 \binom{n + 98}{99} x^{n}  
+ \sum_{n=0}^\infty 7 \binom{n + 100}{100} x^n \,\,\,\text{(replace $p$ by $n$)}
\\
&=
2 \binom{99}{99} + \sum_{n=1}^\infty 2\binom{n + 99}{99} x^n
+ \sum_{n=1}^\infty 5\binom{n + 98}{99} x^{n} 
+ 7\binom{100}{100}  + \sum_{n=1}^\infty 7 \binom{n + 100}{100}  
\\
&=
9 +
\sum_{n=1}^\infty
\left( 2\binom{n + 99}{99} 
+  5\binom{n + 98}{99} 
+ 7 \binom{n + 100}{100}
\right) x^n
\end{align*}
Hence the coefficient of $x^n$ is
\[
\begin{cases}
9 & \text{ if } n = 0 \\
\displaystyle 2\binom{n + 99}{99} 
+  5\binom{n + 98}{99} 
+ 7 \binom{n + 100}{100} & \text{ if } n > 0
\end{cases}
\]


\newpage

Solution to Exercise \ref{ex:power-series-17}\labeltext{}{sol:power-series-17}.

\debug{\tinysidebar{exercises/{power-series-17/answer.tex}}}

(a)
\begin{align*}
\sum_{n=0}^\infty \frac{1}{2^n} x^n
\cdot
\sum_{n=0}^\infty \frac{1}{2^n} x^n
&=
\left( \sum_{n=0}^\infty \frac{1}{2^n} x^n \right)^2
\\
&=
\left( \sum_{n=0}^\infty \left( \frac{x}{2} \right)^n \right)^2
\\
&=
\left( \sum_{n=0}^\infty \left( \frac{x}{2} \right)^n \right)^2
\\
&=
\left( \frac{1}{1 - (x/2)} \right)^2
\\
&=\sum_{n=0}^\infty \binom{2 + n - 1}{n} \left( \frac{x}{2} \right)^n
\\
&=\sum_{n=0}^\infty \binom{n + 1}{n} \left( \frac{1}{2} \right)^n x^n
\\
\sum_{n=0}^\infty \left( \frac{n + 1}{2^n} \right x^n
\end{align*}

(b)
\begin{align*}
\sum_{n=0}^\infty \frac{1}{2^n} x^n
\cdot
\sum_{n=0}^\infty \frac{1}{3^n} x^n
\end{align*}
    


