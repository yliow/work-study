%-*-latex-*-
\section{Substitution method}

Actually a better name is \lq\lq induction'', as in \lq\lq mathematical
induction''.

Suppose you have this recursion:
\begin{align*}
T(n) &= 2 T(n-1) + A \\
T(0) &= B
\end{align*}
You can guess that $T(n) = O(2^n)$
since
\begin{align*}
T(n)
&= 2 T(n-1) + A \\
&= 2 (2T(n-2) + A) + A = 2^2 T(n-2) + 2A + A\\
&= 2^2(2T(n-3) + A) + 2A + A = 2^3T(n-3) + (2^2 + 2 + 1)A 
\end{align*}
A few more of that and you'll see that
\begin{align*}
T(n)
&= 2^n T(0) + (2^{n-1} + \cdots + 2^2 + 2 + 1) A \\
&= B 2^n  + (2^{n-1} + \cdots + 2^2 + 2 + 1) A \\
&= B 2^n  + \frac{2^n - 1}{2 - 1} A \\
&= (A + B) 2^n  - A \\
&= O(2^n)
\end{align*}
So how does the substitution method help prove this fact?

First of all to prove the above by induction, you have
to have a statement to prove.
So ... let's say I try the following statement:
\[
P(n) = [T(n) \leq 2^n]
\]
Of course I can be more flexible than that.
For instance I can let
\[
P'(n) = [T(n) \leq 2^n + n]
\]
or
\[
P''(n) = [T(n) \leq B 2^n + Cn]
\]
etc.
The last two statements will still give me $T(n) = O(2^n)$.
But let's try $P(n)$ first.
(Remember that there's the problem with induction.
You HAVE to come up with a statement to prove.
From discrete math 2, you saw/will see that there are
techniques that will PRODUCE formulas for you so you
don't have to guess-then-prove.)

Now remember that to say about about big-$O$, we don't have to
prove $P(0)$ and $P(1)$.
It can be any starting point.

Assume $P(n)$, i.e.,
\[
T(n) \leq 2^n
\]
Then from
\begin{align*}
 T(n + 1) = 2T(n) + 1 
\end{align*}
we get
\begin{align*}
T(n + 1)
&= 2T(n) + A  \\
&\leq 2(2^n) + A  \\
&\leq 2^{n+1} + A  \\
\end{align*}

Hey ... wait a minute ...
I want to show $P(n+1)$, i.e., I want
\[
T(n+1) \leq 2^{n+1}
\]
But I got
\[
T(n+1) \leq 2^{n+1} + A
\]

Hmmm ... OK ... what if I use this instead:
\[
P(n) = [T(n) \leq 2^n + C]
\]
where $C$ is some constant.
I don't know what $C$ is going to be just yet.
But anyway if I use this $P(n)$,
assuming $P(n)$ holds:
\[
P(n) = [T(n) \leq 2^n + C]
\]
then I can say:
\begin{align*}
  T(n + 1)
  &= 2 T(n) + A \\
  &\leq 2 (2^n + C) + A \\ 
  &\leq 2^{n+1} + 2C + A \\ 
\end{align*}
Again it does not work.

How about we try
\[
P(n) = [T(n) \leq C2^n]
\]
Well then
\begin{align*}
T(n+1)
&= 2T(n) + A \\
&\leq 2(C2^n) + A \\
&= C2^{n+1} + A \\
\end{align*}
Again ... failure!
Arghh!!! OK ... let's try this:
\[
P(n) = [T(n) \leq C2^n + D]
\]
Then
\begin{align*}
T(n+1)
&= 2T(n) + A \\
&\leq 2(C2^n + D) + A \\
&= C2^{n+1} + 2D + A \\
\end{align*}
Hmmmm ... well in this case if I \textit{want} to say
\[
T(n+1)
\leq C2^{n+1} + 2D + A \leq C2^{n+1} + D
\]
then I need
\[
C2^{n+1} + 2D + A \leq C2^{n+1} + D
\]
which means that
\[
D + A \leq 0
\]
i.e.,
\[
D \leq -A
\]
So I must at least choose $D$, say, to be $-A$.
And it seems that $C$ can be any constant.
But wait ... we still have to check for $P(0)$!
In that case I want
\[
P(0) = [T(0) \leq C2^0 + D]
\]
to be true
and if I want $T(0) \leq C2^0 + D = C + D = C - A$
then I need
\[
B \leq C - A
\]
So $C$ must satisfy
\[
A + B \leq C
\]
So maybe I can choose $C = A + B$.
Hmmmm ... OK ... let's try to do this problem all over again.

We're given 
\begin{align*}
T(n) &= 2 T(n-1) + A, \,\,\,\,\, n > 0 \\
T(0) &= B
\end{align*}
Let $P(n)$ be the statement
\[
P(n) = [T(n) \leq (A + B)2^n - A]
\]
Then, if $P(n)$ holds, i.e., if $T(n) \leq (A+B)2^n - A$ is true, we have
\begin{align*}
T(n+1)
&= 2T(n) + A \\
&\leq 2((A+B)2^n - A) + A \\
&= 2(A+B)2^{n} - 2A + A \\
&= (A+B)2^{n+1} - A
\end{align*}
Voila!
That's the inductive case.
For the base case we have
\[
T(0) = B \leq (A + B) - A = (A + B)2^0 - A
\]
Therefore for all $n \geq 0$,
\[
T(n) \leq (A+B)2^n - A
\]
This of course implies that
\[
T(n) = O(2^n)
\]

(Now where have you see
$T(n) \leq (A+B)2^n - A$ before?
Look at the beginning of this section.)

The above illustrates how to prove an upper bound
of $T(n)$ exists 
if $T(n)$
satisfies a recurrence relation.

\begin{ex}
  From the above, it seems that there are other choices for $C$ and $D$.
  Pick another pair of $C$,$D$ and see if you can still
  prove $T(n) = O(2^n)$.
  \qed
\end{ex}


\newpage
Suppose
\[
T(n) = 2T(n - 1) + An + B, \,\,\,\,\, n > 0
\]
and
\[
T(0) = C
\]
where $A, B, C \geq 0$.

I guess that 
\[
T(n) \leq D2^n + En + F
\]
If that's the case, then
\begin{align*}
T(n+1)
&= 2T(n) + An + B \\
&\leq 2(D2^n + En + F) + An + B \\
&= D2^{n+1} + 2En + 2F + An + B \\
&= D2^{n+1} + (2E + A) n + (2F + B) 
\end{align*}
and if I want
\[
D2^{n+1} + (2E + A) n + (2F + B) \leq D2^{n+1} + E(n+1) + F
\]
then
\[
(E + A) n + (F + B) \leq E 
\]
I can achieve this if for instance I pick
\[
E = -A
\]
and
\[
F = E - B = -A - B
\]
For the base condition
\[
C = T(0) \leq D + F
\]
i.e.,
\[
C \leq D + (-A-B)
\]
I can choose $D = A+B+C$.
Now I'm ready to compute the big-$O$ of $T(n)$ with induction ...

Let
\[
T(n) = 2T(n - 1) + An + B, \,\,\,\,\, n > 0
\]
and
\[
T(0) = C
\]
Let
\[
P(n) = [T(n)\leq (A+B+C)2^n - An - A - B]
\]
Then
\[
T(0) = C = (A + B + C)2^0 - A\cdot 0 - A - B
\]
In other words $P(0)$ holds.
Now assume that
\[
T(n) \leq (A+B+C)2^n - An - A - B
\]
Then
\begin{align*}
  T(n+1)
  &= 2T(n) + An + B \\
  &\leq 2 ((A+B+C)2^n - An - A - B) + An + B \\
  &= (A+B+C)2^{n+1} - 2An - 2A - 2B + An + B \\
  &= (A+B+C)2^{n+1} - An - 2A - B \\
  &\leq (A+B+C)2^{n+1} - An - A - B \\
  &= (A+B+C)2^{n+1} - (n + 1)A - B \\
\end{align*}

(?? I need $A \geq 0$ ??)

Hence $P(n+1)$ holds.

\newpage
