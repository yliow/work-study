%-*-latex-*-
\sectionthree{Summation}
\begin{python0}
from solutions import *; clear()
\end{python0}


I'm going to compute the runtime of the bubblesort very soon.
Before I do that I'm going to give you the formula for the arithmetic sum
which will be helpful in runtime computations:
\[
1 + 2 + \cdots + n = \frac{n(n+1)}{2}
\]
In fact sums are common in this game.
So I'm going to use the summation notation to simplify the computation.

Let $a_i$ be a formula in $i$.
The notation
\[
\sum_{i = 3}^7 a_i
\]
is a shorthand notation for 
\[
\sum_{i = 3}^7 a_i = a_3 + a_4 + a_5 + a_6 + a_7
\]
For instance
\[
\sum_{i = 3}^7 i^2 = 3^2 + 4^2 + 5^2 + 6^2 + 7^2
\]

%-*-latex-*-

\begin{ex} 
  \label{ex:prob-00}
  \tinysidebar{\debug{exercises/{disc-prob-28/question.tex}}}

  \solutionlink{sol:prob-00}
  \qed
\end{ex} 
\begin{python0}
from solutions import *
add(label="ex:prob-00",
    srcfilename='exercises/discrete-probability/prob-00/answer.tex') 
\end{python0}

%-*-latex-*-

\begin{ex} 
  \label{ex:prob-00}
  \tinysidebar{\debug{exercises/{disc-prob-28/question.tex}}}

  \solutionlink{sol:prob-00}
  \qed
\end{ex} 
\begin{python0}
from solutions import *
add(label="ex:prob-00",
    srcfilename='exercises/discrete-probability/prob-00/answer.tex') 
\end{python0}




\newpage

Let me summarize the basic summmation formulas here so that you can refer
to them:

\begin{prop}
  \mbox{}
  \begin{myenum}
  \item
    \[
    \sum_{i = 1}^n c a_i = c\sum_{i = 1}^n a_i
    \]
  \item
    \[
    \sum_{i = 1}^n (a_i + b_i) = \sum_{i = 1}^n a_i + \sum_{i=1}^n b_i
    \]
  \item
    \[
    \sum_{i = 1}^n (ca_i + db_i) = c\sum_{i = 1}^n a_i + d\sum_{i=1}^n b_i
    \]
  \end{myenum}
  Of course the above formulas hold when the lower limit of the summation
  are all changed to another value.
    \begin{myenum}
  \item
    \[
    \sum_{i = 1}^n c a_i = c\sum_{i = 1}^n a_i
    \]
  \item
    \[
    \sum_{i = 1}^n (a_i + b_i) = \sum_{i = 1}^n a_i + \sum_{i=1}^n b_i
    \]
  \item
    \[
    \sum_{i = 1}^n (ca_i + db_i) = c\sum_{i = 1}^n a_i + d\sum_{i=1}^n b_i
    \]
  \end{myenum}
\end{prop}

\newpage
The following \lq\lq splitting out terms from the bottom'' is obviously true:
\[
\sum_{i = 0}^{10} a_i = a_0 + a_1 + a_2 + \sum_{i = 2}^{10} a_i
\]
So is \lq\lq splitting out terms from the top'':
\[
\sum_{i = 0}^{10} a_i = \sum_{i = 0}^{8} a_i + a_9 + a_{10} 
\]
Likewise you can split a summation into two like this:
\[
\sum_{i = 0}^{10} a_i = \sum_{i=0}^2 a_i + \sum_{i=3}^{10} a_i
\]

\begin{ex}
You are given 
\[
\sum_{i = 1}^n a_i = 42 \,\,\,\,\, \text{ and } \,\,\,\,\, 
\sum_{i = 1}^n b_i = 60
\]
Compute
\[
\sum_{i = 1}^n (2a_i + 3b_i)
\]
\qed
\end{ex}


\begin{ex}
You are given $\sum_{i=1}^{11} a_i = 120$,
$\sum_{i=11}^{20} a_i = 42$, and $\sum_{i=1}^{20} a_i = 691$.
What can you tell me about $a_{11}$?
\qed
\end{ex}

