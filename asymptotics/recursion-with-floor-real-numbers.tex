%-*-latex-*-
\section{Recursion with Floors - Extending the 
Domain to Real Numbers}

Now I'm going to show you $T(n) \leq g(n) = n + n \log n + (n - 1)$
in a different way.

Notice that earlier when working for $f(n)$, I usually treat $n$
as a variable that takes real values and not just integer values.
(I extend the domain from positive integers to positive real nuimbers.)
I'm going to use the same idea on $T(n)$.

As before (see previous section) suppose I have
\[
T(n) 
= 
\begin{cases}
A                        & \text{if $n = 0, 1$} \\
2T(\floor{n/2}) + Bn + C & \text{otherwise}
\end{cases}
\]
then the big-O of $T(n)$ is the same when you use this instead:
\[
T(n) = 2T(n/2) + Bn + C
\]
To be more precise, of course I meant that if I define another function
with a domain of $[0, \infty)$ (not positive integers):
\[
S(n) 
= 
\begin{cases}
A                & \text{if $n = 0, 1$} \\
2S(n/2) + Bn + C & \text{otherwise}
\end{cases}
\]
then
\[
T(n) = O(S(n))
\]
I hope you realize that the above definition of $S(n)$ has some problems.
Here's the first problem:
Is saying
\[
S(0) = A
\]
(the base case) good enough?
Not quite!!! 
(Are you awake yet???)

Why?
Because for any positive integer $n$, using the recurrence relations,
the computation of $T(n)$ will ultimately arrive at $T(0)$.
Not so for $S(n)$!!!
For instance if I look at $S(3/2)$. 
Then $S(3/2)$ is in terms of $S(3/4)$ which is in terms of 
$S(3/8)$, etc.
It never reaches $S(0)$!
Well ... at least it's moving towards 0 (although never reaching it.)

It's far worse for linear recurrence.
Say
\[
T(n) = T(n - 1) + T(n - 2)
\]
with $T(0) = T(1) = 1$.
If I extend $n$ to have a domain of real numbers, I would have
problems with $T(3/2)$ ... it would ultimately skip $T(0)$ and $T(1)$ and shoot
past $0$ and go on toward $-\infty$!!!

So if given 
\[
T(n) 
= 
\begin{cases}
A                        & \text{if $n = 0, 1$} \\
2T(\floor{n/2}) + Bn + C & \text{otherwise}
\end{cases}
\]
I define
\[
S(n)
=
\begin{cases}
A                 & \text{if $0 \leq n \leq 1$} \\
2S(n/2) + Bn + C  & \text{otherwise}
\end{cases}
\]
(In fact because I only care for large $n$, I can even define
$S(n) = A$ for $0 \leq n \leq 10$.
It doesn't matter at all.)
Let me plot $T(n)$ and $S(n)$ for the case of
\[
A = B = C = 1
\]
Here's the plot of $T(n)$ (blue) and $S(n)$ (red):
\begin{python}
from plot import *
N = 33
STEP = 1
table = {}
def t(n):
    n = int(n)
    if n <= 1: return 1
    else:
        if not n in table:
            table[n] = 2 * t(n/2) + n + 1
        return table[n]
def s(x):
    x = float(x)
    if x <= 1: return 1
    else:
        return 2 * s(x/2.0) + x + 1

p = Plot(domain="0:%s" % N)

points = [(n, t(n)) for n in range(0, N + 1, STEP)]
p.add(points, style='step', color='blue', python=1)

points = [(n/50.0, s(n/50.0)) for n in range(0, N*50 + 1, STEP)]
p.add(points, style='step', color='red', python=1)
#p.add("s(x)", python=1, color='red', num_points=1000) #??? does not work

print(p)
\end{python}

Note that at powers of $2$, there are sharps for both functions, 
but it seems that $T(n)$ and $S(n)$ does not seem to be the same
near powers of 2. Let me compute the function at $2^4$:
\[
T(16) = 95, \,\,\,\,\, S(16) = 95
\]
Now let's see what happens to $S(n)$ for $n$ just before and just after 16:
\[
S(15.9) = 94.6, \,\,\,\,\, S(16.1) = 143.5 
\]


\begin{ex}
Compute $S(15.9), S(16), S(16.1)$ by hand.
\qed
\end{ex}


As before, define
\[
g(n) = n + n \log_2 n + (n - 1)
\]
Recall that $T(n) \leq g(n)$.
Let's plot that too (with green):

\begin{python}
from plot import *
N = 1000
STEP = 1
table = {}
def t(n):
    n = int(n)
    if n <= 1: return 1
    else:
        if not n in table:
            table[n] = 2 * t(n/2) + n + 1
        return table[n]

def s(x):
    x = float(x)
    if x <= 1: return 1
    else:
        return 2 * s(x/2.0) + x + 1

p = Plot(domain="0:%s" % N)

points = [(n, t(n)) for n in range(0, N + 1, STEP)]
p.add(points, style='step', color='blue', python=1)

points = [(n, s(n)) for n in range(0, N + 1, STEP)]
p.add(points, style='step', color='red', python=1)

p.add("x + x*log(x, 2) + (x - 1)", color='green', python=1, num_points=1000)
print(p)
\end{python}

And here's the plot for a large domain:
\begin{python}
from latextool_basic import *
N = 1000000
STEP = 2000
table = {}
def t(n):
    n = int(n)
    if n <= 1: return 1
    else:
        if not n in table:
            table[n] = 2 * t(n/2) + n + 1
        return table[n]

def s(x):
    x = float(x)
    if x <= 1: return 1
    else:
        return 2 * s(x/2.0) + x + 1

p = FunctionPlot(domain="0:%s" % N)

points = [(n, t(n)) for n in range(0, N + 1, STEP)]
p.add(points, style='step', color='blue', python=1)

points = [(n, s(n)) for n in range(0, N + 1, STEP)]
p.add(points, style='step', color='red', python=1)

p.add("x + x*log(x, 2) + (x - 1)", color='green', python=1, num_points=1000)
print(p)
\end{python}
and even large domain:
\begin{python}
from latextool_basic import *
N = 100000000
STEP = 200000
table = {}
def t(n):
    n = int(n)
    if n <= 1: return 1
    else:
        if not n in table:
            table[n] = 2 * t(n/2) + n + 1
        return table[n]
def s(x):
    x = float(x)
    if x <= 1: return 1
    else:
        return 2 * s(x/2.0) + x + 1

p = FunctionPlot(domain="0:%s" % N)

points = [(n, float(t(n))) for n in range(0, N + 1, STEP)]
p.add(points, style='step', color='blue', python=1)

points = [(n, s(n)) for n in range(0, N + 1, STEP)]
p.add(points, style='step', color='red', python=1)

p.add("x + x*log(x, 2) + (x - 1)", color='green', python=1, num_points=1000)
print(p)
\end{python}

We see that 
\[
T(n) \leq S(n) \leq g(n)
\]
with equality when $n$ is a power of $2$.



\begin{ex}
Compute $T(15)$, $T(16)$, and $T(17)$ by hand.
\end{ex}

Note that $S(n)$ is defined simply from $T(n)$
by extending the domain from positive integer to real numbers.
Now how come $S(n) : T(n)$ for $n > 1$.
Let's try compute $S(17)$.
\begin{align*}
S(2^4 + 1) 
&= 2S(2^3 + 1/2) + (2^4 + 1) + 1 \\
&= 2(2S(2^2 + 1/4) + (2^3 + 1/2)1 + 1) + (2^4 + 1)1 + 1 \\
&= 2^2 S(2^2 + 1/4) + 2(2^4 + 1)1 + (1 + 2 + 2^2)1 \\
&= ... \\
&= 2^4 S(2^0 + 1/16) + 4(2^4 + 1)1 + (1 + 2 + 2^3 + 2^4)1
\end{align*}
So you see that 
\begin{align*}
S(2^m + 1) &= 2^m S(2^0 + 1/2^m) + m(2^m + 1)1 + (2^m - 1)1 \\
T(2^m + 1) &= 2^m T(2^0) + m(2^m + 1)1 + (2^m - 1)1 
\end{align*}
The point is that, for $S(n)$, since there is no floor,
after $m$ substitutions of the recurrence relation,
your $S(2^m)$ reached $S(1 + 1/2^m)$, and {\it not} $S(1)$, 
which is not in the range 
of $[0, 1]$ yet.
This means that you need {\it one} more substitution of the 
recurrence relation.
So with one more recurrence on $S$ we get
\begin{align*}
S(2^m + 1) 
&= 2^{m+1} S(2^{-1} + 1/(2^{m+1})) + (m+1)(2^{m+1} + 1) + (2^{m+1} - 1) \\
&= 2^{m+1} + (m+1)(2^{m+1} + 1) + (2^{m+1} - 1)
\end{align*}
whereas
\[
T(2^m + 1) = 2^m  + m(2^m + 1) + (2^m - 1) 
\]
And if I express the above in terms of $n$ (i.e., $n = 2^m$):
\begin{align*}
S(n + 1) 
&= 2n + (\log_2(n)+1)(2n + 1) + (2n - 1) \\
T(n + 1) &= n  + \log_2(n) (n + 1) + (n - 1) 
\end{align*}

In general for $0 \leq x < 2^m$,
\begin{align*}
S(2^m + x) 
&= 2^{m+1} S(2^{-1} + x/(2^{m+1})) + (m+1)(2^{m+1} + 1) + (2^{m+1} - 1) \\
&= 2^{m+1} + (m+1)(2^{m+1} + 1) + (2^{m+1} - 1) 
\end{align*}

This explains the sudden jump in $S(n)$ immediately after powers of $2$.

\begin{python}
from plot import *
N = 20
STEP = 1
table = {}
def t(n):
    n = int(n)
    if n <= 1: return 1
    else:
        if not n in table:
            table[n] = 2 * t(n/2) + n + 1
        return table[n]
def s(x):
    x = float(x)
    if x <= 1: return 1
    else:
        return 2 * s(x/2.0) + x + 1

p = Plot(domain="0:%s" % N)

points = [(n, t(n)) for n in range(0, N + 1, STEP)]
p.add(points, style='step', color='blue', python=1)

xs = [N * x/1000.0 for x in range(0, 1000)]
points = [(n, s(n)) for n in xs]
p.add(points, style='step', color='red', python=1)

p.add("x + x*log(x, 2) + (x - 1)", color='green', python=1, num_points=1000)
print(p)
\end{python}

Now I'm going to modify $S(n)$ so that the extra application of
recurrence is not necessary.
This means that I want $S(n)$ to satify
\[
S(n) = 
\begin{cases}
A               & \text{if $0 \leq n < 2$ (see the difference?)} \\
S(n/2) + Bn + C &\text{otherwise} 
\end{cases}
\]
Now I'm plot the graphs:

\begin{python}
from plot import *
N = 20
STEP = 1
table = {}
def t(n):
    n = int(n)
    if n <= 1: return 1
    else:
        if not n in table:
            table[n] = 2 * t(n/2) + n + 1
        return table[n]

def s(x):
    x = float(x)
    if x < 2: return 1
    else:
        return 2 * s(x/2.0) + x + 1

p = Plot(domain="0:%s" % N)

points = [(n, t(n)) for n in range(0, N + 1, STEP)]
p.add(points, style='step', color='blue', python=1)

xs = [N * x/1000.0 for x in range(0, 1000)]
points = [(n, s(n)) for n in xs]
p.add(points, style='step', color='red', python=1)

p.add("x + x*log(x, 2) + (x - 1)", color='green', python=1, num_points=1000)
print(p)
\end{python}

Now $S(n)$ hugs $T(n)$ a lot closer.
Furthermore, $S(n) \leq g(n)$.
We have
\[
T(n) \leq S(n) \leq g(n)
\]
for $n \geq 1$ and they all agree at powers of $2$.

Now instead of proving $T(n) = O(g)$, by showing $T(n) \leq g(n)$,
I want to proving $T(n) \leq S(n)$ and then $S(n) \leq g(n)$.
Why does one want to do that?
Notice that that the function $S(n)$ is smoother.

Now we know that 
\[
T(n) = S(n) = g(n)
\]
when $n$ is a power of $2$.
Suppose we can show that for all integer $n$,
\[
T(n) \leq S(n)
\]
and furthermore we can show on all intervals $(2^m, 2^{m+1}]$
$S(n)$ and $g(n)$ are increasing
