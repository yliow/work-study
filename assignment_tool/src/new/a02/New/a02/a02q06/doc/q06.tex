This is a continuation of Q5.

It's reasonable to also have augmented operations on fractions so that
we can perform something like \verb!a/b += c/d!, i.e., a function
called \verb!Fraction_addeq!.
For instance if \verb!a/b! is $1/2$ and \verb!c/d! is $1/4$,
after \verb!a/b += c/d!, $a/b$ becomes $3/4$ (and \verb!c/d! is unchanged.)
I'll leave it to you to try that on your own just for practice.
These and other operators will appear in a future assignment.

For this question, we start writing comparison operations.

Now include a function to check when two fractions are the same.
Test it throughly.
The name of the function must be \verb!Fraction_eq!.
The function returns a boolean value.
(There's only one reasonable prototype of this function.)

You must add a test function to the main C++ file and allow user to test.
your function when the user enters option 6. (Refer to Q2.)
Previous code for testing with option 1, 2, 3, 4, and 5 must be kept.


WARNING: $1/2$ is equal to $-2/-4$, $-1/2$ is equal to $1/(-2)$.

I'm sure you know how to check if two fractions are the same.
But I'll give you some test cases anyway, so that we can agree on the
output format. Basically, you print the boolean value returned by the
function you are implementing.

Test 1
\begin{console}[commandchars=\\\{\}]
\underline{\texttt{6 1 2 -2 -4}}
1
\end{console}

Test 2
\begin{console}[commandchars=\\\{\}]
\underline{\texttt{6 -1 2 1 -2}}
1
\end{console}

Test 3
\begin{console}[commandchars=\\\{\}]
\underline{\texttt{6 1 2 2 1}}
0
\end{console}

