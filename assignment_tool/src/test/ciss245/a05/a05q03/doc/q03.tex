%-*-latex-*-
This is a practice on passing pointers to functions.

We want to write some functions that performs \lq\lq increment by
a given amount''.
First run this progrm:
\begin{console}
#include <iostream>

void inc_by_version1(int i, int amt)
{
    i += amt;
}

void inc_by_version2(int & i, int amt)
{
    i += amt;
}

int main()
{
    int i = 40;

    inc_by_version1(i, 2);
    std::cout << i << ' ';
    i = 40;

    inc_by_version2(i, 2);
    std::cout << i << ' ';
    i = 40;

    std::cout << '\n';
    return 0;
}
\end{console}

The function \verb!inc_by_version1()! does not work:
\verb!i! in \verb!main()! is not increment at all.
That's because the in this function call, we are
using pass by value.

The function \verb!inc_by_version2()! does work:
we're using pass by reference so that the 
\verb!i! in \verb!inc_by_version2()!
is a reference to the \verb!i! in \verb!main()!.
Review references if necessary.

Complete the following third version using the following
given code;
The third version uses a pointer as parameter:

\begin{console}
#include <iostream>

void inc_by_version1(int i, int amt)
{
    i += amt;
}

void inc_by_version2(int & i, int amt)
{
    i += amt;
}

void inc_by_version3(int * i, int amt)
{
    // TO BE COMPLETED.
}

int main()
{
    int i = 40;

    inc_by_version1(i, 2);
    std::cout << i << ' ';
    i = 40;

    inc_by_version2(i, 2);
    std::cout << i << ' ';
    i = 40;

    inc_by_version3(&i, 2);
    std::cout << i << ' ';
    i = 40;

    std::cout << '\n';
    return 0;
}
\end{console}
