Q1.
The follow skeleton code is given. There are three files.
Note that the files are (of course) incomplete not just because the
functions are not implemented.
The preprocessor directives (i.e., \verb!ifndef!, ...) are not included:
you must complete them too.

{\small
\begin{console}
// File  : main.cpp
// Author: smaug

#include <iostream>
#include "Fraction.h"

void test_Fraction_print()
{
    int xn = 0, xd = 0; // numerator and denominator of a fraction
    std::cin >> n >> d;
    Fraction_print(xn, xd);
    std::cout << std::endl;
}


int main()
{
    int option;
    std::cin >> option:
    switch (option)
    {
        case 1:
            test_Fraction_print();
            break;
    }
    
    return 0;
}
\end{console}
\begin{console}
// File: Fraction.h
// Author: smaug

//-----------------------------------------------------------------------------
// Print fraction modeled by numerator n and denominator d.
//-----------------------------------------------------------------------------
void Fraction_print(int, int);

\end{console}
\begin{console}
// File: Fraction.cpp
// Author: smaug

#include <iostream>


void Fraction_print(int n, int d)
{
}
\end{console}
}

The following are test cases. Note that underlined texts denotes
user input.
It should be clear from the test cases that the output
is a reduced fraction, i.e., common factors between the numerator
and denominator are removed.

Test 1
\begin{console}[commandchars=\\\{\}]
\underline{\texttt{1 1 3}}
1/3
\end{console}

Test 2
\begin{console}[commandchars=\\\{\}]
\underline{\texttt{1 -1 3}}
-1/3
\end{console}

Test 3
\begin{console}[commandchars=\\\{\}]
\underline{\texttt{1 1 -3}}
-1/3
\end{console}

Test 4
\begin{console}[commandchars=\\\{\}]
\underline{\texttt{1 -1 -3}}
1/3
\end{console}

Test 5
\begin{console}[commandchars=\\\{\}]
\underline{\texttt{1 0 1}}
0
\end{console}

Test 6
\begin{console}[commandchars=\\\{\}]
\underline{\texttt{1 0 5}}
0
\end{console}

Test 7
\begin{console}[commandchars=\\\{\}]
\underline{\texttt{1 0 -6}}
0
\end{console}

Test 8
\begin{console}[commandchars=\\\{\}]
\underline{\texttt{1 10 3}}
10/3
\end{console}

Test 9
\begin{console}[commandchars=\\\{\}]
\underline{\texttt{1 10 2}}
5
\end{console}

Test 10
\begin{console}[commandchars=\\\{\}]
\underline{\texttt{1 10 15}}
2/3
\end{console}

Test 11
\begin{console}[commandchars=\\\{\}]
\underline{\texttt{1 60 45}}
4/3
\end{console}

Test 12
\begin{console}[commandchars=\\\{\}]
\underline{\texttt{1 10 0}}
undefined
\end{console}

Test 13
\begin{console}[commandchars=\\\{\}]
\underline{\texttt{1 -5 0}}
undefined
\end{console}


