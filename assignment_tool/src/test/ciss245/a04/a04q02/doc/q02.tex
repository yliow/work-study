%-*-latex-*-
This is a practice on using pointers and the free store (memory heap).

The goal is to
write a function \verb!pi()! (not the number \verb!pi = 3.14159...!)
such that \verb!pi(x)! is the number of primes $\leq$ \verb!x!.
\verb!x! is a \verb!double!.
We'll use the
sieve of Eratosthenes.
This was mentioned in CISS240.
See next page for a review of the algorithm.
Here's the skeleton code:
\begin{console}
#include <iostream>

int pi(double x)
{
    int n = x; 
    int * p;

    // Allocate an array of size n for p to point to.

    // Perform the sieve of Eratosthenes on the array that p points to.

    // Count the number of primes in the array that p points to and store
    // in ret.
    int ret;

    // Deallocate the memory used by p.
    
    return ret;
}

int main()
{
    double x;
    std::cin >> x;
    std::cout << pi(x) << '\n';
    return 0;
}
\end{console}

\begin{comment}
In function \verb!pi()!, other than variable \verb!ret!, all
other variables must be pointers.

(Note: Again, this is a practice on using pointers and the memory heap.
It is actually more natural to use integer variables rather than pointers
in your function.)

The following skeleton code must be used.
Do not modify \verb!main()!.

\begin{console}
#include <iostream>

int pi(double x)
{
    int * numprimes;
    // Declare pointers and allocate memory for all of them

    // Compute number of primes <= x and store at integer that
    // numprimes points to.

    int ret = numprimes;
    // Deallocate memory for all pointers declared

    return ret;
}

int main()
{
    double x;
    std::cin >> x;
    std::cout << pi(x) << '\n';
}
\end{console}
\end{comment}

\resett
\nextt
\begin{console}[commandchars=\\\{\}]
\underline{\texttt{100}}
25
\end{console}
(There are 25 primes in 0, 1, 2, ..., 100.)

\nextt
\begin{console}[commandchars=\\\{\}]
\underline{\texttt{101.5}}
26
\end{console}
(There are 26 primes in 0, 1, 2, ..., 101.)




\newpage
\textsc{Sieve of Eratosthenes}

The sieve of \href{https://en.wikipedia.org/wiki/Eratosthenes}{Eratosthenes} is a method to compute
primes up to a certain positive integer $n$.
The following illustrates the idea.

Let's compute all the prime up to 99.
Create an array \verb!x! of size 100, i.e.,
you have
\verb!x[0]!,
\verb!x[1]!,
\verb!x[2]!, ...
\verb!x[99]!.
After we're done with the computation, 
\verb!x[i]! will be \verb!1! if \verb!i! is a prime.
For instance when the computation ends,
the first 7 values in \verb!x! which are \verb!1! are
\verb!x[2]!,
\verb!x[3]!,
\verb!x[5]!,
\verb!x[7]!,
\verb!x[11]!,
\verb!x[13]!,
\verb!x[17]!.
Therefore from the array, you can tell that
\verb!2!,
\verb!3!,
\verb!5!,
\verb!7!,
\verb!11!,
\verb!13!,
\verb!17!
are all primes.
However
\verb!x[0]!,
\verb!x[1]!,
\verb!x[4]!,
\verb!x[6]!,
\verb!x[8]!,
\verb!x[9]!,
\verb!x[10]!,
\verb!x[12]!,
\verb!x[14]!,
\verb!x[15]!,
\verb!x[16]!
and all \verb!0! because
\verb!0!,
\verb!1!,
\verb!4!,
\verb!6!,
\verb!8!,
\verb!9!,
\verb!10!,
\verb!12!,
\verb!14!,
\verb!15!,
\verb!16!
are not primes.
By counting the number of \verb!1!'s in \verb!x!, you will then have
the number of primes up to \verb!99!.

Here's the idea of the algorithm.
First initialize 
\verb!x[0]! and 
\verb!x[1]! to \verb!0! (since \verb!0! and \verb!1! are not
primes)
and all other values of \verb!x! to \verb!1!.
Now we begin.

The first value of \verb!x! that is \verb!1! is at index \verb!2!.
Therefore \verb!2! is a prime.
Call it a killer prime.
\verb!2! will now kill off all the multiple of \verb!2! except for \verb!2!,
i.e.,
you set
\verb!x[4]!,
\verb!x[6]!,
\verb!x[8]!,
\verb!x[10]!,...
\verb!x[98]! to \verb!0!.
Remember that you are at index \verb!2!.

Now for the next stage, from index \verb!2!, go to the next
index \verb!i! such that \verb!x[i]! is \verb!1!.
This would be \verb!x[3]!.
So \verb!3! is the next killer prime.
\verb!3! will kill of all the multiples of \verb!3! except for \verb!3!, i.e., 
you set
\verb!x[6]!,
\verb!x[9]!,
\verb!x[12]!,
\verb!x[15]!,...,
\verb!x[99]! to \verb!0!.
(Note that \verb!x[6]! was already set to \verb!0! by
killer prime \verb!2!.
So \verb!x[6]! is set to \verb!0! twice.)
Recall that you are at index \verb!3!.

Now from index \verb!3!, go to the next
index \verb!i! such that \verb!x[i]! is \verb!1!.
This would be \verb!x[5]!.
So \verb!5! is the next killer prime.
\verb!5! will kill of all the multiples of \verb!5! except for 5, i.e., 
you set
\verb!x[10]!,
\verb!x[15]!,
\verb!x[20]!,
\verb!x[25]!,...
\verb!x[95]! to \verb!0!.

Go to the next
index \verb!i! such that \verb!x[i]! is \verb!1!.
This would be \verb!x[7]!.
So you set
\verb!x[14]!,
\verb!x[21]!,
\verb!x[28]!,
\verb!x[35]!,...
\verb!x[98]! to \verb!0!.

If you keep doing this, you'll see that
an index value \verb!i! is a prime exactly when
\verb!x[i]! is \verb!1!.
By counting all such \verb!i! values, you will then
have all the primes from 0 to 99.

Another thing to note is that you don't have to
locate the killer primes all the way up to 99.
You can in fact stop the search for killer primes
beyond $\sqrt{99} \approx 9.95$.
For instance, if you continue algorithm above beyond \verb!7!,
the next killer prime would have been \verb!11!.
But the multiples of \verb!11! greater than \verb!11!
are
\verb!22!,
\verb!33!,
\verb!44!,
\verb!55!,
\verb!66!,
\verb!77!,
\verb!88!, and
\verb!99!,
and these are already killed by
\verb!2!,
\verb!3!,
\verb!2!,
\verb!5!,
\verb!2!,
\verb!7!,
\verb!2!,
\verb!3!.

(Extra challenge especially if you have taken discrete 1, MATH225,
you should try to prove
that for the general case of performing Erathothenes sieve on $n$,
the algorithm does find all the primes when the search for kill primes
terminate at $\sqrt{n}$.)

You should trace the above on paper, then write down the pseudocode,
and then translate into \cpp.
You are also strongly advised to create more test cases.

(Instead of using an integer array that contains
\verb!0!'s
and
\verb!1!'s,
another option is to use
an array of boolean values.)
