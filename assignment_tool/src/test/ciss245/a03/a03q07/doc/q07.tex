%-*-latex-*-
Write a chat bot.
The chat bot will learn to recognize the user.
Here's a test run:
\begin{Verbatim}[frame=single,commandchars=\\\{\}]
Hi, what is your name?
\userinput{My name is John.}
Hi John. How are you?
\end{Verbatim}

The chat bot must recognize the name of the user when given responses of the
following form:
\begin{tightlist}
  \li \verb!My name is John.!
  \li \verb!I am John.!
  \li \verb!I'm John.!
  \li \verb!John.!
\end{tightlist}
and also when the user enters extraneous spaces such as
\begin{tightlist}
  \li \verb!My     name  is       John.!
  \li \verb!I  am    John.!
  \li \verb!I'm   John.  !
  \li \verb!John.    !
\end{tightlist}
and also when the user forgot to enter the period (\verb!'.'!)
and when the user forgets to capitalize correctly such as
\begin{tightlist}
  \li \verb!my     name  is       john.!
  \li \verb!i  am    john.!
  \li \verb!i'm   john.  !
  \li \verb!john.    !
\end{tightlist}

It's helpful to write a \verb!str_capitalize()! function such that
\verb!str_capitalize(x)! will replace \verb!x[0]! with
its uppercase.
It's also useful to have a function that strips away
left trailing space, \verb!str_lstrip! (the left strip function) so that if you
execute
\verb!str_lstrip(x)!
where \verb!x! is
\verb!"   hello world   "!,
\verb!x! becomes
\verb!"hello world   "!.

It's also convenient to have a corresponding \verb!str_rstrip! (the right strip function).
If you
execute
\verb!str_rstrip(x)!
where \verb!x! is
\verb!"   hello world   "!,
\verb!x! becomes
\verb!"   hello world"!.

With the left and right strip functions, it's easy to create \verb!str_strip! (the strip function)
which performs both the left and right strip.
If you
execute
\verb!str_strip(x)!
where \verb!x! is
\verb!"   hello world   "!,
\verb!x! becomes
\verb!"hello world"!.
