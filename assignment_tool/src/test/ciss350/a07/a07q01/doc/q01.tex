%-*-latex-*-
[Introduction to Lexing]

If you're given a string such as
\[
\texttt{"10 + 21 * (33 - 47)"}
\]
which corresponds to a mathematical expression and you want to
evaluate it, you want to convert this into a sequence of
meaningful items:
\[
\texttt{"10", "+",  "21", "*", "(", "33", "-", "47", ")"}
\]
where each string in this sequence has meaning.
For instance \texttt{"10"} corresponds to the meaning of \lq\lq ten", etc.
The strings above (such as \texttt{"10"}) are called lexemes.

It is convenient to attach types to these lexemes.
For instance we want to think of \texttt{"10"} as a lexeme that
is different from the lexeme \texttt{"+"}.
Using object-oriented programming, we will convert these
lexemes into different token types.
For instance \texttt{"10"} will be converted into an
integer token object and
\texttt{"+"} will be converted into a plus token object.

Here are the classes:
{\small
\begin{Verbatim}[frame=single,fontsize=\footnotesize]
class Token
{};

class IntTok: public Token
{
public:
    int value;
};

class PlusTok: public Token
{};

class MinusTok: public Token
{};

class MultTok: public Tok
{};

class DivTok: public Token
{};

class LParenTok: public Token // Left parenthesis
{};

class RParenTok: public Token // Right parenthesis
{};
\end{Verbatim}
}
(You don't have to worry about the mod \verb!%! operator.)

The goal is to write a program that gets a string such as
\[
\texttt{"10 + 21 * (33 - 47)"}
\]
from the user and then produce a \verb!std::vector! of tokens.

You must provide code to print all the above tokens.
For instance
\begin{console}[fontsize=\footnotesize]
std::cout << IntTok(5) << '\n'; // Prints IntTok(5)
std::cout << PlusTok() << '\n'; // Prints PlusTok
\end{console}
You must also provide code to print a \verb!std::vector! of tokens.
\begin{console}[fontsize=\footnotesize]
std::vector< Token > toks;
toks.push_back(IntTok(5));
toks.push_back(PlusTok());
std::cout << toks << '\n'; // Prints [IntTok(5), PlusTok]
\end{console}
The above of course does not work.
You want \verb!toks! to be a vector of \textit{different types of tokens}.
But as it stands you can only put \verb!Token! objects into \verb!toks!.
See hints later.
(Look for the section
\textsc{A \textnormal{\texttt{std::vector}} containing different types of objects}).

Here's an execution of the program.
(Note that you will need to get a string input from the user
that includes spaces.
Check my CISS245 notes on strings.
See chapter 18 on characters and C-strings.)

\textsc{Test 1}
\begin{console}[commandchars=\\\{\},frame=single,fontsize=\scriptsize]
\underline{10 + 21 * (33 - 47)}
[IntTok(10), PlusTok, IntTok(21), MultTok, LParenTok, IntTok(33), MinusTok, IntTok(47), RParenTok]
\end{console}

Ignore \lq\lq negative of'' (the unary $-$).
In other words, when your program sees a $-$, generate a \verb!MinusTok!.

\textsc{Test 2}
{\scriptsize
\begin{console}[commandchars=\\\{\},frame=single,fontsize=\scriptsize]
\underline{10 + 21 * (33 - -47)}
[IntTok(10), PlusTok, IntTok(21), MultTok, LParenTok, IntTok(33), MinusTok, MinusTok, IntTok(47), RParenTok]
\end{console}
}

\textsc{Note.} In a compiler, the process of analysing a source file
is not broken up into lexing as a complete first step:
\begin{Verbatim}[frame=single,fontsize=\footnotesize]
get string s from program source file
get toks, a vector of tokens, from s (lexing)
analyze tokens (parsing)
\end{Verbatim}
The process actually extracts enough characters from a source file
to form a token and then pass the token to the part of the compiler that
analyzes the structure of the program. In other words, the compiler
alternates between lexing and analysis:
\begin{Verbatim}[fontsize=\footnotesize,frame=single]
get string s from program source file
loop as long as there are characters in s:
    extract enough characters from s to get the next token t
    continue analyzing structure of program with new token t 
\end{Verbatim}
So the process is more like lex-parse-lex-parse-lex-parse-...
The reason is because it's too costly to lex the whole string -- what if
there's a very early program error in an extremely huge source file?

\newpage
\textsc{A \textnormal{\texttt{std::vector}} containing different types of objects}

WARNING: INCOMING SPOILERS ...

If \verb!tok! is a \verb!std::vector< Token >!, then you can only put
\verb!Token! objects into \verb!tok!.

You might want to read and review my CISS245 notes on inheritance and polymorphism.

More spoilers on the next page ...

\newpage
continue ... WARNING: MORE INCOMING SPOILERS ...

If \verb!tok! is a \verb!std::vector< Token >!, then you can only put
\verb!Token! objects into \verb!tok!.

If you try to put a \verb!IntTok! object \verb!x!
into a \verb!std::vector< Token >!
object \verb!v!, say at index 0, then you are doing
\begin{Verbatim}[frame=single,fontsize=\footnotesize]
v[0] = x;
\end{Verbatim}
(Review CISS245 notes -- it's possible to assign a child object to a
parent object.)

If \verb!Token! is a concrete class (i.e., no pure virtual method),
sure you can do the above.
But \verb!v[0]! is a \verb!Token! object, not a \verb!IntTok! object.
For instance since \verb!x! is an \verb!IntTok! object \verb!x! will
hold an integer value.
But \verb!v[0]! is a \verb!Token! object and does not hold an
integer value.
So \verb!v[0]! will lose some critical information in \verb!x!.
In fact, you cannot even tell if
\verb!v[0]! came from an \verb!IntTok! or a \verb!PlusTok!.

However from CISS245, you know that this is possible:
\begin{Verbatim}[frame=single,fontsize=\footnotesize]
P * p = new C;
\end{Verbatim}
where \verb!P! is a parent class of class \verb!C!.
Therefore you can put \verb!p! into a \verb!std::vector< P * >!.
This \verb!p! can point to any subclass of \verb!P!.

Suppose \verb!P! is a parent class of classes \verb!C0! and \verb!C1!.
Suppose pointer \verb!p! is declared as
\begin{Verbatim}[frame=single,fontsize=\footnotesize]
P * p;
\end{Verbatim}
and \verb!p! is assigned an address from a \verb!std::vector< P * >! object.
Furthermore \verb!*p! is either a
\verb!C0! object or \verb!C1! object.

The above should be enough hints. (It pretty much gives away everything.)
So you probably want to review inheritance, polymorphism, virtual methods,
pure virtual methods, etc.

Also, if you like you can have a class for \verb!std::vector< Token * >!,
say call it \verb!Tokens!.
Note that if \verb!toks! is a \verb!Tokens! object,
you insert token pointers into \verb!toks!
to be processed \lq\lq at the back"
and you remove token pointers \lq\lq from the front".
So you might want to more or less think of your \verb!Tokens! objects
as queues (or double-ended queue).
