[Review of template class in CISS245 -- dynamic arrays]

This is a review of writing a class template with a pointer
and dynamic arrays in the heap.

The goal is to write a class to make \cpp\ arrays a lot easier to use.

Specifically, the \verb!vector! class is a class template
that provides most of \cpp's array features.
Study the skeleton code for \verb!vector! carefully.
Specifically, note that each \verb!vector! object has three members:
\verb!size_!, \verb!capacity_! and \verb!p_!.
Obviously \verb!p_! points to an array allocated in the free store,
\verb!capacity_! records the number of elements in the array, and
\verb!size_! records the number of elements in the array that are \lq\lq in
use".
The file \verb!test.cpp! includes some usage examples.
\verb!vector! is similar to our \verb!IntDynArray! from CISS245 except that
\verb!IntDynArray! is not a class template.
\verb!vector! is similiar to the \cpp\ STL \verb!std::vector! class
template.

Your goal is to complete the given
skeleton file \verb!vector.h!.
You are advised strongly to test your code.
In particular, you should add more test cases to \verb!test.cpp!;
what I've provided is absolutely not enough.
If you work for a really good company, you will see
that in some cases the test code can be longer than the code
being tested.

It should be clear what you need to complete just by looking at the
skeleton code.
All methods mentioned in the skeleton code must be completed.
(For your own learning, you may add other methods if you like.)
Let me just mention a few things.

\verb!vector! objects can be constructed by specifying a size:
\begin{console}[fontsize=\footnotesize]
vector< int > x(5);    // x models an int array of size 5
vector< double > y(4); // y models a double array of size 4
vector< char > x(5);   // x models a char array of size 3
\end{console}
The memory used is of course from the free store (or memory heap).
Specifically, pointer \verb!x.p_! will point to an array
of 5 integer in the free store and 
\verb!x.size_!
\verb!x.capacity_!
is set to 5.

After the above you can read the values of \lq\lq array'' \verb!x!
(or rather the values of the array that \verb!x.p_! points to):
\begin{console}[fontsize=\footnotesize]
std::cout << x[2] << '\n';
\end{console}
or you can write to the array:
{\small
\begin{console}
x[2] = 42;
\end{console}
}
This means that your class must have
\verb!operator[]! for reading and writing.
Since \verb!vector! objects model arrays in C/\cpp, this is not 
surprising.
However your \verb!operator[]! must provide protection
against invalid index values.
This means that if your \verb!vector! object has size 5,
then calling \verb!operator[]! outside of the
range of 0,1,2,3,4 will result in 
\verb!IndexError! object being thrown as an exception object.
The regular C/\cpp\ arrays do not have this protection
and is a very common programming error resulting not just in 
programs crashing but in fact can cause serious
security issues.

Of course there's the destructor, copy constructor, and assignment
operator.
Once you have memory allocation in your constructor(s), you must
overwrite these, right?

Another important constructor is
\begin{console}[fontsize=\footnotesize]
vector(T arr[], int size);
\end{console}
This allows you to construct a \verb!vector! object from a regular
C/\cpp\ array. For instance the above
constructor allows you to do the following
\begin{console}[fontsize=\footnotesize]
int x[] = {1,3,5,2,4,6};
vector< int > y(x, 6);
\end{console}
Otherwise you would have to do this:
\begin{console}[fontsize=\footnotesize]
int x[] = {1,3,5,2,4,6};
vector< int > y(6);
for (int i = 0; i < 6; i++)
{
    x[i] = y[i]; 
}
\end{console}

Code to print the vector is provided so that you can do this:
\begin{console}[fontsize=\footnotesize]
std::cout << x << '\n';
\end{console}
Study it carefully.
By the way, printing like the above is probably
among the earliest thing to write. 
How in the world do you debug if you can't print?!?
(Also, see the note below on debug printing.)

The comparison operator, \verb!operator==!, returns \verb!true!
exactly when the two relevant \verb!vector! objects
model the same array, i.e., if you call
\begin{console}[fontsize=\footnotesize]
x == y
\end{console}
where \verb!x! and \verb!y! are, say \verb!vector< char >! objects,
then you get \verb!true! 
exactly when their sizes are the same and the values in the arrays
that they point to have the same values.
This operator must also be coded very early in order to 
automate testing. 

Of course the assignment operator, \verb!operator=()!, does the obvious.
If you have
\begin{console}[fontsize=\footnotesize]
vector< double > x(5);
vector< double > y(10);
// code to put values into the arrays of x and y
\end{console}
then on calling the assignment operator:
\begin{console}[fontsize=\footnotesize]
x = y;
\end{console}
the arrays (in the free store) of both \verb!x! and \verb!y!
have the same values; of course the size of \verb!x! and \verb!y!
becomes the same.

All the rest should be obvious from reading the skeleton code.

The above is what you would expect to make \cpp's array more
user-friendly.

Now for the \verb!insert! method.
This is not directly available in your usual C/\cpp\ arrays.

Add a method \verb!insert(int index, const T & value)! so that 
you can insert a 
value into
the array at the given index position. 
This is the expected behavior (the following includes 3 test cases):
\begin{console}[fontsize=\footnotesize]
vector< int > a(3);
a[0] = 5;
a[1] = 10;
a[2] = -6;
std::cout << a << std::endl; // prints [5, 10, -6]

a.insert(0, 42);
std::cout << a << std::endl; // prints [42, 5, 10, -6]

a.insert(2, 999);
std::cout << a << std::endl; // prints [42, 5, 999, 10, -6]

a.insert(5, -11111);
std::cout << a << std::endl; // prints [42, 5, 999, 10, -6, -11111]
\end{console}

Note that if \verb!a! has size 5, you can only call \verb!insert!
for \verb!index = 0, 1, 2, 3, 4, 5!.
You cannot for instance insert at index value -1 and you cannot insert at
index value 6.
In general if the size of \verb!a! is \verb!s!, then you can 
only insert at \verb!index = 0, 1, 2, ..., s!.
If \verb!insert! is invoked at index outside this range your code
should throw an \verb!IndexError! exception object.

Here are test cases that test correct index positions:
\begin{Verbatim}[frame=single,fontsize=\footnotesize]
try
{
    vector< int > a(10);
    a.insert(0, 42);
    std::cout << "pass\n";
}
catch (IndexError & e)
{
    std::cout << "fail\n";
}

try
{
    vector< int > a(10);
    a.insert(5, 42);
    std::cout << "pass\n";
}
catch (IndexError & e)
{
    std::cout << "fail\n";
}

try
{
    vector< int > a(10);
    a.insert(10, 42);
    std::cout << "pass\n";
}
catch (IndexError & e)
{
    std::cout << "fail\n";
}

try
{
    vector< int > a(10);
    a.insert(10, 42);
    std::cout << "pass\n";
}
catch (IndexError & e)
{
    std::cout << "fail\n";
}
\end{Verbatim}

Here are some test cases that test invalid index positions:
\begin{Verbatim}[frame=single,fontsize=\footnotesize]
try
{
    vector< int > a(10);
    a.insert(-6, 42);
    std::cout << "fail\n";
}
catch (IndexError & e)
{
    std::cout << "pass\n";
}

try
{
    vector< int > a(10);
    a.insert(20, 42);
    std::cout << "fail\n";
}
catch (IndexError & e)
{
    std::cout << "pass\n";
}
\end{Verbatim}

Add a method to concatenate two \verb!vector! objects.
This involves overloading the \verb!operator+=! and \verb!operator+!.
For instance if \verb!a! and \verb!b! are \verb!vector< int >! objects,
then \verb!a += b! will result in the values of \verb!b! being appended
to \verb!a! (in the obvious way).
Note that \verb!operator+=! should return a reference to the object
invoking the operator.
In other words, your class should allow you to do this:
\begin{console}[fontsize=\footnotesize]
(a += b) += b;
\end{console}

The \verb!push_back! method adds a value to the end of the
array that your \verb!vector! models.
\begin{console}[frame=single,fontsize=\footnotesize]
int x[] = {1, 3, 5, 2, 4, 6};
vector< int > y(x, 6); // y is {1, 3, 5, 2, 4, 6}
y.push_back(42);       // y is {1, 3, 5, 2, 4, 6, 42}
\end{console}
Note that the \verb!y.size_! is changed from \verb!6! to \verb!7!.
If the array that \verb!y.p_! cannot accomodate 7 values
(i.e., if \verb!y.capacity_! is 6),
then you need to reallocate a larger array for \verb!y.p_! to
point to (just like the \verb!IntDynArray! example/assignment from CISS245).
In this case, when you reallocate, the array must be twice of
what you need.
For instance in the above example, since \verb!y.p_! needs 7 values,
the new array allocated must have 14 values.

The skeleton code is of course not correct. You have to work on it.
I do not need to tell you this but I'm going to anyway:
Make sure that you use pass-by-reference for object parameters
as much as possible.
Also, parameters and methods should be constant whenever possible.

(The \verb!IntDynArray! example/assignment from CISS245 and also
the \cpp\ STL \verb!std::vector! has more methods
than described above.
You need not implement these extra methods.)

Of course you have must ensure that there are no memory issues such as
memory leaks and double free errors.
You should definitely check your code with google asan.

\newpage
\textsc{Note on debug printing}

Frequently, it is helpful to have the ability to print everything
useful for debugging purposes.
Our \verb!operator<<! is only helpful when \textit{using} this
\verb!vector! class and debugging at a higher level assuming that
the \verb!vector! class is already working perfectly.
However for debugging \verb!vector!,
besides printing the values in the 
array, it might be helpful to print the \verb!size! and the pointer
\verb!p! as well.
I suggest you write a method
\begin{console}[fontsize=\footnotesize]
void debugprint() const;
\end{console}
to print everything associated to the object, i.e.,
print the \verb!size! and \verb!p! besides the array values.
This is very helpful especially in chasing pointers.
In more complex classes with pointers, you absolutely have to print the 
pointers for debugging.
This method is not required for this assignment.


\newpage
Skeleton for \verb!vector.h!:
{\footnotesize
\includesourcenonumbers{vector.h}
}

\newpage
Minimal \verb!test.cpp!:
{\footnotesize
\includesourcenonumbers{test.cpp}
}
