%-*-latex-*-
Now implement a Queue class.
Note that a queue is just a deque except that 
you can only add to the back and remove from the front.
Therefore the Queue class can just inherit from the Deque class.
Here's the skeleton:
\begin{console}[fontsize=\footnotesize]
// File: Queue.h
template < typename T >
class Queue
{
public:
    void enqueue(const T & x)
    {
        deque_.push_back(x);
    }
// etc.
private:
    Deque< T > deque_;
};
\end{console}
(I'm using composition instead of inheritance.)

You can do the following
\begin{console}[fontsize=\footnotesize]
Queue< int > queue;
queue.enqueue(5);          // 5 is at the front 
queue.enqueue(6);          // 5 is at the front and 6 is at the back 
int x = queue.front();     // x = 5
queue.front() = 4;         // front of queue is now 4
int y = queue.back();      // y = 6
queue.dequeue();           // queue has 1 value
int size = queue.size();   // size is 1
queue.clear();             // queue is now empty
bool b = queue.is_empty(); // b is true           <-- CORRECTION
queue.enqueue(1);
queue.enqueue(1);
queue.enqueue(1);
queue.clear();             // queue is now empty
\end{console}

You can print a Queue object.
The output is similar to previous questions -- just 
make sure you print the front first.

The Queue class is very easy once the Deque class is done.
So first make sure your Deque class is correct.
