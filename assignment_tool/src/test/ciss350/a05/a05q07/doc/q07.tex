%-*-latex-*-
Collect actual runtime data for each algorithm you have implemented in this
assignment
(bubblesort, bubblesort2, insertionsort, selectionsort,
mergesort, quicksort)
and create a graph.
See the section \textsc{Plotting Runtimes}
on drawing line graphs in \LaTeX.

For each algorithm you have implemented, collect data for 
vector of size 2000, 4000, 6000, 8000, ....
and for each size, you should use an average of about 5 cases.
Make sure you have enough data points to see the trend of your graph.
Of course the size of the array for sorting depends on your laptop.
Faster laptops will need larger array sizes, otherwise all the times
are too small to see the trend.
The values used in your vector should be integers --
see the skeleton code.
Make sure you have a legend for the plot.

Of course when collecting runtime data, you should set
\verb!verbose! to \verb!false! so that output is minimal.
This will ensure that the time collected is due to the sorting process
and not due to output.

SOLUTION:

[Open q07.tex and modify the graph below with your own data.]
\begin{python}
from latextool_basic import *
plot = FunctionPlot(width="3in", height="2in")
plot.add(((100, 1),
          (200, 2),
          (300, 0.5)),
          (1000, 2.5),
          (2000, 5.8),
          line_width='2', color='red', legend='bubblesort')

plot.add(((100, 5),
          (200, 3),
          (300, 7)),
          line_width='2', color='green', legend='bubblesort2')

plot.add(((100, 7),
          (200, 4),
          (300, 5)),
          line_width='2', color='blue', legend='selectionsort')
          
print(plot)
\end{python}

