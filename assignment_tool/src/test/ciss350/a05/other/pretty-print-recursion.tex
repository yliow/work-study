%-*-latex-*-
\textsc{Pretty-print Recursion}

For the verbose output, it's particularly important to know how to display
things in a nice format to help with debugging.
Because of the nature of function calling for recursive functions,
if you simply print everything straight to the stdout, things will be extremely
hard to read.
So let me show you a method that I use for verbose printing for debugging
recursive functions.
First here's the standard Fibonacci sequence. Run it:
\begin{Verbatim}[frame=single,fontsize=\footnotesize]
#include <iostream>

int f(int n)
{
    if (n < 2)
    {
        return 1;
    }
    else
    {
        return f(n - 1) + f(n - 2);
    }
}

int main()
{
    std::cout << "VALUE:" << f(3) << '\n';
    return 0;
}
\end{Verbatim}
It's simple enough that you won't get it wrong.
But suppose for debugging you want to show when you first enter the function and the output from each function.
Run this: 
\begin{Verbatim}[frame=single,fontsize=\footnotesize]
#include <iostream>

int f(int n)
{
    std::cout << "f(" << n << ") enter ...\n";
    if (n < 2)
    {
        int ret = 1;
        std::cout << "f(" << n << ") = " << ret << '\n';
        return ret;
    }
    else
    {
        int ret = f(n - 1) + f(n - 2);
        std::cout << "f(" << n << ") = " << ret << '\n';
        return ret;
    }
}

int main()
{
    std::cout << "VALUE:" << f(3) << '\n';
    return 0;
}
\end{Verbatim}
You'll see that it's rather confusing even for such a small recursive function.
Finally run this:
{\footnotesize
\includesourcenonumbers{pretty-print-recursion.cpp}
}

Here's the output
\begin{Verbatim}[frame=single,fontsize=\footnotesize]
f(3) enter ...
    f(2) enter ...
        f(1) enter ...
        f(1) = 1
        f(0) enter ...
        f(0) = 1
    f(2) = 2
    f(1) enter ...
    f(1) = 1
f(3) = 3
VALUE:3
\end{Verbatim}

Study the output and you'll see the point.
Make sure you study the above code -- it's very important because
I don't see this method of debug-printing for recursion ever mentioned in
beginner's C\texttt{++} books.
Note that I've used a static \verb!int! which you should review 
if you don't recall the point of a static variable.
You will find the above method extremely useful in the future whenever you
have to debug recursion.
