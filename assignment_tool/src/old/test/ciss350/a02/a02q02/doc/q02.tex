[Review of CISS245]

The following is a generalization of Q1. 
Given positive integers $a$, $b$, $d$, we define function $f$ as
follows:
\begin{enumerate}
\item If $n$ is divisible by $d$, then
\[
f(n) = n/d
\]
\item Otherwise
\[
f(n) = an + b
\]
\end{enumerate}
(This is a generalization of the function $f$ in Q1: 
if $a=3, b=1, d=2$, you get the function in Q1.)

Write a program that accepts positive integer values for $a, b, d, n$, and 
$N$ (I'll explain $N$ in a bit) and
does the following.

First, if you can reach 1 after at most $N$ applications of function $f$, 
you print $a, b, d, n, N$, the string \verb!pass!, 
the number of tries, 
and the list of numbers obtained. For instance:
\begin{console}[frame=single,fontsize=\footnotesize, commandchars=\\\{\}]
\userinput{3 1 2 7 1000}
3 1 2 7 1000 pass 16 7 22 11 34 17 52 26 13 40 20 10 5 16 8 4 2 1
\end{console}

Second, if you cannot reach 1 after $N$ tries, 
you print the same information as above except that
you print \verb!fail! instead of \texttt{pass}. For instance:
\begin{console}[frame=single,fontsize=\footnotesize, commandchars=\\\{\}]
\userinput{3 1 2 7 10}
3 1 2 7 10 fail 10 7 22 11 34 17 52 26 13 40 20 10
\end{console}

Note that in this case after applying this particular $f$ 10 times to 7, 
you have 11 numbers (including 7)
and the last is not 1 (it's 10).

In this case you must write a (\cpp) function $f$ that accepts 
$a, b, d$ in such a way that the $f$ function call
in Q1 still works, i.e. your function prototype must be
\[
\verb!int f(int n, int a=3, int b=1, int d=2);!
\]
There is another function \verb!fs! that accepts a pointer to int array,
\[
\text{\texttt{
int fs(int * \& p, int n, int a=3, int b=1, int d=2, int N=1);
}}
\]
allocates an integer array of size $N$ to $p$. 
Note that the value of $n$ is not stored in this array. 
The return value is the number of integers placed in the array. 
For instance when $a=3,b=1,d=2,n=7, N=1000$ 
(see above test case), 
there are 17 values (including $n=7$). 
The values stored in the array \verb!p! points to are
{\small
\begin{console}[frame=single]
22 11 34 17 52 26 13 40 20 10 5 16 8 4 2 1
\end{console}
}
(i.e. 7 is not included) and hence the return value in this case is 16.

Make sure you deallocate memory \verb!p! points to before program halts.

\textsc{Test 1}
\begin{console}[frame=single,fontsize=\footnotesize, commandchars=\\\{\}]
\userinput{3 1 2 7 1000}
3 1 2 7 1000 pass 16 7 22 11 34 17 52 26 13 40 20 10 5 16 8 4 2 1
\end{console}

\textsc{Test 2}
\begin{console}[frame=single,fontsize=\footnotesize, commandchars=\\\{\}]
\userinput{3 1 2 7 10}
3 1 2 7 10 fail 10 7 22 11 34 17 52 26 13 40 20 10
\end{console}

\textsc{Test 3}
\begin{console}[frame=single,fontsize=\footnotesize,commandchars=\\\{\}]
\userinput{7 5 3 7 10}
7 5 3 7 10 fail 10 7 54 18 6 2 19 138 46 327 109 768
\end{console}

In the case of Test 3 the function $f$ is defined as
\begin{enumerate}
\item If $n$ is divisible by 3, then $f(n) = n/3$
\item If $n$ is not divisible by 3, then $f(n) = 7n + 5$
\end{enumerate}

Therefore we have
\begin{align*}
f(7)   &= 7 \times 7+5 = 54 \\
f(54)  &= 54/3 = 18 \\
f(18)  &= 18/3 = 6 \\
f(6)   &= 6/3 = 2 \\
f(2)   &= 2 \times 7+5 = 19 \\
f(19)  &= 3 \times 19+5 = 138 \\
f(138) &= 138/3 = 46 \\
f(46)  &= 7 \times 46+5 = 327 \\
f(327) &= 327/3 = 109 \\
f(109) &= 7 \times 109+5 = 768 \\
\end{align*}
