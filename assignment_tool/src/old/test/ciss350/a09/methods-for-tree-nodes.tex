%-*-latex-*-
\textsc{Methods for tree nodes}

Now I'm going to describe all the methods/functions/operators
that you must implement for all tree node classes.
First let me talk about the instance variables.

Each node class has a \verb!key_! member and a \verb!parent_! 
pointer. (Recall that there are situations where a parent pointer
is not necesary, but for this assignment, we will include parent
pointers.)
Besides that, we need to talk about the member variables
for the children pointers.
The following are the options that we'll deal with for this assignment.

The class that uses \verb!std::vector! of pointers for children looks like
this:
\begin{console}[commandchars=\~\!\@,fontsize=\footnotesize]
template < typename T >
class TreeNodev
{
public:
    const T & key() const
    {
        return key_;
    }
    T & key()
    {
        return key_;
    }
private:
    T key_;
    TreeNodev * parent_;
    std::vector< TreeNodev * > child_;
};
\end{console}
(Memory aid: \verb!TreeNodev! with \verb!v = vector!.)

The class that uses \verb!std::list! looks like this:
\begin{console}[fontsize=\footnotesize]
template < typename T >
class TreeNodel
{
private:
    T key_;
    TreeNodel * parent_;
    std::list< TreeNodel * > child_;
};
\end{console}
(Memory aid: \verb!TreeNodel! with \verb!l = list!.)

The class for binary tree node looks like this:
\begin{console}[fontsize=\footnotesize]
template < typename T >
class BinaryTreeNode
{
private:
    T key_;
    BinaryTreeNode * parent_;
    BinaryTreeNode * left_;
    BinaryTreeNode * right_;
};
\end{console}

The following is a list of methods that must be supported by
all the tree node classes.
In the following, let \verb!node! be an object
from any of the tree node classes.
There are also corresponding functions that do the same thing.
except that the functions accept pointers. For instance there's a
\verb!height! \textit{function} that accepts a pointer to a node object;
the return value is of course an \verb!int!.
In other words the prototype of the \verb!height! \textit{function} 
looks like this:
\begin{Verbatim}[frame=single,fontsize=\footnotesize]
template < typename T >
int height(const TreeNodev< T > * const p);
\end{Verbatim}

Of course each node must be able to return the \verb!key_! as a reference:
\begin{Verbatim}[frame=single,fontsize=\footnotesize]
node.key()           key_ as a reference 
\end{Verbatim}
If the \texttt{node} is constant, the reference returned is a constant.
This has nothing to do with trees.


Now for the common methods for all tree node class.
\begin{Verbatim}[frame=single,fontsize=\footnotesize]
node.is_root()       true iff node is a root
node.is_leaf()       true iff node is a leaf
node.is_nonleaf()    true iff node is non-leaf
node.height()        height of node
node.depth()         depth of node (note: level and depth are the same)
node.size()          number of descendents + 1

node.parent()        pointer to parent (returned as a reference)
node.root()          pointer to root

node.next()          pointer to next sibling, 
                     NULL if there's no next sibling. In the case of binary 
                     tree node, assume left pointer to the 0--th child and 
                     right points to the 1-st child..
node.prev()          pointer to previous sibling,
                     NULL if there's no next sibling. In the case of binary 
                     tree node, assume left pointer to the 0--th child and 
                     right points to the 1-st child.

node.num_children()  number of children.
node.first_child()   pointer to first child (can be NULL)
node.last_child()    pointer to last child (can be NULL)
node.child(i)        pointer to i-th child (i = 0 is the first, can be 
                     NULL)
node.left()          left pointer (only for case of binary tree node)
node.right()         right pointer (only for case of binary tree node)
node.leftmost()      pointer to leftmost child
node.rightmost()     pointer to rightmost child

node.insert(i, data) create a new node n, set n.key_ to data and attach n 
                     as i-th child of node. Note that this might require
                     adding NULL pointers to fill up to child i - 1.
                     If there is already an i-th child, ValueError 
                     exception object is thrown. 
node.insert_parent(data)
                     create a new node n, set n.key_ to data and attach n 
                     as parent. If there is already a parent, a ValueError
                     exception object is thrown.
node.insert_left(data)
                     create a new node n, set n.key_ to data and attach n 
                     as left child of node.
                     If there is already a left child (i.e., left pointer
                     is not NULL) ValueError exception object is thrown.
                     (Only for case of binary tree node.)
node.insert_right(data)
                     create a new node n, set n.key_ to data and attach n 
                     as right child of node.
                     If there is already a right child (i.e., right pointer
                     is not NULL) ValueError exception object is thrown.
                     (Only for case of binary tree node.)

node.remove(i)       remove the subtree at child i.
node.remove_left()   remove the subtree at left child 
                     (Only for case of binary tree node.)
node.remove_right()  remove the subtree at right child 
                     (Only for case of binary tree node.)
\end{Verbatim}

In case of an error, a \verb!ValueError! exception object is thrown.
The following is the \verb!ValueError! class:
\begin{Verbatim}[frame=single,fontsize=\footnotesize]
class ValueError
{};
\end{Verbatim}

Of course the classes also have obvious methods
such as constructors, destructors, 
\verb!operator=()!, \verb!operator==()!, \verb@operator!=()@.
This is also the case for any tree class.

Note that for copy constructor and \verb!operator=()!, 
make sure that objects are actually created.
In other words, suppose a node \verb!node2!
has exactly a child pointer that is pointing
to a node (i.e., it is not \verb!NULL!).
Then after the copy constructor 
\begin{Verbatim}[frame=single,fontsize=\footnotesize]
node1 = node2;
\end{Verbatim}
is called, \verb!node1! also has exactly one child pointer
that points a node.

When the destructor is called, all the memory used must be 
deallocated
correctly. This applies to the whole tree structure, i.e.,
the memory used by all descendents.
Note that this implies that the memory used by a child must be 
deallocated before memory used by a parent is deallocated.

Of course 
\begin{Verbatim}[frame=single,fontsize=\footnotesize]
node1 == node2
\end{Verbatim}
returns \verb!true! exactly when \verb!node1.key_!
has the same value as \verb!node2.key_!,
both have the same tree structure, and the corresponding
descendents have the same \verb!key_! values as well.
Otherwise \verb!false! is returned.

With the above methods, if \verb!node! is a tree node object,
then one can for instance print the children of \verb!node! like this:
\begin{Verbatim}[frame=single,fontsize=\footnotesize]
for (int i = 0; child.child(i) != NULL; i++)
{
    std::cout << *(node.child(i)) << std::endl;
}
\end{Verbatim}
or this 
\begin{Verbatim}[frame=single,fontsize=\footnotesize]
p = node.first_child()
while (p != NULL)
{
    std::cout << *p << std::endl;
    p = p->next();
}
\end{Verbatim}
where \verb!p! is a pointer of the appropriate type.

With the above methods/operators/functions,
one can quickly build a binary tree
\begin{python}
from latextool_basic import *
print(r"""
\begin{center}
%s
\end{center}
""" % graph(yscale=0.9,layout='''
   A
 B   C
D E F G
''',
minimum_size='8mm',
edges='A-B,A-C,B-D,B-E,C-F,C-G',
A=r'label=\texttt{+}',
B=r'label=\texttt{*}',
C=r'label=\texttt{-}',
D=r'label=\texttt{1}',
E=r'label=\texttt{2}',
F=r'label=\texttt{3}',
G=r'label=\texttt{4}')
\end{python}
like this:
\begin{Verbatim}[frame=single,fontsize=\footnotesize]
BinaryTreeNode< std::string > * p = new BinaryTreeNode("+");
p->insert_left("*");
p->insert_right("-");

p->left()->insert_left("1");
p->left()->insert_right("2");

p->right()->insert_left("3");
p->right()->insert_right("4");
\end{Verbatim}
Using the functions instead, we would do this:
\begin{Verbatim}[frame=single,fontsize=\footnotesize]
BinaryTreeNode< std::string > * p = new BinaryTreeNode("+");
insert_left(p, "*");
insert_right(p, "-");

insert_left(left(p), "1");
insert_right(left(p), "2");

insert_left(left(p), "3");
insert_right(right(p), "4");
\end{Verbatim}
