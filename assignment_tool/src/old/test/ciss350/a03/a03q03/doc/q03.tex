Write a recursive linear search function that implements
a linear search recursively.
You must use the following skeleton code which
contains tracing of your function.
The recursive linear search function below returns
the first index \verb!i! where \verb!x[i]! is \verb!target!
in the array \verb!x!.
Otherwise \verb!-1! is returned.

You must use a recursive relation between
\verb!linearsearch_rec(x, start, end, target)!
and
\verb!linearsearch_rec(x, start + 1, end, target)!
when \verb!n! is \verb!start! is $<$ \verb!end - 1!.
Basically,
\\
\verb!linearsearch_rec(x, start, end, target)!
performs a linear search on
\\
\verb!x[start]!,
\verb!x[start + 1]!,
\verb!x[start + 2]!, ...,
\verb!x[end - 1]!,

\begin{Verbatim}[frame=single, fontsize=\small,commandchars=\~\!\@]
#include <iostream>

int linearsearch_rec(int x[], int start, int end, int target)
{
    if (start == end)
    {
        int ret = -99999; // set ret to correct value
        std::cout << "linearsearch_rec(x, " << start << ", " << end << ", "
                  << target << ") base case ... "
                  << "return " << ret << '\n';
        return ret;
    }
    else
    {
        std::cout << "linearsearch_rec(x, " << start << ", " << end << ", "
                  << target << ") recursive case ...\n";

        int ret = -99999; // set ret to correct value





        std::cout << "linearsearch_rec(x, " << start << ", " << end << ", "
                  << target << ") recursive case ... "
                  << "return " << ret << '\n';
        return ret;        
    }
}

int main()
{
    int n, target;
    std::cin >> n;
    int * x = new int[n];
    for (int i = 0; i < n; ++i)
    {
        std::cin >> x[i];
    }
    std::cin >> target;
    int index = linearsearch_rec(x, 0, n, target);
    std::cout << index << '\n';
    delete [] x;
    return 0;
}
\end{Verbatim}

\textsc{Test 1}
\begin{Verbatim}[commandchars=\\\{\}, fontsize=\small, frame=single]
\underline{5}
\underline{1 3 2 5 4}
\underline{2}
linearsearch_rec(x, 0, 5, 2) recursive case ...
linearsearch_rec(x, 1, 5, 2) recursive case ...
linearsearch_rec(x, 2, 5, 2) recursive case ...
linearsearch_rec(x, 2, 5, 2) recursive case ...return 2
linearsearch_rec(x, 1, 5, 2) recursive case ...return 2
linearsearch_rec(x, 0, 5, 2) recursive case ...return 2
2
\end{Verbatim}
(Clearly the inputs are the size of the array, the array, and the target.)

\textsc{Test 2}
\begin{Verbatim}[commandchars=\\\{\}, fontsize=\small, frame=single]
\underline{5}
\underline{1 3 2 5 4}
\underline{4}
linearsearch_rec(x, 0, 5, 4) recursive case ...
linearsearch_rec(x, 1, 5, 4) recursive case ...
linearsearch_rec(x, 2, 5, 4) recursive case ...
linearsearch_rec(x, 3, 5, 4) recursive case ...
linearsearch_rec(x, 4, 5, 4) recursive case ...
linearsearch_rec(x, 4, 5, 4) recursive case ...return 4
linearsearch_rec(x, 3, 5, 4) recursive case ...return 4
linearsearch_rec(x, 2, 5, 4) recursive case ...return 4
linearsearch_rec(x, 1, 5, 4) recursive case ...return 4
linearsearch_rec(x, 0, 5, 4) recursive case ...return 4
4
\end{Verbatim}

\textsc{Test 4}
\begin{Verbatim}[commandchars=\\\{\}, fontsize=\small, frame=single]
\underline{5}
\underline{3 1 4 5 2}
\underline{3}
linearsearch_rec(x, 0, 5, 3) recursive case ...
linearsearch_rec(x, 0, 5, 3) recursive case ...return 0
0
\end{Verbatim}

\textsc{Test 5}
\begin{Verbatim}[commandchars=\\\{\}, fontsize=\small, frame=single]
\underline{5}
\underline{5 4 3 2 1}
\underline{0}
linearsearch_rec(x, 0, 5, 0) recursive case ...
linearsearch_rec(x, 1, 5, 0) recursive case ...
linearsearch_rec(x, 2, 5, 0) recursive case ...
linearsearch_rec(x, 3, 5, 0) recursive case ...
linearsearch_rec(x, 4, 5, 0) recursive case ...
linearsearch_rec(x, 5, 5, 0) base case ...return -1
linearsearch_rec(x, 4, 5, 0) recursive case ...return -1
linearsearch_rec(x, 3, 5, 0) recursive case ...return -1
linearsearch_rec(x, 2, 5, 0) recursive case ...return -1
linearsearch_rec(x, 1, 5, 0) recursive case ...return -1
linearsearch_rec(x, 0, 5, 0) recursive case ...return -1
-1
\end{Verbatim}

Make sure you create your own test cases as well.
