%-*-latex-*-
Vectors are extremely important.
A 2-dimensional vector in math looks like this
\[
\langle 2, 3 \rangle
\]
You can think a vector as representing motion.
For the above vector, you can think of that
vector as a motion of an object that moves
by 2 units along the $x$-axis and
by 3 units along the $y$-axis.

\begin{python}
from latextool_basic import *
p, q = (0,0), (2,3)
print(plot(vector=[p,q]))
\end{python}

Clearly we can model the above vector
$\langle 2, 3 \rangle$
in \cpp\
using two variables with values $2$ and $3$ respectively.
\begin{console}
int x = 2, y = 3;
std::cout << '<' << x << ", " << y << '>' << std::endl;
\end{console}

Vectors appear in many sciences. 
In particular, they appear in Physics.
Therefore in CS they are 
extremely important in computer simulations of moving physical objects, whether
you are talking about rockets, planes, or cars.
And of course there are moving objects in games
and so games use a lot of vectors.
But it's more than that.
Vectors are used to model light particles which allows
computer software to render lighting and shadows of object simulations.
In terms of careers, there's no doubt that if you want to work in
space exploration (SpaceX, NASA) or self-driving cars (Tesla, Waymo,
Mobileye) or aerospace/aviation (Lockheed-Martin, Boeing, Northrop Grumman)
or computer graphics (NVIDIA, Sony, Industrial Light and Magic), you
will need to know vectors.

One important operation on vectors is vector addition.
Here's an example: 
\[
\langle 2, 3 \rangle
+
\langle 1, 4 \rangle
=
\langle 3, 7 \rangle
=
\langle 2+1, 3+4 \rangle
\]
In other words, you just add the values according to their
position or coordinate place.
Here's another example:
\[
\langle 5, -1 \rangle
+
\langle 2, 3 \rangle
=
\langle 7, 2 \rangle
\]


Write a function \verb!vec2d_add()! that adds two 2-dimensional vectors
of integer values.
I have provided a version that uses references for parameters.
Note that the two versions have the same function name.
That is perfectly fine for the compiler since their function
signatures are different -- the function name \verb!vec2d_add!
is overloaded.

\begin{Verbatim}[frame=single]
#include <iostream>

// version of vec2d_add that uses references
void vec2d_add(int & x0, int & y0, 
               int x1, int y1,
               int x2, int y2)
{
    x0 = x1 + x2;
    y0 = y1 + y2;
}


// version of vec2d_add that uses pointers
void vec2d_add(int * x0, int * y0, 
               int x1, int y1,
               int x2, int y2)
{
    // TO BE COMPLETED.
}


void vec2d_println(int x, int y)
{
    std::cout << '<' << x << ", " << y << ">\n";
}


int main()
{
    int x0 = 0, y0 = 0, x1, y1, x2, y2;
    std::cin >> x1 >> y1 >> x2 >> y2;

    int old_x0 = x0, old_y0 = y0;

    vec2d_add(x0, y0, x1, y1, x2, y2);
    vec2d_println(x0, y0);

    x0 = old_x0;
    y0 = old_y0;

    vec2d_add(&x0, &y0, x1, y1, x2, y2);
    vec2d_println(x0, y0);

    return 0;
}
\end{Verbatim}

\textsc{Test 1}
\begin{console}[commandchars=\\\{\}]
\underline{\texttt{1 2 3 4}}
<4, 6>
<4, 6>
\end{console}
