%-*-latex-*-
[Rising hills]
ASCII art.
(Do not use arrays.)

The test cases explains what you need to do.
Make sure the drawing is done by a function with the following prototype:
\[
\texttt{void draw\_rising\_hills(int);}
\]

\textsc{Test 1}
\begin{console}[commandchars=\\\{\}]
\userinput{1}
*
\end{console}

\textsc{Test 2}
\begin{console}[commandchars=\\\{\}]
\userinput{2}
  *
****
\end{console}

\textsc{Test 3}
\begin{console}[commandchars=\\\{\}]
\userinput{3}
      *
  *  ***
*********
\end{console}

\textsc{Test 4}
\begin{console}[commandchars=\\\{\}]
\userinput{4}
            *
      *    ***
  *  ***  *****
****************
\end{console}

\textsc{Test 5}
\begin{console}[commandchars=\\\{\}]
\userinput{5}
                    *
            *      ***
      *    ***    *****
  *  ***  *****  *******
*************************
\end{console}

\textsc{Test 6}
\begin{console}[commandchars=\\\{\}]
\userinput{6}
                              *     
                    *        ***    
            *      ***      *****   
      *    ***    *****    *******  
  *  ***  *****  *******  ********* 
************************************
\end{console}


WARNING: ... INCOMING SPOILERS ...
Hints on the next page.

\newpage
\textsc{Hints}

Look at the test case when the input is $n = 6$:

\begin{python}
from latextool_basic import *
p = Plot()

p += Rect(x0=0.04, y0=0, x1=7, y1=3, linewidth=0, s = \
r'''\begin{Verbatim}
                              *     
                    *        ***    
            *      ***      *****   
      *    ***    *****    *******  
  *  ***  *****  *******  ********* 
************************************
\end{Verbatim}''')

p += Rect(x0=0.03, y0=0.4, x1=7.35, y1=3.03, linewidth=0.01)
print(p)
\end{python}

You might want to think of this:

\begin{python}
from latextool_basic import *
p = Plot()

p += Rect(x0=0.04, y0=0, x1=7, y1=3, linewidth=0, s = \
r'''\begin{Verbatim}
                              *     
                    *        ***    
            *      ***      *****   
      *    ***    *****    *******  
  *  ***  *****  *******  ********* 
************************************
\end{Verbatim}''')

p += Rect(x0=0.03, y0=0.4, x1=7.35, y1=3.04, linewidth=0.01)

d = 0.2435
p += Line(points=[(d,0.4),(d,3.04)], linewidth=0.01)
x = 0.839
p += Line(points=[(x,0.4),(x,3.04)], linewidth=0.01)
x = 1.86
p += Line(points=[(x,0.4),(x,3.04)], linewidth=0.01)
x = 3.27
p += Line(points=[(x,0.4),(x,3.04)], linewidth=0.01)
x = 5.093
p += Line(points=[(x,0.4),(x,3.04)], linewidth=0.01)
print(p)
\end{python}


Why is this helpful?
Because if you want to draw the first star (at the first line of output):
\begin{python}
from latextool_basic import *
p = Plot()

p += Rect(x0=0.04, y0=0, x1=7, y1=3, linewidth=0, s = \
r'''\begin{Verbatim}[commandchars=\\\{\}]
                              \textred{\underline{*}}     
                    *        ***    
            *      ***      *****   
      *    ***    *****    *******  
  *  ***  *****  *******  ********* 
************************************
\end{Verbatim}''')

p += Rect(x0=0.03, y0=0.4, x1=7.35, y1=3.04, linewidth=0.01)

d = 0.2435
p += Line(points=[(d,0.4),(d,3.04)], linewidth=0.01)
x = 0.839
p += Line(points=[(x,0.4),(x,3.04)], linewidth=0.01)
x = 1.86
p += Line(points=[(x,0.4),(x,3.04)], linewidth=0.01)
x = 3.27
p += Line(points=[(x,0.4),(x,3.04)], linewidth=0.01)
x = 5.093
p += Line(points=[(x,0.4),(x,3.04)], linewidth=0.01)
print(p)
\end{python}

you would need to print a bunch of spaces:

\begin{python}
from latextool_basic import *
p = Plot()

p += Rect(x0=0.04, y0=0, x1=7, y1=3, linewidth=0, s = \
r'''\begin{Verbatim}[commandchars=\\\{\}]
                              \textred{\underline{*}}     
                    *        ***    
            *      ***      *****   
      *    ***    *****    *******  
  *  ***  *****  *******  ********* 
************************************
\end{Verbatim}''')

p += Rect(x0=0.03, y0=0.4, x1=7.35, y1=3.04, linewidth=0.01)

d = 0.2435
p += Line(points=[(d,0.4),(d,3.04)], linewidth=0.01)
x = 0.839
p += Line(points=[(x,0.4),(x,3.04)], linewidth=0.01)
x = 1.86
p += Line(points=[(x,0.4),(x,3.04)], linewidth=0.01)
x = 3.27
p += Line(points=[(x,0.4),(x,3.04)], linewidth=0.01)
x = 5.093
p += Line(points=[(x,0.4),(x,3.04)], linewidth=0.01)

p += Line(points=[(0.03,2.94),(6.1,2.94)], linewidth=0.05, startstyle='<', endstyle='>')


print(p)
\end{python}

How many spaces?
For the case when the user input is $n = 6$, the number of spaces
is the base of the rectangles containing the hills
before the largest hill \#6, plus roughly half of the base of hill \#6.
Right?
The first hill has base of length 1, the second hill has base of length 3,
the third hill has base of length 5, etc.

For the printing of the second line of output, you will also need to
use roughly the same idea when you print the top of the second-to-last
hill.
