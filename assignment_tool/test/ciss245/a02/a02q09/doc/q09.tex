Q9. [Brute force factorization of rational polynomial]

This question uses the Fraction library you built in Q1-Q8.

The goal is to write a program that attempts to factorize a polynomial of degree
3. The polynomial has fractional coefficients.
We will assume the coefficient of the highest degree term is 1.
Here's an example:
\[
  x^3 - \frac{13}{12} x^2 + \frac{3}{8}x - \frac{1}{24}
\]
Now it turns out that the above polynomial is the same as
\[
  \left( x - \frac{1}{2} \right)
  \left( x - \frac{1}{3} \right)
  \left( x - \frac{1}{4} \right)
\]
We also say that 1/2, 1/3, 1/4 are roots of the above given polynomial.
(This is recorded as Test 1 below.)

Here's another example:
\[
  x^3 - \frac{25}{12}x^2 + \frac{13}{9}x - \frac{1}{3} =
  \left( x - \frac{2}{3} \right)
  \left( x - \frac{2}{3} \right)
  \left( x - \frac{3}{4} \right)  
\]

Our goal is this: If the user gives you fractions $a,b,c$
\[
  x^3 + ax^2 + bx + c
\]
you need to compute fraction (if possible) $d,e,f$ such that
\[
  x^3 + ax^2 + bx + c = (x - d)(x - e)(x - f)
\]
We will do this by brute force:
We will let $d$ run over a range of fractions, $e$ run over a range of fractions,
and $f$ run over a range of fractions.
Now find $d,e,f$ are fractions, they are made up of numerators and denominators.
I'll let
$d$ be \verb!dn/dd!,
$e$ be \verb!en/ed!,
$f$ be \verb!fn/fd!.
Likewise let 
$a$ be \verb!an/ad!,
$b$ be \verb!bn/bd!,
$c$ be \verb!cn/cd!.
Altogether we have 6 inputs and 6 outputs.
We will let the numerators
\verb!dn!,
\verb!en!,
\verb!fn!
run through the range $[-50, 50]$ and
\verb!dd!,
\verb!ed!,
\verb!fd!
denominators run through the same $[1, 50]$.
So your code structure to enumerate fraction \verb!d! in the above ranges is:
\begin{console}
for dn = -50, ..., 50:
    for dd = 1, ..., 50:
        ... 
\end{console}
Of course the enumeration of all fractions \verb!d! and \verb!f! would look like:      
\begin{console}
for dn = -50, ..., 50:
    for dd = 1, ..., 50:
        for en = -50, ..., 50:
            for ed = 1, ..., 50:
                ... 
\end{console}
The enumeration of each fraction requires two for-loops.
So what do you do in the innermost body of the 6 for-loops:
\begin{console}
for loops over d, e, f: (6 loops)
         if (x-d)(x-e)(x-f) is x^3 + ax^2 + bx + d:
             print (x-d)(x-e)(x-f)
\end{console}
But how do you check
\verb!(x-d)(x-e)(x-f) is x^3 + ax^2 + bx + d! since \texttt{C++}
cannot do such comparison.
\texttt{C++}
don't even understand polynomials since there's no such thing as a polynomial type.

Well it's not that bad.
Since you can only work with integers, and now fractions since you have a small fraction
library, you have to work directly with coefficients.
Note that, using college algebra
\begin{align*}
  (x-d)(x-e)(x-f)
  &= (x^2 - (d + e) x + de)(x - f) \\
  &= x^3 - (d + e + f) x^2 + (de + df + ef) - (def)
\end{align*}
Therefore to say
\[
  (x-d)(x-e)(x-f) = x^3 + ax^2 + bx + c
\]
is the same as saying
\[
  x^3 - (d + e + f) x^2 + (de + df + ef) - (def) = x^3 + ax^2 + bx + c
\]
which is the same as saying
\begin{align*}
  -(d + e + f) &= a \\
  de + df + ef &= b \\
  - (def) &=  c
\end{align*}

[OPTIMIZATION:
From
\begin{align*}
  -(d + e + f) &= a \\
  de + df + ef &= b \\
  - (def) &=  c
\end{align*}
once $d,e$ are given, $f$ is determined:
\begin{align*}
  f &= -a - d - e\\
  f &= \frac{b - de}{d + e}\\
  f &=  -\frac{c}{de}
\end{align*}

]

These are three boolean equality comparisons of fractions (and you have the \verb!Fraction_eq! function)
and the fractions might involve fraction operations (mainly addition and multiplication).
The only operator you need which is not implemented in the previous questions is the
\lq\lq negative of''.
So you might want to add a \verb!Fraction_neg! function.
The general idea is this:
\begin{console}
for loops over d, e, f: (6 loops)
         compute fraction x = -(d + e + f)
         compute fraction y = d * e + d * f + e * f
         compute fraction z = -(d * e * f)
         if x is a and y is b and z is c,
             print (x-d)(x-e)(x-f)
\end{console}
One last thing: you must make sure you organize \verb!d,e,f! so that they are in ascending order:
\verb!d!
$\leq$
\verb!e!
$\leq$
\verb!f!.

If no factorization is found (either the polynomial cannot be factored or the
the roots are outside our range), you print \verb!not found!.

\textsc{Test 1}
\begin{console}[commandchars=\\\{\}]
\underline{\texttt{-13 12 3 8 -1 24}}
x^3 + (-13/12)x^2 + (3/8)x + (-1/24) = (x - (1/4))(x - (1/3))(x - (1/2))
\end{console}
Note: The roots on the right must be listed in ascending order.


\textsc{Test 2}
\begin{console}[commandchars=\\\{\}]
\underline{\texttt{-26 15 8 15 0 1}}
x^3 + (-26/15)x^2 + (8/15)x + (0) = (x - (0))(x - (2/5))(x - (4/3))
\end{console}

\textsc{Test 3}
\begin{console}[commandchars=\\\{\}]
\underline{\texttt{-6 11 0 1 0 1}}
x^3 + (-6/11)x^2 + (0)x + (0) = (x - (0))(x - (0))(x - (6/11))
\end{console}


\textsc{Test 4}
\begin{console}[commandchars=\\\{\}]
\underline{\texttt{-19 2 27 1 -45 2}}
x^3 + (-19/2)x^2 + (27)x + (-45/2) = (x - (3/2))(x - (3))(x - (5))
\end{console}


\textsc{Test 5}
\begin{console}[fontsize=\small,commandchars=\\\{\}]
\underline{\texttt{-244 105 188 105 -16 35}}
x^3 + (-244/105)x^2 + (188/105)x + (-16/35) = (x - (2/3))(x - (4/5))(x - (6/7))
\end{console}

\textsc{Test 6}
\begin{console}[commandchars=\\\{\}]
\underline{\texttt{-217 30 44 3 -91 10}}
x^3 + (-217/30)x^2 + (44/3)x + (-91/10) = (x - (7/5))(x - (3/2))(x - (13/3)) 
\end{console}


\textsc{Test 7}
\begin{console}[commandchars=\\\{\}]
\underline{\texttt{-95 42 61 42 -2 7}}
x^3 + (-95/42)x^2 + (61/42)x + (-2/7) = (x - (3/7))(x - (1/2))(x - (4/3))
\end{console}


\textsc{Test 8}
\begin{console}[commandchars=\\\{\}]
\underline{\texttt{-9 1 87 4 -35 4}}
x^3 + (-9)x^2 + (87/4)x + (-35/4) = (x - (1/2))(x - (7/2))(x - (5))
\end{console}
% 1/4*(2*x - 1)*(2*x - 7)*(x - 5)
% 1/2, 7/2, 5
% (x - 1/2)(x - 7/2)(x - 5)

\textsc{Test 9}
\begin{console}[commandchars=\\\{\}]
\underline{\texttt{-25 12 13 9 -1 3}}
x^3 + (-25/12)x^2 + (13/9)x + (-1/3) = (x - (2/3))(x - (2/3))(x - (3/4))
\end{console}


\textsc{Test 10}
\begin{console}[fontsize=\scriptsize,commandchars=\\\{\}]
\underline{\texttt{-60 47 1200 2209 -8000 103823}}
x^3 + (-60/47)x^2 + (1200/2209)x + (-8000/103823) = (x - (20/47))(x - (20/47))(x - (20/47))
\end{console}

\textsc{Test 11}
\begin{console}[commandchars=\\\{\}]
\underline{\texttt{-143 23 217 23 -97 23}}
not found
\end{console}

