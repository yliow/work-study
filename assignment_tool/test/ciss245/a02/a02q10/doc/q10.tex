Q10. [Brute force factorization of rational polynomial (with some improvement)]


Notice that your program from Q9 is \textit{really} slow.

Now speed up your program from Q9 in the following way:
Note that if $n$ runs through [-50, 50] (integer values)
and $d$ runs through $[1, 50]$, you will get $101 \times 50=5050$ fractions.
But many of these fractions are repeats.
For instance
\[
  1/1 = 2/2 = 3/3 = 4/4 = ... = 50/50
\]
which gives you 50 fractions which are the same.
Also, $-1/1 = -2/2 = -3/3 = -4/4 = ... = -50/50$. 
You also have
\[
  1/2 = 2/4 = 3/6 = 4/8 = ... = 25/50
\]
which gives you 25 fractions which are the same.
Etc.


So there's no hope in factorizing for instance
\[
x^3 - 1973/816*x^2 + 1103/1224*x - 505/7344
\]
since the factorization is
\[
  \left( x - \frac{101}{51} \right)
  \left( x - \frac{1}{3} \right)
  \left( x - \frac{5}{48} \right)
\]
which contains roots with numerators and/or denominators greater than 50.
You can try to expand your range to 100 and see how long it takes.
Try this if you don't believe me:
\begin{console}[fontsize=\scriptsize]
for (int i = -100; i <= 100; ++i)
{
    std::cout << i << '\n';
    for (int j = 1; j <= 100; ++j)
    {
        for (int k = -100; k <= 100; ++k)
        {
            for (int l = 1; l <= 100; ++l)
            {
                for (int m = -100; m <= 100; ++m)
                {
                    for (int n = 1; n <= 100; ++n)
                    {
                        int a;
                    }
                }
            }
        }
    }
}
\end{console}

How would you not check the redundant cases?


Here's the challenge. Factorize this one:
\[
  x^3 - 79477/10212*x^2 + 172871/10212*x - 6790/851
\]
% 1/10212 * (37 * x - 24) * (23 * x - 97) * (12 * x - 35)
% 24/37, 97/23, 35/12
% (x - (24/37))*(x - (97/23))*(x - 35/12)

You may assume the numerator and denominator range in absolute value is at most 100.


