Q3. This is a continuation of Q2.
(In other words, after you finished Q2, copy the files from Q2
to the directory/folder for Q3.)

Now include a function to subtract fractions in your library.
Test it throughly.

You must add a test function to the main C++ file and allow user to test
your function when the user enters option 3. (Refer to Q2.)
Previous code for testing with option 1 and 2 must be kept.

{\small
\begin{console}
// File: Fraction.h

//-----------------------------------------------------------------------------
// Print fraction modeled by numerator n and denominator d.
//-----------------------------------------------------------------------------
void Fraction_print(int n, int d);

//-----------------------------------------------------------------------------
// The fraction modeled by xn and xd as numerator and denominator, i.e., xn/xd,
// is set to the sum of the fractions modeled by yn/yd and zn/zd, i.e.,
// xn/xd = yn/yd + zn/zd 
//-----------------------------------------------------------------------------
void Fraction_add(int & xn, int & xd,
                  int yn, int yd,
                  int zn, int zd);

//-----------------------------------------------------------------------------
// The fraction modeled by xn and xd as numerator and denominator, i.e., xn/xd,
// is set to the difference of the fractions modeled by yn/yd and zn/zd, i.e.,
// xn/xd = yn/yd - zn/zd 
//-----------------------------------------------------------------------------
void Fraction_sub(int & xn, int & xd,
                  int yn, int yd,
                  int zn, int zd);

\end{console}
\begin{console}
// File: Fraction.cpp

void Fraction_print(int n, int d)
{
}


void Fraction_add(int & xn, int & xd,
                  int yn, int yd,
                  int zn, int zd)
{
}


void Fraction_sub(int & xn, int & xd,
                  int yn, int yd,
                  int zn, int zd)
{
}
\end{console}
}


