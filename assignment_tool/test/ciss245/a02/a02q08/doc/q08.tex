Q8.
This is a continuation of Q7.

Now include a function to check when a fraction is less than another.
Test it throughly.
The name of the function must be \verb!Fraction_lt!.
The function returns a boolean value.
(There's only one reasonable prototype of this function.)

You must add a test function to the main C++ file and allow user to test
your function when the user enters option 8. (Refer to Q2.)
Previous code for testing with option 1, 2, 3, 4, 5, 6, and 7 must be kept.

WARNING: Make sure you test negative fractions.

Test 1
\begin{console}[commandchars=\\\{\}]
\underline{\texttt{8 1 2 1 2}}
0
\end{console}

Test 2
\begin{console}[commandchars=\\\{\}]
\underline{\texttt{8 1 2 1 -2}}
0
\end{console}

Test 3
\begin{console}[commandchars=\\\{\}]
\underline{\texttt{8 3 8 1 2}}
1
\end{console}

Test 4
\begin{console}[commandchars=\\\{\}]
\underline{\texttt{8 -3 8 1 2}}
1
\end{console}

Test 5
\begin{console}[commandchars=\\\{\}]
\underline{\texttt{8 -3 8 -1 2}}
0
\end{console}

Test 6
\begin{console}[commandchars=\\\{\}]
\underline{\texttt{8 2 4 -3 8}}
0
\end{console}

Test 7
\begin{console}[commandchars=\\\{\}]
\underline{\texttt{8 2 4 -3 -8}}
0
\end{console}




