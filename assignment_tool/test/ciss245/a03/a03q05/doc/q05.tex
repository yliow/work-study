%-*-latex-*-
Write a lowercase function:
\begin{console}
void str_lower(char x[], char y[]);
\end{console}
that
copies the character of the string \verb!y! to
\verb!x! except that
uppercase characters are replaced by
lowercase.

For instance
if \verb!y! is \verb@"HelLo worlD ... 123!"@,
then after calling
\verb!str_lower(x, y)!, 
\verb!x! is
\\
\verb@"hello world ... 123!"@.

{\small
\begin{Verbatim}[frame=single,commandchars=\~\!\@]
#include <iostream>
#include <limits>
#include "mystring.h"

const int MAX_BUF = 1024;


// earlier test functions


void test_str_lower()
{
    char x[MAX_BUF];
    char y[MAX_BUF];

    std::cin.getline(y, MAX_BUF);
    str_lower(x, y);
    
    std::cout << x << std::endl;
    return;
}


int main()
{
    int i = 0;
    std::cin >> i;
    std::cin.ignore(std::numeric_limits<std::streamsize>::max(), '\n');

    switch (i)
    {
        // earlier cases
        case 5:
            test_str_lower();
            break;
    }
    return 0;
}
\end{Verbatim}
}

\resett

\nextt
\begin{console}[fontsize=\small,commandchars=\\\{\}]
\userinput{5}
\userinput{hEllo woRld}
hello world
\end{console}

\nextt
\begin{console}[fontsize=\small,commandchars=\\\{\}]
\userinput{5}
\userinput{1 2 3 4 5}
1 2 3 4 5
\end{console}

You are strongly advised to try more test cases of your own.
