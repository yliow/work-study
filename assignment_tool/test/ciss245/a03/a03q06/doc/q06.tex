%-*-latex-*-
Write a string \lq\lq tokenizing'' function:
\begin{console}
bool str_tok(char x[], char y[], char delimiters[]);
\end{console}
that does the following.

If \verb!y! is \verb!"hello world"! and
\texttt{delimiters} is \verb!" "!, when
after calling \verb!str_tok(x, y, delimiters)!,
\verb!x! becomes \texttt{"hello"},
\verb!y! becomes \verb!"world"!.
Furthermore \verb!true! is returned.
Basically the characters in \verb!delimiters! is used to cut up
the string \verb!y! (once).
At this point, if 
\\
\verb!str_tok(x, y, delimiters)!
is called again,
\verb!x! becomes \verb!"world"!,
\verb!y! becomes \verb!""!,
and the function return \verb!true!.
If we call \verb!str_tok(x, y, delimiters)!
a third time,
\verb!x! becomes \verb!""!,
\verb!y! stays as \verb!""!,
and \verb!false! is returned.

Note that \verb!delimiters! can contain more than one characters.
For instance, suppose
\verb!y! is
\\ \verb@"hello world,galaxy,universe!"@.
If \verb!delimiters! is \verb!" ,"!,
then
after the first call of
\\ \verb!str_tok(x, y, delimiters)!,
\verb!x! becomes \verb!"hello"!,
\verb!y! becomes \verb@"world,galaxy,universe!"@,
and the function return \verb!true!.
After the second call of
\verb!str_tok(x, y, delimiters)!,
\verb!x! becomes \verb!"world"!,
\verb!y! becomes \verb@"galaxy,universe!"@,
and the function return \verb!true!.
After the third call of
\verb!str_tok(x, y, delimiters)!,
\verb!x! becomes \verb!"galaxy"!,
\verb!y! becomes \verb@"universe!"@,
and the function return \verb!true!.
After the fourth call of
\verb!str_tok(x, y, delimiters)!,
\verb!x! becomes \verb@"universe!"@,
\verb!y! becomes \verb@""@,
and the function return \verb!true!.
After the fifth call of
\verb!str_tok(x, y, delimiters)!,
\verb!x! becomes \verb@""@,
\verb!y! stays as \verb@""@,
and the function return \verb!false!.
In this example, \verb!' '! and \verb!','! are used to cut up
\verb!y!.

Note that if \verb!y! is \verb!",123"!
and \verb!delimiters! is \verb!","!,
then on calling 
\verb!str_tok(x, y, delimiters)!,
\verb!x! is \verb!""!,
\verb!y! is \verb!"123"!,
and
\verb!true! is returned.

Note also that
 because there are multiple characters in delimiters,
 it's possible for string \verb!y! to be cut up by
 the delimiter characters in different ways.
 The character that is used is the one that produces the
 shortest \verb!x!.
 For instance in the above case where
 \verb!y! is
 \\
 \verb@"hello world,galaxy,universe!"@
 and
 \verb!delimiters! is
 \verb!" ,"!.
 \verb!y! can be cut up by \verb!' '! (at index 5),
 by \verb!','! (at index 11), and
 by \verb!' '! (at index 18).
 The delimiter character that is actually used is \verb!' '! (at index 5)
 because
 that creates the shortest left substring \verb!x!.
  
\begin{Verbatim}[frame=single,fontsize=\small,commandchars=\~\!\@]
#include <iostream>
#include <limits>
#include "mystring.h"

const int MAX_BUF = 1024;


// earlier test functions


void test_str_tok()
{
    char x[MAX_BUF];
    char y[MAX_BUF];
    char delimiters[MAX_BUF] ~textred!= " ,."@;
    
    std::cin.getline(y, MAX_BUF);
    bool b = str_tok(x, y, delimiters);
    
    std::cout << b << ' ' << x << ' ' << y << std::endl;
    return;
}


int main()
{
    int i = 0;
    std::cin >> i;
    std::cin.ignore(std::numeric_limits<std::streamsize>::max(), '\n');

    switch (i)
    {
        // earlier cases
        case 6:
            test_str_tok();
            break;
    }
    return 0;
}
\end{Verbatim}

Make sure you try some of your own test cases.
