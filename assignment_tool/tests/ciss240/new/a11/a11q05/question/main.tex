%-*-latex-*-
As mentioned before, an integer value (\verb!int!)  is
actually limited in size. You can see this by running this program:

\begin{Verbatim}[frame=single]
#include <iostream>

int main()
{
    int x = 1;

    for (int i = 0; i < 100; ++i)
    {
        x *= 10;
        std::cout << x << ' ';
    }

    return 0;
}
\end{Verbatim}

You can see from the output that when the integer becomes too big, it becomes
negative. Specifically, on $32$--bit machines, an \verb!int! is usually from 
$-2^{31}$ to $2^{31} - 1$, roughly $-2$ billion to $2$ billion which is roughly
$10$ digits in length. The objective of this assignment is to simulate integers
which are much larger than an \verb!int!. We do this by using arrays. Many
areas of computer science require extremely huge integers. For instance, the
RSA cryptography uses integers which are several hundred digits long. 

Write a program that performs addition for positive integers up to $1000$
digits. Your program will prompt the user for the digits of the two integers to
be added. The digits are entered separated by spaces. The input for each
integer is terminated by entering $-1$. The (column) addition performed by the
program shows the carries when they are nonzero.

[Hint: In your implementation, the digits for each integer is to be stored in
an array. For instance, if the user entered 

\begin{center}
\verb!1 0 0 6 -1!
\end{center}

(for the integer one thousand and six, i.e., $1006$) and the name of the array
is \verb!x!, then
 
\begin{center}
$6$ is to be stored in \verb!x[0]! \\
$0$ is to be stored in \verb!x[1]! \\
$0$ is to be stored in \verb!x[2]! \\
$1$ is to be stored in \verb!x[3]! \\
\end{center}

and all other \verb!x[i]!'s are set to zero for \verb!i! $= 4, \ldots, 999$.
There should be a constant in your code for the maximum size of the array.
(See your notes on arrays.) Note that in your array \verb!x!, you have $1000$
int values. But in the above example only $4$ digits were used, i.e., 

\begin{center}
\verb!x[0], x[1], x[2], x[3]!
\end{center}

You should declare another \verb!int! value, \verb!x_len! (a length variable)
and set it to $4$. \verb!x_len! represents the actual number of digits used to
model the integer value of \verb!x!. Note that the length of \verb!x!, i.e.,
\verb!x_len!, is different from the size of \verb!x!. The size of \verb!x! is
$1000$ (think of it as the capacity) but the length of \verb!x! (think of it as
the number of slots in the array that are actually used) is $4$.]

(Note that with the length variable, the unused part of the array actually need
not be set to $0$.)

$4$ test cases are included. You are strongly advised to include more test
cases.

\resett
\nextt
\begin{console}[frame=single, commandchars=\\\{\}]
\userinput{1 0 0 6 -1}
\userinput{2 3 4 -1}

    1
  1006
+  234
------
  1240
------
\end{console}

\nextt
\begin{console}[frame=single, commandchars=\\\{\}]
\userinput{9 9 9 9 9 9 9 9 9 9 9 9 9 9 9 9 9 9 9 9 -1}
\userinput{1 -1}

 11111111111111111111
  99999999999999999999
+                    1
----------------------
 100000000000000000000
----------------------
\end{console}

\nextt
\begin{console}[frame=single, commandchars=\\\{\}]
\userinput{1 2 3 -1}
\userinput{9 9 9 9 9 9 9 9 9 9 9 9 9 5 9 9 9 9 9 9 -1}

               111111
                   123
+ 99999999999995999999
----------------------
  99999999999996000122
----------------------
\end{console}

\nextt
\begin{console}[frame=single, commandchars=\\\{\}]
\userinput{1 2 3 -1}
\userinput{0 -1}

  123
+   0
-----
  123
-----
\end{console}

This shows you how to simulate large integers using arrays and also to provide
the addition operation for your simulation of such large integers.

[Of course, for such a system to work effectively, you still need to write code
to do multiplication, division, remainder, etc. All these will be dealt with in
a future class.]
