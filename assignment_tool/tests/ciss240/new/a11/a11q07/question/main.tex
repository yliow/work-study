%-*-latex-*-
The goal here is to write a program that prompts the user for
\underline{\textbf{two}} sequences of numbers (think of two columns of numbers), and
prints the sequence sorted (in ascending order) by the second column.
Specifically, the program will continuously prompt the user for pairs of
numbers (an \verb!int! and a \verb!double!) until an \verb!int! of value $-1$
is entered. The program then sorts the data in ascending order by the
\underline{\textbf{second number}}, i.e., the double, and prints out the data.

For instance, if the user enters

\begin{center}
\verb!1 -0.1 2 3.1 5 0.9 -1!
\end{center}

the data entered basically model this table:

\begin{longtable}{c c}
$1$ & \hspace{2 cm} $-0.1$ \\ 
$2$ & \hspace{2 cm} $3.1$ \\
$5$ & \hspace{2 cm} $0.9$
\end{longtable}

and the program should sort the rows of this table by the second column to get

\begin{longtable}{c c}
$1$ & \hspace{2 cm} $-0.1$ \\
$5$ & \hspace{2 cm} $0.9$ \\
$2$ & \hspace{2 cm} $3.1$
\end{longtable}
	
Note that the pairs of data (the rows) move together as a unit.

This is of course very common in real--life applications. For instance, you can
have a spreadsheet of employee data containing firstname, lastname, DOB,
department, salary, etc. and you want to sort the table by salary. 

\resett
\nextt
\begin{console}[frame=single, commandchars=\\\{\}]
\userinput{1 -0.1 2 3.1 5 0.9 -1}
1 -0.1 
5 0.9
2 3.1
\end{console}

\nextt
\begin{console}[frame=single, commandchars=\\\{\}]
\userinput{3 -0.7 9 12.1 21 -12.9 10 11.2 2 -3.2 -1}
21 -12.9
2 -3.2
3 -0.7
10 11.2
9 12.1
\end{console}

\nextt
\begin{console}[frame=single, commandchars=\\\{\}]
\userinput{9 12.1 0 2.0 108 -2.9 17 13.9 3 5.2 9 2.3 -1}
108 -2.9
0 2
9 2.3
3 5.2
9 12.1
17 13.9
\end{console}
