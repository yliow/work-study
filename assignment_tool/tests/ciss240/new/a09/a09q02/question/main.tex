%-*-latex-*-
Write the following program that prints a calendar month.
The program prompts the user for three integers: the month, the year, and the
day--of--week for the first day of the month (with $0$ representing Sunday, $1$
representing Monday, etc.) For instance, if the user entered
\begin{console}[commandchars=\\\{\}]
\userinput{3 2008 6}
\end{console}
It means that he/she wants the calendar for March $2008$ and the first day of
the month is a Saturday. The output is
\begin{console}
March 2008
--------------------
Su Mo Tu We Th Fr Sa  
                   1
 2  3  4  5  6  7  8
 9 10 11 12 13 14 15 
16 17 18 19 20 21 22
23 24 25 26 27 28 29
30 31
\end{console}

See the \lq\lq Hints\rq\rq section for hints and suggestions.

\resett
\nextt
\begin{console}[commandchars=\\\{\}]
\userinput{3 2008 6}
March 2008
--------------------
Su Mo Tu We Th Fr Sa  
                   1
 2  3  4  5  6  7  8
 9 10 11 12 13 14 15 
16 17 18 19 20 21 22
23 24 25 26 27 28 29
30 31
\end{console}

\nextt
\begin{console}[commandchars=\\\{\}]
\userinput{3 2008 4}
March 2008
--------------------
Su Mo Tu We Th Fr Sa  
             1  2  3
 4  5  6  7  8  9 10
11 12 13 14 15 16 17 
18 19 20 21 22 23 24 
25 26 27 28 29 30 31
\end{console}

\nextt
\begin{console}[commandchars=\\\{\}]
\userinput{2 2008 5}
February 2008
--------------------
Su Mo Tu We Th Fr Sa  
                1  2
 3  4  5  6  7  8  9
10 11 12 13 14 15 16 
17 18 19 20 21 22 23 
24 25 26 27 28 29
\end{console}

\nextt
\begin{console}[commandchars=\\\{\}]
\userinput{2 2009 0}
February 2009
--------------------
Su Mo Tu We Th Fr Sa  
 1  2  3  4  5  6  7  
 8  9 10 11 12 13 14 
15 16 17 18 19 20 21 
22 23 24 25 26 27 28
\end{console}

Of course your program should be smart enough to compute the number of days in
the month (including leap year cases for the month of February.)

\newpage

\textsc{\large Spoiler Warning \dots Incoming Hints \dots}

Analyze the problem carefully and try to break it down to smaller sub--problems.
Look at \textbf{Test 1:}
\begin{console}[commandchars=\\\{\}]
\userinput{3 2008 6}
March 2008
--------------------
Su Mo Tu We Th Fr Sa  
                   1
 2  3  4  5  6  7  8
 9 10 11 12 13 14 15 
16 17 18 19 20 21 22
23 24 25 26 27 28 29
30 31
\end{console}

STEP $1$. Try to get your program just to do this:
\begin{console}[commandchars=\\\{\}]
\userinput{3 2008 6}
March 2008
--------------------
Su Mo Tu We Th Fr Sa
\end{console}  
This should be easy. Make sure you try different test cases. 

STEP $2$. When that's done, you can try to get your program to do this:
\begin{console}[commandchars=\\\{\}]
\userinput{3 2008 6}
March 2008
--------------------
Su Mo Tu We Th Fr Sa
                   1
\end{console}
In other words try to print the first day of the month at the right spot.
Again, try as many test cases as possible, making sure that the first day of
the month is always printed at the right place.

STEP $3$. Next print all the days in the month without wrap--around.
{\scriptsize
\begin{console}[commandchars=\\\{\}]
\userinput{4 2008 2}
April 2008
--------------------
Su Mo Tu We Th Fr Sa
       1  2  3  4  5  6  7  8  9 10 11 12 13 14 15 16 17 18 19 20 21 22 23 24 25 26 27 28 29 30
\end{console}
}

STEP $4$. Next get your program to print newline at the right places.
\begin{console}[commandchars=\\\{\}]
\userinput{4 2008 2}
April 2008
--------------------
Su Mo Tu We Th Fr Sa
       1  2  3  4  5  
 6  7  8  9 10 11 12 
13 14 15 16 17 18 19 
20 21 22 23 24 25 26 
27 28 29 30
\end{console}
