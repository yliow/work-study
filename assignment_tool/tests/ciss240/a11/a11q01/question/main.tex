%-*-latex-*-
Declare an array \verb!x! of $10$ integers; initialize all the values in
\verb!x! to $0$. Continually prompt the user for an index value for integer
variable \verb!i! and an integer value \verb!v! and set \verb!x[i]! to
\verb!v!; the program then prints all the values in the array. This repeats
until the user enters $-1$ for \verb!i!.

\resett
\nextt
\begin{console}[frame=single, commandchars=\\\{\}]
index and new value: \userinput{0 4}
array: 4 0 0 0 0 0 0 0 0 0
index and new value: \userinput{1 5}
array: 4 5 0 0 0 0 0 0 0 0
index and new value: \userinput{2 6}
array: 4 5 6 0 0 0 0 0 0 0
index and new value: \userinput{-1}
\end{console}

\nextt
\begin{console}[frame=single, commandchars=\\\{\}]
index and new value: \userinput{9 123}
array: 0 0 0 0 0 0 0 0 0 123
index and new value: \userinput{5 456}
array: 0 0 0 0 0 456 0 0 0 123
index and new value: \userinput{0 789}
array: 789 0 0 0 0 456 0 0 0 123
index and new value: \userinput{9 321}
array: 789 0 0 0 0 456 0 0 0 321
index and new value: \userinput{-1}
\end{console}

[Hint: Note that you should use two input statements for \verb!i! and \verb!v!,
and not one. In other words, you should have this in your code

\begin{Verbatim}
            ...
            std::cin >> i;
            ...
            std::cin >> v;
            ...
\end{Verbatim}

and not 

\begin{Verbatim}
            ...
            std::cin >> i >> v;
            ...
\end{Verbatim}

Why? Because that will allow you to input \verb!i! and then break the loop
without prompting the user for \verb!v!.

\begin{Verbatim}
            ...
            std::cin >> i;
            break if i is -1
            std::cin >> v;
            ...
\end{Verbatim}

On the other hand this code:

\begin{Verbatim}
            ...
            std::cin >> i >> v;
            ...
\end{Verbatim}

forces the user to enter a value for \verb!v!.]
