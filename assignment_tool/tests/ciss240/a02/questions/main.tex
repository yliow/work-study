%-*-latex-*-
\input{myassignmentpreamble}
\input{ciss240}
\input{yliow}
\renewcommand\TITLE{Assignment 2}
\usepackage{import}

\begin{document}
\topmatter

\textsc{Objectives}
 \begin{myenum}
   \li Declare integer variables
   \li Input and output of integer values
   \li Use integer operators
 \end{myenum}
\mbox{}\\

\textsc{Instructions.}
\begin{enumerate}
\li Your program must be well-written. 
    You must follow the style in your notes as closely as possible. 
    Take note of the spaces and blank lines I used in my examples. 
    Badly written programs will very likely result in a poor grade for this 
    assignment. 
    Points will be taken off for sloppy work. 
\li It's important to remember this: In your printouts for all assignments, 
    there must be no wraparound.
\li All outputs must match exactly the output shown. 
    That includes every single space and every blank line.

\li The format of your program must look like this
(replacing \lq\lq smaug'' with your name of course):
\begin{Verbatim}[frame=single,fontsize=\small]
// Name: smaug
// File: a02q01.cpp

#include <iostream>

int main()
{
    *** YOUR WORK HERE ***

    return 0;
}
\end{Verbatim}
In particular:
\begin{enumerate}
\li You must have your name and the name of the file at the top of each 
    C++ source file as shown above.
\li Your C++ source file must end with a blank line.
\end{enumerate}

\li Instructions on uploading your work will be provided in class.

\end{enumerate}

Read the questions carefully before diving in!

Each question contains test cases that your program must pass. For instance for Q1, there are three test cases. Test 1 looks like this:

\textsc{Test 1}
\begin{console}[commandchars=\\\{\}]
Enter w: \userinput{1}
Enter h: \userinput{2}
Enter f: \userinput{3}
IQ: 1
\end{console}

This is a complete test run of your program and should be treated as an exact
screenshot of your console windows.
The underlined text is user input. 
The above test case means that when you run your program, it should print
\begin{console} 
Enter w: 
\end{console}
and wait for the user to enter a value. 
The user enters \verb!1! and press the enter key. 
Your program then prints
\begin{console}
Enter h: 
\end{console}
At this point in Test 1, the user enters \verb!2!. Etc.  
The program ends after the program's output of \verb!IQ: 1! as shown above.
Note that for Q1 there's a Test 2:
\textsc{Test 2}
\begin{console}[commandchars=\\\{\}]
Enter w: \userinput{6}
Enter h: \userinput{5}
Enter f: \userinput{4}
IQ: 3
\end{console}
This shows you what is expected when you run your program against
for a second run.
This is not the continuation of the program execution
in Test 1 which has already ended.

Note that you should create a new project for each question. 
For easy maintenance of your assignments, 
I suggest you have a folder \verb!ciss240! somewhere in your 
\verb!Documents!, and in that you have a folder \verb!a!, 
and in folder \verb!a! you have a folder \verb!a02!, 
and you have solutions folders \verb!a02q01!, \verb!a02q02!, etc. in the 
folder \verb!a02!:

\begin{Verbatim}
    .
    .
    .
    ciss240
    |
    +--- a
         |
         +--- a01
         |    |
         |    +--- a01q01
         |    |
         |    +--- a01q02
         |
         +--- a02
              |
              +--- a02q01
              |
              +--- a02q02
\end{Verbatim}

Note that the name for the C++ source file for question 1 
(i.e. the cpp file in 
project \verb!a02q01!) must be \verb!a02q01.cpp!, etc.


\newpage{Q1. }\subimport*{../a02q01/question/}{main.tex}
\newpage{Q2. }\subimport*{../a02q02/question/}{main.tex}
\newpage{Q3. }\subimport*{../a02q03/question/}{main.tex}
\newpage{Q4. }\subimport*{../a02q04/question/}{main.tex}
%\newpage{\bf Q5. }\subimport*{../a01q05/question/}{main.tex}

\end{document}
