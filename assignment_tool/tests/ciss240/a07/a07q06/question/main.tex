%-*-latex-*-
You are working for a 
game company and you have
been asked to 
write the logic for a smart tic-tac-toe AI agent.
(I assume you know the tic-tac-toe game ... 
otherwise ... hmmm ...)

Here's the game board:
\begin{python}
from latextool_basic import *
print(consolegrid(numrows=3, numcols=3))
\end{python}
and here's what we call a game state (i.e., a snapshot 
of the game in progress):
\begin{python}
from latextool_basic import *
print(consolegrid(numrows=3, numcols=3, s='XO-\nX--\nO-X'))
\end{python}
I use \verb!-! to denote the fact that a square is available.

We can model the game state with 9 variables.
I'll call them 
\verb!board00!,
\verb!board01!,
\verb!board02!,
\verb!board10!,
\verb!board11!,
\verb!board12!,
\verb!board20!,
\verb!board21!, and
\verb!board22!.
The variables will hold character values, i.e., they are \verb!char!
variables.
For instance for the game state above
\begin{python}
from latextool_basic import *
print(consolegrid(numrows=3, numcols=3, s='XO-\nX--\nO-X'))
\end{python}
the variables have the following values:
\begin{console}[frame=none]

        board00 = 'X',  board01 = 'O',  board02 = '-'
        board10 = 'X',  board11 = '-',  board12 = '-'
        board20 = 'O',  board21 = '-',  board22 = 'X'

\end{console}
For memory aid, the variables are named according to this
format: \verb!board! is followed by two numbers, the row number and the
column number where rows and columns are numbered starting from 0.
For instance \verb!board12! refers to the character at row 1 and column 2.
Run this program
\begin{console}[fontsize=\small]
#include <iostream>

int main()
{
    char board00, board01, board02;
    char board10, board11, board12;
    char board20, board21, board22;
    std::cin >> board00 >> board01 >> board02
             >> board10 >> board11 >> board12
             >> board20 >> board21 >> board22;

    std::cout << "+-+-+-+\n"
              << '|' << board00 << '|' << board01 << '|' << board02 << "|\n"
              << "+-+-+-+\n"
              << '|' << board10 << '|' << board11 << '|' << board12 << "|\n"
              << "+-+-+-+\n"
              << '|' << board20 << '|' << board21 << '|' << board22 << "|\n"
              << "+-+-+-+\n";

    return 0;
}
\end{console}
Run it with input
\begin{console}[fontsize=\small,commandchars=\\\{\}]
X O - - - - O - X
\end{console}
and you will get this output
\begin{console}[fontsize=\small,commandchars=\\\{\}]
+-+-+-+
|X|O|-|
+-+-+-+
|-|-|-|
+-+-+-+
|O|-|X|
+-+-+-+
\end{console}
Your goal is to write a program that prompts for a game state
and decide on a good move for the player playing \verb!X! by 
printing a suggested row and column.
\begin{console}[fontsize=\small,commandchars=\\\{\}]
X O - - - - O - X
\end{console}
and you will get this output
\begin{console}[fontsize=\small]
+-+-+-+
|X|O|-|
+-+-+-+
|-|-|-|
+-+-+-+
|O|-|X|
+-+-+-+
1 1
\end{console}
since in this case is \verb!X! is placed at row 1 and column 1, 
the player playing \verb!X! will have a winning diagonal.

The strategy is as follows.
The following is a list of actions to take.
You should use the one listed earlier if it's possible to execute that
action.
For instance if you can either have winning row (see item 1)
or a winning column (see item 2),
then your strategy is to take the winning row (item 1).
\begin{tightlist}
\item[1.] 
If there's a winning row, take the winning row, 
preferring the topmost when there is more than one.
\item[2.]
If there's a winning column, take the winning column,
preferring the leftmost when there is more than one.
\item[3.]
If there's a winning diagonal that goes from
top-left to bottom-right, take the diagonal.
\item[4.]
If there's a winning reverse diagonal that goes from top-right
to bottom-left, take the winning reverse diagonal.
\item[5.]
Block player \verb!O! from making a winning
row, column, diagonal, or reverse diagonal
in the same order of preference as above.
(In other words, if you can block a top row and bottom row,
then you should block a top row.)
\item[6.] Take the center.
\item[7.] 
Take a corner,
looking for an available corner going top-to-bottom, left-to-right.
\item[8.]
If none of the above works, take one of the remaining positions,
looking for an available position going top-to-bottom, left-to-right.
\end{tightlist}
For this question, I will only test your code for scenarios
corresponding to items 1--4.
You are of course more than welcome to code up all the items if you like.

[Note: The above strategy considers the current game state
and not future game ststes, i.e., it does not consider
all the possible moves by the opponent. 
A strategy that does not consider all cases and in particular
only considers the current data is called 
a \textit{greedy} algorithm.
A smarter AI strategy would consider possible future scenarios
as well.
However when a game complex, you usually cannot consider
all scenarios.
For instance the total number of chess games can be as large as 
$10^{120}$; 
the number of atoms in the observable universe is approximately
$10^{80}$.
If your program considers all possible scenarios, then
it's called a \textit{brute force} algorithm.
You can learn more about algorithms for solving 
computationally intensive problems in the AI class CISS450.
For such cases, one would need to use \textit{heuristic}
algorithms, otherwise you would need trillions and trillions
of years to 
compute a single chess move!]

You only need to know that you can compare character values.
Try this on your own:
{\footnotesize
\begin{console}
char a = '!', b = '@', c = '!';
std::cout << (a == b) << '\n';
std::cout << (a == c) << '\n';
std::cout << (a != b) << '\n';
std::cout << (a != c) << '\n';
\end{console}
}
You can of course also perform input/output for \verb!char! variables:
{\footnotesize
\begin{console}
char a = ' ', b = ' ', c = ' ';
std::cin >> a >> b >> c;
std::cout << a << ' ' << b << ' ' << c << '\n';
\end{console}
}

You should use the following skeleton code.
[Note: Skeleton code is definitely incomplete and might
contain errors.]
{\footnotesize
\begin{console}
#include <iostream>

int main()
{
    const char PLAYER = 'X';
    const char ENEMY = 'O';
    const char SPACE = '-';

    char board00, board01, board02;
    char board10, board11, board12;
    char board20, board21, board22;
    std::cin >> board00 >> board01 >> board02
             >> board10 >> board11 >> board12
             >> board20 >> board21 >> board22;

    std::cout << "+-+-+-+\n"
              << '|' << board00 << '|' << board01 << '|' << board02 << "|\n"
              << "+-+-+-+\n"
              << '|' << board10 << '|' << board11 << '|' << board12 << "|\n"
              << "+-+-+-+\n"
              << '|' << board20 << '|' << board21 << '|' << board22 << "|\n"
              << "+-+-+-+\n";

    int row = -1, col = -1;

    if (board00 == SPACE && board01 == PLAYER && board02 == PLAYER)
    {
        // Take (0, 0) for a winning row 0
        row = 0;
        col = 1;
    }
    else
    {
        if (board00 == PLAYER && board01 == SPACE && board02 == PLAYER)
        {
            // Take (0, 1) for a winning row 0
            row = 0;
            col = 2;
        }
        // MORE CODE HERE
    }

    std::cout << row << ' ' << col << std::endl;

    return 0;
}
\end{console}
}

\newpage
\textsc{Preventing indentation in a stack of if-else}

Note that if you have code like this where 
in the \verb!else! block you only have an 
\verb!if! statement or an 
\verb!if-else! statement:
{\footnotesize
\begin{console}
if (bool1)
{
   ...
}
else
{
    if (bool2)
    {
        ...
    }
    else
    {
        if (bool3)
        {
            ...
        }
        else
        {
            ...
        }
    }
}
\end{console}
}
then, it's the same as the following 
(the \verb!else! has only one \verb!if-else! statement
so the block is redundant -- check your notes):
{\footnotesize
\begin{console}
if (bool1)
{
   ...
}
else
    if (bool2)
    {
        ...
    }
    else
        if (bool3)
        {
            ...
        }
        else
        {
            ...
        }
\end{console}
}
which is the same as the following (because
 C++ ignores whitespaces -- check your notes):
{\footnotesize
\begin{console}
if (bool1)
{
   ...
}
else if (bool2)
{
    ...
}
else if (bool3)
{
    ...
}
else
{
    ...
}
\end{console}
}
In other words, we join an \verb!if! with the preceding \verb!else!.

Writing it this way will prevent too much indentation toward the 
right and make your program look like a table.
This is one of the very few cases where we ignore the standard
indentation rules in order not to over--indent.

Note that this won't work if the \verb!else! contains more than
just an \verb!if-else!, for instance
{\footnotesize
\begin{console}[commandchars=\~\!\@]
if (bool1)
{
   ...
}
else
{
    ~underline!x = 1;@
    if (bool2)
    {
        ...
    }
    else
    {
        ...
    }
}
\end{console}
}
In this case, since the outermost \verb!else! has two statements,
you cannot get rid of the block symbols \verb!{}! to join up the 
inner \verb!if! with the outer \verb!else!.
