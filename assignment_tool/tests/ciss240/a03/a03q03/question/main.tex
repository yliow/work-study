%-*-latex-*-
Q3.
Write a program that prompts the user for a $5$--digit integer, encrypts it, and finally displays the original integer (known as the plaintext) and the encrypted integer (known as the ciphertext). The following is the algorithm for our encryption. 

Suppose the digits are $a$, $b$, $c$, $d$, and $e$ where $e$ is the rightmost digit (also known as the least significant digit). Let the digits of the encrypted integer be $a1$, $b1$, $c1$, $d1$, and $e1$. These digits make up the encrypted integer where $e1$ is the rightmost digit.
\begin{tightlist}
  \li First: Add $1$ to $e$ to get $e1$. If $e1$ becomes $10$, change $e1$ to $0$.
  \li Second: Swap $d$ and $c$ into $c1$ and $d1$. In other words the value of $c1$ is the value of $d$ and the value of $d1$ is the value of $c$.
  \li Third: Swap $a$ and $b$ into $b1$ and $a1$. In other words the value of $b1$ is the value of $a$ and the value of $a1$ is the value of $b$.
\end{tightlist}

Refer to the test cases for examples.

[HINT: You want to perform this experiment before trying this program. Use C++ to compute (and display) the $0\,\%\,10$, $1\,\%\,10$,  $2\,\%\,10$,  $3\,\%\,10$, $\dots$, $10\,\%\,10$. Why is this useful for the first step above?]

You must use the following skeleton code for \verb!main()!:
\begin{console}
// Prompt user for an integer value for variable plaintext.
int plaintext = 0;

// a, b, c, d, e are the digits of plaintext

// a1, b1, c1, d1, e1 are the digits of the encrypted ciphertext

// Form the ciphertext (an int)

// Print plaintext and ciphertext

return 0;
\end{console}

\resett
\nextt
\begin{console}[commandchars=\\\{\}]
\userinput{12345}
12345 21436
\end{console}

\nextt
\begin{console}[commandchars=\\\{\}]
\userinput{97531}
97531 79352
\end{console}

\nextt
\begin{console}[commandchars=\\\{\}]
\userinput{56789}
56789 65870
\end{console}
