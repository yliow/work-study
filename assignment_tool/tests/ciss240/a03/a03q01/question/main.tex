%-*-latex-*-
Q1.
It's a well--known fact that when you take a positive integer and compute its remainder after dividing it by $9$, you will find that it's the same as first adding the digits of the number and then computing the remainder after dividing by $9$. For instance the remainder of $38$ after dividing by $9$ is $2$ (because $38$ is $4 \times 9 + 2$) and the sum of the digits of $38$ is $3 + 8 = 11$ whose remainder after dividing by $9$ is also $2$ (because $11 = 1 \times 9 + 2$). Obviously it's easier to compute the remainder of $3 + 8$ when divided by $9$ than the remainder of $38$ when divided by $9$. This fact is used by some to do mental calculations quickly. The goal of this question is to verify the above fact for $4$--digit numbers.

Write a program that verifies the above fact for $4$--digit numbers. Your program must pass the following tests. The first line (\underline{underlined}) is the input from the user. The second line has three output values: 
\begin{tightlist}
\li The first is the remainder of the user--input integer after dividing by $9$
\li The second is the sum of the digits of user--input integer, and 
\li The third is the remainder of the second number after dividing it by $9$. 
\end{tightlist}
If the above fact is true, the first output value must be the same as the third output value. For instance in Test $1$, the first output value is $1$ and the third is $1$.

(With more math one can prove that the above fact is in fact true for a positive integer of any number of digits. For this question we only want to verify it for the cases specified by the user.)

\resett
\nextt
\begin{console}[commandchars=\\\{\}]
\userinput{1234}
1 10 1
\end{console}

\nextt
\begin{console}[commandchars=\\\{\}]
\userinput{2468}
2 20 2
\end{console}

\nextt
\begin{console}[commandchars=\\\{\}]
\userinput{1111}
4 4 4
\end{console}
