%-*-latex-*-
Recall that we had a previous problem that involved printing a calendar month.
Here's the problem:

Write the following program that prints a calendar month.
The program prompts the user for three integers: the month, the year, and the
day--of--week for the first day of the month (with $0$ representing Sunday, $1$
representing Monday, etc.) For instance, if the user entered
\begin{console}[commandchars=\\\{\}]
\userinput{3 2008 6}
\end{console}
It means that he/she wants the calendar for March $2008$ and the first day of
the month is a Saturday. The output is
\begin{console}
March 2008
--------------------
Su Mo Tu We Th Fr Sa  
                   1
 2  3  4  5  6  7  8
 9 10 11 12 13 14 15 
16 17 18 19 20 21 22
23 24 25 26 27 28 29
30 31
\end{console}

Rewrite the calendar month printing program so that it uses the two functions
\verb!is_leap_year()! and \verb!days_in_month()! from the previous questions.
You must use the skeleton code given below.

Note that we are not changing how the original program works. The point is to
\lq\lq factor" out useful code from our original program. This makes the
program more readable -- it will be very obvious once you're done with this
question. Also, the functions factored out can be used for other programs.

\begin{Verbatim}[frame=single]
#include <iostream>


//------------------------------------------------------------------------
// This function returns true if year is a leap year. Otherwise false is
// returned.
// (This function has no output.)
//------------------------------------------------------------------------
bool is_leap_year(int year)
{
    ... your code here ...
}


//------------------------------------------------------------------------
// This function returns the number of days in given month and year.
// (This function has no output.)
//------------------------------------------------------------------------
int days_in_month(int month, int year)
{
    ... your code here ...
}


//------------------------------------------------------------------------
// This function prints the calendar of the given month and year.
// The parameter day_of_first is the day of the 1st of the given month
// and year.
//------------------------------------------------------------------------
void print_calendar_month(int month, int year, int day_of_first)
{
    ... your code here ...
}
 

int main()
{
    int month, year, day_of_first;
    std::cin >> month >> year >> day_of_first;

    print_calendar_month(month, year, day_of_first);

    return 0;
}
\end{Verbatim}

For your reference, here's the relevant program from a previous assignment:

{\small
\begin{Verbatim}[frame=single]
#include <iostream>
#include <iomanip>
#include <cmath>

int main()
{
    int month, year, day_of_first;
    std::cin >> month >> year >> day_of_first;

    switch (month)
    {
        case 1: std::cout << "January "; break;
        case 2: std::cout << "February "; break;
        case 3: std::cout << "March "; break;
        case 4: std::cout << "April "; break;
        case 5: std::cout << "May "; break;
        case 6: std::cout << "June "; break;
        case 7: std::cout << "July "; break;
        case 8: std::cout << "August "; break;
        case 9: std::cout << "September "; break;
        case 10: std::cout << "October "; break;
        case 11: std::cout << "November "; break;
        case 12: std::cout << "December "; break;
    }

    std::cout << year << '\n'
              << "--------------------\n"
              << "Su Mo Tu We Th Fr Sa\n";


    // Compute the number of days in the month and store in days_in_month
    int days_in_month = 0;
    if (1 <= month and month <= 12)
    {
        if (month == 2) // 28 or 29 days
        {
            if (year % 4 == 0 
                && ((year % 100 != 0) || (year % 100 == 0) 
                    && (year % 400 == 0)
                    )
                )
            {
                days_in_month = 29;
            }
            else
            {
                days_in_month = 28;
            }
        }
        else if ((month % 2 == 1 && month <= 7) || 
                 (month % 2 == 0 && month >= 8)) // 31 days
        {
            days_in_month = 31;
        }
        else // 30 days
        {  
            days_in_month = 30;
        }
    }

    // Print spaces before the first day of the month
    for (int i = 0; i < day_of_first; i++)
    {
        std::cout << "   ";
    }

    int day_of_week = day_of_first;
    for (int day_in_month = 1; day_in_month <= days_in_month; day_in_month++)
    {
        std::cout << std::setw(2) << day_in_month << ' ';
        
        day_of_week++;
        if (day_of_week == 7)
        {
            std::cout << std::endl;
            day_of_week = 0;
        }
    }
    
    std::cout << std::endl;
    
    return 0;
}
\end{Verbatim}
}

Make sure you test your program thoroughly.


\resett
\nextt
\begin{console}[frame=single, commandchars=\\\{\}]
\userinput{3 2008 6}
March 2008
--------------------
Su Mo Tu We Th Fr Sa  
                   1
 2  3  4  5  6  7  8
 9 10 11 12 13 14 15 
16 17 18 19 20 21 22
23 24 25 26 27 28 29
30 31
\end{console}

\nextt
\begin{console}[frame=single, commandchars=\\\{\}]
\userinput{3 2008 4}
March 2008
--------------------
Su Mo Tu We Th Fr Sa  
             1  2  3
 4  5  6  7  8  9 10
11 12 13 14 15 16 17 
18 19 20 21 22 23 24 
25 26 27 28 29 30 31
\end{console}

\nextt
\begin{console}[frame=single, commandchars=\\\{\}]
\userinput{2 2008 5}
February 2008
--------------------
Su Mo Tu We Th Fr Sa  
                1  2
 3  4  5  6  7  8  9
10 11 12 13 14 15 16 
17 18 19 20 21 22 23 
24 25 26 27 28 29
\end{console}

\nextt
\begin{console}[frame=single, commandchars=\\\{\}]
\userinput{2 2009 0}
February 2009
--------------------
Su Mo Tu We Th Fr Sa  
 1  2  3  4  5  6  7  
 8  9 10 11 12 13 14 
15 16 17 18 19 20 21 
22 23 24 25 26 27 28
\end{console}
