%-*-latex-*-
\input{myassignmentpreamble}
\input{ciss358}
\input{yliow}
\renewcommand\TITLE{Assignment 2}

\begin{document}
\topmatter

\textsc{Objectives}
\begin{itemize}
  \li Prove a statement using mathematical induction.
\end{itemize}

\newpage
\textsc{Proving $P(n)$ for all $n \geq n_0$ where $n_0$ is a fixed integer}

Recall that if you want to prove
$P(n)$ is true for all $n \geq 42$,
you have to do two things:
\begin{tightlist}
  \li Prove $P(42)$ is true
  \li Let $n \geq 42$. Assume $P(n)$ holds and prove $P(n + 1)$ holds as well.
\end{tightlist}
If you can achieve the above two points, then you can claim
\begin{tightlist}
  \li $P(n)$ is true for all $n \geq 42$
\end{tightlist}
This is one form of mathematical induction called \textbf{weak mathematical induction}.
The \textbf{strong mathematical induction} says that if you can do two things:
\begin{tightlist}
  \li Prove $P(42)$
  \li Let $n \geq 42$. Assume $P(42), P(42 + 1), ..., P(n)$ holds and prove $P(n + 1)$ holds as well.
\end{tightlist}
If you achieve the above two points, then you can claim
\begin{tightlist}
  \li $P(n)$ is true for all $n \geq 42$
\end{tightlist}
The \lq\lq42" above can be replaced by any integer (including some negative integer).

The above two induction techniques proves $P(n)$ are all true for $n \geq n_0$.
Mathematical induction also allows you go \lq\lq go backward".
If you can do the following:
\begin{tightlist}
  \li Prove $P(n_0)$ is true
  \li Let $n \leq n_0$. Assume $P(n)$ holds and prove $P(n - 1)$ holds as well.
\end{tightlist}
then you can claim
\begin{tightlist}
  \li $P(n)$ is true for all $n \leq n_0$
\end{tightlist}
This is also called weak mathematical induction.
The strong induction going backward holds too.
If you can achieve the following:
\begin{tightlist}
  \li Prove $P(n_0)$ is true
  \li Let $n \leq n_0$. Assume $P(n_0), P(n_0 - 1), ..., P(n)$  holds and prove $P(n - 1)$ holds as well.
\end{tightlist}
then you can claim
\begin{tightlist}
  \li $P(n)$ is true for all $n \leq n_0$
\end{tightlist}
So induction allows you to prove $P(n)$ when $n$ goes to infinity or
when $n$ goes to negative infinity.

There is another method that's important and does the same thing for you.

Sometimes, it is possible to prove $P(n)$ for all $n \geq 1$ by using the
\lq\lq proof by contradiction" method by using
the well-ordering principle (WOP) which says that
\begin{tightlist}
  \li Every nonempty subset of $\N = \{0, 1, 2, ...\}$ has a least element,
  i.e., if $X$ is a nonempty subset of $\N$, then there is some
  $m \in X$ such that $m \leq x$ for all $x \in X$.
\end{tightlist}
(See class notes.)
In this case, frequently, the proof involves an argument of the form:
\lq\lq Suppose it's not true that $P(n)$ holds for all $n \geq n_0$".
Then the set
\[
X = \{ n \mid n \geq n_0, \,\,\, P(n) \text{ does not hold}\}
\]
is a nonempty subset of $\N$.
By WOP, $X$ has a least element, i.e.,
there is a smallest $m$ such that $m \geq n_0$ and $P(m)$ is false.
And you continue to prove that something goes wrong,
i.e.,you attempt to arrive at a contradiction.
This is frequently done in one of two ways.
Here's one way to achieve this:
Since $m$ is the least element of $X$,
all $n \in \N$ with $n < m$ must satisfy $P(n)$.
From such $n$, you then show that in fact $P(m)$ holds, which
clearly is a contradiction.
The second method is this:
Try to find some $m' < m$
such that $P(m')$ is also false.
This would contradict the fact that $m$ is the least elment of
$X$.

(\textsc{Aside.}
WOP is related to the proof method called \textbf{Fermat's infinite descent}.
Applying the same argument above on $k'$ and assuming $P(k')$ is false,
you would arrive at another
$k'' < k'$ such that $P(k'')$ is false, etc.
This gives you infinitely many positive integers
$k > k' > k'' > k''' > \cdots$.
This is clearly impossible since there can only be finitely many positive
integer from $n_0$ up to $k$.)

Until you know how to write induction proofs properly, you must follow these
instructions for writing a proof that uses induction.
\begin{tightlist}
\item[1.]
  Paragraph 1: State you $P(n)$ and the range of values for $n$.
  If a problem involves proving multiple statements, you can also use
  $Q(n)$, $R(n)$, etc.
  State what method you are using (weak or strong induction).
  The default is weak induction, i.e., if you want you are using
  mathematical induction, it means you are using weak mathematical induction.
\item [2.]
  Paragraph 2: State you are proving the base case.
  Prove the base case.
\item [3.]
  Paragraph 3: State you are proving the inductive case.
  State your inductive hypothesis and state what you are going to prove.
  Then prove it. If the proof is long, state what you have proven.
  It's even a good idea to state it anyway. That's called good writing:
  State at the beginning of a paragraph what you want to do, do it, then
  remind the reader the goal at the beginning of the paragraph.
  (This is the longest part of the proof. If necessary, you might need
  more than one paragraph.)
\item [4.]
  Paragraph 4:
  State, quoting the method (i.e., induction), what you have proven.
\end{tightlist}
Once you are done with writing about 50 induction proofs, you can use a
freer form.

Here are two examples:

\newpage

\begin{thm}
  If $n \geq 0$, then
  \[
    1 + 2 + \cdots + n = \frac{n(n + 1)}{2}
  \]
  (Note that if $n = 0$, then the expression on the left-hand side of the above
  equation is 0 by definition.
  In other words an empty sum -- sum of no terms -- is defined to be 0.)
\end{thm}

\textit{Proof}.
We will prove this by weak mathematical induction.
For $n \geq 0$, we define
\[
  P(n) = \biggl( 1 + 2 + \cdots + n = \frac{n(n + 1)}{2} \biggr)
\]

\textsc{Base case.}
When $n = 0$, we have
\[
1 + 2 + \cdots + n = 0 = \frac{0(0 + 1)}{2} = \frac{n (n + 1)}{2}
\]
Hence $P(0)$ holds.

\textsc{Inductive case.}
Assume $P(n)$ holds where $n \geq 0$, i.e., we assume
\[
  P(n) = \biggl( 1 + 2 + \cdots n = \frac{n(n + 1)}{2} \biggr)
\]
holds.
We want to show $P(n + 1)$ holds, i.e., we want to show
\[
P(n + 1) =
\biggl(
1 + 2 + \cdots n + 1 = \frac{(n + 1)(n + 1 + 1)}{2}
\biggr)
\]
is true.
Since $P(n)$ holds, we have
\begin{align*}
  1 + 2 + \cdots n
  &= \frac{n(n + 1)}{2} \\
  \THEREFORE 1 + 2 + \cdots n + (n + 1)
  &= \frac{n(n + 1)}{2} + (n + 1) \\
  &= \frac{n(n + 1) + 2(n + 1)}{2} \\
  &= \frac{(n + 1)(n + 2)}{2} \\
  &= \frac{(n + 1)((n + 1) + 1)}{2}
\end{align*}
i.e., $P(n + 1)$ holds.

Therefore, by weak mathematical induction,
$P(n)$ holds for all $n \geq 0$, i.e.,
for $n \geq 0$,
\[
  1 + 2 + \cdots + n = \frac{n(n + 1)}{2}
\]
\qed



\newpage
\begin{thm}
  Let $T$ be a tree wth at least one node,
  i.e., a connected simple graph with at least one node and no cycles.
  Then $e = v - 1$ where $e$ and $v$ are the number of edges and
  nodes of $T$ respectively.
\end{thm}

(Note: A simple graph is a graph with no loops, i.e., no edge
joining a node to itself and no multi-edges, i.e., no multiple edges
joining the same two nodes.
A graph is connected is for every pair of distinct vertices $x,y$,
there is a path of edges from $x$ to $y$.)

\textit{Proof}.
For a graph $G$, we will write $v_G$ and $e_G$ for the number of
nodes and number of edges of $G$ (respectively).
For $n \geq 0$, we define the proposition $P(n)$ as follows:
\[
  P(n) = 
  (
  \text{If $T$ is a tree with at least one node
  \underline{and with $n$ edges}, then $e_T = v_T - 1$}
  )
\]
We will prove $P(n)$ holds for all $n \geq 0$ by strong mathematical induction.

\textsc{Base case}.
We will prove $P(0)$, i.e., we will prove that
if $T$ is a tree with at least one node and 
$0$ edges (i.e., $e_T = 0$), then $e_T = v_T - 1$.
If $T$ has at least two distinct nodes,
say $x$ and $y$, then $x$ and $y$ are not adjacent since there are
no edges in $T$.
But $T$ is connected.
This is a contradiction.
Hence $T$ has exactly one node, i.e., $v_T = 1$.
Therefore
\[
e_T = 0 = 1 - 1 = v_T - 1
\]
i.e., $P(0)$ holds.

\textsc{Inductive case}.
Let $n \geq 0$.
Assume $P(0), P(1), ..., P(n)$ holds.
We will show that $P(n + 1)$ holds, i.e., we will show that
if $T$ is a tree with at least one node and has $n + 1$ edges, then
\[
  e_T = v_T - 1
\]
Let $T$ be a tree with at least one node and $n + 1$ edges.
Since $n \geq 0$, we have $n + 1 \geq 1$.
Hence there is at least one edge in $T$, say
$e$ denote an edge in $T$ joining node $x$ and node $y$.
Since $T$ is simple, $x \neq y$.
Construct the graph
\[
  G = T - e
\]
i.e., $G$ is the graph $T$ with edge $e$ removed.
We have
\begin{align*}
  v_G &= v_T \\
  e_G &= e_T - 1 
\end{align*}
Note that $G$ contains all the nodes of $T$ and in particular
contains $x$ and $y$.
We claim that $G$ is made up of two disjoint trees.

First, we show that $G$ has exactly two connected components, i.e.,
$G$ is made up of two maximal connected subgraphs.
Suppose $k$ is the number of connected components of $G$.
Let $G_1, G_2, ..., G_k$ denote the connected components of $G$.
There are no paths joining a node in $G_i$ to a node in $G_j$ if $i \neq j$.
We will show $k = 2$ and furthermore each $G_i$ is a tree.

The nodes $x$ and $y$ are in $G$.
Therefore $x$ is in some $G_i$ and $y$ is in some $G_j$.
Note that $i \neq j$.
Otherwise, if $i = j$, $x$ and $y$ are in the same connected component
$G_i$ which implies that there is some path $p$ in $G_i$ joining $x$ and $y$.
Since $G_i$ is a subgraph of $G = T - e$ and $G$ does not contain edge $e$,
$G_i$ cannot contain $e$.
The path $p$ (which is in $G_i$) therefore also cannot contain edge $e$.
Hence $p$ and edge $e$ will form a cycle in $T$.
This is a contradiction since $T$ is a tree and cannot have a cycle.
Hence we have shown that $i \neq j$, i.e., $x$ and $y$ are in two different
connnected components of $G$.

Now suppose $k > 2$, i.e.,
suppose there are three connected components.
Recall that $x$ is in some $G_i$ and $y$ is in some $G_j$
with $i \neq j$.
Since there are at least three connected components, there is some
$k$
such that $k \neq i, k \neq j$. 
Let $z$ be a node in $G_k$.
Since $T$ is a tree, there is a path $p$ in $T$ from $x$ to $z$.
There are no repeated nodes in $p$. 
Since $x$ and $z$ are in different connected components,
the path $p$ must leave $G_i$ and must enter $G_k$.
The only edge that leaves $G_i$ is the edge $e$.
Since $y$ is in $G_j$,
path $p$ will leave $G_i$ and enter $G_j$.
Therefore on entering $G_j$, $p$ contains $x$ and $y$.
However to arrive at $G_k$, which is not $G_j$,
the path $p$ has to leave $G_j$.
The only edge leaving $G_j$ is $e$, which means that
the path $p$ on leaving $G_k$ will repeat $x$.
This is a contradiction.
We conclude that $k = 2$, i.e., there are two
connected components in $G = T - e$.

We have now shown that there are two connected components $G_1, G_2$ in $G$.

We now show that the connected components $G_1, G_2$ in $G$
are trees.
Suppose there is a cycle $C$ in $G_1$.
Since $G_1$ is in $G = T - e$, the cycle $C$ is also in $G$
and is therefore in $T$.
This is a contradiction since $T$ is a tree.
(This shows that every subgraph of a tree cannot have cycles.)

We have now shown that $T - e$ is made up of two
disjoint trees, say $T_1$ and $T_2$.

Since $T_1, T_2$ are both subgraphs of $G$ and $G$ is a subgraph of $T$, 
\[
  e_{T_i} \leq e_G = e_T  - 1 < e_T  = n + 1
\]
for $i = 1, 2$.
Therefore by induction hypothesis,
\begin{align*}
    e_{T_1} &= v_{T_1} - 1 \\
    e_{T_2} &= v_{T_2} - 1
\end{align*}
Adding these two equations, we get
\begin{align*}
    e_{T_1} + e_{T_2} &= v_{T_1} - 1 + v_{T_2} - 1 \\
    \THEREFORE e_{T_1} + e_{T_2} &= v_{T_1} + v_{T_2} - 2 \tag{a}
\end{align*}
Now note that since $T = G - e$,
the only difference between $T$ and $G$ is an edge.
Hence
\begin{align*}
  e_T &= e_G + 1 = e_{T_1} + e_{T_2} + 1 \tag{b}  \\
  v_T &= v_G = v_{T_1} + v_{T_2} \tag{c}
\end{align*}
Hence equations (a), (b), (c) gives us
\begin{align*}
  e_T
  &= e_{T_1} + e_{T_2} + 1 \\
  &= \left( v_{T_1} - 1 \right) + \left( v_{T_2} - 1 \right) + 1 \\
  &= v_{T_1} + v_{T_2} - 1 \\
  &= v_T - 1
\end{align*}
We have now shown $P(n + 1)$ holds if $P(0), P(1), ..., P(n)$ hold.

Therefore,
by strong mathematical induction, $P(n)$ holds for all $n \geq 0$, i.e.,
we have shown that if $T$ is a tree with at least one node then
\[
  e_T = v_T - 1
\]
\qed


\newpage\textsc{Template for induction proofs}

\textbf{A. Here is the template for weak induction proofs:}

We will prove the above statemet by weak induction.
For $n \geq ?$, let $P(n)$ be the statements
\[
  P(n) = \biggl( n^2 \text{ is a prime} \biggr)
\]

\textsc{Base case}.
We now prove the base case $P(?)$. [... your proof ...]
Hence $P(?)$ holds.

\textsc{Inductive case}.
We now prove the inductive case.
Assume $P(n)$ holds where $n \geq ?$. 
[... your proof ...]
Hence $P(n + 1)$ holds.

Therefore, by weak mathematical induction,
$P(n)$ is true for all $n \geq ?$, i.e.,
for any $n \geq ?$,
\[
  n^2 \text{ is a prime}
\]
    

\textbf{B. Here is the template for strong induction proofs:}

We will prove $P(n)$ is true for all $n \geq ?$ using
strong mathematical induction.
For $n \geq ?$, let $P(n)$ be the statements
\[
  P(n) = \biggl( n^2 \text{ is a prime} \biggr)
\]

\textsc{Base case}.
We now prove the base case $P(?)$. [... your proof ...]
Hence $P(?)$ holds.

\textsc{Inductive case}.
We now prove the inductive case.
Assume $P(?), P(?+1), ..., P(n)$ hold where $n \geq ?$. 
[...]
Hence $P(n + 1)$ holds.

Therefore, by strong mathematical induction,
$P(n)$ is true for all $n \geq ?$, i.e.,
for any $n \geq ?$,
\[
  n^2 \text{ is a prime}
\]

\textbf{B. Here is a template for WOP proofs:}

We will prove the above statement using the well-ordering principle.
For $n \geq ?$, let $P(n)$ be 
\[
  P(n) = \biggl( n^2 \text{ is a prime} \biggr)
\]
Assume on the contrary that $P(n)$ does not hold for all $n \geq ?$.
Then there is some $n$ such that $P(n)$ does not hold.
Define
\[
X = \{ n \geq ? \mid P(n) \text{ does not hold} \}
\]
[... now prove that $X$ is a nonempty subset of $\Z$ that is bounded below,
  or a nonempty subset of $\N$.]

By the well-ordering principle $X$ has a least element, say $m$.
Since $m$ is the least element of $X$,
$P(n)$ holds for all $? \leq n < m$.
[Now derive a contradiction.
  Warning: You usually have to use some $n$ such that 
  $? \leq n < m$.
  This means that you have to explain why $m > ?$.
]
This is a contradiction because [...]

Therefore, $P(n)$ holds for all $n \geq ?$.




\newpage
Q1.
Using mathematical induction, give a complete proof of the following fact:
\[
  4(1^3 + 2^3 + \cdots + n^3) = n^2(n+1)^2
\]
for $n \geq 1$.

\SOLUTION


The goal is to convert a (regular) class to a template class. The class chosen is our
\verb!vec2d! class.

We already have a \verb!vec2d! class.
Objects of this class models 2-dimensional vectors with \verb!double!
coordinates.
Frequently in games (or scientific simulation),
\verb!float!s are sufficient.
Note that \verb!double!s can represent more real numbers than \verb!float!s.
Furthermore in many \lq\lq simple" games such as 2-d games from the 80s,
integer coordinates are enough and integer operation are a lot faster than
\verb!double! or \verb!float! operations. 

The goal of this question is to build a template 2-dimensional vector class,
\verb!vec2!.
You should use the \verb!vec2d! class from our previous assignment.
With this class we
can create 2-dimensional vectors of different numeric types
(\verb!int!, \verb!float!, \verb!double!) like this:
\begin{Verbatim}[frame=single]
vec2< int > u(1, 2);         // u is a 2-d vector with integer coordinates
vec2< float > v(1.2f, 3.4f); // v is a 2-d vector with float coordinates
vec2< double > w(1.2, 3.4);  // w is a 2-d vector with double coordinates
\end{Verbatim}
  
In your \verb!vec2! header file you should include three typedefs:
\begin{tightlist}
\li \verb!vec2i! which is an alias for \verb!vec2< int >!
\li \verb!vec2f! which is an alias for \verb!vec2< float >!
\li \verb!vec2d! which is an alias for \verb!vec2< double >!
\end{tightlist}
Of course this depends on \verb!vec2!,
so these typedefs should be after the \verb!vec2! class.

With these typedefs the above examples become
\begin{Verbatim}[frame=single]
vec2i u(1, 2); // u is a 2-d vector with integer coordinates
vec2f v(1.2f, 3.4f); // v is a 2-d vector with float coordinates
vec2d w(1.2, 3.4); // w is a 2-d vector with double coordinates
\end{Verbatim}

Note that the length function,
\verb!len()!,
must return a \verb!double! regardless of the template type parameter.

Also, note that the argument for \verb!operator[]! is either 0 or 1.
If a value other than 0 or 1 is given, then you must
throw a \verb!ValueError! exception. This class should be included
at the the top of your \verb!vec2.h! file, i.e.,
\begin{console}
#ifndef VEC2_H
#define VEC2_H

class ValueError
{};

...
#endif
\end{console}
and the following will catch a \verb!ValueError! object:
\begin{console}
vec2f v(1, 2);
try
{
    std::cout << v[42] << '\n';
}
catch (ValueError & e)
{
    std::cout << "caught ValueError object\n";
}
\end{console}

Test your code thoroughly.
The test file must be named \verb!testvec2.cpp!.
(The test cases you should include should be very similar to the test code
for \verb!vec2d!.)


  
\newpage
\textsc{Primes}

Note that a number $n$ is said to be a \textbf{prime} if it is a whole number
that can only be divided by $1$ and itself
(i.e., $n$).
The positive integer $n$ is said to be \textbf{composite} if it is
greater than $1$ and is not a prime, which means
that it is possible to write $n$ as a product, $n = a \cdot b$,
where $a$ and $b$ are positive integers such that
$1 < a < n$ and $1 < b < n$.
A positive integer $n$ (positive means $> 0$) must fall into exactly
one of the 3 cases:
\begin{tightlist}
  \li $n$ is 1
  \li $n$ is prime
  \li $n$ is composite.
\end{tightlist}

Note that if $a$ and $b$ are integers and $a > 0$, $b > 0$,
then $a \leq ab$ and $b \leq ab$.
And if $a > 1$, then $b < ab$.




\newpage
Q2.
Consider the following fact:
Every positive integer $n \geq 1$ is a product of primes.
For instance for $n = 20$,
the ordered collection of primes involved are $(2, 2, 5)$ (in ascending order).
We define the product of the empty collection of primes, i.e. $()$, to be $1$.
Of course the collection can have one single number:
If the collection is $(11)$, then the product is 11.

Define $P(n)$ to be the above statement, i.e.,
\[
  P(n) =
  \biggl(
  \text{Every positive integer $n \geq 1$ is a product of primes}
  \biggr)
\]
Give a complete proof of $P(n)$ for all $n \geq 1$
using strong or weak induction.
Here, I'm giving you the $P(n)$ for free.

You can assume Euclid's lemma: If $p$ is a prime dividing $ab$,
then $p$ divides $a$ or $p$ divides $b$.

(In fact not only is every positive integer $n$ representable
as a product of primes,
it is represented as a product of primes in a \textit{unique way}.
In other words, the ordered collection of primes for $n$ is unique.
This can also be proven using induction.
But you don't have to prove it.)

\SOLUTION

%-*-latex-*-
The concept of an array can simulate the notation of a \lq\lq container".
A container is conceptually a data structure (i.e., variable(s) and 
function(s) together) that allows us to hold values and we can either put
values into or get values out from the container. A container starts off by
being empty:
\[[\,\,]\]
Suppose we put $5$ into the container. Conceptually the container now looks
like this:
\[[\,\,5\,\,]\]
At this point, suppose we want to put $9$ into the container. Where will it
go? After putting the value $9$ into the container, the container can be either
\[[\,\,9,\,\,5\,\,]\]
or
\[[\,\,5,\,\,9\,\,]\]

So we need to be more precise about describing the operation. Note also that
our concept of an array has a fixed size whereas a container's size can change.
How do we use an array to model a container? We need an array and another
variable to denote the number of things already placed in the array. Let's call
our array \verb!x! and suppose that it's an array of integer values of size
$5$. Let's say we initialize the array with zeroes. Let's call the number of
things in the container \verb!len! (for length). For instance,

\begin{longtable}{ll}
concept to model & \hspace{3 cm}variables implementing the concept \\
$[\,\,5,\,\,9\,\,]$ & \hspace{3 cm}\verb!x = {5, 9, 0, 0, 0},  len = 2!
\end{longtable}

On the left we see the concept we want to model and on the right we see the
actual implementation in a programming language (for instance C++.) Of course,
array \verb!x! actually has $5$ values. But the \verb!len! variable tells us
that we should only consider \verb!x[0]! and \verb!x[1]! as values we put into
the container; we simply ignore other values
\verb!x[2]!,
\verb!x[3]!,
\verb!x[4]!.
Since we ignore \verb!x[2]!,
\verb!x[3]!,
\verb!x[4]!.
This could work too:

\begin{longtable}{ll}
concept to model & \hspace{3 cm}variables implementing the concept \\
$[\,\,5,\,\,9\,\,]$ & \hspace{3 cm}\verb!x = {5, 9, 78, 79, 80},  len = 2!
\end{longtable}

If the
container has values $5, 3, 6, 2$ then

\begin{longtable}{ll} 
concept to model & \hspace{3 cm}variables implementing the concept \\
$[\,\,5,\,\,3,\,\,6,\,\,2\,\,]$ & \hspace{3 cm}\verb!x = {5, 3, 6, 2, 0},  len = 4!
\end{longtable}

Since \verb!len! has value \verb!4!, we only consider
\verb!x[0]!,
\verb!x[1]!,
\verb!x[2]!, and
\verb!x[3]! values in the container and
ignore \verb!x[4]!.
The following works too:

\begin{longtable}{ll} 
concept to model & \hspace{3 cm}variables implementing the concept \\
$[\,\,5,\,\,3,\,\,6,\,\,2\,\,]$ & \hspace{3 cm}\verb!x = {5, 3, 6, 2, 99},  len = 4!
\end{longtable}

From now on, I will write \verb!?! for values in the array that
we will ignore.
So in the previous example above, I will write this:

\begin{longtable}{ll} 
concept to model & \hspace{3 cm}variables implementing the concept \\
$[\,\,5,\,\,3,\,\,6,\,\,2\,\,]$ & \hspace{3 cm}\verb!x = {5, 3, 6, 2, ?},  len = 4!
\end{longtable}

Remember that when you see \verb!?! it does not mean that there's no
value there -- there \textit{is} a value.
It's just a value we don't care about.

There are two main things you can do to a container:
you can put something into the container and the you
take something out of the container.
Let me explain these two operations.

Let's call the operation to put a value into the container \lq\lq insert".
Initially, the container looks like this:

\begin{longtable}{ll}
concept to model & \hspace{3 cm}variables implementing the concept \\
$[\,\,]$ & \hspace{3 cm}\verb!x = {?, ?, ?, ?, ?},  len = 0!
\end{longtable}

In other words, initially, the container has nothing.

We can insert a value into the container.
To do so, we have to specify the index position.
For instance when we insert a $5$ at index position $0$ we get this:

\begin{longtable}{ll}
concept to model & \hspace{3 cm}variables implementing the concept \\
$[\,\,5\,\,]$ & \hspace{3 cm}\verb!x = {5, ?, ?, ?, ?},  len = 1!
\end{longtable}

Now the container has one value, i.e., $5$ at index 0.
If we now insert $9$ at index position $0$ we get this:

\begin{longtable}{ll}
concept to model & \hspace{3 cm}variables implementing the concept \\
$[\,\,9,\,\,5\,\,]$ & \hspace{3 cm}\verb!x = {9, 5, ?, ?, ?},  len = 2!
\end{longtable}

Notice that $5$ is shifted to the right by one position.
Now if we insert $11$ at index position $1$ we get this:

\begin{longtable}{ll}
concept to model & \hspace{3 cm}variables implementing the concept \\
$[\,\,9,\,\,11,\,\,5\,\,]$ & \hspace{3 cm}\verb!x = {9, 11, 5, ?, ?},  len = 3!
\end{longtable}

If we insert $1$ at index position $3$ we get

\begin{longtable}{ll}
concept to model & \hspace{3 cm}variables implementing the concept \\
$[\,\,9,\,\,11,\,\,5,\,\,1\,\,]$ & \hspace{3 cm}\verb!x = {9, 11, 5, 1, ?},  len = 4!
\end{longtable}

Now let me talk about removing a value from the container.
If we remove the value at position $2$, we get

\begin{longtable}{ll}
concept to model & \hspace{3 cm}variables implementing the concept \\
$[\,\,9,\,\,11,\,\,1\,\,]$ & \hspace{3 cm}\verb!x = {9, 11, 1, ?, ?},  len = 3!
\end{longtable}

Look at the array \verb!x! and \verb!len! carefully. Note that the
\verb!len! $= 3$ tells us that the container we're modeling has three values,
therefore \verb!x[0]! $= 9$, \verb!x[1]! $= 11$, \verb!x[2]! $= 1$ are the
only values in the container; we ignore \verb!x[3]! and \verb!x[4]!.

If we now remove the value at position $0$, we get

\begin{longtable}{ll}
concept to model & \hspace{3 cm}variables implementing the concept \\
$[\,\,11,\,\,1\,\,]$ & \hspace{3 cm}\verb!x = {11, 1, ?, ?, ?},  len = 2!
\end{longtable}

Get it? Note that you can only insert and remove at certain places.
For instance, right now we have

\begin{longtable}{ll}
concept to model & \hspace{3 cm}variables implementing the concept \\
$[\,\,11,\,\,1\,\,]$ & \hspace{3 cm}\verb!x = {11, 1, ?, ?, ?},  len = 2!
\end{longtable}

You can only insert at position $0, 1, 2$ and remove at position $0, 1$.

For this assignment, if you insert or remove not within the valid range, the
variables are not changed. For instance, if we insert $7$ at position $4$ or
remove the value at position $3$, the container is still the same:

\begin{longtable}{ll}
concept to model & \hspace{3 cm}variables implementing the concept \\
$[\,\,11,\,\,1\,\,]$ & \hspace{3 cm}\verb!x = {11, 1, ?, ?, ?},  len = 2!
\end{longtable}

In the general case, if the container has length \verb!len!, you can insert at
index positions $0, 1, 2, \dots$, \verb!len! if \verb!len! is less than the
size of the array and you can remove the value at index position
$0, 1, 2, \dots$, \verb!len! $-\,1$.

The goals of this assignment is to build such a container using an array and a
length variable. 

(Note that if you restrict the usage of the container so that you only insert
at position \verb!len - 1!  and remove only at the index
position \verb!0!, i.e., values go into the container
at one end and go out from the \textit{other end},
you are then simulating a \textbf{queue}.
Queues occur frequently in the real world.
Examples
include a manufacturing line. An industrial engineer is interested in, for
instance, how frequently the queue of a manufacturing
line is congested based on the probabilistic
distribution of items entering and leaving the queue. Queues are also used
for computer network communication.
Now, suppose you only insert at \verb!len - 1! and remove at
\verb!len - 1!, i.e., the values go into the container at one end and also
the container at the \textit{same end}, then you're simulating a
\textbf{stack}.
You have seen that before: think of a stack of plates at a buffet:
plates go into the container of plates at the top and plates
leave the container at the top.
Stacks are called last-in-first-out (LIFO) containers
while queues are called first-in-first-out (FIFO) containers.
Both the stack and queue are very important
and are used extensively in computer science
including artificial intelligence, computer networks, computer graphics, etc.)

The following code skeleton is given. Here we are simulating our container
with a maximum size of $5$. The container contains integer values (the main
idea can be applied to an array of any type of values.) The program continually
prompts the user for an option: $0$ is to quit, $1$ is to insert, $2$ is to
remove. If the user enters an option not in the above choices, the program
reprompts the user. If the user enters $1$, the program prompts the user for
a position and a value to insert into the container. If the user enters $2$,
the program prompts the user for a position and removes the value at the
position. If the user enters an invalid position, the program does not perform
the requested option. If the user attempts to insert a value into the container
when it's already full, no action is taken. See the test cases.

Your functions must work for arrays of any size. Therefore, you should not
assume the array has at most $5$ elements. (See hints below.)

Note that the \verb!insert! and \verb!remove! functions do not only change the
array, they must change the \verb!len! variable as well. You already know that
functions cannot change the values of variables in the calling function.
For instance,

\begin{Verbatim}[frame=single,fontsize=\footnotesize]
#include <iostream>

void inc(int x)
{
    x++;
}

int main()
{
    int a = 5;
    inc(a);
    std::cout << a << std::endl; // value of a is still 5

    return 0;
}
\end{Verbatim}

If you want the function to modify the variable in the caller function, you
make the variable a reference variable. Try this:

\begin{Verbatim}[frame=single,fontsize=\footnotesize]
#include <iostream>

void inc(int & x) // x *will* affect the a in main()
{
    x++;
}

int main()
{
    int a = 5;
    inc(a);
    std::cout << a << std::endl; // a is changed!

    return 0;
}
\end{Verbatim}

Note that parameter \lq\lq \verb!int x!" is changed to \lq\lq \verb!int & x!".
This tells C++ that the parameter will affect the corresponding variable in
the caller function. Note that in the functions \verb!insert()! and
\verb!remove()!, the \verb!len! parameter is a reference variable. Note once
again that our container is described by three variables: the array \verb!x!,
the \verb!len! which denotes the number of things in the container, and
\verb!size! which describes the maximum number of values you can have in the
container.

\begin{Verbatim}[frame=single,fontsize=\footnotesize]
#include <iostream>

void println(int x[], int len)
{
    std::cout << "[ ";
    for (int i = 0; i < len; i++)
    {
        std::cout << x[i] << ' ';
    }
    std::cout << "]" << std::endl; 
}


// Inserts newvalue into container x at position index
void insert(int x[], int & len, int size, int index, int newvalue)
{
}


// Removes the value at position index of the container x
void remove(int x[], int & len, int size, int index)
{
}


int main()
{
    const int SIZE = 5;
    int x[SIZE] = {0};
    int len = 0;
    println(x, len);

    while (1)
    {
        int option = 0;
        std::cout << "option (0-quit, 1-insert, 2-remove): ";
        std::cin >> option;
        // Break the while loop if option is 0        
        
        // Prompt for index
        switch (option)
        {
             case 1:
                 // Prompt for new value and call the insert() function.
                 break;

             case 2:
                 // Call the remove() function.
                 break;
        }

        std::cout << std::endl;
        println(x, len);
    }

    return 0;
}
\end{Verbatim}


\resett
\nextt
\begin{console}[frame=single, commandchars=\\\{\}]
[ ]
option (0-quit, 1-insert, 2-remove): \userinput{0}
\end{console}

\nextt
\begin{console}[frame=single, commandchars=\\\{\}]
[ ]
option (0-quit, 1-insert, 2-remove): \userinput{1}
index: \userinput{7}
value: \userinput{2}

[ ]
option (0-quit, 1-insert, 2-remove): \userinput{2}
index: \userinput{4}

[ ]
option (0-quit, 1-insert, 2-remove): \userinput{0}
\end{console}

\nextt
\vspace{-6pt}
\begin{Verbatim}[frame=single, fontsize=\footnotesize,commandchars=\\\{\}]
[ ]
option (0-quit, 1-insert, 2-remove): \userinput{1}
index: \userinput{0}
value: \userinput{7}

[ 7 ]
option (0-quit, 1-insert, 2-remove): \userinput{1}
index: \userinput{0}
value: \userinput{3}

[ 3 7 ]
option (0-quit, 1-insert, 2-remove): \userinput{1}
index: \userinput{1}
value: \userinput{2}

[ 3 2 7 ]
option (0-quit, 1-insert, 2-remove): \userinput{1}
index: \userinput{3}
value: \userinput{8}

[ 3 2 7 8 ]
option (0-quit, 1-insert, 2-remove): \userinput{1}
index: \userinput{1}
value: \userinput{4}

[ 3 4 2 7 8 ]
option (0-quit, 1-insert, 2-remove): \userinput{2}
index: \userinput{0}

[ 4 2 7 8 ]
option (0-quit, 1-insert, 2-remove): \userinput{2}
index: \userinput{4}

[ 4 2 7 8 ]
option (0-quit, 1-insert, 2-remove): \userinput{2}
index: \userinput{3}

[ 4 2 7 ]
option (0-quit, 1-insert, 2-remove): \userinput{2}
index: \userinput{1}

[ 4 7 ]
option (0-quit, 1-insert, 2-remove): \userinput{2}
index: \userinput{1}

[ 4 ]
option (0-quit, 1-insert, 2-remove): \userinput{2}
index: \userinput{0}

[ ]
option (0-quit, 1-insert, 2-remove): \userinput{2}
index: \userinput{0}

[ ]
option (0-quit, 1-insert, 2-remove): \userinput{2}
index: \userinput{0}

[ ]
option (0-quit, 1-insert, 2-remove): \userinput{0}
\end{Verbatim}


\newpage
\textsc{Hint: SPOILERS AHEAD!!! WATCHOUT!!!}

\begin{itemize}
  \li If you want to insert at a position, you need to move the values
      starting at the index position to the right by one from the index
      position where insertion is to occur. Use a for--loop!!! Once that's
      done, you put the new value at the index position where insertion
      should occur. You should increment the \verb!len! variable. For instance,
      suppose the array has \verb!len! $5$ (and is, say, of \verb!size! $10$)
      and you need to insert a value of $6$ at index position $2$. Say the
      array looks like this:
      \[\verb!x!: 3, 7, 2, 7, 8, ?, ?, ?, ?, ?\]
      You need to write a for--loop to copy the values from index $2$ onward
      to the right by one step:
      \[\verb!x!: 3, 7, 2, 7, 8, ?, ?, ?, ?, ?\]
      \begin{center}
      \begin{tikzpicture}
        \draw[->](0,0.5) -- (0,0);
      \end{tikzpicture}
      \end{center}
      \[\verb!x!: 3, 7, 2, \underline{\mathbf{2, 7, 8}}, ?, ?, ?, ?\]
and then you put $6$ at index $2$:
      \[\verb!x!: 3, 7, \underline{\mathbf{6}}, 2, 7, 8, ?, ?, ?, ?\]

  \li If you want to remove the value at an index position, you move all the
      values to the right of that index position to the left by one index
      position. Once that's done, you need to decrement the \verb!len!
      variable. The idea is very similar to the case of inserting a value
      into the array except that you're moving a chunk of values in the
      opposite direction. For instance, suppose you have the following with
      \verb!len! $6$:
      \[\verb!x!: 3, 7, 6, 2, 7, 8, ?, ?, ?, ?\]
      Then, to remove the value at index position $3$, (i.e., the value $2$)
      you simply copy all the values to the right of $2$
      \[\verb!x!: 3, 7, 6, 2, \underline{\mathbf{7, 8}}, ?, ?, ?, ?\]
      to the left by one to get this:
      \[\verb!x!: 3, 7, 6, \underline{\mathbf{7, 8}}, ?, ?, ?, ?, ?\]
      and decrement the \verb!len! variable by $1$.
\end{itemize}


  
\newpage
Q3.
Consider the following statement:
Every positive integer is a sum of distinct 2-powers.
In other words, there is a set of distinct 2-powers whose sum is the 
given positive integer.
For instance the number 10 can be expressed as
\[
10 = 2^1 + 2^3
\]

(a)
Prove the above statement using strong or weak induction.

(b)
Prove the above statement using the well-ordering principle.

(Note: This is the missing piece of information in CISS360.
In CISS360, we assume that, for instance, all 32-bit unsigned int
can be written as a binary sequence which is a sum of powers of 2
up to the power of 31.
Also, the set of 2-powers adding up to $n$ is unique.
For instance $2^1 + 2^3 = 10$ and if you look at another
sum of 2--powers such as $2^1 + 2^2$, you will not get $10$.
But you don't have to prove it.)

\SOLUTION

%-*-latex-*-
An intercept missile is capable of exploding and disabling an enemy missile if
it is sufficiently close to the latter. The position $(x,y,z)$ of the enemy
missile is continually collected by a radar mounted on the intercept missile.
The distance between two points $(x,y,z)$ and $(0,0,0)$ (the position of the
intercept missile) in $3$--dimensions is given by the formula
\[\sqrt{x^2 + y^2 + z^2}\]
($3$--dimensional Pythagorus formula). Write a program that detects whether the
distance from the enemy missile to the intercept missile is sufficiently close
for the intercept missile to explode. Your program accepts $4$ doubles: The
first three values gives the $(x,y,z)$ position of the enemy missile from the
intercept missile and the last is the maximum distance the intercept should
explode to disable the enemy missile. Your program must set a boolean variable
to true exactly when the intercept missile should explode and then print this
boolean variable.

\resett
\nextt
\begin{console}[commandchars=\\\{\}]
\userinput{2 0 0 1}
0
\end{console}

\nextt
\begin{console}[commandchars=\\\{\}]
\userinput{2 0 0 2.5}
1
\end{console}

\nextt
\begin{console}[commandchars=\\\{\}]
\userinput{1 0 0 1}
1
\end{console}

\nextt
\begin{console}[commandchars=\\\{\}]
\userinput{1 1 0 1.5}
1
\end{console}

\nextt
\begin{console}[commandchars=\\\{\}]
\userinput{1.1 1.2 1.3 2}
0
\end{console}


\end{document}
