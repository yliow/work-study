Q1.
The goal for Q1 is to implement the function prototype
{\small
\begin{console}
void print(const Robot & robot);
\end{console}
}
of \verb!Robot.h!.
Of course the implementation is placed in \verb!Robot.cpp!.
Note that I have already given you the implementation of
{\small
\begin{console}
void init(Robot &, char name, int x, int y, int energylevel);
\end{console}
}
(See \verb!Robot.cpp!.)

To test the \verb!print! function, 
here's the code for \verb!main.cpp!.
{\small
\begin{Verbatim}[frame=single]
#include <iostream>
#include <cstdlib>
#include "Robot.h"

void test_print()
{
    char name;
    int x, y;
    int energylevel;
    std::cin >> name >> x >> y >> energylevel;
    Robot r;
    init(r, name, x, y, energylevel);
    print(r);
    return;
}


int main()
{
    int seed;
    std::cin >> seed;
    srand(seed);

    int option = 0;
    std::cin >> option;
    switch (option)
    {
        case 1:
            test_print();
            break;
    }
    return 0;
}      
\end{Verbatim}
}
So in the folder for this question,
there are three files:
\verb!main.cpp!,
\verb!Robot.h!, and \verb!Robot.cpp!.


Test 1
\begin{console}[commandchars=\\\{\}]
\underline{1 c 1 2 100}
<Robot: name=c, x=1, y=2, energylevel=100>
\end{console}

Test 2
\begin{console}[commandchars=\\\{\}]
\underline{1 d 0 5 42}
<Robot: name=d, x=0, y=5, energylevel=42>
\end{console}


Perform as many test cases as you need.

