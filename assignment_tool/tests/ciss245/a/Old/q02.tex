Q2.
Now provide an implementation of the function
{\small
\begin{console}
void move_north(Robot & robot);
\end{console}
}
function.
(Of course you put this is \verb!Robot.cpp!.)

{\small
\begin{Verbatim}[frame=single]
#include <iostream>
#include <cstdlib>
#include "Robot.h"

void test_print()
{
    char name;
    int x, y;
    int energylevel;
    std::cin >> name >> x >> y >> energylevel;
    Robot r;
    init(r, name, x, y, energylevel);
    print(r);
    return;
}


void test_move_north()
{
    char name;
    int x, y;
    int energylevel;
    std::cin >> name >> x >> y >> energylevel;
    Robot r;
    init(r, name, x, y, energylevel);
    move_north(r);
    print(r);
    return;
}


int main()
{
    int seed;
    std::cin >> seed;
    srand(seed);
    
    int option = 0;
    std::cin >> option;
    switch (option)
    {
        case 1:
            test_print();
            break;         
        case 2:
            test_move_north();
            break;
    }
    return 0;
}      
\end{Verbatim}
}

The test case below will tell you what you need to do.

Test 1
\begin{console}[commandchars=\\\{\}]
\underline{2 c 5 3 100}
<Robot: name=c, x=5, y=2, energylevel=99>
\end{console}
Note that the energylevel drops by 1 after the Robot moves by 1 step.
You need \textit{not} check that the Robot moves outside the world, i.e.,
you need not check that $x$ and $y$ are in the 0..9 range.
However you need to check that the energy level is positive
before allowing the Robot to move.
For instance if the energylevel is 0, calling \verb!move_north!
will not change the position of the Robot.

Perform as many test cases as you need.

