%-*-latex-*-
\textsc{Objectives}
\begin{myenum}
\item Implement tree node class using \verb!std::vector!
\item Implement tree node class using \verb!std::list!
\item Implement binary tree node
\end{myenum}

For this assignment you will be implementing several tree nodes.
The classes are template classes so that the nodes can be used for
different types of data in the node.
(Recall that it's usually a good idea to implement the non-template version
first. You might need to experiment with the non-template version
from time to time. So I recommend you \textit{keep}
the non-template version and develop the template version in parallel.)

Here's a quick description of the classes:
\begin{myenum}
\li \verb!TreeNodev!: The tree node uses a \verb!std::vector! of pointers
for its children 
\li \verb!TreeNodel!: The tree node uses a \verb!std::list! of pointers
for its children.  
\verb!std::list! is the doubly-linked list class in the 
STL.
Use the web to find out more about this class.
(See the first section below on a quick tutorial.)
\li \verb!BinaryTreeNode!: The tree node uses a left and right pointer
to refer to its children.
\end{myenum} 

In the first two cases, 
there is no restriction on the maximum branching factor, i.e., 
the number of children.
If a tree has a maximum branching factor it 
would probably be stored in the corresponding tree classes:
\begin{myenum}
\li \verb!Treev!
\li \verb!Treel!
\li \verb!BinaryTree!
\end{myenum}
However, we won't be using these tree classes in this assignment.

For submission, make sure each question has it's own folder.
For instance for \verb!a09q01!, 
the code must be in folder \verb!a09q01!.
All question folders must be in a folder \verb!a09!.
Submit using alex.

\textsc{Note}: As in CISS245, skeleton code and pseudocode, where given,
are meant to give you ideas. They are not meant to be complete.
