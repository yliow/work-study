%-*-latex-*-
\textsc{Comamnd-line arguments}

Compile this program say the executable is \texttt{main.exe}:
\begin{Verbatim}[frame=single,fontsize=\footnotesize]
#include <iostream>

int main(int argc, char * argv[])
{
    for (int i = 0; i < argc; ++i)
    {
        std::cout << i << " ... " << argv[i] << '\n';
    }
    return 0;
}
\end{Verbatim}
Now in your shell run the program this way:
\begin{Verbatim}[frame=single,fontsize=\footnotesize]
[student@localhost cpp-cmd-line-args]$ g++ main.cpp -o main.exe
[student@localhost cpp-cmd-line-args]$ ./main.exe abc def hello
0 ... ./main.exe
1 ... abc
2 ... def
3 ... hello
\end{Verbatim}

So you see the following: when you issue a command in your
bash shell to run the program \verb!./main.exe! and a sequence
of strings, i.e., \texttt{abc}, \texttt{def}, \texttt{hello},
The strings
\verb!"./main.exe"!,
\verb!"abc"!,
\verb!"def"!, and
\verb!"hello"!
are stored in the array of C-string \texttt{argv} (the \lq\lq argument vector"
where in this case vector means array).
Furthermore the number of such strings (an integer) is stored in
the integer variable \texttt{argc} (the \lq\lq argument count").
Note that the first C-string, \texttt{argv[0]}, is the program path.

By the way, the type of \verb!argv! is an array of pointers.
If you think about that carefully it means that the following works too:
\begin{Verbatim}[frame=single,fontsize=\footnotesize]
#include <iostream>

int main(int argc, char ** argv)
{
    ...
}
\end{Verbatim}
