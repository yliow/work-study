%-*-latex-*-
[Mergesort]

Implement the mergesort algorithm in C\texttt{++}.
The following is the prototype:

\begin{console}[frame=single, fontsize=\footnotesize]
template < typename T >
void mergesort(T * start, T * end, bool verbose=false);

template < typename T, typename C >
void mergesort(T * start, T * end, const C & less, bool verbose=false);
\end{console}

You should probably have the following functions:
\begin{Verbatim}[frame=single, commandchars=\~\!\@, fontsize=\footnotesize]
template < typename T >
void mergesort_(T * start, T * end, std::vector< T > & t,
                bool verbose=false);

template < typename T, typename C >
void mergesort_(T * start, T * end, std::vector< T > & t,
                const C & less, bool verbose=false);

template < typename T >
void merge(T * start0, T * end0,
           T * start1, T * end1,
           std::vector< T > & t);
\end{Verbatim}
Note that in the \verb!mergesort_! function, there's a parameter
\verb!t!.
This is a reference to a vector that is created in
\verb!mergesort!.
The vector that \verb!t! references is used for the merging process
(Recall: I said in class that you do not need to create a temporary
array for every merging process -- one can be shared among the
recursive function calls of \verb!mergesort_!.
The \verb!mergesort! function looks like this:
\begin{Verbatim}[frame=single,commandchars=\~\!\@, fontsize=\footnotesize]
template < typename T >
void mergesort(T * start, T * end, bool verbose=false)
{
    int n = (end - start) / 2;
    std::vector< T > t(n);
    mergesort_(start, end, t, verbose);
}
\end{Verbatim}
The second \verb!mergesort! with the comparator is similar.

Now for the verbose printing of your mergesort.
Suppose your array/vector is \\
\verb!{3, 1, 5, 6, 4, 2}!,
then your output should look like this:
\begin{Verbatim}[frame=single,commandchars=\~\!\@,fontsize=\footnotesize]
mergesort({3, 1, 5, 6, 4, 2})
split({3, 1, 5, 6, 4, 2}) = {3, 1, 5}, {6, 4, 2}
    mergesort({3, 1, 5})
    split({3, 1, 5}) = {3}, {1, 5}
        mergesort({3})
        return {3}
        mergesort({1, 5})
        split({1, 5}) = {1}, {5}
            mergesort({1}) 
            return {1}
            mergesort({5})
            return {5}
        merge({1}, {5}) = {1, 5}
        return {1, 5}
     merge({3}, {1, 5}) = {1, 3, 5}
     return {1, 3, 5}
     mergesort({6, 4, 2})
     split({6, 4, 2}) = {6}, {4, 2}
         mergesort({6})
         return {6}
         mergesort({4, 2})
         split({4, 2}) = {4}, {2}
             mergesort({4})
             return {4}
             mergesort({2})
             return {2}
         merge({4}, {2}) = {2, 4}
         return {2, 4}
     merge({6}, {2, 4}) = {2, 4, 6}
     return {2, 4, 6}
merge({1, 3, 5}, {2, 4, 6}) = {1, 2, 3, 4, 5, 6}
return {1, 2, 3, 4, 5, 6}
\end{Verbatim} 

(Note that the above mergesort function does not
return a value. Rather it \lq\lq returns by reference'', i.e.,
the reference variable is used to hold the sorted vector.
The \texttt{return} in the printout meant
\lq\lq return via the reference
variable''.)

See section on \textsc{Pretty-print Recursion}.
