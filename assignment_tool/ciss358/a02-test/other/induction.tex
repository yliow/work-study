\textsc{Proving $P(n)$ for all $n \geq n_0$ where $n_0$ is a fixed integer}

Recall that if you want to prove
$P(n)$ is true for all $n \geq 42$,
you have to do two things:
\begin{tightlist}
  \li Prove $P(42)$ is true
  \li Let $n \geq 42$. Assume $P(n)$ holds and prove $P(n + 1)$ holds as well.
\end{tightlist}
If you can achieve the above two points, then you can claim
\begin{tightlist}
  \li $P(n)$ is true for all $n \geq 42$
\end{tightlist}
This is one form of mathematical induction called \textbf{weak mathematical induction}.
The \textbf{strong mathematical induction} says that if you can do two things:
\begin{tightlist}
  \li Prove $P(42)$
  \li Let $n \geq 42$. Assume $P(42), P(42 + 1), ..., P(n)$ holds and prove $P(n + 1)$ holds as well.
\end{tightlist}
If you achieve the above two points, then you can claim
\begin{tightlist}
  \li $P(n)$ is true for all $n \geq 42$
\end{tightlist}
The \lq\lq42" above can be replaced by any integer (including some negative integer).

The above two induction techniques proves $P(n)$ are all true for $n \geq n_0$.
Mathematical induction also allows you go \lq\lq go backward".
If you can do the following:
\begin{tightlist}
  \li Prove $P(n_0)$ is true
  \li Let $n \leq n_0$. Assume $P(n)$ holds and prove $P(n - 1)$ holds as well.
\end{tightlist}
then you can claim
\begin{tightlist}
  \li $P(n)$ is true for all $n \leq n_0$
\end{tightlist}
This is also called weak mathematical induction.
The strong induction going backward holds too.
If you can achieve the following:
\begin{tightlist}
  \li Prove $P(n_0)$ is true
  \li Let $n \leq n_0$. Assume $P(n_0), P(n_0 - 1), ..., P(n)$  holds and prove $P(n - 1)$ holds as well.
\end{tightlist}
then you can claim
\begin{tightlist}
  \li $P(n)$ is true for all $n \leq n_0$
\end{tightlist}
So induction allows you to prove $P(n)$ when $n$ goes to infinity or
when $n$ goes to negative infinity.

There is another method that's important and does the same thing for you.

Sometimes, it is possible to prove $P(n)$ for all $n \geq 1$ by using the
\lq\lq proof by contradiction" method by using
the well-ordering principle (WOP) which says that
\begin{tightlist}
  \li Every nonempty subset of $\N = \{0, 1, 2, ...\}$ has a least element,
  i.e., if $X$ is a nonempty subset of $\N$, then there is some
  $m \in X$ such that $m \leq x$ for all $x \in X$.
\end{tightlist}
(See class notes.)
In this case, frequently, the proof involves an argument of the form:
\lq\lq Suppose it's not true that $P(n)$ holds for all $n \geq n_0$".
Then the set
\[
X = \{ n \mid n \geq n_0, \,\,\, P(n) \text{ does not hold}\}
\]
is a nonempty subset of $\N$.
By WOP, $X$ has a least element, i.e.,
there is a smallest $m$ such that $m \geq n_0$ and $P(m)$ is false.
And you continue to prove that something goes wrong,
i.e.,you attempt to arrive at a contradiction.
This is frequently done in one of two ways.
Here's one way to achieve this:
Since $m$ is the least element of $X$,
all $n \in \N$ with $n < m$ must satisfy $P(n)$.
From such $n$, you then show that in fact $P(m)$ holds, which
clearly is a contradiction.
The second method is this:
Try to find some $m' < m$
such that $P(m')$ is also false.
This would contradict the fact that $m$ is the least elment of
$X$.

(\textsc{Aside.}
WOP is related to the proof method called \textbf{Fermat's infinite descent}.
Applying the same argument above on $k'$ and assuming $P(k')$ is false,
you would arrive at another
$k'' < k'$ such that $P(k'')$ is false, etc.
This gives you infinitely many positive integers
$k > k' > k'' > k''' > \cdots$.
This is clearly impossible since there can only be finitely many positive
integer from $n_0$ up to $k$.)

Until you know how to write induction proofs properly, you must follow these
instructions for writing a proof that uses induction.
\begin{tightlist}
\item[1.]
  Paragraph 1: State you $P(n)$ and the range of values for $n$.
  If a problem involves proving multiple statements, you can also use
  $Q(n)$, $R(n)$, etc.
  State what method you are using (weak or strong induction).
  The default is weak induction, i.e., if you want you are using
  mathematical induction, it means you are using weak mathematical induction.
\item [2.]
  Paragraph 2: State you are proving the base case.
  Prove the base case.
\item [3.]
  Paragraph 3: State you are proving the inductive case.
  State your inductive hypothesis and state what you are going to prove.
  Then prove it. If the proof is long, state what you have proven.
  It's even a good idea to state it anyway. That's called good writing:
  State at the beginning of a paragraph what you want to do, do it, then
  remind the reader the goal at the beginning of the paragraph.
  (This is the longest part of the proof. If necessary, you might need
  more than one paragraph.)
\item [4.]
  Paragraph 4:
  State, quoting the method (i.e., induction), what you have proven.
\end{tightlist}
Once you are done with writing about 50 induction proofs, you can use a
freer form.

Here are two examples:

\newpage

\begin{thm}
  If $n \geq 0$, then
  \[
    1 + 2 + \cdots + n = \frac{n(n + 1)}{2}
  \]
  (Note that if $n = 0$, then the expression on the left-hand side of the above
  equation is 0 by definition.
  In other words an empty sum -- sum of no terms -- is defined to be 0.)
\end{thm}

\textit{Proof}.
We will prove this by weak mathematical induction.
For $n \geq 0$, we define
\[
  P(n) = \biggl( 1 + 2 + \cdots + n = \frac{n(n + 1)}{2} \biggr)
\]

\textsc{Base case.}
When $n = 0$, we have
\[
1 + 2 + \cdots + n = 0 = \frac{0(0 + 1)}{2} = \frac{n (n + 1)}{2}
\]
Hence $P(0)$ holds.

\textsc{Inductive case.}
Assume $P(n)$ holds where $n \geq 0$, i.e., we assume
\[
  P(n) = \biggl( 1 + 2 + \cdots n = \frac{n(n + 1)}{2} \biggr)
\]
holds.
We want to show $P(n + 1)$ holds, i.e., we want to show
\[
P(n + 1) =
\biggl(
1 + 2 + \cdots n + 1 = \frac{(n + 1)(n + 1 + 1)}{2}
\biggr)
\]
is true.
Since $P(n)$ holds, we have
\begin{align*}
  1 + 2 + \cdots n
  &= \frac{n(n + 1)}{2} \\
  \THEREFORE 1 + 2 + \cdots n + (n + 1)
  &= \frac{n(n + 1)}{2} + (n + 1) \\
  &= \frac{n(n + 1) + 2(n + 1)}{2} \\
  &= \frac{(n + 1)(n + 2)}{2} \\
  &= \frac{(n + 1)((n + 1) + 1)}{2}
\end{align*}
i.e., $P(n + 1)$ holds.

Therefore, by weak mathematical induction,
$P(n)$ holds for all $n \geq 0$, i.e.,
for $n \geq 0$,
\[
  1 + 2 + \cdots + n = \frac{n(n + 1)}{2}
\]
\qed



\newpage
\begin{thm}
  Let $T$ be a tree wth at least one node,
  i.e., a connected simple graph with at least one node and no cycles.
  Then $e = v - 1$ where $e$ and $v$ are the number of edges and
  nodes of $T$ respectively.
\end{thm}

(Note: A simple graph is a graph with no loops, i.e., no edge
joining a node to itself and no multi-edges, i.e., no multiple edges
joining the same two nodes.
A graph is connected is for every pair of distinct vertices $x,y$,
there is a path of edges from $x$ to $y$.)

\textit{Proof}.
For a graph $G$, we will write $v_G$ and $e_G$ for the number of
nodes and number of edges of $G$ (respectively).
For $n \geq 0$, we define the proposition $P(n)$ as follows:
\[
  P(n) = 
  (
  \text{If $T$ is a tree with at least one node
  \underline{and with $n$ edges}, then $e_T = v_T - 1$}
  )
\]
We will prove $P(n)$ holds for all $n \geq 0$ by strong mathematical induction.

\textsc{Base case}.
We will prove $P(0)$, i.e., we will prove that
if $T$ is a tree with at least one node and 
$0$ edges (i.e., $e_T = 0$), then $e_T = v_T - 1$.
If $T$ has at least two distinct nodes,
say $x$ and $y$, then $x$ and $y$ are not adjacent since there are
no edges in $T$.
But $T$ is connected.
This is a contradiction.
Hence $T$ has exactly one node, i.e., $v_T = 1$.
Therefore
\[
e_T = 0 = 1 - 1 = v_T - 1
\]
i.e., $P(0)$ holds.

\textsc{Inductive case}.
Let $n \geq 0$.
Assume $P(0), P(1), ..., P(n)$ holds.
We will show that $P(n + 1)$ holds, i.e., we will show that
if $T$ is a tree with at least one node and has $n + 1$ edges, then
\[
  e_T = v_T - 1
\]
Let $T$ be a tree with at least one node and $n + 1$ edges.
Since $n \geq 0$, we have $n + 1 \geq 1$.
Hence there is at least one edge in $T$, say
$e$ denote an edge in $T$ joining node $x$ and node $y$.
Since $T$ is simple, $x \neq y$.
Construct the graph
\[
  G = T - e
\]
i.e., $G$ is the graph $T$ with edge $e$ removed.
We have
\begin{align*}
  v_G &= v_T \\
  e_G &= e_T - 1 
\end{align*}
Note that $G$ contains all the nodes of $T$ and in particular
contains $x$ and $y$.
We claim that $G$ is made up of two disjoint trees.

First, we show that $G$ has exactly two connected components, i.e.,
$G$ is made up of two maximal connected subgraphs.
Suppose $k$ is the number of connected components of $G$.
Let $G_1, G_2, ..., G_k$ denote the connected components of $G$.
There are no paths joining a node in $G_i$ to a node in $G_j$ if $i \neq j$.
We will show $k = 2$ and furthermore each $G_i$ is a tree.

The nodes $x$ and $y$ are in $G$.
Therefore $x$ is in some $G_i$ and $y$ is in some $G_j$.
Note that $i \neq j$.
Otherwise, if $i = j$, $x$ and $y$ are in the same connected component
$G_i$ which implies that there is some path $p$ in $G_i$ joining $x$ and $y$.
Since $G_i$ is a subgraph of $G = T - e$ and $G$ does not contain edge $e$,
$G_i$ cannot contain $e$.
The path $p$ (which is in $G_i$) therefore also cannot contain edge $e$.
Hence $p$ and edge $e$ will form a cycle in $T$.
This is a contradiction since $T$ is a tree and cannot have a cycle.
Hence we have shown that $i \neq j$, i.e., $x$ and $y$ are in two different
connnected components of $G$.

Now suppose $k > 2$, i.e.,
suppose there are three connected components.
Recall that $x$ is in some $G_i$ and $y$ is in some $G_j$
with $i \neq j$.
Since there are at least three connected components, there is some
$k$
such that $k \neq i, k \neq j$. 
Let $z$ be a node in $G_k$.
Since $T$ is a tree, there is a path $p$ in $T$ from $x$ to $z$.
There are no repeated nodes in $p$. 
Since $x$ and $z$ are in different connected components,
the path $p$ must leave $G_i$ and must enter $G_k$.
The only edge that leaves $G_i$ is the edge $e$.
Since $y$ is in $G_j$,
path $p$ will leave $G_i$ and enter $G_j$.
Therefore on entering $G_j$, $p$ contains $x$ and $y$.
However to arrive at $G_k$, which is not $G_j$,
the path $p$ has to leave $G_j$.
The only edge leaving $G_j$ is $e$, which means that
the path $p$ on leaving $G_k$ will repeat $x$.
This is a contradiction.
We conclude that $k = 2$, i.e., there are two
connected components in $G = T - e$.

We have now shown that there are two connected components $G_1, G_2$ in $G$.

We now show that the connected components $G_1, G_2$ in $G$
are trees.
Suppose there is a cycle $C$ in $G_1$.
Since $G_1$ is in $G = T - e$, the cycle $C$ is also in $G$
and is therefore in $T$.
This is a contradiction since $T$ is a tree.
(This shows that every subgraph of a tree cannot have cycles.)

We have now shown that $T - e$ is made up of two
disjoint trees, say $T_1$ and $T_2$.

Since $T_1, T_2$ are both subgraphs of $G$ and $G$ is a subgraph of $T$, 
\[
  e_{T_i} \leq e_G = e_T  - 1 < e_T  = n + 1
\]
for $i = 1, 2$.
Therefore by induction hypothesis,
\begin{align*}
    e_{T_1} &= v_{T_1} - 1 \\
    e_{T_2} &= v_{T_2} - 1
\end{align*}
Adding these two equations, we get
\begin{align*}
    e_{T_1} + e_{T_2} &= v_{T_1} - 1 + v_{T_2} - 1 \\
    \THEREFORE e_{T_1} + e_{T_2} &= v_{T_1} + v_{T_2} - 2 \tag{a}
\end{align*}
Now note that since $T = G - e$,
the only difference between $T$ and $G$ is an edge.
Hence
\begin{align*}
  e_T &= e_G + 1 = e_{T_1} + e_{T_2} + 1 \tag{b}  \\
  v_T &= v_G = v_{T_1} + v_{T_2} \tag{c}
\end{align*}
Hence equations (a), (b), (c) gives us
\begin{align*}
  e_T
  &= e_{T_1} + e_{T_2} + 1 \\
  &= \left( v_{T_1} - 1 \right) + \left( v_{T_2} - 1 \right) + 1 \\
  &= v_{T_1} + v_{T_2} - 1 \\
  &= v_T - 1
\end{align*}
We have now shown $P(n + 1)$ holds if $P(0), P(1), ..., P(n)$ hold.

Therefore,
by strong mathematical induction, $P(n)$ holds for all $n \geq 0$, i.e.,
we have shown that if $T$ is a tree with at least one node then
\[
  e_T = v_T - 1
\]
\qed


\newpage\textsc{Template for induction proofs}

\textbf{A. Here is the template for weak induction proofs:}

We will prove the above statemet by weak induction.
For $n \geq ?$, let $P(n)$ be the statements
\[
  P(n) = \biggl( n^2 \text{ is a prime} \biggr)
\]

\textsc{Base case}.
We now prove the base case $P(?)$. [... your proof ...]
Hence $P(?)$ holds.

\textsc{Inductive case}.
We now prove the inductive case.
Assume $P(n)$ holds where $n \geq ?$. 
[... your proof ...]
Hence $P(n + 1)$ holds.

Therefore, by weak mathematical induction,
$P(n)$ is true for all $n \geq ?$, i.e.,
for any $n \geq ?$,
\[
  n^2 \text{ is a prime}
\]
    

\textbf{B. Here is the template for strong induction proofs:}

We will prove $P(n)$ is true for all $n \geq ?$ using
strong mathematical induction.
For $n \geq ?$, let $P(n)$ be the statements
\[
  P(n) = \biggl( n^2 \text{ is a prime} \biggr)
\]

\textsc{Base case}.
We now prove the base case $P(?)$. [... your proof ...]
Hence $P(?)$ holds.

\textsc{Inductive case}.
We now prove the inductive case.
Assume $P(?), P(?+1), ..., P(n)$ hold where $n \geq ?$. 
[...]
Hence $P(n + 1)$ holds.

Therefore, by strong mathematical induction,
$P(n)$ is true for all $n \geq ?$, i.e.,
for any $n \geq ?$,
\[
  n^2 \text{ is a prime}
\]

\textbf{B. Here is a template for WOP proofs:}

We will prove the above statement using the well-ordering principle.
For $n \geq ?$, let $P(n)$ be 
\[
  P(n) = \biggl( n^2 \text{ is a prime} \biggr)
\]
Assume on the contrary that $P(n)$ does not hold for all $n \geq ?$.
Then there is some $n$ such that $P(n)$ does not hold.
Define
\[
X = \{ n \geq ? \mid P(n) \text{ does not hold} \}
\]
[... now prove that $X$ is a nonempty subset of $\Z$ that is bounded below,
  or a nonempty subset of $\N$.]

By the well-ordering principle $X$ has a least element, say $m$.
Since $m$ is the least element of $X$,
$P(n)$ holds for all $? \leq n < m$.
[Now derive a contradiction.
  Warning: You usually have to use some $n$ such that 
  $? \leq n < m$.
  This means that you have to explain why $m > ?$.
]
This is a contradiction because [...]

Therefore, $P(n)$ holds for all $n \geq ?$.

