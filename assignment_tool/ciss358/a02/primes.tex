\textsc{Primes}

Note that a number $n$ is said to be a \textbf{prime} if it is a whole number
that can only be divided by $1$ and itself
(i.e., $n$).
The positive integer $n$ is said to be \textbf{composite} if it is
greater than $1$ and is not a prime, which means
that it is possible to write $n$ as a product, $n = a \cdot b$,
where $a$ and $b$ are positive integers such that
$1 < a < n$ and $1 < b < n$.
A positive integer $n$ (positive means $> 0$) must fall into exactly
one of the 3 cases:
\begin{tightlist}
  \li $n$ is 1
  \li $n$ is prime
  \li $n$ is composite.
\end{tightlist}

Note that if $a$ and $b$ are integers and $a > 0$, $b > 0$,
then $a \leq ab$ and $b \leq ab$.
And if $a > 1$, then $b < ab$.

