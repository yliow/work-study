%-*-latex-*-
%-*-latex-*-
\input{mybookpreamble.tex}
\input{yliow}
\textwidth=5.5in

%-*-latex-*-
%\usepackage{makeidx}
%\usetikzlibrary{shapes.geometric}
%\usetikzlibrary{arrows}
\usetikzlibrary{fit}
%\usetikzlibrary{positioning}

\renewcommand\debug[1]{#1}
%\renewcommand\debug[1]{}

\newcommand\starprob{ * }
\newcommand\bangprob{ ! }


\makeindex
\begin{document}
\topmatter


\renewcommand\AUTHOR{jdoe5@cougars.ccis.edu} % CHANGE TO YOURS

\begin{document}
\topmattertwo


In any of the following questions, write ERROR if there's
an error in the code fragment.

\nextq
The following program does not run.
Insert \textit{one} statement to correct the problem.
\\
\textsc{Answer:}\vspace{-2mm}
\begin{answercode}
#include <iostream>

int sum(int start, int end, int step);

int main()
{
    std::cout << sum(5, 10, 1) << std::endl;
    return 0;
}

int sum(int start, int end, int step)
{
    int s = 0;
    for (int i = start; i <= end; i += step)
    {
        s += i;
    }
    return s;
}
\end{answercode}

%------------------------------------------------------------------------------
\nextq
The function \verb!play_audio()! plays music file an audio file. The parameters are:
\begin{enumerate}[nosep]
\li \verb!filename!: a C-string that is the name of the audio file to be load
\li \verb!track_number!: an integer.
$0$ is the first track, $1$ is the second track, etc.
The default value is $-1$ which is \lq\lq play all tracks".
\li \verb!repeat!: a boolean. The default is \verb!false!.
\end{enumerate}
The function returns a $0$ if there are no errors,
a $-1$ if the file cannot be found using the \verb!filename!,
$-2$ if the \verb!track_number! is not $-1$ and the track number is not found
in the audio file.
Write down the prototype of this function.
\\
\textsc{Answer:}\vspace{-2mm}
\begin{answercode}
int play_audio(char filename[], int track_number, bool repeat = false);
\end{answercode}

%------------------------------------------------------------------------------
\nextq
The following function call
\begin{console}
push_back(x, x_len, 42);
\end{console}
sets \verb!x[x_len]! to \verb!42! and increments \verb!x_len! by 1.
Write down the function prototype of \verb!push_back!.
\\
\textsc{Answer:}\vspace{-2mm}
\begin{answercode}
void push_back(int x[], int & x_len, int value);
\end{answercode}

%------------------------------------------------------------------------------
\newpage
\textsc{Instructions}
\begin{enumerate}[topsep=0pt,nosep]
\li Your program must be well-written. 
    You must follow the style in your notes as closely as possible. 
    Take note of the spaces and blank lines I used in my examples. 
    Badly written programs will very likely result in a poor grade for this 
    assignment. 
    Points will be taken off for sloppy work. 
\li It's important to remember this: In your printouts for all assignments, 
    there must be no wraparound.
\li All outputs must match exactly the output shown. 
    That includes every single space and every blank line.

\li The format of your program must look like this
(replacing \lq\lq smaug'' with your name of course!):
\begin{Verbatim}[frame=single,fontsize=\small]
// File: a01q01.cpp
// Name: smaug

#include <iostream>

int main()
{
    *** YOUR WORK HERE ***

    return 0;
}
\end{Verbatim}
In particular:
\begin{myenum}
\li You must have your name and the name of the file at the top of each 
    C++ source file as shown above.
\li Your C++ source file must end with a blank line.
\end{myenum}

\li Instructions on uploading your work will be provided in class.

\end{enumerate}


Read the questions carefully before diving in!

Note that you should create a new project for each question. 
For easy maintenance of your assignments, 
I suggest you have a folder \verb!ciss240! somewhere in your 
\verb!Documents!, and in that you have a folder \verb!a!, 
and in folder \verb!a! you have a folder \verb!a01!, 
and you have solutions folders \verb!a01q01!, \verb!a01q02!, etc. in the 
folder \verb!a01!:

\begin{Verbatim}
    .
    .
    .
    ciss240
    |
    +--- a
         |
         +--- a01
              |
              +--- a01q01
              |
              +--- a01q02
\end{Verbatim}

Note that the name for the C++ source file for question 1 
(i.e. the cpp file in 
project \verb!a01q01!) must be \verb!a01q01.cpp!, etc.

\end{document}