%-*-latex-*-
\newcommand\COURSE{ciss350}
\newcommand\ASSESSMENT{a01}
\newcommand\ASSESSMENTTYPE{Assignment}
\newcommand\POINTS{	extwhite{xxx/xxx}}

\input{myquizpreamble}
\input{yliow}
\renewcommand\TITLE{\ASSESSMENTTYPE \ \ASSESSMENT}

\renewcommand\EMAIL{}
\input{\COURSE}
\textwidth=6in

 

\newcommand\BLANK{\sqcup}
% Used in DFA minimization
\newcommand\ind{\operatorname{index}}


\makeindex
\begin{document}
\topmatter


\renewcommand\AUTHOR{jdoe5@cougars.ccis.edu} % CHANGE TO YOURS

\begin{document}
\topmattertwo


\nextq
What is the output of the following program:
\begin{console}
#include <iostream>

void change(int x[], int x_len, index, value)
{
    x[index] = value;
}

int main()
{
    int x[1000] = {2, 3, 5};
    int x_len = 3;
    change(x, xlen, 2, 7);
    std::cout << x[2];
    return 0;
}
\end{console}
\\
\textsc{Answer:}\vspace{-2mm}
\begin{answercode}

\end{answercode}

%------------------------------------------------------------------------------
\nextq
The purpose of \verb!push_back! is to \lq\lq extend" the array:
\begin{console}
#include <iostream>

void push_back(int x[], int x_len, value)
{
    x[x_len] = value;
    ++x_len;
}

int main()
{
    int x[1000] = {2, 3, 5};
    int x_len = 3;
    push_back(x, x_len, 7);

    std::cout << x_len           // should be 4
              << ' '        
              << x[3] << '\n';   // should be 7
    return 0;
}
\end{console}
But the function does not work.
Correct the function if necessary
\\
\textsc{Answer:}\vspace{-2mm}
\begin{answercode}
#include <iostream>

void push_back(int x[], int x_len, value)
{
    x[x_len] = value;
    ++x_len;
}

int main()
{
    int x[1000] = {2, 3, 5};
    int x_len = 3;
    push_back(x, x_len, 7);

    std::cout << x_len           // should be 4
              << ' '        
              << x[3] << '\n';   // should be 7
    return 0;
}
\end{answercode}

%------------------------------------------------------------------------------
\nextq
Correct the following function if necsssary.
\\
\textsc{Answer:}\vspace{-2mm}
\begin{answercode}
#include <iostream>

void f(int x)
{
    ++x;
}

void g(const int & x)
{
    ++x;
}

int main()
{
    int a = 0;
    f(a);      // on return, a should be 1
    g(a);      // on return, a should be 2
    return 0;
}
\end{answercode}

%------------------------------------------------------------------------------
\nextq
Write down the output or write ERROR if there's an error in the
code fragment.
\begin{Verbatim}[frame=single,fontsize=\footnotesize]
int a = 0;
int b = 1;
int & c = a;
int & d = b;
const int & e = d;
a = 2;
c = 3;
d = 4;
std::cout << a + b + c + d + e;
\end{Verbatim}
\textsc{Answer:}\vspace{-2mm}
\begin{answercode}

\end{answercode}

%------------------------------------------------------------------------------
\nextq
Write down the output or write ERROR if there's an error in the
code fragment.
\begin{Verbatim}[frame=single,fontsize=\footnotesize]
int a = 0;
int b = 1;
int & c = a;
int & d = b;
const int & e = d;
a = 2;
d = 3;
e = 4;
std::cout << a + b + c + d + e;
\end{Verbatim}
\textsc{Answer:}\vspace{-2mm}
\begin{answercode}

\end{answercode}

%------------------------------------------------------------------------------
\newpage
You are given the following (possibly incomplete files):
\begin{tightlist}
  \li \texttt{Rational.h}
  \li \texttt{Rational.cpp}
  \li \texttt{main.cpp} (the test code)
\end{tightlist}
\textsc{Important Warning:}
Again, the files are meant to be skeleton file and might not be
complete and might have deliberate missing details or even errors.

Create directory
\texttt{ciss245/a/a07/a07q01}.
Keep all your files in this directory.

If you're doing a copy-and-paste of the given code,
note that some character might be changed by PDF to other characters.
In particular the - character might actually not be the dash character.
Looking at the compiler error message will help you find these minor
annoying issues so that you can correct them.

Study the given test code.
Add tests if necessary to test all methods and functions.
Such low level function/method tests are called \defterm{unit tests}.

Observe the following very carefully:
\begin{tightlist}
\li All methods must be constant whenever possible. 
\li All parameters which are objects (or struct variables)
must be pass by reference or pass by constant reference as much as possible.
\li Reuse code as much as possible.
For instance \verb@operator!=()@ should use \verb!operator==()!. 
\end{tightlist}
Let me know ASAP if you see a typo.

\end{document}