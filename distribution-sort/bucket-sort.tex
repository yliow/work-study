%-*-latex-*-
\sectionthree{Bucket sort}
\begin{python0}
from solutions import *; clear()
\end{python0}

The bucket sort is very similar to the radix sort.

Recall that when I was talking about the radix sort, I use
10 linked lists as queues to sort integers in each pass of the radix sort.
I'm going to do it again, but in a different way.

It's really pretty easy.
Suppose I know that my arrays always have values from 0 to 99.
Then I can create 10 bucket.
I scan the array and if a value of the element
has value from 0 to 9, I put it into bucket 0.
If the value is from 10 to 19, I put it into bucket has 1. 
Etc.
If the number of values is a bucket is small,
I can use any suitable sorting algorithm to sort values in the bucket.
If there are too many values, of course I can use a suitable 
sorting algorithm, including the bucket sorting algorithm,
So suppose for my first bucket that holds values 0 to 9, I have 1000.
Since there are only 10 possible values, I would choose the counting sort
and sort the bucket. 
That's it.

But what if the values are from 0 to 999999999?
I can have have 10 buckets again and if the 
range of values of the buckets are the same, then
the first bucket would hold values from 0 to 99999999.
Unfortunately, it might not be a good idea to use the counting sort.
You have to decide on what other sorting algorithm to use.
If you use bucket sort again, then, of course each bucket this
time will have a smaller range of values.
If you do this recursively, of course, at some point, the number of
values in a bucket will shrink -- except for the case when there's clustering,
i.e., there are lots of values clustered
together.

Assume the values are uniformly distributed
and you want to minimize memory usage, you can use a linked list (see chapter
on linked list) for each bucket.
The space complexity is minimal when compared to using vectors.
Note that insertion sort can be performed on linked lists (the buckets in
this case).
Then the values from the buckets can be copied back to the original input
array.
If $n$ linked list are used and the keys are uniformly distributed
over an interval $[a, b]$, then each buckey can be used to collect
keys in a subinterval or length $(b - a)/n$.

