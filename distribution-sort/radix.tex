%-*-latex-*-
\section{LSD radix sort}

There are two basic variants of radix sort.
First I'll talk about the LSD (least significant digit) radix sort.


Suppose you have the following array (I'm going to draw it vertically):
\begin{python}
from latextool_basic import *
p = Plot()

def arr(x=0, y=0, xs=[]):
    xs = [[_] for _ in xs]
    return Array2d(x, y, xs, width=0.7, height=0.7)

def string(x=0, y=0, label=''):
    label = r'\text{\texttt{%s}}' % label
    return Rect2(x, y, x, y, label=label, linewidth=0)

a = arr(xs=[31,22,'02',25,34,41,'09'])
p += a
x, y = a[0][0].left()
p += string(x-0.3, y, 'x')
print(p)
\end{python}
First we sort the array \textit{by the ones digit} using a 
\textit{stable} sorting algorithm (example: mergesort).
\begin{python}
from latextool_basic import *
p = Plot()

def arr(x=0, y=0, xs=[]):
    xs = [[_] for _ in xs]
    return Array2d(x, y, xs, width=0.7, height=0.7)

def string(x=0, y=0, label=''):
    label = r'\text{\texttt{%s}}' % label
    return Rect2(x, y, x, y, label=label, linewidth=0)

def bubble(xs, f):
    n = len(xs)
    #print "xs:", xs
    for i in range(n - 2, -1, -1):
        for j in range(0, i + 1):
            if f(xs[j]) > f(xs[j + 1]):
                xs[j], xs[j + 1] = xs[j + 1], xs[j]
        #print "xs:", xs

a = ['31','22','02','25','34','41','09']
a = [(_[-1],_) for _ in a]
bubble(a, lambda x:x[0])
a = [_ for __,_ in a] 
a = arr(xs=a)
p += a
x, y = a[0][0].left()
p += string(x-0.3, y, 'x')
print(p)
\end{python}
Because the sorting is stable, note for instance that 
\verb!22! appear before \verb!02! before and after the sort.

Next I'm going to perform a stable sort on the tens digit:
\begin{python}
from latextool_basic import *
p = Plot()

def arr(x=0, y=0, xs=[]):
    xs = [[_] for _ in xs]
    return Array2d(x, y, xs, width=0.7, height=0.7)

def string(x=0, y=0, label=''):
    label = r'\text{\texttt{%s}}' % label
    return Rect2(x, y, x, y, label=label, linewidth=0)

def bubble(xs, f):
    n = len(xs)
    #print "xs:", xs
    for i in range(n - 2, -1, -1):
        for j in range(0, i + 1):
            if f(xs[j]) > f(xs[j + 1]):
                xs[j], xs[j + 1] = xs[j + 1], xs[j]
        #print "xs:", xs

a = ['31','22','02','25','34','41','09']
a = [(_[-1],_) for _ in a]
bubble(a, lambda x:x[0])
bubble(a, lambda x:x[1])
a = [_ for __,_ in a] 
a = arr(xs=a)
p += a
x, y = a[0][0].left()
p += string(x-0.3, y, 'x')
print(p)
\end{python}

Note for each pass of the radix sort, you can use any stable sort.
You want to pick the fastest (of course): so of course mergesort is better than bubblesort!
If you are using mergesort to sort the ones and then the tens, then the
runtime is
\[
  O(2 n \lg n) = O(n \lg n)
\]
If there are $d$ digits (instead of $2$), then the runtime is
\[
  O(d n \lg n)
\]
If most cases, the $d$ might be a constant (i.e., does not change).
For instance if your array is an array of integers in a 32-bit
computer, then $d = 10$.
In this case the runtime is
\[
  O(d n \lg n) = O(n \ln n)
\]

For radix sort, when you look at the key \verb!31! (the first value of the
original array) and you are sorting based on the first digit of \verb!31!,
i.e., \verb!1!, we say that \verb!1! is the \defone{subkey}.
If there's no digit there, for instance look at the key \verb!02! which is
actually \verb!2!, then the second digit subkey of \verb!2! = \verb!02! is
\verb!0!, i.e., the smallest possible subkey value (if you sorting in
ascending order). 

For implementation, you might want to have a \verb!subkey! function.
\verb!subkey(key, i)! returns the $i$--th digit of \verb!key! where
$i$ starts with $0$ (the leftmost digit).
If you call \verb!subkey(31, 0)!, you get \verb!1!.
If you call \verb!subkey(31, 1)!, you get \verb!3!.
Remember that since \verb!2! has only one digit,
if you call \verb!subkey(2, 1)!, you get \verb!0!.

There's another way to do this with a better runtime.

Recall the concept of a queue, a data structure there data enters at
the tail of the data structure and leave through the head of the data structure:

\begin{python}
from latextool_basic import *
from latexcircuit import *
p = Plot()
r = Rect(x0=0, y0=0, x1=5, y1=0.8)
p += r

x,y = r.right()
p += Line(points=[(x+2,y), (x+0.2,y)], endstyle='>', linewidth=0.1)

x,y = r.left()
p += Line(points=[(x-0.2,y), (x - 2,y)], endstyle='>', linewidth=0.1)

x,y = r.bottomleft()
X = POINT(x=x, y=y, r=0, label='head', anchor='north west')
p += str(X)

x,y = r.bottomright()
X = POINT(x=x, y=y, r=0, label='tail', anchor='north east')
p += str(X)

print(p)
\end{python}

We saw the queue in CISS245 where we implemented it using an array
(say dynamic arrays in the heap).
The worst runtimes are insert and delete are
\begin{tightlist}
\item entering the queue: $T(n) = O(n)$  
\item leaving the queue: $T(n) = O(n)$ 
\end{tightlist}
where $n$ is the size of the queue.

There's a better way to implement a queue ...

After we study linked list, you will see that you can implement a queue using
linked lists.
The operations 
on a queue implemented using a linked list has the following runtimes:
\begin{tightlist}
\item entering the queue: $T(n) = O(1)$ 
\item leaving the queue: $T(n) = O(1)$ 
\end{tightlist}
where $n$ is the size of the queue.

We then sort based on the digit value in the following way:
First we create an array \verb!q! of 10 queues.
\verb!q[0]! is a queue that will hold values with digit value of \verb!0!,
\verb!q[1]! is a queue that will hold values with digit value of \verb!1!, etc.
For each value of \verb!x!, we place the value into the appropriate queue
based on the digit of that value.
When we're done, we remove all the values in \verb!q[0]! and place them
in order into \verb!x!.
We then repeat with \verb!q[1]!, etc.
Because the operations on the queue all have runtime of $O(1)$,
the total runtime is $O(n)$.

We then repeat the process, but now determine which queue a value of \verb!x!
should go into by the tens digit of that value.
Altogether the runtime is $O(2n)$.

In our example above, the numbers have 2 digits.
If the numbers have $d$ digits, then the runtime is $O(dn)$.
If the number of digits $d$ is fixed (i.e., is a constant),
then the runtime is $O(n)$.

For space, the total amount of memory used by the queues 
adds up to 
$O(n)$.
So although the queue-based version of radix sort is faster,
it's at the expense of memory.
Remember that.
Nothing is for free.

Note that by using queues, 
the sorting process is stable:
a number that has joined a queue will always be ahead of another
that joined the queue later.
After each pass, the values of a queues are dequeued and put back 
into the original array, of course starting with 
the queue corresonding to \verb!0!.

Note that when we use queues for distribution, a value
enters the queue without being compared when other values.
Therefore in this context radix sort is a distribution (non-comparison) sorting
algorithm.

Another thing to note is that we are assuming all the numbers
we are sorting have the same number of digits.
What if the number of digits are different?
As mentioned above, if you call \verb!subkey(key, i)!
and \verb!key! has no digit at \verb!i!,
\verb!subkey(key, i)! returns the smallest possible digit value, i.e., \verb!0!.

The above starts with the \lq\lq smallest" digit (i.e., the ones digit).
That's why it's called \defone{least signicant digit (LSD) radix sort}.

\newpage

\begin{ex}
  Perform LSD radix sort on the following values, showing the
  values of the array after each pass:
  \[
    121, 23, 231, 89, 123, 536, 697, 457, 355, 243, 657, 345, 174, 248, 346, 376
  \]
  \qed
\end{ex}

\begin{ex}
What happens when you sort not by single digit for each pass, but
\textit{two} digits?
\qed
\end{ex}



\newpage
\section{MSD radix sort}

Instead of sorting numbers by sorting on their digits from right to left, i.e.,
starting with the least significant digit,
you can do the same thing but going left to right.
That's \defone{MSD radix sort} (MSD = most significant digit).
If you
sort numbers using MSD
radix sort on
\[
22, 19, 31, 91, 90, 10, 45
\]
after the first pass (i.e., after stable sorting on the tens), you 
you will see this:
\[
19, 10, 22, 31, 45, 91, 90
\]
Now we (stable) sort by the ones:
\[
10, 90, 31, 91, 22, 45, 19
\]




For numbers with different lengths, you can
pad on the right with an extra \lq\lq digit'' that is infinitely small.
For instance, say you pad the above on the right with \verb!?!
and assume that $? < 0 < 1 < 2 < 3 < 4 < 5 < 6 < 7 < 8 < 9$ to get
\[
2?, 1?, 3?, 91, 9?, 19, 45
\]
Then after performing a stable sorting on the leftmost digit, I get
\[
1?, 19, 2?, 3?, 45, 91, 9?, 
\]
and on after the second pass of stable sorting on the second digit, I get
\[
1?, 19, 2?, 3?, 45, 9?, 91  
\]
After the array is sorted, you can remove this extra data \verb!?! to get
\[
1, 19, 2, 3, 45, 9, 91  
\]

If you think that's bizarre, then think again -- you have
seen this before.
Let's look at strings.
Suppose you have an array of string
\verb!"apple"!,
\verb!"abe"!,
\verb!"cat"!,
\verb!"ab"!,
\verb!"bee"!,
\verb!"cap"!.
Using MSD radix sort, you would get
\[
\texttt{"ab",
"abe",
"apple",
"bee",
"cap".
"cat"}
\]
where the order of each character is determined by its ASCII code.

The above ordering is called \defterm{dictionary order}. Duh.

Note that you  don't really need to \lq\lq pad on the right''.
For instance when comparing strings using the dictionary ordering
on C-strings $x$ and $y$ to compute the boolean value of
$x < y$,
you run a loop comparing \verb!x[i]! and \verb!y[i]!.
If \verb!i! is less than the length of \verb!x! but $\geq$ the length of
\verb!y!,
you return true.
If \verb!i! is less than the length of \verb!y! but $\geq$ the length of
\verb!x!,
you return false.
If \verb!x[i]! $<$ \verb!y[i]!, you return true.
If \verb!x[i]! $>$ \verb!y[i]!, you return false.
If \verb!x[i]! \verb!==! \verb!y[i]!, you increment \verb!i! and
continue on to the next iteration of the loop.
