%-*-latex-*-
\section{Distribution sort: counting sort}

Beside sorting an array using a comparison-based sorting
algorithm, i.e., using $\leq$ or $<$ or $\geq$ or $>$ to compare values in the array,
there's
another class of sorting algorithms that work in a difference way.
Distribution sorting algoroithms sort by \textit{distributing}
the values in an array.
Sometimes distribution sort will use a comparison-based sort.

The simplest distribution sort is the counting sort.

Suppose you have an array \verb!x! of integers with values
from 0 to 4 (inclusive):

\begin{python}
from latextool_basic import *
p = Plot()

def arr(x=0, y=0, xs=[]):
    return Array2d(x, y, [xs], width=0.7, height=0.7)
def string(x=0, y=0, label=''):
    label = r'\text{\texttt{%s}}' % label
    return Rect2(x, y, x, y, label=label, linewidth=0)
a = arr(xs=[3,2,0,2,3,4,0,3,2,4])
p += a
x, y = a[0][0].left()
p += string(x-0.3, y, 'x')
print(p)
\end{python}

You scan the array and count the number of times each value
from 0 to 4 occurs in \verb!x!.

\begin{python}
from latextool_basic import *
p = Plot()

def arr(x=0, y=0, xs=[]):
    return Array2d(x, y, [xs], width=0.7, height=0.7)
def string(x=0, y=0, label=''):
    label = r'\text{\texttt{%s}}' % label
    return Rect2(x, y, x, y, label=label, linewidth=0)
a = arr(xs=[2,0,3,3,2])
p += a
x, y = a[0][0].left()
p += string(x-0.8, y, 'count')
print(p)
\end{python}

For instance \verb!count[2]! is the number of times \verb!2!
occurs in array \verb!x!.

Now the final thing to do is to fill \verb!x! using
information in \verb!count!.
For instance \verb!count[0]! is \verb!2!, 
so I would put two \verb!0!'s into \verb!x!:

\begin{python}
from latextool_basic import *
p = Plot()

def arr(x=0, y=0, xs=[]):
    return Array2d(x, y, [xs], width=0.7, height=0.7)
def string(x=0, y=0, label=''):
    label = r'\text{\texttt{%s}}' % label
    return Rect2(x, y, x, y, label=label, linewidth=0)
a = arr(xs=[0,0,'','','','','','','',''])
p += a
x, y = a[0][0].left()
p += string(x-0.3, y, 'x')
print(p)
\end{python}

Next, since \verb!count[1]! is \verb!0!, \verb!1! does not appear in 
\verb!x!.
So the re-organized \verb!x! looks like this (i.e., no change):

\begin{python}
from latextool_basic import *
p = Plot()

def arr(x=0, y=0, xs=[]):
    return Array2d(x, y, [xs], width=0.7, height=0.7)
def string(x=0, y=0, label=''):
    label = r'\text{\texttt{%s}}' % label
    return Rect2(x, y, x, y, label=label, linewidth=0)
a = arr(xs=[0,0,'','','','','','','',''])
p += a
x, y = a[0][0].left()
p += string(x-0.3, y, 'x')
print(p)
\end{python}

However note that \verb!count[2]! is \verb!3!, which means that
\verb!2! appears \verb!3! times.
Therefore my \verb!x! now look like this:

\begin{python}
from latextool_basic import *
p = Plot()

def arr(x=0, y=0, xs=[]):
    return Array2d(x, y, [xs], width=0.7, height=0.7)
def string(x=0, y=0, label=''):
    label = r'\text{\texttt{%s}}' % label
    return Rect2(x, y, x, y, label=label, linewidth=0)
a = arr(xs=[0,0,'2','2','2','','','','',''])
p += a
x, y = a[0][0].left()
p += string(x-0.3, y, 'x')
print(p)
\end{python}

You get the point.

The pseudocode looks like this.
I assume that the values in \verb!x! are from 0 to \verb!M!.
\begin{console}
ALGORITHM: counting_sort
INPUT: array x of size n

    count = int array of size M + 1, initialized with 0s
    for i = 0, 1, 2, ..., n - 1:
        count[x[i]] = count[x[i]] + 1
    j = 0
    for i = 0, 1, 2, ..., M:
        for k = 1, ..., count[i]:
            x[j] = i
            j = j + 1
\end{console}

If \verb!M! is not 
known, you have to run through \verb!x! to figure that out.


Here's the runtime analysis.
The time to create \verb!count!, if it's in the heap,
might very well be $O(1)$ (or the memory allocation might fail because of lack of
available memory). So that's hard to say.
So let's just assume memory allocation for \verb!count! is negligible.
First you have to set the \verb!count! array to zeroes:
that takes $A(M + 1)$ ($A$ is a constant).
You need to run through \verb!x! to fill \verb!count! with
values: that takes a time of $Bn$.
You will to put values back into \verb!x! using \verb!count!.
That requires touching each cell in \verb!x!.
So for this part, the runtime is $Cn$.
Altogether, the runtime is
\[
O(n + M)
\]
Of course the space requirement is
\[
O(M + 1) = O(M)
\]

\newpage

\begin{ex}
  How would you sort a huge array if the values are integer in the interval $[-10, 10]$?
  \qed
\end{ex}


\begin{ex}
  How would you sort a huge array if the values are
  1000000, 1100000, 1200000, 1300000, ..., 1900000.
  \qed
\end{ex}


\begin{ex}
  Modify the counting sort algorithm to sort a huge array of \verb!double!s
  where there is a (known) small number of distinct \verb!double! values in the 
  array.
  Say the size of the array is $10^6$ and there are at most $1000$ distinct
  \verb!double!s in the array.
  What if the values are not known ahead of time?
  (The idea should also work for other types.)
  \qed
\end{ex}


\begin{ex}
Take a look at the sieve of Eratosthenes and you'll see that it's related to 
the counting sort.
\qed
\end{ex}
