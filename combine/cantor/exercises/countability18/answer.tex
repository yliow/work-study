\tinysidebar{\debug{exercises/{countability18/answer.tex}}}
Note that
\[
0.99999\ldots
\]
and
\[
1.00000\ldots
\]
represent the same real number.
Also,
\[
0.1234599999\ldots
\]
and
\[
0.1234600000\ldots
\]
represent the same real number.
In general when does distinct two decimal representations represent the
same real number?
First I'll focus on the representation problem in the interval $[0,1)$.

Let
$x = 0.x_1x_2x_3...$
and
$y = 0.y_1y_2y_3...$.
Assume the sequence of decimal places are different:
\[
(x_i)_{i=1}^\infty
\neq
(y_i)_{i=1}^\infty
\]
I claim that $x = y$ iff there is some $N$ such that
$x_i = y_i$ for $i < N$,
$x_N = y_N - 1$, and
$x_j = 9, y_j = 0$ for $j > N$, or the conditions are switched between
$x$ and $y$.

$\impliedby$: This is easy.

$\implies$:
If
\[
(x_i)_{i=1}^\infty
\neq
(y_i)_{i=1}^\infty
\]
then there must be a smallest $N \geq 1$ such that
\[
x_i = y_i
\]
for $i < N$
and $x_N \neq y_N$.
Of course either $x_N < y_N$ or $x_N > y_N$.
Without loss of generalize, assume $x_N < y_N$.
I want to prove that $x_N, y_N$ differ by $1$.
Let's see what happens.
Since $x = y$ and $x_i = y_i$ for $i < N$, I have
\[
y - x = (y_N - x_N)10^{-N} + \sum_{j=N+1}^\infty (y_j - x_j)10^{-j}
\]
I claim that in this case $y - x > 0$.
Since $y - x = 0$,
\[
0 = (y_N - x_N)10^{-N} + \sum_{j=N+1}^\infty (y_j - x_j)10^{-j}
\]
i.e.,
\[
(x_N - y_N)10^{-N} = \sum_{j=N+1}^\infty (y_j - x_j)10^{-j}
\]
i.e.,
\[
x_N - y_N = \sum_{j=1}^\infty (y_{N+j} - x_{N+j})10^{-j}
\]
Now note that since $x_j, y_j$ are integers in $[0, 9]$,
\[
-9 \leq y_j - x_j \leq 9
\]
Hence
\[
-9 \sum_{j=1}^\infty 10^{-j} \leq
\sum_{j=1}^\infty (y_{N+j} - x_{N+j})10^{-j} \leq
9 \sum_{j=1}^\infty 10^{-j}
\]
i.e.,
\[
-9 \cdot \frac{1}{10^{1}} \cdot \frac{1}{1 - 1/10}
\leq
\sum_{j=1}^\infty (y_{N+j} - x_{N+j})10^{-j}
\leq
9 \cdot \frac{1}{10^{1}} \cdot \frac{1}{1 - 1/10} 
\]
i.e.,
\[
-1 
\leq
\sum_{j=N+1}^\infty (y_{N+j} - x_{N+j})10^{-j}
\leq
1
\]
Together with the above equation
\[
x_N - y_N = \sum_{j=1}^\infty (y_{N+j} - x_{N+j})10^{-j}
\]
I get
\[
-1 
\leq
x_N - y_N
\leq
1
\]
This means that
$x_N - y_N$
is $-1, 0, 1$.
But hang on: remember that $x_N \neq y_N$.
So I now know that
$x_N,  y_N$ differ by $1$.
Since I'm assuming $x_N < y_n$, I get
\[
x_N = y_N - 1
\]

So now from
\[
x_N - y_N = \sum_{j=1}^\infty (y_{N + j} - x_{N+j})10^{-j}
\]
I have
\[
-1 = x_N - y_N = \sum_{j=1}^\infty (y_{N+j} - x_{N+j})10^{-j}
\]
I claim that $y_{N+1} = 0$ and $x_{N+i} = 9$.
From
\[
-1 = \sum_{j=1}^\infty (y_{N+j} - x_{N+j})10^{-j}
\]
I get
\[
-1 - (y_{N+1} - x_{N+1})10^{-1} = \sum_{j=2}^\infty (y_{N+j} - x_{N+j})10^{-j}
\]
and therefore
\[
-10 - (y_{N+1} - x_{N+1}) = \sum_{j=2}^\infty (y_{N+j} - x_{N+j})10^{-j}
\]
i.e.,
\[
-10 - (y_{N+1} - x_{N+1}) = \sum_{j=1}^\infty (y_{N+1+j} - x_{N+1+j})10^{-j}
\]
I claim that $y_{N+1} = 0$ and $x_{N+1} = 9$.
The sum above can be bounded:
\[
-1 = 
-9 \cdot \frac{1}{10} \cdot \frac{1}{1 - 1/10}
\leq
\sum_{j=1}^\infty (y_{N+1+j} - x_{N+1+j})10^{-j}
\leq
9 \cdot \frac{1}{10} \cdot \frac{1}{1 - 1/10}
= 1
\]
Hence
\[
-1 \leq -10 - (y_{N+1} - x_{N+1}) \leq 1 
\]
i.e. 
\[
9 \leq - (y_{N+1} - x_{N+1}) \leq 11 
\]
i.e.,
\[
-9 \geq y_{N+1} - x_{N+1} \geq -11 
\]
However the since $x_{j}, y_{j}$ are in $[0, 9]$,
\[
-9 \leq y_{N+1} - x_{N+1} \leq 9
\]
Hence
\[
y_{N+1} - x_{N+1} = -9
\]
which is achieved only when
\[
y_{N+1} = 0, \,\,\, x_{N+1} = 9
\]
which is what I claimed earlier.
With this information,
\[
-10 - (y_{N+1} - x_{N+1}) = \sum_{j=1}^\infty (y_{N+1+j} - x_{N+1+j})10^{-j}
\]
becomes
\[
-1 = \sum_{j=1}^\infty (y_{N+1+j} - x_{N+1+j})10^{-j}
\]
and the same argument would yield
\[
y_{N+2} = 0, x_{N+2} = 9
\]
Etc.
By induction, one can prove that
\[
y_{N+j} = 0, x_{N+j} = 9
\]
for $j = 1, 2, 3, ...$.

I have now shown that if
$0.x_1x_2x_3...$
and
$0.y_1y_2y_3...$
are two decimal representations, then
either $x_i = y_i$ for all $i$ or, if not,
then there is a smallest $N$ such that
$x_i = y_i$ for $i < N$,
and $x_N \neq y_N$.
Furthermore,
$x_N$ and $y_N$ differs by exactly $1$.
Without loss of generality, if $x_N = y_N - 1$, then
$x_j = 9$ and $y_j = 0$ for $j > N$.

What about decimal representations with nonzero integer part?
Easy!
Consider
\[
x = x_{-m}x_{-m+1}x_{-m+2}\cdots x_0 \ . \ x_1x_2x_3...
\]
and
\[
y = y_{-n}y_{-n+1}y_{-n+2}\cdots y_0 \ . \ y_1y_2y_3...
\]
All I need to do is to multiply these two numbers
by $10^{-k}$ for a sufficiently large $k$
so that $x\cdot 10^{-k}$ and $y \cdot 10^{-k}$ are
two numbers in $[0,1)$.
  (Basically I'm moving their decimal point to the left
  by the same number of steps.)
  Then use the above result to get the following:
  Either $x_i = y_i$ for all $i$ of
  there is some $N$ such that $x_i = y_i$ for $i < N$,
  $x_N \neq y_N$.
  Furthermore $x_N, y_N$ differs by $1$.
  Assuming $x_N = y_N - 1$, then
  $x_j = 9$ and $y_j = 0$ for $j > N$.

  Let me state this as a theorem.
  I'm going to use descending index values.
  
  \begin{thm}
    Let $x$ and $y$ be real numbers.
    Suppose
    \[
    x = x_m x_{m - 1} \ldots x_0 \cdot x_{-1}x_{-2} \ldots
    \]
    and
    \[
    y = y_m y_{m - 1} \ldots y_0 \cdot y_{-1}y_{-2} \ldots
    \]
    be decimal representations for $x$ and $y$.
    Then $x = y$ iff
    either $x_i = y_i$ for all $i \leq m$ or
    there is some $N$ such that
    \begin{itemize}
    \item[\textnormal{(a)}] $x_i = y_i$ for $m \leq i < N$
    \item[\textnormal{(b)}] $x_N = y_N - 1$ (or $y_N = x_N - 1$),
      and
    \item[\textnormal{(c)}]
      $x_j = 9$, $y_j = 0$ for $j < N$
      (or $x_j = 0$, $y_j = 9$ for $j < N$, respectively)
    \end{itemize}
  \end{thm}

  The above is also true (and the proof is similar), if the base
  of the representation is changed to any base $B > 1$.
  In that case,
  the \lq\lq 9" in the statement of the theorem has to be changed to
  $B - 1$.
  

