\tinysidebar{\debug{exercises/{bernstein4/answer.tex}}}
\proof
I will show $|P(\N)| = |[0, 1)|$.
A real number $x$ in $[0, 1)$ can be writing in binary representation
  \[
  x = 0.x_1 x_2 x_3 \ldots
  \]
  I will assume that $x$ does not have a string of $1$s as the tail end
  of the binary sequence, i.e., there is no $N$ such that
  $x_i = 1$ for all $i > N$.
  Define
  a set
  \[
  f(x) \subseteq \N
  \]
  where
  \[
  n \in f(x) \iff x_n > 0
  \]
  For instance if $x = 0.001101$, then
  \[
  f(x) = \{3,4,6\}
  \]
  and if $x = 0.0101010101010\ldots$, then
  \[
  f(x) = \{2, 4, 6, 8,10, 12, ...\} 
  \]
  i.e., it's the set of positive even integer.
  Now I'll show that $f$ is 1--1.
  If $f(x) = f(x')$, then the
  $x_i = 1$ iff $x'_i = 1$.
  Hence $x$ and $x'$ has the same binary representation.
  Therefore $x = x'$.
  Hence $f$ is 1--1.
  To show $f$ is onto,
  if $X$ is a subset of $\N$, from $X$ I construct $x$ as
  $x = 0.x_1 x_2 x_3 \ldots$ where $x_i = 1$ iff $i \in X$.
  Hence $|P(\N)| = |[0,1)|$.

Note that Cantor's theorem
\[
|X| < |P(X)|
\]
implies that there there is no \lq\lq
largest" set since
\[
|X| < |P(X)| < |P(P(X))| < |P(P(P(X)))| < \cdots
\]
In particular we have $|\N| < |P(\N)|$.
From the above exercise, $|P(\N)| = |\R|$.
Hence
\[
|\N| < |\R|
\]
An interesting question is
whether you can find a set $X$ such that
\[
|\N| < |X| < |\R|
\]
In other words, is there anything in between $|\N|$ and $|\R|$?
In some books you will find the symbol
\defone{$\aleph_0$} (pronounced \lq\lq $\aleph$-null", google for the
pronounciation of $\aleph$)
which stands for
$|\N|$; we also write
\sidebarskip{12pt}\defone{$2^{\aleph_0}$}\sidebarskip{0pt}
for $|P(\N)|$.
The orders of
infinite sets together with natural numbers are called
\defone{ordinals}.
You can think of ordinals as sizes of sets.
Here are some ordinals:
\[
0 < 1 < 2 < 3 < \ldots < |\N| < |P(\N)| = |\R| < |P(\R)| < |P(P(\R))|
< |P(P(P(\R)))|
< \ldots
\]
So the
above question can be rephrased as whether there is any ordinal $x$
such that $\aleph_0 < x < 2^{\aleph_0}$.
Cantor asked this question 1878 and believed that there is no such $x$.
This is called the
\defone{continuum hypothesis}
\sidebarskip{12pt}\defone{CH}\sidebarskip{0pt}:

\textsc{Continuum Hypothesis}.
There is no ordinal $x$ between $|\N|$ and $|\R|$, i.e.,
there is no ordinal $x$ such that
\[
\aleph_0 < x < 2^{\alpha_0}
\]

Cantor spent his whole life trying to prove the CH but was not able to.
The continuum hypothesis is so important that it was
one of the famous
\href{https://en.wikipedia.org/wiki/Hilbert%27s_problems}{23 problems of David Hilbert}
announced at the
1900 centennial meeting of the international congress of mathematicians.
Without going into details, it was later proved that
CH can neither be proved right nor proved wrong.
Specifically, in 1940 G\"odel proved that the statement
\lq\lq there is \textit{some} $x$ such that $|\N| < x < |\R|$"
cannot be proven
and then in 1963 Paul Cohen proved that the statement
\lq\lq there is \textit{no} $x$ such that $|\N| < x < |\R|$"
cannot be proven, both
proved under some \lq\lq general and reasonably assumptions
on what is meant by sets and logic".
It's important to understand that
Godel and Cohen did not say that CH is true.
And they did not say it's false.
They are saying that CH 
and the logical opposite of CH, i.e., $\lnot\operatorname{CH}$ cannot be proved.
Set theory and logic of this type is still under active research.

