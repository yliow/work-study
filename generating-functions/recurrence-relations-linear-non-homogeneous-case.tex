%\section{Recurrence Relations: Linear Non-homogeneous Case}
\begin{python0}
from solutions import *; clear() 
\end{python0}

The general linear non-homogeneous case looks like this:
\[
a_n = c_1 a_{n-1} + \cdots c_d a_{n-d} + f(n)
\]
where $f(n)$ is a nonzero function in $n$.
Here's one example:
\[
a_n = 2a_{n-1} + n
\]
and here's another:
\[
a_n = a_{n-1} + 3a_{n-2} + 2n^3 + 1
\]
and yet another:
\[
a_n = a_{n-1} + 3a_{n-2} + \lfloor n \log n \rfloor
\]

Let's consider the a degree 2 nonhomogeneous recurrence relation:
\[
a_n = c_1 a_{n-1} + c_2 a_{n-2} + f(n), \,\,\,\,\, n \geq 2
\]
Of course using our method of generating functions,
in order to compute a closed form for $a_n$, we do this:
We let $a(x) = \sum_{n=0}^\infty a_n x^n$.
From 
\[
a_n = c_1 a_{n-1} + c_2 a_{n-2} + f(n), \,\,\,\,\, n \geq 2
\]
we get
\begin{align*}
a(x)
&= \sum_{n=0}^\infty a_n x^n \\
&= a_0 + a_1x + \sum_{n=0}^\infty a_n x^n \\
&= a_0 + a_1x + \sum_{n=2}^\infty (c_1 a_{n-1} + c_2 a_{n-2} + f(n))x^n \\
&= a_0 + a_1x + \sum_{n=2}^\infty (c_1 a_{n-1} + c_2 a_{n-2})x^n + 
\sum_{n=2}^\infty f(n)x^n \\
\end{align*}
Without going into details, we know that had the recurrence been this:
\[
a_n = c_1 a_{n-1} + c_2 a_{n-2}, \,\,\,\,\, n \geq 2
\]
we would have a rational function:
\[
a(x) = \frac{P(x)}{Q(x)}
\]
With the nonhomogeneous factor $f(n)$ in the recurrence relation we
get this:
\[
a(x) = \frac{P(x) + \sum_{n=2}^\infty f(n)x^n}{Q(x)}
\]
Therefore as long as we can express
\[
\sum_{n=2}^\infty f(n)x^n
\]
as a rational function then there is hope of getting the coefficient 
of $x^n$ of the power series of $a(x)$.

Let's try an example:
\[
a_n = 2a_{n-1} + n
\]
for $n \geq 1$.
Of course we let $a(x) = \sum_{n=0}^\infty a_n x^n$.
Then
\begin{align*}
a(x) 
&= \sum_{n=0} a_nx^n \\
&= a_0 + \sum_{n = 1}^\infty a_n x^n \\
&= a_0 + \sum_{n = 1}^\infty (2a_{n-1} + n) x^n \\
&= a_0 + 2\sum_{n = 1}^\infty a_{n-1}x^n + \sum_{n=1}^\infty n x^n \\
&= a_0 + 2x\sum_{n = 1}^\infty a_{n-1}x^{n-1} + \sum_{n=0}^\infty n x^n \\
\end{align*}
Notice that I've secretly actually added $0x^0$.
Continuing the computation we have
\begin{align*}
a(x) 
&= a_0 + 2x\sum_{n = 0}^\infty a_{n}x^{n} + \sum_{n=0}^\infty n x^n \\
&= a_0 + 2x a(x) + \sum_{n=0}^\infty n x^n \\
\therefore\,\,\,\,\, (1-2x) a(x) 
&= a_0 + \sum_{n=0}^\infty n x^n \\
\therefore\,\,\,\,\, a(x) 
&= \frac{a_0 + \sum_{n=0}^\infty n x^n}{1-2x} \\
\end{align*}
And by the way ... we can rewrite $\sum_{n=0}^\infty nx^n$ as a
rational function! It's just
\[
x\frac{d}{dx} \sum_{n=0}^\infty x^n
\]
remember?
Since
\begin{align*}
x\frac{d}{dx} \sum_{n=0}^\infty x^n
&= x \frac{d}{dx}\frac{1}{1-x} \\
x\frac{d}{dx} \sum_{n=0}^\infty x^n
&= x \frac{d}{dx} \frac{1}{1-x} \\
&= x \frac{1}{(1-x)^2} \\
\end{align*}
Hence
\begin{align*}
a(x) 
&= \frac{a_0 + \sum_{n=0}^\infty n x^n}{1-2x} \\
&= \frac{a_0 + \frac{x}{(1-x)^2}}{1-2x} \\
&= \frac{a_0(1-x)^2 + x}{(1-2x)(1-x)^2} \\
\end{align*}
Now we use the theory of partial fractions:
There are constants $A, B, C$ such that
\[
\frac{1}{(1-2x)(1-x)^2}
= 
\frac{A}{1-2x}
+
\frac{B}{1-x}
+
\frac{C}{(1-x)^2}
\]
We have
\[
1
= 
A(1-x)^2
+
B(1-x)(1-2x)
+
C(1-2x)
\]
When $x = 1$, we get $C = -1$.
When $x = 1/2$, we obtain $A = 4$.
Finally when $x = 0$, $1 = A + B + C$, and hence
$B = 1 - 4 + 1 = -2$.
Hence
\[
\frac{1}{(1-2x)(1-x)^2}
= 
\frac{4}{1-2x}
+
\frac{-2}{1-x}
+
\frac{-1}{(1-x)^2}
\]
Hence
\begin{align*}
a(x) 
&= 
(a_0(1-x)^2 + x) 
\biggl(
\frac{4}{1-2x}
+
\frac{-2}{1-x}
+
\frac{-1}{(1-x)^2}
\biggr) \\
&= 
(a_0 + (1-2a_0)x + a_0 x^2) 
\biggl(
4 \sum_{n=0}^\infty 2^n x^n
-2 \sum_{n=0}^\infty x^n
- \sum_{n=0}^\infty (n+1) x^n
\biggr) \\
&= 
(a_0 + (1-2a_0)x + a_0 x^2) 
\sum_{n=0}^\infty (4 \cdot 2^n - 2 - (n+1))x^n
\end{align*}
Hence 
\[
a_n = 
\begin{cases}
a_0 &\text{ if } n = 0 \\
%
a_0(4 \cdot 2 - 2 - (1+1)) + (1-2a_0)(4 - 2 - (0+1)) &\text{ if } n = 1 \\
%
a_0(4 \cdot 2^n - 2 - (n+1)) + (1-2a_0)(4 \cdot 2^{n-1} - 2 - n) +
a_0(4 \cdot 2^{n-2} - 2 - (n-1))
& \text{ if } n \geq 2 \\
\end{cases}
\]
Simplifying the above horrendous closed forms we get:
\[
a_n = 
\begin{cases}
a_0 &\text{ if } n = 0 \\
%
1 + 2a_0 &\text{ if } n = 1 \\
%
(2 + a_0)2^n - n  -2 & \text{ if } n \geq 2 \\
\end{cases}
\]

Let's check that the first few terms are correct.
First of all, $a_n = a_0$ for $n = 0$.
That's good.
For $n = 1$, from the closed form we have $a_1 = 1 + 2a_0$.
Now the recurrence we started with is
\[
a_n = 2a_{n-1} + n
\]
Therefore from the recurrence relation, $a_1 = 2a_0 + 1$.
That's also good.
One last check: Let's look at $n = 2$.
From the recurrence relation we have
\begin{align*}
a_2 &= 2a_1 + 2 \\
a_1 &= 2a_0 + 1
\end{align*}
Hence
\begin{align*}
a_2 = 2(2a_0 + 1) + 2 = 4a_0 + 4
\end{align*}
From the derived closed form we have
\[
(2 + a_0)2^2 - 2 - 2 = 8 + 4a_0 - 4 = 4a_0 + 4
\]



\newpage
\begin{ex}
Find a closed form for $a_n$ where
\[
a_n = 2a_{n-1} + a_{n-2} + n^2 + 1
\]
\end{ex}


\newpage
\begin{ex}
Solve completely the degree 2 linear nonhomogenous case
where the nonhomogenous factor is a polynomial of degree 1:
Find a closed form for $a_n$ where
\[
a_n = c_1 a_{n-1} + c_2 a_{n-2} + d_1x + d_2 x^2
\]
\end{ex}

\newpage
\subsection*{Solutions}

\newpage
\section*{Solutions}
Solution to Exercise \ref{ex:power-series-11}\labeltext{}{sol:power-series-11}.

\debug{\tinysidebar{exercises/{power-series-11/answer.tex}}}
 
(a) From
\begin{align*}
\sum_{n = 0}^\infty \frac{1}{2^n} x^n \cdot \sum_{n = 0}^\infty \frac{1}{2^n} x^n
&=
\left(
1 + \frac{1}{2}x + \frac{1}{4}x^2 + \frac{1}{8}x^3 + \cdots
\right)
\left(
1 + \frac{1}{2}x + \frac{1}{4}x^2 + \frac{1}{8}x^3 + \cdots
\right)
\end{align*}
the coefficient of $x^3$ is
\[
1 \cdot \frac{1}{8}
+ \frac{1}{2} \cdot \frac{1}{4}
+ \frac{1}{4} \cdot \frac{1}{2}
+ \frac{1}{8} \cdot 1
= 4 \cdot \frac{1}{8} = \frac{1}{2}
\]
The coefficient of $x^n$ is
\[
\sum_{k=0}^n \frac{1}{2^k} \cdot \frac{1}{2^{n-k}}
= \sum_{k=0}^n \frac{1}{2^k \cdot 2^{n-k}}
= \sum_{k=0}^n \frac{1}{2^n}
= \frac{n + 1}{2^n}
\]

(b)
First let's derive the coefficient of $x^n$ in general. 
The coefficient of $x^n$ is
\begin{align*}
\sum_{k=0}^n \frac{1}{2^k} \cdot \frac{1}{3^{n-k}}
&= \sum_{k=0}^n \frac{1}{2^k} \cdot \frac{3^k}{3^n} 
= \frac{1}{3^n} \sum_{k=0}^n \left(\frac{3}{2}\right)^k \\
&= \frac{1}{3^n} \cdot \frac{1 - (3/2)^{n+1}}{1 - 3/2} \\
&= \frac{1}{3^n} \cdot \frac{1 - (3/2)^{n+1}}{-1/2} \\
&= \frac{1}{3^n} \cdot \frac{(3/2)^{n+1} - 1}{1/2} \\
&= \frac{2}{3^n} \cdot \left( \frac{3^{n+1}}{2^{n+1}} - 1 \right) \\
&= 2 \cdot \left( \frac{3^{n+1} - 2^{n+1}}{2^{n+1}3^n} \right) \\
&= \frac{3^{n+1} - 2^{n+1}}{6^n}
\end{align*}
The coefficient of $x^3$ is
\[
\frac{3^4 - 2^{4}}{6^4} = \frac{65}{216}
\]



\newpage

Solution to Exercise \ref{ex:power-series-15}\labeltext{}{sol:power-series-15}.

\debug{\tinysidebar{exercises/{power-series-15/answer.tex}}}
We have
\begin{align*}
\left( \sum_{n=0}^\infty x^n \right)^{100} 
&= \left( \frac{1}{1 - x} \right)^{100} \\
&= \sum_{n=0}^\infty \binom{100 + n - 1}{n} x^n \\
&= \sum_{n=0}^\infty \binom{n + 99}{n} x^n \\
&= \sum_{n=0}^\infty \binom{n + 99}{99} x^n
\end{align*}
Hence the coefficient of $x^n$ is $\binom{n + 99}{99} x^n$ for $n \geq 0$.


\newpage

Solution to Exercise \ref{ex:power-series-16}\labeltext{}{sol:power-series-16}.

\debug{\tinysidebar{exercises/{power-series-16/answer.tex}}}

We have
\begin{align*}
&\left( 
2 + 5x + \frac{7}{1 - x}
\right)
\left( \sum_{n=0}^\infty x^n \right)^{100}
\\
&= \left( 
2 + 5x + \frac{7}{1 - x}
\right)
\left( \frac{1}{1 - x} \right)^{100}
\\
&=
2\left( \frac{1}{1 - x} \right)^{100}
+ 5x \left( \frac{1}{1 - x} \right)^{100}
+ \frac{7}{1 - x} \left( \frac{1}{1 - x} \right)^{100}
\\
&=
2 \sum_{n=0}^\infty \binom{100 + n - 1}{n} x^n
+ 5x \sum_{n=0}^\infty \binom{100 + n - 1}{n} x^n
+ 7 \left( \frac{1}{1 - x} \right)^{101} 
\\
&=
2 \sum_{n=0}^\infty \binom{n + 99}{n} x^n
+ 5x \sum_{n=0}^\infty \binom{n + 99}{n} x^n
+ 7 \sum_{n=0}^\infty \binom{101 + n - 1}{n}
\\
&=
\sum_{n=0}^\infty 2 \binom{n + 99}{99} x^n
+ \sum_{n=0}^\infty 5 \binom{n + 99}{99} x^{n+1}
+ \sum_{n=0}^\infty 7 \binom{n + 100}{n} x^n
\\
&=
\sum_{n=0}^\infty 2 \binom{n + 99}{99} x^n
+ \sum_{p=1}^\infty 5 \binom{p + 98}{99} x^{p}
+ \sum_{n=0}^\infty 7          \binom{n + 100}{n} x^n  \,\,\, \text{(let $p = n + 1$)}
\\
&=
\sum_{n=0}^\infty 2\binom{n + 99}{99} x^n
+ \sum_{n=1}^\infty 5 \binom{n + 98}{99} x^{n}  
+ \sum_{n=0}^\infty 7 \binom{n + 100}{100} x^n \,\,\,\text{(replace $p$ by $n$)}
\\
&=
2 \binom{99}{99} + \sum_{n=1}^\infty 2\binom{n + 99}{99} x^n
+ \sum_{n=1}^\infty 5\binom{n + 98}{99} x^{n} 
+ 7\binom{100}{100}  + \sum_{n=1}^\infty 7 \binom{n + 100}{100}  
\\
&=
9 +
\sum_{n=1}^\infty
\left( 2\binom{n + 99}{99} 
+  5\binom{n + 98}{99} 
+ 7 \binom{n + 100}{100}
\right) x^n
\end{align*}
Hence the coefficient of $x^n$ is
\[
\begin{cases}
9 & \text{ if } n = 0 \\
\displaystyle 2\binom{n + 99}{99} 
+  5\binom{n + 98}{99} 
+ 7 \binom{n + 100}{100} & \text{ if } n > 0
\end{cases}
\]


\newpage

Solution to Exercise \ref{ex:power-series-17}\labeltext{}{sol:power-series-17}.

\debug{\tinysidebar{exercises/{power-series-17/answer.tex}}}

(a)
\begin{align*}
\sum_{n=0}^\infty \frac{1}{2^n} x^n
\cdot
\sum_{n=0}^\infty \frac{1}{2^n} x^n
&=
\left( \sum_{n=0}^\infty \frac{1}{2^n} x^n \right)^2
\\
&=
\left( \sum_{n=0}^\infty \left( \frac{x}{2} \right)^n \right)^2
\\
&=
\left( \sum_{n=0}^\infty \left( \frac{x}{2} \right)^n \right)^2
\\
&=
\left( \frac{1}{1 - (x/2)} \right)^2
\\
&=\sum_{n=0}^\infty \binom{2 + n - 1}{n} \left( \frac{x}{2} \right)^n
\\
&=\sum_{n=0}^\infty \binom{n + 1}{n} \left( \frac{1}{2} \right)^n x^n
\\
\sum_{n=0}^\infty \left( \frac{n + 1}{2^n} \right x^n
\end{align*}

(b)
\begin{align*}
\sum_{n=0}^\infty \frac{1}{2^n} x^n
\cdot
\sum_{n=0}^\infty \frac{1}{3^n} x^n
\end{align*}
    

 % input solutions.tex
