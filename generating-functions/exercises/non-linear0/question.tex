\tinysidebar{\debug{exercises/{non-linear0/question.tex}}}
Note that the above recurrence looks like this:
\[
a_n = \frac{1}{n} a_{n-1} + n
\]
In particular, the multiple in front of $a_{n-1}$ is $\frac{1}{n}$
which is not constant (and therefore the recurrence is
not linear) and more importantly it's not a polynomial ...
that's the whole point of this section.
The appearance of $\frac{1}{n}$ can appear in cases where
there's an \lq\lq average'' computation.
This happens, for instance,
when $a_n$ is the average runtime of quicksort.
What do I mean by \lq\lq average'' here?
Let's look at quicksort (see section on quicksort algorithms).

Suppose the array to be sorted is \verb!x[0..(n-1)]!.
The pivot can be at \verb!x[0]! or it can be at \verb!x[1]! or ...
In any case the array is re--organized using the pivot.
Specifically,
the array is partitioned into two parts using the pivot
so that the array is re-organized into
\lq\lq left-hand-side array, pivot, right-hand-side array''.
There are several cases.
The left-hand-side array of values $\leq$ the pivot can have 0 elements.
This is followed by the pivot.
The right-hand-side array of values $>$ the pivot can have $n - 1$ values.
That's one case.
For this case, the time taken (recursively) is
\[
An + B + T(0) + T(n - 1)
\]
where $T(n)$ is the time taken to quicksort an array of size $n$.
The $An + B$ is time to scan the array
and place the values of the array into the structure
of 
\lq\lq left-hand side array, pivot, right-hand side array''.
Another case is when the left-hand-side array has 1 value.
This is followed by the pivot and then followed by the right-hand-side
array which has $n - 2$ values.
In this case, the time taken (recursively) is
\[
An + B + T(1) + T(n - 2)
\]
The time taken when the left-hand-side array has 2 values is
\[
An + B + T(2) + T(n - 3)
\]
Etc.
When we take the average over all $n$ cases, we get
\begin{align*}
T(n)
&= An + B \\
&\textwhite{= \ }
+ \frac{1}{n} 
\biggl(
  \left( T(0) + T(n-1) \right)
  + \left( T(1) + T(n-2) \right)
  + \cdots
  + \left( T(n - 1) + T(0) \right)
\biggr) \\
&= An + B + \frac{1}{n}\sum_{i=0}^{n-1} \left( T(i) + T(n - 1 - i) \right)
\end{align*}
Of course $T(0) = T(1) = C$ where $C$ is a constant.
Note that, if you're sharp, you realized that I'm assuming that the
values in the array \verb!x! are unique.
Furthermore, I'm assuming all the above cases are equally likely.

In the above, you can see that each $T(i)$ appears twice.
So I can simplify it a little like this:
\begin{align*}
T(n)
&= An + B + \frac{2}{n}\sum_{i=0}^{n-1} T(i)
\end{align*}
