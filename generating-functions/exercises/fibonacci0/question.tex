\tinysidebar{\debug{exercises/{fibonacci0/question.tex}}}
You may skip this exercise.
The above expression for $F_n$ is
\[
F_n = 
\frac{1}{\sqrt{5}} 
\left( 
\left( \frac{1 + \sqrt{5}}{2} \right)^n
-
\left( \frac{1 - \sqrt{5}}{2} \right)^n
\right)
\]
and it involves $\sqrt{5}$. 
Of course we know that $F_n$ is an integer.
So the $\sqrt{5}$ must some how disappear.
Let's expand the terms of the $n$--th powers
using binomial theorem and
verify that \textit{all} the $\sqrt{5}$ disappears.
(There's also the mysterious division
by a very high power of 2.)
\begin{align*}
F_n 
&= 
\frac{1}{\sqrt{5}} 
\left( 
\left( \frac{1 + \sqrt{5}}{2} \right)^n
-
\left( \frac{1 - \sqrt{5}}{2} \right)^n
\right) \\
&=
\frac{1}{2^n\sqrt{5}} 
\left( 
\bigl(
1 + \sqrt{5}
\bigr)^n
-
\bigl(
1 - \sqrt{5}
\bigr)^n
\right) \\
&=
\frac{1}{2^n\sqrt{5}} 
\left(
\sum_{i=0}^n \binom{n}{i}\sqrt{5}^i
- 
\sum_{i=0}^n (-1)^i\binom{n}{i}\sqrt{5}^i
\right) \\
&=
\frac{1}{2^n\sqrt{5}} 
\sum_{i=0}^n 
\left(
\binom{n}{i}
- 
(-1)^i\binom{n}{i}
\right)
\sqrt{5}^i \\
&=
\frac{1}{2^n\sqrt{5}} 
\sum_{i=0}^n 
(1
- 
(-1)^i)
\binom{n}{i}
\sqrt{5}^i
\end{align*}
Of course $1 - (-1)^i$ is 0 when $i$ is even.
Therefore
\begin{align*}
F_n 
&=
\frac{1}{2^n\sqrt{5}} 
\sum_{\substack{i=0 \\ i \text{ odd}}}^n 
2
\binom{n}{i}
\sqrt{5}^i \\
&=
\frac{2}{2^n} 
\sum_{\substack{i=0 \\ i \text{ odd}}}^n 
\binom{n}{i}
\sqrt{5}^{i-1} \\
&=
\frac{1}{2^{n-1}} 
\sum_{\substack{i=0 \\ i \text{ odd}}}^n 
\binom{n}{i}
5^{\frac{i-1}{2}}
\end{align*}
Note that since $i$ is odd, $i-1$ is even and 
hence $(i-1)/2$ is an integer.
Therefore we know now
that the expression of $F_n$ in fact does not involve the
mysterious $\sqrt{5}$.

In particular for the case of $n = 10$ we have
\begin{align*}
F_{10}
= \frac{1}{2^{9}} 
\left(
\binom{10}{1} 5^{0} +
\binom{10}{3} 5^{1} +
\binom{10}{5} 5^{2} +
\binom{10}{7} 5^{3} +
\binom{10}{9} 5^{4}
\right)
\end{align*}
which can be \lq\lq folded'':
\begin{align*}
F_{10}
= \frac{1}{2^{9}} 
\left(
2\binom{10}{1} 5^{0} +
2\binom{10}{3} 5^{1} +
\binom{10}{5} 5^{2}
\right)
\end{align*}
Likewise for odd $n$, say $n = 9$, 
\begin{align*}
F_{9}
= \frac{1}{2^{8}} 
\left(
2\binom{9}{1} 5^{0} +
2\binom{9}{3} 5^{1}
\right)
\end{align*}

Note however that the closed form will evaluate faster
than the form that uses a summation.
Furthermore it's also easier to see the big-O of $F_n$
using the closed form.
