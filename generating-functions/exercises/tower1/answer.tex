\tinysidebar{\debug{exercises/{tower1/answer.tex}}}
Let our original procedure be $P(n, A, B, C)$. I have already proved
that the number of moves made by $P(n, A, B, C)$ is $T(n) = 2^n - 1$.
Now suppose there's another procedure $P'(n, A, B, C)$ that uses
$T'(n)$ moves. I will prove by induction that $T'(n) = T(n)$.
(Yes, it's that \lq\lq induction" thing again.)

First of all if you have one disk (i.e., $n = 1$), then of course
$P'(1, A, B, C)$, being optimal, will execute $A \rightarrow C$.
That means $T'(1) = 1 = T(1)$.

Now suppose $T'(k) = T(k)$ for $k = 1, 2, 3, ..., n - 1$
and we consider the moves made by $P'(n, A, B, C)$.
Disk $n$ (the largest) has to move from $A$ to either $B$ or $C$.
Note that this is the first move made by disk $n$.
(Of course in the end it will land in $C$, but I'm not even assuming that yet.
I'm just saying this disk has to move.
If this disk does not move, there's no way it's going to land in C!)
So $P'$ at this point will either execute $A \rightarrow B$
or $A \rightarrow C$.
Remember that $P'$ is optimal.

\textsc{Case: The move is $A\rightarrow C$.}
This costs 1 step.
After this, I use the optimal strategy to move the first $n - 1$ disks
from $B$ to $C$ and I'm done.
But by induction, the optimal strategy takes
$T'(n-1) = 2^{n - 1} + 1$ moves.
Therefore altogether the number of numbers is
$T'(n - 1) + 1 + T'(n-1) = 2^n - 1 = T(n)$.

\textsc{Case: The move is $A\rightarrow B$.}
At this point, $T'(n-1) + 1$ moves has been made.
Consider what will happen next.
At some point (in the future), after $\alpha \geq 0$ moves, 
disk $n$ has to land in $C$ -- that cost at least one step.
This move for disk $n$ is either $A \rightarrow C$ or $B \rightarrow C$.
If it's $A \rightarrow C$, then the first $n - 1$ disks must be at $B$.
If it's $B \rightarrow C$, then the first $n - 1$ disks must be at $A$.
This means that the first $n - 1$ disks has to be moved (in the future)
to $B$ or $A$.
But the number of moves has to be $\geq T'(n-1)$.
So for this case, the total number of moves is
$\geq T'(n-1) + 1 + \alpha + T'(n-1) + 1 = 2T'(n-1) + 2 + \alpha$ where
$\alpha \geq 0$.
By inductive hypothesis $T'(n-1) = 2^{n-1} - 1$ which means that 
the number of moves is at least
\[
2T'(n - 1) + 2 + \alpha = 2(2^{n-1} - 1) + 2 + \alpha = 2^n + \alpha
\]
But this is greater than the first case.
And since we are using the optimal strategy $P'$, only the first case occurs
-- the second case does
not happen.

We conclude that $T'(n) = 2^n - 1 = T(n)$.

By inductive hypothesis, we have shown that any optimal strategy will
make $2^n - 1$ moves.
In particular, our earlier strategy is the optimal strategy.
