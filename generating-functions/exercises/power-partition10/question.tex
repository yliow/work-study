\tinysidebar{\debug{exercises/{power-partition10/question.tex}}}
Here's another method for creating new partitions.
First consider this Ferrers diagram:
\begin{verbatim}
    * * * *
    * * 
    *
\end{verbatim}
This gives a partition of $4 + 2 + 1$, i.e. a partition of $7$.
First ignore the leftmost column:
\begin{verbatim}
    *     * * *
    *     * 
    *
\end{verbatim}
You can the partition 
\begin{verbatim}
          * * *
          * 
\end{verbatim}
i.e. $3 + 1$.
Perform a reflection on it to get
\begin{verbatim}
          * *
          *
          * 
\end{verbatim}
and join it with the original leftmost column 
\begin{verbatim}
    *     * *
    *  <- *   
    *     *
\end{verbatim}
to get this:
\begin{verbatim}
    * * *
    * *   
    * *
\end{verbatim}
i.e. you get $3 + 2 + 2$.
Study this new technique.
