%\section{Space subdivision problems}
\begin{python0}
from solutions import *; clear() 
\end{python0}

What is the maximum number of regions do you get when you
cut up a 2-dimensional space by $n$ lines.
With one line:
\begin{python}
from latextool_basic import *
plot = Plot()
plot += Line(points=[(-1, -1), (1, 1)])
print(plot)
\end{python}
the 2-D space is cut up into two pieces.
Let $a_n$ be the number of regions cut out by $n$ lines.
With two lines we get 4.
\begin{python}
from latextool_basic import *
plot = Plot()
plot += Line(points=[(-1, -1), (1, 1)])
plot += Line(points=[(-1, 0), (1, 0)])
print(plot)
\end{python}
So far we have
\[
a_0 = 1, \,\,\,\,\,
a_1 = 2, \,\,\,\,\,
a_2 = 4
\]
To simplify the problem, a line cannot go through a point of
intersection.
With three:
\begin{python}
from latextool_basic import *
plot = Plot()
plot += Line(points=[(-1, -1), (1, 1)])
plot += Line(points=[(-1, 0), (1, 0)])
plot += Line(points=[(-1, 1), (2, -1)])
print(plot)
\end{python}
you get 7.
Note that the last line above, $L_3$, cuts through 3 regions,
and therefore adds 3 more
regions to the regions already obtained with the first two lines.
The next line will cut through 4 regions.
Come to think of it, $L_1$ (the first line) cuts through 1 region --
the uncut region,
$L_2$ (the second line) cuts through
2 regions.
In general line $L_n$ cuts through $n$ regions.
Now before $L_n$, therefore $a_{n-1}$ regions.
With $L_n$, the number of regions is $a_{n-1}$ and
$n$ regions are cut up by $L_n$. 
Therefore
\[
a_n = a_{n-1} + n
\]
for $n \geq 1$.


\newpage
\begin{ex}
What is $a_0$?
\qed
\end{ex}

\newpage
Let $a(x) = \sum_{n=0}^\infty a_n x^n$.
\begin{align*}
a(x) 
&= a_0 + \sum_{n=1}^\infty (a_{n-1} + n) x^n \\
&= a_0 + \sum_{n=1}^\infty a_{n-1} x^n + \sum_{n=0}^\infty n x^n \\
&= a_0 + x a(x) + \sum_{n=0}^\infty n x^n \\
\therefore\,\,\,\,\,\,
a(x) 
&= 
\frac{1 + x\frac{d}{dx} \sum_{n=0}^\infty x^n}{1 - x} \\
&=\frac{1 + x\frac{d}{dx} \frac{1}{1-x} }{1 - x} \\
&=\frac{1 + x \frac{1}{(1-x)^2} }{1 - x} \\
&=\frac{(1-x)^2 + x}{(1 - x)^2} \\
&=\frac{1 - x + x^2}{(1 - x)^3} \\
&=(1 - x + x^2) \sum_{n=0}^\infty \binom{3 + n - 1}{n} x^n \\
&=(1 - x + x^2) \sum_{n=0}^\infty \binom{n + 2}{2} x^n \\
&=
\sum_{n=0}^\infty \binom{n + 2}{2} x^n
- \sum_{n=0}^\infty \binom{n + 2}{2} x^{n+1}
+ \sum_{n=0}^\infty \binom{n + 2}{2} x^{n+2} \\
&=
\sum_{n=0}^\infty \binom{n + 2}{2} x^n
- \sum_{n=1}^\infty \binom{n + 1}{2} x^{n}
+ \sum_{n=2}^\infty \binom{n}{2} x^{n}
\end{align*}
Hence
\[
a_0 = 1, \,\,\,\,\,
a_1 = \binom{0 + 2}{2} + \binom{1 + 1}{2} = 2
\]
and for $n \geq 2$
\begin{align*}
a_n 
&= 
\binom{n + 2}{2} +
- \binom{n + 1}{2}
+ \binom{n}{2} \\
&= \frac{1}{2} 
((n+2)(n+1)
- (n+1)n
+ n(n-1)) \\
&= \frac{1}{2} 
(n^2 
+ n
+ 2)
\end{align*}
and note that this closed form 
can also be used for $a_0, a_1$.
Hence for $n \geq 0$
\begin{align*}
a_n 
= \frac{1}{2}
(n^2 
+ n
+ 2)
\end{align*}
or if you like
\begin{align*}
a_n 
= \frac{n(n+1)}{2} + 1
\end{align*}


\newpage
\begin{ex}
Instead of straight lines, what if we use lines bent at exactly one
point:
\begin{python}
from latextool_basic import *
plot = Plot()
plot += Line(points=[(-1,1),(0,-1),(1,1)])
print(plot)
\end{python}
For instance with two such lines
\begin{python}
from latextool_basic import *
plot = Plot()
plot += Line(points=[(-1,1),(0,-1),(1,1)])
plot += Line(points=[(-1,-1),(0,1),(1,-1)])
print(plot)
\end{python}
we can have 5 regions.
However this is not the maximum possible.
Show that it should be 7.
(HINT: See it? No?
Try to rotate one of the bent lines.)
Find a closed form for this problem.
\qed
\end{ex}

\newpage
\begin{ex}
  What if instead of lines or lines with one bend we have
  lines with 2 bends, i.e. something that looks like z and its reflection?
 What about lines with $k$ bends where $k$ is fixed?
  \qed
\end{ex}
  
\newpage
\begin{ex}
What if for the original problem we only count finite regions?
(i.e. we only count regions which are bounded on all sides by lines.)
\qed
\end{ex}

\newpage
\begin{ex}
What if instead of lines (or bent lines) we use
triangles?
\qed
\end{ex}


\newpage
\begin{ex}
What if instead of lines we use circles?
\qed
\end{ex}

\newpage
\subsection*{Solutions}

\newpage
\section*{Solutions}
Solution to Exercise \ref{ex:power-series-11}\labeltext{}{sol:power-series-11}.

\debug{\tinysidebar{exercises/{power-series-11/answer.tex}}}
 
(a) From
\begin{align*}
\sum_{n = 0}^\infty \frac{1}{2^n} x^n \cdot \sum_{n = 0}^\infty \frac{1}{2^n} x^n
&=
\left(
1 + \frac{1}{2}x + \frac{1}{4}x^2 + \frac{1}{8}x^3 + \cdots
\right)
\left(
1 + \frac{1}{2}x + \frac{1}{4}x^2 + \frac{1}{8}x^3 + \cdots
\right)
\end{align*}
the coefficient of $x^3$ is
\[
1 \cdot \frac{1}{8}
+ \frac{1}{2} \cdot \frac{1}{4}
+ \frac{1}{4} \cdot \frac{1}{2}
+ \frac{1}{8} \cdot 1
= 4 \cdot \frac{1}{8} = \frac{1}{2}
\]
The coefficient of $x^n$ is
\[
\sum_{k=0}^n \frac{1}{2^k} \cdot \frac{1}{2^{n-k}}
= \sum_{k=0}^n \frac{1}{2^k \cdot 2^{n-k}}
= \sum_{k=0}^n \frac{1}{2^n}
= \frac{n + 1}{2^n}
\]

(b)
First let's derive the coefficient of $x^n$ in general. 
The coefficient of $x^n$ is
\begin{align*}
\sum_{k=0}^n \frac{1}{2^k} \cdot \frac{1}{3^{n-k}}
&= \sum_{k=0}^n \frac{1}{2^k} \cdot \frac{3^k}{3^n} 
= \frac{1}{3^n} \sum_{k=0}^n \left(\frac{3}{2}\right)^k \\
&= \frac{1}{3^n} \cdot \frac{1 - (3/2)^{n+1}}{1 - 3/2} \\
&= \frac{1}{3^n} \cdot \frac{1 - (3/2)^{n+1}}{-1/2} \\
&= \frac{1}{3^n} \cdot \frac{(3/2)^{n+1} - 1}{1/2} \\
&= \frac{2}{3^n} \cdot \left( \frac{3^{n+1}}{2^{n+1}} - 1 \right) \\
&= 2 \cdot \left( \frac{3^{n+1} - 2^{n+1}}{2^{n+1}3^n} \right) \\
&= \frac{3^{n+1} - 2^{n+1}}{6^n}
\end{align*}
The coefficient of $x^3$ is
\[
\frac{3^4 - 2^{4}}{6^4} = \frac{65}{216}
\]



\newpage

Solution to Exercise \ref{ex:power-series-15}\labeltext{}{sol:power-series-15}.

\debug{\tinysidebar{exercises/{power-series-15/answer.tex}}}
We have
\begin{align*}
\left( \sum_{n=0}^\infty x^n \right)^{100} 
&= \left( \frac{1}{1 - x} \right)^{100} \\
&= \sum_{n=0}^\infty \binom{100 + n - 1}{n} x^n \\
&= \sum_{n=0}^\infty \binom{n + 99}{n} x^n \\
&= \sum_{n=0}^\infty \binom{n + 99}{99} x^n
\end{align*}
Hence the coefficient of $x^n$ is $\binom{n + 99}{99} x^n$ for $n \geq 0$.


\newpage

Solution to Exercise \ref{ex:power-series-16}\labeltext{}{sol:power-series-16}.

\debug{\tinysidebar{exercises/{power-series-16/answer.tex}}}

We have
\begin{align*}
&\left( 
2 + 5x + \frac{7}{1 - x}
\right)
\left( \sum_{n=0}^\infty x^n \right)^{100}
\\
&= \left( 
2 + 5x + \frac{7}{1 - x}
\right)
\left( \frac{1}{1 - x} \right)^{100}
\\
&=
2\left( \frac{1}{1 - x} \right)^{100}
+ 5x \left( \frac{1}{1 - x} \right)^{100}
+ \frac{7}{1 - x} \left( \frac{1}{1 - x} \right)^{100}
\\
&=
2 \sum_{n=0}^\infty \binom{100 + n - 1}{n} x^n
+ 5x \sum_{n=0}^\infty \binom{100 + n - 1}{n} x^n
+ 7 \left( \frac{1}{1 - x} \right)^{101} 
\\
&=
2 \sum_{n=0}^\infty \binom{n + 99}{n} x^n
+ 5x \sum_{n=0}^\infty \binom{n + 99}{n} x^n
+ 7 \sum_{n=0}^\infty \binom{101 + n - 1}{n}
\\
&=
\sum_{n=0}^\infty 2 \binom{n + 99}{99} x^n
+ \sum_{n=0}^\infty 5 \binom{n + 99}{99} x^{n+1}
+ \sum_{n=0}^\infty 7 \binom{n + 100}{n} x^n
\\
&=
\sum_{n=0}^\infty 2 \binom{n + 99}{99} x^n
+ \sum_{p=1}^\infty 5 \binom{p + 98}{99} x^{p}
+ \sum_{n=0}^\infty 7          \binom{n + 100}{n} x^n  \,\,\, \text{(let $p = n + 1$)}
\\
&=
\sum_{n=0}^\infty 2\binom{n + 99}{99} x^n
+ \sum_{n=1}^\infty 5 \binom{n + 98}{99} x^{n}  
+ \sum_{n=0}^\infty 7 \binom{n + 100}{100} x^n \,\,\,\text{(replace $p$ by $n$)}
\\
&=
2 \binom{99}{99} + \sum_{n=1}^\infty 2\binom{n + 99}{99} x^n
+ \sum_{n=1}^\infty 5\binom{n + 98}{99} x^{n} 
+ 7\binom{100}{100}  + \sum_{n=1}^\infty 7 \binom{n + 100}{100}  
\\
&=
9 +
\sum_{n=1}^\infty
\left( 2\binom{n + 99}{99} 
+  5\binom{n + 98}{99} 
+ 7 \binom{n + 100}{100}
\right) x^n
\end{align*}
Hence the coefficient of $x^n$ is
\[
\begin{cases}
9 & \text{ if } n = 0 \\
\displaystyle 2\binom{n + 99}{99} 
+  5\binom{n + 98}{99} 
+ 7 \binom{n + 100}{100} & \text{ if } n > 0
\end{cases}
\]


\newpage

Solution to Exercise \ref{ex:power-series-17}\labeltext{}{sol:power-series-17}.

\debug{\tinysidebar{exercises/{power-series-17/answer.tex}}}

(a)
\begin{align*}
\sum_{n=0}^\infty \frac{1}{2^n} x^n
\cdot
\sum_{n=0}^\infty \frac{1}{2^n} x^n
&=
\left( \sum_{n=0}^\infty \frac{1}{2^n} x^n \right)^2
\\
&=
\left( \sum_{n=0}^\infty \left( \frac{x}{2} \right)^n \right)^2
\\
&=
\left( \sum_{n=0}^\infty \left( \frac{x}{2} \right)^n \right)^2
\\
&=
\left( \frac{1}{1 - (x/2)} \right)^2
\\
&=\sum_{n=0}^\infty \binom{2 + n - 1}{n} \left( \frac{x}{2} \right)^n
\\
&=\sum_{n=0}^\infty \binom{n + 1}{n} \left( \frac{1}{2} \right)^n x^n
\\
\sum_{n=0}^\infty \left( \frac{n + 1}{2^n} \right x^n
\end{align*}

(b)
\begin{align*}
\sum_{n=0}^\infty \frac{1}{2^n} x^n
\cdot
\sum_{n=0}^\infty \frac{1}{3^n} x^n
\end{align*}
    


