\sectionthree{Partial fractions}
\begin{python0}
from solutions import *; clear() 
\end{python0}

In the previous section, I rewrote the product of two power series into one
power series.
Here's a summary from the previous section:

{\footnotesize
\begin{align*}
&\tmpAA \\ 
&\xlongequal{(a)} \tmpBB & & \text{convert power series to rational expression} \\
&\xlongequal{(b)} \tmpCC & & \text{multiply rational expressions (instead of power series)}\\
&\xlongequal{(c)} \tmpDD & & \text{write rational expression as linear sum of rational expressions}\\
&\xlongequal{(d)} \tmpEE & & \text{convert rational expressions to power series} \\
&\xlongequal{(d)} \tmpFF & & \text{tidy up to one power series}
\end{align*}
}

The basic machinery is the theory of partial fractions.
In the previous section, I showed you some techniques for
converting some power series to rational functions.

Once we're done with that we want to express the rational function
into a sum of simpler rational functions.
In an earlier section I mentioned that you can always find constants
$A$ and $B$ so that
\[
\frac{1}{(4-x)(3-2x)}
\]
is broken up into
\[
\frac{1}{(4-x)(3-2x)}
=
\frac{A}{4-x} +
\frac{B}{3-2x}
\]

What other rational functions can be broken up this way?

Suppose you're given a rational function:
\[
\frac{P(x)}{Q(x)}
\]
where $P(x)$ and $Q(x)$ are polynomials.
The first thing you need to do is to check the degree of $P(x)$
and $Q(x)$.
If the degree of $P(x)$ is at least the degree of $Q(x)$,
you perform long division to get
\[
\frac{P(x)}{Q(x)} = p(x) + \frac{R(x)}{Q(x)}
\]
The point: the theory of partial fractions can only apply when
the numerator has degree less than the degree of the denominator.
Long division of polynomials is taught in earlier classes
so I will not talk about it.
Once you're done with that you focus on
\[
\frac{R(x)}{Q(x)}
\]
The quotient
\[
p(x)
\]
is a polynomial and is easy to handle when it comes to collecting the 
coefficient of $x^n$.

So let's assume that the rational function
\[
\frac{P(x)}{Q(x)}
\]
has degree of $P(x)$ less than the degree of $Q(x)$.

The first thing you do is to factorize $Q(x)$.
You know that over the real numbers polynomials factorize into powers
of linear polynomials:
\[
(ax + b)^k
\]
and powers of quadratic ones:
\[
(ax^2 + bx + c)^k
\]
($a, b, c$ are real numbers).
Once this is done the product can be broken up into
a sum of simpler rational functions based on the
factors.

Instead of giving you the general theory,
I'll give you some examples.

Say for instance you have the following
\[
\frac{x^3 + 10x^2 + 3x + 36}{x^5 - x^4 + 8x^3 - 8x^2 + 16x - 16}
\]
Note that the degree of the numerator is 3 and the degree of the
denominator is 5.
Therefore we do not need to perform long division.

The denominator can be factorized as:
\begin{align*}
x^5 - x^4 + 8x^3 - 8x^2 + 16x - 16
&= x^4(x - 1) + 8x^2(x - 1) + 16(x - 1) \\
&= (x^4 + 8x^2 + 16)(x - 1) \\
&= (x^2 + 4)^2(x - 1) 
\end{align*}

As you can see you get a product of powers of linear and quadratic 
polynomials.

The general form of the sum depends only on the denominator.
For our example the theory of partial fractions
guarantees that
\begin{align*}
\frac{x^3 + 10x^2  3x + 36}{(x - 1)(x^2 + 4)^2}
&=
\frac{A}{x - 1} +
\frac{Bx + C}{x^2 + 4} + \frac{Dx + E}{(x^2 + 4)^2}
\end{align*}
for some constants $A, B, C, D, E$.
Again the form on the left does not depend on the
numerator $x^3 + 10x^2  3x + 36$ so let's 
just forget about that for the time being:
\begin{align*}
\frac{BLAH}{(x - 1)(x^2 + 4)^2}
&=
\frac{A}{x - 1} +
\frac{Bx + C}{x^2 + 4} + \frac{Dx + E}{(x^2 + 4)^2}
\end{align*}
This is what you do:

First you make sure that same factors in the denominator
are grouped up together into a single power.
For instance note that we have two factors $(x^2+4)$
collected together to a power of 2:
\[
\frac{BLAH}{(x - 1)(x^2 + 4)^2}
\]
The other factors occurs with power 1.
Do \textit{not} write
\[
\frac{BLAH}{(x - 1)(x^2 + 4)(x^2 + 4)}
\]
You will confuse yourself.

Next for each power factor in the denominator
of your rational function, you write rational
functions with denominators
starting from power 1 up to the power of that factor:
\begin{align*}
\frac{BLAH}{(x - 1)(x^2 + 4)^2}
&=
\frac{?}{x - 1} + \\
&\hskip 0.5cm \frac{?}{x^2 + 4} + \frac{?}{(x^2 + 4)^2} 
\end{align*}

Note for instance that the factor $x - 1$
has power 1.
You only have one rational function for that factor so you write down
\[
\frac{?}{x - 1}
\]
on the right.
For the factor $(x^2 + 4)^2$ the power is 2.
Therefore you write down two rational functions on the left,
one for power 1 and another for power 2;
you stop at 2.
Therefore for this factor you write down
\[
\frac{?}{x^2 + 4} + \frac{?}{(x^2 + 4)^2}
\]
on the right.

So at this point you have this:
\begin{align*}
\frac{BLAH}{(x - 1)(x^2 + 4)^2}
&=
\frac{?}{x - 1} + \\
&\hskip 0.5cm \frac{?}{x^2 + 4} + \frac{?}{(x^2 + 4)^2} 
\end{align*}
Now for the numerators on the right.
You always write a polynomial that is one degree
less than the corresonding denominator factor.

For instance for the factor $x - 1$,
the degree is 1.
So you write a polynomial of degree 0, i.e. a constant:
\[
\frac{A}{x - 1}
\]

For the factor $x^2 + 4$, the degree is two.
So you write down polynomials of degree one:
\[
\frac{Bx + C}{x^2 + 4} + \frac{Dx + E}{(x^2 + 4)^2} 
\]
(in this case you need two numerator polynomials since
there are two fractions to complete.)

Therefore altogether you have this:
\begin{align*}
\frac{BLAH}{(x - 1)(x^2 + 4)^2}
&=
\frac{A}{x - 1} + \\
&\hskip 0.5cm \frac{Bx + C}{x^2 + 4} + \frac{Dx + E}{(x^2 + 4)^2} 
\end{align*}
The important thing to note is that the form of the 
sum of rational functions on the right
depends only on the denominator of the rational function
that you started with.

Now we put in the numerator of our rational function to get
\begin{align*}
\frac{x^3 + 10x^2  3x + 36}{(x - 1)(x^2 + 4)^2}
&=
\frac{A}{x - 1} +
\frac{Bx + C}{x^2 + 4} + \frac{Dx + E}{(x^2 + 4)^2}
\end{align*}
Now all you need to do is to solve for $A, B, C, D, E$.

Again one of the simplest methods is to clear denominators,
select values of $x$ to get equations for $A, B, C, D, E$ 
(you need 5 equations in this case), and solve
the linear equations on the unknown quantities $A, B, C, D, E$.
That's it.
This is pretty much just basic polynomial and solving
linear equations from college algebra.

I will show you only the just the first few steps.

First on multiplying the equation
\begin{align*}
\frac{x^3 + 10x^2  3x + 36}{(x - 1)(x^2 + 4)^2}
&=
\frac{A}{x - 1} +
\frac{Bx + C}{x^2 + 4} + \frac{Dx + E}{(x^2 + 4)^2}
\end{align*}
by $(x - 1)(x^2 + 4)^2$ we get
\begin{align*}
x^3 + 10x^2 + 3x + 36
&=
A(x^2 + 4)^2 +
(Bx + C)(x^2 + 4)(x - 1) +
(Dx + E)(x-1)
\end{align*}

If I let $x = 1$, the above identity becomes the equation
\begin{align*}
1 + 10 + 3 + 36 &= A(1 + 4)^2 + 0 + 0 \\
\THEREFORE 50 &= 25A \\
\THEREFORE A &= 2
\end{align*}
(Note that as in a previous example, I try to choose
simple values for $x$. The best are those that knock out lots
of terms.)
We manage to solve $A$ completely.
So now we need 4 equations for the rest.

Next I choose $x = 0$ to get
\[
36 = A(4)^2 + (0 + C)(4)(-1) + E(-1)
\]
and simplifying and substituting $A = 2$, I get
\[
36 =  32 - 4C - E
\]
i.e.
\[
4C + E = -4
\]
Now I choose $x = -1$ to get
\[
-1 + 10 - 3 + 36 = A(5)^2 + (-B+C)(-2)(5) + (-D+E)(-2)
\]
i.e,
\[
42 = 50 -10(C - B) -2(E - D)
\]
which gives me
\[
10(C - B) + 2(E - D) = 8
\]
i.e.
\[
5(C - B) + (E - D) = 4
\]
On choosing two more values for $x$ you will get
altogether 4 equations in $B, C, D, E$
and solving them you get
\[
B = -2, \,\,\,
C = -1, \,\,\,
D = 1, \,\,\,
E = 0
\]

This implies that
\[
\frac{x^3 + 10x^2  3x + 36}{x^5 - x^4 + 8x^3 - 8x^2 + 16x - 16}
=
\frac{2}{x - 1} +
\frac{-2x -1}{x^2 + 4} + \frac{x}{(x^2 + 4)^2} 
\]

And now you can apply all you know from the previous sections
to rewrite the three sums into power series and then
collect them together into a single power series so that
the coefficient can be read off easily.

Of course it's easy to check if your work is correct if you have
a calculator or by writing a simple program.
You simply evaluate the two functions (the left and right hand side of the
above for instance) for several values of $x$.



%-*-latex-*-

\begin{ex} 
  \label{ex:prob-00}
  \tinysidebar{\debug{exercises/{disc-prob-28/question.tex}}}

  \solutionlink{sol:prob-00}
  \qed
\end{ex} 
\begin{python0}
from solutions import *
add(label="ex:prob-00",
    srcfilename='exercises/discrete-probability/prob-00/answer.tex') 
\end{python0}


%-*-latex-*-

\begin{ex} 
  \label{ex:prob-00}
  \tinysidebar{\debug{exercises/{disc-prob-28/question.tex}}}

  \solutionlink{sol:prob-00}
  \qed
\end{ex} 
\begin{python0}
from solutions import *
add(label="ex:prob-00",
    srcfilename='exercises/discrete-probability/prob-00/answer.tex') 
\end{python0}


%-*-latex-*-

\begin{ex} 
  \label{ex:prob-00}
  \tinysidebar{\debug{exercises/{disc-prob-28/question.tex}}}

  \solutionlink{sol:prob-00}
  \qed
\end{ex} 
\begin{python0}
from solutions import *
add(label="ex:prob-00",
    srcfilename='exercises/discrete-probability/prob-00/answer.tex') 
\end{python0}


%-*-latex-*-

\begin{ex} 
  \label{ex:prob-00}
  \tinysidebar{\debug{exercises/{disc-prob-28/question.tex}}}

  \solutionlink{sol:prob-00}
  \qed
\end{ex} 
\begin{python0}
from solutions import *
add(label="ex:prob-00",
    srcfilename='exercises/discrete-probability/prob-00/answer.tex') 
\end{python0}


%-*-latex-*-

\begin{ex} 
  \label{ex:prob-00}
  \tinysidebar{\debug{exercises/{disc-prob-28/question.tex}}}

  \solutionlink{sol:prob-00}
  \qed
\end{ex} 
\begin{python0}
from solutions import *
add(label="ex:prob-00",
    srcfilename='exercises/discrete-probability/prob-00/answer.tex') 
\end{python0}


%-*-latex-*-

\begin{ex} 
  \label{ex:prob-00}
  \tinysidebar{\debug{exercises/{disc-prob-28/question.tex}}}

  \solutionlink{sol:prob-00}
  \qed
\end{ex} 
\begin{python0}
from solutions import *
add(label="ex:prob-00",
    srcfilename='exercises/discrete-probability/prob-00/answer.tex') 
\end{python0}


%-*-latex-*-

\begin{ex} 
  \label{ex:prob-00}
  \tinysidebar{\debug{exercises/{disc-prob-28/question.tex}}}

  \solutionlink{sol:prob-00}
  \qed
\end{ex} 
\begin{python0}
from solutions import *
add(label="ex:prob-00",
    srcfilename='exercises/discrete-probability/prob-00/answer.tex') 
\end{python0}


%-*-latex-*-

\begin{ex} 
  \label{ex:prob-00}
  \tinysidebar{\debug{exercises/{disc-prob-28/question.tex}}}

  \solutionlink{sol:prob-00}
  \qed
\end{ex} 
\begin{python0}
from solutions import *
add(label="ex:prob-00",
    srcfilename='exercises/discrete-probability/prob-00/answer.tex') 
\end{python0}


%-*-latex-*-

\begin{ex} 
  \label{ex:prob-00}
  \tinysidebar{\debug{exercises/{disc-prob-28/question.tex}}}

  \solutionlink{sol:prob-00}
  \qed
\end{ex} 
\begin{python0}
from solutions import *
add(label="ex:prob-00",
    srcfilename='exercises/discrete-probability/prob-00/answer.tex') 
\end{python0}

