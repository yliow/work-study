\sectionthree{Partition of integers}
\begin{python0}
from solutions import *; clear() 
\end{python0}

A theory of partition of integers has an extremely long and rich history.
It falls under an area of study called additive number theory.

Let's consider the number 6.
You can write 6 as a sum of positive integers (not including 0) like so:
\begin{align*}
6
&= 6 \\
&= 5 + 1\\
&= 4 + 2 \\
&= 4 + 1 + 1 \\
&= 3 + 3 \\
&= 3 + 2 + 1 \\
&= 3 + 1 + 1 + 1 \\
&= 2 + 2 + 2 \\
&= 2 + 2 + 1 + 1 \\
&= 2 + 1 + 1 + 1 + 1 \\
&= 1 + 1 + 1 + 1 + 1 + 1 
\end{align*}
Each sum on the right (including $6$) is called a \defone{partition} of $6$.
Note that $1 + 5$ is not listed: $1 + 5$ and $5 + 1$ are considered
the same partition.
We say that there are 11 partitions for $6$.
We write $p(n)$\tinysidebar{$p(n)$}\index{p@$p(n)$} for the number of partitions of $n$;
$p(6) = 11$.



\begin{ex} 
  \label{ex:some-decision1}
  \tinysidebar{\debug{exercises/{empty0/question.tex}}}
  \solutionlink{sol:some-decision1}
  \qed
\end{ex} 
\begin{python0}
from solutions import *
add(label="ex:some-decision1",
    srcfilename='exercises/some-decision1/answer.tex') 
\end{python0}



\begin{ex} 
  \label{ex:some-decision1}
  \tinysidebar{\debug{exercises/{empty0/question.tex}}}
  \solutionlink{sol:some-decision1}
  \qed
\end{ex} 
\begin{python0}
from solutions import *
add(label="ex:some-decision1",
    srcfilename='exercises/some-decision1/answer.tex') 
\end{python0}


\newpage
Note that $p(6)$ is the number of solutions $(a_1, a_2, a_3, a_4, a_5, a_6)$ to
\[
a_1 +  a_2 +  a_3 + a_4 + a_5 + a_6 = 6
\]
where 
\begin{align*}
a_1 &\in \{0, 1, 2, 3, \ldots\} \\
a_2 &\in \{0, 2, 4, 6, \ldots\} \\
a_3 &\in \{0, 3, 6, 9, \ldots\} \\
a_4 &\in \{0, 4, 8, 12, \ldots\} \\
a_5 &\in \{0, 5, 10, 15, \ldots\} \\
a_6 &\in \{0, 6, 12, 18, \ldots\} 
\end{align*}
Right? 
And of course the generation function is
\begin{align*}
f(x) 
&= (1 + x + x^2 + x^3 + \cdots ) \\
& \,\,\,\,\, \times (1 + x^2 + x^4 + x^6 + \cdots ) \\
& \,\,\,\,\, \times (1 + x^3 + x^6 + x^9 + \cdots ) \\
& \,\,\,\,\, \times (1 + x^4 + x^8 + x^{12} + \cdots ) \\
& \,\,\,\,\, \times (1 + x^5 + x^{10} + x^{15} + \cdots ) \\
& \,\,\,\,\, \times (1 + x^6 + x^{12} + x^{18} + \cdots ) 
\end{align*}
(Refer to the previous section on linear forms.)

Of course if you want to consider all the $p(n)$ you need to consider
\begin{align*}
f(x) 
&= (1 + x + x^2 + x^3 + \cdots ) \\
& \,\,\,\,\, \times (1 + x^2 + x^4 + x^6 + \cdots ) \\
& \,\,\,\,\, \times (1 + x^3 + x^6 + x^9 + \cdots ) \\
& \,\,\,\,\, \times (1 + x^4 + x^8 + x^{12} + \cdots ) \\
& \,\,\,\,\, \times (1 + x^5 + x^{10} + x^{15} + \cdots ) \\
& \,\,\,\,\, \times (1 + x^6 + x^{12} + x^{18} + \cdots ) \\
& \,\,\,\,\, \times \cdots \\
&= \frac{1}{1 - x} \cdot \frac{1}{1 - x^2} \cdot \frac{1}{1 - x^3} \cdot \frac{1}{1 - x^4} \cdot \cdots \\
&= \prod_{n=1}^\infty \frac{1}{1 - x^n}
\end{align*}
The $\prod_{n=0}^\infty$ is analogous to the $\sum_{n=0}^\infty$ notation
except that you take the product instead of the sum.
So now I have the generating function for $p(n)$ ($n=0, 1, 2, 3, \ldots$):
\[
\sum_{n=0}^\infty p(n) x^n =  \prod_{n=1}^\infty \frac{1}{1 - x^n}
\]

Besides talking about partitions of $n$,
one can also restrict to certain special partitions.
For instance from the partition of 6
\begin{align*}
6
&= 6 \\
&= 5 + 1\\
&= 4 + 2 \\
&= 4 + 1 + 1 \\
&= 3 + 3 \\
&= 3 + 2 + 1 \\
&= 3 + 1 + 1 + 1 \\
&= 2 + 2 + 2 \\
&= 2 + 2 + 1 + 1 \\
&= 2 + 1 + 1 + 1 + 1 \\
&= 1 + 1 + 1 + 1 + 1 + 1 
\end{align*}
we can look at the \defone{odd partitions} (i.e. partitions of $6$ using only odd terms):
\begin{align*}
6
&= 5 + 1\\
&= 3 + 3 \\
&= 3 + 1 + 1 + 1 \\
&= 1 + 1 + 1 + 1 + 1 + 1 
\end{align*}
We write
\[
p_o(6) = 4\index{p0@$p_0(n)$}
\]

We can also talk about \defone{partitions with distinct terms} of $6$.
In this case we throw away the partition $4 + 1 + 1$ since $1$ is used twice.
The partitions with distinct terms of 6 are
\begin{align*}
6
&= 6 \\
&= 5 + 1\\
&= 4 + 2 \\
&= 3 + 2 + 1 
\end{align*}
We write
\[
p_d(6) = 4\index{pd@$p_d(n)$}
\]



\begin{ex} 
  \label{ex:some-decision1}
  \tinysidebar{\debug{exercises/{empty0/question.tex}}}
  \solutionlink{sol:some-decision1}
  \qed
\end{ex} 
\begin{python0}
from solutions import *
add(label="ex:some-decision1",
    srcfilename='exercises/some-decision1/answer.tex') 
\end{python0}



\begin{ex} 
  \label{ex:some-decision1}
  \tinysidebar{\debug{exercises/{empty0/question.tex}}}
  \solutionlink{sol:some-decision1}
  \qed
\end{ex} 
\begin{python0}
from solutions import *
add(label="ex:some-decision1",
    srcfilename='exercises/some-decision1/answer.tex') 
\end{python0}



\begin{ex} 
  \label{ex:some-decision1}
  \tinysidebar{\debug{exercises/{empty0/question.tex}}}
  \solutionlink{sol:some-decision1}
  \qed
\end{ex} 
\begin{python0}
from solutions import *
add(label="ex:some-decision1",
    srcfilename='exercises/some-decision1/answer.tex') 
\end{python0}


\newpage
We will show that $p_d(n) = p_o(n)$ for \textit{all} $n$.
In fact we'll show that their generating functions are the same
\[
\sum_{n=0}^\infty p_d(n) x^n = 
\sum_{n=0}^\infty p_o(n) x^n 
\]
thereby proving all the equalities in one step.
To do that we'll need to write down the generating functions for $p_d(n)$
and $p_o(n)$.

The generating function for $p_d(n)$ ($n=0,1,...$) is
\[
\sum_{n=0}^\infty p_d(n) x^n
= (1 + x) (1 + x^2) (1 + x^3) \cdots
\]

The generating function for $p_o(n)$ ($n=0,1,...$) is
\begin{align*}
\sum_{n=0}^\infty p_o(n) x^n 
&= (1 + x + x^2 + x^3 + \cdots ) \\
& \,\,\,\,\, \times (1 + x^3 + x^6 + x^9 + \cdots ) \\
& \,\,\,\,\, \times (1 + x^5 + x^{10} + x^{15} + \cdots ) \\
& \,\,\,\,\, \times \cdots \\
&= \frac{1}{1 - x} \cdot \frac{1}{1 - x^3} \cdot\frac{1}{1 - x^5} \cdots 
\end{align*}
We want to show that this is $(1 + x) (1 + x^2) (1 + x^3) \cdots$.
Hmmm ... let's just introduce the factors of this function into $\sum_{n=0}^\infty p_o(n)x^n$.
Let me push $1+x$ into the function like this:
\begin{align*}
\sum_{n=0}^\infty p_o(n) x^n 
&= \frac{1+x}{1+x} \cdot \frac{1}{1 - x} \cdot \frac{1}{1 - x^3} \cdot\frac{1}{1 - x^5} \cdots 
\end{align*}
Now this is
\begin{align*}
\sum_{n=0}^\infty p_o(n) x^n 
&= (1+x) \cdot \frac{1}{(1+x)(1-x)} \cdot \frac{1}{1 - x^3} \cdot\frac{1}{1 - x^5} \cdots \\ 
&= (1+x) \cdot \frac{1}{1 - x^2} \cdot \frac{1}{1 - x^3} \cdot\frac{1}{1 - x^5} \cdots 
\end{align*}
Now what if we do the same by insert $(1+x^2)/(1+x^2)$?
\begin{align*}
\sum_{n=0}^\infty p_o(n) x^n 
&= (1+x) \frac{1+x^2}{1+x^2}\cdot \frac{1}{1 - x^2} \cdot \frac{1}{1 - x^3} \cdot\frac{1}{1 - x^5} \cdots 
\end{align*}
Whoa! It becomes:
\begin{align*}
\sum_{n=0}^\infty p_o(n) x^n 
&= (1+x)(1+x^2) \frac{1}{(1+x^2)(1 - x^2)} \cdot \frac{1}{1 - x^3} \cdot\frac{1}{1 - x^5} \cdots \\
&= (1+x)(1+x^2) \cdot \frac{1}{1 - x^4} \cdot \frac{1}{1 - x^3} \cdot\frac{1}{1 - x^5} \cdots 
\end{align*}
and putting $1/(1-x^4)$ into the right place:
\begin{align*}
\sum_{n=0}^\infty p_o(n) x^n 
&= (1+x)(1+x^2) \frac{1}{(1+x^2)(1 - x^2)} \cdot \frac{1}{1 - x^3} \cdot\frac{1}{1 - x^5} \cdots \\
&= (1+x)(1+x^2) \cdot  \frac{1}{1 - x^3} \cdot \frac{1}{1 - x^4} \cdot \frac{1}{1 - x^5} \cdots 
\end{align*}
It should be clear now that the next step gives us
\begin{align*}
\sum_{n=0}^\infty p_o(n) x^n 
&= (1+x)(1+x^2)(1+x^3) \cdot  \frac{1}{1 - x^6} \cdot \frac{1}{1 - x^4} \cdot \frac{1}{1 - x^5} \cdots  \\
&= (1+x)(1+x^2)(1+x^3) \cdot   \frac{1}{1 - x^4} \cdot \frac{1}{1 - x^5} \cdot \frac{1}{1 - x^6} \cdots 
\end{align*}
And then
\begin{align*}
\sum_{n=0}^\infty p_o(n) x^n 
&= (1+x)(1+x^2)(1+x^3)(1+x^4) \cdot \frac{1}{1 - x^8} \cdot \frac{1}{1 - x^5} \cdot \frac{1}{1 - x^6} \cdots \\
&= (1+x)(1+x^2)(1+x^3)(1+x^4) \cdot \frac{1}{1 - x^5} \cdot 
\frac{1}{1 - x^6} \cdot \frac{1}{1 - x^7} \cdot \frac{1}{1 - x^8} \cdots \\
&= \ldots \\
\end{align*}
You see that the power series is the same as $\sum_{n=0}^\infty p_d(n) x^n$.

In the study of combinatorics and discrete math, 
there are many \lq\lq diagram'' techniques.
The \defone{Ferrers diagram} is a diagram for finding new partitions.
Here's a partition of 7:
\[
4 + 2 + 1
\]
The Ferrers diagram of this partition is
\begin{verbatim}
    * * * *
    * * 
    *
\end{verbatim}
Now if you reflect this diagram about the diagonal
that runs from the top left corner
to the bottom right corner of the diagram 
you get this:
\begin{verbatim}
    * * * 
    * * 
    *
    *
\end{verbatim}
(i.e. make each row of the diagram a column).
You get a new partition
\[
3 + 2 + 1 + 1
\]
Of course this is a partition of 7 since the number of dots
is the same.
You will of course get a new partition if the diagram 
is not symmetric about the mirror line.
For instance the partition
\[
4 + 2 + 1 + 1
\]
of 8 has the Ferrers diagram
\begin{verbatim}
    * * * *
    * *
    *
    *
\end{verbatim}
which when reflected also gives
\begin{verbatim}
    * * * *
    * *
    *
    *
\end{verbatim}
which gives the same partition.



\begin{ex} 
  \label{ex:some-decision1}
  \tinysidebar{\debug{exercises/{empty0/question.tex}}}
  \solutionlink{sol:some-decision1}
  \qed
\end{ex} 
\begin{python0}
from solutions import *
add(label="ex:some-decision1",
    srcfilename='exercises/some-decision1/answer.tex') 
\end{python0}



\begin{ex} 
  \label{ex:some-decision1}
  \tinysidebar{\debug{exercises/{empty0/question.tex}}}
  \solutionlink{sol:some-decision1}
  \qed
\end{ex} 
\begin{python0}
from solutions import *
add(label="ex:some-decision1",
    srcfilename='exercises/some-decision1/answer.tex') 
\end{python0}



\begin{ex} 
  \label{ex:some-decision1}
  \tinysidebar{\debug{exercises/{empty0/question.tex}}}
  \solutionlink{sol:some-decision1}
  \qed
\end{ex} 
\begin{python0}
from solutions import *
add(label="ex:some-decision1",
    srcfilename='exercises/some-decision1/answer.tex') 
\end{python0}

%
%    *****
%    *
%    *
%    *
%    *
%


\begin{ex} 
  \label{ex:some-decision1}
  \tinysidebar{\debug{exercises/{empty0/question.tex}}}
  \solutionlink{sol:some-decision1}
  \qed
\end{ex} 
\begin{python0}
from solutions import *
add(label="ex:some-decision1",
    srcfilename='exercises/some-decision1/answer.tex') 
\end{python0}

% 2 of the 5 x's on both arms can be moved:
%
%    **xxxxx    
%    *o         
%    x          
%    x          
%    x          
%    x          
%    x          
%
%    **xxxx    
%    *ox         
%    xx          
%    x          
%    x          
%    x                  
%
%    **xxx    
%    *oxx         
%    xx          
%    xx          
%    x



\begin{ex} 
  \label{ex:some-decision1}
  \tinysidebar{\debug{exercises/{empty0/question.tex}}}
  \solutionlink{sol:some-decision1}
  \qed
\end{ex} 
\begin{python0}
from solutions import *
add(label="ex:some-decision1",
    srcfilename='exercises/some-decision1/answer.tex') 
\end{python0}



\begin{ex} 
  \label{ex:some-decision1}
  \tinysidebar{\debug{exercises/{empty0/question.tex}}}
  \solutionlink{sol:some-decision1}
  \qed
\end{ex} 
\begin{python0}
from solutions import *
add(label="ex:some-decision1",
    srcfilename='exercises/some-decision1/answer.tex') 
\end{python0}

