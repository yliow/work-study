%\section{Products of power series}
\begin{python0}
from solutions import *; clear() 
\end{python0}

Note that from the previous section, sums of two power series
can be easily calculated to form a single power series.
We also talked about multiplying a \textit{polynomial} with a power series.
What about the product of a \textit{power series} with another power series?
We have this:
\begin{align*}
  \sum_{n=0}^\infty a_n x^n \cdot \sum_{n=0}^\infty b_n x^n
  &= \left(a_0 + a_1x + a_2x^2 + a_3x^3 + \cdots \right) \cdot
  \left(b_0 + b_1x + b_2x^2 + b_3x^3 + \cdots \right)
  \\
  &= a_0 b_0 \\
  &\hskip 0.5cm + (a_0 b_1 + a_1 b_0) x \\ 
  &\hskip 0.5cm + (a_0 b_2 + a_1 b_1 + a_2 b_0) x^2 \\ 
  &\hskip 0.5cm + (a_0 b_3 + a_1 b_2 + a_2 b_1 + a_3 b_0) x^3 \\
  &\hskip 0.5cm + (a_0 b_4 + a_1 b_3 + a_2 b_2 + a_3 b_1 + a_4 b_0) x^4 \\
  &\hskip 0.5cm + \cdots
\end{align*}
Note that the $x^n$--term looks like this:
\[
\left( a_0 b_n + a_1 b_{n-1} + \cdots a_n b_{0} \right) x^n = \left( \sum_{k=0}a_kb_{n-k} \right) x^n
\]
Some people also write the summation on the right this way:
\[
\sum_{k=0}^n a_kb_{n-k}
= \sum_{k + \ell = n} a_k b_\ell
\]
where the summation on the right
\lq\lq $\sum_{k + \ell = n} (...)$" means
\lq\lq sum over all terms $(...)$
where integers $k\geq 0$ and $\ell\geq 0$ such that 
$k + \ell = n$".
Let me state the above as a small proposition:

\begin{prop}\label{prop:product-of-power-series}
  \[
  \sum_{n=0}^\infty a_n x^n
  \cdot
  \sum_{n=0}^\infty b_n x^n
  =
  \sum_{n=0}^\infty \left( \sum_{k=0}^n a_k b_{n-k} \right) x^n
  \]
\end{prop}

By the way, I hope it's clear that the above also applies to
the product of two polynomials.
Given two polynomials to multiply, you can
extend them to power series by including terms of the form
$0x^n$ to each of them:
\[
(1 + 2x + 5x^2)(3 + x -7x^2)
=
(1 + 2x + 5x^2 + 0x^3 + 0x^4 + \cdots)
(3 + x -7x^2 + 0x^3 + 0x^4 + \cdots)
\]

As an aside, if you are given two finite sequences of the same length,
$(a_0, ..., a_n)$ and
$(b_0, ..., b_n)$, there are two \lq\lq products" for these two sequences
that you will see frequently in CS, math, engineering, physics, etc.
The \defone{dot product} of 
$(a_0, ..., a_n)$ and
$(b_0, ..., b_n)$ is defined to be
\[
(a_0, ..., a_n) \cdot (b_0, ..., b_n) = a_0b_0 + \cdots a_nb_n = \sum_{k=0}^n a_k b_k
\]
The \defone{convolution} 
of 
$(a_0, ..., a_n)$ and
$(b_0, ..., b_n)$ is defined to be
\[
(a_0, ..., a_n) * (b_0, ..., b_n)
= a_0b_n + a_1b_{n-1} + \cdots + a_{n-1}b_1 + a_n b_0 = \sum_{k + \ell = n} a_kb_\ell
\]
For dot products, the sequences come from vectors
and can be used to detect if two vectors are parallel and can be used
to define the angle between the two vectors.
Convolutions come from many areas, not just in the coefficients of
products of power series.
If you want to analyze some image data and $(b_0,...,b_n)$ is a collection of adjacent
pixels, then for a specially chosen $(a_0, ..., a_n)$,
the convolution will give you some useful about $(b_0,...,b_n)$.
These special $(a_0, ..., a_n)$ are called filters and appear in
image processing and computer vision such as
cleaning up an image, blurring an image, 
sharpening objects in the image, edge detection of objects in the image,
facial detection, etc.
Convolution is also used in analyzing audio streams and video streams.

\newpage
\begin{ex} 
  \label{ex:some-decision1}
  \tinysidebar{\debug{exercises/{empty0/question.tex}}}
  \solutionlink{sol:some-decision1}
  \qed
\end{ex} 
\begin{python0}
from solutions import *
add(label="ex:some-decision1",
    srcfilename='exercises/some-decision1/answer.tex') 
\end{python0}


\newpage
Let's look at one of the simplest power series:
\[
\sum_{n=0}^\infty x^n
\]
What if you need to multiply two of them?
\[
\sum_{n=0}^\infty x^n \cdot \sum_{n=0}^\infty x^n 
\]
Of course you can expand the series out and multiply term-by-term:
\begin{align*}
&\sum_{n=0}^\infty x^n \cdot \sum_{n=0}^\infty x^n  \\
&= (1 + x + x^2 + x^3 + \cdots ) (1 + x + x^2 + x^3 + \cdots)  \\
&= 
1\cdot 1 + 
(1\cdot x + x\cdot 1) + 
(1 \cdot x^2 + x \cdot x + x^2 \cdot 1) + 
(1 \cdot x^3 + x \cdot x^2 + x^2 \cdot x + x^3 \cdot 1) + 
\cdots
\\
&= 
1 + 
2x + 
3x^2 + 
4x^3 + 
\cdots
\end{align*}
Once you've done enough terms (say up to $x^5$), you suspect the product is
\[
\sum_{n = 0}^\infty (n+1)x^n
\]
But a verification up $n = 5$ does not mean a conjecture is true for all $n \geq 0$.
Furthermore, what if the product is more complex:
\[
\sum_{n=0}^\infty \frac{1}{4^n} x^n \cdot 
\sum_{n=0}^\infty
\frac{2^n}{3^n} x^n 
\]
so that making a guess is not that easy,
plus the verification is usually too time 
consuming and error prone. So what we should we do?
Is there a faster and easier way to compute the product?

If we go back to the first example:
\[
\sum_{n=0}^\infty x^n \cdot \sum_{n=0}^\infty x^n 
\]
The trick is to note that
\[
\sum_{n=0}^\infty x^n = \frac{1}{1 - x}
\]
and hence
\[
\sum_{n=0}^\infty x^n \cdot \sum_{n=0}^\infty x^n 
=
\left( \frac{1}{1 - x} \right) \cdot
\left( \frac{1}{1 - x} \right)
=
\left( \frac{1}{1 - x} \right)^2
\]
And using the formula for powers of the geometric series (case of $k = 2$):
\[
\biggl( \frac{1}{1 - x} \biggr)^k
= \sum_{n=0}^\infty \binom{k + n - 1}{n} x^n
\]
we have
\begin{align*}
\sum_{n=0}^\infty x^n \cdot \sum_{n=0}^\infty x^n
&= \left( \frac{1}{1 - x} \right)^2 \\
&= \sum_{n=0}^\infty \binom{2 + n - 1}{n} x^n \\
&= \sum_{n=0}^\infty \binom{n + 1}{n} x^n \\
&= \sum_{n=0}^\infty \binom{n + 1}{1} x^n \\
&= \sum_{n=0}^\infty (n+1) x^n
\end{align*}
The above does not involve verification and making a guess on coefficients
up to $n = 5$.
The computation is simply true \textit{for all} $n \geq 0$.

I hope you realize the reason the second computation
is simpler than the first attempt is because this computation
\begin{align*}
&\sum_{n=0}^\infty x^n \cdot \sum_{n=0}^\infty x^n  \\
&= (1 + x + x^2 + x^3 + \cdots ) (1 + x + x^2 + x^3 + \cdots)  \\
&= 
1\cdot 1 + 
(1\cdot x + x\cdot 1) + 
(1 \cdot x^2 + x \cdot x + x^2 \cdot) + 
(1 \cdot x^3 + x \cdot x^2 + x^2 \cdot x + x^3 \cdot 1) + 
\cdots
\end{align*}
involves pairing
\textit{infinitely} many $x^n$ from the left power series
with 
\textit{infinitely} many $x^n$ from the right power series.
But the second computation
\begin{align*}
\sum_{n=0}^\infty x^n \cdot \sum_{n=0}^\infty x^n
&= \biggl( \frac{1}{1 - x} \biggr) \cdot \biggl( \frac{1}{1 - x} \biggr)
\end{align*}
converts the power series to rational expressions
$\displaystyle \frac{1}{1 - x}$ which are finitary objects.
See it?

\newpage
\begin{ex} 
  \label{ex:some-decision1}
  \tinysidebar{\debug{exercises/{empty0/question.tex}}}
  \solutionlink{sol:some-decision1}
  \qed
\end{ex} 
\begin{python0}
from solutions import *
add(label="ex:some-decision1",
    srcfilename='exercises/some-decision1/answer.tex') 
\end{python0}

\newpage
\begin{ex} 
  \label{ex:some-decision1}
  \tinysidebar{\debug{exercises/{empty0/question.tex}}}
  \solutionlink{sol:some-decision1}
  \qed
\end{ex} 
\begin{python0}
from solutions import *
add(label="ex:some-decision1",
    srcfilename='exercises/some-decision1/answer.tex') 
\end{python0}


\newpage
Of course this helps only when the series we're multiplying
together are all
\[
\sum_{n=0}^\infty x^n
\]
What if they are different?
What about 
\[
\sum_{n=0}^\infty \frac{1}{4^n} x^n 
\cdot 
\sum_{n=0}^\infty \frac{2^n}{3^n} x^n 
\]
Then using the geometric series, with $x$ replaced by $x/4$,
\[
\sum_{n=0}^\infty \biggl( \frac{x}{4} \biggr) ^n = \frac{1}{1-x/4}
= \frac{4}{4 - x}
\]
Likewise
\begin{align*}
\sum_{n=0}^\infty \frac{2^n}{3^n} x^n 
&= \sum_{n=0}^\infty \biggr( \frac{2x}{3} \biggl)^n \\
&= \frac{1}{1 - 2x/3} \\
&= \frac{3}{3 - 2x} \\
\end{align*}
So we now know that
\[
\sum_{n=0}^\infty \frac{1}{4^n} x^n 
\cdot 
\sum_{n=0}^\infty \frac{2^n}{3^n} x^n 
= \frac{4}{4 - x} \cdot  \frac{3}{3 - 2x}
= \frac{12}{(4-x)(3-2x)} 
\]
Of course we can try to 
apply the Maclaurin/Taylor series to develop a power series for the 
function
\[
\frac{12}{(4-x)(3-2x)} 
\]
But let's not forget that coefficients provided by the Maclaurin/Taylor
series is
\[
\frac{f^{(n)}(0)}{n!}
\]
which requires you to compute the $n$--th derivative until you see
a pattern.
That's time-consuming and also error-prone.

We'll solve this problem via a different route:
We'll use the theory of partial fractions.
According to the theory of partial fractions, there are constants $A$ and $B$ such that
\[
\frac{1}{(4-x)(3-2x)}
=
\frac{A}{4-x} +
\frac{B}{3-2x}
\]
More generally, if $ax + b$ and $cx + d$ are degree 1 polynomials with
distinct roots (i.e., $-b/a \neq -d/c$), then  
\[
\frac{1}{(ax+b)(cx+d)}
=
\frac{A}{ax+b} +
\frac{B}{cx+d}
\]
where $A,B$ are constants.

Assuming we can find the constants $A$ and $B$, and here's the
kicker, notice that ... \textbf{the theory of partial fractions has
changed the product problem into an addition problem!}
Get it? Look at this:
\[
\frac{12}{(4-x)(3-2x)} 
= 12 \biggl( 
\frac{A}{4-x} +
\frac{B}{3-2x}
\biggr)
\]
If we have $A$, we can definitely write 
\[
\frac{A}{4-x}
\]
as a power series.
Likewise for $\displaystyle \frac{B}{3-2x}$.
We then add up the series to get a new series.
We'll get back to the general theory of partial fractions later.
Right now you only need to know that 
this process of breaking up product of functions into sums of functions
does not work for all functions.
For instance it is \textit{not} true that you can find constants $A$ and $B$
such that
\[
\frac{1}{\sin x \cdot \cos x}
=
\frac{A}{\sin x} + 
\frac{B}{\cos x}
\]
The theory of partial fractions applies only to rational functions, 
i.e. functions which are fractions of polynomials.
Examples are
\[
\frac{x}{(2x + 1)(3x+5)}, \,\,\,\,\,
\frac{2x + 1}{(2x + 1)^2(3x^2+5)}, \,\,\,\,\,
\text{etc.}
\]

OK, let's get back to finding $A$ and $B$.
Here's what the theory of partial fractions says:
\[
\frac{1}{(4-x)(3-2x)}
=
\frac{A}{4-x} +
\frac{B}{3-2x}
\]
There are two standard ways to find $A$ and $B$.
I'll show you one of them.
You first clear denominators by multiplying both sides
of the equation with $(4-x)(3-2x)$ to get
\[
1
=
A(3 - 2x) +
B(4 - x)
\]
This is an identity involving two unknowns.
All you need to do is to substitute two values of $x$ into the 
identity to get two equations.
You can then solve for $A$ and $B$ quickly.
I'll pick the easiest values for $x$,
i.e., $x$ values that will force some of the terms to be zero.
First I'll choose $x = 4$.
This gives me
\[
1 = A(3 - 8) + 0
\]
which gives me
\[
A = -\frac{1}{5}
\]
Next I'll choose $x = 3/2$.
This gives us
\[
1 = 0 + B(4 - 3/2)
\]
i.e.
\[
B = \frac{2}{5}
\]
Substituting the values of $A$ and $B$ we get
\[
\sum_{n=0}^\infty \frac{1}{4^n} x^n 
\cdot 
\sum_{n=0}^\infty \frac{2^n}{3^n} x^n 
= \frac{12}{(4-x)(3-2x)} 
= 12 \biggl( 
-\frac{1}{5} \cdot \frac{1}{4-x} +
\frac{2}{5}\cdot \frac{1}{3-2x}
\biggr)
\]
Now we simply note that
\begin{align*}
\frac{1}{4-x} 
&= \frac{1}{4} \cdot \frac{1}{1 - x/4}  \\
&= \frac{1}{4} \sum_{n=0}^\infty \biggl(\frac{x}{4} \biggr)^n \\
&= \sum_{n=0}^\infty \frac{1}{4} \biggl( \frac{1}{4} \biggr)^n x^n\\
\end{align*}
and
\begin{align*}
\frac{1}{3-2x} 
&= \frac{1}{3} \cdot \frac{1}{1 - 2x/3}  \\
&= \frac{1}{3} \sum_{n=0}^\infty \biggl(\frac{2x}{3} \biggr)^n \\
&= \sum_{n=0}^\infty \frac{1}{3} \biggl( \frac{2}{3} \biggr)^n x^n\\
\end{align*}
Putting these series into the above equation we get
\begin{align*}
\sum_{n=0}^\infty \frac{1}{4^n} x^n 
\cdot 
\sum_{n=0}^\infty \frac{2^n}{3^n} x^n  
&= 
12 \biggl( 
-\frac{1}{5} \cdot \frac{1}{4-x} +
\frac{2}{5} \cdot \frac{1}{3-2x}
\biggr) \\
&=
12 \biggl( 
-\frac{1}{5}  \sum_{n=0}^\infty \frac{1}{4} \biggl( \frac{1}{4} \biggr)^n x^n +
\frac{2}{5}  \sum_{n=0}^\infty \frac{1}{3} \biggl( \frac{2}{3} \biggr)^n x^n
\biggr) \\
&=
\sum_{n=0}^\infty 
12 \biggl( 
-\frac{1}{5} \cdot \frac{1}{4} \biggl( \frac{1}{4} \biggr)^n +
\frac{2}{5} \cdot \frac{1}{3} \biggl( \frac{2}{3} \biggr)^n
\biggr) x^n\\
&=
\sum_{n=0}^\infty 
\biggl( 
-\frac{3}{5} \biggl( \frac{1}{4} \biggr)^n +
\frac{8}{5} \biggl( \frac{2}{3} \biggr)^n
\biggr) x^n
\end{align*}
I'll check my work with this program
\VerbatimInput[frame=single,fontsize=\footnotesize]{test0.py}
The output is
\VerbatimInput[frame=single,fontsize=\footnotesize]{test0.txt}

Voil\`a!
As you can see I've bypassed the 
method of multiplying lots of terms and then try to guess
the formula for the coefficients (more or less the brute-force
method).
Note that we use the theory of partial fractions to 
break up a complex rational expression into
into a linear sum of simpler rational expressions (called partial fractions).

Make sure you get the big picture:
You want to multiply two power series to get a power series.
Instead of multiplying the two power series, you take a detour,
and rewrite the two power series as rational expressions
and multiply the two rational expressions because
multiplying two rational expressions is easier.
Here's a summary of the above computations:

\newcommand\tmpAA{\sum_{n=0}^\infty \frac{1}{4^n} x^n \cdot \sum_{n=0}^\infty \frac{2^n}{3^n} x^n }
\newcommand\tmpBB{\frac{4}{4 - x} \cdot  \frac{3}{3 - 2x}}
\newcommand\tmpCC{\frac{12}{(4-x)(3-2x)}}
\newcommand\tmpDD{12 \biggl( -\frac{1}{5} \cdot \frac{1}{4-x} + \frac{2}{5}\cdot \frac{1}{3-2x} \biggr)}
\newcommand\tmpEE{12 \biggl( 
-\frac{1}{5}  \sum_{n=0}^\infty \frac{1}{4} \biggl( \frac{1}{4} \biggr)^n x^n +
\frac{2}{5}  \sum_{n=0}^\infty \frac{1}{3} \biggl( \frac{2}{3} \biggr)^n x^n
\biggr) 
}
\newcommand\tmpFF{
\sum_{n=0}^\infty 
\biggl( 
-\frac{3}{5} \biggl( \frac{1}{4} \biggr)^n +
\frac{8}{5} \biggl( \frac{2}{3} \biggr)^n
\biggr) x^n
}

{\footnotesize
\begin{align*}
&\tmpAA \\ 
&\xlongequal{(a)} \tmpBB & & \text{convert power series to rational expression} \\
&\xlongequal{(b)} \tmpCC & & \text{multiply rational expressions (instead of power series)}\\
&\xlongequal{(c)} \tmpDD & & \text{write rational expression as linear sum of rational expressions}\\
&\xlongequal{(d)} \tmpEE & & \text{convert rational expressions to power series} \\
&\xlongequal{(d)} \tmpFF & & \text{tidy up to one power series}
\end{align*}
}

This section is a quick introduction to (c), one special case of
writing a complex rational expression to a linear sum of
rational functions:
if $ax + b$ and $cx + d$ are linear polynomials with distinct roots,
then there are constants $A, B$ such that
\[
\frac{1}{(ax + b)(cx + d)} = \frac{A}{ax + b} + \frac{B}{cx + d}
\]
In the next section, I'll go into details about
partial fractions decomposition to handle more cases.
And of course if the product is a product of the form
\[
\frac{1}{1 - x}
\cdot
\frac{1}{1 - x}
\]
then you do not need partial fractions, since you can
use the formula for powers of geometric series.


\newpage
\begin{ex} 
  \label{ex:some-decision1}
  \tinysidebar{\debug{exercises/{empty0/question.tex}}}
  \solutionlink{sol:some-decision1}
  \qed
\end{ex} 
\begin{python0}
from solutions import *
add(label="ex:some-decision1",
    srcfilename='exercises/some-decision1/answer.tex') 
\end{python0}


\newpage
\subsection*{Solutions}

\newpage
\section*{Solutions}
Solution to Exercise \ref{ex:dfa0}\labeltext{}{sol:dfa0}.

\tinysidebar{\debug{exercises/{dfa0/answer.tex}}}

    Solution not provided.
    

\newpage

Solution to Exercise \ref{ex:dfa1}\labeltext{}{sol:dfa1}.

\tinysidebar{\debug{exercises/{dfa1/answer.tex}}}
  The ID computation is
  \begin{align*}
    (q_0, aba)
    &\vdash (\delta(q_0, a), ba) = (q_0, ba) \\ 
    &\vdash (\delta(q_0, b), a) = (q_1, a) \\
    &\vdash (\delta(q_1, a), \ep) = (q_0, \ep)
  \end{align*}
  $q_0$ is not an accept state. Therefore $aba$ is not accepted.


\newpage

Solution to Exercise \ref{ex:dfa4}\labeltext{}{sol:dfa4}.

\tinysidebar{\debug{exercises/{dfa4/answer.tex}}}

    Solution not provided.
    

\newpage

Solution to Exercise \ref{ex:dfa5}\labeltext{}{sol:dfa5}.

\tinysidebar{\debug{exercises/{dfa5/answer.tex}}}

    Solution not provided.
    

\newpage

Solution to Exercise \ref{ex:implementing-a-single-dfa0}\labeltext{}{sol:implementing-a-single-dfa0}.

\tinysidebar{\debug{exercises/{implementing-a-single-dfa0/answer.tex}}}

    Solution not provided.
    

\newpage

Solution to Exercise \ref{ex:nfastatediag0}\labeltext{}{sol:nfastatediag0}.

\tinysidebar{\debug{exercises/{nfastatediag0/answer.tex}}}

    Solution not provided.
    

\newpage

Solution to Exercise \ref{ex:nfastatediag1}\labeltext{}{sol:nfastatediag1}.

\tinysidebar{\debug{exercises/{nfastatediag1/answer.tex}}}

    Solution not provided.
    

\newpage

Solution to Exercise \ref{ex:nfastatediag2}\labeltext{}{sol:nfastatediag2}.

\tinysidebar{\debug{exercises/{nfastatediag2/answer.tex}}}

    Solution not provided.
    

\newpage

Solution to Exercise \ref{ex:nfastatediag3}\labeltext{}{sol:nfastatediag3}.

\tinysidebar{\debug{exercises/{nfastatediag3/answer.tex}}}

    Solution not provided.
    

\newpage

Solution to Exercise \ref{ex:nfastatediag4}\labeltext{}{sol:nfastatediag4}.

\tinysidebar{\debug{exercises/{nfastatediag4/answer.tex}}}

    Solution not provided.
    

\newpage

Solution to Exercise \ref{ex:nfastatediag5}\labeltext{}{sol:nfastatediag5}.

\tinysidebar{\debug{exercises/{nfastatediag5/answer.tex}}}

    Solution not provided.
    

\newpage

Solution to Exercise \ref{ex:nfastatediag6}\labeltext{}{sol:nfastatediag6}.

\tinysidebar{\debug{exercises/{nfastatediag6/answer.tex}}}

    Solution not provided.
    

\newpage

Solution to Exercise \ref{ex:nfastatediag7}\labeltext{}{sol:nfastatediag7}.

\tinysidebar{\debug{exercises/{nfastatediag7/answer.tex}}}

    Solution not provided.
    

\newpage

Solution to Exercise \ref{ex:nfastatediag8}\labeltext{}{sol:nfastatediag8}.

\tinysidebar{\debug{exercises/{nfastatediag8/answer.tex}}}

    Solution not provided.
    

\newpage

Solution to Exercise \ref{ex:nfastatediag9}\labeltext{}{sol:nfastatediag9}.

\tinysidebar{\debug{exercises/{nfastatediag9/answer.tex}}}

    Solution not provided.
    

\newpage

Solution to Exercise \ref{ex:nfastatediag10}\labeltext{}{sol:nfastatediag10}.

\tinysidebar{\debug{exercises/{nfastatediag10/answer.tex}}}

    Solution not provided.
    

\newpage

Solution to Exercise \ref{ex:nfastatediag11}\labeltext{}{sol:nfastatediag11}.

\tinysidebar{\debug{exercises/{nfastatediag11/answer.tex}}}

    Solution not provided.
    

\newpage

Solution to Exercise \ref{ex:nfastatediag12}\labeltext{}{sol:nfastatediag12}.

\tinysidebar{\debug{exercises/{nfastatediag12/answer.tex}}}

    Solution not provided.
    

\newpage

Solution to Exercise \ref{ex:nfastatediag13}\labeltext{}{sol:nfastatediag13}.

\tinysidebar{\debug{exercises/{nfastatediag13/answer.tex}}}

    Solution not provided.
    

\newpage

Solution to Exercise \ref{ex:nfa0}\labeltext{}{sol:nfa0}.

\tinysidebar{\debug{exercises/{nfa0/answer.tex}}}
The formal definition of this NFA is $(\Sigma, Q, q_0, \delta, F)$ where
\begin{tightlist}
\li $\Sigma = \{a,b\}$
\li $Q = \{q_0\}$
\li $\delta$ is the function
\[
\delta : Q \times \Sigma_\epsilon \rightarrow P(Q)
\]
given by
\begin{align*}
  \delta(q_0, \epsilon) &= \{\} \\
  \delta(q_0, a) &= \{\} \\
  \delta(q_0, b) &= \{\} 
\end{align*}
\end{tightlist}


\newpage

Solution to Exercise \ref{ex:nfa1}\labeltext{}{sol:nfa1}.

\tinysidebar{\debug{exercises/{nfa1/answer.tex}}}

    Solution not provided.
    

\newpage

Solution to Exercise \ref{ex:nfa2}\labeltext{}{sol:nfa2}.

\tinysidebar{\debug{exercises/{nfa2/answer.tex}}}

    Solution not provided.
    

\newpage

Solution to Exercise \ref{ex:nfa3}\labeltext{}{sol:nfa3}.

\tinysidebar{\debug{exercises/{nfa3/answer.tex}}}

    Solution not provided.
    

\newpage

Solution to Exercise \ref{ex:nfa4}\labeltext{}{sol:nfa4}.

\tinysidebar{\debug{exercises/{nfa4/answer.tex}}}

    Solution not provided.
    

\newpage

Solution to Exercise \ref{ex:nfa5}\labeltext{}{sol:nfa5}.

\tinysidebar{\debug{exercises/{nfa5/answer.tex}}}

    Solution not provided.
    

\newpage

Solution to Exercise \ref{ex:dfa-as-powerful-as-nfa0}\labeltext{}{sol:dfa-as-powerful-as-nfa0}.

\tinysidebar{\debug{exercises/{dfa-as-powerful-as-nfa0/answer.tex}}}
Here's the solution.
Let $\delta$ denote the transition function of $N$.
Note that 
\begin{align*}
  \delta(q_0, \epsilon) = \{\} \\
  \delta(q_0, a) = \{\} \\
  \delta(q_0, b) = \{\} 
\end{align*}
First of all the states are labeled as all the subsets of $\{q_0\}$.


\begin{center}
\begin{tikzpicture}[>=triangle 60,shorten >=0.5pt,node distance=2cm,auto,initial text=, double distance=2pt]
\node[state] (A) at (  0,  0) {$\{q_0\}$};
\node[state] (B) at (  3,  0) {$\{\}$};

\path[->]

;
\end{tikzpicture}
\end{center}
    


The start state is the $\epsilon$-closure of $\{q_0\}$.
However in $N$, there are no $\epsilon$--transitions out of 
$q_0$.
So the $\epsilon$-closure of $\{q_0\}$ is in fact $\{q_0\}$, i.e.
$\overline{\{q_0\}} = \{q_0\}$
The $\DFA$ is now this:


\begin{longtable}{|r||r|r|r|r|r|}
\hline 
         & $w_1$ & $w_2$ & $w_3$ & $w_4$ & $\ldots$ \\ \hline \hline 
$M_1$    &       &       &       &       &          \\ \hline 
$M_2$    &       &       &       &       &          \\ \hline 
$M_3$    &       &       &       &       &          \\ \hline 
$M_4$    &       &       &       &       &          \\ \hline 
$\ldots$ &       &       &       &       &          \\ \hline 
\end{longtable}
        


Now I will compute the $a$--transition of the state $\{q_0\}$.
Let $\delta^\DFA$ denote the transition function of the $\DFA$
that we're building.
Then
\begin{align*}
\delta( \{q_0, a\} ) 
&= \overline{ \bigcup_{q \in \{q_0\}} \delta(q, a)} \\
&= \overline{ \delta(q_0, a) } \\
&= \overline{ \emptyset } \\
&= \emptyset
\end{align*}
The (incomplete) $\DFA$ now looks like this:


\begin{longtable}{|r||r|r|r|r|r|}
\hline 
         & $w_1$ & $w_2$ & $w_3$ & $w_4$ & $\ldots$ \\ \hline \hline 
$M_1$    & 0     & 0     & 1     & 0     & ...      \\ \hline 
$M_2$    & 1     & 0     & 1     & 1     & ...      \\ \hline 
$M_3$    & 0     & 1     & 1     & 1     & ...      \\ \hline 
$M_4$    & 1     & 0     & 1     & 1     & ...      \\ \hline 
$\ldots$ &       &       &       &       &          \\ \hline 
\end{longtable}
        


Using the same reasoning we have

\begin{center}
\begin{tikzpicture}

\fill[blue!10] (0.0, 0.0) circle (0.35);
\node [line width=0.03cm,black,minimum size=0.6699999999999999cm,draw,circle] at (0.0,0.0)(10){};\draw (0.0, 0.0) node[color=black] {\texttt{10}};
\fill[blue!10] (-1.9, -1.0) circle (0.35);
\node [line width=0.03cm,black,minimum size=0.6699999999999999cm,draw,circle] at (-1.9,-1.0)(5){};\draw (-1.9, -1.0) node[color=black] {\texttt{5}};
\fill[blue!10] (1.9, -1.0) circle (0.35);
\node [line width=0.03cm,black,minimum size=0.6699999999999999cm,draw,circle] at (1.9,-1.0)(8){};\draw (1.9, -1.0) node[color=black] {\texttt{8}};
\fill[blue!10] (-2.85, -2.0) circle (0.35);
\node [line width=0.03cm,black,minimum size=0.6699999999999999cm,draw,circle] at (-2.85,-2.0)(2){};\draw (-2.85, -2.0) node[color=black] {\texttt{2}};
\fill[blue!10] (-0.95, -2.0) circle (0.35);
\node [line width=0.03cm,black,minimum size=0.6699999999999999cm,draw,circle] at (-0.95,-2.0)(1){};\draw (-0.95, -2.0) node[color=black] {\texttt{1}};
\fill[blue!10] (0.95, -2.0) circle (0.35);
\node [line width=0.03cm,black,minimum size=0.6699999999999999cm,draw,circle] at (0.95,-2.0)(0){};\draw (0.95, -2.0) node[color=black] {\texttt{0}};
\fill[blue!10] (2.85, -2.0) circle (0.35);
\node [line width=0.03cm,black,minimum size=0.6699999999999999cm,draw,circle] at (2.85,-2.0)(7){};\draw (2.85, -2.0) node[color=black] {\texttt{7}};
\fill[blue!10] (-3.33, -3.0) circle (0.35);
\node [line width=0.03cm,black,minimum size=0.6699999999999999cm,draw,circle] at (-3.33,-3.0)(9){};\draw (-3.33, -3.0) node[color=black] {\texttt{9}};\draw[line width=0.03cm,black] (10) to  (5);
\draw[line width=0.03cm,black] (10) to  (8);
\draw[line width=0.03cm,black] (5) to  (2);
\draw[line width=0.03cm,black] (5) to  (1);
\draw[line width=0.03cm,black] (8) to  (0);
\draw[line width=0.03cm,black] (8) to  (7);
\draw[line width=0.03cm,black] (2) to  (9);
\end{tikzpicture}

\end{center}



It's easy to see that in the DFA, the $a$--
and $b$--transitions from the state $\{\}$ goes back to itself.
Therefore the completed DFA is this:


\begin{center}
\begin{tikzpicture}[>=triangle 60,shorten >=0.5pt,node distance=2cm,auto,initial text=, double distance=2pt]
\node[state,initial] (A) at (  0,  0) {$\{q_0\}$};
\node[state] (B) at (  3,  0) {$\{\}$};

\path[->]
(A) edge [bend left=0,pos=0.5,above] node {$a,b$} (B)
(B) edge [loop above] node {$a,b$} ()

;
\end{tikzpicture}
\end{center}
    



\newpage

Solution to Exercise \ref{ex:dfa-as-powerful-as-nfa1}\labeltext{}{sol:dfa-as-powerful-as-nfa1}.

\tinysidebar{\debug{exercises/{dfa-as-powerful-as-nfa1/answer.tex}}}

    Solution not provided.
    

\newpage

Solution to Exercise \ref{ex:dfa-as-powerful-as-nfa2}\labeltext{}{sol:dfa-as-powerful-as-nfa2}.

\tinysidebar{\debug{exercises/{dfa-as-powerful-as-nfa2/answer.tex}}}

    Solution not provided.
    

\newpage

Solution to Exercise \ref{ex:dfa-as-powerful-as-nfa3}\labeltext{}{sol:dfa-as-powerful-as-nfa3}.

\tinysidebar{\debug{exercises/{dfa-as-powerful-as-nfa3/answer.tex}}}

    Solution not provided.
    

\newpage

Solution to Exercise \ref{ex:dfa-as-powerful-as-nfa4}\labeltext{}{sol:dfa-as-powerful-as-nfa4}.

\tinysidebar{\debug{exercises/{dfa-as-powerful-as-nfa4/answer.tex}}}

    Solution not provided.
    

\newpage

Solution to Exercise \ref{ex:closure0}\labeltext{}{sol:closure0}.

\tinysidebar{\debug{exercises/{closure0/answer.tex}}}

    Solution not provided.
    

\newpage

Solution to Exercise \ref{ex:closure1}\labeltext{}{sol:closure1}.

\tinysidebar{\debug{exercises/{closure1/answer.tex}}}

    Solution not provided.
    

\newpage

Solution to Exercise \ref{ex:closure2}\labeltext{}{sol:closure2}.

\tinysidebar{\debug{exercises/{closure2/answer.tex}}}

    Solution not provided.
    

\newpage

Solution to Exercise \ref{ex:closure3}\labeltext{}{sol:closure3}.

\tinysidebar{\debug{exercises/{closure3/answer.tex}}}

    Solution not provided.
    

\newpage

Solution to Exercise \ref{ex:closure4}\labeltext{}{sol:closure4}.

\tinysidebar{\debug{exercises/{closure4/answer.tex}}}

    Solution not provided.
    

\newpage

Solution to Exercise \ref{ex:closure5}\labeltext{}{sol:closure5}.

\tinysidebar{\debug{exercises/{closure5/answer.tex}}}

    Solution not provided.
    

\newpage

Solution to Exercise \ref{ex:closure6}\labeltext{}{sol:closure6}.

\tinysidebar{\debug{exercises/{closure6/answer.tex}}}

    Solution not provided.
    

\newpage

Solution to Exercise \ref{ex:closure7}\labeltext{}{sol:closure7}.

\tinysidebar{\debug{exercises/{closure7/answer.tex}}}

    Solution not provided.
    

\newpage

Solution to Exercise \ref{ex:closure8}\labeltext{}{sol:closure8}.

\tinysidebar{\debug{exercises/{closure8/answer.tex}}}

    Solution not provided.
    

\newpage

Solution to Exercise \ref{ex:closure9}\labeltext{}{sol:closure9}.

\tinysidebar{\debug{exercises/{closure9/answer.tex}}}

    Solution not provided.
    

\newpage

Solution to Exercise \ref{ex:closure10}\labeltext{}{sol:closure10}.

\tinysidebar{\debug{exercises/{closure10/answer.tex}}}

    Solution not provided.
    

\newpage

Solution to Exercise \ref{ex:closure11}\labeltext{}{sol:closure11}.

\tinysidebar{\debug{exercises/{closure11/answer.tex}}}

    Solution not provided.
    

\newpage

Solution to Exercise \ref{ex:closure12}\labeltext{}{sol:closure12}.

\tinysidebar{\debug{exercises/{closure12/answer.tex}}}

    Solution not provided.
    
 % input solutions.tex
