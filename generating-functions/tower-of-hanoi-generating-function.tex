Can generating functions be used to derive a closed form for $t_n$?
Let's go ahead and try it.
Again we define the generating function for $t_n$:
\[
t(x) = \sum_{n=0}^\infty t_n x^n
\]
Hence
\begin{align*}
t(x) 
&= t_0 + \sum_{n=1}^\infty t_n x^n \\
&= 0 + \sum_{n=1}^\infty (2t_{n-1} + 1) x^n \\
&= 2 \sum_{n=1}^\infty t_{n-1} x^n + \sum_{n=1}^\infty x^n \\
&= 2 x \sum_{n=1}^\infty t_{n-1} x^{n-1} + \sum_{n=1}^\infty x^n \\
&= 2 x \sum_{n=0}^\infty t_{n} x^{n} + \sum_{n=1}^\infty x^n 
\end{align*}
Now note that we see that $t(x)$ pops up and ... a geoemtric
series appears as well (with the $x^0$ missing)!
But that's no  big deal.
Remember that ultimately we want to write $t(x)$ as a rational
function.
Geometric series are rational functions too.
Having one around doesn't make the solution harder.
So we just plow ahead.
\begin{align*}
t(x) 
&= 2 x \sum_{n=0}^\infty t_{n} x^{n} + \sum_{n=0}^\infty x^n - 1\\
&= 2x t(x) + \frac{1}{1-x} - 1\\
\therefore
t(x) - 2x t(x)  
&= \frac{1}{1-x} - 1 = \frac{1-1+x}{1-x}\\
\therefore
(1 - 2x)t(x)  
&= \frac{x}{1-x}\\
\therefore
t(x)  
&= \frac{x}{(1-x)(1 - 2x)}
\end{align*}



\newpage
\begin{ex}
Check that 
\[
\frac{1}{(1-x)(1 - 2x)}
= \frac{-1}{1-x} + \frac{2}{1-2x}
\]
\qed
\end{ex}

Hence
\begin{align*}
t(x) 
&= x \frac{1}{(1-x)(1-2x)} \\
&= x \biggl( \frac{-1}{1-x} + \frac{2}{1-2x} \biggr) \\
&= x \biggl( -\sum_{n=0}^\infty x^n + 2 \sum_{n=0}^\infty 2^n x^n \biggr) \\
&= \sum_{n=0}^\infty \bigl ( -1 + 2^{n+1} \bigr )x^{n+1} \\
&= \sum_{n=1}^\infty \bigl ( -1 + 2^n \bigr )x^{n}
\end{align*}
Therefore we have
\[
t_n = 
\begin{cases}
0       &\text{ if } n = 0 \\
2^n - 1 &\text{ if } n > 0
\end{cases}
\]
Note that $2^0 - 1 = 0$. 
Therefore
\[
t_n = 2^n - 1
\]
for $n \geq 0$.
So it's clear that the method of generating function can solve
recurrences of degree one as well.
Note that the \lq\lq + 1'' in the recurrence relation, the nonhomogeneous term,
is not a problem either.
