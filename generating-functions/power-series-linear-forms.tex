\sectionthree{Linear forms}
\begin{python0}
from solutions import *; clear() 
\end{python0}

Now we use power series to solve some combinatorial / discrete problems.
Before we begin, let's step back and look at the big picture.

In many combinatorial problems (including problems from CS), one is usually
interested in understanding not just a number such as
\lq\lq how many steps do you need to sort these 1000 numbers if we use 
algorithm $X$?''
Instead we're usually interested in a family of related problems:
\lq\lq how many steps do you need to sort these $n$ numbers if we use 
algorithm $X$?''
If $t_n$ is the number of steps, then in fact we're looking at not just
one number but an infinite sequence of numbers:
\[
t_0, t_1, t_2, t_3, \ldots
\]

So far we've been studying such numbers individually, or
in the case of mathematical induction, we're looking at $t_{n+1}$
from the point of view of $t_n$.
Note also that although we've been proving theorems such as
\[
1 + 2 + \cdots + n = \frac{n(n+1)}{2}
\]
using mathematical induction, you should realize by now that
mathematical induction requires us to start with a 
hypothesis:
You have to do some experiments to come up with a plausible
formula or fact before you can apply induction to prove
the correctness of your conjecture.

The technique and power of the power series is that instead of
looking at numbers one at a time, we will build a single object to study.
So instead of studying a sequence of infinite things:
\[
t_0, t_1, t_2, t_3, \ldots
\]
one at a time, we will combine them together to form a power series:
\[
t(x) = \sum_{n=0}^\infty t_n x^n
\]

We say that 
\[
t(x) = \sum_{n=0}^\infty t_n x^n
\]
is the \defone{ordinary generating function} of the sequence
\[
t_0, t_1, t_2, t_3, \ldots
\]

Of course since I'm calling this the \textit{ordinary} generating
function, you would expect other types of generating functions.
That is in fact the case although the ordinary is the most common.
Our generating function is built from $t_0, t_1, t_2, \ldots$
together with $x^0, x^1, x^2, \ldots$.
In general it's possible to talk about generating functions 
with a collection of kernel functions $K_0(x), K_1(x), K_2(x), \ldots$:
\[
\sum_{n=0}^\infty t_n K_n(x)
\]
Different kernel functions solve different problems.
Anyway let's get back to the ordinary case.

One extremely important thing about polynomial multiplication
is that it is a combinatorial operation.
For instance look at this:
\[
(1 + x + x^2 + x^3 + x^4)
(1 + x + x^2 + x^3)
\]
Of course you know how to multiply two polynomials.
But notice that each term in the product looks like this
\[
x^i x^j
\]
where $x^i$ is from $1 + x + x^2 + x^3 + x^4$ and 
$x^i$ is from $1 + x + x^2 + x^3$.
In particular look at the term with $x$-power of $x^3$.
The following lists all the pairs (underlined) that will contribute
towards the $x^3$ term in the final product:
\begin{align*}
&(\underline{1} + x + x^2 + x^3 + x^4)(1 + x + x^2 + \underline{x^3}) \\
&(1 + \underline{x} + x^2 + x^3 + x^4)(1 + x + \underline{x^2} + x^3) \\
&(1 + x + \underline{x^2} + x^3 + x^4)(1 + \underline{x} + x^2 + x^3) \\
&(1 + x + x^2 + \underline{x^3} + x^4)(\underline{1} + x + x^2 + x^3) \\
\end{align*}
i.e. the $x^3$ term is
\[
x^0x^3 + x^1x^2 + x^2x^1 + x^3x^0 = 4x^3
\]
The multiplication process is combinatorial in the sense that
one has to \textit{choose} terms.
In this case,
I have to choose a term from the first polynomial and
a term from the second.

Now what if I multiply the following:
\[
(1 + x + x^3 + x^4)
(1 + x + x^2 + x^3)
\]
Note that $x^2$ is missing from the first polynomial.
What is the term with $x$-power of $x^3$?
\begin{align*}
&(\underline{1} + x + x^3 + x^4)(1 + x + x^2 + \underline{x^3}) \\
&(1 + \underline{x} + x^3 + x^4)(1 + x + \underline{x^2} + x^3) \\
&(1 + x + \underline{x^3} + x^4)(\underline{1} + x + x^2 + x^3) \\
\end{align*}
In this case I have
\[
x^0x^3 + x^1x^2 + x^3x^0 = 3x^3
\]

Now look at the above simple computations very carefully.
In the first example, do you see that
\[
\text{ The 4 in 4$x^3$ comes from } 3 = 0+3, 1+2, 2+1, 3+0
\]
and the second example
\[
\text{ The 3 in $3x^3$ comes from } 3 = 0+3, 1+2, 3+0
\]
If you don't see that, you should go over the examples above again.

If I want to count the number of ways to select $a$ from $\{0, 1, 2, 3, 4\}$
and $b$ from $\{0, 1, 2, 3\}$ such that
\[
a + b = 3
\]
then all I need is to compute the coefficient of $x^3$ in 
\[
(x^0 + x^1 + x^2 + x^3 + x^4)
(x^0 + x^1 + x^2 + x^3)
\]
Of course the first polynomial is created from $\{0,1,2,3,4\}$:
\[
(x^0 + x^1 + x^2 + x^3 + x^4) = \sum_{n \in \{0, 1, 2, 3, 4\}} x^n
\]
Likewise the second polynomial is created from $\{0, 1, 2, 3\}$
resulting in 
\[
(x^0 + x^1 + x^2 + x^3) = \sum_{n \in \{0, 1, 2, 3\}} x^n
\]

On the other hand if the $a$ must be chosen from $\{1, 3, 5\}$ 
and $b$ is chosen from $\{0,1,2,3\}$ then the 
number of ways to select $a$ and $b$ such that $a + b = 3$
is the coefficient of 
\[
(x^1 + x^3 + x^5)(x^0 + x^1 + x^2 + x^3)
\]
See that?


%-*-latex-*-

\begin{ex} 
  \label{ex:prob-00}
  \tinysidebar{\debug{exercises/{disc-prob-28/question.tex}}}

  \solutionlink{sol:prob-00}
  \qed
\end{ex} 
\begin{python0}
from solutions import *
add(label="ex:prob-00",
    srcfilename='exercises/discrete-probability/prob-00/answer.tex') 
\end{python0}


%-*-latex-*-

\begin{ex} 
  \label{ex:prob-00}
  \tinysidebar{\debug{exercises/{disc-prob-28/question.tex}}}

  \solutionlink{sol:prob-00}
  \qed
\end{ex} 
\begin{python0}
from solutions import *
add(label="ex:prob-00",
    srcfilename='exercises/discrete-probability/prob-00/answer.tex') 
\end{python0}


%-*-latex-*-

\begin{ex} 
  \label{ex:prob-00}
  \tinysidebar{\debug{exercises/{disc-prob-28/question.tex}}}

  \solutionlink{sol:prob-00}
  \qed
\end{ex} 
\begin{python0}
from solutions import *
add(label="ex:prob-00",
    srcfilename='exercises/discrete-probability/prob-00/answer.tex') 
\end{python0}


%-*-latex-*-

\begin{ex} 
  \label{ex:prob-00}
  \tinysidebar{\debug{exercises/{disc-prob-28/question.tex}}}

  \solutionlink{sol:prob-00}
  \qed
\end{ex} 
\begin{python0}
from solutions import *
add(label="ex:prob-00",
    srcfilename='exercises/discrete-probability/prob-00/answer.tex') 
\end{python0}


%-*-latex-*-

\begin{ex} 
  \label{ex:prob-00}
  \tinysidebar{\debug{exercises/{disc-prob-28/question.tex}}}

  \solutionlink{sol:prob-00}
  \qed
\end{ex} 
\begin{python0}
from solutions import *
add(label="ex:prob-00",
    srcfilename='exercises/discrete-probability/prob-00/answer.tex') 
\end{python0}


%-*-latex-*-

\begin{ex} 
  \label{ex:prob-00}
  \tinysidebar{\debug{exercises/{disc-prob-28/question.tex}}}

  \solutionlink{sol:prob-00}
  \qed
\end{ex} 
\begin{python0}
from solutions import *
add(label="ex:prob-00",
    srcfilename='exercises/discrete-probability/prob-00/answer.tex') 
\end{python0}


%-*-latex-*-

\begin{ex} 
  \label{ex:prob-00}
  \tinysidebar{\debug{exercises/{disc-prob-28/question.tex}}}

  \solutionlink{sol:prob-00}
  \qed
\end{ex} 
\begin{python0}
from solutions import *
add(label="ex:prob-00",
    srcfilename='exercises/discrete-probability/prob-00/answer.tex') 
\end{python0}


%-*-latex-*-

\begin{ex} 
  \label{ex:prob-00}
  \tinysidebar{\debug{exercises/{disc-prob-28/question.tex}}}

  \solutionlink{sol:prob-00}
  \qed
\end{ex} 
\begin{python0}
from solutions import *
add(label="ex:prob-00",
    srcfilename='exercises/discrete-probability/prob-00/answer.tex') 
\end{python0}

