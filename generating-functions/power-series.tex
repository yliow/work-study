\sectionthree{Power series}
\begin{python0}
from solutions import *; clear() 
\end{python0}

A \defone{power series} is an expression of the form
\[
\sum_{n=0}^\infty a_n x^n
\]
where $a_n$ is an expression in $n$. 
$\sum_{n=0}^\infty a_n x^n$ is just a short-hand notation:
\[
\sum_{n=0}^\infty a_n x^n
= a_0 x^0 + a_1 x^1 + a_2 x^2 + \cdots
\]

Here's an example:
\[
\sum_{n=0}^\infty \frac{1}{2^n} x^n
\]
In this example, 
\[
a_n = \frac{1}{2^n}
\]

If you like, you can think of a power series as a polynomial that 
\lq\lq goes on forever'',
i.e. a polynomial with infinite degree.

A series need not start with $n = 0$.
For instance this is a perfectly good series:
\[
\sum_{n=5}^\infty \frac{1}{2^n} x^n
\]
And it doesn't have to go on forever.
This is OK too:
\[
\sum_{n=5}^{1000000} \frac{1}{2^n} x^n
\]
Of course this means
\[
\sum_{n=5}^{1000000} \frac{1}{2^n} x^n
= 
\frac{1}{2^5} x^5 +
\frac{1}{2^6} x^6 +
\frac{1}{2^7} x^7 +
\cdots +
\frac{1}{2^{999999}} x^{999999} +
\frac{1}{2^{1000000}} x^{1000000}
\]
Of course this \textit{is} a polynomial (of degree 1000000).


More generally, power series are examples of series of functions:
\[
\sum_{n = 0}^\infty A_n(x)
\]
where each $A_n(x)$ is a function of $x$.
One can also talk about $\sum_{n=0}^\infty c_n$ where
each $c_n$ are constants
and of course when you substitute a value for $x$ in
\[
\sum_{n = 0}^\infty A_n(x)
\]
say $x = 3.1415$, you \textit{do} get a series of constants:
\[
\sum_{n = 0}^\infty A_n(3.1415)
\]
At this point, of course you have to think about the serious
consequence of adding infinitely many numbers.
For instance when you substitute $x = 1$ into
\[
\sum_{n=0}^\infty x^n
\]
you get
\[
\sum_{n=0}^\infty 1^n = \sum_{n=0}^\infty 1 = 1 + 1 + 1 + \cdots
\]
which is infinite!
There are however cases where the sum is not infinite:
When $x = 1/2$,
\[
\sum_{n=0}^\infty (1/2)^n
\]
\textit{is} finite.
If you run this program
\begin{console}
s = 0.0
term = 1.0
while 1:
    s = s + term
    print(s)
    term = term / 2.0 
\end{console}
(rewritten in your favorite programming lang) 
you will see that \verb!s! seems to stop growing after some time.
In fact from your Pre-calc you already know that
\[
\sum_{n=0}^\infty x^n = \frac{1}{1 - x}
\]
This is the famous \defone{geometric series}.
When you substitute $x = 1/2$ into the above, you get
\[
\sum_{n=0}^\infty (1/2)^n = \frac{1}{1 - 1/2} = 2
\]
I already mentioned that if you substitute $x = 1$,
the above series explodes in your face.
In fact for the geometric series identity to be correct,
you can only substitute $x$ when $|x| < 1$.
Only then will the series \textit{converge}.
For us however, we frequently
do not substitute values for 
$x$ in a power series. You can think of the powers of $x$
as place holders for values (i.e. coefficients).


Here are some basic facts about series.
The following facts are easy since they mirror the same facts for
polynomials:

\begin{prop}
Let $\sum_{n \in I} A_n$, $\sum_{n \in I} B_n$ be two series;
here $\sum_{n \in I}$ means \lq\lq sum over all $i$ in index set $I$''
and the $A_i$ and $B_i$ can be functions of $x$.
$c$ can be a constant, a polynomial, or a power series.
\begin{enumerate}
\item[\textnormal{(a)}] $c \sum_{n \in I} A_n$ = $\sum_{n \in I} cA_n$
\item[\textnormal{(b)}] $\sum_{n \in I} A_n + \sum_{n \in I} B_n$ = 
$\sum_{n \in I} (A_n + B_n)$
\end{enumerate}
\end{prop}

Note that the above applies only when the index sets are all the same.
In terms of power series we have the following:

\begin{prop}
Let $\sum_{n \in I} a_n x^n$, $\sum_{n \in I} b_n x^n$ be two series.
\begin{enumerate}
\item[\textnormal{(a)}] $c \sum_{n \in I} a_n x^n$ = $\sum_{n \in I} ca_n x^n$
\item[\textnormal{(b)}] $\sum_{n \in I} a_nx^n+ \sum_{n \in I} b_nx^n$ = 
$\sum_{n \in I} (a_n + b_n) x^n$
\end{enumerate}
\end{prop}

For instance
\[
3 \sum_{n = 0}^\infty \frac{1}{2^n}x^n
= \sum_{n = 0}^\infty 3 \cdot \frac{1}{2^n}x^n
= \sum_{n = 0}^\infty \frac{3}{2^n}x^n
\]
and 
\[
(3 + 2x + x^2) \sum_{n = 0}^\infty \frac{1}{2^n}x^n
= \sum_{n = 0}^\infty (3 + 2x + x^2) \frac{1}{2^n}x^n
\]

Note that when working with power series, we are usually interested
in extracting the coefficient of $x^n$ from a power series.
In the case of 
\[
3 \sum_{n = 0}^\infty \frac{1}{2^n}x^n
= \sum_{n = 0}^\infty \frac{3}{2^n}x^n
\]
we see that the coefficient of $x^n$ is
\[
\frac{3}{2^n}
\]
However for the case of 
\[
(3 + 2x + x^2) \sum_{n = 0}^\infty \frac{1}{2^n}x^n
= \sum_{n = 0}^\infty (3 + 2x + x^2) \frac{1}{2^n}x^n
\]
the coefficient of $x^n$ is not obvious.
In fact we have to do this:
\begin{align*}
(3 + 2x + x^2) \sum_{n = 0}^\infty \frac{1}{2^n}x^n
&= 3 \sum_{n = 0}^\infty \frac{1}{2^n}x^n
+ 2x \sum_{n = 0}^\infty \frac{1}{2^n}x^n
+ x^2 \sum_{n = 0}^\infty \frac{1}{2^n}x^n \\
&= \sum_{n = 0}^\infty 3 \frac{1}{2^n}x^n
+ \sum_{n = 0}^\infty 2x \frac{1}{2^n}x^n
+ \sum_{n = 0}^\infty x^2 \frac{1}{2^n}x^n \\
&= \sum_{n = 0}^\infty \frac{3}{2^n}x^n
+ \sum_{n = 0}^\infty \frac{2}{2^n}x^{n+1}
+ \sum_{n = 0}^\infty \frac{1}{2^n}x^{n+2} \\
\end{align*}

If you stare at the above carefully you should be able to
see the coefficient of $x^n$.
If you don't see it you can write down a few terms:

\begin{align*}
(3 + 2x + x^2) \sum_{n = 0}^\infty \frac{1}{2^n}x^n
&= \sum_{n = 0}^\infty \frac{3}{2^n}x^n
+ \sum_{n = 0}^\infty \frac{2}{2^n}x^{n+1}
+ \sum_{n = 0}^\infty \frac{1}{2^n}x^{n+2} \\
&=
\biggl( \frac{3}{2^0}x^0 + \frac{3}{2^1}x^1 + \frac{3}{2^2}x^2 + \cdots \biggr) \\ 
&\hskip 1.4cm + \biggl( \frac{2}{2^0}x^1 + \frac{2}{2^1}x^2 + \frac{2}{2^2}x^3 + \cdots \biggr) \\
&\hskip 2.7cm + \biggl( \frac{1}{2^0}x^2 + \frac{1}{2^1}x^3 + \frac{1}{2^2}x^4 + \cdots \biggr)
\end{align*}

Do you see that when $n \geq 2$, the coefficient is
\[
\frac{3}{2^n} + \frac{2}{2^{n-1}} + \frac{1}{2^{n-2}}
\]
For the remaining coefficients (i.e. for $x^0$ and $x^1$) we have to 
handle them separately. They are respectively
\[
\frac{3}{2^0}, \,\,\,\,\, \frac{3}{2^1} + \frac{2}{2^0}
\] 
In order not to rely on all the above term-by-term expansion, we
can formalize the computations as follows.

Let's go back to our series:
\[
(3 + 2x + x^2) \sum_{n = 0}^\infty \frac{1}{2^n}x^n
= \sum_{n = 0}^\infty \frac{3}{2^n}x^n
+ \sum_{n = 0}^\infty \frac{2}{2^n}x^{n+1}
+ \sum_{n = 0}^\infty \frac{1}{2^n}x^{n+2}
\]
For the second term of the sum of three series:
\[
\sum_{n = 0}^\infty \frac{2}{2^n}x^{n+1}
\]
we really want $x^n$ and not $x^{n+1}$.
Again by expanding out a few terms you can see what it should be.
Another way to do this is to do a substitution to replace the
index variable $n$.
Let
\[
p = n + 1
\]
Note that the sum is from $n = 0$ to $\infty$.
In terms of $p$, this means the sum is from $p = 0 + 1$ to $\infty$.
Therefore we have
\[
\sum_{n = 0}^\infty \frac{2}{2^n}x^{n+1}
=
\sum_{p = 1}^\infty \frac{2}{2^{p-1}}x^{p}
\]
Of course $p$ is just a \lq\lq dummy'' variable; you can replace it with $n$:
\[
\sum_{n = 0}^\infty \frac{2}{2^n}x^{n+1}
=
\sum_{p = 1}^\infty \frac{2}{2^{p-1}}x^{p}
= 
\sum_{n = 1}^\infty \frac{2}{2^{n-1}}x^{n}
\]

Now for the third term:
\[
\sum_{n = 0}^\infty \frac{1}{2^n}x^{n+2}
\]
Let
\[
q = n + 2
\]
Then
\[
\sum_{n = 0}^\infty \frac{1}{2^n}x^{n+2}
=
\sum_{q = 2}^\infty \frac{1}{2^{q-2}}x^{q}
\]

Therefore
\begin{align*}
(3 + 2x + x^2) \sum_{n = 0}^\infty \frac{1}{2^n}x^n
&= \sum_{n = 0}^\infty \frac{3}{2^n}x^n
+ \sum_{n = 0}^\infty \frac{2}{2^n}x^{n+1}
+ \sum_{n = 0}^\infty \frac{1}{2^n}x^{n+2} \\
&=
\sum_{n = 0}^\infty \frac{3}{2^n}x^n
+ \sum_{n = 1}^\infty \frac{2}{2^{n-1}}x^{n}
+ \sum_{n = 2}^\infty \frac{1}{2^{n-2}}x^{n} \\
\end{align*}

You see that all the power series have the same power of $x$ for each of their
terms.
We are now ready to combine the three power series into one.
But note that their index values run over different values.
Note that if $I$ and $J$ are two disjoint index sets, 
\[
\sum_{i \in I \cup J} A_i
=
\sum_{i \in I} A_i
+
\sum_{i \in J} A_i
\]
This implies for instance that
\[
\sum_{n = 0}^\infty A_i = 
A_0 + \sum_{n = 1}^\infty A_i 
\]
and 
\[
\sum_{n = 0}^\infty A_i = 
A_0 + A_1 + \sum_{n = 2}^\infty A_i 
\]

In other words you can split up your series
into parts.
This is useful for instance when you have the 
sum of two series and their indices run over different values.
You simply split out the values not common to both.
For instance 
\begin{align*}
\sum_{n = 1}^\infty A_n + \sum_{n = 3}^\infty B_n
= A_1 + A_2 + \sum_{n = 3}^\infty A_n + \sum_{n = 3}^\infty B_n \\
= A_1 + A_2 + \sum_{n = 3}^\infty (A_n + B_n)
\end{align*}

Now let's go back to our series:
\begin{align*}
(3 + 2x + x^2) \sum_{n = 0}^\infty \frac{1}{2^n}x^n
&= \sum_{n = 0}^\infty \frac{3}{2^n}x^n
+ \sum_{n = 0}^\infty \frac{2}{2^n}x^{n+1}
+ \sum_{n = 0}^\infty \frac{1}{2^n}x^{n+2} \\
&=
\sum_{n = 0}^\infty \frac{3}{2^n}x^n
+ \sum_{n = 1}^\infty \frac{2}{2^{n-1}}x^{n}
+ \sum_{n = 2}^\infty \frac{1}{2^{n-2}}x^{n}
\end{align*}
The common index values for all three series are $n \geq 2$. 
Therefore
\begin{align*}
(3 + 2x + x^2) \sum_{n = 0}^\infty \frac{1}{2^n}x^n
&= \sum_{n = 0}^\infty \frac{3}{2^n}x^n
+ \sum_{n = 0}^\infty \frac{2}{2^n}x^{n+1}
+ \sum_{n = 0}^\infty \frac{1}{2^n}x^{n+2} \\
&=
\sum_{n = 0}^\infty \frac{3}{2^n}x^n
+ \sum_{n = 1}^\infty \frac{2}{2^{n-1}}x^{n}
+ \sum_{n = 2}^\infty \frac{1}{2^{n-2}}x^{n} \\
&=
\frac{3}{2^0}x^0 +
\frac{3}{2^1}x^1 +
\sum_{n = 2}^\infty \frac{3}{2^n}x^n
+ \frac{2}{2^{1-1}}x^{1} + \sum_{n = 2}^\infty \frac{2}{2^{n-1}}x^{n}
+ \sum_{n = 2}^\infty \frac{1}{2^{n-2}}x^{n}
\end{align*}

Note that from the first series, I split off two terms ($n=0, 1$)
and from the second I split off one term ($n = 1$).
Continuing the computation we have
\begin{align*}
(3 + 2x + x^2) \sum_{n = 0}^\infty \frac{1}{2^n}x^n
&= \sum_{n = 0}^\infty \frac{3}{2^n}x^n
+ \sum_{n = 0}^\infty \frac{2}{2^n}x^{n+1}
+ \sum_{n = 0}^\infty \frac{1}{2^n}x^{n+2} \\
&=
\frac{3}{2^0}x^0 +
\frac{3}{2^1}x^1 +
\sum_{n = 2}^\infty \frac{3}{2^n}x^n \\
& \hskip 1cm + \frac{2}{2^{1-1}}x^{1} + \sum_{n = 2}^\infty \frac{2}{2^{n-1}}x^{n} \\
& \hskip 2.7cm + \sum_{n = 2}^\infty \frac{1}{2^{n-2}}x^{n} \\
&=
\frac{3}{2^0} 
+ \biggl( \frac{3}{2^1} +  \frac{2}{2^{0}} \biggr) x
+ 
\sum_{n = 2}^\infty  
\biggl( 
\frac{3}{2^n} +
\frac{2}{2^{n-1}} +
\frac{1}{2^{n-2}}
\biggr) 
x^n
\end{align*}
And we're done! You can read off the coefficients very quickly from the above:
the coefficient of $x^n$ is
\[
\begin{cases}
3  & \text{if $n = 0$} \\
\frac{7}{2} & \text{if $n = 1$} \\
\displaystyle \frac{3}{2^n} +
\displaystyle \frac{2}{2^{n-1}} +
\displaystyle \frac{1}{2^{n-2}} & \text{if $n \geq 2$} \\
\end{cases}
\]

We see that for a constant $c$ (and $k$ any positive integer):
\[
c \sum_{n = k}^\infty a_n x^n
= \sum_{n = k}^\infty ca_n x^n
\]
which does not change the power of the $x$ in the general term
of the power series.
However 
\[
x \sum_{n = k}^\infty a_n x^n
= \sum_{n = k}^\infty a_n x^{n+1}
\]
In general
\[
x^\ell \sum_{n = k}^\infty a_n x^n
= \sum_{n = k}^\infty a_n x^{n+\ell}
\]
This results in a change of the power of $x$ in the general term of the power
series.
A change of index variable
\[
p = n + \ell
\]
will allow you to change the power of $x$ so that the power is the 
same as the index variable of the summation.

If you list the the coefficient of a power series $\sum_{n=0}^\infty a_nx^n$
as an infinite-dimensional vector:
\[
(a_0, a_1, a_2, a_3, \ldots)
\]
for every $x^n$ and you do likewise for the coefficients for 
$x^2 \sum_{n=0}^\infty a_nx^n$:
\[
(0, 0, a_0, a_1, a_2, a_3, \ldots)
\]
you see that \lq\lq multiplication-by-$x^2$'' acts like a \lq\lq shift-by-2'' 
operator.
In general \lq\lq multiplication-by-$x^\ell$'' acts like a 
\lq\lq shift-by-$\ell$'' operator.
The multiplication-by-$c$ is like a scalar multiplication: the 
coefficients of $c \sum_{n=0}^\infty a_nx^n$ is
\[
(ca_0, ca_1, ca_2, ca_3, \ldots)
\]

\newpage
\begin{ex} 
  \label{ex:some-decision1}
  \tinysidebar{\debug{exercises/{empty0/question.tex}}}
  \solutionlink{sol:some-decision1}
  \qed
\end{ex} 
\begin{python0}
from solutions import *
add(label="ex:some-decision1",
    srcfilename='exercises/some-decision1/answer.tex') 
\end{python0}


\newpage
\begin{ex}
Rewrite 
\[
(1 - 3x^2 + 5x^7) \sum_{n = 1}^\infty \frac{2^n}{3^n}x^n
+ \sum_{n=100}^\infty \frac{1}{5^n}x^n 
\]
as a
power series.
What is the coefficient of $x^n$?
\end{ex}

\newpage
\begin{ex}
Rewrite 
\[
(1 - 3x^{110} + 5x^{200}) \sum_{n = 1}^\infty \frac{2^n}{3^n}x^n
+ \sum_{n=100}^\infty \frac{1}{11^n - 1}x^n
\] as a
power series.
What is the coefficient of $x^n$?
\end{ex}

\newpage
\begin{ex}
Rewrite 
\[
(1 - 3x^{110} + 5x^{200}) \sum_{n = 1}^\infty \frac{2^n}{3^n}x^n
+ \sum_{n=100}^\infty \frac{1}{5^n}x^n
+ \sum_{n=5}^{90} \frac{1}{5^n}x^n 
\] as a
power series.
What is the coefficient of $x^n$?
\end{ex}

\newpage
Now that you have a couple of warmups you can prove
a couple of little theorems of your own.

\begin{ex}
Rewrite 
\[
x^\ell \sum_{n = 0}^\infty a_n x^n
\] 
as a
power series.
What is the coefficient of $x^n$?
\end{ex}

\newpage
\begin{ex}
Rewrite 
\[
cx^\ell \sum_{n = 0}^\infty a_n x^n
\] 
as a
power series.
What is the coefficient of $x^n$?
\end{ex}

\newpage
\begin{ex}
Rewrite 
\[
(c_0 + c_1x + c_2 x^2) \cdot \sum_{n = 0}^\infty a_n x^n
\] 
as a
power series.
What is the coefficient of $x^n$?
\end{ex}

\newpage
\begin{ex}
Rewrite 
\[
(c_0 + c_1x + c_2 x^2 + \cdots + c_d x^d) \cdot \sum_{n = 0}^\infty a_n x^n
\] 
as a
power series.
What is the coefficient of $x^n$?
\end{ex}


\newpage
One last fact about power series:

\[
\sum_{n = 0}^\infty a_nx^n =
\sum_{n = 0}^\infty b_nx^n
\text{ iff }
a_n = b_n \text{ for } n \geq 
\]
