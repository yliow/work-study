%-*-latex-*-
\sectionthree{Balanced expressions}
\begin{python0}
from solutions import *; clear()
\end{python0}

Another use of the stack is to check for balanced expressions.
What do I mean by that?
Suppose you look at this arithmetic expression
\[
\text{
  \texttt{(1 + (2 + 3)) * ((4 - 5 / (7 * 8) + 9) \% 0)}
  }
\]
When you only look at the parentheses you see this:
\[
\text{\texttt{(())((()))}}
\]
When the parentheses come from a well-formed arithmetic expression,
we say that the parentheses are balanced.
This is not balanced:
\[
\text{\texttt{()(())()(()(()))(())((())))((()())}}
\]
See it?

\begin{ex}
Write a program that prompts the user for an arithmetic expression
and checks if the left and right parentheses form a balanced expression.
This should NOT take you more than 5 minutes.
\qed
\end{ex}

Now, the problem can be a little bit more complicated:
instead of just \texttt{(} and \texttt{)},
suppose we allow \texttt{[} and \texttt{]}.
How about throwing in
\texttt{<} and \texttt{>}?
We might as well include
\texttt{\{} and \texttt{\}} too.

The problem of whether the parentheses in an expression is balanced or not
is similar to the problem of checking whether a string is a palindrome.
Why?
Look at this palindrome again
\[
\text{\texttt{madam}}
\]
You are matching the two \texttt{m}'s:
\[
\text{\texttt{\underline{m}ada\underline{m}}}
\]
You then match the \texttt{a}'s:
\[
\text{\texttt{m\underline{a}d\underline{a}m}}
\]
If you look at this:
\[
\text{\texttt{([()])}}
\]
you see quickly that you're also matching two things in a manner
similar to a palindromic check.
There's of course a difference because this is also balanced
but not palindromic:
\[
\text{\texttt{[()](())}}
\]

\begin{ex}\label{ex:balanced-expression-multiple-pairs}
Design an algorithm 
that checks if a string containing the characters
\verb!(,),{,},<,>,[,]! is balanced.
(Hint: Stack. Solution at end of this subsection.)
\qed
\end{ex}


Don't be fooled by the above
\lq\lq puzzles'' ... reversing a string, checking palindromes,
balancing expressions.
Seems like little tricky problems just to tickle the mind.
NO!
It's a lot more than just meets the eye.
The presence of a stack used for a computation
indicates that there's a certain theoretical level of
complexity in the problem.
It's even important and present in nature:
palindromic subsequences in DNA in related to the ability of the
molecules to perform folding. (Google protein folder.)

\begin{ex} *
This one is much harder than the previous.
Write a program that prompts the user for 
a string containing left and right parentheses
and return a balanced expression by making the 
least changes to the given string, either by 
deleting or inserting characters.
\qed
\end{ex}

\newpage
\textsc{Solutions}

Solution to Exercise \ref{ex:balanced-expression-multiple-pairs}.
The idea is to keep the \lq\lq openings"
\verb!(!,
\verb![!,
\verb!<!, or
\verb!{! in a stack when scanning the string.
And when a \lq\lq closing"
\verb!)!,
\verb!]!,
\verb!>!, or
\verb!}!
is seen, check is it matches the top of the stack.
Here's the pseudocode:
\begin{console}
Take a character from the string.
If the character is (, [, <, or {, push it onto the stack.
Otherwise,
    If the character is ):
        if the top of the stack is (, pop it off
        if the top of the stack is not, stop with ERROR
    If the character is ]:
        if the top of the stack is [, pop it off
        if the top of the stack is not, stop with ERROR
    If the character is }:
        if the top of the stack is {, pop it off
        if the top of the stack is not, stop with ERROR
    If the character is >:
        if the top of the stack is <, pop it off
        if the top of the stack is not, stop with ERROR
Repeat the above until all characters are processes.
\end{console}
