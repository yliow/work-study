%-*-latex-*-
\sectionthree{Stack machine}
\begin{python0}
from solutions import *; clear()
\end{python0}

In the previous section, you notice that there's a common
method of computation where you operator on the top two
values of a stack and then put the result back to the stack.
This type of computation appears frequently in computer science.
Because of that, if such a computation is needed in what you're
tryng to computation, you might want to
add this feature to your stack, so that you
stack becomes what is known as a \defone{stack machine}.

A \defone{stack machine} is a stack that has the usual
\texttt{stack.push(v)} and
\texttt{v = stack.pop()} operation, but it also has
\texttt{stack.add()},
\texttt{stack.subtract()},
\texttt{stack.mult()},
\texttt{stack.div()},
\texttt{stack.mod()},
etc.
In the case of an integer stack machine, those
are some of the common operation.
But of course you can add more to your stack machine if you like.
For instance you can add the exponentiation operation
or a bit shift operation.

The above operation (\texttt{add()}, etc.)
  will apply the relevant operator to the top two
values on the stack (these two values are removed from the stack)
and the result is pushed back onto the stack.

Such a computational model is implemented in software or in hardware.
In software, such a computation model is present in compilers
and virtual machines.

If you prefer, instead of the above, you can have
\verb!stack.op('+')!,
\verb!stack.op('-')!,
\verb!stack.op('*')!,
\verb!stack.op('/')!,
\verb!stack.op('%')!, etc.

\begin{ex}
Write a \texttt{IntStackMachine} class.
Use it to compute \texttt{1 + 2 - 3 * 4 + 5} by doing this:
\begin{Verbatim}[frame=single]
IntStackMachine s;
s.push(1);
std::cout << s << std::endl; // prints [1]
s.push(2);
std::cout << s << std::endl; // prints [1, 2]
s.add();
std::cout << s << std::endl; // prints [3]
s.push(3);
std::cout << s << std::endl; // prints [3, 3]
s.push(4);
std::cout << s << std::endl; // prints [3, 3, 4]
s.mult();
std::cout << s << std::endl; // prints [3, 12]
s.subtract();
std::cout << s << std::endl; // prints [-9]
s.push(5);
std::cout << s << std::endl; // prints [-9, 5]
s.add();
std::cout << s << std::endl; // prints [-4]
\end{Verbatim}
\qed
\end{ex}

Note that a stack machine evaluates one operator at a time --
it's not that smart.
The responsibility to
take an expression, break up into
steps and feed the operations one at a time to
stack machine lies somewhere else.

\begin{ex}
Add exponentiation to your stack machine: \texttt{stack.pow()}.
\qed
\end{ex}

\begin{ex}
Using \texttt{0} for false and non--zero for true,
add \texttt{stack.ifelse()} to your stack machine.
This is a ternary operator.
The top of the stack is an integer value that represents
true/false and is followed by two integer values \texttt{a} and \texttt{b}.
If the top of stack is true, the the computation pushes \texttt{a}
onto the stack; otherwise \texttt{b} is pushed onto the stack.
\qed
\end{ex}

\begin{ex}
Clean up your infix evaluation using a stack machine.
\qed
\end{ex}

%\begin{ex} [TODO: two-stack machines]
%\end{ex}
