%-*-latex-*-
\sectionthree{Polish notation}
\begin{python0}
from solutions import *; clear()
\end{python0}

The common mathematical language of arithmetic expression contains
a \lq\lq sentence'' of this form:
\[
\text{\texttt{1 + 2 - 3 * 5}} 
\]
The above way of writing arithmetic expression is called
\defone{infix notation}: a binary operator is placed \textit{in the middle}.
When given the above, I am interested in evaluating it.
Of course I want the algorithm to obey standard precedence rules.
You have been using this algorithm since elementary school.

Infix notation requires parentheses.
For instance if you look at the following in infix notation
\[
\texttt{1 + 2 * 3}
\]
it means \verb!1 + (2 * 3)!: the multiplication goes first.
If you wanted to perform the \verb!+! first, you are forced to 
write 
\verb!(1 + 2) * 3!.
Let me repeat the above: if I
say to you \lq\lq Compute this \texttt{1 + 2 * 3} but make sure
you do multiplication first'', then mathematically
(without all the above words), I have to write
\[
\texttt{1 + (2 * 3)}
\]
but if I say
\lq\lq Compute this \texttt{1 + 2 * 3} but make sure
you do addition first'', then mathematically
(without all the above words), I have to write
\[
\texttt{(1 + 2) * 3}
\]
The infix notation requires parenthesis.

Another way to write down an operator with two parameters is to put the
operator \textit{in front}.
This is called
\defterm{prefix notation}\index{prefix notation}\tinysidebar{prefix notation \\ Polish notation} or
\defterm{Polish notation}\index{Polish notation}
(the Polish mathematician
\href{https://en.wikipedia.org/wiki/Jan_%C5%81ukasiewicz}{Jan \L{}ukasiewicz}
invented this notation).
For instance instead of writing \texttt{3 + 5}, in prefix notation
I would write
\[
\texttt{+ 3 5}
\]
The prefix notation does \textit{not} require parenthesis.

In the prefix notation (or Polish notation),
if you want \verb!1 + (2 * 3)! (usual infix) in the prefix/Polish notation,
you
write
\[
\texttt{+ 1 * 2 3}
\]
because you evaluate it like this:
\[
\texttt{\underline{+ 1 \underline{* 2 3}}}
\]
i.e., (let me do it slowly):
\begin{align*}
&\text{\textwhite{=}} \hspace{0.2cm} \texttt{+ 1 * 2 3} \\
&= \texttt{\underline{+ 1 \underline{* 2 3}}} \\
&= \texttt{\underline{+ 1 6}} \\
&= \texttt{7} \\
\end{align*}
If you really want \verb!(1 + 2) * 3!, then you write this in prefix
notation
\[
\texttt{* + 1 2 3}
\]
i.e.,
\begin{align*}
&\text{\textwhite{=}} \hspace{0.2cm} \texttt{* + 1 2 3} \\
&= \texttt{\underline{* \underline{+ 1 2} 3}} \\
&= \texttt{\underline{* 3 3}} \\
&= \texttt{9} 
\end{align*}

To evaluate an expression in infix notation, 
note that you can simplify the expression if it's of the form
\verb![operator] [num1] [num2]!.
So if you're given a longer expression such as
\[
\texttt{* 1 + 2 - 3 5 4}
\]
you scan left-to-right, and once you see \verb![operator] [num1] [num2]!,
you simplify.
Here's an example where I'm using \verb!|! as my finger
and it moves left-to-right, and you can only see the data on the 
left of \verb!|!:
\begin{console}
 |* 1 - + 2 - 3 5 4
= *|1 - + 2 - 3 5 4
= * 1|- + 2 - 3 5 4
= * 1 -|+ 2 - 3 5 4
= * 1 - +|2 - 3 5 4
= * 1 - + 2|- 3 5 4
= * 1 - + 2 -|3 5 4
= * 1 - + 2 - 3|5 4
= * 1 - + 2 - 3 5|4
= * 1 - + 2 -2|4       by - 3 5 = -2
= * 1 - 0|4            by + 2 -2 = 0
= * 1 - 0 4|
= * 1 -4|              by - 0 4 = -4
= -4|                  by * 1 -4 = -4
\end{console}
The above method is kind of messy as an algorithm.
For instance if you look this line:
\begin{console}
= * 1 - + 2 - 3 5|4
= * 1 - + 2 -2|4       by - 3 5 = -2
\end{console}
The only reason we can simplify is because there's a \texttt{-}
and then two numbers.
However your finger is at \texttt{5} and you need to
know that two steps to the left is an operation.
Now \textit{if} we scan \textit{right-to-left},
remembering numbers
\begin{console}
  * 1 - + 2 - 3 5 4
                  ^
            
  * 1 - + 2 - 3 5 4
                ^
                
  * 1 - + 2 - 3 5 4
              ^
                            
  * 1 - + 2 - 3 5 4
            ^
\end{console}
then on reaching the \texttt{-}, we take the last two numbers
from our bag and perform subtraction.
Of course we the result (i.e., \texttt{-2}) back into our bag
and continue.
The bag contains only numbers if I use this method.

Clearly the bag is organized like this:
I need to be able to put numbers into the bag
and I need the bag to give me the \textit{last} two numbers.
So ... what type of container should we use for this bag of integers?
Look at the last sentence again:
\[
\text{... I need the bag to give me the \textit{last} two numbers.}
\]

So ...
you can perform polish notation evaluation
you scan right to left
and if you use a stack.
Note that if we do it this way, the stack only contains integers.
For the previous method, the stack contains integers and operators.
Here's the computation but using the new method:
\begin{console} 
                       stack (top on left)
  * 1 - + 2 - 3 5 4    []
  * 1 - + 2 - 3 5      [4] 
  * 1 - + 2 - 3        [5,  4]
  * 1 - + 2 -          [3,  5, 4]             
  * 1 - + 2            [-2, 4]       because - 3 5 is -2
  * 1 - +              [2, -2, 4]
  * 1 -                [0,  4]       because + 2 -2 is 0
  * 1                  [-4]          because - 0 4  is -4
  *                    [1, -4]
                       [-4]          because * 1 -4 is -4 
\end{console}
In other words, as you scan right to left:
\begin{tightlist}
\li If the data read is a number, push it on to the stack.
\li If the data read is an operator \verb!op!, 
    pop \verb!num1! from stack,
    pop \verb!num2! from stack, compute 
    \verb!num3 = op num1 num2! and push \verb!num3! onto the stack.
\li If there's no more data, the result is the top of the stack
(the stack should have only one value).
\end{tightlist}
Of course there's an error in the prefix expression if you have the
following situations:
\begin{tightlist}
\li When the expression is completely processed, the stack has more than 
one value
\li When you read an operator, you cannot pop two values from the
stack.
\end{tightlist}

Note that two main ingredients of the above algorithms is
\begin{tightlist}
  \item You have a stack (for holding numbers).
  \item You can perform \texttt{op num1 num2}, i.e., the simplest polish notation
  operation. The \texttt{op} is what you see from the input and
  the \texttt{num1} and \texttt{num2} are the top two values on the stack.
\end{tightlist}

\begin{ex}
What is the runtime in the above algorithm
(in terms of the length of the expression)?
\qed
\end{ex}

\begin{ex}
Evaluate the following expressions written in
prefix/Polish notation or write ERROR if the expression
is invalid.
\begin{tightlist}
  \item \texttt{* + 3 5 - 7 2}
  \item \texttt{+ / 8 2 - 4 * 7 2}
  \item \texttt{* 8 * 2 + 3 - 2 / 7 2}
  \item \texttt{/ + - * 2 3 7 4 2}
\end{tightlist}
\qed
\end{ex}


\begin{ex}
  Convert the following expression in infix notation to
  prefix/Polish notation. (Do not simplify - just convert).
  \begin{tightlist}
    \item \texttt{4 + 2 * 3 - 8 / 7}
    \item \texttt{4 + 2 * (3 - 8) / 7}
    \item \texttt{(4 + 2 * 3) - 8 / 7}
    \item \texttt{(4 + 2 * (3 - 8)) / 7}    
  \end{tightlist}
  \qed
\end{ex}


\begin{ex}
\begin{tightlist}
  \item Write a wrong expression in polish notation that will
  result in a stack with 2 numbers at the end of the algorithm.
  \item Write a wrong expression in polish notation that will
  result in a stack with 1 number when an operator is read.
\end{tightlist}
\qed
\end{ex}


\begin{ex}
Implement a polish notation calculator.
The input is a vector of strings where the strings represent integers (example
\texttt{"123"}, \texttt{"-321"})
or binary integer operators, i.e.
\verb!"+"!,
\verb!"-"!,
\verb!"*"!,
\verb!"/"!,
\verb!"%"!.
The return value is an integer.
(Besides the algorithm above, clearly you will
need to be able to convert a numeric string to an integer.
Check your CISS240 notes.)
If you do not have enough values in the stack, throw the exception
\texttt{StackUnderflowError}.
If you have too many values in the stack, throw the exception
\texttt{StackOverflowError}.
\qed
\end{ex}

