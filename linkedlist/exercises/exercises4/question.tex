\tinysidebar{\debug{exercises/{exercises4/question.tex}}}
[Random maze generation]
The goal is to generate a maze in a 2D grid.
Suppose you have the following 4x4 grid:
\begin{console}
+-+-+-+-+
| | | | |
+-+-+-+-+
| | | | |
+-+-+-+-+
| | | | |
+-+-+-+-+
| | | | |
+-+-+-+-+
\end{console}
You can think of the above as a grid of 4-by-4 rooms.
Initially think of each room as completely sealed with walls.
You can think of building a maze as punching a hole in the wall.
For instance here's a hole, from (0,0) to (0,1):
\begin{console}
+-+-+-+-+
|  x| | |
+-+-+-+-+
| | | | |
+-+-+-+-+
| | | | |
+-+-+-+-+
| | | | |
+-+-+-+-+
\end{console}
I'm using \texttt{x} to match where I am in the grid.
And now I punch another hole from (0,1) to (1,1):
\begin{console}
+-+-+-+-+
|   | | |
+-+ +-+-+
| |x| | |
+-+-+-+-+
| | | | |
+-+-+-+-+
| | | | |
+-+-+-+-+
\end{console}
Now if I punch (1,1) to (1,2) to (1,3) to (0,3) to (0,2) I get this:
\begin{console}
+-+-+-+-+
|   |x  |
+-+ +-+ +
| |     |
+-+-+-+-+
| | | | |
+-+-+-+-+
| | | | |
+-+-+-+-+
\end{console}
At this point, I'm stuck -- see the \texttt{x} in the incomplete maze.
I'm stuck in the sense that I can only punch two walls but each will
go into a room that is already visited. Again see the diagram above.
What I need to do is to go backwards.
Going one step back, I get to this point:
\begin{console}
+-+-+-+-+
|   |  x|
+-+ +-+ +
| |     |
+-+-+-+-+
| | | | |
+-+-+-+-+
| | | | |
+-+-+-+-+
\end{console}
But I'm still stuck: I don't any walls to punch (you have to stay in
the grid so you cannot punch the perimeter wall).
Butif I cgo one more step back I arrive at this point:
\begin{console}
+-+-+-+-+
|   |   |
+-+ +-+ +
| |    x|
+-+-+-+-+
| | | | |
+-+-+-+-+
| | | | |
+-+-+-+-+
\end{console}
At this point I can go from (1,3) to (2,3):
\begin{console}
+-+-+-+-+
|   |   |
+-+ +-+ +
| |     |
+-+-+-+ +
| | | |x|
+-+-+-+-+
| | | | |
+-+-+-+-+
\end{console}
To be able to build the maze I will need the following idea.
Make sure you study and think about it and make
changes/corrections/modification when necessary.
\begin{console}
ALGORITHM: build_maze
INPUT: (r,c) - the starting point

let UNVISITED be the container of unvisited cells:
    initially this should be all cells except for (r,c)
let PATH be the container containing only (r,c)
let VISITED be a container containing only (r,c)
let PUNCHED be an empty container of "(r0,c0) -- (r1,c1)"
    that represents punched walls

while PATH is not empty:
    look at the last step stored in PATH -- call it x
    look at all the surrounding cells around x
    the available cells to go to are the
        cells to the north, south, east, west of x
        which are within the grid and not visited yet,
        i.e., in UNVISITED.
    if there is at least one available cell:
        randomly choose one available cell -- call it y
        store y in VISITED and add that to PATH
        remove y from UNVISITED
        store x--y (the punched wall) in PUNCHED
    else:
        there are are no available cells
        we have to go backwards -- remove x from PATH 
\end{console}
Note that the \texttt{PATH} is a stack
(this should be clear from the way I describe how I use
\texttt{PATH}).
(Do I need both  \texttt{VISITED}.
and \texttt{UNVISITED}?)

Note that you can view the maze as a graph.
The nodes are of the form $(r,c)$.
An edge joins \texttt{(r0,c0)} to
\texttt{(r1,c1)} is the same as you can go from
room \texttt{(r0,c0)} to \texttt{(r1,c1)}.
The maze can be described by a vector of \texttt{PunchedWall}. 
Call the function
\begin{console}
class Cell
{
public:
    int r, c;
};

class PunchedWall
{
public:
    Cell c0, c1;
};
  
// Return an adjacency list describing the maze
// The maze is n-by-n and (r,c) is the starting point of the
// maze.
std::vector< PunchedWall > build_maze(int n,
                                      int r, int c);

void print_maze( int n, const std::vector<PunchedWall> & v)
\end{console}
The \verb!print_maze! will print a maze in the following manner:
\begin{console}
+-+-+-+-+
|   |   |
+-+ +-+ +
| |     |
+ +-+-+ +
|     | |
+-+ +-+ +
|       |
+-+-+-+-+
\end{console}
