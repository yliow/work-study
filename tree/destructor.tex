%-*-latex-*-
\sectionthree{Destructor/deallocation}
\begin{python0}
from solutions import *; clear()
\end{python0}

Suppose you have a tree $T$ in the heap.
Say a pointer \verb!p! points to the root of the tree (with
value $\alpha$ in the diagram):


\begin{center}
\begin{tikzpicture}[>=triangle 60,shorten >=0.5pt,node distance=2cm,auto,initial text=, double distance=2pt]
\node[state,initial] (A) at (  0,  0) {};

\path[->]
(A) edge [loop above] node {$a,b$} ()

;
\end{tikzpicture}
\end{center}
    


The memory used by the node (with value $\alpha$) and the
subtrees ($T_1$ and $T_2$) must all be returned to the heap.
Suppose the function is called \verb!deallocate! and the
parameter the a pointer to the root.
{\small
\begin{Verbatim}
    deallocate(p);
\end{Verbatim}
}
Of course this would result in
{\small
\begin{Verbatim}
    delete p;
    deallocate(q);
    deallocate(r);
\end{Verbatim}
}
where \verb!q! and \verb!r! are pointers to the left and right
subtrees of node with value $\alpha$.
Therefore the code would look like:
{\small
\begin{Verbatim}[frame=single]
void deallocate(Node * p)
{
    if (p != NULL)
    {
        deallocate(p->left_);
        deallocate(p->right_);
        delete p;
    }
}
\end{Verbatim}
}
Of course you \textit{cannot} do this:
{\small
\begin{Verbatim}[frame=single]
void deallocate(Node * p)
{
    if (p != NULL)
    {
        delete p;
        deallocate(p->left_);
        deallocate(p->right_);
    }
}
\end{Verbatim}
}
Do you see why?


The above is for the case where the tree is a binary tree.
It's clear what you need to do for a general tree.

Notice that the above uses the postorder traveral on the tree, i.e.,
we perform a postorder traveral of the tree and
perform deallocate the memory
used by each node.

Note that if you have a tree class (for instance a binary tree class):
{\small
\begin{Verbatim}[frame=single]
class Tree
{
private:
    Node * proot_;
};
\end{Verbatim}
}
the destructor can then be
{\small
\begin{Verbatim}[frame=single]
class Tree
{
public:
    ...
    ~Tree()
    {
        deallocate(proot_);
    }
...
};
\end{Verbatim}
}
You should probably also have a method to clear the tree:
{\small
\begin{Verbatim}[frame=single]
class Tree
{
public:
    ...
    void clear()
    {
        deallocate(proot_);
        proot_ = NULL;
    }
...
};
\end{Verbatim}
}
This is sort of like the destructor except that the tree object does not go out of
scope and you are in control of calling this method.

