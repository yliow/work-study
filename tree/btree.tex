%-*-latex-*-
\sectionthree{$k$-ary trees and B-trees}
\begin{python0}
from solutions import *; clear()
\end{python0}

Recall that a $k$--ary tree is just a tree where node can have
at most $k$ children.
So a binary tree is a $2$--ary tree.
Recall that the height of a binary tree of size $n$ is in the best case
\[
\lg n
\]
and in the worse case
\[
n
\]
In general if you have a $k$--ary tree, in the best case the height is
\[
\log_k n
\]
and $n$ in the worse case.

It's not difficult to imagine that the same idea used in a
binary search tree can be applied to a
$k$--ary search tree. For instance, in a 5--ary tree would look like

AAA
\input{stdout112.tex}

The first leaf has keys in $(-\infty, 5)$,
the second has leaves with keys $[5, 13)$,
the third has leaves with keys $[13, 17)$, etc.

You can make all the leaves into a doubly-linked list
so that in the iteration to access all leave nodes, you need not
to the parent nodes.
This is the case for B+ trees.
In a B+ tree, only the keys in the leaf nodes correspond to real keys.
The keys in the nonleaf nodes are meant to subdivide the keys (in the leaves)
into the leaf nodes.

For a tree where each node has 2 keys, the nodes have 3 children pointers.
Such a tree is called a (2,3)-tree.
A B+ tree is more general.

Refer to CISS430 for more information on this variation of the binary
search tree.

%\url{http://quiz.geeksforgeeks.org/data-structure/b-and-b-trees/}
