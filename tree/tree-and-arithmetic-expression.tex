\sectionthree{Trees and arithmetic expressions}
\begin{python0}
from solutions import *; clear()
\end{python0}

Trees can be used to represent arithmetic expressions.
For instance look at this:

\begin{console}[frame=single, , commandchars=~!@]
...

void insert_head(SLNode ** phead, int i)
{
    *phead = new SLNode(i, *phead);
}

void insert_head(SLNode *& phead, int i)
{
    phead = new SLNode(i, phead);
}

int main()
{
    SLNode * phead = NULL;
    insert_head(&phead, 5);
    print(phead);
    
    return 0;
}
\end{console}

The output is this:
\begin{console}[frame=single,fontsize=\small]
[student@localhost linkedlist] g++ tmp12345678.cpp; ./a.out
<SLNode 0xc01eb0 key:5, next:0>
\end{console}



The value obtained from \lq\lq evaluating'' the tree is this:
\[
\texttt{(((3 - 5) * (7 \% 2)) + ((0 * 7) + 1))}
\]

You can think of the tree as a device that can be used to describe
order of operations, or if you like, it's a device that can be 
used to play the role of parentheses.
So really, you have been doing trees since elementary school, right?

\begin{ex}
Assuming the above tree is a tree with node that contain characters,
Write a function to traverse the tree and produce the 
string
\[
\texttt{(((3 - 5) * (7 \% 2)) + ((0 * 7) + 1))}
\]
What tree traversal would you use?
\qed
\end{ex}
