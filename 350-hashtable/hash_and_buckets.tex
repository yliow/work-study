\sectionthree{Connection between hash table and buckets}

Do you remember bucket sort?

You can and \textit{should} think of hash tables
as a kind of bucketing structure.
What we have seen so far, each row in the array is a bucket.
A hash basically place a key into a bucket.
Each bucket has either one or zero key.
You can think of probes as a way to overflow from a bucket to another.

(Later we will see that there's another method to let buckets
have more than one key.)
\newpage
