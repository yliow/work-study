\sectionthree{Load factor}
\begin{python0}
from solutions import *; clear()
\end{python0}

The \defone{load factor} of a hash table, usually denoted $\alpha$, 
is defined to be
\[
\alpha = \frac{n}{m}
\]
where $n$ is the number of keys in the table and $m$ is the number of buckets.
On the average, the runtime for insert, delete, search is
\[
1 + \alpha
\]
where the 1 is due to the constant time to compute the hash and accessing the array.
All of this assume that the hash function more or less is well behaved and spread
the keys uniformly.
If the load factor is maintained at a constant value, this means
that the runtimes are $O(1)$ on the average.

In the \textit{worse} case 
insert, delete and search becomes $O(n)$.
For instance if all the keys congregate at one hash value and the 
key you're search for is the last in the chain or last in the probe sequence,
the runtime will be $O(n)$.
The same reasoning applies to the case of delete and insert.

In general, a hash table usually has a minimum load factor threshold
$\alpha_{\operatorname{min}}$ (example: $0.25$)
and a maximum load factor threshold
$\alpha_{\operatorname{max}}$ (example: $0.75$)
and the hash table maintain some statistics on the load factor.
When the load factor $\alpha$ goes beyond the maximum threshold, 
the hash table increases the table size and rehash.
If the load factor drops below the minimum threshold, the hash table
decreases the table size and rehash.

The space requirement is $O(n)$ since 
a good hash table requires the $\alpha$ be a reasonable
proportion of $n$, for instance $4n$ to maintain a $0.25$ load factor.



\newpage
