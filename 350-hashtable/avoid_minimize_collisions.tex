\sectionthree{Avoiding/minimizing collisions}

In general you want your hash functions to randomly
scatter the keys you are interested in to different index values.
In other words, you do not want to have too many collisions.
The best is not to have collisions at all.
Why?
Because if there's a collision, then you need to probe.

The technical term to use is that you have your
hash functions to behave like a uniform random distribution.

In general, designing a good hash function is not easy.
Our hash function:
\[
\text{Abe}
\rightarrow
(\operatorname{int}(\texttt{A}) \cdot 10^0 +
\operatorname{int}(\texttt{b}) \cdot 10^1 +
\operatorname{int}(\texttt{e}) \cdot 10^2) 
\mod 10
\]
is in fact pretty bad.
Why?
Because if you mod by 10 like the above, you actually get
\[
\text{Abe}
\rightarrow
(\operatorname{int}(\texttt{A}) \cdot 10^0)
\mod 10
\]
That's why \verb!Annie! was hashed to the same index as \verb!Abe!.

It's because of this that table/array sizes should be primes.
So for instance we 
could have choosen 13 for a table/array size.




\newpage
