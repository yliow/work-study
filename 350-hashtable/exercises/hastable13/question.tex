\tinysidebar{\debug{exercises/{hastable13/question.tex}}}
  There are times when the set of keys are fixed.
  Say there are 16 keys.
  If you use a hashtable, you might need a size (number of buckets)
  much larger than 16 to avoid collision.
  Add a method to your Hashtable class called \verb!compactify!
  that uses the default hash function followed by the unsigned int
  multiplication of \verb!m! such that
  the resulting hash mod size of table gives you
  values 0, 1, 2, 3, ..., 15 which implies that
  you only need a hashtable of size 16.
  If 0, 1, 2, 3, ..., 15 is impossible, since a range that that is the
  smallest possible.
  (See exercise below as well.)
