\sectionthree{Implementation}
\begin{python0}
from solutions import *; clear()
\end{python0}

Here's an abstract base class:
{\small
\begin{Verbatim}[frame=single]
class Hashable
{
public:
    virtual unsigned int hash(unsigned int) = 0;
};
\end{Verbatim}
}
This is just to impose a hash method on a subclass.
So for instance you can do this:
{\small
\begin{Verbatim}[frame=single]
class Name
{
public:
    Name(const std::string & s)
        : s_(s)
    {}

private:
    std::string s_;
};


class HashableName: public Name, Hashable
{
public:
    HashableName(const std::string & s)
        : Name(s) 
    {}
    
    unsigned int hash(unsigned int s)
    {
        unsigned int h = 0;
        ... compute h using s_ ...
        return h % s;
    }
};
\end{Verbatim}
}
Here's the class for a row of key-value in the hash table:
{\small
\begin{Verbatim}[frame=single]
template < typename Key, typename Value >
class KeyValue
{
private:
    Key key_;
    Value value_;
};
\end{Verbatim}
}
and here's the class for a row in the hash table together with a 
flag:
{\small
\begin{Verbatim}[frame=single]
template < typename Key, typename Value >
class HashtableRow
{
public:
    enum State
    { 
        AVAILABLE, NOT_AVAILABLE, DELETED 
    };
private:
    State state_;
    KeyValue< Key, Value > keyvalue_;
};
\end{Verbatim}
}
And now we have the hash table:
{\small
\begin{Verbatim}[frame=single]
template < typename Key, typename Value >
class Hashtable
{
private:
    unsigned int size_;
    HashtableRow< Key, Value > * p_;
};
\end{Verbatim}
}

Note that the \verb!HashtableRow! class can be nested in the \verb!Hashtable!
class.

%-*-latex-*-

\begin{ex} 
  \label{ex:prob-00}
  \tinysidebar{\debug{exercises/{disc-prob-28/question.tex}}}

  \solutionlink{sol:prob-00}
  \qed
\end{ex} 
\begin{python0}
from solutions import *
add(label="ex:prob-00",
    srcfilename='exercises/discrete-probability/prob-00/answer.tex') 
\end{python0}



%-*-latex-*-

\begin{ex} 
  \label{ex:prob-00}
  \tinysidebar{\debug{exercises/{disc-prob-28/question.tex}}}

  \solutionlink{sol:prob-00}
  \qed
\end{ex} 
\begin{python0}
from solutions import *
add(label="ex:prob-00",
    srcfilename='exercises/discrete-probability/prob-00/answer.tex') 
\end{python0}


%-*-latex-*-

\begin{ex} 
  \label{ex:prob-00}
  \tinysidebar{\debug{exercises/{disc-prob-28/question.tex}}}

  \solutionlink{sol:prob-00}
  \qed
\end{ex} 
\begin{python0}
from solutions import *
add(label="ex:prob-00",
    srcfilename='exercises/discrete-probability/prob-00/answer.tex') 
\end{python0}





\newpage
